% ---------------------------------------------------------------------------- %
% Honours Thesis                                                               %
% Chapter 3 - A Strategic Perspective                                          %
% ---------------------------------------------------------------------------- %

\chapter{A Strategic Perspective}  \label{chp:a-strategic-perspective}
    A prominent feature in several of the game value graphs from \autoref{fig:parameterised-incompetent-games} is the presence of plateaus, or regions over which the game value is constant.
    Indeed, aligning with our observations, Beck \parencite{Beck2013} speculates that the game value might become constant as the players approach complete competence.
    Here, primarily motivated by these plateaus, we will display various properties of incompetent games by exploring incompetence from a strategic perspective.

    The introduction of executed strategies and executable strategies in \autoref{sec:executed-strategies} establishes this ``strategic perspective'' of incompetence.
    This differs from previous approaches to incompetence in that it focuses on how a player's skill modifies their strategy space rather than their utility function.
    These concepts are used in \autoref{sec:explaining-plateaus} to explain the appearance of plateaus in the game value of parameterised incompetent games.
    Later, in \autoref{sec:optimal-learning-parameters}, we use executable strategies to describe optimal learning parameter choices under certain conditions.



\section{Executed Strategies}  \label{sec:executed-strategies}
    Consider a matrix game $G$ and a corresponding incompetent game $G_\ms{Q}$, with all definitions from \autoref{chp:game-theory} being reused.
    Recall that, in their introduction of incompetence to matrix games, Beck and Filar \parencite{Beck2007} distinguish between selectable and executable actions.\footnote{
        Unlike Beck and Filar \parencite{Beck2007}, we require each player's set of selectable and executable strategies to coincide.
        Nevertheless, with some additional notation to distinguish between selectable and executable strategies, it is straightforward to adapt our conversation to their broader setting.
    }
    A player is required to select an action (called the selected action) that, after any accidental deviations occur, realises a potentially different action (called the executed action).
    We might say that, if Player 1 chooses $a_i$ for some $i = 1, 2, \ldots, m_1$ and Player 2 chooses $b_j$ for some $j = 1, 2, \ldots, m_2$, then $\vec{q}_1(i)$ and $\vec{q}_2(j)$ are the \emph{executed strategies} realised by Player 1 and Player 2, respectively.
    This reflects the definition of $\vec{q}_1(i)$ and $\vec{q}_2(j)$ as the incompetence-induced probability distributions over each player's executable actions.

    Similarly, we can extend the concept of an executed strategy to a player's entire mixed strategy space.
    Suppose $\vec{x}' \in \vec{X}$ is Player 1's selected strategy and $\vec{y}' \in \vec{Y}$ is Player 2's selected strategy.
    Define, for all $\alpha = 1, 2, \ldots, m_1$ and $\beta = 1, 2, \ldots, m_2$, the quantities
    \begin{equation}  \label{eq:executed-action-probability}
        x_\alpha
            = \sum_{i = 1}^{m_1} x'_i q_1(i, \alpha)
            = \vec{x}' Q_1 \vec{e}_\alpha^\transp
        \quad\text{and}\quad
        y_\beta
            = \sum_{j = 1}^{m_2} y'_j q_2(j, \beta)
            = \vec{y}' Q_2 \vec{e}_\beta^\transp
    \end{equation}
    where $\vec{e}_\alpha \in \RR^{1 \times m_1}$ and $\vec{e}_\beta \in \RR^{1 \times m_2}$ are the standard basis vectors in the $\alpha^{\text{th}}$ and $\beta^{\text{th}}$ directions.
    We interpret $x_\alpha$ as being the probability that Player 1 executes action $a_\alpha$ and $y_\beta$ as being the probability that Player 2 executes action $b_\beta$.
    These probabilities are compiled to form the \emph{executed strategies} $\vec{x} = (x_1, x_2, \ldots, x_{m_1})$ and $\vec{y} = (y_1, y_2, \ldots, y_{m_2})$, which can be expressed as
    \begin{equation}  \label{eq:executed-strategies}
        \vec{x}
            = \vec{x}' Q_1
        \quad\text{and}\quad
        \vec{y}
            = \vec{y}' Q_2.
    \end{equation}
    Hence, given that the strategy profile $(\vec{x}' , \vec{y}') \in \vec{X} \times \vec{Y}$ has been selected, we know that the corresponding strategy profile $(\vec{x}' Q_1, \vec{y}' Q_2)$ is executed.

    What strategies are the players able to execute?
    We will say that $\vec{x} \in \vec{X}$ is an \emph{executable strategy} for Player 1 (under the incompetence matrix $Q_1$) whenever there exists $\vec{x}' \in \vec{X}$ satisfying $\vec{x} = \vec{x}' Q_1$.
    Analogously, we will say that $\vec{y} \in \vec{Y}$ is an \emph{executable strategy} for Player 2 (under the incompetence matrix $Q_2$) whenever there exists $\vec{y}' \in \vec{Y}$ satisfying $\vec{y} = \vec{y}' Q_2$.
    The set of executable strategies belonging to Player 1 is
    \begin{equation} \label{eq:executable-strategies-1}
        \vec{E}_1(Q_1)
            = \left\{
                \vec{x} \in \vec{X}
                : \vec{x} = \vec{x}' Q_1 \text{ for some } \vec{x}' \in \vec{X}
            \right\}
    \end{equation}
    and the set of executable strategies belonging to Player 2 is
    \begin{equation} \label{eq:executable-strategies-2}
        \vec{E}_2(Q_2)
            = \left\{
                \vec{y} \in \vec{Y}
                : \vec{y} = \vec{y}' Q_2 \text{ for some } \vec{y}' \in \vec{Y}
            \right\}.
    \end{equation}
    \nomenclature[D, 06]{$\vec{E}_1(Q_1)$}{The executable strategies belonging to Player 1. \nomrefpage}%
    \nomenclature[D, 07]{$\vec{E}_2(Q_2)$}{The executable strategies belonging to Player 2. \nomrefpage}%
    If the choice of incompetence matrices is unambiguous, then it is often convenient to write $\vec{E}_1$ instead of $\vec{E}_1(Q_1)$ and $\vec{E}_2$ instead of $\vec{E}_2(Q_2)$.
    Moreover, provided a pair of learning trajectories $Q_1 : [0, 1] \to \RR^{m_1 \times m_1}$ and $Q_2 : [0, 1] \to \RR^{m_2 \times m_2}$, we will replace $\vec{E}_1(Q_1(\lambda))$ with $\vec{E}_1(\lambda)$ and $\vec{E}_2(Q_2(\mu))$ with $\vec{E}_2(\mu)$.
    \nomenclature[E, 05]{$\vec{E}_1(\lambda)$}{The executable strategies belonging to Player 1 for any $\lambda \in [0, 1]$. \nomrefpage}%
    \nomenclature[E, 06]{$\vec{E}_2(\mu)$}{The executable strategies belonging to Player 2 for any $\mu \in [0, 1]$. \nomrefpage}%

    Next, to characterise the geometry of the spaces of executable strategies $\vec{E}_1$ and $\vec{E}_2$, observe that the strategies $\vec{x} \in \vec{X}$ and $\vec{y} \in \vec{Y}$ are executable if and only if
    \[
        \vec{x}
            = \vec{x}' Q_1
            = \sum_{i = 1}^{m_1} x'_i \vec{q}_1(i)
        \quad\text{and}\quad
        \vec{y}
            = \vec{y}' Q_2
            = \sum_{j = 1}^{m_2} y'_j \vec{q}_2(j)
    \]
    for some $\vec{x}' \in \vec{X}$ and $\vec{y}' \in \vec{Y}$.
    This means that $\vec{x}$ is a convex combination of the row vectors $\vec{q}_1(1), \vec{q}_1(2), \ldots, \vec{q}_1(m_1)$ of $Q_1$ and $\vec{y}$ is a convex combination of the row vectors $\vec{q}_2(1), \vec{q}_2(2), \ldots, \vec{q}_2(m_2)$ of $Q_2$.
    So, Player 1's set of executable strategies $\vec{E}_1$ is the convex hull of $\{\vec{q}_1(i) : i = 1, 2, \ldots, m_1\}$ and Player 2's set of executable strategies $\vec{E}_2$ is the convex hull of $\{\vec{q}_2(j) : j = 1, 2, \ldots, m_2\}$.

    Suppose a game of ``Rock, Paper, Scissors'' (shown in \autoref{fig:extensive-rock-paper-scissors} and \autoref{fig:normal-rock-paper-scissors}) is being played between a pair of players with incompetence matrices
    \[
        Q_1
            =
            \begin{pmatrix}
                \nicefrac{2}{3} & \nicefrac{1}{6} & \nicefrac{1}{6} \\
                \nicefrac{1}{6} & \nicefrac{2}{3} & \nicefrac{1}{6} \\
                \nicefrac{1}{6} & \nicefrac{1}{6} & \nicefrac{2}{3} \\
            \end{pmatrix}
        \quad\text{and}\quad
        Q_2
            =
            \begin{pmatrix}
                0 & \nicefrac{1}{3} & \nicefrac{2}{3} \\
                \nicefrac{2}{3} & 0 & \nicefrac{1}{3} \\
                \nicefrac{1}{3} & \nicefrac{2}{3} & 0 \\
            \end{pmatrix}.
    \]
    The executable strategy spaces $\vec{E}_1$ and $\vec{E}_2$ are the convex hulls of the rows within the matrices $Q_1$ and $Q_2$, respectively.
    These convex hulls are shown as shaded regions in \autoref{fig:rock-paper-scissors-executable-strategies} and, in response to our previous question, we see that there are some strategies that neither player can execute.

    \begin{figure}[t]
        \centering
        % ---------------------------------------------------------------------------- %
% Honours Thesis                                                               %
% Figure: Executable Strategies in "Rock, Paper, Scissors"                     %
% ---------------------------------------------------------------------------- %

\begin{tikzpicture}
    % Player 1
    \node (P1) at (2, -1) {$\vec{E}_1(Q_1)$}; 
    \draw (0, 0)        node[anchor=north] {$(1, 0, 0)$}
       -- (4, 0)        node[anchor=north] {$(0, 0, 1)$}
       -- (2, 3.464)    node[anchor=south] {$(0, 1, 0)$}
       -- cycle;
    \draw [fill=color2!25] (1, 0.577) -- (2, 2.309) -- (3, 0.577) -- cycle;

    % Player 2
    \node (P2) at (8,-1) {$\vec{E}_2(Q_2)$};
    \draw (6, 0)       node[anchor=north] {$(1, 0, 0)$}
       -- (10, 0)      node[anchor=north] {$(0, 0, 1)$}
       -- (8, 3.464)   node[anchor=south] {$(0, 1, 0)$}
       -- cycle;
    \draw [fill=color4!25] (9.333, 1.155) -- (7.333, 0) -- (7.333, 2.309) -- cycle;

\end{tikzpicture}
        \caption[Executable Strategies in Incompetent ``Rock, Paper, Scissors'']{The executable strategy spaces belonging to Player 1 (blue) and Player 2 (red) under incompetence in ``Rock, Paper, Scissors''.}
        \label{fig:rock-paper-scissors-executable-strategies}
    \end{figure}



\section{Game Value Plateaus}  \label{sec:explaining-plateaus}
    Now, we want to apply the concepts of executed and executable strategies to explain the presence of plateaus in \autoref{fig:parameterised-incompetent-games}.
    Beck \parencite[Theorem 3.3]{Beck2013} notes that, if a pair of completely mixed incompetent games are derived from the same matrix game, then their game values are equal.
    We will further explore this observation by showing that rectangular plateaus emerge when, despite their limited skills, players are able to execute their competent optimal strategies.

    Again, reusing the notation introduced in \autoref{chp:game-theory}, consider a matrix game $G$ and an associated incompetent game $G_\ms{Q}$.
    Our initial goal is to connect the equilibria of $G$ and $G_\ms{Q}$ through the lens of executed strategies.
    This leads us to question when an equilibrium $(\vec{x}^*, \vec{y}^*) \in \vec{X} \times \vec{Y}$ of $G_\ms{Q}$ produces an equilibrium $(\vec{x}^* Q_1, \vec{y}^* Q_2)$ of $G$, and vice versa.
    Below, \autoref{prop:incompetent-game-equilibrium} answers the ``backward direction'' of this question by showing that, given an equilibrium $(\vec{x}^* Q_1, \vec{y}^* Q_2)$ of $G$ for some $(\vec{x}^*, \vec{y}^*) \in \vec{X} \times \vec{Y}$, the strategy profile $(\vec{x}^*, \vec{y}^*)$ is always an equilibrium of $G_\ms{Q}$.

    \begin{proposition}  \label{prop:incompetent-game-equilibrium}
        Consider a strategy profile $(\vec{x}^*, \vec{y}^*) \in \vec{X} \times \vec{Y}$ such that $(\vec{x}^* Q_1, \vec{y}^* Q_2)$ is an equilibrium of the competent game $G$.
        Then, $(\vec{x}^*, \vec{y}^*)$ is an equilibrium of the incompetent game $G_\ms{Q}$.
    \end{proposition}

    \begin{proof}
        We want to prove that neither player possesses a profitable unilateral deviation from the strategy profile $(\vec{x}^*, \vec{y}^*)$ in $G_\ms{Q}$.
        First, to the contrary, assume that Player 1 has a strategy $\vec{x} \in \vec{X}$ such that $v_\ms{Q}(\vec{x}, \vec{y}^*) > v_\ms{Q}(\vec{x}^*, \vec{y}^*)$ and $\vec{x} R_\ms{Q} (\vec{y}^*)^\transp > \vec{x}^* R_\ms{Q} (\vec{y}^*)^\transp$.
        Observe that
        \begin{equation}  \label{temp:incompetent-game-equilibrium-1}
        \begin{split}
            v(\vec{x}^* Q_1, \vec{y}^* Q_2)
                & = \vec{x}^* Q_1 R Q_2^\transp (\vec{y}^*)^\transp
                = \vec{x}^* R_\ms{Q} (\vec{y}^*)^\transp \\
                & < \vec{x} R_\ms{Q} (\vec{y}^*)^\transp
                = \vec{x} Q_1 R Q_2^\transp (\vec{y}^*)^\transp
                = v(\vec{x} Q_1, \vec{y}^* Q_2).
        \end{split}
        \end{equation}
        This contradicts the equilibrium inequalities for $(\vec{x}^* Q_1, \vec{y}^* Q_2)$ in $G$ and, as a consequence, we must have $v_\ms{Q}(\vec{x}, \vec{y}^*) \le v_\ms{Q}(\vec{x}^*, \vec{y}^*)$ for all $\vec{x} \in \vec{X}$.
        Second, assuming that Player 2 possesses a strategy $\vec{y} \in \vec{Y}$ such that $v_\ms{Q}(\vec{x}^*, \vec{y}) < v_\ms{Q}(\vec{x}^*, \vec{y}^*)$ and $\vec{x}^* R_\ms{Q}  \vec{y}^\transp < \vec{x}^* R_\ms{Q} (\vec{y}^*)^\transp$, we obtain
        \begin{equation} \label{temp:incompetent-game-equilibrium-2}
        \begin{split}
            v(\vec{x}^* Q_1, \vec{y}^* Q_2)
                & = \vec{x}^* Q_1 R Q_2^\transp (\vec{y}^*)^\transp
                = \vec{x}^* R_\ms{Q} (\vec{y}^*)^\transp \\
                & > \vec{x}^* R_\ms{Q} \vec{y}^\transp
                = \vec{x}^* Q_1 R Q_2^\transp \vec{y}^\transp
                = v(\vec{x}^* Q_1, \vec{y}, Q_2).
        \end{split}
        \end{equation}
        Again, this contradicts the equilibrium inequalities for $(\vec{x}^* Q_1, \vec{y}^* Q_2)$ in $G$ and allows us to conclude that $v_\ms{Q}(\vec{x}^*, \vec{y}) \ge v_\ms{Q}(\vec{x}^*, \vec{y}^*)$ for all $\vec{y} \in \vec{Y}$.
        Therefore, having established the necessary inequalities, we see that $(\vec{x}^*, \vec{y}^*)$ is an equilibrium of the incompetent game $G_\ms{Q}$.
        It is straightforward to show that $\val(G) = \val(G_\ms{Q})$ by writing
        \begin{equation} \label{temp:incompetent-game-equilibrium-3}
            \val(G)
                = v(\vec{x}^* Q_1, \vec{y}^* Q_2)
                = \vec{x}^* Q_1 R Q_2^\transp (\vec{y}^*)^\transp
                = \vec{x}^* R_\ms{Q} (\vec{y}^*)^\transp
                = v_\ms{Q}(\vec{x}^*, \vec{y}^*)
                = \val(G_\ms{Q}),
        \end{equation}
        as required.
    \end{proof}

    A more interesting problem is encountered when answering the ``forward'' direction of our previous question; that is, when finding the conditions under which an equilibrium $(\vec{x}^*, \vec{y}^*)$ of $G_\ms{Q}$ executes an equilibrium $(\vec{x}^* Q_1, \vec{y}^* Q_2)$ of $G$.
    We may be unable to write every strategy profile $(\vec{x}, \vec{y}) \in \vec{X} \times \vec{Y}$ in the form $(\vec{x}' Q_1, \vec{y}' Q_2)$ for some $\vec{x}' \in \vec{X}$ and $\vec{y}' \in \vec{Y}$.
    This means that the constituent strategies of $(\vec{x}, \vec{y})$ cannot be executed, with either $\vec{x} \notin \vec{E}_1$ or $\vec{y} \notin \vec{E}_2$.
    Accordingly, whenever $\vec{E}_1 \subset \vec{X}$ or $\vec{E}_2 \subset \vec{Y}$ hold as strict inclusions, the equilibrium inequalities for $(\vec{x}^*, \vec{y}^*)$ in $G_\ms{Q}$ do not disprove the existence of a unilateral deviation from $(\vec{x}^* Q_1, \vec{y}^* Q_2)$ in $G$.

    Consider, for instance, a simple $2 \times 2$ matrix game $G$ called ``Matching Pennies'', which has the utility matrix
    \[
        R
            =
            \begin{pmatrix*}[r]
                1 & -1 \\
                -1 & 1
            \end{pmatrix*}.
    \]
    The players, who are each given their own penny, must simultaneously place these pennies to show either heads or tails.
    Player 1 is declared the winner if the face-up sides match and Player 2 is declared the winner if the face-up sides do not match.
    The game value of $G$ is $\val(G) = 0$ and its unique equilibrium $(\vec{x}^\sdagger, \vec{y}^\sdagger)$ has $\vec{x}^\sdagger = (\nicefrac{1}{2}, \nicefrac{1}{2})$ and $\vec{y}^\sdagger = (\nicefrac{1}{2}, \nicefrac{1}{2})$.
    So, optimal play involves both players mixing uniformly at random between heads or tails.
    Suppose Player 1 and Player 2 are assigned the incompetence matrices
    \[
        Q_1
            =
            \begin{pmatrix}
                \nicefrac{1}{3} & \nicefrac{2}{3} \\
                \nicefrac{1}{4} & \nicefrac{3}{4} \\
            \end{pmatrix}
        \quad\text{and}\quad
        Q_2
            =
            \begin{pmatrix}
                \nicefrac{3}{4} & \nicefrac{1}{4} \\
                \nicefrac{2}{3} & \nicefrac{1}{3} \\
            \end{pmatrix},
    \]
    respectively.
    The resulting incompetent game $G_\ms{Q}$ is represented by the utility matrix
    \[
        R_\ms{Q}
            = Q_1 R Q_2^\transp
            =
            \begin{pmatrix}
                \nicefrac{-1}{6} & \nicefrac{-1}{9} \\
                \nicefrac{-1}{4} & \nicefrac{-1}{6} \\
            \end{pmatrix}
    \]
    and has a unique equilibrium solution $(\vec{x}^*, \vec{y}^*)$ with $\vec{x}^* = (1, 0)$ and $\vec{y}^* = (1, 0)$.
    Clearly, under these incompetence matrices, optimal play requires that both players attempt to place their pennies showing heads.
    This implies that the strategies $\vec{x}^* Q_1 = (\nicefrac{1}{3}, \nicefrac{2}{3})$ and $\vec{y}^* Q_2 = (\nicefrac{3}{4}, \nicefrac{1}{4})$ do not form an equilibrium in $G$.
    Additionally, since
    \[
        \vec{x}^\sdagger Q_1^{-1}
            = (3, -2)
            \notin \vec{X}
        \quad\text{and}\quad
        \vec{y}^\sdagger Q_2^{-1}
            = (-2, 3)
            \notin \vec{Y},
    \]
    the unique optimal strategy $\vec{x}^\sdagger$ in $G$ belonging to Player 1 is not executable under $Q_1$ and the unique optimal strategy $\vec{y}^\sdagger$ in $G$ belonging to Player 2 is not executable under $Q_2$.

    Shortly, we will provide a sufficient condition under which an equilibrium of $G_\ms{Q}$ executes an equilibrium of $G$.
    Specifically, \autoref{thm:competent-game-equilibrium} states that, given a completely mixed  incompetent game $G_\ms{Q}$, every equilibrium $(\vec{x}^*, \vec{y}^*) \in \vec{X} \times \vec{Y}$ of $G_\ms{Q}$ produces an equilibrium $(\vec{x}^* Q_1, \vec{y}^* Q_2)$ of $G$.\footnote{Implicitly,
        in the assumption that $G_\ms{Q}$ is completely mixed, we are insisting that both players possess the same number of actions, or $m_1 = m_2$ (see \parencite[Theorem 3]{Kaplansky1945}).
    }
    Although this result is similar to both \parencite[Theorem 3.3]{Beck2013} and \parencite[Lemma 4.2]{Beck2007}, it differs in that it describes not only the game value under optimal play but also the strategies capable of achieving optimality.

    \begin{lemma}  \label{lem:non-singular-incompetence-matrices}
        If $G_\ms{Q}$ is completely mixed, then $Q_1$ and $Q_2$ are non-singular.
    \end{lemma}

    \begin{proof}
        Here, to argue by contraposition, consider an arbitrary incompetent game $G_\ms{Q}$.
        Additionally, we will assume that Player 1's incompetence matrix $Q_1$ is singular such that, for some distinct indices $I \subseteq \{1, 2, \ldots, m_1\}$, the row vectors $\{\vec{q}_1(i) : i \in I\}$ are linearly dependent.
        So, there exists non-zero coefficients $\{\theta_i \in \RR : i \in I\}$ satisfying
        \begin{equation} \label{temp:non-singular-incompetence-matrices-1}
            \sum_{i \in I} \theta_i \vec{q}_1(i)
                = 0.
        \end{equation}
        Let $I^+ = \{i \in I : \theta_i > 0\}$ denote the set of indices corresponding to positive coefficients and $I^- = \{i \in I : \theta_i < 0\}$ denote the set of indices corresponding to negative coefficients.
        Note that, since the rows of $Q_1$ are stochastic vectors, both positive and negative coefficients must be present, or $I^+ \neq \varnothing$ and $I^- \neq \varnothing$.
        We obtain
        \begin{equation}  \label{temp:non-singular-incompetence-matrices-2}
            \sum_{i \in I^+} \theta_i \vec{q}_1(i)
                = - \sum_{i \in I^-} \theta_i \vec{q}_1(i).
        \end{equation}
        after rearranging the equality in \eqref{temp:non-singular-incompetence-matrices-1}.
        Next, to identify a relationship between the coefficients, observe that
        \begin{equation} \label{temp:non-singular-incompetence-matrices-3}
            \left( \sum_{i \in I^+} \theta_i \vec{q}_1(i) \right) \vec{1}_{m_1}^\transp
                = - \left( \sum_{i \in I^-} \theta_i \vec{q}_1(i) \right) \vec{1}_{m_1}^\transp
            \quad\text{implies}\quad
            \sum_{i \in I^+} \theta_i
                = - \sum_{i \in I^-} \theta_i
        \end{equation}
        where $\vec{1}_{m_1}^\transp \in \RR^{1 \times m_1}$ is the $1 \times m_1$ all-ones row vector.
        Of course, we can rescale these coefficients to ensure that both the left-hand side and right-hand side of \eqref{temp:non-singular-incompetence-matrices-3} sum to unity.
        This allows us to interpret the vectors $(\theta_i : i \in I^+)$ and $(-\theta_i : i \in I^-)$ as probability distributions over the sets of actions $\{a_i : i \in I^+\}$ and $\{a_i : i \in I^-\}$, respectively.

        Let $(\vec{x}^*, \vec{y}^*) \in \vec{X} \times \vec{Y}$ be an equilibrium of the incompetent game $G_\ms{Q}$.
        Moreover, define a constant $\alpha \in \RR^+$ such that $x^*_i + \alpha \theta_i \ge 0$ for all $i \in I^-$ and $x^*_{i'} + \alpha \theta_{i'} = 0$ for some $i' \in I^-$.
        A suitable choice for this constant is
        \begin{equation} \label{temp:non-singular-incompetence-matrices-4}
            \alpha
                = \max_{i \in I^-} \left(-\frac{x_i^*}{\theta_i}\right).
        \end{equation}
        We can construct an alternative strategy $\vec{x}^\sdagger \in \vec{X}$, which is not completely mixed, where
        \begin{equation} \label{temp:non-singular-incompetence-matrices-5}
            x^\sdagger_i
                =
                \begin{cases}
                    x^*_i + \alpha \theta_i, & i \in I, \\
                    x^*_i, & i \notin I, \\
                \end{cases}
        \end{equation}
        for all $i = 1, 2, \ldots, m_1$.
        Certainly, this is a stochastic vector because
        \begin{equation}
            \sum_{i = 1}^{m_1} x^\sdagger_i
                = \sum_{i \not\in I} x^*_i + \sum_{i \in I} (x^*_i + \alpha \theta_i)
                = \sum_{i = 1}^{m_1} x^*_i + \alpha \sum_{i \in I} \theta_i
                = \sum_{i = 1}^{m_1} x^*_i
                = 1
        \end{equation}
        and $x^\sdagger_i \ge 0$ for every $i = 1, 2, \ldots, m_1$.
        Finally, to prove that $\vec{x}^\sdagger$ is an optimal strategy for Player 1 in $G_\ms{Q}$, it suffices to show that $\vec{x}^*$ and $\vec{x}^\sdagger$ deliver equal expected utility regardless of Player 2's selected strategy.
        Fix an arbitrary strategy $\vec{y} \in \vec{Y}$ and write
        \begin{equation}
        \begin{split}
            v_\ms{Q}(\vec{x}^\sdagger, \vec{y})
                & = \vec{x}^\sdagger R_\ms{Q} \vec{y}^\transp
                = \vec{x}^\sdagger Q_1 R Q_2^\transp \vec{y}^\transp \\
                & = \sum_{i = 1}^{m_1} x^\sdagger_i \vec{q}_1(i) R Q_2^\transp \vec{y}^\transp
                = \left(\sum_{i \not\in I} x^*_i \vec{q}_1(i) + \sum_{i \in I} (x^*_i + \alpha \theta_i) \vec{q}_1(i) \right) R Q_2^\transp \vec{y}^\transp \\
                & = \left(\sum_{i = 1}^{m_1} x^*_i \vec{q}_1(i)\right) R Q_2^\transp \vec{y}^\transp + \alpha \left(\sum_{i \in I}  \theta_i \vec{q}_1(i)\right) R Q_2^\transp \vec{y}^\transp \\
                & = \vec{x}^* Q_1 R Q_2^\transp \vec{y}^\transp
                = \vec{x}^* R_\ms{Q} \vec{y}^\transp
                = v_\ms{Q}(\vec{x}^*, \vec{y}).
        \end{split}
        \end{equation}
        Hence, from an expected utility perspective, the strategies $\vec{x}^*$ and $\vec{x}^\sdagger$ behave identically.
        This shows that $\vec{x}^\sdagger$ is an optimal strategy for Player 1 in $G_\ms{Q}$ and, as a result, the incompetent game $G_\ms{Q}$ is not completely mixed.

        We can conclude that, whenever $G_\ms{Q}$ is completely mixed, Player 1's incompetence matrix $Q_1$ is non-singular.
        Similarly, by an analagous argument, Player 2's incompetence matrix $Q_2$ will also be non-singular.
    \end{proof}

    Note that, because $Q_1$ and $Q_2$ are non-singular whenever $G_\ms{Q}$ is completely mixed, the set of Player 1's executable strategies $\vec{E}_1$ is $m_1$-dimensional and the set of Player 2's executable strategies $\vec{E}_2$ is $m_2$-dimensional.
    Furthermore, the row vectors $\{\vec{q}_1(i) : 1, 2, \ldots, m_1\}$ are extreme points of $\vec{E}_1$ and the row vectors $\{\vec{q}_2(j) : j = 1, 2, \ldots, m_2\}$ are extreme points of $Q_2$.
    Why?
    Certainly, if a row could be expressed as a convex combination of other rows, then these rows would be linearly dependent and the associated incompetence matrix would be singular.

    \begin{theorem} \label{thm:competent-game-equilibrium}
        Consider an equilibrium $(\vec{x}^*, \vec{y}^*) \in \vec{X} \times \vec{Y}$ of a completely mixed incompetent game $G_\ms{Q}$.
        The strategy profile $(\vec{x}^* Q_1, \vec{y}^* Q_2)$ is an equilibrium of the competent game $G$ and $\val(G_\ms{Q}) = \val(G)$.
    \end{theorem}

    \begin{proof}
        We will apply the indifference principle to the equilibrium $(\vec{x}^*, \vec{y}^*)$ of $G_\ms{Q}$.
        This states that Player 1's actions award equal expected utility $\gamma_1$ when Player 2 chooses $\vec{y}^*$ and Player 2's actions award equal expected utility $\gamma_2$ when Player 1 chooses $\vec{x}^*$.
        Mathematically, this is expressed in the equalities
        \begin{equation} \label{temp:competent-game-equilibrium-1}
            R_\ms{Q} (\vec{y}^*)^\transp
                = Q_1 R Q_2^\transp (\vec{y}^*)^\transp
                = \gamma_1 \vec{1}_{m_1}^\transp
            \quad\text{and}\quad
            \vec{x}^* R_\ms{Q}
                = \vec{x}^* Q_1 R Q_2^\transp
                = \gamma_2 \vec{1}_{m_2}
        \end{equation}
        where $\gamma_1 = -\gamma_2 = \val(G)$ and $\vec{1}_{m_1} \in \RR^{1 \times m_1}$ and $\vec{1}_{m_2} \in \RR^{1 \times m_2}$ are all-ones row vectors.
        Given that $Q_1$ and $Q_2$ are stochastic matrices, setting $R Q_2^\transp (\vec{y}^*)^\transp = \gamma_1 \vec{1}_{m_1}^\transp$ and $\vec{x}^* Q_1 R = \gamma_2 \vec{1}_{m_2}$ gives
        \begin{equation} \label{temp:competent-game-equilibrium-2}
            Q_1 R Q_2^\transp (\vec{y}^*)^\transp
                = \gamma_1 Q_1 \vec{1}_{m_1}^\transp
                = \gamma_1 \vec{1}_{m_1}^\transp
            \quad\text{and}\quad
            \vec{x}^* Q_1 R Q_2^\transp
                = \gamma_2 (Q_2 \vec{1}_{m_2}^\transp)^\transp
                = \gamma_2 \vec{1}_{m_2}.
        \end{equation}
        The uniqueness of these solutions is guaranteed by our finding in \autoref{lem:non-singular-incompetence-matrices} that $Q_1$ and $Q_2$ are non-singular.
        Thus, we must have $R Q_2^\transp (\vec{y}^*)^\transp = \gamma_1 \vec{1}_{m_1}^\transp$ and $\vec{x}^* Q_1 R = \gamma_2 \vec{1}_{m_2}$.
        We interpret these equalities as saying that, in the competent game $G$, Player 1 is indifferent between their actions when Player 2 selects $\vec{y}^* Q_2$ and Player 2 is indifferent between their actions when Player 1 selects $\vec{x}^* Q_1$.
        Evidently, by way of the indifference principle, this proves that $(\vec{x}^* Q_1, \vec{y}^* Q_2)$ is an equilibrium of $G$ and
        \begin{equation} \label{eq:temp:competent-game-equilibrium-3}
            \val(G)
                = v(\vec{x}^* Q_1, \vec{y}^* Q_2)
                = \vec{x}^* Q_1 R Q_2^\transp (\vec{y}^*)^\transp
                = \vec{x}^* R_\ms{Q} (\vec{y}^*)^\transp
                = v_\ms{Q}(\vec{x}^*, \vec{y}^*)
                = \val(G_\ms{Q}),
        \end{equation}
        as desired.
    \end{proof}

    Briefly, shifting to the context of a parameterised incompetent game $G_\ms{Q(\funcdot)}$ with learning trajectories $Q_1 : [0, 1] \to \RR^{m_1 \times m_1}$ and $Q_2 : [0, 1] \to \RR^{m_2 \times m_2}$, define the set
    \begin{equation} \label{eq:completely-mixed-parameters}
        \scrC
            =
            \left\{
                (\lambda, \mu) \in [0, 1] \times [0, 1]
                : G_{\lambda, \mu} \text{ is completely mixed}
            \right\}
    \end{equation}
    of learning parameter pairs.
    \autoref{thm:competent-game-equilibrium} tells us that the game value of $G_\ms{Q(\funcdot)}$ is constant on $\scrC$; that is, we have $\val(G_{\lambda, \mu}) = \val(G)$ for all $(\lambda, \mu) \in \scrC$.
    Our immediate task is to prove that, when $\scrC$ is non-empty, it constitutes the unions of rectangular plateaus present in \autoref{fig:parameterised-incompetent-games}.
    Below, the rectangular structure is established in \autoref{prop:completely-mixed-interchangability} and the plateauing behaviour is established in \autoref{prop:completely-mixed-open}.

    \begin{lemma}  \label{lem:completely-mixed-competent-game}
        If $G_\ms{Q}$ is completely mixed, then $G$ is also completely mixed.
    \end{lemma}

    \begin{proof}
        We know that the completely mixed game $G_\ms{Q}$ has a unique equilibrium $(\vec{x}^*, \vec{y}^*) \in \vec{X} \times \vec{Y}$ (see \parencite[Theorem 2]{Kaplansky1945}).
        Assume that, in pursuit of a contradiction, the corresponding competent game $G$ possesses an equilibrium $(\vec{x}^\sdagger, \vec{y}^\sdagger) \in \vec{X} \times \vec{Y}$ that is not completely mixed.

        First, having shown that the rows of $Q_1$ and $Q_2$ are linearly independent in \autoref{lem:non-singular-incompetence-matrices}, the executed strategies $\vec{x}^* Q_1 \in \vec{E}_1$  and $\vec{y}^* Q_2 \in \vec{E}_2$ are strictly positive convex combinations of the extreme points $\{\vec{q}_1(i) : i = 1, 2, \ldots, m_1\}$ and $\{\vec{q}_2(j) : j = 1, 2, \ldots, m_2\}$, respectively.
        Therefore, the interior of $\vec{E}_1$ includes $\vec{x}^* Q_1$ (or $\vec{x}^* Q_1 \in \Int \vec{E}_1)$ and the interior of $\vec{E}_2$ includes $\vec{y}^* Q_2$ (or $\vec{y}^* Q_2 \in \Int \vec{E}_2$) by \parencite[Corollary 4.19]{Soltan2020}.

        Second, recall that $(\vec{x}^\sdagger, \vec{y}^\sdagger)$ is not a completely mixed strategy profile and, as a consequence, a constituent strategy must lie on the boundary of its associated strategy space: either $\vec{x}^\sdagger \in \Bnd \vec{X}$ or $\vec{y}^\sdagger \in \Bnd \vec{Y}$.
        Assume, without loss of generality, that $\vec{x}^\sdagger$ is not completely mixed and $\vec{x}^\sdagger \in \Bnd \vec{X}$.
        Given that the set of executable strategies $\vec{E}_1$ is a subset of the entire strategy space $\vec{X}$, we have $\Int \vec{E}_1 \cap \Bnd \vec{X} = \varnothing$ and $\vec{x}^\sdagger \neq \vec{x}^* Q_1$.
        A strategy $\vec{x}^{\sdagger \sdagger} \in \vec{X}$ exists such that
        \begin{equation} \label{temp:completely-mixed-competent-game-1}
            \vec{x}^{\sdagger \sdagger}
                = \alpha \vec{x}^* Q_1 + (1 - \alpha) \vec{x}^\sdagger
                \in \Int \vec{E}_1
        \end{equation}
        for some $\alpha \in (0, 1)$ (see \parencite[Theorem 3.26]{Soltan2020}).
        Additionally, given the executability of $\vec{x}^{\sdagger \sdagger} \in \Int \vec{E}_1$, there exists $\vec{x}^{**} \in \vec{X}$ satisfying the equality $\vec{x}^{\sdagger \sdagger} = \vec{x}^{* *} Q_1$.
        We know that $\vec{x}^{\sdagger \sdagger}$ is an optimal strategy in $G$ (by closure under convex combinations, as in \parencite[Proposition G.5]{Filar1997}) and that $\vec{x}^{**}$ is an optimal strategy in $G_\ms{Q}$ (by \autoref{prop:incompetent-game-equilibrium}).
        This contradicts the uniqueness of the equilibrium $(\vec{x}^*, \vec{y}^*)$ in $G_\ms{Q}$ because the distinct strategies $\vec{x}^*$ and $\vec{x}^{**}$ are both supposedly optimal for Player 1.

        Finally, after applying an analagaous argument when $\vec{y}^\sdagger \in \Bnd \vec{Y}$, we find that $G$ is always completely mixed whenever $G_\ms{Q}$ is completely mixed.
    \end{proof}

    \begin{proposition} \label{prop:completely-mixed-interchangability}
        Suppose that, for some incompetence matrices $Q_1, T_1 \in \RR^{m_1 \times m_1}$ and $Q_2, T_2 \in \RR^{m_2 \times m_2}$, the incompetent games $G_\ms{Q_1, Q_2}$ and $G_\ms{T_1, T_2}$ are completely mixed.
        Then, $G_{\ms{Q_1, T_2}}$ and $G_{\ms{T_1, Q_2}}$ are also completely mixed.
    \end{proposition}

    \begin{proof}
        Here, without loss of generality, assume that $G$ has a non-zero game value.\footnote{
            If $\val(G) = 0$, then we can simply shift its utility matrix $R$ by a non-zero constant, as in \parencite[Theorem 5.35]{Maschler2013}.
            This produces a strategically equivalent game $G'$ with $\val(G') \neq 0$.
            The sets of equilibria in $G$ and $G'$ are identical.
        }
        Consider a pair of equilibria $(\vec{x}^*, \vec{y}^*), (\vec{x}^\sdagger, \vec{y}^\sdagger) \in \vec{X} \times \vec{Y}$ belonging to the incompetent games $G_\ms{Q_1, Q_2}$ and $G_\ms{T_1, T_2}$, respectively.
        \autoref{thm:competent-game-equilibrium} tells us that the executed strategies $(\vec{x}^* Q_1, \vec{y}^* Q_2)$ and $(\vec{x}^\sdagger T_1, \vec{y}^\sdagger T_2)$ are equilibria of $G$.
        Thus, $(\vec{x}^* Q_1, \vec{y}^\sdagger T_2)$ is also an equilibrium of $G$ and, after applying \autoref{prop:incompetent-game-equilibrium}, we find that $(\vec{x}^*, \vec{y}^\sdagger)$ is an equilibrium of $G_\ms{Q_1, T_2}$.
        We want to prove the uniqueness of this equilibrium.

        Note that the incompetence matrices $Q_1$ and $T_2$ are non-singular (by \autoref{lem:non-singular-incompetence-matrices}) and the utility matrix $R$ is non-singular (by \autoref{lem:completely-mixed-competent-game} and \parencite{Kaplansky1945}).
        Fix an arbitrary equilibrium $(\vec{x}^\sdiamond, \vec{y}^\sdiamond) \in \vec{X} \times \vec{Y}$ of $G_\ms{Q_1, T_2}$.
        Kaplansky \parencite[Theorem 1]{Kaplansky1945} states that, because both players possess completely mixed optimal strategies, we have
        \begin{equation} \label{temp:completely-mixed-interchangability-1}
            R_\ms{Q_1, T_2} (\vec{y}^\sdiamond)^\transp
                = Q_1 R T_2^\transp (\vec{y}^\sdiamond)^\transp
                = \gamma_1 \vec{1}_{m_1}^\transp
            \quad\text{and}\quad
            \vec{x}^\sdiamond R_\ms{Q_1, T_2}
                = \vec{x}^\sdiamond Q_1 R T_2^\transp
                = \gamma_2 \vec{1}_{m_2}
        \end{equation}
        where $\vec{1}_{m_1} \in \RR^{1 \times m_1}$ and $\vec{1}_{m_2} \in \RR^{1 \times m_2}$ are all-ones row vectors and $\gamma_1 = -\gamma_2 = \val(G)$.
        We can appeal to the invertability of $Q_1$, $R$, and $Q_2$ to show that the equalities in \eqref{temp:completely-mixed-interchangability-1} are uniquely solved by
        \begin{equation}
            (\vec{y}^\sdiamond)^\transp
                = \gamma_1 (Q_1 R T_2^\transp)^\inv \vec{1}_{m_1}^\transp 
            \quad\text{and}\quad
            \vec{x}^\sdiamond
                = \gamma_2 \vec{1}_{m_2} (Q_1 R T_2^\transp)^\inv.
        \end{equation}
        So, we must have $\vec{x}^\sdiamond = \vec{x}^*$ and $\vec{y}^\sdiamond = \vec{y}^\sdagger$.
        This proves that the incompetent game $G_\ms{Q_1, T_2}$ has a unique, completely mixed equilibrium $(\vec{x}^*, \vec{y}^\sdagger)$ and, as a result, it is a completely mixed game.
    \end{proof}

    Recall that the set $\scrC$ contains every pair of learning parameters $(\lambda, \mu) \in [0, 1] \times [0, 1]$ making $G_{\lambda, \mu}$ completely mixed.
    \autoref{prop:completely-mixed-interchangability} tells us that the constituent parameters of these pairs are interchangeable; that is, given any $(\lambda, \mu), (\lambda', \mu') \in \scrC$, we know that $(\lambda, \mu') \in \scrC$ and $(\lambda', \mu) \in \scrC$.
    It is convenient to define the set
    \begin{equation} \label{eq:completely-mixed-parameters-1}
        \scrC_1
            =
            \left\{
                \lambda \in [0, 1]
                : (\lambda, \mu) \in \scrC \text{ for some } \mu \in [0, 1]
            \right\}
    \end{equation}
    belonging to Player 1 and the set
    \begin{equation} \label{eq:completely-mixed-parameters-2}
        \scrC_2
            =
            \left\{
                \mu \in [0, 1]
                : (\lambda, \mu) \in \scrC \text{ for some } \lambda \in [0, 1]
            \right\}
    \end{equation}
    belonging to Player 2.
    The interchangeability property immediately implies that  $\scrC = \scrC_1 \times \scrC_2$.
    Now, having explored the rectangular structure of $\scrC$, we investigate its plateauing behaviour in \autoref{prop:completely-mixed-open}.

    \begin{proposition} \label{prop:completely-mixed-open}
        If $Q_1 : [0, 1] \to \RR^{m_1 \times m_1}$ and $Q_2 : [0, 1] \to \RR^{m_2 \times m_2}$ are continuous, then the set $\scrC$ is open in $[0, 1] \times [0, 1]$.
    \end{proposition}

    \begin{proof}
        Jansen \parencite[Theorem 3.15]{Jansen1981} proves that the set of $m_1 \times m_2$ utility matrix pairs associated with completely mixed bimatrix games is open in $\RR^{m_1 \times m_2} \times \RR^{m_1 \times m_2}$.
        This, in turn, implies that the set of $m_1 \times m_2$ utility matrices belonging to completely mixed matrix games is open in $\RR^{m_1 \times m_2}$.
        Indeed, if a matrix $R \in \RR^{m_1 \times m_2}$ exists such that the ball $\ball{\epsilon}{R}$ in $\RR^{m_1 \times m_2}$ does not exclusively contain completely mixed utility matrices for all $\epsilon \in \RR^+$, then the ball $\ball{\epsilon}{R, -R}$ in $\RR^{m_1 \times m_2} \times \RR^{m_1 \times m_2}$ would serve as a counterexample to \parencite[Theorem 3.15]{Jansen1981}.

        Fix a pair of learning parameters $(\lambda, \mu) \in \scrC$.
        Evidently, for some $\epsilon \in \RR^+$, every matrix game represented by a utility matrix in $\ball{\epsilon}{R_{\lambda, \mu}}$ is completely mixed.
        The continuity of $Q_1(\funcdot)$ and $Q_2(\funcdot)$ guarantees that the mapping from $[0, 1] \times [0, 1]$ to $\RR^{m_1 \times m_2}$ given by $(\lambda', \mu') \mapsto R_{\lambda', \mu'}$ is continuous on its entire domain.
        Therefore, there exists $\delta \in \RR^+$ such that, for all $(\lambda', \mu') \in \ball{\delta}{\lambda, \mu} \cap ([0, 1] \times [0, 1])$, we have
        \begin{equation} \label{temp:completely-mixed-open-1}
            R_{\lambda', \mu'}
                = Q_1(\lambda') R Q_2(\mu')^\transp
                \in \ball{\epsilon}{R_{\lambda, \mu}}
        \end{equation}
        and $G_{\lambda', \mu'}$ is completely mixed.
        This allows us to conclude that $\scrC$ is open in the space $[0, 1] \times [0, 1]$ of learning parameter pairs, as required.
    \end{proof}

    Finally, through \autoref{prop:completely-mixed-interchangability} and \autoref{prop:completely-mixed-open}, we have found that, whenever $G_{\lambda, \mu}$ is completely mixed for some $\lambda, \mu \in [0, 1]$, the game value of $G_\ms{Q(\funcdot)}$ forms a rectangular plateau around $(\lambda, \mu)$.
    This provides an explanation for the plateaus present in \autoref{fig:parameterised-incompetent-games}; however, it is not a necessary condition for the appearance of plateaus.

\section{Optimal Learning Parameters}  \label{sec:optimal-learning-parameters}
    The concept of an executable strategy can also be leveraged to describe a player's optimal learning parameter choices under certain conditions.
    Let $G_\ms{Q(\funcdot)}$ be a parameterised incompetent game in which Player 1 uses the learning trajectory $Q_1 : [0, 1] \to \RR^{m_1 \times m_1}$ and Player 2 uses the learning trajectory $Q_2 : [0, 1] \to \RR^{m_2 \times m_2}$.
    We will allow both players to freely choose their learning parameters before playing the associated incompetent game.
    This is modelled as a multi-stage game $\Gamma$ where
    \begin{enumerate}[
        leftmargin=*,
        align=left,
        label=\textbf{Stage \arabic*.}
    ]
        \item (\textit{Learning Parameter Selection})
            Player 1 selects a learning parameter $\lambda \in [0, 1]$ and Player 2 selects a learning parameter $\mu \in [0, 1]$,

        \item (\textit{Action Selection})
            Player 1 selects an action $a_i \in A$ and Player 2 selects an action $b_j \in B$ for some $i = 1, 2, \ldots, m_1$ and $j = 1, 2, \ldots, m_2$,
        
        \item[\textbf{Utility.}]
            Player 1 expects to receive utility $u_{\lambda, \mu}(i, j)$ and Player 2 expects to receive utility $-u_{\lambda, \mu}(i, j)$.
    \end{enumerate}
    The players are limited to their pure strategies in Stage 1 (where the action spaces are continuous) and their mixed strategies in Stage 2 (where the action spaces are discrete).
    We are exclusively interested in the subgame perfect equilibria of $\Gamma$, which require the players to select optimal strategies in $G_{\lambda, \mu}$ during Stage 2.
    Consequently, we are justified in treating $\Gamma$ as a game wherein, after the players choose learning parameters $\lambda, \mu \in [0, 1]$, they play $G_{\lambda, \mu}$ optimally to achieve expected utilities $\val(G_{\lambda, \mu})$ and $-\val(G_{\lambda, \mu})$ for Player 1 and Player 2, respectively.

    The remaining problem is to identify optimal learning parameter choices in Stage 1.
    Thus, we seek learning parameters $\lambda^*, \mu^* \in [0, 1]$ satisfying the usual zero-sum equilibrium inequalities
    \begin{equation} \label{eq:multistage-equilibrium-inequalities}
        \val\big(G_{\lambda, \mu^*}\big)
            \le \val\big(G_{\lambda^*, \mu^*}\big)
            \le \val\big(G_{\lambda^*, \mu}\big)
    \end{equation}
    for every $\lambda, \mu \in [0, 1]$.
    We will investigate the learning parameters that allow the players to execute a competent optimal strategy.
    Let $\vec{O}_1$ denote Player 1's set of optimal strategies in $G$ and $\vec{O}_2$ denote Player 2's set of optimal strategies in $G$.
    Then, define the sets of learning parameters
    \begin{equation}
        \scrE_1
            =
            \left\{
                \lambda \in [0, 1]
                : \vec{E}_1(\lambda) \cap \vec{O}_1 \neq \varnothing
            \right\}
    \end{equation}
    and
    \begin{equation}
        \scrE_2
            =
            \left\{
                \mu \in [0, 1]
                : \vec{E}_2(\mu) \cap \vec{O}_2
                \neq \varnothing
            \right\}.
    \end{equation}
    Notice that the game value of $G_\ms{Q(\funcdot)}$ is constant on the region $\scrE = \scrE_1 \times \scrE_2$ by \autoref{prop:incompetent-game-equilibrium}.
    Moreover, because \autoref{thm:competent-game-equilibrium} says that the equilibrium of a completely mixed incompetent game executes a competent optimal strategy, we have $\scrC_1 \subseteq \scrE_1$ and $\scrC_2 \subseteq \scrE_2$.

    Next, \autoref{prop:optimal-learning-parameters-sufficient} shows that every pair of learning parameters $(\lambda^*, \mu^*) \in \scrE$ is an equilibrium of the multi-stage game $\Gamma$.
    This means that a player should, whenever it is possible, choose a learning parameter that allows them to execute a competent optimal strategy.

    \begin{proposition} \label{prop:optimal-learning-parameters-sufficient}
        If $(\lambda^*, \mu^*) \in \scrE$, then $(\lambda^*, \mu^*)$ is an equilibrium in $\Gamma$.
    \end{proposition}

    \begin{proof}
        We know that, for some $\vec{x}^* \in \vec{X}$ and $\vec{y}^* \in \vec{Y}$, the strategy profile $(\vec{x}^* Q_1, \vec{y}^* Q_2)$ is an equilibrium of $G$.
        We will show that a contradiction arises whenever Player 1 or Player 2 possesses a profitable unilateral deviation from the strategy profile $(\lambda^*, \mu^*)$.

        First, assume that Player 1 has an alternative learning parameter $\lambda \in [0, 1]$ satisfying $\val(G_{\lambda, \mu^*}) > \val(G_{\lambda^*, \mu^*})$.
        Observe that, given an equilibrium $(\vec{x}^\sdagger, \vec{y}^\sdagger) \in \vec{X} \times \vec{Y}$ of $G_{\lambda^*, \mu}$, we have
        \begin{equation} \label{temp:optimal-learning-parameters-sufficient-1}
        \begin{split}
            v(\vec{x}^* Q_1(\lambda^*), \vec{y}^* Q_2(\mu^*))
                & = \vec{x}^* Q_1(\lambda^*) R Q_2(\mu^*)^\transp (\vec{y}^*)^\transp
                = \vec{x}^* R_{\lambda^*, \mu^*} (\vec{y}^*)^\transp \\
                & = \val\big(G_{\lambda^*, \mu^*}\big)
                < \val\big(G_{\lambda, \mu^*}\big)
                = \vec{x}^\sdagger R_{\lambda, \mu^*} (\vec{y}^\sdagger)^\transp
                \le \vec{x}^\sdagger R_{\lambda, \mu^*} (\vec{y}^*)^\transp \\
                & = \vec{x}^\sdagger Q_1(\lambda) R Q_2(\mu^*)^\transp (\vec{y}^*)^\transp
                = v(\vec{x}^\sdagger Q_1(\lambda), \vec{y}^* Q_2(\mu^*)).
        \end{split}
        \end{equation}
        This implies that the strategy $\vec{x}^\sdagger Q_1(\lambda)$ is a profitable deviation from the equilibrium $(\vec{x}^* Q_1, \vec{y}^* Q_2)$ in $G$.
        Hence, since this violates the equilibrium conditions, Player 1 cannot possess any profitable deviations from $(\lambda^*, \mu^*)$ in $\Gamma$.

        Second, consider a learning parameter $\mu \in [0, 1]$ for Player 2 with $\val(G_{\lambda^*, \mu}) < \val(G_{\lambda^*, \mu^*})$.
        If $(\vec{x}^\sdagger, \vec{y}^\sdagger) \in \vec{X} \times \vec{Y}$ denotes an equilibrium of $G_{\lambda^*, \mu}$, then
        \begin{equation} \label{temp:optimal-learning-parameters-sufficient-2}
        \begin{split}
            v(\vec{x}^* Q_1(\lambda^*), \vec{y}^* Q_2(\mu^*))
                & = \vec{x}^* Q_1(\lambda^*) R Q_2(\mu^*)^\transp (\vec{y}^*)^\transp
                = \vec{x}^* R_{\lambda^*, \mu^*} (\vec{y}^*)^\transp \\
                & = \val\big(G_{\lambda^*, \mu^*}\big)
                > \val\big(G_{\lambda^*, \mu}\big)
                = \vec{x}^\sdagger R_{\lambda^*, \mu} (\vec{y}^\sdagger)^\transp
                \ge \vec{x}^* R_{\lambda^*, \mu} (\vec{y}^\sdagger)^\transp \\
                & = \vec{x}^* Q_1(\lambda^*) R Q_2(\mu)^\transp (\vec{y}^\sdagger)^\transp
                = v(\vec{x}^* Q_1(\lambda^*), \vec{y}^\sdagger Q_2(\mu)).
        \end{split}
        \end{equation}
        Again, this suggests that Player 2 would benefit from switching to the strategy $\vec{y}^\sdagger Q_2(\mu)$ in $G$.
        This is a contradiction and, as a consequence, Player 2 also does not possess a profitable deviation from $(\lambda^*, \mu^*)$ in $\Gamma$.

        Therefore, after showing that both players do not have any incentive to deviate from $(\lambda^*, \mu^*)$, we conclude that it is an equilibrium of $\Gamma$.
    \end{proof}

    Although \autoref{prop:optimal-learning-parameters-sufficient} gives a sufficient condition for $(\lambda^*, \mu^*) \in [0, 1] \times [0, 1]$ to be an equilibrium, it does not exclude the possibility of other optimal learning parameters.
    Instead, to ensure that $\scrE$ contains every equilibrium of $\Gamma$, we can impose the additional restriction that the learning trajectories $Q_1(\funcdot)$ and $Q_2(\funcdot)$ must achieve complete competence on the domain $[0, 1]$.
    This necessary condition is addressed in \autoref{thm:optimal-learning-parameters-necessary}.

    \begin{theorem} \label{thm:optimal-learning-parameters-necessary}
        Assume that the learning trajectories $Q_1(\funcdot)$ and $Q_2(\funcdot)$ achieve complete competence on the interval $[0, 1]$, or $Q_1(\lambda^*) = I_{m_1}$ and $Q_2(\mu^*) = I_{m_2}$ for some $\lambda^\sdagger, \mu^\sdagger \in [0, 1]$.
        If $(\lambda^\sdagger, \mu^\sdagger) \in [0, 1] \times [0, 1]$ is an equilibrium of $\Gamma$, then $(\lambda^\sdagger, \mu^\sdagger) \in \scrE$.
    \end{theorem}

    \begin{proof}
        Notice that, because each player's entire strategy space is executable under the identity matrices $Q_1(\lambda^*)$ and $Q_2(\mu^*)$, we must have $\lambda^* \in \scrE_1$ and $\mu^* \in \scrE_2$.
        Then, \autoref{prop:optimal-learning-parameters-sufficient} tells us that $(\lambda^*, \mu^*)$ is an equilibrium of $\Gamma$ and, by the interchangeability of optimal strategies, $(\lambda^*, \mu^\sdagger)$ and $(\lambda^\sdagger, \mu^*)$ are also equilibria, with
        \begin{equation}  \label{temp:optimal-learning-parameters-necessary-1}
            \val(G)
                = \val\big(G_{\lambda^*, \mu^*}\big)
                = \val\big(G_{\lambda^*, \mu^\sdagger}\big)
                = \val\big(G_{\lambda^\sdagger, \mu^*}\big)
                = \val\big(G_{\lambda^\sdagger, \mu^\sdagger}\big)
        \end{equation}
        (see \parencite[Theorem 2.1.2]{Owen2013}).
    
        Let the strategy profiles $(\vec{x}^{*}, \vec{y}^*) \in \vec{X} \times \vec{Y}$ and $(\vec{x}^\sdiamond, \vec{y}^\sdiamond) \in \vec{X} \times \vec{Y}$ be equilibria of $G_{\lambda^*, \mu^*}$ and $G_{\lambda^\sdagger, \mu^*}$, respectively.
        Observe that
        \begin{equation}  \label{temp:optimal-learning-parameters-necessary-2}
        \begin{split}
            v(\vec{x}^\sdiamond Q_1(\lambda^\sdagger), \vec{y}^*)
                & = \vec{x}^\sdiamond Q_1(\lambda^\sdagger) R (\vec{y}^*)^\transp
                \le \vec{x}^* R (\vec{y}^*)^\transp
                = \vec{x}^* Q_1(\lambda^*) R Q_2(\mu^*)^\transp (\vec{y}^*)^\transp \\
                & = \vec{x}^* R_{\lambda^*, \mu^*} (\vec{y}^*)^\transp
                = v_{\lambda^*, \mu^*} (\vec{x}^*, \vec{y}^*)
                = \val\big(G_{\lambda^*, \mu^*}\big)
                = \val(G)
        \end{split}
        \end{equation}
        and
        \begin{equation}  \label{temp:optimal-learning-parameters-necessary-3}
        \begin{split}
            v(\vec{x}^\sdiamond Q_1(\lambda^\sdagger), \vec{y}^*)
                & = \vec{x}^\sdiamond Q_1(\lambda^\sdagger) R (\vec{y}^*)^\transp
                = \vec{x}^\sdiamond Q_1(\lambda^\sdagger) R Q_2(\mu^*)^\transp (\vec{y}^*)^\transp
                = \vec{x}^\sdiamond R_{\lambda^\sdagger, \mu^*} (\vec{y}^*)^\transp \\
                & \ge \vec{x}^\sdiamond R_{\lambda^\sdagger, \mu^*} (\vec{y}^\sdiamond)^\transp
                = v_{\lambda^\sdagger, \mu^*}(\vec{x}^\sdiamond, \vec{y}^\sdiamond)
                = \val\big(G_{\lambda^\sdagger, \mu^*}\big)
                = \val(G).
        \end{split}
        \end{equation}
        The inequalities in \eqref{temp:optimal-learning-parameters-necessary-2} and \eqref{temp:optimal-learning-parameters-necessary-3} combine to prove that, in the competent game $G$, the expected utility of the strategy profile $(\vec{x}^\sdiamond Q_1(\lambda^\sdagger), \vec{y}^*)$ is $\val(G)$.
        Hence, for any $\vec{x} \in \vec{X}$ and $\vec{y} \in \vec{Y}$, we have
        \begin{equation} \label{temp:optimal-learning-parameters-necessary-4}
        \begin{split}
            v(\vec{x}, \vec{y}^*)
                & = \vec{x} R (\vec{y}^*)^\transp
                = \vec{x} Q_1(\lambda^*) R Q_2(\mu^*)^\transp (\vec{y}^*)^\transp
                = \vec{x} R_{\lambda^*, \mu^*} (\vec{y}^*)^\transp \\
                & \le \vec{x}^* R_{\lambda^*, \mu^*} (\vec{y}^*)^\transp
                = v_{\lambda^*, \mu^*} (\vec{x}^*, \vec{y}^*)
                = \val\big(G_{\lambda^*, \mu^*}\big)
                = v(\vec{x}^\sdiamond Q_1(\lambda^\sdagger), \vec{y}^*)
        \end{split}
        \end{equation}
        and
        \begin{equation} \label{temp:optimal-learning-parameters-necessary-5}
        \begin{split}
            v(\vec{x}^\sdiamond Q_1(\lambda^\sdagger), \vec{y})
                & = \vec{x}^\sdiamond Q_1(\lambda^\sdagger) R \vec{y}^\transp
                = \vec{x}^\sdiamond Q_1(\lambda^\sdagger) R Q_2(\mu^*)^\transp \vec{y}^\transp
                = \vec{x}^\sdiamond R_{\lambda^\sdagger, \mu^*} \vec{y}^\transp \\
                & \ge \vec{x}^\sdiamond R_{\lambda^\sdagger, \mu^*} (\vec{y}^\sdiamond)^\transp
                = v_{\lambda^\sdagger, \mu^*}(\vec{x}^\sdiamond, \vec{y}^\sdiamond)
                = \val\big(G_{\lambda^\sdagger, \mu^*}\big)
                = v(\vec{x}^\sdiamond Q_1(\lambda^\sdagger), \vec{y}^*)
        \end{split}
        \end{equation}
        Clearly, since \eqref{temp:optimal-learning-parameters-necessary-4} and \eqref{temp:optimal-learning-parameters-necessary-5} are the equilibrium inequalities, Player 1 possesses the optimal strategy $\vec{x}^\sdiamond Q_1(\lambda^\sdagger)$ in $G$ and $\lambda^\sdagger \in \scrE_1$.
        A similar argument for the incompetent game $G_{\lambda^*, \mu^\sdagger}$ constructs an optimal strategy for Player 2 such that $\mu^\sdagger \in \scrE_2$.
        So, we obtain $(\lambda^\sdagger, \mu^\sdagger) \in \scrE_1 \times \scrE_2$, as desired.
    \end{proof}

    \begin{figure}[t]
        \centering
        % ---------------------------------------------------------------------------- %
% Honours Thesis                                                               %
% Figure: Learning Trajectories with Incomparable Skill Levels             a    %
% ---------------------------------------------------------------------------- %

\begin{tikzpicture}
    % Player 1
    \node (P1-1) at (2, -1) {$\vec{E}_1(\nicefrac{1}{3})$}; 
    \draw (0, 0)        node[anchor=north] {$(1, 0, 0)$}
       -- (4, 0)        node[anchor=north] {$(0, 0, 1)$}
       -- (2, 3.464)    node[anchor=south] {$(0, 1, 0)$}
       -- cycle;
    \draw [fill=color2!25] (1.333, 2.309) -- (3.333, 1.155) -- (1.333, 0) -- cycle;

    \node (P1-1) at (8,-1) {$\vec{E}_1(\nicefrac{2}{3})$};
    \draw (6, 0)       node[anchor=north] {$(1, 0, 0)$}
       -- (10, 0)      node[anchor=north] {$(0, 0, 1)$}
       -- (8, 3.464)   node[anchor=south] {$(0, 1, 0)$}
       -- cycle;
    \draw [fill=color2!25] (6.667, 1.155) -- (8.667, 2.309) -- (8.667, 0) -- cycle;

    % Player 2
    \node (P2-1) at (2, -7) {$\vec{E}_2(0)$}; 
    \draw (0, -6)        node[anchor=north] {$(1, 0, 0)$}
       -- (4, -6)        node[anchor=north] {$(0, 0, 1)$}
       -- (2, -2.536)    node[anchor=south] {$(0, 1, 0)$}
       -- cycle;
    \draw [fill=color4!25] (3, -4.268) -- (2, -6) -- (1, -4.238) -- cycle;

    \node (P2-1) at (8,-7) {$\vec{E}_2(\nicefrac{2}{3})$};
    \draw (6, -6)       node[anchor=north] {$(1, 0, 0)$}
       -- (10, -6)      node[anchor=north] {$(0, 0, 1)$}
       -- (8, -2.536)   node[anchor=south] {$(0, 1, 0)$}
       -- cycle;
    \draw [fill=color4!25] (7, -5.423) -- (8, -3.691) -- (9, -5.423) -- cycle;

\end{tikzpicture}
        \caption[Learning Trajectories with Incomparable Skill Levels]{The executable strategy spaces belonging to Player 1 (blue) and Player 2 (red) at incomparable skill levels.}
        \label{fig:incomparable-skill}
    \end{figure}

    Lastly, to provide another sufficient condition for identifying optimal learning parameters, note that the mathematical notion of incompetence does not always admit straightforward comparisons of skill.
    An increase in a player's skill corresponds to an expansion of their executable strategy space without ``forgetting'' any previously executable strategy.
    Accordingly, we say that Player 1 and Player 2 achieve \emph{greater skill} at $\lambda_2, \mu_2 \in [0, 1]$ than at $\lambda_1, \mu_1 \in [0, 1]$ whenever
    \begin{equation}  \label{eq:skill-comparison}
        \vec{E}_1(\lambda_1)
            \subseteq \vec{E}_1(\lambda_2)
        \quad\text{and}\quad
        \vec{E}_2(\mu_1)
            \subseteq \vec{E}_2(\mu_2),
    \end{equation}
    respectively.
    The binary relation of having greater skill does not necessarily relate every pair of learning parameters.
    Consider, for example, the learning trajectories $Q_1 : [0, 1] \to \RR^{3 \times 3}$ and $Q_2 : [0, 1] \to \RR^{3 \times 3}$ where
    \[
        Q_1(\lambda)
            =
            \begin{pmatrix}
                0 & 1 & 0 \\
                0 & 0 & 1 \\
                1 & 0 & 0 \\
            \end{pmatrix}
            (1 - \lambda) +
            \begin{pmatrix}
                1 & 0 & 0 \\
                0 & 1 & 0 \\
                0 & 0 & 1 \\
            \end{pmatrix}
            \lambda
    \]
    for all $\lambda \in [0, 1]$ and
    \[
        Q_2(\mu)
            =
            \begin{pmatrix}
                0 & \nicefrac{1}{2} & \nicefrac{1}{2} \\
                \nicefrac{1}{2} & 0 & \nicefrac{1}{2} \\
               \nicefrac{1}{2} & \nicefrac{1}{2} & 0 \\
            \end{pmatrix}
            (1 - \mu) +
            \begin{pmatrix}
                1 & 0 & 0 \\
                0 & 1 & 0 \\
                0 & 0 & 1 \\
            \end{pmatrix}
            \mu
    \]
    for all $\mu \in [0, 1]$.
    \autoref{fig:incomparable-skill} shows the executable strategies available to Player 1 at $\lambda = \nicefrac{1}{3}, \nicefrac{2}{3}$ and the executable strategies available to Player 2 at $\mu = 0, \nicefrac{2}{3}$.
    Player 1's learning parameters are incomparable because $\vec{E}_1(\nicefrac{1}{3}) \not\subseteq \vec{E}_1(\nicefrac{2}{3})$ and $\vec{E}_1(\nicefrac{2}{3}) \not\subseteq \vec{E}_1(\nicefrac{1}{3})$.
    Similarly, Player 2's learning parameters are incomparable because $\vec{E}_2(0) \not\subseteq \vec{E}_2(\nicefrac{2}{3})$ and $\vec{E}_2(\nicefrac{2}{3}) \not\subseteq \vec{E}_2(0)$.
    Although the ability to compare a player's skill at different parameters is not guaranteed, \autoref{lem:comparable-skill-inequality} relates the game value of $G_\ms{Q(\funcdot)}$ at those learning parameters satisfying \eqref{eq:skill-comparison}.

    \begin{lemma} \label{lem:comparable-skill-inequality}
        If $\vec{E}_1(\lambda_1) \subseteq \vec{E}_1(\lambda_2)$ and $\vec{E}_2(\mu_1) \subseteq \vec{E}_2(\mu_2)$ for some $\lambda_1, \lambda_2, \mu_1, \mu_2 \in [0, 1]$, then
        \begin{equation} \label{eq:comparable-skill-inequality}
            \val\big(G_{\lambda_1, \mu}\big)
                \le \val\big(G_{\lambda_2, \mu}\big)
            \quad\text{and}\quad
            \val\big(G_{\lambda, \mu_1}\big)
                \ge \val\big(G_{\lambda, \mu_2}\big)
        \end{equation}
        for every $\lambda, \mu \in [0, 1]$.
    \end{lemma}

    \begin{proof}
        First, fix $\mu \in [0, 1]$ and suppose $\vec{E}_1(\lambda_1) \subseteq \vec{E}_1(\lambda_2)$ for some $\lambda_1, \lambda_2 \in [0, 1]$.
        Let $(\vec{x}^*, \vec{y}^*) \in \vec{X} \times \vec{Y}$ and $(\vec{x}^\sdagger, \vec{y}^\sdagger) \in \vec{X} \times \vec{Y}$ be equilibria of $G_{\lambda_1, \mu}$ and $G_{\lambda_2, \mu}$, respectively.
        Evidently, since $\vec{x}^* Q_1(\lambda_1) \in \vec{E}_1(\lambda_1) \subseteq \vec{E}_1(\lambda_2)$, there exists $\vec{x} \in \vec{X}$ such that $\vec{x} Q_1(\lambda_2) = \vec{x}^* Q_1(\lambda_1)$.
        Observe that
        \begin{equation} \label{temp:comparable-skill-inequality-1}
        \begin{split}
        \val\big(G_{\lambda_1, \mu}\big)
            & = \vec{x}^* R_{\lambda_1, \mu} (\vec{y}^*)^\transp
            \le \vec{x}^* R_{\lambda_1, \mu} (\vec{y}^\sdagger)^\transp
            = \vec{x}^* Q_1(\lambda_1) R Q_2(\mu)^\transp (\vec{y}^\sdagger)^\transp \\
            & = \vec{x} Q_1(\lambda_2) R Q_2(\mu)^\transp (\vec{y}^\sdagger)^\transp
            = \vec{x} R_{\lambda_2, \mu} (\vec{y}^\sdagger)^\transp
            \le \vec{x}^\sdagger R_{\lambda_2, \mu} (\vec{y}^\sdagger)^\transp
            = \val\big(G_{\lambda_2, \mu}\big),
        \end{split}
        \end{equation}
        as required.

        Second, fix $\lambda \in [0, 1]$ and suppose $\vec{E}_2(\mu_1) \subseteq \vec{E}_2(\mu_2)$ for some $\mu_1, \mu_2 \in [0, 1]$.
        Consider an equilibrium $(\vec{x}^*, \vec{y}^*) \in \vec{X} \times \vec{Y}$ of $G_{\lambda, \mu_1}$ and an equilibrium $(\vec{x}^\sdagger, \vec{y}^\sdagger) \in \vec{X} \times \vec{Y}$ of $G_{\lambda, \mu_2}$.
        Then, $\vec{y}^* Q_2(\mu_1) \in \vec{E}_2(\mu_1) \subseteq \vec{E}_2(\mu_2)$ and $\vec{y} Q_2(\mu_2) = \vec{y}^* Q_2(\mu_1)$ for some $\vec{y} \in \vec{Y}$.
        Again, we have
        \begin{equation} \label{temp:comparable-skill-inequality-2}
        \begin{split}
            \val\big(G_{\lambda, \mu_1}\big)
                & = \vec{x}^* R_{\lambda, \mu_1} (\vec{y}^*)^\transp
                \ge \vec{x}^\sdagger R_{\lambda, \mu_1} (\vec{y}^*)^\transp
                = \vec{x}^\sdagger Q_1(\lambda) R Q_2(\mu_1)^\transp (\vec{y}^*)^\transp \\
                & = \vec{x}^\sdagger Q_1(\lambda) R Q_2(\mu_2)^\transp \vec{y}^\transp
                = \vec{x}^\sdagger R_{\lambda, \mu_2} \vec{y}^\transp
                \ge \vec{x}^\sdagger R_{\lambda, \mu_2} (\vec{y}^\sdagger)^\transp
                = \val\big(G_{\lambda, \mu_2}\big),
        \end{split}
        \end{equation}
        which proves the desired conclusion.
    \end{proof}

    Next, we say that Player 1 achieves \emph{maximum skill} at $\lambda^* \in [0, 1]$ whenever $\vec{E}_1(\lambda) \subseteq \vec{E}_1(\lambda^*)$ for all $\lambda \in [0, 1]$.
    Analogously, we say that Player 2 achieves \emph{maximum skill} at $\mu^* \in [0, 1]$ whenever $\vec{E}_2(\mu) \subseteq \vec{E}_2(\mu^*)$ for all $\lambda \in [0, 1]$.
    This definition captures the idea that a maximally skilled player should be able to execute every learnable strategy.
    \autoref{prop:maximum-skill-optimality} proves that a pair of maximum skill learning parameters is necessarily an equilibrium of $\Gamma$.

    \begin{proposition} \label{prop:maximum-skill-optimality}
        If Player 1 achieves maximum skill at $\lambda^* \in [0, 1]$ and Player 2 achieves maximum skill at $\mu^* \in [0, 1]$, then $(\lambda^*, \mu^*)$ is an equilibrium of $\Gamma$.
    \end{proposition}

    \begin{proof}
        We know that, for all $\lambda, \mu \in [0, 1]$, the spaces of executable strategy satisfy $\vec{E}_1(\lambda) \subseteq \vec{E}_1(\lambda^*)$ and $\vec{E}_2(\mu) \subseteq \vec{E}_2(\mu^*)$.
        Thus, according to \autoref{lem:comparable-skill-inequality}, we obtain
        \[
            \val\big(G_{\lambda, \mu^*}\big)
                \le \val\big(G_{\lambda^*, \mu^*}\big)
                \le \val\big(G_{\lambda^*, \mu}\big).
        \]
        These are the necessary equilibrium inequalities from \eqref{eq:multistage-equilibrium-inequalities} and, as a result, $(\lambda^*, \mu^*)$ is an equilibrium of $\Gamma$.
    \end{proof}

    The existence of maximum skill learning parameters depends entirely on the selected learning trajectories; however, a special case occurs when Player 1 and Player 2 are capable of achieving complete competence.
    This means that $Q_1(\lambda^*) = I_{m_1}$ and $Q_2(\mu^*) = I_{m_2}$ for some $\lambda^*, \mu^* \in [0, 1]$.
    Clearly, since
    \[
        \vec{E}_1(\lambda^*)
            = \vec{E}_1(I_{m_1})
            = \vec{X}
        \quad\text{and}\quad
        \vec{E}_2(\mu^*)
            = \vec{E}_2(I_{m_2})
            = \vec{Y},
    \]
    Player 1 achieves maximum skill at $\lambda^*$ and Player 2 achieves maximum skill at $\mu^*$.
    Therefore, \autoref{prop:maximum-skill-optimality} reduces to the ``optimality of complete competence'' discussed in \parencite[Section 4.4]{Beck2007} and \parencite[Proposition 3.1]{Beck2013}.
