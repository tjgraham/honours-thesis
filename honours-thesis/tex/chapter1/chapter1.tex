% ---------------------------------------------------------------------------- %
% Honours Thesis                                                               %
% Chapter 1 - Introduction                                                     %
% ---------------------------------------------------------------------------- %

\chapter{Introduction} \label{chp:introduction}
    Often, in a repeated strategic situation whose outcome is dependent on the skill of its participants, we might expect these participants to improve their skills by learning or training.
    Consider, for instance, a tennis player who must practice to increase their competitive performance or a military that must invest to increase their capabilities.
    These seemingly disparate examples share two common characteristics: short-term skill-dependent interactions with opponents (tennis matches or military engagements) and long-term opportunities to improve skills (practicing tennis shots or investing in military equipment).
    We are interested in studying situations with these characteristics from both a short-term perspective and a long-term perspective; that is, we want to provide insight into the strategies that can be employed to achieve desired outcomes.

    A natural quantitative framework to explore this problem is provided by game theory, which investigates strategic interactions between rational agents whose objectives are not necessarily aligned.
    This mathematical notion of a game, inspired by its namesake, consists of multiple players who select actions with the intention of maximising their rewards.
    The modern game-theoretic paradigm was introduced in 1928 by von Neumann's ``On the Theory of Games of Strategy'' \parencite{vonNeumann1959} and popularised in 1944 by von Neumann and Morgenstern's ``Theory of Games and Economic Behaviour'' \parencite{vonNeumann2004}.
    Although their methodology was originally presented in the context of economic applications, it has become a useful tool for analysing problems in a variety of other domains, including political science, theoretical biology, and military planning \parencite{Maschler2013}.
    Here, we will motivate our discussion of skill by reviewing the evolution and applications of game theory.

    Unsurprisingly, several mathematical discussions of strategy---often in the context of parlour games---predate the methodology of modern game theory.
    Note that, similar to von Neumann and Morgenstern's \parencite{vonNeumann2004} contributions, these discussions predominantly focus on two-player zero-sum games, or games in which total rewards are conserved.
    The earliest application of a minimax solution, a pair of strategies that minimise each player's maximum loss, is attributed to Waldegrave's analysis of a two-player variant of the card game \textit{Le Her} in 1713 \parencite{Dimand1992}.
    Precisely, each player is assigned a strategy that maximises their probability of winning regardless of their opponent's choices.
    Although Waldegrave expresses scepticism of the solution's need to randomise (or mix) over actions, these mixed strategies play a critical role in further developments \parencite{Dimand1992}.
    Indeed, Borel \parencite{Borel1921} allows players to randomise over their actions in games whose ``winnings depend [symmetrically] both on chance and the skill of the players.''
    The existence of a minimax solution is proved in \parencite{Borel1921} and \parencite{Borel1924} under the assumption that players possess three and five actions, respectively.
    Borel \parencite{Borel1927} concludes by speculating that a minimax solution exists when players have seven available actions; however, he remains unconvinced as to whether the statement can be generalised to arbitrary finite collections of actions.

    The scope of the problem is widened by von Neumann \parencite{vonNeumann1959} to include two-player zero-sum games with finitely many actions and arbitrary, not necessarily symmetric, rewards.
    It is demonstrated, by way of a counterexample, that a minimax solution might not exist when players are restricted to using non-randomised (or pure) strategies.
    Then, in a similar vein to Borel \parencite{Borel1921}, this motivates the need for unpredictable behaviour in the form of mixed strategies.
    The mathematical details behind this ``shift'' are justified by von Neumann and Morgenstern's \parencite{vonNeumann2004} axiomatisation of player preferences and utility functions.
    Consequently, von Neumann \parencite{vonNeumann1959} answers Borel's \parencite{Borel1927} earlier question in the affirmative; that is, he shows that a minimax solution exists in mixed strategies for any two-player zero-sum game.
    The concept of minimising a player's maximum loss, despite its successful application to zero-sum environments, begins to present problems in non-zero-sum (or general-sum) games.
    Although a minimax solution to a zero-sum game is stable in the sense that neither player has an incentive to deviate, the same cannot be said for the general-sum case \parencite{Maschler2013}.
    Instead, Nash \parencite{Nash1950} introduces the now-ubiquitous Nash equilibrium, which guarantees that a player cannot benefit from unilaterally deviating from their prescribed strategy.
    The existence of a Nash equilibrium is proved for any general-sum game with finitely many actions \parencite{Nash1950}.
    Note that, since the set of minimax solutions and set of equilibria belonging to a zero-sum game are identical, we will henceforth prefer the terminology ``equilibrium'' over ``minimax solution''.

    What does an equilibrium tell us when game theory is applied to real-world situations?
    Well, the answer depends on the lens through which equilibria are viewed; there are two possible interpretations:
    \begin{itemize}
        \item \textbf{Descriptive}, the view that game theory predicts (human or non-human) strategies, and
        \item \textbf{Normative}, the view that game theory recommends ``best'' strategies \parencite{Wooldridge2012}.
    \end{itemize}
    Here, by approaching the task of ``solving'' a game as individually recommending strategies to its players, we will adopt a normative interpretation of game theory.
    This viewpoint is particularly successful in zero-sum settings because equilibria offer both stability, dissuading unilateral deviations, and security, guaranteeing a minimum reward \parencite{Maschler2013}.
    The security of equilibria is not necessary in general-sum games and, as a consequence, different equilibria might confer different rewards.
    A normative approach encounters problems when needing to compare these rewards and nominate a ``best'' equilibrium \parencite{Maschler2013}.\footnote{A
        response to the difficulty of recommending ``best'' strategies in general-sum games is to modify the concept of an equilibrium.
        For example, the subgame perfect equilibrium is an equilibrium refinement and the correlated equilibrium is an equilibrium generalisation (see \parencite[Chapter 7, Chapter 8]{Maschler2013}).
    }

    Of course, whenever our intention is to suggest suitable strategies to a player, we should always account for their ability to implement these recommendations.
    Larkey, Kadane, Austin, and Zamir \parencite{Larkey1997} observe that, in traditional game theoretic discussions, ``[t]he cognitive or physical difficulties for players in devising and executing strategies for playing particular games are essentially assumed away.''
    Then, seeking to identify the difficulties a player might encounter, they propose a typology of skill consisting of:
    \begin{itemize}
        \item \textbf{Strategic Skill}, the ability to select which games should be played,
        \item \textbf{Planning Skill}, the ability to develop a desirable strategy within a game, and
        \item \textbf{Execution Skill}, the ability to execute desired actions throughout a game.
    \end{itemize}
    Larkey et al. \parencite{Larkey1997} apply their typology to experimentally compare strategies in ``Sum Poker'' under different skill limitations; however, a precise description of skill is not provided.
    We are only interested in further exploring execution skill because strategic skill is not critical in individual games and planning skill is not critical in normative game theory.

    A mathematical framework that incorporates execution skill---or lack thereof---is provided by Beck and Filar \parencite{Beck2007} for zero-sum settings and Beck, Ejov, and Filar \parencite{Beck2012} for general-sum settings.
    Essentially, a notion of incompetence is introduced that quantifies a player's tendency to accidentally deviate from their intended actions.
    The key advantage of incompetence, despite being superficially similar to Selten's \parencite{Selten1975} concept of a ``slight mistake'' (or, as it has since been called, a trembling hand), is that it is expressly designed to describe execution skill.
    Conversely, the motivation behind Selten's \parencite{Selten1975} discussion of accidental deviations is to define an equilibrium refinement.
    So, while a trembling hand involves players making mistakes with negligible probability, incompetence allows players to make mistakes according to arbitrary probability distributions.

    The application of incompetence to military planning is discussed by Beck in \parencite{Beck2013} and \parencite{Beck2011}.
    Moreover, within the context of evolutionary biology, incompetence has been applied by Kleshnina, Filar, Ejov, and McKerral in \parencite{Kleshnina2018} to study the behaviour and adaptation of species that are prone to mistakes.
    Kleshnina, Streipert, Filar, and Chatterjee \parencite{Kleshnina2020} also study the optimal stepwise learning strategies in evolutionary games under incompetence.

    Presently, our main objective is to further explore the nature of incompetence in zero-sum games from a normative perspective.
    \autoref{chp:game-theory} begins by reviewing the relevant mathematical concepts in traditional games and incompetent games.
    Recall that, in the earlier examples of a tennis player and a military force, a distinction was established between short-term and long-term strategies.
    The short-term strategic horizon, which involves optimally playing games under incompetence, is discussed in \autoref{chp:a-strategic-perspective}.
    This mainly investigates various properties of incompetence by building upon the already-established properties from \parencite{Beck2013} and \parencite{Beck2007}.
    Additionally, the long-term strategic horizon, which involves improving execution skill and modifying incompetence, is discussed in \autoref{chp:incremental-learning}.
    We describe a model of incremental learning, develop a backward induction technique to compute equilibria, and apply this process to a simple tennis game from \parencite{Beck2013}.
    The overarching goal is that, by understanding the short-term and long-term strategic considerations, we can create better strategies to mitigate and reduce incompetence.