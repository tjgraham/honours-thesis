% ---------------------------------------------------------------------------- %
% Honours Thesis                                                               %
% Chapter 5 - Conclusion                                                       %
% ---------------------------------------------------------------------------- %

\chapter{Conclusion}
    Adopting the perspective of normative game theory, we sought to address the strategic considerations arising from execution skill, or the possibility that a player accidentally deviates from their intended action.
    The mathematical notion of an incompetent matrix game, which was introduced by Beck and Filar \parencite{Beck2007}, captured these accidental deviations as probability distributions over a player's available actions.
    We explored the properties of incompetent games in \autoref{chp:a-strategic-perspective} and the ``best'' learning strategies in \autoref{chp:incremental-learning}.

    First, when exploring properties of incompetent games in \autoref{chp:a-strategic-perspective}, we focused on two interesting features: game value plateaus and optimal learning parameters.
    These problems were addressed by viewing incompetence as modifying each player's strategy space---to create the space of executable strategies---instead of modifying each player's expected utility.
    Accordingly, we were able to show that rectangular game value plateaus (as in \autoref{fig:parameterised-incompetent-games}) arise whenever a pair of learning parameters create a completely mixed incompetent game.
    Additionally, under the assumption that complete competence is achievable, we proved that a learning parameter is optimal if and only if the corresponding space of executable strategies contains a competent optimal strategy.
    However, the established optimality conditions do not always apply and, as a consequence, further investigation might explore learning parameter choices in situations where competent optimal strategies are never executable.
    Furthermore, since the multi-stage game from \autoref{sec:optimal-learning-parameters} was free of cost, a possible extension could involve adding learning costs and finding properties of the resulting equilibria.

    Second, in \autoref{chp:incremental-learning}, we attempted to find the ``best'' learning strategies for playing incompetent games.
    An incremental learning model was proposed that required two players to repeatedly play an incompetent game with opportunities to increment their learning parameters between stages.
    We were able to leverage the resulting transition structure to apply a backward induction algorithm and exhaustively compute equilibria.
    Precisely, while traditional backward induction cannot be applied because incremental learning unfolds over infinitely many stages, we instead iterate through the game's finite collection of states.
    A simple tennis game was shown as an example of the potential strategic insights gained from understanding the equilibria in incremental learning games.
    Of course, given that we were primarily focused on finding methods to solve these games, we did not attempt to characterise structural patterns present in their equilibria.
    So, future research might explore the properties of the ``best'' learning strategies with the goal of developing easy-to-apply learning heuristics.

    Overall, we have demonstrated the usefulness of the incompetence concept in modelling execution skill and expanded the scope of learning dynamics to include incremental learning.
    We provided insight into various properties of incompetence using executable strategies and insight into incremental learning strategies using backward induction.
    Hopefully, through continued incorporation of skill, the recommendations obtained from game-theoretic analysis will become more realistic and robust.
