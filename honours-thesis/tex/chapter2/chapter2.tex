% ---------------------------------------------------------------------------- %
% Honours Thesis                                                               %
% Chapter 2 - Game Theory                                                      %
% ---------------------------------------------------------------------------- %

\chapter{Game Theory} \label{chp:game-theory}
    What is a game?
    We certainly cannot promise a single mathematical model that captures the entire diversity of real-world strategic interactions.
    Instead, games can be broadly described as ``situations involving several decision makers with different goals, in which the decision of each affects the outcome for all the decision makers'' \parencite{Maschler2013}.
    Adopting the terminology of traditional games, these decision makers are called \emph{players} and their available choices are called \emph{actions}.
    It is assumed that players select actions with the intention of maximising the outcome-dependent reward or \emph{utility} they receive.\footnote{Although
        a complete discussion of utility theory is beyond the scope of this brief introduction, the development of utility functions from player preferences is explained in \parencite[Chapter 2]{Maschler2013}.
        Here, utility can simply be viewed as an abstract quantity resembling a monetary incentive.
    }

    After a game-theoretic model has been formulated, we seek to solve it by finding player strategies that satisfy a \emph{solution concept}, which captures salient properties of rational behaviour.
    Accordingly, this chapter reviews several well-established models and solution concepts that are encountered throughout our exploration of incompetence and learning.
    \autoref{sec:game-representations-and-strategies} and \autoref{sec:matrix-games-and-bimatrix-games} explain introductory game theory (from \parencite{Maschler2013}, \parencite{Osborne1994}, and \parencite{Owen2013}) and \autoref{sec:incompetent-games} explains incompetent matrix games (from \parencite{Beck2013} and \parencite{Beck2007}).



\section{Game Representations and Strategies} \label{sec:game-representations-and-strategies}
    Unsurprisingly, some real-world strategic interactions share similarities with popular parlour games---for example, chess, backgammon, or poker.
    These games involve players sequentially selecting actions in pursuit of a desired outcome but could also include randomness (dice in backgammon) or private information (face-down cards in poker).
    A situation with these characteristics may be described using an \emph{extensive-form game} where, within a suitably constructed directed tree, each vertex is assigned to a player who must select an incident arc to traverse.
    This sequence of choices traces a path through the tree until a terminal vertex is reached and utility is awarded.
    Randomness is incorporated by adding vertices where a lottery determines the traversed arc and private information is incorporated by partitioning a player's decision vertices into \emph{information sets}.
    Precisely, the players are only aware of the current information set and, from among the vertices in this information set, are unable to discern the exact current vertex.

    Recall the simple two-player game ``Rock, Paper, Scissors'' in which both players simultaneously reveal a rock, paper, or scissors hand sign.
    Then, a winner is determined by applying the dominance relationships: rock beats scissors, scissors beats paper, and paper beats rock.
    An extensive-form representation of this game is shown in \autoref{fig:extensive-rock-paper-scissors}.
    The shaded regions represent information sets and a label $(x, y)$ on a terminal vertex indicates that Player 1 receives utility $x$ and Player 2 receives utility $y$.
    We imitate simultaneous action selection by combining the possible outcomes of Player 1's actions into a single information set such that Player 2 must act without knowing the choice that was made.


    \begin{figure}[t]
        \centering
        % ---------------------------------------------------------------------------- %
% Honours Thesis                                                               %
% Figure: "Rock, Paper, Scissors" in Extensive Form                            %
% ---------------------------------------------------------------------------- %


\begin{istgame}[font=\small]
    % Grow East
    \setistgrowdirection'{east}



    % Player 1
    \xtdistance{40mm}{22.5mm}

    \istroot(0)<[yshift=15pt]>{Player 1}
        % Rock
        \istb{Rock}[above, sloped, yshift=-2pt, font=\scriptsize]
        
        % Paper
        \istb{Paper}[above, yshift=-2pt, font=\scriptsize]
        
        % Scissors
        \istb{Scissors}[above, sloped, yshift=-2pt, font=\scriptsize]
    \endist



    % Player 2
    \xtdistance{40mm}{7.5mm}
    
    \istroot(1)(0-1)<[yshift=15pt]>{Player 2}
        % Rock
        \istb{Rock}[above, sloped, xshift=5pt, yshift=-2pt, font=\scriptsize]{$(0, 0)$}
        
        % Paper
        \istb{Paper}[above, xshift=7.5pt, yshift=-2pt, font=\scriptsize]{$(-1, 1)$}
        
        % Scissors
        \istb{Scissors}[above, sloped, xshift=10pt, yshift=-2pt, font=\scriptsize]{$(1, -1)$}
    \endist

    \istroot(2)(0-2){}
        % Rock
        \istb{Rock}[above, sloped, xshift=5pt, yshift=-2pt, font=\scriptsize]{$(1, -1)$}
        
        % Paper
        \istb{Paper}[above, xshift=7.5pt, yshift=-2pt, font=\scriptsize]{$(0, 0)$}
        
        % Scissors
        \istb{Scissors}[above, sloped, xshift=10pt, yshift=-2pt, font=\scriptsize]{$(-1, 1)$}
    \endist

    \istroot(3)(0-3){}
        % Rock
        \istb{Rock}[above, sloped, xshift=5pt, yshift=-2pt, font=\scriptsize]{$(-1, 1)$}
        
        % Paper
        \istb{Paper}[above, xshift=7.5pt, yshift=-2pt, font=\scriptsize]{$(1, -1)$}
        
        % Scissors
        \istb{Scissors}[above, sloped, xshift=10pt, yshift=-2pt, font=\scriptsize]{$(0, 0)$}
    \endist

    % Information Sets
    \xtInfosetO[fill=color2!25](0)(0) % Player 1
    \xtInfosetO[fill=color4!25](1)(3) % Player 2

\end{istgame}
        \caption[``Rock, Paper, Scissors'' in Extensive Form]{An extensive-form representation of ``Rock, Paper, Scissors'' \parencite{Maschler2013}.}
        \label{fig:extensive-rock-paper-scissors}
    \end{figure}

    Already, from the example of ``Rock, Paper, Scissors'', it should be obvious that the complete definition of an extensive-form game quickly becomes cumbersome.
    A simplified model requires every player to implement a fixed strategy that will determine their behaviour throughout the game.
    Specifically, a \emph{pure strategy} assigns unique actions to a player's information sets.
    Knowing that player behaviour is entirely determined by the implemented strategies, we may view the game as a simultaneous selection of strategies rather than a sequential selection of actions.
    This is captured using a \emph{normal-form game} $G = (N, (S_k)_{k \in N}, (u_k)_{k \in N})$ with
    \begin{itemize}
        \item a set of players $N = \{1, 2, \ldots, n\}$ for some $n \in \ZZ^+$,
        \item a set of actions $S_k$ for each $k \in N$, and
        \item a utility function $u_k : S_k \to \RR$ for each $k \in N$ where $S = S_1 \times S_2 \times \ldots, S_n$.
    \end{itemize}
    \nomenclature[A, 01]{$G$}{A normal-form game. \nomrefpage}%
    \nomenclature[A, 02]{$N$}{The set of players in $G$. \nomrefpage}%
    \nomenclature[A, 03]{$S_k$}{The set of actions belonging to Player $k \in N$ in $G$. \nomrefpage}%
    \nomenclature[A, 04]{$S$}{The set of action profiles in $G$. \nomrefpage}%
    \nomenclature[A, 05]{$u_k$}{The utility function belonging to Player $k \in N$ in $G$. \nomrefpage}%
    Here, every player $k \in N$ simultaneously chooses an action $s_k \in S_k$ to form an \emph{action profile} (or \emph{pure strategy profile}) $s = (s_1, s_2, \ldots, s_n) \in S$ containing the choices of all participants.
    The realisation of this action profile causes the player $k \in N$ to receive utility $u_k(s) = u_k(s_1, s_2, \ldots, s_n)$.
    Notice that the actions within the normal-form representation of an extensive-form game correspond to the available pure strategies.

    A player wanting to behave unpredictably might employ a \emph{mixed strategy}, which randomises among their pure strategies.
    The space of mixed strategies belonging to Player $k \in N$ is denoted by
    \[
        \Delta_k
            =
            \left\{
                \delta_k : S_k \to [0, 1] : \sum_{s_k \in S_k} \delta_k(s_k) = 1
            \right\}
    \]
    and contains every probability distribution over the action set $S_k$.\footnote{
        This characterisation of mixed strategies, to avoid measure-theoretic complications, implicitly assumes that the action sets $S_1, S_2, \ldots, S_n$ are finite.
        An extension of this discussion to games with infinitely many actions can be found in \parencite[Chapter 4]{Owen2013}.
    }
    \nomenclature[A, 06]{$\Delta_k$}{The set of mixed strategies belonging to Player $k \in N$ in $G$. \nomrefpage}%
    If the mixed strategy $\delta_k \in \Delta_k$ is selected, then $\delta_k(s_k)$ is interpreted as the probability that Player $k \in N$ plays the action $s_k \in S_k$.
    Now, each player $k \in N$ can select a mixed strategy $\delta_k \in \Delta_k$ to form a \emph{(mixed) strategy profile} $\delta \in \Delta$ where $\Delta = \Delta_1 \times \Delta_2 \times \ldots \times \Delta_n$.
    \nomenclature[A, 07]{$\Delta$}{The set of mixed strategy profiles in $G$. \nomrefpage}%
    We say that a strategy profile $\delta \in \Delta$ is \emph{completely mixed} whenever $\delta_k(s_k) > 0$ for all players $k \in N$ and actions $s_k \in S_k$.

    How can a player value a strategy profile $\delta \in \Delta$ when the utility functions $u_1, u_2, \ldots, u_n$ are only defined on the set of action profiles $S$?
    Naturally, given any $k \in N$, we create an expected utility function $v_k : \Delta \to \RR$ such that
    \begin{equation} \label{eq:expected-utility}
        v_k(\delta)
            = \sum_{s_1 \in S_1} \sum_{s_2 \in S_2} \ldots \sum_{s_n \in S_n} u_k(s_1, s_2, \ldots, s_n) \delta_1(s_1) \delta_2(s_2) \ldots \delta_n(s_n)
    \end{equation}
    for all $\delta \in \Delta$.
    \nomenclature[A, 08]{$v_k$}{The expected utility function belonging to Player $k \in N$ in $G$. \nomrefpage}%
    We say that $v_k(\delta)$ is the \emph{value} of the strategy profile $\delta \in \Delta$ to Player $k \in N$.\footnote{The
        ability to consistently value mixed strategies via expected utility is a consequence of player preferences satisfying the von Neumann-Morgenstern axioms (see \parencite[Theorem 2.18]{Maschler2013}).
    }
    Formally, the normal-form game $(N, (\Delta_k)_{k \in N}, (v_k)_{k \in N})$ is called the \emph{mixed extension} of $G$ but, because our players have access to their mixed strategies, we will use $G$ itself to mean this mixed extension.

    Intending to introduce incompetence to a narrower class of normal-form games in \autoref{sec:incompetent-games}, we should clarify the notions of a finite game and a zero-sum game.
    We say that our normal-form game $G$ is \emph{finite} if, for every $k \in N$, the action set $S_k$ is finite.
    We say that $G$ is \emph{zero-sum} when overall utility is conserved regardless of the game's outcome or, equivalently, whenever
    \begin{equation} \label{eq:zero-sum-property}
        \sum_{k \in N} u_k(s)
            = \sum_{k \in N} u_k(s_1, s_2, \ldots, s_n)
            = 0
    \end{equation}
    for all $s \in S$.
    It is straightforward to show from \eqref{eq:expected-utility} and \eqref{eq:zero-sum-property} that the mixed extension of a zero-sum game is also zero-sum.

    The solution concept that we are generally interested in finding when working with a normal-form game is the \emph{Nash equilibrium}.
    This is a collection of strategies for which no player would benefit from unilaterally deviating and adopting an alternative strategy.
    If $\delta \in \Delta$ is a strategy profile and $\delta_k \in \Delta_k$ is a strategy for Player $k \in N$, then $(\delta_k, \delta_{-k})$ denotes the strategy profile obtained by replacing the $k$\textsuperscript{th} entry of $\delta$ with $\delta_k$.
    Using this notation to describe unilateral deviations, a Nash equilibrium is a profile $\delta^* = (\delta^*_1, \delta^*_2, \ldots, \delta^*_n) \in \Delta$ such that, for any player $k \in N$ and strategy $\delta_k \in \Delta(S_k)$, we have
    \begin{equation} \label{eq:nash-equilibrium}
        v_k\bigl(\delta_k, \delta^*_{-k}\bigr)
            \le v_k(\delta^*).
    \end{equation}
    Nash \parencite{Nash1950} proved that an equilibrium always exists in a finite normal-form game.
    A useful tool when computing these equilibria is the \emph{indifference principle}, which states that, under a mixed strategy equilibrium, any actions in $S_k$ played with non-zero probability must yield equal expected utility to Player $k \in N$ (see, for instance, \parencite[Theorem 5.18]{Maschler2013}).
    If every equilibrium $\delta^* \in \Delta$ is completely mixed, then the game $G$ is said to be \emph{completely mixed}.

    \autoref{fig:normal-rock-paper-scissors} shows a normal-form representation of ``Rock, Paper, Scissors'' derived from the extensive-form game in \autoref{fig:extensive-rock-paper-scissors}.
    The actions belonging to Player 1 are presented in rows, the actions belonging to Player 2 are presented in columns, and an entry $(x, y)$ indicates that they receive utilities $x$ and $y$, respectively.
    Aligning with the common usage of ``Rock, Paper, Scissors'' as a randomisation device, it is not a surprise that this game's single equilibrium---both players mixing uniformly over their actions---causes the players to win with equal probability.

    Occasionally, the Nash equilibrium is not a suitable solution concept and a refinement is needed to further restrict the definition of rational behaviour.
    A useful refinement in extensive-form games is the \emph{subgame perfect equilibrium}.
    This is a Nash equilibrium that remains resilient to unilateral deviations whenever it is restricted to an arbitrary \emph{subgame}---a subtree that does not divide any information sets.
    The desirability of a subgame perfect equilibrium comes from its elimination of incredible threats or irrational strategies that attempt to dissuade opponents from particular actions.
    Schelling \parencite{Schelling1980} gives an example of an incredible threat where
    \begin{quote}
        ``if I threaten to blow us both to bits unless you close the window, you know that I won't unless I have somehow managed to leave myself no choice in the matter.''
    \end{quote}
    We will always seek to eliminate these rational inconsistencies when solving games with sequential action selection.

    \begin{figure}[h]
        \centering
        % ---------------------------------------------------------------------------- %
% Honours Thesis                                                               %
% Figure: "Rock, Paper, Scissors" in Normal Form                               %
% ---------------------------------------------------------------------------- %

\setlength{\extrarowheight}{3pt}

\begin{tabular}{cc|c|c|c|}
                            & \multicolumn{1}{c}{} & \multicolumn{3}{c}{Player 2}                                                            \\
                            & \multicolumn{1}{c}{} & \multicolumn{1}{c}{Rock}     & \multicolumn{1}{c}{Paper} & \multicolumn{1}{c}{Scissors} \\ \cline{3-5}
    \multirow{3}*{Player 1} & Rock                 & $0, 0$                       & $-1, 1$                   & $1, -1$                      \\ \cline{3-5}
                            & Paper                & $1, -1$                      & $0, 0$                    & $-1, 1$                      \\ \cline{3-5}
                            & Scissors             & $-1, 1$                      & $1, -1$                   & $0, 0$                       \\ \cline{3-5}
\end{tabular}

\setlength{\extrarowheight}{0pt}
        \caption[``Rock, Paper, Scissors'' in Normal Form]{A normal-form representation of ``Rock, Paper, Scissors'' \parencite{Maschler2013}.}
        \label{fig:normal-rock-paper-scissors}
    \end{figure}



\section{Matrix Games and Bimatrix Games} \label{sec:matrix-games-and-bimatrix-games}
    The tabular depiction of ``Rock, Paper, Scissors'' in \autoref{fig:normal-rock-paper-scissors} correctly suggests that finite two-player normal-form games can be represented as matrices.
    Suppose $G$ is a finite two-player normal-form game and, without loss of generality, take Player 1's action set to be $S_1 = A = \{a_1, a_2, \ldots, a_{m_1}\}$ and Player 2's action set to be $S_2 = B = \{b_1, b_2, \ldots, b_{m_2}\}$ for some $m_1, m_2 \in \ZZ^+$.
    \nomenclature[B, 01]{$G$}{A bimatrix game. \nomrefpage}%
    \nomenclature[B, 02]{$A$}{The set of actions belonging to Player 1 in $G$. \nomrefpage}%
    \nomenclature[B, 03]{$B$}{The set of actions belonging to Player 2 in $G$. \nomrefpage}%
    \nomenclature[B, 04]{$m_k$}{The number of actions available to Player $k \in \{1, 2\}$ in $G$. \nomrefpage}%
    The \emph{utility matrices} $R_1 \in \RR^{m_1 \times m_2}$ and $R_2 \in \RR^{m_1 \times m_2}$ encode the utilities allocated to Player 1 and Player 2 for every possible combination of actions; that is,
    \[
        R_1[i, j]
            = u_1(a_i, b_j)
        \quad\text{and}\quad
        R_2[i, j]
            = u_2(a_i, b_j)
    \]
    for all $i = 1, 2, \ldots, m_1$ and $j = 1, 2, \ldots, m_2$.
    \nomenclature[B, 05]{$R_k$}{The utility matrix belonging to Player $k \in \{1, 2\}$ in $G$. \nomrefpage}%
    We might think of $G$ as a game wherein, after Player 1 chooses a row index $i = 1, 2, \ldots, m_1$ and Player 2 chooses a column index $j = 1, 2, \ldots, m_2$, they are awarded utilities $u_1(a_i, b_j)$ and $u_2(a_i, b_j)$, respectively.
    This interpretation allows us to write $u_1(i, j) = u_1(a_i, b_j)$ and $u_2(i, j) = u_2(a_i, b_j)$ for any $i = 1, 2, \ldots, m_1$ and $j = 1, 2, \ldots, m_2$.
    A game admitting this representation is called an $m_1 \times m_2$ \emph{bimatrix game}.

    The mixed strategy spaces $\Delta_1$ and $\Delta_2$ contain probability distributions over the finite action sets $A$ and $B$.
    It is convenient to represent these mixed strategies as stochastic row vectors from the sets
    \[
        \vec{X}
            =
            \left\{
                \vec{x}
                    = (x_1, x_2, \ldots, x_{m_1}) \in [0, 1]^{m_1}
                    : \sum_{i = 1}^{m_1} x_i = 1
            \right\}
    \]
    and
    \[
        \vec{Y}
            =
            \left\{
                \vec{y}
                    = (y_1, y_2, \ldots, y_{m_2}) \in [0, 1]^{m_2}
                    : \sum_{j = 1}^{m_2} y_j = 1
            \right\}.
    \]
    \nomenclature[B, 06]{$\vec{X}$}{The set of mixed strategies belonging to Player 1 in $G$. \nomrefpage}%
    \nomenclature[B, 07]{$\vec{Y}$}{The set of mixed strategies belonging to Player 2 in $G$. \nomrefpage}%
    If Player 1 chooses the mixed strategy $\vec{x} \in \vec{X}$ and Player 2 chooses the mixed strategy $\vec{y} \in \vec{Y}$, then the actions $a_i$ and $b_j$ are played with probability $x_i$ and $y_j$, respectively.
    Adapting our expected utility functions $v_1 : \Delta \to \RR$ and $v_2 : \Delta \to \RR$ to the domain $\vec{X} \times \vec{Y}$, the value of these strategies to Player $k = 1, 2$ is
    \begin{equation} \label{eq:bimatrix-expected-utility}
        v_k(\vec{x}, \vec{y})
            = \sum_{i = 1}^{m_1} \sum_{j = 1}^{m_2} x_i u_k(i, j) y_j
            = \vec{x} R_k \vec{y}
    \end{equation}
    Next, by combining \eqref{eq:nash-equilibrium} and \eqref{eq:bimatrix-expected-utility}, observe that a strategy profile $(\vec{x}^*, \vec{y}^*) \in \vec{X} \times \vec{Y}$ is a Nash equilibrium in the bimatrix game $G$ if and only if
    \begin{equation} \label{eq:bimatrix-nash-equilibrium}
        \vec{x} R_1 (\vec{y}^*)^\transp
            \le \vec{x}^* R_1 (\vec{y}^*)^\transp
        \quad\text{and}\quad
        \vec{x}^* R_2 \vec{y}^\transp
            \le \vec{x}^* R_2 (\vec{y}^*)^\transp
    \end{equation}
    given any deviations $\vec{x} \in \vec{X}$ and $\vec{y} \in \vec{Y}$.

    \begin{figure}[t]
        \centering
        % ---------------------------------------------------------------------------- %
% Honours Thesis                                                               %
% Figure: "Battle of the Sexes" in Normal Form                                 %
% ---------------------------------------------------------------------------- %

\setlength{\extrarowheight}{3pt}

\begin{tabular}{cc|c|c|}
                            & \multicolumn{1}{c}{} & \multicolumn{2}{c}{Player 2}                               \\
                            & \multicolumn{1}{c}{} & \multicolumn{1}{c}{Football} & \multicolumn{1}{c}{Concert} \\ \cline{3-4}
    \multirow{2}*{Player 1} & Football             & $2, 1$                       & $0, 0$                      \\ \cline{3-4}
                            & Concert              & $0, 0$                       & $1, 2$                      \\ \cline{3-4}
\end{tabular}

\setlength{\extrarowheight}{0pt}

        \caption[``Battle of the Sexes'' in Normal-Form]{A normal-form representation of ``Battle of the Sexes''.}
        \label{fig:battle-of-the-sexes}
    \end{figure}

    A simple bimatrix game known as ``Battle of the Sexes'' is shown in \autoref{fig:battle-of-the-sexes} and is represented by the utility matrices
    \[
        R_1
            =
            \begin{pmatrix}
                2 & 0 \\
                0 & 1 \\
            \end{pmatrix}
        \quad\text{and}\quad
        R_2
            =
            \begin{pmatrix}
                1 & 0 \\
                0 & 2 \\
            \end{pmatrix}.
    \]
    Here, without any communication, two players must individually decide between going to a football match or a concert.
    Although Player 1 prefers the football match and Player 2 prefers the concert, they must attend the same event to receive a non-zero utility reward.
    The three possible equilibrium solutions $(\vec{x}^*, \vec{y}^*)$, which each achieve different values of $\vec{v}(\vec{x}^*, \vec{y}^*) = (v_1(\vec{x}^*, \vec{y}^*), v_2(\vec{x}^*, \vec{y}^*))$, are:
    \begin{itemize}
        \item $\vec{x}^* = (1, 0)$ and $\vec{y}^* = (1, 0)$ with expected utility $\vec{v}(\vec{x}^*, \vec{y}^*) = (2, 1)$,
        \item $\vec{x}^* = (0, 1)$ and $\vec{y}^* = (0, 1)$ with expected utility $\vec{v}(\vec{x}^*, \vec{y}^*) = (1, 2)$, and
        \item $\vec{x}^* = (\nicefrac{2}{3}, \nicefrac{1}{3})$ and $\vec{y}^* = (\nicefrac{1}{3}, \nicefrac{2}{3})$ with expected utility $\vec{v}(\vec{x}^*, \vec{y}^*) = (\nicefrac{2}{3}, \nicefrac{2}{3})$.
    \end{itemize}
    This demonstrates that a general bimatrix game might possess multiple Nash equilibria and that these do not necessarily award the same expected utilities.

    Under the additional assumption that the bimatrix game $G$ is zero-sum, we know that $R_1 = - R_2$ since $u_1(i, j) + u_2(i, j) = 0$ for all $i = 1, 2, \ldots, m_1$ and $j = 1, 2, \ldots, m_2$.
    Hence, these utility allocations can be encoded in a single matrix $R \in \RR^{m_1 \times m_2}$ where
    \[
        R[i, j]
            = u_1(i, j)
            = - u_2(i, j)
    \]
    for each $i = 1, 2, \ldots, m_1$ and $j = 1, 2, \ldots, m_2$.
    A finite two-player zero-sum game $G$ is called an $m_1 \times m_2$ \emph{matrix game} and $R$ is its \emph{utility matrix}.
    \nomenclature[C, 01]{$G$}{A matrix game. \nomrefpage}%
    \nomenclature[C, 04]{$R$}{The utility matrix of $G$. \nomrefpage}%
    We can describe a matrix game by giving a single utility function $u : A \times B \to \RR$ where
    \[
        u(a_i, b_j)
            = u(i, j)
            = u_1(i, j)
            = - u_2(i, j)
    \]
    for all $i = 1, 2, \ldots, m_1$ and $j = 1, 2, \ldots, m_2$.
    \nomenclature[C, 02]{$u$}{The utility function of $G$. \nomrefpage}%
    The expected utilities of the strategy profile $(\vec{x}, \vec{y}) \in \vec{X} \times \vec{Y}$ are expressed by rewriting \eqref{eq:bimatrix-expected-utility} as
    \begin{equation} \label{eq:matrix-expected-utility}
        v_1(\vec{x}, \vec{y})
            = \sum_{i = 1}^{m_1} \sum_{j = 1}^{m_2} x_i u_1(i, j) y_j
            = \vec{x} R \vec{y}^\transp
            = \sum_{i = 1}^{m_1} \sum_{j = 1}^{m_2} x_i u_2(i, j) y_j
            = - v_2(\vec{x}, \vec{y}).
    \end{equation}
    This motivates our definition of a value function $v : \vec{X} \times \vec{Y} \to \RR$ where $v(\vec{x}, \vec{y}) = \vec{x} R \vec{y}^\transp$ for all $(\vec{x}, \vec{y}) \in \vec{X} \times \vec{Y}$.
    \nomenclature[C, 03]{$v$}{The expected utility function of $G$. \nomrefpage}%
    If Player 1 selects $\vec{x} \in \vec{X}$ and Player 2 selects $\vec{y} \in \vec{Y}$, then they expect to receive utilities $v(\vec{x}, \vec{y})$ and $-v(\vec{x}, \vec{y})$, respectively.
    We might view $G$ as a game in which Player 1 chooses their strategy $\vec{x} \in \vec{X}$ to maximise $v(\vec{x}, \vec{y}) = \vec{x} R \vec{y}^\transp$ and Player 2 chooses their strategy $\vec{y} \in \vec{Y}$ to minimise $v(\vec{x}, \vec{y}) = \vec{x} R \vec{y}^\transp$.
    The equilibrium inequalities from \eqref{eq:bimatrix-nash-equilibrium} can be rearranged to show that the strategy profile $(\vec{x}^*, \vec{y}^*) \in \vec{X} \times \vec{Y}$ is an equilibrium if and only if
    \begin{equation} \label{eq:matrix-nash-equilibrium}
        \vec{x} R (\vec{y}^*)^\transp
            \le \vec{x}^* R (\vec{y}^*)^\transp
            \le \vec{x}^* R \vec{y}^\transp
    \end{equation}
    for every $\vec{x} \in \vec{X}$ and $\vec{y} \in \vec{Y}$.
    The strategies $\vec{x}^*$ and $\vec{y}^*$ are called \emph{optimal strategies} to emphasise their unique properties in zero-sum games.
    Namely, taking arbitrary equilibria $(\vec{x}^*, \vec{y}^*), (\vec{x}^\sdagger, \vec{y}^\sdagger) \in \vec{X} \times \vec{Y}$, we know that they have equal value $v(\vec{x}^*, \vec{y}^*) = v(\vec{x}^\sdagger, \vec{y}^\sdagger)$ and that $(\vec{x}^*, \vec{y}^\sdagger)$ and $(\vec{x}^\sdagger, \vec{y}^*)$ are also equilibria (see \parencite[Theorem 2.1.2]{Owen2013}).\footnote{
        Generally, these properties cannot be extended to equilibrium solutions in bimatrix games.
        Recall that the equilibria in ``Battle of the Sexes'' did not yield the same expected utility and that their component strategies could not be interchanged to produce new equilibria.
    }
    This common value shared among equilibria is called the \emph{game value} of $G$ and is denoted by $\val(G)$.
    \nomenclature[C, 05]{$\val(G)$}{The game value of $G$. \nomrefpage}%
    The existence of optimal mixed strategies in matrix games is established by von Neumann's \parencite{vonNeumann1959} minimax theorem, which proves the equality
    \begin{equation} \label{eq:minimax-equality}
        \val(G)
            = \max_{\vec{x} \in \vec{X}} \min_{\vec{y} \in \vec{Y}} \vec{x} R \vec{y}^\transp
            = \min_{\vec{y} \in \vec{Y}} \max_{\vec{x} \in \vec{X}} \vec{x} R \vec{y}^\transp.
    \end{equation}
    Accordingly, the equilibrium of a zero-sum game is also a \emph{minimax solution} in which both players minimise their worst-case losses.
    We will reguarly need to find optimal strategies and game values in matrix games, a task that is often achieved through linear programming.
    Specifically, if $\vec{x}^* \in \vec{X}$, $\vec{y}^* \in \vec{Y}$, and $\gamma \in \RR$ solve the primal linear program
    \begin{equation} \label{lp:player1-program}
    \tag{LP1}
        \begin{array}{lr@{}ll}
            \text{maximise} & \multicolumn{1}{l}{\gamma} & & \\
            \text{subject to} & \gamma - \displaystyle\sum\limits_{i = 1}^{m_1} u(i, j) & x^*_i \le 0, &  j = 1, 2, \ldots, m_2, \\
             & \displaystyle\sum\limits_{i = 1}^{m_1} & x^*_i = 1, & \\
             & & x^*_i \ge 0, & i = 1, 2, \ldots, m_1, \\
        \end{array}
    \end{equation}
    and the dual linear program
    \begin{equation} \label{lp:player2-program}
    \tag{LP2}
        \begin{array}{lr@{}ll}
            \text{minimise} & \multicolumn{1}{l}{\gamma} & & \\
            \text{subject to} & \gamma - \displaystyle\sum\limits_{j = 1}^{m_2} u(i, j) & y^*_j \ge 0, & i = 1, 2, \ldots, m_1, \\
             & \displaystyle\sum\limits_{j = 1}^{m_2} & y^*_j = 1, & \\
             & & y^*_j \ge 0, & j = 1, 2, \ldots, m_2, \\
        \end{array}
    \end{equation}
    then $\vec{x}^*$ is an optimal strategy for Player 1, $\vec{y}^*$ is an optimal strategy for Player 2, and $\gamma$ is the game value
    (see \parencite[Chapter 3]{Owen2013}).



\section{Incompetent Games}
\label{sec:incompetent-games}
    Beck and Filar \parencite{Beck2007} introduce incompetence to matrix games by allowing players to accidentally deviate from their proposed strategies.\footnote{We
        must, for the sake of brevity, only discuss incompetence in matrix games whose selectable and executable actions coincide.
        A broader definition of incompetence in matrix and bimatrix games can be found in \parencite{Beck2013}, \parencite{Beck2012}, and \parencite{Beck2007}.
    }
    They construct a pair of \emph{incompetence matrices} $Q_1 \in \RR^{m_1 \times m_1}$ and $Q_2 \in \RR^{m_2 \times m_2}$ such that:
    \begin{itemize}
        \item $q_1(a_i, a_\alpha) = q_1(i, \alpha) = Q_1[i, \alpha]$ is the probability that Player 1 executes action $a_\alpha$ after selecting action $a_i$ for all $i, \alpha = 1, 2, \ldots, m_1$, and
        \item $q_2(b_j, b_\beta) = q_2(j, \beta) = Q_2[j, \beta]$ is the probability that Player 2 executes action $b_\beta$ after selecting action $b_j$ for all $j, \beta = 1, 2, \ldots, m_2$.
    \end{itemize}
    \nomenclature[D, 01]{$Q_k$}{An incompetence matrix belonging to Player $k \in \{1, 2\}$. \nomrefpage}%
    If Player 1 chooses $a_i$ and Player 2 chooses $b_j$ for some $i = 1, 2, \ldots, m_1$ and $j = 1, 2, \ldots, m_2$, then the stochastic row vectors
    \[
        \vec{q}_1(a_i)
            = \vec{q}_1(i)
            = (q_1(i, \alpha))_{\alpha = 1}^{m_1}
        \quad\text{and}\quad
            \vec{q}_2(b_j)
            = \vec{q}_2(j)
            = (q_2(j, \beta))_{\beta = 1}^{m_2}
    \]
    are interpreted as probability distributions over the executable actions belonging to Player 1 and Player 2, respectively.
    What are the expected utilities of the action profile $(a_i, b_j)$ under incompetence?
    Consider a function $u_\ms{Q_1, Q_2} : A \times B \to \RR$ where, by taking the probability-weighted sum over the possible action profiles, we have
    \begin{equation} \label{eq:incompetent-utility}
        u_\ms{Q_1, Q_2}(a_i, b_j)
            = u_\ms{Q_1, Q_2}(i, j)
            = \sum_{\alpha = 1}^{m_1} \sum_{\beta = 1}^{m_2} q_1(i, \alpha) u(\alpha, \beta) q_2(j, \beta)
            = \vec{q}_1(i) R \vec{q}_2(j)^\transp
    \end{equation}
    for all $i = 1, 2, \ldots, m_1$ and $j = 1, 2, \ldots, m_2$.
    \nomenclature[D, 03]{$u_\ms{Q_1, Q_2}$}{The utility function of $G_\ms{Q_1, Q_2}$. \nomrefpage}%
    Clearly, when accounting for the effects of incompetence, $u_\ms{Q_1, Q_2}$ becomes Player 1's utility function and $-u_\ms{Q_1, Q_2}$ becomes Player 2's utility function.
    An \emph{incompetent (matrix) game} $G_\ms{Q_1, Q_2}$ replaces the utility functions of the competent matrix game $G$ with these incompetence-adjusted utility functions.
    \nomenclature[D, 02]{$G_\ms{Q_1, Q_2}$}{An incompetent game. \nomrefpage}%
    A mixed strategy profile $(\vec{x}, \vec{y}) \in \vec{X} \times \vec{Y}$ in this incompetent game has value $v_\ms{Q_1, Q_2}(\vec{x}, \vec{y})$ to Player 1 and value $-v_\ms{Q_1, Q_2}(\vec{x}, \vec{y})$ to Player 2 where
    \begin{equation} \label{eq:incompetent-expected-utility}
        v_\ms{Q_1, Q_2}(\vec{x}, \vec{y})
            = \sum_{i = 1}^{m_1} \sum_{j = 1}^{m_2} x_i u_\ms{Q_1, Q_2}(i, j) y_j
            = \sum_{i = 1}^{m_1} \sum_{j = 1}^{m_2} x_i \vec{q}_1(i) R \vec{q}_2(j)^\transp y_j
            = \vec{x} Q_1 R Q_2^\transp \vec{y}^\transp.
    \end{equation}
    \nomenclature[D, 04]{$v_\ms{Q_1, Q_2}$}{The expected utility function of $G_\ms{Q_1, Q_2}$. \nomrefpage}%
    This shows that $G_\ms{Q_1, Q_2}$ is represented by the utility matrix $R_\ms{Q_1, Q_2} = Q_1 R Q_2^\transp$.
    \nomenclature[D, 05]{$R_\ms{Q_1, Q_2}$}{The utility matrix of $G_\ms{Q_1, Q_2}$. \nomrefpage}%
    Henceforth, whenever the choice of incompetence matrices is unambiguous, we substitute the subscript ``$Q$'' for ``$Q_1, Q_2$'', as in $G_\ms{Q}$, $u_\ms{Q}$, $v_\ms{Q}$, and $R_\ms{Q}$.

    Primarily, we are interested in the behaviour of incompetent games under variations in their incompetence matrices.
    These variations are captured by Beck and Filar \parencite{Beck2007} using \emph{learning trajectories}, which are functions that map from the interval $[0, 1]$ to the set of $m \times m$ stochastic matrices for some $m \in \ZZ^+$.
    The learning trajectories $Q_1 : [0, 1] \to \RR^{m_1 \times m_1}$ and $Q_2 : [0, 1] \to \RR^{m_2 \times m_2}$ parameterise a family of incompetent games
    \[
        \left\{
            G_{\lambda, \mu}
                = G_{Q_1(\lambda), Q_2(\mu)}
                : \lambda, \mu \in [0, 1]
        \right\}.
    \]
    \nomenclature[E, 01]{$Q_k(\funcdot)$}{A learning trajectory belonging to Player $k \in \{1, 2\}$. \nomrefpage}%
    We shall refer to this family of games as a \emph{parameterised incompetent (matrix) game} and write $G_\ms{Q_1(\funcdot), Q_2(\funcdot)}$ or $G_\ms{Q(\funcdot)}$.
    \nomenclature[E, 02]{$G_\ms{Q_1(\cdot), Q_2(\cdot)}$}{A parameterised incompetent game. \nomrefpage}
    Here, the expressions $Q_1(\funcdot)$ and $Q_2(\funcdot)$ serve as a reminder that these are learning trajectories, not incompetence matrices.
    Observe that, given any \emph{learning parameters} $\lambda, \mu \in [0, 1]$, the incompetent game $G_{\lambda, \mu}$ has $Q_1(\lambda)$ as Player 1's incompetence matrix, $Q_2(\mu)$ as Player 2's incompetence matrix, and $R_{\lambda, \mu} = Q_1(\lambda) R Q_2(\mu)^\transp$ as its utility matrix.
    \nomenclature[E, 03]{$G_{\lambda, \mu}$}{The incompetent game for any $\lambda, \mu \in [0, 1]$. \nomrefpage}%
    \nomenclature[E, 04]{$R_{\lambda, \mu}$}{The utility matrix of $G_{\lambda, \mu}$ for any $\lambda, \mu \in [0, 1]$. \nomrefpage}%

    Among the collection of $m \times m$ incompetence matrices for some $m \in \ZZ^+$, Beck and Filar \parencite{Beck2007} associate $\nicefrac{1}{m} \cdot J_m$ with \emph{uniform incompetence} and $I_m$ with \emph{complete competence} where $J_m$ is the $m \times m$ all-one matrix and $I_m$ is the $m \times m$ identity matrix.
    Intuitively, uniform incompetence causes a player to select actions uniformly at random and complete competence causes a player to select actions with absolute precision.
    They also restrict their discussion to linear learning trajectories $Q : [0, 1] \to \RR^{m \times m}$ satisfying
    \[
        Q(\lambda)
            = Q(0) (1 - \lambda) + Q(1) \lambda
    \]
    for all $\lambda, \mu \in [0, 1]$.\footnote{An
        empirical exploration of several additional learning trajectories---namely, sigmoidal, exponential, power-law, and discontinuous learning trajectories---is provided in \parencite[Section 4.4]{Beck2013}.
    }

    Consider, for instance, a $2 \times 2$ matrix game $G$ represented by the utility matrix $R \in \RR^{2 \times 2}$ where
    \[
        R
            =
            \begin{pmatrix*}[r]
                1 & -1 \\
                3 & 1 \\
            \end{pmatrix*}.
    \]
    We might introduce incompetence by assigning Player 1 the learning trajectory $Q_1 : [0, 1] \to \RR^{2 \times 2}$ and Player 2 the learning trajectory $Q_2 : [0, 1] \to \RR^{2 \times 2}$ where
    \[
        Q_1(\lambda)
            =
            \begin{pmatrix}
                \nicefrac{1}{2} & \nicefrac{1}{2} \\
                \nicefrac{1}{2} & \nicefrac{1}{2}
            \end{pmatrix}
            (1 - \lambda)
            +
            \begin{pmatrix}
                1 & 0 \\
                0 & 1 \\
            \end{pmatrix}
            \lambda
        \quad\text{and}\quad
        Q_2(\mu)
            =
            \begin{pmatrix}
                \nicefrac{1}{2} & \nicefrac{1}{2} \\
                \nicefrac{1}{2} & \nicefrac{1}{2}
            \end{pmatrix}
            (1 - \mu)
            +
            \begin{pmatrix}
                1 & 0 \\
                0 & 1 \\
            \end{pmatrix}
            \mu
    \]
    for every $\lambda, \mu \in [0, 1]$.
    Notice that, under these learning trajectories, both players transition linearly from uniform incompetence to complete competence.
    This produces a parameterised incompetent game $G_\ms{Q(\funcdot)}$ and, for all $\lambda, \mu \in [0, 1]$, the incompetent game $G_{\lambda, \mu}$ is a matrix game represented by the utility matrix $R_{\lambda, \mu} = Q_1(\lambda) R Q_2(\mu)^\transp$.
    If Player 1's learning parameter is $\lambda = 1$ and Player 2's learning parameter is $\mu = 0$, then the incompetent game $G_{1, 0}$ has the utility matrix
    \[
        R_{1, 0}
            = Q_1(1) R Q_2(0)^\transp
            =
            \begin{pmatrix}
                1 & 0 \\
                0 & 1 \\
            \end{pmatrix}
            \begin{pmatrix*}[r]
                1 & -1 \\
                3 & 1 \\
            \end{pmatrix*}
            \begin{pmatrix}
                \nicefrac{1}{2} & \nicefrac{1}{2} \\
                \nicefrac{1}{2} & \nicefrac{1}{2} \\
            \end{pmatrix}
            =
            \begin{pmatrix}
                0 & 0 \\
                2 & 2 \\
            \end{pmatrix}.
    \]
    Hence, $\vec{x}^* = (0, 1)$ is an optimal strategy for Player 1, $\vec{y}^* = (q, 1 - q)$ with $q \in [0, 1]$ is an optimal strategy for Player 2, and the game value is $\val(G_{1, 0}) = 2$.
    Alternatively, if Player 1's learning parameter is $\lambda = 0$ and Player 2's learning parameter is $\mu = 1$, then the utility matrix of $G_{0, 1}$ is
    \[
        R_{0, 1}
            = Q_1(0) R Q_2(1)^\transp
            =
            \begin{pmatrix}
                \nicefrac{1}{2} & \nicefrac{1}{2} \\
                \nicefrac{1}{2} & \nicefrac{1}{2} \\
            \end{pmatrix}
            \begin{pmatrix*}[r]
                1 & -1 \\
                3 & 1 \\
            \end{pmatrix*}
            \begin{pmatrix}
                 1 & 0 \\
                 0 & 1 \\
            \end{pmatrix}
            =
            \begin{pmatrix}
                2 & 0 \\
                2 & 0 \\
            \end{pmatrix}.
    \]
    This means that $\vec{x}^* = (p, 1 - p)$ with $p \in [0, 1]$ is an optimal strategy for Player 1, $\vec{y}^* = (0, 1)$ is an optimal strategy for Player 2, and the game value is $\val(G_{0, 1}) = 0$.
    The dependence of the game value $\val(G_{\lambda, \mu})$ on the learning parameters $\lambda, \mu \in [0, 1]$ is shown in \autoref{fig:parameterised-incompetent-games-a}.

    Finally, to motivate our subsequent discussion of the variational properties of parameterised incompetent games, several additional examples have been compiled in \autoref{tab:parameterised-incompetent-games} and \autoref{fig:parameterised-incompetent-games}.
    A different $G_\ms{Q(\funcdot)}$ is produced for every combination of a utility matrix $R$, Player 1's learning trajectory $Q_1(\funcdot)$, and Player 2's learning trajectory $Q_2(\funcdot)$.
    We will further explore the features of these parameterised incompetent games in \autoref{chp:a-strategic-perspective}.
    
    \begin{table}[b]
        \centering
        \caption[Collection of Parameterised Incompetent Games]{The utility matrices ($R$) and learning trajectories ($Q_1$ and $Q_2$) that define a collection of parameterised incompetent games.}
        \label{tab:parameterised-incompetent-games}
        % ---------------------------------------------------------------------------- %
% Honours Thesis                                                               %
% Table: Parameterised Incompetent Games                                       %
% ---------------------------------------------------------------------------- %

\renewcommand{\tabcolsep}{5pt}

\[
\begin{array}{ c c c c }
    \toprule
    & R & Q_1(\funcdot) & Q_2(\funcdot) \\ \midrule

    \Gape[8pt]{\text{\autoref{fig:parameterised-incompetent-games-a}}} &
        % Utility Matrix
        \begin{psmallmatrix*}[r] 
            -1 & -3 \\
            3 & 1 \\
        \end{psmallmatrix*} &
        % Player 1 Learning Trajectory
        \frac{1}{2}
        \begin{psmallmatrix}
            1 & 1 \\
            1 & 1 \\
        \end{psmallmatrix}
        (1 - \lambda) +
        \begin{psmallmatrix}
            1 & 0 \\
            0 & 1 \\
        \end{psmallmatrix}
        \lambda &
        % Player 2 Learning Trajectory
        \frac{1}{2}
        \begin{psmallmatrix}
            1 & 1 \\
            1 & 1 \\
        \end{psmallmatrix}
        (1 - \mu) +
        \begin{psmallmatrix}
            1 & 0 \\
            0 & 1 \\
        \end{psmallmatrix}
        \mu \\

    \Gape[8pt]{\text{\autoref{fig:parameterised-incompetent-games-b}}} &
        % Utility Matrix
        \begin{psmallmatrix*}[r] 
            3 & 0 \\
            -1 & -2 \\
        \end{psmallmatrix*} &
        % Player 1 Learning Trajectory
        \frac{1}{2}
        \begin{psmallmatrix}
            1 & 1 \\
            1 & 1 \\
        \end{psmallmatrix}
        (1 - \lambda) +
        \begin{psmallmatrix}
            1 & 0 \\
            0 & 1 \\
        \end{psmallmatrix}
        \lambda &
        % Player 2 Learning Trajectory
        \frac{1}{3}
        \begin{psmallmatrix}
            1 & 2 \\
            2 & 1 \\
        \end{psmallmatrix}
        (1 - \mu) +
        \begin{psmallmatrix}
            1 & 0 \\
            0 & 1 \\
        \end{psmallmatrix}
        \mu \\

    \Gape[8pt]{\text{\autoref{fig:parameterised-incompetent-games-c}}} &
        % Utility Matrix
        \begin{psmallmatrix*}[r] 
            3 & -2 \\
            -2 & 1 \\
        \end{psmallmatrix*} &
        % Player 1 Learning Trajectory
        \frac{1}{3}
        \begin{psmallmatrix}
            1 & 2 \\
            2 & 1 \\
        \end{psmallmatrix}
        (1 - \lambda) +
        \begin{psmallmatrix}
            1 & 0 \\
            0 & 1 \\
        \end{psmallmatrix}
        \lambda &
        % Player 2 Learning Trajectory
        \frac{1}{4}
        \begin{psmallmatrix}
            1 & 3 \\
            3 & 1 \\
        \end{psmallmatrix}
        (1 - \mu) +
        \begin{psmallmatrix}
            1 & 0 \\
            0 & 1 \\
        \end{psmallmatrix}
        \mu \\

    \Gape[8pt]{\text{\autoref{fig:parameterised-incompetent-games-d}}} &
        % Utility Matrix
        \begin{psmallmatrix*}[r] 
            3 & -3 \\
            -1 & 2 \\
        \end{psmallmatrix*} &
        % Player 1 Learning Trajectory
        \frac{1}{2}
        \begin{psmallmatrix}
            1 & 1 \\
            1 & 1 \\
        \end{psmallmatrix}
        (1 - \lambda) + \frac{1}{10}
        \begin{psmallmatrix}
            9 & 1 \\
            1 & 9 \\
        \end{psmallmatrix}
        \lambda &
        % Player 2 Learning Trajectory
        \frac{1}{10}
        \begin{psmallmatrix}
            2 & 8 \\
            5 & 5 \\
        \end{psmallmatrix}
        (1 - \mu) + \frac{1}{5}
        \begin{psmallmatrix}
            1 & 4 \\
            4 & 1 \\
        \end{psmallmatrix}
        \mu \\

    \Gape[8pt]{\text{\autoref{fig:parameterised-incompetent-games-e}}} &
        % Utility Matrix
        \begin{psmallmatrix*}[r] 
            0 & 3 & -2 \\
            -2 & 0 & 1 \\
            1 & -1 & 0 \\
        \end{psmallmatrix*} &
        % Player 1 Learning Trajectory
        \frac{1}{7}
        \begin{psmallmatrix}
            1 & 3 & 3 \\
            3 & 1 & 3 \\
            3 & 3 & 1 \\
        \end{psmallmatrix}
        (1 - \lambda) +
        \begin{psmallmatrix}
            1 & 0 & 0 \\
            0 & 1 & 0 \\
            0 & 0 & 1 \\
        \end{psmallmatrix}
        \lambda &
        % Player 2 Learning Trajectory
        \frac{1}{7}
        \begin{psmallmatrix}
            1 & 3 & 3 \\
            3 & 1 & 3 \\
            3 & 3 & 1 \\
        \end{psmallmatrix}
        (1 - \mu) +
        \begin{psmallmatrix}
            1 & 0 & 0 \\
            0 & 1 & 0 \\
            0 & 0 & 1 \\
        \end{psmallmatrix}
        \mu \\

    \Gape[8pt]{\text{\autoref{fig:parameterised-incompetent-games-f}}} &
        % Utility Matrix
        \begin{psmallmatrix*}[r] 
            2 & 3 & -1 \\
            0 & -3 & 0 \\
            -3 & 0 & 2 \\
        \end{psmallmatrix*} &
        % Player 1 Learning Trajectory
        \frac{1}{7}
        \begin{psmallmatrix}
            1 & 3 & 3 \\
            2 & 2 & 3 \\
            1 & 4 & 2 \\
        \end{psmallmatrix}
        (1 - \lambda) +
        \begin{psmallmatrix}
            1 & 0 & 0 \\
            0 & 1 & 0 \\
            0 & 0 & 1 \\
        \end{psmallmatrix}
        \lambda &
        % Player 2 Learning Trajectory
        \frac{1}{7}
        \begin{psmallmatrix}
            0 & 0 & 7 \\
            4 & 1 & 2 \\
            4 & 3 & 0 \\
        \end{psmallmatrix}
        (1 - \mu) +
        \begin{psmallmatrix}
            1 & 0 & 0 \\
            0 & 1 & 0 \\
            0 & 0 & 1 \\
        \end{psmallmatrix}
        \mu \\ \bottomrule
\end{array}
\]

\renewcommand{\tabcolsep}{6pt}

    \end{table}



    \begin{figure}[p]
        \centerfloat
        \begin{minipage}{\textwidth + 1.5in}
            \subbottom[\label{fig:parameterised-incompetent-games-a}]%
                {%% Creator: Matplotlib, PGF backend
%%
%% To include the figure in your LaTeX document, write
%%   \input{<filename>.pgf}
%%
%% Make sure the required packages are loaded in your preamble
%%   \usepackage{pgf}
%%
%% Figures using additional raster images can only be included by \input if
%% they are in the same directory as the main LaTeX file. For loading figures
%% from other directories you can use the `import` package
%%   \usepackage{import}
%% and then include the figures with
%%   \import{<path to file>}{<filename>.pgf}
%%
%% Matplotlib used the following preamble
%%   \usepackage{fontspec}
%%   \setmainfont{DejaVuSerif.ttf}[Path=C:/Users/Thomas/anaconda3/lib/site-packages/matplotlib/mpl-data/fonts/ttf/]
%%   \setsansfont{DejaVuSans.ttf}[Path=C:/Users/Thomas/anaconda3/lib/site-packages/matplotlib/mpl-data/fonts/ttf/]
%%   \setmonofont{DejaVuSansMono.ttf}[Path=C:/Users/Thomas/anaconda3/lib/site-packages/matplotlib/mpl-data/fonts/ttf/]
%%
\begingroup%
\makeatletter%
\begin{pgfpicture}%
\pgfpathrectangle{\pgfpointorigin}{\pgfqpoint{3.000000in}{2.250000in}}%
\pgfusepath{use as bounding box, clip}%
\begin{pgfscope}%
\pgfsetbuttcap%
\pgfsetmiterjoin%
\definecolor{currentfill}{rgb}{1.000000,1.000000,1.000000}%
\pgfsetfillcolor{currentfill}%
\pgfsetlinewidth{0.000000pt}%
\definecolor{currentstroke}{rgb}{1.000000,1.000000,1.000000}%
\pgfsetstrokecolor{currentstroke}%
\pgfsetdash{}{0pt}%
\pgfpathmoveto{\pgfqpoint{0.000000in}{0.000000in}}%
\pgfpathlineto{\pgfqpoint{3.000000in}{0.000000in}}%
\pgfpathlineto{\pgfqpoint{3.000000in}{2.250000in}}%
\pgfpathlineto{\pgfqpoint{0.000000in}{2.250000in}}%
\pgfpathclose%
\pgfusepath{fill}%
\end{pgfscope}%
\begin{pgfscope}%
\pgfsetbuttcap%
\pgfsetmiterjoin%
\definecolor{currentfill}{rgb}{1.000000,1.000000,1.000000}%
\pgfsetfillcolor{currentfill}%
\pgfsetlinewidth{0.000000pt}%
\definecolor{currentstroke}{rgb}{0.000000,0.000000,0.000000}%
\pgfsetstrokecolor{currentstroke}%
\pgfsetstrokeopacity{0.000000}%
\pgfsetdash{}{0pt}%
\pgfpathmoveto{\pgfqpoint{0.150000in}{0.150000in}}%
\pgfpathlineto{\pgfqpoint{2.850000in}{0.150000in}}%
\pgfpathlineto{\pgfqpoint{2.850000in}{2.100000in}}%
\pgfpathlineto{\pgfqpoint{0.150000in}{2.100000in}}%
\pgfpathclose%
\pgfusepath{fill}%
\end{pgfscope}%
\begin{pgfscope}%
\pgfsetbuttcap%
\pgfsetmiterjoin%
\definecolor{currentfill}{rgb}{0.950000,0.950000,0.950000}%
\pgfsetfillcolor{currentfill}%
\pgfsetfillopacity{0.500000}%
\pgfsetlinewidth{1.003750pt}%
\definecolor{currentstroke}{rgb}{0.950000,0.950000,0.950000}%
\pgfsetstrokecolor{currentstroke}%
\pgfsetstrokeopacity{0.500000}%
\pgfsetdash{}{0pt}%
\pgfpathmoveto{\pgfqpoint{2.573296in}{0.776948in}}%
\pgfpathlineto{\pgfqpoint{1.536486in}{1.299017in}}%
\pgfpathlineto{\pgfqpoint{1.536486in}{2.074448in}}%
\pgfpathlineto{\pgfqpoint{2.652584in}{1.554387in}}%
\pgfusepath{stroke,fill}%
\end{pgfscope}%
\begin{pgfscope}%
\pgfsetbuttcap%
\pgfsetmiterjoin%
\definecolor{currentfill}{rgb}{0.900000,0.900000,0.900000}%
\pgfsetfillcolor{currentfill}%
\pgfsetfillopacity{0.500000}%
\pgfsetlinewidth{1.003750pt}%
\definecolor{currentstroke}{rgb}{0.900000,0.900000,0.900000}%
\pgfsetstrokecolor{currentstroke}%
\pgfsetstrokeopacity{0.500000}%
\pgfsetdash{}{0pt}%
\pgfpathmoveto{\pgfqpoint{0.499677in}{0.776948in}}%
\pgfpathlineto{\pgfqpoint{1.536486in}{1.299017in}}%
\pgfpathlineto{\pgfqpoint{1.536486in}{2.074448in}}%
\pgfpathlineto{\pgfqpoint{0.420389in}{1.554387in}}%
\pgfusepath{stroke,fill}%
\end{pgfscope}%
\begin{pgfscope}%
\pgfsetbuttcap%
\pgfsetmiterjoin%
\definecolor{currentfill}{rgb}{0.925000,0.925000,0.925000}%
\pgfsetfillcolor{currentfill}%
\pgfsetfillopacity{0.500000}%
\pgfsetlinewidth{1.003750pt}%
\definecolor{currentstroke}{rgb}{0.925000,0.925000,0.925000}%
\pgfsetstrokecolor{currentstroke}%
\pgfsetstrokeopacity{0.500000}%
\pgfsetdash{}{0pt}%
\pgfpathmoveto{\pgfqpoint{1.536486in}{0.199655in}}%
\pgfpathlineto{\pgfqpoint{2.573296in}{0.776948in}}%
\pgfpathlineto{\pgfqpoint{1.536486in}{1.299017in}}%
\pgfpathlineto{\pgfqpoint{0.499677in}{0.776948in}}%
\pgfusepath{stroke,fill}%
\end{pgfscope}%
\begin{pgfscope}%
\pgfsetrectcap%
\pgfsetroundjoin%
\pgfsetlinewidth{0.803000pt}%
\definecolor{currentstroke}{rgb}{0.000000,0.000000,0.000000}%
\pgfsetstrokecolor{currentstroke}%
\pgfsetdash{}{0pt}%
\pgfpathmoveto{\pgfqpoint{2.573296in}{0.776948in}}%
\pgfpathlineto{\pgfqpoint{1.536486in}{0.199655in}}%
\pgfusepath{stroke}%
\end{pgfscope}%
\begin{pgfscope}%
\definecolor{textcolor}{rgb}{0.000000,0.000000,0.000000}%
\pgfsetstrokecolor{textcolor}%
\pgfsetfillcolor{textcolor}%
\pgftext[x=2.017747in,y=0.045475in,left,base,rotate=29.108966]{\color{textcolor}\sffamily\fontsize{8.000000}{9.600000}\selectfont Player 2 (\(\displaystyle \mu\))}%
\end{pgfscope}%
\begin{pgfscope}%
\pgfsetbuttcap%
\pgfsetroundjoin%
\pgfsetlinewidth{0.803000pt}%
\definecolor{currentstroke}{rgb}{0.690196,0.690196,0.690196}%
\pgfsetstrokecolor{currentstroke}%
\pgfsetdash{}{0pt}%
\pgfpathmoveto{\pgfqpoint{1.605722in}{0.238205in}}%
\pgfpathlineto{\pgfqpoint{0.568749in}{0.811728in}}%
\pgfpathlineto{\pgfqpoint{0.494997in}{1.589151in}}%
\pgfusepath{stroke}%
\end{pgfscope}%
\begin{pgfscope}%
\pgfsetbuttcap%
\pgfsetroundjoin%
\pgfsetlinewidth{0.803000pt}%
\definecolor{currentstroke}{rgb}{0.690196,0.690196,0.690196}%
\pgfsetstrokecolor{currentstroke}%
\pgfsetdash{}{0pt}%
\pgfpathmoveto{\pgfqpoint{1.793262in}{0.342627in}}%
\pgfpathlineto{\pgfqpoint{0.755965in}{0.905998in}}%
\pgfpathlineto{\pgfqpoint{0.697035in}{1.683294in}}%
\pgfusepath{stroke}%
\end{pgfscope}%
\begin{pgfscope}%
\pgfsetbuttcap%
\pgfsetroundjoin%
\pgfsetlinewidth{0.803000pt}%
\definecolor{currentstroke}{rgb}{0.690196,0.690196,0.690196}%
\pgfsetstrokecolor{currentstroke}%
\pgfsetdash{}{0pt}%
\pgfpathmoveto{\pgfqpoint{1.977414in}{0.445162in}}%
\pgfpathlineto{\pgfqpoint{0.939964in}{0.998647in}}%
\pgfpathlineto{\pgfqpoint{0.895342in}{1.775698in}}%
\pgfusepath{stroke}%
\end{pgfscope}%
\begin{pgfscope}%
\pgfsetbuttcap%
\pgfsetroundjoin%
\pgfsetlinewidth{0.803000pt}%
\definecolor{currentstroke}{rgb}{0.690196,0.690196,0.690196}%
\pgfsetstrokecolor{currentstroke}%
\pgfsetdash{}{0pt}%
\pgfpathmoveto{\pgfqpoint{2.158267in}{0.545861in}}%
\pgfpathlineto{\pgfqpoint{1.120829in}{1.089719in}}%
\pgfpathlineto{\pgfqpoint{1.090021in}{1.866411in}}%
\pgfusepath{stroke}%
\end{pgfscope}%
\begin{pgfscope}%
\pgfsetbuttcap%
\pgfsetroundjoin%
\pgfsetlinewidth{0.803000pt}%
\definecolor{currentstroke}{rgb}{0.690196,0.690196,0.690196}%
\pgfsetstrokecolor{currentstroke}%
\pgfsetdash{}{0pt}%
\pgfpathmoveto{\pgfqpoint{2.335912in}{0.644773in}}%
\pgfpathlineto{\pgfqpoint{1.298639in}{1.179253in}}%
\pgfpathlineto{\pgfqpoint{1.281170in}{1.955480in}}%
\pgfusepath{stroke}%
\end{pgfscope}%
\begin{pgfscope}%
\pgfsetbuttcap%
\pgfsetroundjoin%
\pgfsetlinewidth{0.803000pt}%
\definecolor{currentstroke}{rgb}{0.690196,0.690196,0.690196}%
\pgfsetstrokecolor{currentstroke}%
\pgfsetdash{}{0pt}%
\pgfpathmoveto{\pgfqpoint{2.510430in}{0.741945in}}%
\pgfpathlineto{\pgfqpoint{1.473472in}{1.267287in}}%
\pgfpathlineto{\pgfqpoint{1.468885in}{2.042948in}}%
\pgfusepath{stroke}%
\end{pgfscope}%
\begin{pgfscope}%
\pgfsetrectcap%
\pgfsetroundjoin%
\pgfsetlinewidth{0.803000pt}%
\definecolor{currentstroke}{rgb}{0.000000,0.000000,0.000000}%
\pgfsetstrokecolor{currentstroke}%
\pgfsetdash{}{0pt}%
\pgfpathmoveto{\pgfqpoint{1.596992in}{0.243033in}}%
\pgfpathlineto{\pgfqpoint{1.623203in}{0.228537in}}%
\pgfusepath{stroke}%
\end{pgfscope}%
\begin{pgfscope}%
\definecolor{textcolor}{rgb}{0.000000,0.000000,0.000000}%
\pgfsetstrokecolor{textcolor}%
\pgfsetfillcolor{textcolor}%
\pgftext[x=1.680378in,y=0.147403in,,top]{\color{textcolor}\sffamily\fontsize{6.000000}{7.200000}\selectfont \(\displaystyle 0.0\)}%
\end{pgfscope}%
\begin{pgfscope}%
\pgfsetrectcap%
\pgfsetroundjoin%
\pgfsetlinewidth{0.803000pt}%
\definecolor{currentstroke}{rgb}{0.000000,0.000000,0.000000}%
\pgfsetstrokecolor{currentstroke}%
\pgfsetdash{}{0pt}%
\pgfpathmoveto{\pgfqpoint{1.784534in}{0.347367in}}%
\pgfpathlineto{\pgfqpoint{1.810740in}{0.333134in}}%
\pgfusepath{stroke}%
\end{pgfscope}%
\begin{pgfscope}%
\definecolor{textcolor}{rgb}{0.000000,0.000000,0.000000}%
\pgfsetstrokecolor{textcolor}%
\pgfsetfillcolor{textcolor}%
\pgftext[x=1.866959in,y=0.252496in,,top]{\color{textcolor}\sffamily\fontsize{6.000000}{7.200000}\selectfont \(\displaystyle 0.2\)}%
\end{pgfscope}%
\begin{pgfscope}%
\pgfsetrectcap%
\pgfsetroundjoin%
\pgfsetlinewidth{0.803000pt}%
\definecolor{currentstroke}{rgb}{0.000000,0.000000,0.000000}%
\pgfsetstrokecolor{currentstroke}%
\pgfsetdash{}{0pt}%
\pgfpathmoveto{\pgfqpoint{1.968688in}{0.449817in}}%
\pgfpathlineto{\pgfqpoint{1.994886in}{0.435840in}}%
\pgfusepath{stroke}%
\end{pgfscope}%
\begin{pgfscope}%
\definecolor{textcolor}{rgb}{0.000000,0.000000,0.000000}%
\pgfsetstrokecolor{textcolor}%
\pgfsetfillcolor{textcolor}%
\pgftext[x=2.050175in,y=0.355693in,,top]{\color{textcolor}\sffamily\fontsize{6.000000}{7.200000}\selectfont \(\displaystyle 0.4\)}%
\end{pgfscope}%
\begin{pgfscope}%
\pgfsetrectcap%
\pgfsetroundjoin%
\pgfsetlinewidth{0.803000pt}%
\definecolor{currentstroke}{rgb}{0.000000,0.000000,0.000000}%
\pgfsetstrokecolor{currentstroke}%
\pgfsetdash{}{0pt}%
\pgfpathmoveto{\pgfqpoint{2.149546in}{0.550433in}}%
\pgfpathlineto{\pgfqpoint{2.175732in}{0.536706in}}%
\pgfusepath{stroke}%
\end{pgfscope}%
\begin{pgfscope}%
\definecolor{textcolor}{rgb}{0.000000,0.000000,0.000000}%
\pgfsetstrokecolor{textcolor}%
\pgfsetfillcolor{textcolor}%
\pgftext[x=2.230114in,y=0.457045in,,top]{\color{textcolor}\sffamily\fontsize{6.000000}{7.200000}\selectfont \(\displaystyle 0.6\)}%
\end{pgfscope}%
\begin{pgfscope}%
\pgfsetrectcap%
\pgfsetroundjoin%
\pgfsetlinewidth{0.803000pt}%
\definecolor{currentstroke}{rgb}{0.000000,0.000000,0.000000}%
\pgfsetstrokecolor{currentstroke}%
\pgfsetdash{}{0pt}%
\pgfpathmoveto{\pgfqpoint{2.327195in}{0.649264in}}%
\pgfpathlineto{\pgfqpoint{2.353366in}{0.635779in}}%
\pgfusepath{stroke}%
\end{pgfscope}%
\begin{pgfscope}%
\definecolor{textcolor}{rgb}{0.000000,0.000000,0.000000}%
\pgfsetstrokecolor{textcolor}%
\pgfsetfillcolor{textcolor}%
\pgftext[x=2.406864in,y=0.556601in,,top]{\color{textcolor}\sffamily\fontsize{6.000000}{7.200000}\selectfont \(\displaystyle 0.8\)}%
\end{pgfscope}%
\begin{pgfscope}%
\pgfsetrectcap%
\pgfsetroundjoin%
\pgfsetlinewidth{0.803000pt}%
\definecolor{currentstroke}{rgb}{0.000000,0.000000,0.000000}%
\pgfsetstrokecolor{currentstroke}%
\pgfsetdash{}{0pt}%
\pgfpathmoveto{\pgfqpoint{2.501720in}{0.746357in}}%
\pgfpathlineto{\pgfqpoint{2.527872in}{0.733108in}}%
\pgfusepath{stroke}%
\end{pgfscope}%
\begin{pgfscope}%
\definecolor{textcolor}{rgb}{0.000000,0.000000,0.000000}%
\pgfsetstrokecolor{textcolor}%
\pgfsetfillcolor{textcolor}%
\pgftext[x=2.580510in,y=0.654408in,,top]{\color{textcolor}\sffamily\fontsize{6.000000}{7.200000}\selectfont \(\displaystyle 1.0\)}%
\end{pgfscope}%
\begin{pgfscope}%
\pgfsetrectcap%
\pgfsetroundjoin%
\pgfsetlinewidth{0.803000pt}%
\definecolor{currentstroke}{rgb}{0.000000,0.000000,0.000000}%
\pgfsetstrokecolor{currentstroke}%
\pgfsetdash{}{0pt}%
\pgfpathmoveto{\pgfqpoint{0.499677in}{0.776948in}}%
\pgfpathlineto{\pgfqpoint{1.536486in}{0.199655in}}%
\pgfusepath{stroke}%
\end{pgfscope}%
\begin{pgfscope}%
\definecolor{textcolor}{rgb}{0.000000,0.000000,0.000000}%
\pgfsetstrokecolor{textcolor}%
\pgfsetfillcolor{textcolor}%
\pgftext[x=0.492803in,y=0.358631in,left,base,rotate=330.891034]{\color{textcolor}\sffamily\fontsize{8.000000}{9.600000}\selectfont Player 1 (\(\displaystyle \lambda\))}%
\end{pgfscope}%
\begin{pgfscope}%
\pgfsetbuttcap%
\pgfsetroundjoin%
\pgfsetlinewidth{0.803000pt}%
\definecolor{currentstroke}{rgb}{0.690196,0.690196,0.690196}%
\pgfsetstrokecolor{currentstroke}%
\pgfsetdash{}{0pt}%
\pgfpathmoveto{\pgfqpoint{2.577976in}{1.589151in}}%
\pgfpathlineto{\pgfqpoint{2.504223in}{0.811728in}}%
\pgfpathlineto{\pgfqpoint{1.467251in}{0.238205in}}%
\pgfusepath{stroke}%
\end{pgfscope}%
\begin{pgfscope}%
\pgfsetbuttcap%
\pgfsetroundjoin%
\pgfsetlinewidth{0.803000pt}%
\definecolor{currentstroke}{rgb}{0.690196,0.690196,0.690196}%
\pgfsetstrokecolor{currentstroke}%
\pgfsetdash{}{0pt}%
\pgfpathmoveto{\pgfqpoint{2.375938in}{1.683294in}}%
\pgfpathlineto{\pgfqpoint{2.317008in}{0.905998in}}%
\pgfpathlineto{\pgfqpoint{1.279711in}{0.342627in}}%
\pgfusepath{stroke}%
\end{pgfscope}%
\begin{pgfscope}%
\pgfsetbuttcap%
\pgfsetroundjoin%
\pgfsetlinewidth{0.803000pt}%
\definecolor{currentstroke}{rgb}{0.690196,0.690196,0.690196}%
\pgfsetstrokecolor{currentstroke}%
\pgfsetdash{}{0pt}%
\pgfpathmoveto{\pgfqpoint{2.177631in}{1.775698in}}%
\pgfpathlineto{\pgfqpoint{2.133009in}{0.998647in}}%
\pgfpathlineto{\pgfqpoint{1.095559in}{0.445162in}}%
\pgfusepath{stroke}%
\end{pgfscope}%
\begin{pgfscope}%
\pgfsetbuttcap%
\pgfsetroundjoin%
\pgfsetlinewidth{0.803000pt}%
\definecolor{currentstroke}{rgb}{0.690196,0.690196,0.690196}%
\pgfsetstrokecolor{currentstroke}%
\pgfsetdash{}{0pt}%
\pgfpathmoveto{\pgfqpoint{1.982952in}{1.866411in}}%
\pgfpathlineto{\pgfqpoint{1.952144in}{1.089719in}}%
\pgfpathlineto{\pgfqpoint{0.914705in}{0.545861in}}%
\pgfusepath{stroke}%
\end{pgfscope}%
\begin{pgfscope}%
\pgfsetbuttcap%
\pgfsetroundjoin%
\pgfsetlinewidth{0.803000pt}%
\definecolor{currentstroke}{rgb}{0.690196,0.690196,0.690196}%
\pgfsetstrokecolor{currentstroke}%
\pgfsetdash{}{0pt}%
\pgfpathmoveto{\pgfqpoint{1.791803in}{1.955480in}}%
\pgfpathlineto{\pgfqpoint{1.774334in}{1.179253in}}%
\pgfpathlineto{\pgfqpoint{0.737061in}{0.644773in}}%
\pgfusepath{stroke}%
\end{pgfscope}%
\begin{pgfscope}%
\pgfsetbuttcap%
\pgfsetroundjoin%
\pgfsetlinewidth{0.803000pt}%
\definecolor{currentstroke}{rgb}{0.690196,0.690196,0.690196}%
\pgfsetstrokecolor{currentstroke}%
\pgfsetdash{}{0pt}%
\pgfpathmoveto{\pgfqpoint{1.604088in}{2.042948in}}%
\pgfpathlineto{\pgfqpoint{1.599501in}{1.267287in}}%
\pgfpathlineto{\pgfqpoint{0.562543in}{0.741945in}}%
\pgfusepath{stroke}%
\end{pgfscope}%
\begin{pgfscope}%
\pgfsetrectcap%
\pgfsetroundjoin%
\pgfsetlinewidth{0.803000pt}%
\definecolor{currentstroke}{rgb}{0.000000,0.000000,0.000000}%
\pgfsetstrokecolor{currentstroke}%
\pgfsetdash{}{0pt}%
\pgfpathmoveto{\pgfqpoint{1.475981in}{0.243033in}}%
\pgfpathlineto{\pgfqpoint{1.449770in}{0.228537in}}%
\pgfusepath{stroke}%
\end{pgfscope}%
\begin{pgfscope}%
\definecolor{textcolor}{rgb}{0.000000,0.000000,0.000000}%
\pgfsetstrokecolor{textcolor}%
\pgfsetfillcolor{textcolor}%
\pgftext[x=1.392595in,y=0.147403in,,top]{\color{textcolor}\sffamily\fontsize{6.000000}{7.200000}\selectfont \(\displaystyle 0.0\)}%
\end{pgfscope}%
\begin{pgfscope}%
\pgfsetrectcap%
\pgfsetroundjoin%
\pgfsetlinewidth{0.803000pt}%
\definecolor{currentstroke}{rgb}{0.000000,0.000000,0.000000}%
\pgfsetstrokecolor{currentstroke}%
\pgfsetdash{}{0pt}%
\pgfpathmoveto{\pgfqpoint{1.288439in}{0.347367in}}%
\pgfpathlineto{\pgfqpoint{1.262233in}{0.333134in}}%
\pgfusepath{stroke}%
\end{pgfscope}%
\begin{pgfscope}%
\definecolor{textcolor}{rgb}{0.000000,0.000000,0.000000}%
\pgfsetstrokecolor{textcolor}%
\pgfsetfillcolor{textcolor}%
\pgftext[x=1.206013in,y=0.252496in,,top]{\color{textcolor}\sffamily\fontsize{6.000000}{7.200000}\selectfont \(\displaystyle 0.2\)}%
\end{pgfscope}%
\begin{pgfscope}%
\pgfsetrectcap%
\pgfsetroundjoin%
\pgfsetlinewidth{0.803000pt}%
\definecolor{currentstroke}{rgb}{0.000000,0.000000,0.000000}%
\pgfsetstrokecolor{currentstroke}%
\pgfsetdash{}{0pt}%
\pgfpathmoveto{\pgfqpoint{1.104285in}{0.449817in}}%
\pgfpathlineto{\pgfqpoint{1.078087in}{0.435840in}}%
\pgfusepath{stroke}%
\end{pgfscope}%
\begin{pgfscope}%
\definecolor{textcolor}{rgb}{0.000000,0.000000,0.000000}%
\pgfsetstrokecolor{textcolor}%
\pgfsetfillcolor{textcolor}%
\pgftext[x=1.022798in,y=0.355693in,,top]{\color{textcolor}\sffamily\fontsize{6.000000}{7.200000}\selectfont \(\displaystyle 0.4\)}%
\end{pgfscope}%
\begin{pgfscope}%
\pgfsetrectcap%
\pgfsetroundjoin%
\pgfsetlinewidth{0.803000pt}%
\definecolor{currentstroke}{rgb}{0.000000,0.000000,0.000000}%
\pgfsetstrokecolor{currentstroke}%
\pgfsetdash{}{0pt}%
\pgfpathmoveto{\pgfqpoint{0.923427in}{0.550433in}}%
\pgfpathlineto{\pgfqpoint{0.897241in}{0.536706in}}%
\pgfusepath{stroke}%
\end{pgfscope}%
\begin{pgfscope}%
\definecolor{textcolor}{rgb}{0.000000,0.000000,0.000000}%
\pgfsetstrokecolor{textcolor}%
\pgfsetfillcolor{textcolor}%
\pgftext[x=0.842859in,y=0.457045in,,top]{\color{textcolor}\sffamily\fontsize{6.000000}{7.200000}\selectfont \(\displaystyle 0.6\)}%
\end{pgfscope}%
\begin{pgfscope}%
\pgfsetrectcap%
\pgfsetroundjoin%
\pgfsetlinewidth{0.803000pt}%
\definecolor{currentstroke}{rgb}{0.000000,0.000000,0.000000}%
\pgfsetstrokecolor{currentstroke}%
\pgfsetdash{}{0pt}%
\pgfpathmoveto{\pgfqpoint{0.745778in}{0.649264in}}%
\pgfpathlineto{\pgfqpoint{0.719607in}{0.635779in}}%
\pgfusepath{stroke}%
\end{pgfscope}%
\begin{pgfscope}%
\definecolor{textcolor}{rgb}{0.000000,0.000000,0.000000}%
\pgfsetstrokecolor{textcolor}%
\pgfsetfillcolor{textcolor}%
\pgftext[x=0.666109in,y=0.556601in,,top]{\color{textcolor}\sffamily\fontsize{6.000000}{7.200000}\selectfont \(\displaystyle 0.8\)}%
\end{pgfscope}%
\begin{pgfscope}%
\pgfsetrectcap%
\pgfsetroundjoin%
\pgfsetlinewidth{0.803000pt}%
\definecolor{currentstroke}{rgb}{0.000000,0.000000,0.000000}%
\pgfsetstrokecolor{currentstroke}%
\pgfsetdash{}{0pt}%
\pgfpathmoveto{\pgfqpoint{0.571253in}{0.746357in}}%
\pgfpathlineto{\pgfqpoint{0.545101in}{0.733108in}}%
\pgfusepath{stroke}%
\end{pgfscope}%
\begin{pgfscope}%
\definecolor{textcolor}{rgb}{0.000000,0.000000,0.000000}%
\pgfsetstrokecolor{textcolor}%
\pgfsetfillcolor{textcolor}%
\pgftext[x=0.492463in,y=0.654408in,,top]{\color{textcolor}\sffamily\fontsize{6.000000}{7.200000}\selectfont \(\displaystyle 1.0\)}%
\end{pgfscope}%
\begin{pgfscope}%
\pgfsetrectcap%
\pgfsetroundjoin%
\pgfsetlinewidth{0.803000pt}%
\definecolor{currentstroke}{rgb}{0.000000,0.000000,0.000000}%
\pgfsetstrokecolor{currentstroke}%
\pgfsetdash{}{0pt}%
\pgfpathmoveto{\pgfqpoint{0.499677in}{0.776948in}}%
\pgfpathlineto{\pgfqpoint{0.420389in}{1.554387in}}%
\pgfusepath{stroke}%
\end{pgfscope}%
\begin{pgfscope}%
\definecolor{textcolor}{rgb}{0.000000,0.000000,0.000000}%
\pgfsetstrokecolor{textcolor}%
\pgfsetfillcolor{textcolor}%
\pgftext[x=0.041630in,y=1.401767in,left,base,rotate=275.823265]{\color{textcolor}\sffamily\fontsize{8.000000}{9.600000}\selectfont \(\displaystyle \mathsf{val}(G_{\lambda, \mu}\))}%
\end{pgfscope}%
\begin{pgfscope}%
\pgfsetbuttcap%
\pgfsetroundjoin%
\pgfsetlinewidth{0.803000pt}%
\definecolor{currentstroke}{rgb}{0.690196,0.690196,0.690196}%
\pgfsetstrokecolor{currentstroke}%
\pgfsetdash{}{0pt}%
\pgfpathmoveto{\pgfqpoint{0.498202in}{0.791413in}}%
\pgfpathlineto{\pgfqpoint{1.536486in}{1.313496in}}%
\pgfpathlineto{\pgfqpoint{2.574771in}{0.791413in}}%
\pgfusepath{stroke}%
\end{pgfscope}%
\begin{pgfscope}%
\pgfsetbuttcap%
\pgfsetroundjoin%
\pgfsetlinewidth{0.803000pt}%
\definecolor{currentstroke}{rgb}{0.690196,0.690196,0.690196}%
\pgfsetstrokecolor{currentstroke}%
\pgfsetdash{}{0pt}%
\pgfpathmoveto{\pgfqpoint{0.486247in}{0.908629in}}%
\pgfpathlineto{\pgfqpoint{1.536486in}{1.430757in}}%
\pgfpathlineto{\pgfqpoint{2.586726in}{0.908629in}}%
\pgfusepath{stroke}%
\end{pgfscope}%
\begin{pgfscope}%
\pgfsetbuttcap%
\pgfsetroundjoin%
\pgfsetlinewidth{0.803000pt}%
\definecolor{currentstroke}{rgb}{0.690196,0.690196,0.690196}%
\pgfsetstrokecolor{currentstroke}%
\pgfsetdash{}{0pt}%
\pgfpathmoveto{\pgfqpoint{0.474014in}{1.028577in}}%
\pgfpathlineto{\pgfqpoint{1.536486in}{1.550616in}}%
\pgfpathlineto{\pgfqpoint{2.598959in}{1.028577in}}%
\pgfusepath{stroke}%
\end{pgfscope}%
\begin{pgfscope}%
\pgfsetbuttcap%
\pgfsetroundjoin%
\pgfsetlinewidth{0.803000pt}%
\definecolor{currentstroke}{rgb}{0.690196,0.690196,0.690196}%
\pgfsetstrokecolor{currentstroke}%
\pgfsetdash{}{0pt}%
\pgfpathmoveto{\pgfqpoint{0.461493in}{1.151351in}}%
\pgfpathlineto{\pgfqpoint{1.536486in}{1.673160in}}%
\pgfpathlineto{\pgfqpoint{2.611480in}{1.151351in}}%
\pgfusepath{stroke}%
\end{pgfscope}%
\begin{pgfscope}%
\pgfsetbuttcap%
\pgfsetroundjoin%
\pgfsetlinewidth{0.803000pt}%
\definecolor{currentstroke}{rgb}{0.690196,0.690196,0.690196}%
\pgfsetstrokecolor{currentstroke}%
\pgfsetdash{}{0pt}%
\pgfpathmoveto{\pgfqpoint{0.448673in}{1.277054in}}%
\pgfpathlineto{\pgfqpoint{1.536486in}{1.798481in}}%
\pgfpathlineto{\pgfqpoint{2.624300in}{1.277054in}}%
\pgfusepath{stroke}%
\end{pgfscope}%
\begin{pgfscope}%
\pgfsetbuttcap%
\pgfsetroundjoin%
\pgfsetlinewidth{0.803000pt}%
\definecolor{currentstroke}{rgb}{0.690196,0.690196,0.690196}%
\pgfsetstrokecolor{currentstroke}%
\pgfsetdash{}{0pt}%
\pgfpathmoveto{\pgfqpoint{0.435544in}{1.405791in}}%
\pgfpathlineto{\pgfqpoint{1.536486in}{1.926674in}}%
\pgfpathlineto{\pgfqpoint{2.637429in}{1.405791in}}%
\pgfusepath{stroke}%
\end{pgfscope}%
\begin{pgfscope}%
\pgfsetbuttcap%
\pgfsetroundjoin%
\pgfsetlinewidth{0.803000pt}%
\definecolor{currentstroke}{rgb}{0.690196,0.690196,0.690196}%
\pgfsetstrokecolor{currentstroke}%
\pgfsetdash{}{0pt}%
\pgfpathmoveto{\pgfqpoint{0.422093in}{1.537674in}}%
\pgfpathlineto{\pgfqpoint{1.536486in}{2.057839in}}%
\pgfpathlineto{\pgfqpoint{2.650880in}{1.537674in}}%
\pgfusepath{stroke}%
\end{pgfscope}%
\begin{pgfscope}%
\pgfsetrectcap%
\pgfsetroundjoin%
\pgfsetlinewidth{0.803000pt}%
\definecolor{currentstroke}{rgb}{0.000000,0.000000,0.000000}%
\pgfsetstrokecolor{currentstroke}%
\pgfsetdash{}{0pt}%
\pgfpathmoveto{\pgfqpoint{0.506923in}{0.795798in}}%
\pgfpathlineto{\pgfqpoint{0.480740in}{0.782632in}}%
\pgfusepath{stroke}%
\end{pgfscope}%
\begin{pgfscope}%
\definecolor{textcolor}{rgb}{0.000000,0.000000,0.000000}%
\pgfsetstrokecolor{textcolor}%
\pgfsetfillcolor{textcolor}%
\pgftext[x=0.355298in,y=0.791413in,,top]{\color{textcolor}\sffamily\fontsize{6.000000}{7.200000}\selectfont \(\displaystyle -1.0\)}%
\end{pgfscope}%
\begin{pgfscope}%
\pgfsetrectcap%
\pgfsetroundjoin%
\pgfsetlinewidth{0.803000pt}%
\definecolor{currentstroke}{rgb}{0.000000,0.000000,0.000000}%
\pgfsetstrokecolor{currentstroke}%
\pgfsetdash{}{0pt}%
\pgfpathmoveto{\pgfqpoint{0.495073in}{0.913017in}}%
\pgfpathlineto{\pgfqpoint{0.468574in}{0.899843in}}%
\pgfusepath{stroke}%
\end{pgfscope}%
\begin{pgfscope}%
\definecolor{textcolor}{rgb}{0.000000,0.000000,0.000000}%
\pgfsetstrokecolor{textcolor}%
\pgfsetfillcolor{textcolor}%
\pgftext[x=0.341698in,y=0.908629in,,top]{\color{textcolor}\sffamily\fontsize{6.000000}{7.200000}\selectfont \(\displaystyle -0.5\)}%
\end{pgfscope}%
\begin{pgfscope}%
\pgfsetrectcap%
\pgfsetroundjoin%
\pgfsetlinewidth{0.803000pt}%
\definecolor{currentstroke}{rgb}{0.000000,0.000000,0.000000}%
\pgfsetstrokecolor{currentstroke}%
\pgfsetdash{}{0pt}%
\pgfpathmoveto{\pgfqpoint{0.482948in}{1.032966in}}%
\pgfpathlineto{\pgfqpoint{0.456125in}{1.019787in}}%
\pgfusepath{stroke}%
\end{pgfscope}%
\begin{pgfscope}%
\definecolor{textcolor}{rgb}{0.000000,0.000000,0.000000}%
\pgfsetstrokecolor{textcolor}%
\pgfsetfillcolor{textcolor}%
\pgftext[x=0.327782in,y=1.028577in,,top]{\color{textcolor}\sffamily\fontsize{6.000000}{7.200000}\selectfont \(\displaystyle 0.0\)}%
\end{pgfscope}%
\begin{pgfscope}%
\pgfsetrectcap%
\pgfsetroundjoin%
\pgfsetlinewidth{0.803000pt}%
\definecolor{currentstroke}{rgb}{0.000000,0.000000,0.000000}%
\pgfsetstrokecolor{currentstroke}%
\pgfsetdash{}{0pt}%
\pgfpathmoveto{\pgfqpoint{0.470537in}{1.155741in}}%
\pgfpathlineto{\pgfqpoint{0.443382in}{1.142560in}}%
\pgfusepath{stroke}%
\end{pgfscope}%
\begin{pgfscope}%
\definecolor{textcolor}{rgb}{0.000000,0.000000,0.000000}%
\pgfsetstrokecolor{textcolor}%
\pgfsetfillcolor{textcolor}%
\pgftext[x=0.313537in,y=1.151351in,,top]{\color{textcolor}\sffamily\fontsize{6.000000}{7.200000}\selectfont \(\displaystyle 0.5\)}%
\end{pgfscope}%
\begin{pgfscope}%
\pgfsetrectcap%
\pgfsetroundjoin%
\pgfsetlinewidth{0.803000pt}%
\definecolor{currentstroke}{rgb}{0.000000,0.000000,0.000000}%
\pgfsetstrokecolor{currentstroke}%
\pgfsetdash{}{0pt}%
\pgfpathmoveto{\pgfqpoint{0.457830in}{1.281444in}}%
\pgfpathlineto{\pgfqpoint{0.430335in}{1.268264in}}%
\pgfusepath{stroke}%
\end{pgfscope}%
\begin{pgfscope}%
\definecolor{textcolor}{rgb}{0.000000,0.000000,0.000000}%
\pgfsetstrokecolor{textcolor}%
\pgfsetfillcolor{textcolor}%
\pgftext[x=0.298953in,y=1.277054in,,top]{\color{textcolor}\sffamily\fontsize{6.000000}{7.200000}\selectfont \(\displaystyle 1.0\)}%
\end{pgfscope}%
\begin{pgfscope}%
\pgfsetrectcap%
\pgfsetroundjoin%
\pgfsetlinewidth{0.803000pt}%
\definecolor{currentstroke}{rgb}{0.000000,0.000000,0.000000}%
\pgfsetstrokecolor{currentstroke}%
\pgfsetdash{}{0pt}%
\pgfpathmoveto{\pgfqpoint{0.444817in}{1.410179in}}%
\pgfpathlineto{\pgfqpoint{0.416973in}{1.397005in}}%
\pgfusepath{stroke}%
\end{pgfscope}%
\begin{pgfscope}%
\definecolor{textcolor}{rgb}{0.000000,0.000000,0.000000}%
\pgfsetstrokecolor{textcolor}%
\pgfsetfillcolor{textcolor}%
\pgftext[x=0.284016in,y=1.405791in,,top]{\color{textcolor}\sffamily\fontsize{6.000000}{7.200000}\selectfont \(\displaystyle 1.5\)}%
\end{pgfscope}%
\begin{pgfscope}%
\pgfsetrectcap%
\pgfsetroundjoin%
\pgfsetlinewidth{0.803000pt}%
\definecolor{currentstroke}{rgb}{0.000000,0.000000,0.000000}%
\pgfsetstrokecolor{currentstroke}%
\pgfsetdash{}{0pt}%
\pgfpathmoveto{\pgfqpoint{0.431486in}{1.542058in}}%
\pgfpathlineto{\pgfqpoint{0.403284in}{1.528895in}}%
\pgfusepath{stroke}%
\end{pgfscope}%
\begin{pgfscope}%
\definecolor{textcolor}{rgb}{0.000000,0.000000,0.000000}%
\pgfsetstrokecolor{textcolor}%
\pgfsetfillcolor{textcolor}%
\pgftext[x=0.268714in,y=1.537674in,,top]{\color{textcolor}\sffamily\fontsize{6.000000}{7.200000}\selectfont \(\displaystyle 2.0\)}%
\end{pgfscope}%
\begin{pgfscope}%
\pgfpathrectangle{\pgfqpoint{0.150000in}{0.150000in}}{\pgfqpoint{2.700000in}{1.950000in}}%
\pgfusepath{clip}%
\pgfsetbuttcap%
\pgfsetroundjoin%
\definecolor{currentfill}{rgb}{0.781556,0.603631,0.617724}%
\pgfsetfillcolor{currentfill}%
\pgfsetlinewidth{0.000000pt}%
\definecolor{currentstroke}{rgb}{0.000000,0.000000,0.000000}%
\pgfsetstrokecolor{currentstroke}%
\pgfsetdash{}{0pt}%
\pgfpathmoveto{\pgfqpoint{2.408706in}{0.781515in}}%
\pgfpathlineto{\pgfqpoint{2.442626in}{0.791413in}}%
\pgfpathlineto{\pgfqpoint{2.406437in}{0.829099in}}%
\pgfpathlineto{\pgfqpoint{2.372418in}{0.819301in}}%
\pgfpathclose%
\pgfusepath{fill}%
\end{pgfscope}%
\begin{pgfscope}%
\pgfpathrectangle{\pgfqpoint{0.150000in}{0.150000in}}{\pgfqpoint{2.700000in}{1.950000in}}%
\pgfusepath{clip}%
\pgfsetbuttcap%
\pgfsetroundjoin%
\definecolor{currentfill}{rgb}{0.800551,0.638097,0.650965}%
\pgfsetfillcolor{currentfill}%
\pgfsetlinewidth{0.000000pt}%
\definecolor{currentstroke}{rgb}{0.000000,0.000000,0.000000}%
\pgfsetstrokecolor{currentstroke}%
\pgfsetdash{}{0pt}%
\pgfpathmoveto{\pgfqpoint{2.372418in}{0.819301in}}%
\pgfpathlineto{\pgfqpoint{2.406437in}{0.829099in}}%
\pgfpathlineto{\pgfqpoint{2.370243in}{0.866789in}}%
\pgfpathlineto{\pgfqpoint{2.336126in}{0.857092in}}%
\pgfpathclose%
\pgfusepath{fill}%
\end{pgfscope}%
\begin{pgfscope}%
\pgfpathrectangle{\pgfqpoint{0.150000in}{0.150000in}}{\pgfqpoint{2.700000in}{1.950000in}}%
\pgfusepath{clip}%
\pgfsetbuttcap%
\pgfsetroundjoin%
\definecolor{currentfill}{rgb}{0.819547,0.672564,0.684206}%
\pgfsetfillcolor{currentfill}%
\pgfsetlinewidth{0.000000pt}%
\definecolor{currentstroke}{rgb}{0.000000,0.000000,0.000000}%
\pgfsetstrokecolor{currentstroke}%
\pgfsetdash{}{0pt}%
\pgfpathmoveto{\pgfqpoint{2.336126in}{0.857092in}}%
\pgfpathlineto{\pgfqpoint{2.370243in}{0.866789in}}%
\pgfpathlineto{\pgfqpoint{2.334045in}{0.904485in}}%
\pgfpathlineto{\pgfqpoint{2.299828in}{0.894888in}}%
\pgfpathclose%
\pgfusepath{fill}%
\end{pgfscope}%
\begin{pgfscope}%
\pgfpathrectangle{\pgfqpoint{0.150000in}{0.150000in}}{\pgfqpoint{2.700000in}{1.950000in}}%
\pgfusepath{clip}%
\pgfsetbuttcap%
\pgfsetroundjoin%
\definecolor{currentfill}{rgb}{0.838542,0.707031,0.717448}%
\pgfsetfillcolor{currentfill}%
\pgfsetlinewidth{0.000000pt}%
\definecolor{currentstroke}{rgb}{0.000000,0.000000,0.000000}%
\pgfsetstrokecolor{currentstroke}%
\pgfsetdash{}{0pt}%
\pgfpathmoveto{\pgfqpoint{2.299828in}{0.894888in}}%
\pgfpathlineto{\pgfqpoint{2.334045in}{0.904485in}}%
\pgfpathlineto{\pgfqpoint{2.297842in}{0.942186in}}%
\pgfpathlineto{\pgfqpoint{2.263526in}{0.932688in}}%
\pgfpathclose%
\pgfusepath{fill}%
\end{pgfscope}%
\begin{pgfscope}%
\pgfpathrectangle{\pgfqpoint{0.150000in}{0.150000in}}{\pgfqpoint{2.700000in}{1.950000in}}%
\pgfusepath{clip}%
\pgfsetbuttcap%
\pgfsetroundjoin%
\definecolor{currentfill}{rgb}{0.857537,0.741498,0.750689}%
\pgfsetfillcolor{currentfill}%
\pgfsetlinewidth{0.000000pt}%
\definecolor{currentstroke}{rgb}{0.000000,0.000000,0.000000}%
\pgfsetstrokecolor{currentstroke}%
\pgfsetdash{}{0pt}%
\pgfpathmoveto{\pgfqpoint{2.263526in}{0.932688in}}%
\pgfpathlineto{\pgfqpoint{2.297842in}{0.942186in}}%
\pgfpathlineto{\pgfqpoint{2.261634in}{0.979891in}}%
\pgfpathlineto{\pgfqpoint{2.227219in}{0.970494in}}%
\pgfpathclose%
\pgfusepath{fill}%
\end{pgfscope}%
\begin{pgfscope}%
\pgfpathrectangle{\pgfqpoint{0.150000in}{0.150000in}}{\pgfqpoint{2.700000in}{1.950000in}}%
\pgfusepath{clip}%
\pgfsetbuttcap%
\pgfsetroundjoin%
\definecolor{currentfill}{rgb}{0.876532,0.775965,0.783931}%
\pgfsetfillcolor{currentfill}%
\pgfsetlinewidth{0.000000pt}%
\definecolor{currentstroke}{rgb}{0.000000,0.000000,0.000000}%
\pgfsetstrokecolor{currentstroke}%
\pgfsetdash{}{0pt}%
\pgfpathmoveto{\pgfqpoint{2.227219in}{0.970494in}}%
\pgfpathlineto{\pgfqpoint{2.261634in}{0.979891in}}%
\pgfpathlineto{\pgfqpoint{2.225422in}{1.017602in}}%
\pgfpathlineto{\pgfqpoint{2.190908in}{1.008305in}}%
\pgfpathclose%
\pgfusepath{fill}%
\end{pgfscope}%
\begin{pgfscope}%
\pgfpathrectangle{\pgfqpoint{0.150000in}{0.150000in}}{\pgfqpoint{2.700000in}{1.950000in}}%
\pgfusepath{clip}%
\pgfsetbuttcap%
\pgfsetroundjoin%
\definecolor{currentfill}{rgb}{0.891728,0.803539,0.810524}%
\pgfsetfillcolor{currentfill}%
\pgfsetlinewidth{0.000000pt}%
\definecolor{currentstroke}{rgb}{0.000000,0.000000,0.000000}%
\pgfsetstrokecolor{currentstroke}%
\pgfsetdash{}{0pt}%
\pgfpathmoveto{\pgfqpoint{2.190908in}{1.008305in}}%
\pgfpathlineto{\pgfqpoint{2.225422in}{1.017602in}}%
\pgfpathlineto{\pgfqpoint{2.189204in}{1.055317in}}%
\pgfpathlineto{\pgfqpoint{2.154591in}{1.046121in}}%
\pgfpathclose%
\pgfusepath{fill}%
\end{pgfscope}%
\begin{pgfscope}%
\pgfpathrectangle{\pgfqpoint{0.150000in}{0.150000in}}{\pgfqpoint{2.700000in}{1.950000in}}%
\pgfusepath{clip}%
\pgfsetbuttcap%
\pgfsetroundjoin%
\definecolor{currentfill}{rgb}{0.910723,0.838006,0.843765}%
\pgfsetfillcolor{currentfill}%
\pgfsetlinewidth{0.000000pt}%
\definecolor{currentstroke}{rgb}{0.000000,0.000000,0.000000}%
\pgfsetstrokecolor{currentstroke}%
\pgfsetdash{}{0pt}%
\pgfpathmoveto{\pgfqpoint{2.154591in}{1.046121in}}%
\pgfpathlineto{\pgfqpoint{2.189204in}{1.055317in}}%
\pgfpathlineto{\pgfqpoint{2.152982in}{1.093038in}}%
\pgfpathlineto{\pgfqpoint{2.118270in}{1.083941in}}%
\pgfpathclose%
\pgfusepath{fill}%
\end{pgfscope}%
\begin{pgfscope}%
\pgfpathrectangle{\pgfqpoint{0.150000in}{0.150000in}}{\pgfqpoint{2.700000in}{1.950000in}}%
\pgfusepath{clip}%
\pgfsetbuttcap%
\pgfsetroundjoin%
\definecolor{currentfill}{rgb}{0.929718,0.872472,0.877007}%
\pgfsetfillcolor{currentfill}%
\pgfsetlinewidth{0.000000pt}%
\definecolor{currentstroke}{rgb}{0.000000,0.000000,0.000000}%
\pgfsetstrokecolor{currentstroke}%
\pgfsetdash{}{0pt}%
\pgfpathmoveto{\pgfqpoint{2.118270in}{1.083941in}}%
\pgfpathlineto{\pgfqpoint{2.152982in}{1.093038in}}%
\pgfpathlineto{\pgfqpoint{2.116756in}{1.130763in}}%
\pgfpathlineto{\pgfqpoint{2.081944in}{1.121767in}}%
\pgfpathclose%
\pgfusepath{fill}%
\end{pgfscope}%
\begin{pgfscope}%
\pgfpathrectangle{\pgfqpoint{0.150000in}{0.150000in}}{\pgfqpoint{2.700000in}{1.950000in}}%
\pgfusepath{clip}%
\pgfsetbuttcap%
\pgfsetroundjoin%
\definecolor{currentfill}{rgb}{0.948713,0.906939,0.910248}%
\pgfsetfillcolor{currentfill}%
\pgfsetlinewidth{0.000000pt}%
\definecolor{currentstroke}{rgb}{0.000000,0.000000,0.000000}%
\pgfsetstrokecolor{currentstroke}%
\pgfsetdash{}{0pt}%
\pgfpathmoveto{\pgfqpoint{2.081944in}{1.121767in}}%
\pgfpathlineto{\pgfqpoint{2.116756in}{1.130763in}}%
\pgfpathlineto{\pgfqpoint{2.080524in}{1.168493in}}%
\pgfpathlineto{\pgfqpoint{2.045614in}{1.159597in}}%
\pgfpathclose%
\pgfusepath{fill}%
\end{pgfscope}%
\begin{pgfscope}%
\pgfpathrectangle{\pgfqpoint{0.150000in}{0.150000in}}{\pgfqpoint{2.700000in}{1.950000in}}%
\pgfusepath{clip}%
\pgfsetbuttcap%
\pgfsetroundjoin%
\definecolor{currentfill}{rgb}{0.967708,0.941406,0.943490}%
\pgfsetfillcolor{currentfill}%
\pgfsetlinewidth{0.000000pt}%
\definecolor{currentstroke}{rgb}{0.000000,0.000000,0.000000}%
\pgfsetstrokecolor{currentstroke}%
\pgfsetdash{}{0pt}%
\pgfpathmoveto{\pgfqpoint{2.045614in}{1.159597in}}%
\pgfpathlineto{\pgfqpoint{2.080524in}{1.168493in}}%
\pgfpathlineto{\pgfqpoint{2.044288in}{1.206228in}}%
\pgfpathlineto{\pgfqpoint{2.009278in}{1.197433in}}%
\pgfpathclose%
\pgfusepath{fill}%
\end{pgfscope}%
\begin{pgfscope}%
\pgfpathrectangle{\pgfqpoint{0.150000in}{0.150000in}}{\pgfqpoint{2.700000in}{1.950000in}}%
\pgfusepath{clip}%
\pgfsetbuttcap%
\pgfsetroundjoin%
\definecolor{currentfill}{rgb}{0.986703,0.975873,0.976731}%
\pgfsetfillcolor{currentfill}%
\pgfsetlinewidth{0.000000pt}%
\definecolor{currentstroke}{rgb}{0.000000,0.000000,0.000000}%
\pgfsetstrokecolor{currentstroke}%
\pgfsetdash{}{0pt}%
\pgfpathmoveto{\pgfqpoint{2.009278in}{1.197433in}}%
\pgfpathlineto{\pgfqpoint{2.044288in}{1.206228in}}%
\pgfpathlineto{\pgfqpoint{2.008047in}{1.243968in}}%
\pgfpathlineto{\pgfqpoint{1.972938in}{1.235273in}}%
\pgfpathclose%
\pgfusepath{fill}%
\end{pgfscope}%
\begin{pgfscope}%
\pgfpathrectangle{\pgfqpoint{0.150000in}{0.150000in}}{\pgfqpoint{2.700000in}{1.950000in}}%
\pgfusepath{clip}%
\pgfsetbuttcap%
\pgfsetroundjoin%
\definecolor{currentfill}{rgb}{0.990671,0.991820,0.993428}%
\pgfsetfillcolor{currentfill}%
\pgfsetlinewidth{0.000000pt}%
\definecolor{currentstroke}{rgb}{0.000000,0.000000,0.000000}%
\pgfsetstrokecolor{currentstroke}%
\pgfsetdash{}{0pt}%
\pgfpathmoveto{\pgfqpoint{1.972938in}{1.235273in}}%
\pgfpathlineto{\pgfqpoint{2.008047in}{1.243968in}}%
\pgfpathlineto{\pgfqpoint{1.971802in}{1.281713in}}%
\pgfpathlineto{\pgfqpoint{1.936593in}{1.273119in}}%
\pgfpathclose%
\pgfusepath{fill}%
\end{pgfscope}%
\begin{pgfscope}%
\pgfpathrectangle{\pgfqpoint{0.150000in}{0.150000in}}{\pgfqpoint{2.700000in}{1.950000in}}%
\pgfusepath{clip}%
\pgfsetbuttcap%
\pgfsetroundjoin%
\definecolor{currentfill}{rgb}{0.959574,0.964553,0.971523}%
\pgfsetfillcolor{currentfill}%
\pgfsetlinewidth{0.000000pt}%
\definecolor{currentstroke}{rgb}{0.000000,0.000000,0.000000}%
\pgfsetstrokecolor{currentstroke}%
\pgfsetdash{}{0pt}%
\pgfpathmoveto{\pgfqpoint{1.936593in}{1.273119in}}%
\pgfpathlineto{\pgfqpoint{1.971802in}{1.281713in}}%
\pgfpathlineto{\pgfqpoint{1.935552in}{1.319463in}}%
\pgfpathlineto{\pgfqpoint{1.900244in}{1.310969in}}%
\pgfpathclose%
\pgfusepath{fill}%
\end{pgfscope}%
\begin{pgfscope}%
\pgfpathrectangle{\pgfqpoint{0.150000in}{0.150000in}}{\pgfqpoint{2.700000in}{1.950000in}}%
\pgfusepath{clip}%
\pgfsetbuttcap%
\pgfsetroundjoin%
\definecolor{currentfill}{rgb}{0.934697,0.942739,0.953998}%
\pgfsetfillcolor{currentfill}%
\pgfsetlinewidth{0.000000pt}%
\definecolor{currentstroke}{rgb}{0.000000,0.000000,0.000000}%
\pgfsetstrokecolor{currentstroke}%
\pgfsetdash{}{0pt}%
\pgfpathmoveto{\pgfqpoint{1.900244in}{1.310969in}}%
\pgfpathlineto{\pgfqpoint{1.935552in}{1.319463in}}%
\pgfpathlineto{\pgfqpoint{1.899297in}{1.357218in}}%
\pgfpathlineto{\pgfqpoint{1.863890in}{1.348824in}}%
\pgfpathclose%
\pgfusepath{fill}%
\end{pgfscope}%
\begin{pgfscope}%
\pgfpathrectangle{\pgfqpoint{0.150000in}{0.150000in}}{\pgfqpoint{2.700000in}{1.950000in}}%
\pgfusepath{clip}%
\pgfsetbuttcap%
\pgfsetroundjoin%
\definecolor{currentfill}{rgb}{0.903600,0.915472,0.932093}%
\pgfsetfillcolor{currentfill}%
\pgfsetlinewidth{0.000000pt}%
\definecolor{currentstroke}{rgb}{0.000000,0.000000,0.000000}%
\pgfsetstrokecolor{currentstroke}%
\pgfsetdash{}{0pt}%
\pgfpathmoveto{\pgfqpoint{1.863890in}{1.348824in}}%
\pgfpathlineto{\pgfqpoint{1.899297in}{1.357218in}}%
\pgfpathlineto{\pgfqpoint{1.863037in}{1.394977in}}%
\pgfpathlineto{\pgfqpoint{1.827530in}{1.386684in}}%
\pgfpathclose%
\pgfusepath{fill}%
\end{pgfscope}%
\begin{pgfscope}%
\pgfpathrectangle{\pgfqpoint{0.150000in}{0.150000in}}{\pgfqpoint{2.700000in}{1.950000in}}%
\pgfusepath{clip}%
\pgfsetbuttcap%
\pgfsetroundjoin%
\definecolor{currentfill}{rgb}{0.872503,0.888205,0.910187}%
\pgfsetfillcolor{currentfill}%
\pgfsetlinewidth{0.000000pt}%
\definecolor{currentstroke}{rgb}{0.000000,0.000000,0.000000}%
\pgfsetstrokecolor{currentstroke}%
\pgfsetdash{}{0pt}%
\pgfpathmoveto{\pgfqpoint{1.827530in}{1.386684in}}%
\pgfpathlineto{\pgfqpoint{1.863037in}{1.394977in}}%
\pgfpathlineto{\pgfqpoint{1.826772in}{1.432742in}}%
\pgfpathlineto{\pgfqpoint{1.791167in}{1.424549in}}%
\pgfpathclose%
\pgfusepath{fill}%
\end{pgfscope}%
\begin{pgfscope}%
\pgfpathrectangle{\pgfqpoint{0.150000in}{0.150000in}}{\pgfqpoint{2.700000in}{1.950000in}}%
\pgfusepath{clip}%
\pgfsetbuttcap%
\pgfsetroundjoin%
\definecolor{currentfill}{rgb}{0.841406,0.860938,0.888281}%
\pgfsetfillcolor{currentfill}%
\pgfsetlinewidth{0.000000pt}%
\definecolor{currentstroke}{rgb}{0.000000,0.000000,0.000000}%
\pgfsetstrokecolor{currentstroke}%
\pgfsetdash{}{0pt}%
\pgfpathmoveto{\pgfqpoint{1.791167in}{1.424549in}}%
\pgfpathlineto{\pgfqpoint{1.826772in}{1.432742in}}%
\pgfpathlineto{\pgfqpoint{1.790503in}{1.470511in}}%
\pgfpathlineto{\pgfqpoint{1.754798in}{1.462420in}}%
\pgfpathclose%
\pgfusepath{fill}%
\end{pgfscope}%
\begin{pgfscope}%
\pgfpathrectangle{\pgfqpoint{0.150000in}{0.150000in}}{\pgfqpoint{2.700000in}{1.950000in}}%
\pgfusepath{clip}%
\pgfsetbuttcap%
\pgfsetroundjoin%
\definecolor{currentfill}{rgb}{0.810309,0.833670,0.866376}%
\pgfsetfillcolor{currentfill}%
\pgfsetlinewidth{0.000000pt}%
\definecolor{currentstroke}{rgb}{0.000000,0.000000,0.000000}%
\pgfsetstrokecolor{currentstroke}%
\pgfsetdash{}{0pt}%
\pgfpathmoveto{\pgfqpoint{1.754798in}{1.462420in}}%
\pgfpathlineto{\pgfqpoint{1.790503in}{1.470511in}}%
\pgfpathlineto{\pgfqpoint{1.754229in}{1.508286in}}%
\pgfpathlineto{\pgfqpoint{1.718425in}{1.500295in}}%
\pgfpathclose%
\pgfusepath{fill}%
\end{pgfscope}%
\begin{pgfscope}%
\pgfpathrectangle{\pgfqpoint{0.150000in}{0.150000in}}{\pgfqpoint{2.700000in}{1.950000in}}%
\pgfusepath{clip}%
\pgfsetbuttcap%
\pgfsetroundjoin%
\definecolor{currentfill}{rgb}{0.779213,0.806403,0.844470}%
\pgfsetfillcolor{currentfill}%
\pgfsetlinewidth{0.000000pt}%
\definecolor{currentstroke}{rgb}{0.000000,0.000000,0.000000}%
\pgfsetstrokecolor{currentstroke}%
\pgfsetdash{}{0pt}%
\pgfpathmoveto{\pgfqpoint{1.718425in}{1.500295in}}%
\pgfpathlineto{\pgfqpoint{1.754229in}{1.508286in}}%
\pgfpathlineto{\pgfqpoint{1.717951in}{1.546065in}}%
\pgfpathlineto{\pgfqpoint{1.682047in}{1.538175in}}%
\pgfpathclose%
\pgfusepath{fill}%
\end{pgfscope}%
\begin{pgfscope}%
\pgfpathrectangle{\pgfqpoint{0.150000in}{0.150000in}}{\pgfqpoint{2.700000in}{1.950000in}}%
\pgfusepath{clip}%
\pgfsetbuttcap%
\pgfsetroundjoin%
\definecolor{currentfill}{rgb}{0.748116,0.779136,0.822564}%
\pgfsetfillcolor{currentfill}%
\pgfsetlinewidth{0.000000pt}%
\definecolor{currentstroke}{rgb}{0.000000,0.000000,0.000000}%
\pgfsetstrokecolor{currentstroke}%
\pgfsetdash{}{0pt}%
\pgfpathmoveto{\pgfqpoint{1.682047in}{1.538175in}}%
\pgfpathlineto{\pgfqpoint{1.717951in}{1.546065in}}%
\pgfpathlineto{\pgfqpoint{1.681667in}{1.583849in}}%
\pgfpathlineto{\pgfqpoint{1.645664in}{1.576060in}}%
\pgfpathclose%
\pgfusepath{fill}%
\end{pgfscope}%
\begin{pgfscope}%
\pgfpathrectangle{\pgfqpoint{0.150000in}{0.150000in}}{\pgfqpoint{2.700000in}{1.950000in}}%
\pgfusepath{clip}%
\pgfsetbuttcap%
\pgfsetroundjoin%
\definecolor{currentfill}{rgb}{0.717019,0.751869,0.800659}%
\pgfsetfillcolor{currentfill}%
\pgfsetlinewidth{0.000000pt}%
\definecolor{currentstroke}{rgb}{0.000000,0.000000,0.000000}%
\pgfsetstrokecolor{currentstroke}%
\pgfsetdash{}{0pt}%
\pgfpathmoveto{\pgfqpoint{1.645664in}{1.576060in}}%
\pgfpathlineto{\pgfqpoint{1.681667in}{1.583849in}}%
\pgfpathlineto{\pgfqpoint{1.645379in}{1.621639in}}%
\pgfpathlineto{\pgfqpoint{1.609276in}{1.613949in}}%
\pgfpathclose%
\pgfusepath{fill}%
\end{pgfscope}%
\begin{pgfscope}%
\pgfpathrectangle{\pgfqpoint{0.150000in}{0.150000in}}{\pgfqpoint{2.700000in}{1.950000in}}%
\pgfusepath{clip}%
\pgfsetbuttcap%
\pgfsetroundjoin%
\definecolor{currentfill}{rgb}{0.692142,0.730055,0.783134}%
\pgfsetfillcolor{currentfill}%
\pgfsetlinewidth{0.000000pt}%
\definecolor{currentstroke}{rgb}{0.000000,0.000000,0.000000}%
\pgfsetstrokecolor{currentstroke}%
\pgfsetdash{}{0pt}%
\pgfpathmoveto{\pgfqpoint{1.609276in}{1.613949in}}%
\pgfpathlineto{\pgfqpoint{1.645379in}{1.621639in}}%
\pgfpathlineto{\pgfqpoint{1.609086in}{1.659433in}}%
\pgfpathlineto{\pgfqpoint{1.572884in}{1.651844in}}%
\pgfpathclose%
\pgfusepath{fill}%
\end{pgfscope}%
\begin{pgfscope}%
\pgfpathrectangle{\pgfqpoint{0.150000in}{0.150000in}}{\pgfqpoint{2.700000in}{1.950000in}}%
\pgfusepath{clip}%
\pgfsetbuttcap%
\pgfsetroundjoin%
\definecolor{currentfill}{rgb}{0.661045,0.702788,0.761229}%
\pgfsetfillcolor{currentfill}%
\pgfsetlinewidth{0.000000pt}%
\definecolor{currentstroke}{rgb}{0.000000,0.000000,0.000000}%
\pgfsetstrokecolor{currentstroke}%
\pgfsetdash{}{0pt}%
\pgfpathmoveto{\pgfqpoint{1.572884in}{1.651844in}}%
\pgfpathlineto{\pgfqpoint{1.609086in}{1.659433in}}%
\pgfpathlineto{\pgfqpoint{1.572789in}{1.697232in}}%
\pgfpathlineto{\pgfqpoint{1.536486in}{1.689744in}}%
\pgfpathclose%
\pgfusepath{fill}%
\end{pgfscope}%
\begin{pgfscope}%
\pgfpathrectangle{\pgfqpoint{0.150000in}{0.150000in}}{\pgfqpoint{2.700000in}{1.950000in}}%
\pgfusepath{clip}%
\pgfsetbuttcap%
\pgfsetroundjoin%
\definecolor{currentfill}{rgb}{0.629948,0.675521,0.739323}%
\pgfsetfillcolor{currentfill}%
\pgfsetlinewidth{0.000000pt}%
\definecolor{currentstroke}{rgb}{0.000000,0.000000,0.000000}%
\pgfsetstrokecolor{currentstroke}%
\pgfsetdash{}{0pt}%
\pgfpathmoveto{\pgfqpoint{1.536486in}{1.689744in}}%
\pgfpathlineto{\pgfqpoint{1.572789in}{1.697232in}}%
\pgfpathlineto{\pgfqpoint{1.536486in}{1.735036in}}%
\pgfpathlineto{\pgfqpoint{1.500085in}{1.727649in}}%
\pgfpathclose%
\pgfusepath{fill}%
\end{pgfscope}%
\begin{pgfscope}%
\pgfpathrectangle{\pgfqpoint{0.150000in}{0.150000in}}{\pgfqpoint{2.700000in}{1.950000in}}%
\pgfusepath{clip}%
\pgfsetbuttcap%
\pgfsetroundjoin%
\definecolor{currentfill}{rgb}{0.792953,0.624311,0.637669}%
\pgfsetfillcolor{currentfill}%
\pgfsetlinewidth{0.000000pt}%
\definecolor{currentstroke}{rgb}{0.000000,0.000000,0.000000}%
\pgfsetstrokecolor{currentstroke}%
\pgfsetdash{}{0pt}%
\pgfpathmoveto{\pgfqpoint{2.374605in}{0.771564in}}%
\pgfpathlineto{\pgfqpoint{2.408706in}{0.781515in}}%
\pgfpathlineto{\pgfqpoint{2.372418in}{0.819301in}}%
\pgfpathlineto{\pgfqpoint{2.338217in}{0.809450in}}%
\pgfpathclose%
\pgfusepath{fill}%
\end{pgfscope}%
\begin{pgfscope}%
\pgfpathrectangle{\pgfqpoint{0.150000in}{0.150000in}}{\pgfqpoint{2.700000in}{1.950000in}}%
\pgfusepath{clip}%
\pgfsetbuttcap%
\pgfsetroundjoin%
\definecolor{currentfill}{rgb}{0.811949,0.658778,0.670910}%
\pgfsetfillcolor{currentfill}%
\pgfsetlinewidth{0.000000pt}%
\definecolor{currentstroke}{rgb}{0.000000,0.000000,0.000000}%
\pgfsetstrokecolor{currentstroke}%
\pgfsetdash{}{0pt}%
\pgfpathmoveto{\pgfqpoint{2.338217in}{0.809450in}}%
\pgfpathlineto{\pgfqpoint{2.372418in}{0.819301in}}%
\pgfpathlineto{\pgfqpoint{2.336126in}{0.857092in}}%
\pgfpathlineto{\pgfqpoint{2.301825in}{0.847342in}}%
\pgfpathclose%
\pgfusepath{fill}%
\end{pgfscope}%
\begin{pgfscope}%
\pgfpathrectangle{\pgfqpoint{0.150000in}{0.150000in}}{\pgfqpoint{2.700000in}{1.950000in}}%
\pgfusepath{clip}%
\pgfsetbuttcap%
\pgfsetroundjoin%
\definecolor{currentfill}{rgb}{0.827145,0.686351,0.697503}%
\pgfsetfillcolor{currentfill}%
\pgfsetlinewidth{0.000000pt}%
\definecolor{currentstroke}{rgb}{0.000000,0.000000,0.000000}%
\pgfsetstrokecolor{currentstroke}%
\pgfsetdash{}{0pt}%
\pgfpathmoveto{\pgfqpoint{2.301825in}{0.847342in}}%
\pgfpathlineto{\pgfqpoint{2.336126in}{0.857092in}}%
\pgfpathlineto{\pgfqpoint{2.299828in}{0.894888in}}%
\pgfpathlineto{\pgfqpoint{2.265428in}{0.885239in}}%
\pgfpathclose%
\pgfusepath{fill}%
\end{pgfscope}%
\begin{pgfscope}%
\pgfpathrectangle{\pgfqpoint{0.150000in}{0.150000in}}{\pgfqpoint{2.700000in}{1.950000in}}%
\pgfusepath{clip}%
\pgfsetbuttcap%
\pgfsetroundjoin%
\definecolor{currentfill}{rgb}{0.846140,0.720818,0.730744}%
\pgfsetfillcolor{currentfill}%
\pgfsetlinewidth{0.000000pt}%
\definecolor{currentstroke}{rgb}{0.000000,0.000000,0.000000}%
\pgfsetstrokecolor{currentstroke}%
\pgfsetdash{}{0pt}%
\pgfpathmoveto{\pgfqpoint{2.265428in}{0.885239in}}%
\pgfpathlineto{\pgfqpoint{2.299828in}{0.894888in}}%
\pgfpathlineto{\pgfqpoint{2.263526in}{0.932688in}}%
\pgfpathlineto{\pgfqpoint{2.229026in}{0.923140in}}%
\pgfpathclose%
\pgfusepath{fill}%
\end{pgfscope}%
\begin{pgfscope}%
\pgfpathrectangle{\pgfqpoint{0.150000in}{0.150000in}}{\pgfqpoint{2.700000in}{1.950000in}}%
\pgfusepath{clip}%
\pgfsetbuttcap%
\pgfsetroundjoin%
\definecolor{currentfill}{rgb}{0.865135,0.755285,0.763986}%
\pgfsetfillcolor{currentfill}%
\pgfsetlinewidth{0.000000pt}%
\definecolor{currentstroke}{rgb}{0.000000,0.000000,0.000000}%
\pgfsetstrokecolor{currentstroke}%
\pgfsetdash{}{0pt}%
\pgfpathmoveto{\pgfqpoint{2.229026in}{0.923140in}}%
\pgfpathlineto{\pgfqpoint{2.263526in}{0.932688in}}%
\pgfpathlineto{\pgfqpoint{2.227219in}{0.970494in}}%
\pgfpathlineto{\pgfqpoint{2.192620in}{0.961047in}}%
\pgfpathclose%
\pgfusepath{fill}%
\end{pgfscope}%
\begin{pgfscope}%
\pgfpathrectangle{\pgfqpoint{0.150000in}{0.150000in}}{\pgfqpoint{2.700000in}{1.950000in}}%
\pgfusepath{clip}%
\pgfsetbuttcap%
\pgfsetroundjoin%
\definecolor{currentfill}{rgb}{0.884130,0.789752,0.797227}%
\pgfsetfillcolor{currentfill}%
\pgfsetlinewidth{0.000000pt}%
\definecolor{currentstroke}{rgb}{0.000000,0.000000,0.000000}%
\pgfsetstrokecolor{currentstroke}%
\pgfsetdash{}{0pt}%
\pgfpathmoveto{\pgfqpoint{2.192620in}{0.961047in}}%
\pgfpathlineto{\pgfqpoint{2.227219in}{0.970494in}}%
\pgfpathlineto{\pgfqpoint{2.190908in}{1.008305in}}%
\pgfpathlineto{\pgfqpoint{2.156209in}{0.998958in}}%
\pgfpathclose%
\pgfusepath{fill}%
\end{pgfscope}%
\begin{pgfscope}%
\pgfpathrectangle{\pgfqpoint{0.150000in}{0.150000in}}{\pgfqpoint{2.700000in}{1.950000in}}%
\pgfusepath{clip}%
\pgfsetbuttcap%
\pgfsetroundjoin%
\definecolor{currentfill}{rgb}{0.903125,0.824219,0.830469}%
\pgfsetfillcolor{currentfill}%
\pgfsetlinewidth{0.000000pt}%
\definecolor{currentstroke}{rgb}{0.000000,0.000000,0.000000}%
\pgfsetstrokecolor{currentstroke}%
\pgfsetdash{}{0pt}%
\pgfpathmoveto{\pgfqpoint{2.156209in}{0.998958in}}%
\pgfpathlineto{\pgfqpoint{2.190908in}{1.008305in}}%
\pgfpathlineto{\pgfqpoint{2.154591in}{1.046121in}}%
\pgfpathlineto{\pgfqpoint{2.119793in}{1.036875in}}%
\pgfpathclose%
\pgfusepath{fill}%
\end{pgfscope}%
\begin{pgfscope}%
\pgfpathrectangle{\pgfqpoint{0.150000in}{0.150000in}}{\pgfqpoint{2.700000in}{1.950000in}}%
\pgfusepath{clip}%
\pgfsetbuttcap%
\pgfsetroundjoin%
\definecolor{currentfill}{rgb}{0.922120,0.858686,0.863710}%
\pgfsetfillcolor{currentfill}%
\pgfsetlinewidth{0.000000pt}%
\definecolor{currentstroke}{rgb}{0.000000,0.000000,0.000000}%
\pgfsetstrokecolor{currentstroke}%
\pgfsetdash{}{0pt}%
\pgfpathmoveto{\pgfqpoint{2.119793in}{1.036875in}}%
\pgfpathlineto{\pgfqpoint{2.154591in}{1.046121in}}%
\pgfpathlineto{\pgfqpoint{2.118270in}{1.083941in}}%
\pgfpathlineto{\pgfqpoint{2.083372in}{1.074796in}}%
\pgfpathclose%
\pgfusepath{fill}%
\end{pgfscope}%
\begin{pgfscope}%
\pgfpathrectangle{\pgfqpoint{0.150000in}{0.150000in}}{\pgfqpoint{2.700000in}{1.950000in}}%
\pgfusepath{clip}%
\pgfsetbuttcap%
\pgfsetroundjoin%
\definecolor{currentfill}{rgb}{0.941115,0.893153,0.896952}%
\pgfsetfillcolor{currentfill}%
\pgfsetlinewidth{0.000000pt}%
\definecolor{currentstroke}{rgb}{0.000000,0.000000,0.000000}%
\pgfsetstrokecolor{currentstroke}%
\pgfsetdash{}{0pt}%
\pgfpathmoveto{\pgfqpoint{2.083372in}{1.074796in}}%
\pgfpathlineto{\pgfqpoint{2.118270in}{1.083941in}}%
\pgfpathlineto{\pgfqpoint{2.081944in}{1.121767in}}%
\pgfpathlineto{\pgfqpoint{2.046946in}{1.112723in}}%
\pgfpathclose%
\pgfusepath{fill}%
\end{pgfscope}%
\begin{pgfscope}%
\pgfpathrectangle{\pgfqpoint{0.150000in}{0.150000in}}{\pgfqpoint{2.700000in}{1.950000in}}%
\pgfusepath{clip}%
\pgfsetbuttcap%
\pgfsetroundjoin%
\definecolor{currentfill}{rgb}{0.960110,0.927619,0.930193}%
\pgfsetfillcolor{currentfill}%
\pgfsetlinewidth{0.000000pt}%
\definecolor{currentstroke}{rgb}{0.000000,0.000000,0.000000}%
\pgfsetstrokecolor{currentstroke}%
\pgfsetdash{}{0pt}%
\pgfpathmoveto{\pgfqpoint{2.046946in}{1.112723in}}%
\pgfpathlineto{\pgfqpoint{2.081944in}{1.121767in}}%
\pgfpathlineto{\pgfqpoint{2.045614in}{1.159597in}}%
\pgfpathlineto{\pgfqpoint{2.010516in}{1.150654in}}%
\pgfpathclose%
\pgfusepath{fill}%
\end{pgfscope}%
\begin{pgfscope}%
\pgfpathrectangle{\pgfqpoint{0.150000in}{0.150000in}}{\pgfqpoint{2.700000in}{1.950000in}}%
\pgfusepath{clip}%
\pgfsetbuttcap%
\pgfsetroundjoin%
\definecolor{currentfill}{rgb}{0.975306,0.955193,0.956786}%
\pgfsetfillcolor{currentfill}%
\pgfsetlinewidth{0.000000pt}%
\definecolor{currentstroke}{rgb}{0.000000,0.000000,0.000000}%
\pgfsetstrokecolor{currentstroke}%
\pgfsetdash{}{0pt}%
\pgfpathmoveto{\pgfqpoint{2.010516in}{1.150654in}}%
\pgfpathlineto{\pgfqpoint{2.045614in}{1.159597in}}%
\pgfpathlineto{\pgfqpoint{2.009278in}{1.197433in}}%
\pgfpathlineto{\pgfqpoint{1.974081in}{1.188590in}}%
\pgfpathclose%
\pgfusepath{fill}%
\end{pgfscope}%
\begin{pgfscope}%
\pgfpathrectangle{\pgfqpoint{0.150000in}{0.150000in}}{\pgfqpoint{2.700000in}{1.950000in}}%
\pgfusepath{clip}%
\pgfsetbuttcap%
\pgfsetroundjoin%
\definecolor{currentfill}{rgb}{0.994301,0.989660,0.990028}%
\pgfsetfillcolor{currentfill}%
\pgfsetlinewidth{0.000000pt}%
\definecolor{currentstroke}{rgb}{0.000000,0.000000,0.000000}%
\pgfsetstrokecolor{currentstroke}%
\pgfsetdash{}{0pt}%
\pgfpathmoveto{\pgfqpoint{1.974081in}{1.188590in}}%
\pgfpathlineto{\pgfqpoint{2.009278in}{1.197433in}}%
\pgfpathlineto{\pgfqpoint{1.972938in}{1.235273in}}%
\pgfpathlineto{\pgfqpoint{1.937641in}{1.226532in}}%
\pgfpathclose%
\pgfusepath{fill}%
\end{pgfscope}%
\begin{pgfscope}%
\pgfpathrectangle{\pgfqpoint{0.150000in}{0.150000in}}{\pgfqpoint{2.700000in}{1.950000in}}%
\pgfusepath{clip}%
\pgfsetbuttcap%
\pgfsetroundjoin%
\definecolor{currentfill}{rgb}{0.978232,0.980913,0.984666}%
\pgfsetfillcolor{currentfill}%
\pgfsetlinewidth{0.000000pt}%
\definecolor{currentstroke}{rgb}{0.000000,0.000000,0.000000}%
\pgfsetstrokecolor{currentstroke}%
\pgfsetdash{}{0pt}%
\pgfpathmoveto{\pgfqpoint{1.937641in}{1.226532in}}%
\pgfpathlineto{\pgfqpoint{1.972938in}{1.235273in}}%
\pgfpathlineto{\pgfqpoint{1.936593in}{1.273119in}}%
\pgfpathlineto{\pgfqpoint{1.901196in}{1.264478in}}%
\pgfpathclose%
\pgfusepath{fill}%
\end{pgfscope}%
\begin{pgfscope}%
\pgfpathrectangle{\pgfqpoint{0.150000in}{0.150000in}}{\pgfqpoint{2.700000in}{1.950000in}}%
\pgfusepath{clip}%
\pgfsetbuttcap%
\pgfsetroundjoin%
\definecolor{currentfill}{rgb}{0.947135,0.953646,0.962760}%
\pgfsetfillcolor{currentfill}%
\pgfsetlinewidth{0.000000pt}%
\definecolor{currentstroke}{rgb}{0.000000,0.000000,0.000000}%
\pgfsetstrokecolor{currentstroke}%
\pgfsetdash{}{0pt}%
\pgfpathmoveto{\pgfqpoint{1.901196in}{1.264478in}}%
\pgfpathlineto{\pgfqpoint{1.936593in}{1.273119in}}%
\pgfpathlineto{\pgfqpoint{1.900244in}{1.310969in}}%
\pgfpathlineto{\pgfqpoint{1.864747in}{1.302429in}}%
\pgfpathclose%
\pgfusepath{fill}%
\end{pgfscope}%
\begin{pgfscope}%
\pgfpathrectangle{\pgfqpoint{0.150000in}{0.150000in}}{\pgfqpoint{2.700000in}{1.950000in}}%
\pgfusepath{clip}%
\pgfsetbuttcap%
\pgfsetroundjoin%
\definecolor{currentfill}{rgb}{0.916039,0.926379,0.940855}%
\pgfsetfillcolor{currentfill}%
\pgfsetlinewidth{0.000000pt}%
\definecolor{currentstroke}{rgb}{0.000000,0.000000,0.000000}%
\pgfsetstrokecolor{currentstroke}%
\pgfsetdash{}{0pt}%
\pgfpathmoveto{\pgfqpoint{1.864747in}{1.302429in}}%
\pgfpathlineto{\pgfqpoint{1.900244in}{1.310969in}}%
\pgfpathlineto{\pgfqpoint{1.863890in}{1.348824in}}%
\pgfpathlineto{\pgfqpoint{1.828292in}{1.340386in}}%
\pgfpathclose%
\pgfusepath{fill}%
\end{pgfscope}%
\begin{pgfscope}%
\pgfpathrectangle{\pgfqpoint{0.150000in}{0.150000in}}{\pgfqpoint{2.700000in}{1.950000in}}%
\pgfusepath{clip}%
\pgfsetbuttcap%
\pgfsetroundjoin%
\definecolor{currentfill}{rgb}{0.884942,0.899112,0.918949}%
\pgfsetfillcolor{currentfill}%
\pgfsetlinewidth{0.000000pt}%
\definecolor{currentstroke}{rgb}{0.000000,0.000000,0.000000}%
\pgfsetstrokecolor{currentstroke}%
\pgfsetdash{}{0pt}%
\pgfpathmoveto{\pgfqpoint{1.828292in}{1.340386in}}%
\pgfpathlineto{\pgfqpoint{1.863890in}{1.348824in}}%
\pgfpathlineto{\pgfqpoint{1.827530in}{1.386684in}}%
\pgfpathlineto{\pgfqpoint{1.791833in}{1.378347in}}%
\pgfpathclose%
\pgfusepath{fill}%
\end{pgfscope}%
\begin{pgfscope}%
\pgfpathrectangle{\pgfqpoint{0.150000in}{0.150000in}}{\pgfqpoint{2.700000in}{1.950000in}}%
\pgfusepath{clip}%
\pgfsetbuttcap%
\pgfsetroundjoin%
\definecolor{currentfill}{rgb}{0.853845,0.871844,0.897044}%
\pgfsetfillcolor{currentfill}%
\pgfsetlinewidth{0.000000pt}%
\definecolor{currentstroke}{rgb}{0.000000,0.000000,0.000000}%
\pgfsetstrokecolor{currentstroke}%
\pgfsetdash{}{0pt}%
\pgfpathmoveto{\pgfqpoint{1.791833in}{1.378347in}}%
\pgfpathlineto{\pgfqpoint{1.827530in}{1.386684in}}%
\pgfpathlineto{\pgfqpoint{1.791167in}{1.424549in}}%
\pgfpathlineto{\pgfqpoint{1.755370in}{1.416313in}}%
\pgfpathclose%
\pgfusepath{fill}%
\end{pgfscope}%
\begin{pgfscope}%
\pgfpathrectangle{\pgfqpoint{0.150000in}{0.150000in}}{\pgfqpoint{2.700000in}{1.950000in}}%
\pgfusepath{clip}%
\pgfsetbuttcap%
\pgfsetroundjoin%
\definecolor{currentfill}{rgb}{0.822748,0.844577,0.875138}%
\pgfsetfillcolor{currentfill}%
\pgfsetlinewidth{0.000000pt}%
\definecolor{currentstroke}{rgb}{0.000000,0.000000,0.000000}%
\pgfsetstrokecolor{currentstroke}%
\pgfsetdash{}{0pt}%
\pgfpathmoveto{\pgfqpoint{1.755370in}{1.416313in}}%
\pgfpathlineto{\pgfqpoint{1.791167in}{1.424549in}}%
\pgfpathlineto{\pgfqpoint{1.754798in}{1.462420in}}%
\pgfpathlineto{\pgfqpoint{1.718901in}{1.454284in}}%
\pgfpathclose%
\pgfusepath{fill}%
\end{pgfscope}%
\begin{pgfscope}%
\pgfpathrectangle{\pgfqpoint{0.150000in}{0.150000in}}{\pgfqpoint{2.700000in}{1.950000in}}%
\pgfusepath{clip}%
\pgfsetbuttcap%
\pgfsetroundjoin%
\definecolor{currentfill}{rgb}{0.797871,0.822763,0.857613}%
\pgfsetfillcolor{currentfill}%
\pgfsetlinewidth{0.000000pt}%
\definecolor{currentstroke}{rgb}{0.000000,0.000000,0.000000}%
\pgfsetstrokecolor{currentstroke}%
\pgfsetdash{}{0pt}%
\pgfpathmoveto{\pgfqpoint{1.718901in}{1.454284in}}%
\pgfpathlineto{\pgfqpoint{1.754798in}{1.462420in}}%
\pgfpathlineto{\pgfqpoint{1.718425in}{1.500295in}}%
\pgfpathlineto{\pgfqpoint{1.682428in}{1.492261in}}%
\pgfpathclose%
\pgfusepath{fill}%
\end{pgfscope}%
\begin{pgfscope}%
\pgfpathrectangle{\pgfqpoint{0.150000in}{0.150000in}}{\pgfqpoint{2.700000in}{1.950000in}}%
\pgfusepath{clip}%
\pgfsetbuttcap%
\pgfsetroundjoin%
\definecolor{currentfill}{rgb}{0.766774,0.795496,0.835708}%
\pgfsetfillcolor{currentfill}%
\pgfsetlinewidth{0.000000pt}%
\definecolor{currentstroke}{rgb}{0.000000,0.000000,0.000000}%
\pgfsetstrokecolor{currentstroke}%
\pgfsetdash{}{0pt}%
\pgfpathmoveto{\pgfqpoint{1.682428in}{1.492261in}}%
\pgfpathlineto{\pgfqpoint{1.718425in}{1.500295in}}%
\pgfpathlineto{\pgfqpoint{1.682047in}{1.538175in}}%
\pgfpathlineto{\pgfqpoint{1.645950in}{1.530242in}}%
\pgfpathclose%
\pgfusepath{fill}%
\end{pgfscope}%
\begin{pgfscope}%
\pgfpathrectangle{\pgfqpoint{0.150000in}{0.150000in}}{\pgfqpoint{2.700000in}{1.950000in}}%
\pgfusepath{clip}%
\pgfsetbuttcap%
\pgfsetroundjoin%
\definecolor{currentfill}{rgb}{0.735677,0.768229,0.813802}%
\pgfsetfillcolor{currentfill}%
\pgfsetlinewidth{0.000000pt}%
\definecolor{currentstroke}{rgb}{0.000000,0.000000,0.000000}%
\pgfsetstrokecolor{currentstroke}%
\pgfsetdash{}{0pt}%
\pgfpathmoveto{\pgfqpoint{1.645950in}{1.530242in}}%
\pgfpathlineto{\pgfqpoint{1.682047in}{1.538175in}}%
\pgfpathlineto{\pgfqpoint{1.645664in}{1.576060in}}%
\pgfpathlineto{\pgfqpoint{1.609467in}{1.568228in}}%
\pgfpathclose%
\pgfusepath{fill}%
\end{pgfscope}%
\begin{pgfscope}%
\pgfpathrectangle{\pgfqpoint{0.150000in}{0.150000in}}{\pgfqpoint{2.700000in}{1.950000in}}%
\pgfusepath{clip}%
\pgfsetbuttcap%
\pgfsetroundjoin%
\definecolor{currentfill}{rgb}{0.704580,0.740962,0.791896}%
\pgfsetfillcolor{currentfill}%
\pgfsetlinewidth{0.000000pt}%
\definecolor{currentstroke}{rgb}{0.000000,0.000000,0.000000}%
\pgfsetstrokecolor{currentstroke}%
\pgfsetdash{}{0pt}%
\pgfpathmoveto{\pgfqpoint{1.609467in}{1.568228in}}%
\pgfpathlineto{\pgfqpoint{1.645664in}{1.576060in}}%
\pgfpathlineto{\pgfqpoint{1.609276in}{1.613949in}}%
\pgfpathlineto{\pgfqpoint{1.572979in}{1.606219in}}%
\pgfpathclose%
\pgfusepath{fill}%
\end{pgfscope}%
\begin{pgfscope}%
\pgfpathrectangle{\pgfqpoint{0.150000in}{0.150000in}}{\pgfqpoint{2.700000in}{1.950000in}}%
\pgfusepath{clip}%
\pgfsetbuttcap%
\pgfsetroundjoin%
\definecolor{currentfill}{rgb}{0.673483,0.713695,0.769991}%
\pgfsetfillcolor{currentfill}%
\pgfsetlinewidth{0.000000pt}%
\definecolor{currentstroke}{rgb}{0.000000,0.000000,0.000000}%
\pgfsetstrokecolor{currentstroke}%
\pgfsetdash{}{0pt}%
\pgfpathmoveto{\pgfqpoint{1.572979in}{1.606219in}}%
\pgfpathlineto{\pgfqpoint{1.609276in}{1.613949in}}%
\pgfpathlineto{\pgfqpoint{1.572884in}{1.651844in}}%
\pgfpathlineto{\pgfqpoint{1.536486in}{1.644215in}}%
\pgfpathclose%
\pgfusepath{fill}%
\end{pgfscope}%
\begin{pgfscope}%
\pgfpathrectangle{\pgfqpoint{0.150000in}{0.150000in}}{\pgfqpoint{2.700000in}{1.950000in}}%
\pgfusepath{clip}%
\pgfsetbuttcap%
\pgfsetroundjoin%
\definecolor{currentfill}{rgb}{0.642387,0.686428,0.748085}%
\pgfsetfillcolor{currentfill}%
\pgfsetlinewidth{0.000000pt}%
\definecolor{currentstroke}{rgb}{0.000000,0.000000,0.000000}%
\pgfsetstrokecolor{currentstroke}%
\pgfsetdash{}{0pt}%
\pgfpathmoveto{\pgfqpoint{1.536486in}{1.644215in}}%
\pgfpathlineto{\pgfqpoint{1.572884in}{1.651844in}}%
\pgfpathlineto{\pgfqpoint{1.536486in}{1.689744in}}%
\pgfpathlineto{\pgfqpoint{1.499989in}{1.682216in}}%
\pgfpathclose%
\pgfusepath{fill}%
\end{pgfscope}%
\begin{pgfscope}%
\pgfpathrectangle{\pgfqpoint{0.150000in}{0.150000in}}{\pgfqpoint{2.700000in}{1.950000in}}%
\pgfusepath{clip}%
\pgfsetbuttcap%
\pgfsetroundjoin%
\definecolor{currentfill}{rgb}{0.611290,0.659161,0.726180}%
\pgfsetfillcolor{currentfill}%
\pgfsetlinewidth{0.000000pt}%
\definecolor{currentstroke}{rgb}{0.000000,0.000000,0.000000}%
\pgfsetstrokecolor{currentstroke}%
\pgfsetdash{}{0pt}%
\pgfpathmoveto{\pgfqpoint{1.499989in}{1.682216in}}%
\pgfpathlineto{\pgfqpoint{1.536486in}{1.689744in}}%
\pgfpathlineto{\pgfqpoint{1.500085in}{1.727649in}}%
\pgfpathlineto{\pgfqpoint{1.463487in}{1.720222in}}%
\pgfpathclose%
\pgfusepath{fill}%
\end{pgfscope}%
\begin{pgfscope}%
\pgfpathrectangle{\pgfqpoint{0.150000in}{0.150000in}}{\pgfqpoint{2.700000in}{1.950000in}}%
\pgfusepath{clip}%
\pgfsetbuttcap%
\pgfsetroundjoin%
\definecolor{currentfill}{rgb}{0.800551,0.638097,0.650965}%
\pgfsetfillcolor{currentfill}%
\pgfsetlinewidth{0.000000pt}%
\definecolor{currentstroke}{rgb}{0.000000,0.000000,0.000000}%
\pgfsetstrokecolor{currentstroke}%
\pgfsetdash{}{0pt}%
\pgfpathmoveto{\pgfqpoint{2.340320in}{0.761559in}}%
\pgfpathlineto{\pgfqpoint{2.374605in}{0.771564in}}%
\pgfpathlineto{\pgfqpoint{2.338217in}{0.809450in}}%
\pgfpathlineto{\pgfqpoint{2.303832in}{0.799547in}}%
\pgfpathclose%
\pgfusepath{fill}%
\end{pgfscope}%
\begin{pgfscope}%
\pgfpathrectangle{\pgfqpoint{0.150000in}{0.150000in}}{\pgfqpoint{2.700000in}{1.950000in}}%
\pgfusepath{clip}%
\pgfsetbuttcap%
\pgfsetroundjoin%
\definecolor{currentfill}{rgb}{0.819547,0.672564,0.684206}%
\pgfsetfillcolor{currentfill}%
\pgfsetlinewidth{0.000000pt}%
\definecolor{currentstroke}{rgb}{0.000000,0.000000,0.000000}%
\pgfsetstrokecolor{currentstroke}%
\pgfsetdash{}{0pt}%
\pgfpathmoveto{\pgfqpoint{2.303832in}{0.799547in}}%
\pgfpathlineto{\pgfqpoint{2.338217in}{0.809450in}}%
\pgfpathlineto{\pgfqpoint{2.301825in}{0.847342in}}%
\pgfpathlineto{\pgfqpoint{2.267340in}{0.837540in}}%
\pgfpathclose%
\pgfusepath{fill}%
\end{pgfscope}%
\begin{pgfscope}%
\pgfpathrectangle{\pgfqpoint{0.150000in}{0.150000in}}{\pgfqpoint{2.700000in}{1.950000in}}%
\pgfusepath{clip}%
\pgfsetbuttcap%
\pgfsetroundjoin%
\definecolor{currentfill}{rgb}{0.838542,0.707031,0.717448}%
\pgfsetfillcolor{currentfill}%
\pgfsetlinewidth{0.000000pt}%
\definecolor{currentstroke}{rgb}{0.000000,0.000000,0.000000}%
\pgfsetstrokecolor{currentstroke}%
\pgfsetdash{}{0pt}%
\pgfpathmoveto{\pgfqpoint{2.267340in}{0.837540in}}%
\pgfpathlineto{\pgfqpoint{2.301825in}{0.847342in}}%
\pgfpathlineto{\pgfqpoint{2.265428in}{0.885239in}}%
\pgfpathlineto{\pgfqpoint{2.230843in}{0.875538in}}%
\pgfpathclose%
\pgfusepath{fill}%
\end{pgfscope}%
\begin{pgfscope}%
\pgfpathrectangle{\pgfqpoint{0.150000in}{0.150000in}}{\pgfqpoint{2.700000in}{1.950000in}}%
\pgfusepath{clip}%
\pgfsetbuttcap%
\pgfsetroundjoin%
\definecolor{currentfill}{rgb}{0.857537,0.741498,0.750689}%
\pgfsetfillcolor{currentfill}%
\pgfsetlinewidth{0.000000pt}%
\definecolor{currentstroke}{rgb}{0.000000,0.000000,0.000000}%
\pgfsetstrokecolor{currentstroke}%
\pgfsetdash{}{0pt}%
\pgfpathmoveto{\pgfqpoint{2.230843in}{0.875538in}}%
\pgfpathlineto{\pgfqpoint{2.265428in}{0.885239in}}%
\pgfpathlineto{\pgfqpoint{2.229026in}{0.923140in}}%
\pgfpathlineto{\pgfqpoint{2.194341in}{0.913540in}}%
\pgfpathclose%
\pgfusepath{fill}%
\end{pgfscope}%
\begin{pgfscope}%
\pgfpathrectangle{\pgfqpoint{0.150000in}{0.150000in}}{\pgfqpoint{2.700000in}{1.950000in}}%
\pgfusepath{clip}%
\pgfsetbuttcap%
\pgfsetroundjoin%
\definecolor{currentfill}{rgb}{0.876532,0.775965,0.783931}%
\pgfsetfillcolor{currentfill}%
\pgfsetlinewidth{0.000000pt}%
\definecolor{currentstroke}{rgb}{0.000000,0.000000,0.000000}%
\pgfsetstrokecolor{currentstroke}%
\pgfsetdash{}{0pt}%
\pgfpathmoveto{\pgfqpoint{2.194341in}{0.913540in}}%
\pgfpathlineto{\pgfqpoint{2.229026in}{0.923140in}}%
\pgfpathlineto{\pgfqpoint{2.192620in}{0.961047in}}%
\pgfpathlineto{\pgfqpoint{2.157834in}{0.951548in}}%
\pgfpathclose%
\pgfusepath{fill}%
\end{pgfscope}%
\begin{pgfscope}%
\pgfpathrectangle{\pgfqpoint{0.150000in}{0.150000in}}{\pgfqpoint{2.700000in}{1.950000in}}%
\pgfusepath{clip}%
\pgfsetbuttcap%
\pgfsetroundjoin%
\definecolor{currentfill}{rgb}{0.891728,0.803539,0.810524}%
\pgfsetfillcolor{currentfill}%
\pgfsetlinewidth{0.000000pt}%
\definecolor{currentstroke}{rgb}{0.000000,0.000000,0.000000}%
\pgfsetstrokecolor{currentstroke}%
\pgfsetdash{}{0pt}%
\pgfpathmoveto{\pgfqpoint{2.157834in}{0.951548in}}%
\pgfpathlineto{\pgfqpoint{2.192620in}{0.961047in}}%
\pgfpathlineto{\pgfqpoint{2.156209in}{0.998958in}}%
\pgfpathlineto{\pgfqpoint{2.121323in}{0.989561in}}%
\pgfpathclose%
\pgfusepath{fill}%
\end{pgfscope}%
\begin{pgfscope}%
\pgfpathrectangle{\pgfqpoint{0.150000in}{0.150000in}}{\pgfqpoint{2.700000in}{1.950000in}}%
\pgfusepath{clip}%
\pgfsetbuttcap%
\pgfsetroundjoin%
\definecolor{currentfill}{rgb}{0.910723,0.838006,0.843765}%
\pgfsetfillcolor{currentfill}%
\pgfsetlinewidth{0.000000pt}%
\definecolor{currentstroke}{rgb}{0.000000,0.000000,0.000000}%
\pgfsetstrokecolor{currentstroke}%
\pgfsetdash{}{0pt}%
\pgfpathmoveto{\pgfqpoint{2.121323in}{0.989561in}}%
\pgfpathlineto{\pgfqpoint{2.156209in}{0.998958in}}%
\pgfpathlineto{\pgfqpoint{2.119793in}{1.036875in}}%
\pgfpathlineto{\pgfqpoint{2.084807in}{1.027579in}}%
\pgfpathclose%
\pgfusepath{fill}%
\end{pgfscope}%
\begin{pgfscope}%
\pgfpathrectangle{\pgfqpoint{0.150000in}{0.150000in}}{\pgfqpoint{2.700000in}{1.950000in}}%
\pgfusepath{clip}%
\pgfsetbuttcap%
\pgfsetroundjoin%
\definecolor{currentfill}{rgb}{0.929718,0.872472,0.877007}%
\pgfsetfillcolor{currentfill}%
\pgfsetlinewidth{0.000000pt}%
\definecolor{currentstroke}{rgb}{0.000000,0.000000,0.000000}%
\pgfsetstrokecolor{currentstroke}%
\pgfsetdash{}{0pt}%
\pgfpathmoveto{\pgfqpoint{2.084807in}{1.027579in}}%
\pgfpathlineto{\pgfqpoint{2.119793in}{1.036875in}}%
\pgfpathlineto{\pgfqpoint{2.083372in}{1.074796in}}%
\pgfpathlineto{\pgfqpoint{2.048286in}{1.065602in}}%
\pgfpathclose%
\pgfusepath{fill}%
\end{pgfscope}%
\begin{pgfscope}%
\pgfpathrectangle{\pgfqpoint{0.150000in}{0.150000in}}{\pgfqpoint{2.700000in}{1.950000in}}%
\pgfusepath{clip}%
\pgfsetbuttcap%
\pgfsetroundjoin%
\definecolor{currentfill}{rgb}{0.948713,0.906939,0.910248}%
\pgfsetfillcolor{currentfill}%
\pgfsetlinewidth{0.000000pt}%
\definecolor{currentstroke}{rgb}{0.000000,0.000000,0.000000}%
\pgfsetstrokecolor{currentstroke}%
\pgfsetdash{}{0pt}%
\pgfpathmoveto{\pgfqpoint{2.048286in}{1.065602in}}%
\pgfpathlineto{\pgfqpoint{2.083372in}{1.074796in}}%
\pgfpathlineto{\pgfqpoint{2.046946in}{1.112723in}}%
\pgfpathlineto{\pgfqpoint{2.011760in}{1.103630in}}%
\pgfpathclose%
\pgfusepath{fill}%
\end{pgfscope}%
\begin{pgfscope}%
\pgfpathrectangle{\pgfqpoint{0.150000in}{0.150000in}}{\pgfqpoint{2.700000in}{1.950000in}}%
\pgfusepath{clip}%
\pgfsetbuttcap%
\pgfsetroundjoin%
\definecolor{currentfill}{rgb}{0.967708,0.941406,0.943490}%
\pgfsetfillcolor{currentfill}%
\pgfsetlinewidth{0.000000pt}%
\definecolor{currentstroke}{rgb}{0.000000,0.000000,0.000000}%
\pgfsetstrokecolor{currentstroke}%
\pgfsetdash{}{0pt}%
\pgfpathmoveto{\pgfqpoint{2.011760in}{1.103630in}}%
\pgfpathlineto{\pgfqpoint{2.046946in}{1.112723in}}%
\pgfpathlineto{\pgfqpoint{2.010516in}{1.150654in}}%
\pgfpathlineto{\pgfqpoint{1.975229in}{1.141663in}}%
\pgfpathclose%
\pgfusepath{fill}%
\end{pgfscope}%
\begin{pgfscope}%
\pgfpathrectangle{\pgfqpoint{0.150000in}{0.150000in}}{\pgfqpoint{2.700000in}{1.950000in}}%
\pgfusepath{clip}%
\pgfsetbuttcap%
\pgfsetroundjoin%
\definecolor{currentfill}{rgb}{0.986703,0.975873,0.976731}%
\pgfsetfillcolor{currentfill}%
\pgfsetlinewidth{0.000000pt}%
\definecolor{currentstroke}{rgb}{0.000000,0.000000,0.000000}%
\pgfsetstrokecolor{currentstroke}%
\pgfsetdash{}{0pt}%
\pgfpathmoveto{\pgfqpoint{1.975229in}{1.141663in}}%
\pgfpathlineto{\pgfqpoint{2.010516in}{1.150654in}}%
\pgfpathlineto{\pgfqpoint{1.974081in}{1.188590in}}%
\pgfpathlineto{\pgfqpoint{1.938694in}{1.179700in}}%
\pgfpathclose%
\pgfusepath{fill}%
\end{pgfscope}%
\begin{pgfscope}%
\pgfpathrectangle{\pgfqpoint{0.150000in}{0.150000in}}{\pgfqpoint{2.700000in}{1.950000in}}%
\pgfusepath{clip}%
\pgfsetbuttcap%
\pgfsetroundjoin%
\definecolor{currentfill}{rgb}{0.990671,0.991820,0.993428}%
\pgfsetfillcolor{currentfill}%
\pgfsetlinewidth{0.000000pt}%
\definecolor{currentstroke}{rgb}{0.000000,0.000000,0.000000}%
\pgfsetstrokecolor{currentstroke}%
\pgfsetdash{}{0pt}%
\pgfpathmoveto{\pgfqpoint{1.938694in}{1.179700in}}%
\pgfpathlineto{\pgfqpoint{1.974081in}{1.188590in}}%
\pgfpathlineto{\pgfqpoint{1.937641in}{1.226532in}}%
\pgfpathlineto{\pgfqpoint{1.902153in}{1.217743in}}%
\pgfpathclose%
\pgfusepath{fill}%
\end{pgfscope}%
\begin{pgfscope}%
\pgfpathrectangle{\pgfqpoint{0.150000in}{0.150000in}}{\pgfqpoint{2.700000in}{1.950000in}}%
\pgfusepath{clip}%
\pgfsetbuttcap%
\pgfsetroundjoin%
\definecolor{currentfill}{rgb}{0.959574,0.964553,0.971523}%
\pgfsetfillcolor{currentfill}%
\pgfsetlinewidth{0.000000pt}%
\definecolor{currentstroke}{rgb}{0.000000,0.000000,0.000000}%
\pgfsetstrokecolor{currentstroke}%
\pgfsetdash{}{0pt}%
\pgfpathmoveto{\pgfqpoint{1.902153in}{1.217743in}}%
\pgfpathlineto{\pgfqpoint{1.937641in}{1.226532in}}%
\pgfpathlineto{\pgfqpoint{1.901196in}{1.264478in}}%
\pgfpathlineto{\pgfqpoint{1.865608in}{1.255791in}}%
\pgfpathclose%
\pgfusepath{fill}%
\end{pgfscope}%
\begin{pgfscope}%
\pgfpathrectangle{\pgfqpoint{0.150000in}{0.150000in}}{\pgfqpoint{2.700000in}{1.950000in}}%
\pgfusepath{clip}%
\pgfsetbuttcap%
\pgfsetroundjoin%
\definecolor{currentfill}{rgb}{0.934697,0.942739,0.953998}%
\pgfsetfillcolor{currentfill}%
\pgfsetlinewidth{0.000000pt}%
\definecolor{currentstroke}{rgb}{0.000000,0.000000,0.000000}%
\pgfsetstrokecolor{currentstroke}%
\pgfsetdash{}{0pt}%
\pgfpathmoveto{\pgfqpoint{1.865608in}{1.255791in}}%
\pgfpathlineto{\pgfqpoint{1.901196in}{1.264478in}}%
\pgfpathlineto{\pgfqpoint{1.864747in}{1.302429in}}%
\pgfpathlineto{\pgfqpoint{1.829058in}{1.293844in}}%
\pgfpathclose%
\pgfusepath{fill}%
\end{pgfscope}%
\begin{pgfscope}%
\pgfpathrectangle{\pgfqpoint{0.150000in}{0.150000in}}{\pgfqpoint{2.700000in}{1.950000in}}%
\pgfusepath{clip}%
\pgfsetbuttcap%
\pgfsetroundjoin%
\definecolor{currentfill}{rgb}{0.903600,0.915472,0.932093}%
\pgfsetfillcolor{currentfill}%
\pgfsetlinewidth{0.000000pt}%
\definecolor{currentstroke}{rgb}{0.000000,0.000000,0.000000}%
\pgfsetstrokecolor{currentstroke}%
\pgfsetdash{}{0pt}%
\pgfpathmoveto{\pgfqpoint{1.829058in}{1.293844in}}%
\pgfpathlineto{\pgfqpoint{1.864747in}{1.302429in}}%
\pgfpathlineto{\pgfqpoint{1.828292in}{1.340386in}}%
\pgfpathlineto{\pgfqpoint{1.792504in}{1.331902in}}%
\pgfpathclose%
\pgfusepath{fill}%
\end{pgfscope}%
\begin{pgfscope}%
\pgfpathrectangle{\pgfqpoint{0.150000in}{0.150000in}}{\pgfqpoint{2.700000in}{1.950000in}}%
\pgfusepath{clip}%
\pgfsetbuttcap%
\pgfsetroundjoin%
\definecolor{currentfill}{rgb}{0.872503,0.888205,0.910187}%
\pgfsetfillcolor{currentfill}%
\pgfsetlinewidth{0.000000pt}%
\definecolor{currentstroke}{rgb}{0.000000,0.000000,0.000000}%
\pgfsetstrokecolor{currentstroke}%
\pgfsetdash{}{0pt}%
\pgfpathmoveto{\pgfqpoint{1.792504in}{1.331902in}}%
\pgfpathlineto{\pgfqpoint{1.828292in}{1.340386in}}%
\pgfpathlineto{\pgfqpoint{1.791833in}{1.378347in}}%
\pgfpathlineto{\pgfqpoint{1.755944in}{1.369965in}}%
\pgfpathclose%
\pgfusepath{fill}%
\end{pgfscope}%
\begin{pgfscope}%
\pgfpathrectangle{\pgfqpoint{0.150000in}{0.150000in}}{\pgfqpoint{2.700000in}{1.950000in}}%
\pgfusepath{clip}%
\pgfsetbuttcap%
\pgfsetroundjoin%
\definecolor{currentfill}{rgb}{0.841406,0.860938,0.888281}%
\pgfsetfillcolor{currentfill}%
\pgfsetlinewidth{0.000000pt}%
\definecolor{currentstroke}{rgb}{0.000000,0.000000,0.000000}%
\pgfsetstrokecolor{currentstroke}%
\pgfsetdash{}{0pt}%
\pgfpathmoveto{\pgfqpoint{1.755944in}{1.369965in}}%
\pgfpathlineto{\pgfqpoint{1.791833in}{1.378347in}}%
\pgfpathlineto{\pgfqpoint{1.755370in}{1.416313in}}%
\pgfpathlineto{\pgfqpoint{1.719380in}{1.408032in}}%
\pgfpathclose%
\pgfusepath{fill}%
\end{pgfscope}%
\begin{pgfscope}%
\pgfpathrectangle{\pgfqpoint{0.150000in}{0.150000in}}{\pgfqpoint{2.700000in}{1.950000in}}%
\pgfusepath{clip}%
\pgfsetbuttcap%
\pgfsetroundjoin%
\definecolor{currentfill}{rgb}{0.810309,0.833670,0.866376}%
\pgfsetfillcolor{currentfill}%
\pgfsetlinewidth{0.000000pt}%
\definecolor{currentstroke}{rgb}{0.000000,0.000000,0.000000}%
\pgfsetstrokecolor{currentstroke}%
\pgfsetdash{}{0pt}%
\pgfpathmoveto{\pgfqpoint{1.719380in}{1.408032in}}%
\pgfpathlineto{\pgfqpoint{1.755370in}{1.416313in}}%
\pgfpathlineto{\pgfqpoint{1.718901in}{1.454284in}}%
\pgfpathlineto{\pgfqpoint{1.682811in}{1.446105in}}%
\pgfpathclose%
\pgfusepath{fill}%
\end{pgfscope}%
\begin{pgfscope}%
\pgfpathrectangle{\pgfqpoint{0.150000in}{0.150000in}}{\pgfqpoint{2.700000in}{1.950000in}}%
\pgfusepath{clip}%
\pgfsetbuttcap%
\pgfsetroundjoin%
\definecolor{currentfill}{rgb}{0.779213,0.806403,0.844470}%
\pgfsetfillcolor{currentfill}%
\pgfsetlinewidth{0.000000pt}%
\definecolor{currentstroke}{rgb}{0.000000,0.000000,0.000000}%
\pgfsetstrokecolor{currentstroke}%
\pgfsetdash{}{0pt}%
\pgfpathmoveto{\pgfqpoint{1.682811in}{1.446105in}}%
\pgfpathlineto{\pgfqpoint{1.718901in}{1.454284in}}%
\pgfpathlineto{\pgfqpoint{1.682428in}{1.492261in}}%
\pgfpathlineto{\pgfqpoint{1.646237in}{1.484183in}}%
\pgfpathclose%
\pgfusepath{fill}%
\end{pgfscope}%
\begin{pgfscope}%
\pgfpathrectangle{\pgfqpoint{0.150000in}{0.150000in}}{\pgfqpoint{2.700000in}{1.950000in}}%
\pgfusepath{clip}%
\pgfsetbuttcap%
\pgfsetroundjoin%
\definecolor{currentfill}{rgb}{0.748116,0.779136,0.822564}%
\pgfsetfillcolor{currentfill}%
\pgfsetlinewidth{0.000000pt}%
\definecolor{currentstroke}{rgb}{0.000000,0.000000,0.000000}%
\pgfsetstrokecolor{currentstroke}%
\pgfsetdash{}{0pt}%
\pgfpathmoveto{\pgfqpoint{1.646237in}{1.484183in}}%
\pgfpathlineto{\pgfqpoint{1.682428in}{1.492261in}}%
\pgfpathlineto{\pgfqpoint{1.645950in}{1.530242in}}%
\pgfpathlineto{\pgfqpoint{1.609658in}{1.522266in}}%
\pgfpathclose%
\pgfusepath{fill}%
\end{pgfscope}%
\begin{pgfscope}%
\pgfpathrectangle{\pgfqpoint{0.150000in}{0.150000in}}{\pgfqpoint{2.700000in}{1.950000in}}%
\pgfusepath{clip}%
\pgfsetbuttcap%
\pgfsetroundjoin%
\definecolor{currentfill}{rgb}{0.717019,0.751869,0.800659}%
\pgfsetfillcolor{currentfill}%
\pgfsetlinewidth{0.000000pt}%
\definecolor{currentstroke}{rgb}{0.000000,0.000000,0.000000}%
\pgfsetstrokecolor{currentstroke}%
\pgfsetdash{}{0pt}%
\pgfpathmoveto{\pgfqpoint{1.609658in}{1.522266in}}%
\pgfpathlineto{\pgfqpoint{1.645950in}{1.530242in}}%
\pgfpathlineto{\pgfqpoint{1.609467in}{1.568228in}}%
\pgfpathlineto{\pgfqpoint{1.573075in}{1.560354in}}%
\pgfpathclose%
\pgfusepath{fill}%
\end{pgfscope}%
\begin{pgfscope}%
\pgfpathrectangle{\pgfqpoint{0.150000in}{0.150000in}}{\pgfqpoint{2.700000in}{1.950000in}}%
\pgfusepath{clip}%
\pgfsetbuttcap%
\pgfsetroundjoin%
\definecolor{currentfill}{rgb}{0.692142,0.730055,0.783134}%
\pgfsetfillcolor{currentfill}%
\pgfsetlinewidth{0.000000pt}%
\definecolor{currentstroke}{rgb}{0.000000,0.000000,0.000000}%
\pgfsetstrokecolor{currentstroke}%
\pgfsetdash{}{0pt}%
\pgfpathmoveto{\pgfqpoint{1.573075in}{1.560354in}}%
\pgfpathlineto{\pgfqpoint{1.609467in}{1.568228in}}%
\pgfpathlineto{\pgfqpoint{1.572979in}{1.606219in}}%
\pgfpathlineto{\pgfqpoint{1.536486in}{1.598447in}}%
\pgfpathclose%
\pgfusepath{fill}%
\end{pgfscope}%
\begin{pgfscope}%
\pgfpathrectangle{\pgfqpoint{0.150000in}{0.150000in}}{\pgfqpoint{2.700000in}{1.950000in}}%
\pgfusepath{clip}%
\pgfsetbuttcap%
\pgfsetroundjoin%
\definecolor{currentfill}{rgb}{0.661045,0.702788,0.761229}%
\pgfsetfillcolor{currentfill}%
\pgfsetlinewidth{0.000000pt}%
\definecolor{currentstroke}{rgb}{0.000000,0.000000,0.000000}%
\pgfsetstrokecolor{currentstroke}%
\pgfsetdash{}{0pt}%
\pgfpathmoveto{\pgfqpoint{1.536486in}{1.598447in}}%
\pgfpathlineto{\pgfqpoint{1.572979in}{1.606219in}}%
\pgfpathlineto{\pgfqpoint{1.536486in}{1.644215in}}%
\pgfpathlineto{\pgfqpoint{1.499893in}{1.636545in}}%
\pgfpathclose%
\pgfusepath{fill}%
\end{pgfscope}%
\begin{pgfscope}%
\pgfpathrectangle{\pgfqpoint{0.150000in}{0.150000in}}{\pgfqpoint{2.700000in}{1.950000in}}%
\pgfusepath{clip}%
\pgfsetbuttcap%
\pgfsetroundjoin%
\definecolor{currentfill}{rgb}{0.629948,0.675521,0.739323}%
\pgfsetfillcolor{currentfill}%
\pgfsetlinewidth{0.000000pt}%
\definecolor{currentstroke}{rgb}{0.000000,0.000000,0.000000}%
\pgfsetstrokecolor{currentstroke}%
\pgfsetdash{}{0pt}%
\pgfpathmoveto{\pgfqpoint{1.499893in}{1.636545in}}%
\pgfpathlineto{\pgfqpoint{1.536486in}{1.644215in}}%
\pgfpathlineto{\pgfqpoint{1.499989in}{1.682216in}}%
\pgfpathlineto{\pgfqpoint{1.463295in}{1.674648in}}%
\pgfpathclose%
\pgfusepath{fill}%
\end{pgfscope}%
\begin{pgfscope}%
\pgfpathrectangle{\pgfqpoint{0.150000in}{0.150000in}}{\pgfqpoint{2.700000in}{1.950000in}}%
\pgfusepath{clip}%
\pgfsetbuttcap%
\pgfsetroundjoin%
\definecolor{currentfill}{rgb}{0.598851,0.648254,0.717417}%
\pgfsetfillcolor{currentfill}%
\pgfsetlinewidth{0.000000pt}%
\definecolor{currentstroke}{rgb}{0.000000,0.000000,0.000000}%
\pgfsetstrokecolor{currentstroke}%
\pgfsetdash{}{0pt}%
\pgfpathmoveto{\pgfqpoint{1.463295in}{1.674648in}}%
\pgfpathlineto{\pgfqpoint{1.499989in}{1.682216in}}%
\pgfpathlineto{\pgfqpoint{1.463487in}{1.720222in}}%
\pgfpathlineto{\pgfqpoint{1.426693in}{1.712756in}}%
\pgfpathclose%
\pgfusepath{fill}%
\end{pgfscope}%
\begin{pgfscope}%
\pgfpathrectangle{\pgfqpoint{0.150000in}{0.150000in}}{\pgfqpoint{2.700000in}{1.950000in}}%
\pgfusepath{clip}%
\pgfsetbuttcap%
\pgfsetroundjoin%
\definecolor{currentfill}{rgb}{0.811949,0.658778,0.670910}%
\pgfsetfillcolor{currentfill}%
\pgfsetlinewidth{0.000000pt}%
\definecolor{currentstroke}{rgb}{0.000000,0.000000,0.000000}%
\pgfsetstrokecolor{currentstroke}%
\pgfsetdash{}{0pt}%
\pgfpathmoveto{\pgfqpoint{2.305850in}{0.751500in}}%
\pgfpathlineto{\pgfqpoint{2.340320in}{0.761559in}}%
\pgfpathlineto{\pgfqpoint{2.303832in}{0.799547in}}%
\pgfpathlineto{\pgfqpoint{2.269262in}{0.789590in}}%
\pgfpathclose%
\pgfusepath{fill}%
\end{pgfscope}%
\begin{pgfscope}%
\pgfpathrectangle{\pgfqpoint{0.150000in}{0.150000in}}{\pgfqpoint{2.700000in}{1.950000in}}%
\pgfusepath{clip}%
\pgfsetbuttcap%
\pgfsetroundjoin%
\definecolor{currentfill}{rgb}{0.827145,0.686351,0.697503}%
\pgfsetfillcolor{currentfill}%
\pgfsetlinewidth{0.000000pt}%
\definecolor{currentstroke}{rgb}{0.000000,0.000000,0.000000}%
\pgfsetstrokecolor{currentstroke}%
\pgfsetdash{}{0pt}%
\pgfpathmoveto{\pgfqpoint{2.269262in}{0.789590in}}%
\pgfpathlineto{\pgfqpoint{2.303832in}{0.799547in}}%
\pgfpathlineto{\pgfqpoint{2.267340in}{0.837540in}}%
\pgfpathlineto{\pgfqpoint{2.232669in}{0.827685in}}%
\pgfpathclose%
\pgfusepath{fill}%
\end{pgfscope}%
\begin{pgfscope}%
\pgfpathrectangle{\pgfqpoint{0.150000in}{0.150000in}}{\pgfqpoint{2.700000in}{1.950000in}}%
\pgfusepath{clip}%
\pgfsetbuttcap%
\pgfsetroundjoin%
\definecolor{currentfill}{rgb}{0.846140,0.720818,0.730744}%
\pgfsetfillcolor{currentfill}%
\pgfsetlinewidth{0.000000pt}%
\definecolor{currentstroke}{rgb}{0.000000,0.000000,0.000000}%
\pgfsetstrokecolor{currentstroke}%
\pgfsetdash{}{0pt}%
\pgfpathmoveto{\pgfqpoint{2.232669in}{0.827685in}}%
\pgfpathlineto{\pgfqpoint{2.267340in}{0.837540in}}%
\pgfpathlineto{\pgfqpoint{2.230843in}{0.875538in}}%
\pgfpathlineto{\pgfqpoint{2.196071in}{0.865784in}}%
\pgfpathclose%
\pgfusepath{fill}%
\end{pgfscope}%
\begin{pgfscope}%
\pgfpathrectangle{\pgfqpoint{0.150000in}{0.150000in}}{\pgfqpoint{2.700000in}{1.950000in}}%
\pgfusepath{clip}%
\pgfsetbuttcap%
\pgfsetroundjoin%
\definecolor{currentfill}{rgb}{0.865135,0.755285,0.763986}%
\pgfsetfillcolor{currentfill}%
\pgfsetlinewidth{0.000000pt}%
\definecolor{currentstroke}{rgb}{0.000000,0.000000,0.000000}%
\pgfsetstrokecolor{currentstroke}%
\pgfsetdash{}{0pt}%
\pgfpathmoveto{\pgfqpoint{2.196071in}{0.865784in}}%
\pgfpathlineto{\pgfqpoint{2.230843in}{0.875538in}}%
\pgfpathlineto{\pgfqpoint{2.194341in}{0.913540in}}%
\pgfpathlineto{\pgfqpoint{2.159469in}{0.903889in}}%
\pgfpathclose%
\pgfusepath{fill}%
\end{pgfscope}%
\begin{pgfscope}%
\pgfpathrectangle{\pgfqpoint{0.150000in}{0.150000in}}{\pgfqpoint{2.700000in}{1.950000in}}%
\pgfusepath{clip}%
\pgfsetbuttcap%
\pgfsetroundjoin%
\definecolor{currentfill}{rgb}{0.884130,0.789752,0.797227}%
\pgfsetfillcolor{currentfill}%
\pgfsetlinewidth{0.000000pt}%
\definecolor{currentstroke}{rgb}{0.000000,0.000000,0.000000}%
\pgfsetstrokecolor{currentstroke}%
\pgfsetdash{}{0pt}%
\pgfpathmoveto{\pgfqpoint{2.159469in}{0.903889in}}%
\pgfpathlineto{\pgfqpoint{2.194341in}{0.913540in}}%
\pgfpathlineto{\pgfqpoint{2.157834in}{0.951548in}}%
\pgfpathlineto{\pgfqpoint{2.122861in}{0.941999in}}%
\pgfpathclose%
\pgfusepath{fill}%
\end{pgfscope}%
\begin{pgfscope}%
\pgfpathrectangle{\pgfqpoint{0.150000in}{0.150000in}}{\pgfqpoint{2.700000in}{1.950000in}}%
\pgfusepath{clip}%
\pgfsetbuttcap%
\pgfsetroundjoin%
\definecolor{currentfill}{rgb}{0.903125,0.824219,0.830469}%
\pgfsetfillcolor{currentfill}%
\pgfsetlinewidth{0.000000pt}%
\definecolor{currentstroke}{rgb}{0.000000,0.000000,0.000000}%
\pgfsetstrokecolor{currentstroke}%
\pgfsetdash{}{0pt}%
\pgfpathmoveto{\pgfqpoint{2.122861in}{0.941999in}}%
\pgfpathlineto{\pgfqpoint{2.157834in}{0.951548in}}%
\pgfpathlineto{\pgfqpoint{2.121323in}{0.989561in}}%
\pgfpathlineto{\pgfqpoint{2.086249in}{0.980113in}}%
\pgfpathclose%
\pgfusepath{fill}%
\end{pgfscope}%
\begin{pgfscope}%
\pgfpathrectangle{\pgfqpoint{0.150000in}{0.150000in}}{\pgfqpoint{2.700000in}{1.950000in}}%
\pgfusepath{clip}%
\pgfsetbuttcap%
\pgfsetroundjoin%
\definecolor{currentfill}{rgb}{0.922120,0.858686,0.863710}%
\pgfsetfillcolor{currentfill}%
\pgfsetlinewidth{0.000000pt}%
\definecolor{currentstroke}{rgb}{0.000000,0.000000,0.000000}%
\pgfsetstrokecolor{currentstroke}%
\pgfsetdash{}{0pt}%
\pgfpathmoveto{\pgfqpoint{2.086249in}{0.980113in}}%
\pgfpathlineto{\pgfqpoint{2.121323in}{0.989561in}}%
\pgfpathlineto{\pgfqpoint{2.084807in}{1.027579in}}%
\pgfpathlineto{\pgfqpoint{2.049632in}{1.018233in}}%
\pgfpathclose%
\pgfusepath{fill}%
\end{pgfscope}%
\begin{pgfscope}%
\pgfpathrectangle{\pgfqpoint{0.150000in}{0.150000in}}{\pgfqpoint{2.700000in}{1.950000in}}%
\pgfusepath{clip}%
\pgfsetbuttcap%
\pgfsetroundjoin%
\definecolor{currentfill}{rgb}{0.941115,0.893153,0.896952}%
\pgfsetfillcolor{currentfill}%
\pgfsetlinewidth{0.000000pt}%
\definecolor{currentstroke}{rgb}{0.000000,0.000000,0.000000}%
\pgfsetstrokecolor{currentstroke}%
\pgfsetdash{}{0pt}%
\pgfpathmoveto{\pgfqpoint{2.049632in}{1.018233in}}%
\pgfpathlineto{\pgfqpoint{2.084807in}{1.027579in}}%
\pgfpathlineto{\pgfqpoint{2.048286in}{1.065602in}}%
\pgfpathlineto{\pgfqpoint{2.013010in}{1.056358in}}%
\pgfpathclose%
\pgfusepath{fill}%
\end{pgfscope}%
\begin{pgfscope}%
\pgfpathrectangle{\pgfqpoint{0.150000in}{0.150000in}}{\pgfqpoint{2.700000in}{1.950000in}}%
\pgfusepath{clip}%
\pgfsetbuttcap%
\pgfsetroundjoin%
\definecolor{currentfill}{rgb}{0.960110,0.927619,0.930193}%
\pgfsetfillcolor{currentfill}%
\pgfsetlinewidth{0.000000pt}%
\definecolor{currentstroke}{rgb}{0.000000,0.000000,0.000000}%
\pgfsetstrokecolor{currentstroke}%
\pgfsetdash{}{0pt}%
\pgfpathmoveto{\pgfqpoint{2.013010in}{1.056358in}}%
\pgfpathlineto{\pgfqpoint{2.048286in}{1.065602in}}%
\pgfpathlineto{\pgfqpoint{2.011760in}{1.103630in}}%
\pgfpathlineto{\pgfqpoint{1.976383in}{1.094488in}}%
\pgfpathclose%
\pgfusepath{fill}%
\end{pgfscope}%
\begin{pgfscope}%
\pgfpathrectangle{\pgfqpoint{0.150000in}{0.150000in}}{\pgfqpoint{2.700000in}{1.950000in}}%
\pgfusepath{clip}%
\pgfsetbuttcap%
\pgfsetroundjoin%
\definecolor{currentfill}{rgb}{0.975306,0.955193,0.956786}%
\pgfsetfillcolor{currentfill}%
\pgfsetlinewidth{0.000000pt}%
\definecolor{currentstroke}{rgb}{0.000000,0.000000,0.000000}%
\pgfsetstrokecolor{currentstroke}%
\pgfsetdash{}{0pt}%
\pgfpathmoveto{\pgfqpoint{1.976383in}{1.094488in}}%
\pgfpathlineto{\pgfqpoint{2.011760in}{1.103630in}}%
\pgfpathlineto{\pgfqpoint{1.975229in}{1.141663in}}%
\pgfpathlineto{\pgfqpoint{1.939752in}{1.132623in}}%
\pgfpathclose%
\pgfusepath{fill}%
\end{pgfscope}%
\begin{pgfscope}%
\pgfpathrectangle{\pgfqpoint{0.150000in}{0.150000in}}{\pgfqpoint{2.700000in}{1.950000in}}%
\pgfusepath{clip}%
\pgfsetbuttcap%
\pgfsetroundjoin%
\definecolor{currentfill}{rgb}{0.994301,0.989660,0.990028}%
\pgfsetfillcolor{currentfill}%
\pgfsetlinewidth{0.000000pt}%
\definecolor{currentstroke}{rgb}{0.000000,0.000000,0.000000}%
\pgfsetstrokecolor{currentstroke}%
\pgfsetdash{}{0pt}%
\pgfpathmoveto{\pgfqpoint{1.939752in}{1.132623in}}%
\pgfpathlineto{\pgfqpoint{1.975229in}{1.141663in}}%
\pgfpathlineto{\pgfqpoint{1.938694in}{1.179700in}}%
\pgfpathlineto{\pgfqpoint{1.903116in}{1.170762in}}%
\pgfpathclose%
\pgfusepath{fill}%
\end{pgfscope}%
\begin{pgfscope}%
\pgfpathrectangle{\pgfqpoint{0.150000in}{0.150000in}}{\pgfqpoint{2.700000in}{1.950000in}}%
\pgfusepath{clip}%
\pgfsetbuttcap%
\pgfsetroundjoin%
\definecolor{currentfill}{rgb}{0.978232,0.980913,0.984666}%
\pgfsetfillcolor{currentfill}%
\pgfsetlinewidth{0.000000pt}%
\definecolor{currentstroke}{rgb}{0.000000,0.000000,0.000000}%
\pgfsetstrokecolor{currentstroke}%
\pgfsetdash{}{0pt}%
\pgfpathmoveto{\pgfqpoint{1.903116in}{1.170762in}}%
\pgfpathlineto{\pgfqpoint{1.938694in}{1.179700in}}%
\pgfpathlineto{\pgfqpoint{1.902153in}{1.217743in}}%
\pgfpathlineto{\pgfqpoint{1.866474in}{1.208907in}}%
\pgfpathclose%
\pgfusepath{fill}%
\end{pgfscope}%
\begin{pgfscope}%
\pgfpathrectangle{\pgfqpoint{0.150000in}{0.150000in}}{\pgfqpoint{2.700000in}{1.950000in}}%
\pgfusepath{clip}%
\pgfsetbuttcap%
\pgfsetroundjoin%
\definecolor{currentfill}{rgb}{0.947135,0.953646,0.962760}%
\pgfsetfillcolor{currentfill}%
\pgfsetlinewidth{0.000000pt}%
\definecolor{currentstroke}{rgb}{0.000000,0.000000,0.000000}%
\pgfsetstrokecolor{currentstroke}%
\pgfsetdash{}{0pt}%
\pgfpathmoveto{\pgfqpoint{1.866474in}{1.208907in}}%
\pgfpathlineto{\pgfqpoint{1.902153in}{1.217743in}}%
\pgfpathlineto{\pgfqpoint{1.865608in}{1.255791in}}%
\pgfpathlineto{\pgfqpoint{1.829828in}{1.247057in}}%
\pgfpathclose%
\pgfusepath{fill}%
\end{pgfscope}%
\begin{pgfscope}%
\pgfpathrectangle{\pgfqpoint{0.150000in}{0.150000in}}{\pgfqpoint{2.700000in}{1.950000in}}%
\pgfusepath{clip}%
\pgfsetbuttcap%
\pgfsetroundjoin%
\definecolor{currentfill}{rgb}{0.916039,0.926379,0.940855}%
\pgfsetfillcolor{currentfill}%
\pgfsetlinewidth{0.000000pt}%
\definecolor{currentstroke}{rgb}{0.000000,0.000000,0.000000}%
\pgfsetstrokecolor{currentstroke}%
\pgfsetdash{}{0pt}%
\pgfpathmoveto{\pgfqpoint{1.829828in}{1.247057in}}%
\pgfpathlineto{\pgfqpoint{1.865608in}{1.255791in}}%
\pgfpathlineto{\pgfqpoint{1.829058in}{1.293844in}}%
\pgfpathlineto{\pgfqpoint{1.793178in}{1.285212in}}%
\pgfpathclose%
\pgfusepath{fill}%
\end{pgfscope}%
\begin{pgfscope}%
\pgfpathrectangle{\pgfqpoint{0.150000in}{0.150000in}}{\pgfqpoint{2.700000in}{1.950000in}}%
\pgfusepath{clip}%
\pgfsetbuttcap%
\pgfsetroundjoin%
\definecolor{currentfill}{rgb}{0.884942,0.899112,0.918949}%
\pgfsetfillcolor{currentfill}%
\pgfsetlinewidth{0.000000pt}%
\definecolor{currentstroke}{rgb}{0.000000,0.000000,0.000000}%
\pgfsetstrokecolor{currentstroke}%
\pgfsetdash{}{0pt}%
\pgfpathmoveto{\pgfqpoint{1.793178in}{1.285212in}}%
\pgfpathlineto{\pgfqpoint{1.829058in}{1.293844in}}%
\pgfpathlineto{\pgfqpoint{1.792504in}{1.331902in}}%
\pgfpathlineto{\pgfqpoint{1.756522in}{1.323372in}}%
\pgfpathclose%
\pgfusepath{fill}%
\end{pgfscope}%
\begin{pgfscope}%
\pgfpathrectangle{\pgfqpoint{0.150000in}{0.150000in}}{\pgfqpoint{2.700000in}{1.950000in}}%
\pgfusepath{clip}%
\pgfsetbuttcap%
\pgfsetroundjoin%
\definecolor{currentfill}{rgb}{0.853845,0.871844,0.897044}%
\pgfsetfillcolor{currentfill}%
\pgfsetlinewidth{0.000000pt}%
\definecolor{currentstroke}{rgb}{0.000000,0.000000,0.000000}%
\pgfsetstrokecolor{currentstroke}%
\pgfsetdash{}{0pt}%
\pgfpathmoveto{\pgfqpoint{1.756522in}{1.323372in}}%
\pgfpathlineto{\pgfqpoint{1.792504in}{1.331902in}}%
\pgfpathlineto{\pgfqpoint{1.755944in}{1.369965in}}%
\pgfpathlineto{\pgfqpoint{1.719861in}{1.361537in}}%
\pgfpathclose%
\pgfusepath{fill}%
\end{pgfscope}%
\begin{pgfscope}%
\pgfpathrectangle{\pgfqpoint{0.150000in}{0.150000in}}{\pgfqpoint{2.700000in}{1.950000in}}%
\pgfusepath{clip}%
\pgfsetbuttcap%
\pgfsetroundjoin%
\definecolor{currentfill}{rgb}{0.822748,0.844577,0.875138}%
\pgfsetfillcolor{currentfill}%
\pgfsetlinewidth{0.000000pt}%
\definecolor{currentstroke}{rgb}{0.000000,0.000000,0.000000}%
\pgfsetstrokecolor{currentstroke}%
\pgfsetdash{}{0pt}%
\pgfpathmoveto{\pgfqpoint{1.719861in}{1.361537in}}%
\pgfpathlineto{\pgfqpoint{1.755944in}{1.369965in}}%
\pgfpathlineto{\pgfqpoint{1.719380in}{1.408032in}}%
\pgfpathlineto{\pgfqpoint{1.683196in}{1.399707in}}%
\pgfpathclose%
\pgfusepath{fill}%
\end{pgfscope}%
\begin{pgfscope}%
\pgfpathrectangle{\pgfqpoint{0.150000in}{0.150000in}}{\pgfqpoint{2.700000in}{1.950000in}}%
\pgfusepath{clip}%
\pgfsetbuttcap%
\pgfsetroundjoin%
\definecolor{currentfill}{rgb}{0.797871,0.822763,0.857613}%
\pgfsetfillcolor{currentfill}%
\pgfsetlinewidth{0.000000pt}%
\definecolor{currentstroke}{rgb}{0.000000,0.000000,0.000000}%
\pgfsetstrokecolor{currentstroke}%
\pgfsetdash{}{0pt}%
\pgfpathmoveto{\pgfqpoint{1.683196in}{1.399707in}}%
\pgfpathlineto{\pgfqpoint{1.719380in}{1.408032in}}%
\pgfpathlineto{\pgfqpoint{1.682811in}{1.446105in}}%
\pgfpathlineto{\pgfqpoint{1.646526in}{1.437882in}}%
\pgfpathclose%
\pgfusepath{fill}%
\end{pgfscope}%
\begin{pgfscope}%
\pgfpathrectangle{\pgfqpoint{0.150000in}{0.150000in}}{\pgfqpoint{2.700000in}{1.950000in}}%
\pgfusepath{clip}%
\pgfsetbuttcap%
\pgfsetroundjoin%
\definecolor{currentfill}{rgb}{0.766774,0.795496,0.835708}%
\pgfsetfillcolor{currentfill}%
\pgfsetlinewidth{0.000000pt}%
\definecolor{currentstroke}{rgb}{0.000000,0.000000,0.000000}%
\pgfsetstrokecolor{currentstroke}%
\pgfsetdash{}{0pt}%
\pgfpathmoveto{\pgfqpoint{1.646526in}{1.437882in}}%
\pgfpathlineto{\pgfqpoint{1.682811in}{1.446105in}}%
\pgfpathlineto{\pgfqpoint{1.646237in}{1.484183in}}%
\pgfpathlineto{\pgfqpoint{1.609851in}{1.476062in}}%
\pgfpathclose%
\pgfusepath{fill}%
\end{pgfscope}%
\begin{pgfscope}%
\pgfpathrectangle{\pgfqpoint{0.150000in}{0.150000in}}{\pgfqpoint{2.700000in}{1.950000in}}%
\pgfusepath{clip}%
\pgfsetbuttcap%
\pgfsetroundjoin%
\definecolor{currentfill}{rgb}{0.735677,0.768229,0.813802}%
\pgfsetfillcolor{currentfill}%
\pgfsetlinewidth{0.000000pt}%
\definecolor{currentstroke}{rgb}{0.000000,0.000000,0.000000}%
\pgfsetstrokecolor{currentstroke}%
\pgfsetdash{}{0pt}%
\pgfpathmoveto{\pgfqpoint{1.609851in}{1.476062in}}%
\pgfpathlineto{\pgfqpoint{1.646237in}{1.484183in}}%
\pgfpathlineto{\pgfqpoint{1.609658in}{1.522266in}}%
\pgfpathlineto{\pgfqpoint{1.573171in}{1.514247in}}%
\pgfpathclose%
\pgfusepath{fill}%
\end{pgfscope}%
\begin{pgfscope}%
\pgfpathrectangle{\pgfqpoint{0.150000in}{0.150000in}}{\pgfqpoint{2.700000in}{1.950000in}}%
\pgfusepath{clip}%
\pgfsetbuttcap%
\pgfsetroundjoin%
\definecolor{currentfill}{rgb}{0.704580,0.740962,0.791896}%
\pgfsetfillcolor{currentfill}%
\pgfsetlinewidth{0.000000pt}%
\definecolor{currentstroke}{rgb}{0.000000,0.000000,0.000000}%
\pgfsetstrokecolor{currentstroke}%
\pgfsetdash{}{0pt}%
\pgfpathmoveto{\pgfqpoint{1.573171in}{1.514247in}}%
\pgfpathlineto{\pgfqpoint{1.609658in}{1.522266in}}%
\pgfpathlineto{\pgfqpoint{1.573075in}{1.560354in}}%
\pgfpathlineto{\pgfqpoint{1.536486in}{1.552438in}}%
\pgfpathclose%
\pgfusepath{fill}%
\end{pgfscope}%
\begin{pgfscope}%
\pgfpathrectangle{\pgfqpoint{0.150000in}{0.150000in}}{\pgfqpoint{2.700000in}{1.950000in}}%
\pgfusepath{clip}%
\pgfsetbuttcap%
\pgfsetroundjoin%
\definecolor{currentfill}{rgb}{0.673483,0.713695,0.769991}%
\pgfsetfillcolor{currentfill}%
\pgfsetlinewidth{0.000000pt}%
\definecolor{currentstroke}{rgb}{0.000000,0.000000,0.000000}%
\pgfsetstrokecolor{currentstroke}%
\pgfsetdash{}{0pt}%
\pgfpathmoveto{\pgfqpoint{1.536486in}{1.552438in}}%
\pgfpathlineto{\pgfqpoint{1.573075in}{1.560354in}}%
\pgfpathlineto{\pgfqpoint{1.536486in}{1.598447in}}%
\pgfpathlineto{\pgfqpoint{1.499797in}{1.590633in}}%
\pgfpathclose%
\pgfusepath{fill}%
\end{pgfscope}%
\begin{pgfscope}%
\pgfpathrectangle{\pgfqpoint{0.150000in}{0.150000in}}{\pgfqpoint{2.700000in}{1.950000in}}%
\pgfusepath{clip}%
\pgfsetbuttcap%
\pgfsetroundjoin%
\definecolor{currentfill}{rgb}{0.642387,0.686428,0.748085}%
\pgfsetfillcolor{currentfill}%
\pgfsetlinewidth{0.000000pt}%
\definecolor{currentstroke}{rgb}{0.000000,0.000000,0.000000}%
\pgfsetstrokecolor{currentstroke}%
\pgfsetdash{}{0pt}%
\pgfpathmoveto{\pgfqpoint{1.499797in}{1.590633in}}%
\pgfpathlineto{\pgfqpoint{1.536486in}{1.598447in}}%
\pgfpathlineto{\pgfqpoint{1.499893in}{1.636545in}}%
\pgfpathlineto{\pgfqpoint{1.463103in}{1.628833in}}%
\pgfpathclose%
\pgfusepath{fill}%
\end{pgfscope}%
\begin{pgfscope}%
\pgfpathrectangle{\pgfqpoint{0.150000in}{0.150000in}}{\pgfqpoint{2.700000in}{1.950000in}}%
\pgfusepath{clip}%
\pgfsetbuttcap%
\pgfsetroundjoin%
\definecolor{currentfill}{rgb}{0.611290,0.659161,0.726180}%
\pgfsetfillcolor{currentfill}%
\pgfsetlinewidth{0.000000pt}%
\definecolor{currentstroke}{rgb}{0.000000,0.000000,0.000000}%
\pgfsetstrokecolor{currentstroke}%
\pgfsetdash{}{0pt}%
\pgfpathmoveto{\pgfqpoint{1.463103in}{1.628833in}}%
\pgfpathlineto{\pgfqpoint{1.499893in}{1.636545in}}%
\pgfpathlineto{\pgfqpoint{1.463295in}{1.674648in}}%
\pgfpathlineto{\pgfqpoint{1.426403in}{1.667038in}}%
\pgfpathclose%
\pgfusepath{fill}%
\end{pgfscope}%
\begin{pgfscope}%
\pgfpathrectangle{\pgfqpoint{0.150000in}{0.150000in}}{\pgfqpoint{2.700000in}{1.950000in}}%
\pgfusepath{clip}%
\pgfsetbuttcap%
\pgfsetroundjoin%
\definecolor{currentfill}{rgb}{0.580193,0.631893,0.704274}%
\pgfsetfillcolor{currentfill}%
\pgfsetlinewidth{0.000000pt}%
\definecolor{currentstroke}{rgb}{0.000000,0.000000,0.000000}%
\pgfsetstrokecolor{currentstroke}%
\pgfsetdash{}{0pt}%
\pgfpathmoveto{\pgfqpoint{1.426403in}{1.667038in}}%
\pgfpathlineto{\pgfqpoint{1.463295in}{1.674648in}}%
\pgfpathlineto{\pgfqpoint{1.426693in}{1.712756in}}%
\pgfpathlineto{\pgfqpoint{1.389699in}{1.705249in}}%
\pgfpathclose%
\pgfusepath{fill}%
\end{pgfscope}%
\begin{pgfscope}%
\pgfpathrectangle{\pgfqpoint{0.150000in}{0.150000in}}{\pgfqpoint{2.700000in}{1.950000in}}%
\pgfusepath{clip}%
\pgfsetbuttcap%
\pgfsetroundjoin%
\definecolor{currentfill}{rgb}{0.819547,0.672564,0.684206}%
\pgfsetfillcolor{currentfill}%
\pgfsetlinewidth{0.000000pt}%
\definecolor{currentstroke}{rgb}{0.000000,0.000000,0.000000}%
\pgfsetstrokecolor{currentstroke}%
\pgfsetdash{}{0pt}%
\pgfpathmoveto{\pgfqpoint{2.271194in}{0.741387in}}%
\pgfpathlineto{\pgfqpoint{2.305850in}{0.751500in}}%
\pgfpathlineto{\pgfqpoint{2.269262in}{0.789590in}}%
\pgfpathlineto{\pgfqpoint{2.234505in}{0.779579in}}%
\pgfpathclose%
\pgfusepath{fill}%
\end{pgfscope}%
\begin{pgfscope}%
\pgfpathrectangle{\pgfqpoint{0.150000in}{0.150000in}}{\pgfqpoint{2.700000in}{1.950000in}}%
\pgfusepath{clip}%
\pgfsetbuttcap%
\pgfsetroundjoin%
\definecolor{currentfill}{rgb}{0.838542,0.707031,0.717448}%
\pgfsetfillcolor{currentfill}%
\pgfsetlinewidth{0.000000pt}%
\definecolor{currentstroke}{rgb}{0.000000,0.000000,0.000000}%
\pgfsetstrokecolor{currentstroke}%
\pgfsetdash{}{0pt}%
\pgfpathmoveto{\pgfqpoint{2.234505in}{0.779579in}}%
\pgfpathlineto{\pgfqpoint{2.269262in}{0.789590in}}%
\pgfpathlineto{\pgfqpoint{2.232669in}{0.827685in}}%
\pgfpathlineto{\pgfqpoint{2.197811in}{0.817776in}}%
\pgfpathclose%
\pgfusepath{fill}%
\end{pgfscope}%
\begin{pgfscope}%
\pgfpathrectangle{\pgfqpoint{0.150000in}{0.150000in}}{\pgfqpoint{2.700000in}{1.950000in}}%
\pgfusepath{clip}%
\pgfsetbuttcap%
\pgfsetroundjoin%
\definecolor{currentfill}{rgb}{0.857537,0.741498,0.750689}%
\pgfsetfillcolor{currentfill}%
\pgfsetlinewidth{0.000000pt}%
\definecolor{currentstroke}{rgb}{0.000000,0.000000,0.000000}%
\pgfsetstrokecolor{currentstroke}%
\pgfsetdash{}{0pt}%
\pgfpathmoveto{\pgfqpoint{2.197811in}{0.817776in}}%
\pgfpathlineto{\pgfqpoint{2.232669in}{0.827685in}}%
\pgfpathlineto{\pgfqpoint{2.196071in}{0.865784in}}%
\pgfpathlineto{\pgfqpoint{2.161112in}{0.855978in}}%
\pgfpathclose%
\pgfusepath{fill}%
\end{pgfscope}%
\begin{pgfscope}%
\pgfpathrectangle{\pgfqpoint{0.150000in}{0.150000in}}{\pgfqpoint{2.700000in}{1.950000in}}%
\pgfusepath{clip}%
\pgfsetbuttcap%
\pgfsetroundjoin%
\definecolor{currentfill}{rgb}{0.876532,0.775965,0.783931}%
\pgfsetfillcolor{currentfill}%
\pgfsetlinewidth{0.000000pt}%
\definecolor{currentstroke}{rgb}{0.000000,0.000000,0.000000}%
\pgfsetstrokecolor{currentstroke}%
\pgfsetdash{}{0pt}%
\pgfpathmoveto{\pgfqpoint{2.161112in}{0.855978in}}%
\pgfpathlineto{\pgfqpoint{2.196071in}{0.865784in}}%
\pgfpathlineto{\pgfqpoint{2.159469in}{0.903889in}}%
\pgfpathlineto{\pgfqpoint{2.124408in}{0.894185in}}%
\pgfpathclose%
\pgfusepath{fill}%
\end{pgfscope}%
\begin{pgfscope}%
\pgfpathrectangle{\pgfqpoint{0.150000in}{0.150000in}}{\pgfqpoint{2.700000in}{1.950000in}}%
\pgfusepath{clip}%
\pgfsetbuttcap%
\pgfsetroundjoin%
\definecolor{currentfill}{rgb}{0.891728,0.803539,0.810524}%
\pgfsetfillcolor{currentfill}%
\pgfsetlinewidth{0.000000pt}%
\definecolor{currentstroke}{rgb}{0.000000,0.000000,0.000000}%
\pgfsetstrokecolor{currentstroke}%
\pgfsetdash{}{0pt}%
\pgfpathmoveto{\pgfqpoint{2.124408in}{0.894185in}}%
\pgfpathlineto{\pgfqpoint{2.159469in}{0.903889in}}%
\pgfpathlineto{\pgfqpoint{2.122861in}{0.941999in}}%
\pgfpathlineto{\pgfqpoint{2.087699in}{0.932397in}}%
\pgfpathclose%
\pgfusepath{fill}%
\end{pgfscope}%
\begin{pgfscope}%
\pgfpathrectangle{\pgfqpoint{0.150000in}{0.150000in}}{\pgfqpoint{2.700000in}{1.950000in}}%
\pgfusepath{clip}%
\pgfsetbuttcap%
\pgfsetroundjoin%
\definecolor{currentfill}{rgb}{0.910723,0.838006,0.843765}%
\pgfsetfillcolor{currentfill}%
\pgfsetlinewidth{0.000000pt}%
\definecolor{currentstroke}{rgb}{0.000000,0.000000,0.000000}%
\pgfsetstrokecolor{currentstroke}%
\pgfsetdash{}{0pt}%
\pgfpathmoveto{\pgfqpoint{2.087699in}{0.932397in}}%
\pgfpathlineto{\pgfqpoint{2.122861in}{0.941999in}}%
\pgfpathlineto{\pgfqpoint{2.086249in}{0.980113in}}%
\pgfpathlineto{\pgfqpoint{2.050986in}{0.970615in}}%
\pgfpathclose%
\pgfusepath{fill}%
\end{pgfscope}%
\begin{pgfscope}%
\pgfpathrectangle{\pgfqpoint{0.150000in}{0.150000in}}{\pgfqpoint{2.700000in}{1.950000in}}%
\pgfusepath{clip}%
\pgfsetbuttcap%
\pgfsetroundjoin%
\definecolor{currentfill}{rgb}{0.929718,0.872472,0.877007}%
\pgfsetfillcolor{currentfill}%
\pgfsetlinewidth{0.000000pt}%
\definecolor{currentstroke}{rgb}{0.000000,0.000000,0.000000}%
\pgfsetstrokecolor{currentstroke}%
\pgfsetdash{}{0pt}%
\pgfpathmoveto{\pgfqpoint{2.050986in}{0.970615in}}%
\pgfpathlineto{\pgfqpoint{2.086249in}{0.980113in}}%
\pgfpathlineto{\pgfqpoint{2.049632in}{1.018233in}}%
\pgfpathlineto{\pgfqpoint{2.014267in}{1.008837in}}%
\pgfpathclose%
\pgfusepath{fill}%
\end{pgfscope}%
\begin{pgfscope}%
\pgfpathrectangle{\pgfqpoint{0.150000in}{0.150000in}}{\pgfqpoint{2.700000in}{1.950000in}}%
\pgfusepath{clip}%
\pgfsetbuttcap%
\pgfsetroundjoin%
\definecolor{currentfill}{rgb}{0.948713,0.906939,0.910248}%
\pgfsetfillcolor{currentfill}%
\pgfsetlinewidth{0.000000pt}%
\definecolor{currentstroke}{rgb}{0.000000,0.000000,0.000000}%
\pgfsetstrokecolor{currentstroke}%
\pgfsetdash{}{0pt}%
\pgfpathmoveto{\pgfqpoint{2.014267in}{1.008837in}}%
\pgfpathlineto{\pgfqpoint{2.049632in}{1.018233in}}%
\pgfpathlineto{\pgfqpoint{2.013010in}{1.056358in}}%
\pgfpathlineto{\pgfqpoint{1.977544in}{1.047064in}}%
\pgfpathclose%
\pgfusepath{fill}%
\end{pgfscope}%
\begin{pgfscope}%
\pgfpathrectangle{\pgfqpoint{0.150000in}{0.150000in}}{\pgfqpoint{2.700000in}{1.950000in}}%
\pgfusepath{clip}%
\pgfsetbuttcap%
\pgfsetroundjoin%
\definecolor{currentfill}{rgb}{0.967708,0.941406,0.943490}%
\pgfsetfillcolor{currentfill}%
\pgfsetlinewidth{0.000000pt}%
\definecolor{currentstroke}{rgb}{0.000000,0.000000,0.000000}%
\pgfsetstrokecolor{currentstroke}%
\pgfsetdash{}{0pt}%
\pgfpathmoveto{\pgfqpoint{1.977544in}{1.047064in}}%
\pgfpathlineto{\pgfqpoint{2.013010in}{1.056358in}}%
\pgfpathlineto{\pgfqpoint{1.976383in}{1.094488in}}%
\pgfpathlineto{\pgfqpoint{1.940816in}{1.085296in}}%
\pgfpathclose%
\pgfusepath{fill}%
\end{pgfscope}%
\begin{pgfscope}%
\pgfpathrectangle{\pgfqpoint{0.150000in}{0.150000in}}{\pgfqpoint{2.700000in}{1.950000in}}%
\pgfusepath{clip}%
\pgfsetbuttcap%
\pgfsetroundjoin%
\definecolor{currentfill}{rgb}{0.986703,0.975873,0.976731}%
\pgfsetfillcolor{currentfill}%
\pgfsetlinewidth{0.000000pt}%
\definecolor{currentstroke}{rgb}{0.000000,0.000000,0.000000}%
\pgfsetstrokecolor{currentstroke}%
\pgfsetdash{}{0pt}%
\pgfpathmoveto{\pgfqpoint{1.940816in}{1.085296in}}%
\pgfpathlineto{\pgfqpoint{1.976383in}{1.094488in}}%
\pgfpathlineto{\pgfqpoint{1.939752in}{1.132623in}}%
\pgfpathlineto{\pgfqpoint{1.904083in}{1.123534in}}%
\pgfpathclose%
\pgfusepath{fill}%
\end{pgfscope}%
\begin{pgfscope}%
\pgfpathrectangle{\pgfqpoint{0.150000in}{0.150000in}}{\pgfqpoint{2.700000in}{1.950000in}}%
\pgfusepath{clip}%
\pgfsetbuttcap%
\pgfsetroundjoin%
\definecolor{currentfill}{rgb}{0.990671,0.991820,0.993428}%
\pgfsetfillcolor{currentfill}%
\pgfsetlinewidth{0.000000pt}%
\definecolor{currentstroke}{rgb}{0.000000,0.000000,0.000000}%
\pgfsetstrokecolor{currentstroke}%
\pgfsetdash{}{0pt}%
\pgfpathmoveto{\pgfqpoint{1.904083in}{1.123534in}}%
\pgfpathlineto{\pgfqpoint{1.939752in}{1.132623in}}%
\pgfpathlineto{\pgfqpoint{1.903116in}{1.170762in}}%
\pgfpathlineto{\pgfqpoint{1.867345in}{1.161776in}}%
\pgfpathclose%
\pgfusepath{fill}%
\end{pgfscope}%
\begin{pgfscope}%
\pgfpathrectangle{\pgfqpoint{0.150000in}{0.150000in}}{\pgfqpoint{2.700000in}{1.950000in}}%
\pgfusepath{clip}%
\pgfsetbuttcap%
\pgfsetroundjoin%
\definecolor{currentfill}{rgb}{0.959574,0.964553,0.971523}%
\pgfsetfillcolor{currentfill}%
\pgfsetlinewidth{0.000000pt}%
\definecolor{currentstroke}{rgb}{0.000000,0.000000,0.000000}%
\pgfsetstrokecolor{currentstroke}%
\pgfsetdash{}{0pt}%
\pgfpathmoveto{\pgfqpoint{1.867345in}{1.161776in}}%
\pgfpathlineto{\pgfqpoint{1.903116in}{1.170762in}}%
\pgfpathlineto{\pgfqpoint{1.866474in}{1.208907in}}%
\pgfpathlineto{\pgfqpoint{1.830603in}{1.200024in}}%
\pgfpathclose%
\pgfusepath{fill}%
\end{pgfscope}%
\begin{pgfscope}%
\pgfpathrectangle{\pgfqpoint{0.150000in}{0.150000in}}{\pgfqpoint{2.700000in}{1.950000in}}%
\pgfusepath{clip}%
\pgfsetbuttcap%
\pgfsetroundjoin%
\definecolor{currentfill}{rgb}{0.934697,0.942739,0.953998}%
\pgfsetfillcolor{currentfill}%
\pgfsetlinewidth{0.000000pt}%
\definecolor{currentstroke}{rgb}{0.000000,0.000000,0.000000}%
\pgfsetstrokecolor{currentstroke}%
\pgfsetdash{}{0pt}%
\pgfpathmoveto{\pgfqpoint{1.830603in}{1.200024in}}%
\pgfpathlineto{\pgfqpoint{1.866474in}{1.208907in}}%
\pgfpathlineto{\pgfqpoint{1.829828in}{1.247057in}}%
\pgfpathlineto{\pgfqpoint{1.793855in}{1.238276in}}%
\pgfpathclose%
\pgfusepath{fill}%
\end{pgfscope}%
\begin{pgfscope}%
\pgfpathrectangle{\pgfqpoint{0.150000in}{0.150000in}}{\pgfqpoint{2.700000in}{1.950000in}}%
\pgfusepath{clip}%
\pgfsetbuttcap%
\pgfsetroundjoin%
\definecolor{currentfill}{rgb}{0.903600,0.915472,0.932093}%
\pgfsetfillcolor{currentfill}%
\pgfsetlinewidth{0.000000pt}%
\definecolor{currentstroke}{rgb}{0.000000,0.000000,0.000000}%
\pgfsetstrokecolor{currentstroke}%
\pgfsetdash{}{0pt}%
\pgfpathmoveto{\pgfqpoint{1.793855in}{1.238276in}}%
\pgfpathlineto{\pgfqpoint{1.829828in}{1.247057in}}%
\pgfpathlineto{\pgfqpoint{1.793178in}{1.285212in}}%
\pgfpathlineto{\pgfqpoint{1.757103in}{1.276534in}}%
\pgfpathclose%
\pgfusepath{fill}%
\end{pgfscope}%
\begin{pgfscope}%
\pgfpathrectangle{\pgfqpoint{0.150000in}{0.150000in}}{\pgfqpoint{2.700000in}{1.950000in}}%
\pgfusepath{clip}%
\pgfsetbuttcap%
\pgfsetroundjoin%
\definecolor{currentfill}{rgb}{0.872503,0.888205,0.910187}%
\pgfsetfillcolor{currentfill}%
\pgfsetlinewidth{0.000000pt}%
\definecolor{currentstroke}{rgb}{0.000000,0.000000,0.000000}%
\pgfsetstrokecolor{currentstroke}%
\pgfsetdash{}{0pt}%
\pgfpathmoveto{\pgfqpoint{1.757103in}{1.276534in}}%
\pgfpathlineto{\pgfqpoint{1.793178in}{1.285212in}}%
\pgfpathlineto{\pgfqpoint{1.756522in}{1.323372in}}%
\pgfpathlineto{\pgfqpoint{1.720345in}{1.314796in}}%
\pgfpathclose%
\pgfusepath{fill}%
\end{pgfscope}%
\begin{pgfscope}%
\pgfpathrectangle{\pgfqpoint{0.150000in}{0.150000in}}{\pgfqpoint{2.700000in}{1.950000in}}%
\pgfusepath{clip}%
\pgfsetbuttcap%
\pgfsetroundjoin%
\definecolor{currentfill}{rgb}{0.841406,0.860938,0.888281}%
\pgfsetfillcolor{currentfill}%
\pgfsetlinewidth{0.000000pt}%
\definecolor{currentstroke}{rgb}{0.000000,0.000000,0.000000}%
\pgfsetstrokecolor{currentstroke}%
\pgfsetdash{}{0pt}%
\pgfpathmoveto{\pgfqpoint{1.720345in}{1.314796in}}%
\pgfpathlineto{\pgfqpoint{1.756522in}{1.323372in}}%
\pgfpathlineto{\pgfqpoint{1.719861in}{1.361537in}}%
\pgfpathlineto{\pgfqpoint{1.683583in}{1.353064in}}%
\pgfpathclose%
\pgfusepath{fill}%
\end{pgfscope}%
\begin{pgfscope}%
\pgfpathrectangle{\pgfqpoint{0.150000in}{0.150000in}}{\pgfqpoint{2.700000in}{1.950000in}}%
\pgfusepath{clip}%
\pgfsetbuttcap%
\pgfsetroundjoin%
\definecolor{currentfill}{rgb}{0.810309,0.833670,0.866376}%
\pgfsetfillcolor{currentfill}%
\pgfsetlinewidth{0.000000pt}%
\definecolor{currentstroke}{rgb}{0.000000,0.000000,0.000000}%
\pgfsetstrokecolor{currentstroke}%
\pgfsetdash{}{0pt}%
\pgfpathmoveto{\pgfqpoint{1.683583in}{1.353064in}}%
\pgfpathlineto{\pgfqpoint{1.719861in}{1.361537in}}%
\pgfpathlineto{\pgfqpoint{1.683196in}{1.399707in}}%
\pgfpathlineto{\pgfqpoint{1.646816in}{1.391337in}}%
\pgfpathclose%
\pgfusepath{fill}%
\end{pgfscope}%
\begin{pgfscope}%
\pgfpathrectangle{\pgfqpoint{0.150000in}{0.150000in}}{\pgfqpoint{2.700000in}{1.950000in}}%
\pgfusepath{clip}%
\pgfsetbuttcap%
\pgfsetroundjoin%
\definecolor{currentfill}{rgb}{0.779213,0.806403,0.844470}%
\pgfsetfillcolor{currentfill}%
\pgfsetlinewidth{0.000000pt}%
\definecolor{currentstroke}{rgb}{0.000000,0.000000,0.000000}%
\pgfsetstrokecolor{currentstroke}%
\pgfsetdash{}{0pt}%
\pgfpathmoveto{\pgfqpoint{1.646816in}{1.391337in}}%
\pgfpathlineto{\pgfqpoint{1.683196in}{1.399707in}}%
\pgfpathlineto{\pgfqpoint{1.646526in}{1.437882in}}%
\pgfpathlineto{\pgfqpoint{1.610045in}{1.429615in}}%
\pgfpathclose%
\pgfusepath{fill}%
\end{pgfscope}%
\begin{pgfscope}%
\pgfpathrectangle{\pgfqpoint{0.150000in}{0.150000in}}{\pgfqpoint{2.700000in}{1.950000in}}%
\pgfusepath{clip}%
\pgfsetbuttcap%
\pgfsetroundjoin%
\definecolor{currentfill}{rgb}{0.748116,0.779136,0.822564}%
\pgfsetfillcolor{currentfill}%
\pgfsetlinewidth{0.000000pt}%
\definecolor{currentstroke}{rgb}{0.000000,0.000000,0.000000}%
\pgfsetstrokecolor{currentstroke}%
\pgfsetdash{}{0pt}%
\pgfpathmoveto{\pgfqpoint{1.610045in}{1.429615in}}%
\pgfpathlineto{\pgfqpoint{1.646526in}{1.437882in}}%
\pgfpathlineto{\pgfqpoint{1.609851in}{1.476062in}}%
\pgfpathlineto{\pgfqpoint{1.573268in}{1.467897in}}%
\pgfpathclose%
\pgfusepath{fill}%
\end{pgfscope}%
\begin{pgfscope}%
\pgfpathrectangle{\pgfqpoint{0.150000in}{0.150000in}}{\pgfqpoint{2.700000in}{1.950000in}}%
\pgfusepath{clip}%
\pgfsetbuttcap%
\pgfsetroundjoin%
\definecolor{currentfill}{rgb}{0.717019,0.751869,0.800659}%
\pgfsetfillcolor{currentfill}%
\pgfsetlinewidth{0.000000pt}%
\definecolor{currentstroke}{rgb}{0.000000,0.000000,0.000000}%
\pgfsetstrokecolor{currentstroke}%
\pgfsetdash{}{0pt}%
\pgfpathmoveto{\pgfqpoint{1.573268in}{1.467897in}}%
\pgfpathlineto{\pgfqpoint{1.609851in}{1.476062in}}%
\pgfpathlineto{\pgfqpoint{1.573171in}{1.514247in}}%
\pgfpathlineto{\pgfqpoint{1.536486in}{1.506185in}}%
\pgfpathclose%
\pgfusepath{fill}%
\end{pgfscope}%
\begin{pgfscope}%
\pgfpathrectangle{\pgfqpoint{0.150000in}{0.150000in}}{\pgfqpoint{2.700000in}{1.950000in}}%
\pgfusepath{clip}%
\pgfsetbuttcap%
\pgfsetroundjoin%
\definecolor{currentfill}{rgb}{0.692142,0.730055,0.783134}%
\pgfsetfillcolor{currentfill}%
\pgfsetlinewidth{0.000000pt}%
\definecolor{currentstroke}{rgb}{0.000000,0.000000,0.000000}%
\pgfsetstrokecolor{currentstroke}%
\pgfsetdash{}{0pt}%
\pgfpathmoveto{\pgfqpoint{1.536486in}{1.506185in}}%
\pgfpathlineto{\pgfqpoint{1.573171in}{1.514247in}}%
\pgfpathlineto{\pgfqpoint{1.536486in}{1.552438in}}%
\pgfpathlineto{\pgfqpoint{1.499700in}{1.544478in}}%
\pgfpathclose%
\pgfusepath{fill}%
\end{pgfscope}%
\begin{pgfscope}%
\pgfpathrectangle{\pgfqpoint{0.150000in}{0.150000in}}{\pgfqpoint{2.700000in}{1.950000in}}%
\pgfusepath{clip}%
\pgfsetbuttcap%
\pgfsetroundjoin%
\definecolor{currentfill}{rgb}{0.661045,0.702788,0.761229}%
\pgfsetfillcolor{currentfill}%
\pgfsetlinewidth{0.000000pt}%
\definecolor{currentstroke}{rgb}{0.000000,0.000000,0.000000}%
\pgfsetstrokecolor{currentstroke}%
\pgfsetdash{}{0pt}%
\pgfpathmoveto{\pgfqpoint{1.499700in}{1.544478in}}%
\pgfpathlineto{\pgfqpoint{1.536486in}{1.552438in}}%
\pgfpathlineto{\pgfqpoint{1.499797in}{1.590633in}}%
\pgfpathlineto{\pgfqpoint{1.462909in}{1.582776in}}%
\pgfpathclose%
\pgfusepath{fill}%
\end{pgfscope}%
\begin{pgfscope}%
\pgfpathrectangle{\pgfqpoint{0.150000in}{0.150000in}}{\pgfqpoint{2.700000in}{1.950000in}}%
\pgfusepath{clip}%
\pgfsetbuttcap%
\pgfsetroundjoin%
\definecolor{currentfill}{rgb}{0.629948,0.675521,0.739323}%
\pgfsetfillcolor{currentfill}%
\pgfsetlinewidth{0.000000pt}%
\definecolor{currentstroke}{rgb}{0.000000,0.000000,0.000000}%
\pgfsetstrokecolor{currentstroke}%
\pgfsetdash{}{0pt}%
\pgfpathmoveto{\pgfqpoint{1.462909in}{1.582776in}}%
\pgfpathlineto{\pgfqpoint{1.499797in}{1.590633in}}%
\pgfpathlineto{\pgfqpoint{1.463103in}{1.628833in}}%
\pgfpathlineto{\pgfqpoint{1.426113in}{1.621079in}}%
\pgfpathclose%
\pgfusepath{fill}%
\end{pgfscope}%
\begin{pgfscope}%
\pgfpathrectangle{\pgfqpoint{0.150000in}{0.150000in}}{\pgfqpoint{2.700000in}{1.950000in}}%
\pgfusepath{clip}%
\pgfsetbuttcap%
\pgfsetroundjoin%
\definecolor{currentfill}{rgb}{0.598851,0.648254,0.717417}%
\pgfsetfillcolor{currentfill}%
\pgfsetlinewidth{0.000000pt}%
\definecolor{currentstroke}{rgb}{0.000000,0.000000,0.000000}%
\pgfsetstrokecolor{currentstroke}%
\pgfsetdash{}{0pt}%
\pgfpathmoveto{\pgfqpoint{1.426113in}{1.621079in}}%
\pgfpathlineto{\pgfqpoint{1.463103in}{1.628833in}}%
\pgfpathlineto{\pgfqpoint{1.426403in}{1.667038in}}%
\pgfpathlineto{\pgfqpoint{1.389312in}{1.659388in}}%
\pgfpathclose%
\pgfusepath{fill}%
\end{pgfscope}%
\begin{pgfscope}%
\pgfpathrectangle{\pgfqpoint{0.150000in}{0.150000in}}{\pgfqpoint{2.700000in}{1.950000in}}%
\pgfusepath{clip}%
\pgfsetbuttcap%
\pgfsetroundjoin%
\definecolor{currentfill}{rgb}{0.567754,0.620987,0.695512}%
\pgfsetfillcolor{currentfill}%
\pgfsetlinewidth{0.000000pt}%
\definecolor{currentstroke}{rgb}{0.000000,0.000000,0.000000}%
\pgfsetstrokecolor{currentstroke}%
\pgfsetdash{}{0pt}%
\pgfpathmoveto{\pgfqpoint{1.389312in}{1.659388in}}%
\pgfpathlineto{\pgfqpoint{1.426403in}{1.667038in}}%
\pgfpathlineto{\pgfqpoint{1.389699in}{1.705249in}}%
\pgfpathlineto{\pgfqpoint{1.352506in}{1.697701in}}%
\pgfpathclose%
\pgfusepath{fill}%
\end{pgfscope}%
\begin{pgfscope}%
\pgfpathrectangle{\pgfqpoint{0.150000in}{0.150000in}}{\pgfqpoint{2.700000in}{1.950000in}}%
\pgfusepath{clip}%
\pgfsetbuttcap%
\pgfsetroundjoin%
\definecolor{currentfill}{rgb}{0.827145,0.686351,0.697503}%
\pgfsetfillcolor{currentfill}%
\pgfsetlinewidth{0.000000pt}%
\definecolor{currentstroke}{rgb}{0.000000,0.000000,0.000000}%
\pgfsetstrokecolor{currentstroke}%
\pgfsetdash{}{0pt}%
\pgfpathmoveto{\pgfqpoint{2.236350in}{0.731220in}}%
\pgfpathlineto{\pgfqpoint{2.271194in}{0.741387in}}%
\pgfpathlineto{\pgfqpoint{2.234505in}{0.779579in}}%
\pgfpathlineto{\pgfqpoint{2.199559in}{0.769514in}}%
\pgfpathclose%
\pgfusepath{fill}%
\end{pgfscope}%
\begin{pgfscope}%
\pgfpathrectangle{\pgfqpoint{0.150000in}{0.150000in}}{\pgfqpoint{2.700000in}{1.950000in}}%
\pgfusepath{clip}%
\pgfsetbuttcap%
\pgfsetroundjoin%
\definecolor{currentfill}{rgb}{0.846140,0.720818,0.730744}%
\pgfsetfillcolor{currentfill}%
\pgfsetlinewidth{0.000000pt}%
\definecolor{currentstroke}{rgb}{0.000000,0.000000,0.000000}%
\pgfsetstrokecolor{currentstroke}%
\pgfsetdash{}{0pt}%
\pgfpathmoveto{\pgfqpoint{2.199559in}{0.769514in}}%
\pgfpathlineto{\pgfqpoint{2.234505in}{0.779579in}}%
\pgfpathlineto{\pgfqpoint{2.197811in}{0.817776in}}%
\pgfpathlineto{\pgfqpoint{2.162763in}{0.807814in}}%
\pgfpathclose%
\pgfusepath{fill}%
\end{pgfscope}%
\begin{pgfscope}%
\pgfpathrectangle{\pgfqpoint{0.150000in}{0.150000in}}{\pgfqpoint{2.700000in}{1.950000in}}%
\pgfusepath{clip}%
\pgfsetbuttcap%
\pgfsetroundjoin%
\definecolor{currentfill}{rgb}{0.865135,0.755285,0.763986}%
\pgfsetfillcolor{currentfill}%
\pgfsetlinewidth{0.000000pt}%
\definecolor{currentstroke}{rgb}{0.000000,0.000000,0.000000}%
\pgfsetstrokecolor{currentstroke}%
\pgfsetdash{}{0pt}%
\pgfpathmoveto{\pgfqpoint{2.162763in}{0.807814in}}%
\pgfpathlineto{\pgfqpoint{2.197811in}{0.817776in}}%
\pgfpathlineto{\pgfqpoint{2.161112in}{0.855978in}}%
\pgfpathlineto{\pgfqpoint{2.125963in}{0.846119in}}%
\pgfpathclose%
\pgfusepath{fill}%
\end{pgfscope}%
\begin{pgfscope}%
\pgfpathrectangle{\pgfqpoint{0.150000in}{0.150000in}}{\pgfqpoint{2.700000in}{1.950000in}}%
\pgfusepath{clip}%
\pgfsetbuttcap%
\pgfsetroundjoin%
\definecolor{currentfill}{rgb}{0.884130,0.789752,0.797227}%
\pgfsetfillcolor{currentfill}%
\pgfsetlinewidth{0.000000pt}%
\definecolor{currentstroke}{rgb}{0.000000,0.000000,0.000000}%
\pgfsetstrokecolor{currentstroke}%
\pgfsetdash{}{0pt}%
\pgfpathmoveto{\pgfqpoint{2.125963in}{0.846119in}}%
\pgfpathlineto{\pgfqpoint{2.161112in}{0.855978in}}%
\pgfpathlineto{\pgfqpoint{2.124408in}{0.894185in}}%
\pgfpathlineto{\pgfqpoint{2.089157in}{0.884429in}}%
\pgfpathclose%
\pgfusepath{fill}%
\end{pgfscope}%
\begin{pgfscope}%
\pgfpathrectangle{\pgfqpoint{0.150000in}{0.150000in}}{\pgfqpoint{2.700000in}{1.950000in}}%
\pgfusepath{clip}%
\pgfsetbuttcap%
\pgfsetroundjoin%
\definecolor{currentfill}{rgb}{0.903125,0.824219,0.830469}%
\pgfsetfillcolor{currentfill}%
\pgfsetlinewidth{0.000000pt}%
\definecolor{currentstroke}{rgb}{0.000000,0.000000,0.000000}%
\pgfsetstrokecolor{currentstroke}%
\pgfsetdash{}{0pt}%
\pgfpathmoveto{\pgfqpoint{2.089157in}{0.884429in}}%
\pgfpathlineto{\pgfqpoint{2.124408in}{0.894185in}}%
\pgfpathlineto{\pgfqpoint{2.087699in}{0.932397in}}%
\pgfpathlineto{\pgfqpoint{2.052346in}{0.922744in}}%
\pgfpathclose%
\pgfusepath{fill}%
\end{pgfscope}%
\begin{pgfscope}%
\pgfpathrectangle{\pgfqpoint{0.150000in}{0.150000in}}{\pgfqpoint{2.700000in}{1.950000in}}%
\pgfusepath{clip}%
\pgfsetbuttcap%
\pgfsetroundjoin%
\definecolor{currentfill}{rgb}{0.922120,0.858686,0.863710}%
\pgfsetfillcolor{currentfill}%
\pgfsetlinewidth{0.000000pt}%
\definecolor{currentstroke}{rgb}{0.000000,0.000000,0.000000}%
\pgfsetstrokecolor{currentstroke}%
\pgfsetdash{}{0pt}%
\pgfpathmoveto{\pgfqpoint{2.052346in}{0.922744in}}%
\pgfpathlineto{\pgfqpoint{2.087699in}{0.932397in}}%
\pgfpathlineto{\pgfqpoint{2.050986in}{0.970615in}}%
\pgfpathlineto{\pgfqpoint{2.015531in}{0.961064in}}%
\pgfpathclose%
\pgfusepath{fill}%
\end{pgfscope}%
\begin{pgfscope}%
\pgfpathrectangle{\pgfqpoint{0.150000in}{0.150000in}}{\pgfqpoint{2.700000in}{1.950000in}}%
\pgfusepath{clip}%
\pgfsetbuttcap%
\pgfsetroundjoin%
\definecolor{currentfill}{rgb}{0.941115,0.893153,0.896952}%
\pgfsetfillcolor{currentfill}%
\pgfsetlinewidth{0.000000pt}%
\definecolor{currentstroke}{rgb}{0.000000,0.000000,0.000000}%
\pgfsetstrokecolor{currentstroke}%
\pgfsetdash{}{0pt}%
\pgfpathmoveto{\pgfqpoint{2.015531in}{0.961064in}}%
\pgfpathlineto{\pgfqpoint{2.050986in}{0.970615in}}%
\pgfpathlineto{\pgfqpoint{2.014267in}{1.008837in}}%
\pgfpathlineto{\pgfqpoint{1.978711in}{0.999389in}}%
\pgfpathclose%
\pgfusepath{fill}%
\end{pgfscope}%
\begin{pgfscope}%
\pgfpathrectangle{\pgfqpoint{0.150000in}{0.150000in}}{\pgfqpoint{2.700000in}{1.950000in}}%
\pgfusepath{clip}%
\pgfsetbuttcap%
\pgfsetroundjoin%
\definecolor{currentfill}{rgb}{0.960110,0.927619,0.930193}%
\pgfsetfillcolor{currentfill}%
\pgfsetlinewidth{0.000000pt}%
\definecolor{currentstroke}{rgb}{0.000000,0.000000,0.000000}%
\pgfsetstrokecolor{currentstroke}%
\pgfsetdash{}{0pt}%
\pgfpathmoveto{\pgfqpoint{1.978711in}{0.999389in}}%
\pgfpathlineto{\pgfqpoint{2.014267in}{1.008837in}}%
\pgfpathlineto{\pgfqpoint{1.977544in}{1.047064in}}%
\pgfpathlineto{\pgfqpoint{1.941885in}{1.037720in}}%
\pgfpathclose%
\pgfusepath{fill}%
\end{pgfscope}%
\begin{pgfscope}%
\pgfpathrectangle{\pgfqpoint{0.150000in}{0.150000in}}{\pgfqpoint{2.700000in}{1.950000in}}%
\pgfusepath{clip}%
\pgfsetbuttcap%
\pgfsetroundjoin%
\definecolor{currentfill}{rgb}{0.975306,0.955193,0.956786}%
\pgfsetfillcolor{currentfill}%
\pgfsetlinewidth{0.000000pt}%
\definecolor{currentstroke}{rgb}{0.000000,0.000000,0.000000}%
\pgfsetstrokecolor{currentstroke}%
\pgfsetdash{}{0pt}%
\pgfpathmoveto{\pgfqpoint{1.941885in}{1.037720in}}%
\pgfpathlineto{\pgfqpoint{1.977544in}{1.047064in}}%
\pgfpathlineto{\pgfqpoint{1.940816in}{1.085296in}}%
\pgfpathlineto{\pgfqpoint{1.905055in}{1.076055in}}%
\pgfpathclose%
\pgfusepath{fill}%
\end{pgfscope}%
\begin{pgfscope}%
\pgfpathrectangle{\pgfqpoint{0.150000in}{0.150000in}}{\pgfqpoint{2.700000in}{1.950000in}}%
\pgfusepath{clip}%
\pgfsetbuttcap%
\pgfsetroundjoin%
\definecolor{currentfill}{rgb}{0.994301,0.989660,0.990028}%
\pgfsetfillcolor{currentfill}%
\pgfsetlinewidth{0.000000pt}%
\definecolor{currentstroke}{rgb}{0.000000,0.000000,0.000000}%
\pgfsetstrokecolor{currentstroke}%
\pgfsetdash{}{0pt}%
\pgfpathmoveto{\pgfqpoint{1.905055in}{1.076055in}}%
\pgfpathlineto{\pgfqpoint{1.940816in}{1.085296in}}%
\pgfpathlineto{\pgfqpoint{1.904083in}{1.123534in}}%
\pgfpathlineto{\pgfqpoint{1.868221in}{1.114396in}}%
\pgfpathclose%
\pgfusepath{fill}%
\end{pgfscope}%
\begin{pgfscope}%
\pgfpathrectangle{\pgfqpoint{0.150000in}{0.150000in}}{\pgfqpoint{2.700000in}{1.950000in}}%
\pgfusepath{clip}%
\pgfsetbuttcap%
\pgfsetroundjoin%
\definecolor{currentfill}{rgb}{0.978232,0.980913,0.984666}%
\pgfsetfillcolor{currentfill}%
\pgfsetlinewidth{0.000000pt}%
\definecolor{currentstroke}{rgb}{0.000000,0.000000,0.000000}%
\pgfsetstrokecolor{currentstroke}%
\pgfsetdash{}{0pt}%
\pgfpathmoveto{\pgfqpoint{1.868221in}{1.114396in}}%
\pgfpathlineto{\pgfqpoint{1.904083in}{1.123534in}}%
\pgfpathlineto{\pgfqpoint{1.867345in}{1.161776in}}%
\pgfpathlineto{\pgfqpoint{1.831381in}{1.152741in}}%
\pgfpathclose%
\pgfusepath{fill}%
\end{pgfscope}%
\begin{pgfscope}%
\pgfpathrectangle{\pgfqpoint{0.150000in}{0.150000in}}{\pgfqpoint{2.700000in}{1.950000in}}%
\pgfusepath{clip}%
\pgfsetbuttcap%
\pgfsetroundjoin%
\definecolor{currentfill}{rgb}{0.947135,0.953646,0.962760}%
\pgfsetfillcolor{currentfill}%
\pgfsetlinewidth{0.000000pt}%
\definecolor{currentstroke}{rgb}{0.000000,0.000000,0.000000}%
\pgfsetstrokecolor{currentstroke}%
\pgfsetdash{}{0pt}%
\pgfpathmoveto{\pgfqpoint{1.831381in}{1.152741in}}%
\pgfpathlineto{\pgfqpoint{1.867345in}{1.161776in}}%
\pgfpathlineto{\pgfqpoint{1.830603in}{1.200024in}}%
\pgfpathlineto{\pgfqpoint{1.794536in}{1.191092in}}%
\pgfpathclose%
\pgfusepath{fill}%
\end{pgfscope}%
\begin{pgfscope}%
\pgfpathrectangle{\pgfqpoint{0.150000in}{0.150000in}}{\pgfqpoint{2.700000in}{1.950000in}}%
\pgfusepath{clip}%
\pgfsetbuttcap%
\pgfsetroundjoin%
\definecolor{currentfill}{rgb}{0.916039,0.926379,0.940855}%
\pgfsetfillcolor{currentfill}%
\pgfsetlinewidth{0.000000pt}%
\definecolor{currentstroke}{rgb}{0.000000,0.000000,0.000000}%
\pgfsetstrokecolor{currentstroke}%
\pgfsetdash{}{0pt}%
\pgfpathmoveto{\pgfqpoint{1.794536in}{1.191092in}}%
\pgfpathlineto{\pgfqpoint{1.830603in}{1.200024in}}%
\pgfpathlineto{\pgfqpoint{1.793855in}{1.238276in}}%
\pgfpathlineto{\pgfqpoint{1.757686in}{1.229447in}}%
\pgfpathclose%
\pgfusepath{fill}%
\end{pgfscope}%
\begin{pgfscope}%
\pgfpathrectangle{\pgfqpoint{0.150000in}{0.150000in}}{\pgfqpoint{2.700000in}{1.950000in}}%
\pgfusepath{clip}%
\pgfsetbuttcap%
\pgfsetroundjoin%
\definecolor{currentfill}{rgb}{0.884942,0.899112,0.918949}%
\pgfsetfillcolor{currentfill}%
\pgfsetlinewidth{0.000000pt}%
\definecolor{currentstroke}{rgb}{0.000000,0.000000,0.000000}%
\pgfsetstrokecolor{currentstroke}%
\pgfsetdash{}{0pt}%
\pgfpathmoveto{\pgfqpoint{1.757686in}{1.229447in}}%
\pgfpathlineto{\pgfqpoint{1.793855in}{1.238276in}}%
\pgfpathlineto{\pgfqpoint{1.757103in}{1.276534in}}%
\pgfpathlineto{\pgfqpoint{1.720832in}{1.267808in}}%
\pgfpathclose%
\pgfusepath{fill}%
\end{pgfscope}%
\begin{pgfscope}%
\pgfpathrectangle{\pgfqpoint{0.150000in}{0.150000in}}{\pgfqpoint{2.700000in}{1.950000in}}%
\pgfusepath{clip}%
\pgfsetbuttcap%
\pgfsetroundjoin%
\definecolor{currentfill}{rgb}{0.853845,0.871844,0.897044}%
\pgfsetfillcolor{currentfill}%
\pgfsetlinewidth{0.000000pt}%
\definecolor{currentstroke}{rgb}{0.000000,0.000000,0.000000}%
\pgfsetstrokecolor{currentstroke}%
\pgfsetdash{}{0pt}%
\pgfpathmoveto{\pgfqpoint{1.720832in}{1.267808in}}%
\pgfpathlineto{\pgfqpoint{1.757103in}{1.276534in}}%
\pgfpathlineto{\pgfqpoint{1.720345in}{1.314796in}}%
\pgfpathlineto{\pgfqpoint{1.683973in}{1.306174in}}%
\pgfpathclose%
\pgfusepath{fill}%
\end{pgfscope}%
\begin{pgfscope}%
\pgfpathrectangle{\pgfqpoint{0.150000in}{0.150000in}}{\pgfqpoint{2.700000in}{1.950000in}}%
\pgfusepath{clip}%
\pgfsetbuttcap%
\pgfsetroundjoin%
\definecolor{currentfill}{rgb}{0.822748,0.844577,0.875138}%
\pgfsetfillcolor{currentfill}%
\pgfsetlinewidth{0.000000pt}%
\definecolor{currentstroke}{rgb}{0.000000,0.000000,0.000000}%
\pgfsetstrokecolor{currentstroke}%
\pgfsetdash{}{0pt}%
\pgfpathmoveto{\pgfqpoint{1.683973in}{1.306174in}}%
\pgfpathlineto{\pgfqpoint{1.720345in}{1.314796in}}%
\pgfpathlineto{\pgfqpoint{1.683583in}{1.353064in}}%
\pgfpathlineto{\pgfqpoint{1.647108in}{1.344545in}}%
\pgfpathclose%
\pgfusepath{fill}%
\end{pgfscope}%
\begin{pgfscope}%
\pgfpathrectangle{\pgfqpoint{0.150000in}{0.150000in}}{\pgfqpoint{2.700000in}{1.950000in}}%
\pgfusepath{clip}%
\pgfsetbuttcap%
\pgfsetroundjoin%
\definecolor{currentfill}{rgb}{0.797871,0.822763,0.857613}%
\pgfsetfillcolor{currentfill}%
\pgfsetlinewidth{0.000000pt}%
\definecolor{currentstroke}{rgb}{0.000000,0.000000,0.000000}%
\pgfsetstrokecolor{currentstroke}%
\pgfsetdash{}{0pt}%
\pgfpathmoveto{\pgfqpoint{1.647108in}{1.344545in}}%
\pgfpathlineto{\pgfqpoint{1.683583in}{1.353064in}}%
\pgfpathlineto{\pgfqpoint{1.646816in}{1.391337in}}%
\pgfpathlineto{\pgfqpoint{1.610239in}{1.382921in}}%
\pgfpathclose%
\pgfusepath{fill}%
\end{pgfscope}%
\begin{pgfscope}%
\pgfpathrectangle{\pgfqpoint{0.150000in}{0.150000in}}{\pgfqpoint{2.700000in}{1.950000in}}%
\pgfusepath{clip}%
\pgfsetbuttcap%
\pgfsetroundjoin%
\definecolor{currentfill}{rgb}{0.766774,0.795496,0.835708}%
\pgfsetfillcolor{currentfill}%
\pgfsetlinewidth{0.000000pt}%
\definecolor{currentstroke}{rgb}{0.000000,0.000000,0.000000}%
\pgfsetstrokecolor{currentstroke}%
\pgfsetdash{}{0pt}%
\pgfpathmoveto{\pgfqpoint{1.610239in}{1.382921in}}%
\pgfpathlineto{\pgfqpoint{1.646816in}{1.391337in}}%
\pgfpathlineto{\pgfqpoint{1.610045in}{1.429615in}}%
\pgfpathlineto{\pgfqpoint{1.573365in}{1.421302in}}%
\pgfpathclose%
\pgfusepath{fill}%
\end{pgfscope}%
\begin{pgfscope}%
\pgfpathrectangle{\pgfqpoint{0.150000in}{0.150000in}}{\pgfqpoint{2.700000in}{1.950000in}}%
\pgfusepath{clip}%
\pgfsetbuttcap%
\pgfsetroundjoin%
\definecolor{currentfill}{rgb}{0.735677,0.768229,0.813802}%
\pgfsetfillcolor{currentfill}%
\pgfsetlinewidth{0.000000pt}%
\definecolor{currentstroke}{rgb}{0.000000,0.000000,0.000000}%
\pgfsetstrokecolor{currentstroke}%
\pgfsetdash{}{0pt}%
\pgfpathmoveto{\pgfqpoint{1.573365in}{1.421302in}}%
\pgfpathlineto{\pgfqpoint{1.610045in}{1.429615in}}%
\pgfpathlineto{\pgfqpoint{1.573268in}{1.467897in}}%
\pgfpathlineto{\pgfqpoint{1.536486in}{1.459688in}}%
\pgfpathclose%
\pgfusepath{fill}%
\end{pgfscope}%
\begin{pgfscope}%
\pgfpathrectangle{\pgfqpoint{0.150000in}{0.150000in}}{\pgfqpoint{2.700000in}{1.950000in}}%
\pgfusepath{clip}%
\pgfsetbuttcap%
\pgfsetroundjoin%
\definecolor{currentfill}{rgb}{0.704580,0.740962,0.791896}%
\pgfsetfillcolor{currentfill}%
\pgfsetlinewidth{0.000000pt}%
\definecolor{currentstroke}{rgb}{0.000000,0.000000,0.000000}%
\pgfsetstrokecolor{currentstroke}%
\pgfsetdash{}{0pt}%
\pgfpathmoveto{\pgfqpoint{1.536486in}{1.459688in}}%
\pgfpathlineto{\pgfqpoint{1.573268in}{1.467897in}}%
\pgfpathlineto{\pgfqpoint{1.536486in}{1.506185in}}%
\pgfpathlineto{\pgfqpoint{1.499603in}{1.498079in}}%
\pgfpathclose%
\pgfusepath{fill}%
\end{pgfscope}%
\begin{pgfscope}%
\pgfpathrectangle{\pgfqpoint{0.150000in}{0.150000in}}{\pgfqpoint{2.700000in}{1.950000in}}%
\pgfusepath{clip}%
\pgfsetbuttcap%
\pgfsetroundjoin%
\definecolor{currentfill}{rgb}{0.673483,0.713695,0.769991}%
\pgfsetfillcolor{currentfill}%
\pgfsetlinewidth{0.000000pt}%
\definecolor{currentstroke}{rgb}{0.000000,0.000000,0.000000}%
\pgfsetstrokecolor{currentstroke}%
\pgfsetdash{}{0pt}%
\pgfpathmoveto{\pgfqpoint{1.499603in}{1.498079in}}%
\pgfpathlineto{\pgfqpoint{1.536486in}{1.506185in}}%
\pgfpathlineto{\pgfqpoint{1.499700in}{1.544478in}}%
\pgfpathlineto{\pgfqpoint{1.462714in}{1.536476in}}%
\pgfpathclose%
\pgfusepath{fill}%
\end{pgfscope}%
\begin{pgfscope}%
\pgfpathrectangle{\pgfqpoint{0.150000in}{0.150000in}}{\pgfqpoint{2.700000in}{1.950000in}}%
\pgfusepath{clip}%
\pgfsetbuttcap%
\pgfsetroundjoin%
\definecolor{currentfill}{rgb}{0.642387,0.686428,0.748085}%
\pgfsetfillcolor{currentfill}%
\pgfsetlinewidth{0.000000pt}%
\definecolor{currentstroke}{rgb}{0.000000,0.000000,0.000000}%
\pgfsetstrokecolor{currentstroke}%
\pgfsetdash{}{0pt}%
\pgfpathmoveto{\pgfqpoint{1.462714in}{1.536476in}}%
\pgfpathlineto{\pgfqpoint{1.499700in}{1.544478in}}%
\pgfpathlineto{\pgfqpoint{1.462909in}{1.582776in}}%
\pgfpathlineto{\pgfqpoint{1.425821in}{1.574877in}}%
\pgfpathclose%
\pgfusepath{fill}%
\end{pgfscope}%
\begin{pgfscope}%
\pgfpathrectangle{\pgfqpoint{0.150000in}{0.150000in}}{\pgfqpoint{2.700000in}{1.950000in}}%
\pgfusepath{clip}%
\pgfsetbuttcap%
\pgfsetroundjoin%
\definecolor{currentfill}{rgb}{0.611290,0.659161,0.726180}%
\pgfsetfillcolor{currentfill}%
\pgfsetlinewidth{0.000000pt}%
\definecolor{currentstroke}{rgb}{0.000000,0.000000,0.000000}%
\pgfsetstrokecolor{currentstroke}%
\pgfsetdash{}{0pt}%
\pgfpathmoveto{\pgfqpoint{1.425821in}{1.574877in}}%
\pgfpathlineto{\pgfqpoint{1.462909in}{1.582776in}}%
\pgfpathlineto{\pgfqpoint{1.426113in}{1.621079in}}%
\pgfpathlineto{\pgfqpoint{1.388922in}{1.613284in}}%
\pgfpathclose%
\pgfusepath{fill}%
\end{pgfscope}%
\begin{pgfscope}%
\pgfpathrectangle{\pgfqpoint{0.150000in}{0.150000in}}{\pgfqpoint{2.700000in}{1.950000in}}%
\pgfusepath{clip}%
\pgfsetbuttcap%
\pgfsetroundjoin%
\definecolor{currentfill}{rgb}{0.580193,0.631893,0.704274}%
\pgfsetfillcolor{currentfill}%
\pgfsetlinewidth{0.000000pt}%
\definecolor{currentstroke}{rgb}{0.000000,0.000000,0.000000}%
\pgfsetstrokecolor{currentstroke}%
\pgfsetdash{}{0pt}%
\pgfpathmoveto{\pgfqpoint{1.388922in}{1.613284in}}%
\pgfpathlineto{\pgfqpoint{1.426113in}{1.621079in}}%
\pgfpathlineto{\pgfqpoint{1.389312in}{1.659388in}}%
\pgfpathlineto{\pgfqpoint{1.352019in}{1.651695in}}%
\pgfpathclose%
\pgfusepath{fill}%
\end{pgfscope}%
\begin{pgfscope}%
\pgfpathrectangle{\pgfqpoint{0.150000in}{0.150000in}}{\pgfqpoint{2.700000in}{1.950000in}}%
\pgfusepath{clip}%
\pgfsetbuttcap%
\pgfsetroundjoin%
\definecolor{currentfill}{rgb}{0.555316,0.610080,0.686749}%
\pgfsetfillcolor{currentfill}%
\pgfsetlinewidth{0.000000pt}%
\definecolor{currentstroke}{rgb}{0.000000,0.000000,0.000000}%
\pgfsetstrokecolor{currentstroke}%
\pgfsetdash{}{0pt}%
\pgfpathmoveto{\pgfqpoint{1.352019in}{1.651695in}}%
\pgfpathlineto{\pgfqpoint{1.389312in}{1.659388in}}%
\pgfpathlineto{\pgfqpoint{1.352506in}{1.697701in}}%
\pgfpathlineto{\pgfqpoint{1.315111in}{1.690112in}}%
\pgfpathclose%
\pgfusepath{fill}%
\end{pgfscope}%
\begin{pgfscope}%
\pgfpathrectangle{\pgfqpoint{0.150000in}{0.150000in}}{\pgfqpoint{2.700000in}{1.950000in}}%
\pgfusepath{clip}%
\pgfsetbuttcap%
\pgfsetroundjoin%
\definecolor{currentfill}{rgb}{0.838542,0.707031,0.717448}%
\pgfsetfillcolor{currentfill}%
\pgfsetlinewidth{0.000000pt}%
\definecolor{currentstroke}{rgb}{0.000000,0.000000,0.000000}%
\pgfsetstrokecolor{currentstroke}%
\pgfsetdash{}{0pt}%
\pgfpathmoveto{\pgfqpoint{2.201317in}{0.720997in}}%
\pgfpathlineto{\pgfqpoint{2.236350in}{0.731220in}}%
\pgfpathlineto{\pgfqpoint{2.199559in}{0.769514in}}%
\pgfpathlineto{\pgfqpoint{2.164424in}{0.759395in}}%
\pgfpathclose%
\pgfusepath{fill}%
\end{pgfscope}%
\begin{pgfscope}%
\pgfpathrectangle{\pgfqpoint{0.150000in}{0.150000in}}{\pgfqpoint{2.700000in}{1.950000in}}%
\pgfusepath{clip}%
\pgfsetbuttcap%
\pgfsetroundjoin%
\definecolor{currentfill}{rgb}{0.857537,0.741498,0.750689}%
\pgfsetfillcolor{currentfill}%
\pgfsetlinewidth{0.000000pt}%
\definecolor{currentstroke}{rgb}{0.000000,0.000000,0.000000}%
\pgfsetstrokecolor{currentstroke}%
\pgfsetdash{}{0pt}%
\pgfpathmoveto{\pgfqpoint{2.164424in}{0.759395in}}%
\pgfpathlineto{\pgfqpoint{2.199559in}{0.769514in}}%
\pgfpathlineto{\pgfqpoint{2.162763in}{0.807814in}}%
\pgfpathlineto{\pgfqpoint{2.127525in}{0.797798in}}%
\pgfpathclose%
\pgfusepath{fill}%
\end{pgfscope}%
\begin{pgfscope}%
\pgfpathrectangle{\pgfqpoint{0.150000in}{0.150000in}}{\pgfqpoint{2.700000in}{1.950000in}}%
\pgfusepath{clip}%
\pgfsetbuttcap%
\pgfsetroundjoin%
\definecolor{currentfill}{rgb}{0.876532,0.775965,0.783931}%
\pgfsetfillcolor{currentfill}%
\pgfsetlinewidth{0.000000pt}%
\definecolor{currentstroke}{rgb}{0.000000,0.000000,0.000000}%
\pgfsetstrokecolor{currentstroke}%
\pgfsetdash{}{0pt}%
\pgfpathmoveto{\pgfqpoint{2.127525in}{0.797798in}}%
\pgfpathlineto{\pgfqpoint{2.162763in}{0.807814in}}%
\pgfpathlineto{\pgfqpoint{2.125963in}{0.846119in}}%
\pgfpathlineto{\pgfqpoint{2.090622in}{0.836206in}}%
\pgfpathclose%
\pgfusepath{fill}%
\end{pgfscope}%
\begin{pgfscope}%
\pgfpathrectangle{\pgfqpoint{0.150000in}{0.150000in}}{\pgfqpoint{2.700000in}{1.950000in}}%
\pgfusepath{clip}%
\pgfsetbuttcap%
\pgfsetroundjoin%
\definecolor{currentfill}{rgb}{0.891728,0.803539,0.810524}%
\pgfsetfillcolor{currentfill}%
\pgfsetlinewidth{0.000000pt}%
\definecolor{currentstroke}{rgb}{0.000000,0.000000,0.000000}%
\pgfsetstrokecolor{currentstroke}%
\pgfsetdash{}{0pt}%
\pgfpathmoveto{\pgfqpoint{2.090622in}{0.836206in}}%
\pgfpathlineto{\pgfqpoint{2.125963in}{0.846119in}}%
\pgfpathlineto{\pgfqpoint{2.089157in}{0.884429in}}%
\pgfpathlineto{\pgfqpoint{2.053714in}{0.874620in}}%
\pgfpathclose%
\pgfusepath{fill}%
\end{pgfscope}%
\begin{pgfscope}%
\pgfpathrectangle{\pgfqpoint{0.150000in}{0.150000in}}{\pgfqpoint{2.700000in}{1.950000in}}%
\pgfusepath{clip}%
\pgfsetbuttcap%
\pgfsetroundjoin%
\definecolor{currentfill}{rgb}{0.910723,0.838006,0.843765}%
\pgfsetfillcolor{currentfill}%
\pgfsetlinewidth{0.000000pt}%
\definecolor{currentstroke}{rgb}{0.000000,0.000000,0.000000}%
\pgfsetstrokecolor{currentstroke}%
\pgfsetdash{}{0pt}%
\pgfpathmoveto{\pgfqpoint{2.053714in}{0.874620in}}%
\pgfpathlineto{\pgfqpoint{2.089157in}{0.884429in}}%
\pgfpathlineto{\pgfqpoint{2.052346in}{0.922744in}}%
\pgfpathlineto{\pgfqpoint{2.016801in}{0.913038in}}%
\pgfpathclose%
\pgfusepath{fill}%
\end{pgfscope}%
\begin{pgfscope}%
\pgfpathrectangle{\pgfqpoint{0.150000in}{0.150000in}}{\pgfqpoint{2.700000in}{1.950000in}}%
\pgfusepath{clip}%
\pgfsetbuttcap%
\pgfsetroundjoin%
\definecolor{currentfill}{rgb}{0.929718,0.872472,0.877007}%
\pgfsetfillcolor{currentfill}%
\pgfsetlinewidth{0.000000pt}%
\definecolor{currentstroke}{rgb}{0.000000,0.000000,0.000000}%
\pgfsetstrokecolor{currentstroke}%
\pgfsetdash{}{0pt}%
\pgfpathmoveto{\pgfqpoint{2.016801in}{0.913038in}}%
\pgfpathlineto{\pgfqpoint{2.052346in}{0.922744in}}%
\pgfpathlineto{\pgfqpoint{2.015531in}{0.961064in}}%
\pgfpathlineto{\pgfqpoint{1.979883in}{0.951462in}}%
\pgfpathclose%
\pgfusepath{fill}%
\end{pgfscope}%
\begin{pgfscope}%
\pgfpathrectangle{\pgfqpoint{0.150000in}{0.150000in}}{\pgfqpoint{2.700000in}{1.950000in}}%
\pgfusepath{clip}%
\pgfsetbuttcap%
\pgfsetroundjoin%
\definecolor{currentfill}{rgb}{0.948713,0.906939,0.910248}%
\pgfsetfillcolor{currentfill}%
\pgfsetlinewidth{0.000000pt}%
\definecolor{currentstroke}{rgb}{0.000000,0.000000,0.000000}%
\pgfsetstrokecolor{currentstroke}%
\pgfsetdash{}{0pt}%
\pgfpathmoveto{\pgfqpoint{1.979883in}{0.951462in}}%
\pgfpathlineto{\pgfqpoint{2.015531in}{0.961064in}}%
\pgfpathlineto{\pgfqpoint{1.978711in}{0.999389in}}%
\pgfpathlineto{\pgfqpoint{1.942961in}{0.989891in}}%
\pgfpathclose%
\pgfusepath{fill}%
\end{pgfscope}%
\begin{pgfscope}%
\pgfpathrectangle{\pgfqpoint{0.150000in}{0.150000in}}{\pgfqpoint{2.700000in}{1.950000in}}%
\pgfusepath{clip}%
\pgfsetbuttcap%
\pgfsetroundjoin%
\definecolor{currentfill}{rgb}{0.967708,0.941406,0.943490}%
\pgfsetfillcolor{currentfill}%
\pgfsetlinewidth{0.000000pt}%
\definecolor{currentstroke}{rgb}{0.000000,0.000000,0.000000}%
\pgfsetstrokecolor{currentstroke}%
\pgfsetdash{}{0pt}%
\pgfpathmoveto{\pgfqpoint{1.942961in}{0.989891in}}%
\pgfpathlineto{\pgfqpoint{1.978711in}{0.999389in}}%
\pgfpathlineto{\pgfqpoint{1.941885in}{1.037720in}}%
\pgfpathlineto{\pgfqpoint{1.906033in}{1.028325in}}%
\pgfpathclose%
\pgfusepath{fill}%
\end{pgfscope}%
\begin{pgfscope}%
\pgfpathrectangle{\pgfqpoint{0.150000in}{0.150000in}}{\pgfqpoint{2.700000in}{1.950000in}}%
\pgfusepath{clip}%
\pgfsetbuttcap%
\pgfsetroundjoin%
\definecolor{currentfill}{rgb}{0.986703,0.975873,0.976731}%
\pgfsetfillcolor{currentfill}%
\pgfsetlinewidth{0.000000pt}%
\definecolor{currentstroke}{rgb}{0.000000,0.000000,0.000000}%
\pgfsetstrokecolor{currentstroke}%
\pgfsetdash{}{0pt}%
\pgfpathmoveto{\pgfqpoint{1.906033in}{1.028325in}}%
\pgfpathlineto{\pgfqpoint{1.941885in}{1.037720in}}%
\pgfpathlineto{\pgfqpoint{1.905055in}{1.076055in}}%
\pgfpathlineto{\pgfqpoint{1.869100in}{1.066764in}}%
\pgfpathclose%
\pgfusepath{fill}%
\end{pgfscope}%
\begin{pgfscope}%
\pgfpathrectangle{\pgfqpoint{0.150000in}{0.150000in}}{\pgfqpoint{2.700000in}{1.950000in}}%
\pgfusepath{clip}%
\pgfsetbuttcap%
\pgfsetroundjoin%
\definecolor{currentfill}{rgb}{0.990671,0.991820,0.993428}%
\pgfsetfillcolor{currentfill}%
\pgfsetlinewidth{0.000000pt}%
\definecolor{currentstroke}{rgb}{0.000000,0.000000,0.000000}%
\pgfsetstrokecolor{currentstroke}%
\pgfsetdash{}{0pt}%
\pgfpathmoveto{\pgfqpoint{1.869100in}{1.066764in}}%
\pgfpathlineto{\pgfqpoint{1.905055in}{1.076055in}}%
\pgfpathlineto{\pgfqpoint{1.868221in}{1.114396in}}%
\pgfpathlineto{\pgfqpoint{1.832163in}{1.105208in}}%
\pgfpathclose%
\pgfusepath{fill}%
\end{pgfscope}%
\begin{pgfscope}%
\pgfpathrectangle{\pgfqpoint{0.150000in}{0.150000in}}{\pgfqpoint{2.700000in}{1.950000in}}%
\pgfusepath{clip}%
\pgfsetbuttcap%
\pgfsetroundjoin%
\definecolor{currentfill}{rgb}{0.959574,0.964553,0.971523}%
\pgfsetfillcolor{currentfill}%
\pgfsetlinewidth{0.000000pt}%
\definecolor{currentstroke}{rgb}{0.000000,0.000000,0.000000}%
\pgfsetstrokecolor{currentstroke}%
\pgfsetdash{}{0pt}%
\pgfpathmoveto{\pgfqpoint{1.832163in}{1.105208in}}%
\pgfpathlineto{\pgfqpoint{1.868221in}{1.114396in}}%
\pgfpathlineto{\pgfqpoint{1.831381in}{1.152741in}}%
\pgfpathlineto{\pgfqpoint{1.795221in}{1.143657in}}%
\pgfpathclose%
\pgfusepath{fill}%
\end{pgfscope}%
\begin{pgfscope}%
\pgfpathrectangle{\pgfqpoint{0.150000in}{0.150000in}}{\pgfqpoint{2.700000in}{1.950000in}}%
\pgfusepath{clip}%
\pgfsetbuttcap%
\pgfsetroundjoin%
\definecolor{currentfill}{rgb}{0.934697,0.942739,0.953998}%
\pgfsetfillcolor{currentfill}%
\pgfsetlinewidth{0.000000pt}%
\definecolor{currentstroke}{rgb}{0.000000,0.000000,0.000000}%
\pgfsetstrokecolor{currentstroke}%
\pgfsetdash{}{0pt}%
\pgfpathmoveto{\pgfqpoint{1.795221in}{1.143657in}}%
\pgfpathlineto{\pgfqpoint{1.831381in}{1.152741in}}%
\pgfpathlineto{\pgfqpoint{1.794536in}{1.191092in}}%
\pgfpathlineto{\pgfqpoint{1.758273in}{1.182111in}}%
\pgfpathclose%
\pgfusepath{fill}%
\end{pgfscope}%
\begin{pgfscope}%
\pgfpathrectangle{\pgfqpoint{0.150000in}{0.150000in}}{\pgfqpoint{2.700000in}{1.950000in}}%
\pgfusepath{clip}%
\pgfsetbuttcap%
\pgfsetroundjoin%
\definecolor{currentfill}{rgb}{0.903600,0.915472,0.932093}%
\pgfsetfillcolor{currentfill}%
\pgfsetlinewidth{0.000000pt}%
\definecolor{currentstroke}{rgb}{0.000000,0.000000,0.000000}%
\pgfsetstrokecolor{currentstroke}%
\pgfsetdash{}{0pt}%
\pgfpathmoveto{\pgfqpoint{1.758273in}{1.182111in}}%
\pgfpathlineto{\pgfqpoint{1.794536in}{1.191092in}}%
\pgfpathlineto{\pgfqpoint{1.757686in}{1.229447in}}%
\pgfpathlineto{\pgfqpoint{1.721321in}{1.220571in}}%
\pgfpathclose%
\pgfusepath{fill}%
\end{pgfscope}%
\begin{pgfscope}%
\pgfpathrectangle{\pgfqpoint{0.150000in}{0.150000in}}{\pgfqpoint{2.700000in}{1.950000in}}%
\pgfusepath{clip}%
\pgfsetbuttcap%
\pgfsetroundjoin%
\definecolor{currentfill}{rgb}{0.872503,0.888205,0.910187}%
\pgfsetfillcolor{currentfill}%
\pgfsetlinewidth{0.000000pt}%
\definecolor{currentstroke}{rgb}{0.000000,0.000000,0.000000}%
\pgfsetstrokecolor{currentstroke}%
\pgfsetdash{}{0pt}%
\pgfpathmoveto{\pgfqpoint{1.721321in}{1.220571in}}%
\pgfpathlineto{\pgfqpoint{1.757686in}{1.229447in}}%
\pgfpathlineto{\pgfqpoint{1.720832in}{1.267808in}}%
\pgfpathlineto{\pgfqpoint{1.684364in}{1.259035in}}%
\pgfpathclose%
\pgfusepath{fill}%
\end{pgfscope}%
\begin{pgfscope}%
\pgfpathrectangle{\pgfqpoint{0.150000in}{0.150000in}}{\pgfqpoint{2.700000in}{1.950000in}}%
\pgfusepath{clip}%
\pgfsetbuttcap%
\pgfsetroundjoin%
\definecolor{currentfill}{rgb}{0.841406,0.860938,0.888281}%
\pgfsetfillcolor{currentfill}%
\pgfsetlinewidth{0.000000pt}%
\definecolor{currentstroke}{rgb}{0.000000,0.000000,0.000000}%
\pgfsetstrokecolor{currentstroke}%
\pgfsetdash{}{0pt}%
\pgfpathmoveto{\pgfqpoint{1.684364in}{1.259035in}}%
\pgfpathlineto{\pgfqpoint{1.720832in}{1.267808in}}%
\pgfpathlineto{\pgfqpoint{1.683973in}{1.306174in}}%
\pgfpathlineto{\pgfqpoint{1.647402in}{1.297505in}}%
\pgfpathclose%
\pgfusepath{fill}%
\end{pgfscope}%
\begin{pgfscope}%
\pgfpathrectangle{\pgfqpoint{0.150000in}{0.150000in}}{\pgfqpoint{2.700000in}{1.950000in}}%
\pgfusepath{clip}%
\pgfsetbuttcap%
\pgfsetroundjoin%
\definecolor{currentfill}{rgb}{0.810309,0.833670,0.866376}%
\pgfsetfillcolor{currentfill}%
\pgfsetlinewidth{0.000000pt}%
\definecolor{currentstroke}{rgb}{0.000000,0.000000,0.000000}%
\pgfsetstrokecolor{currentstroke}%
\pgfsetdash{}{0pt}%
\pgfpathmoveto{\pgfqpoint{1.647402in}{1.297505in}}%
\pgfpathlineto{\pgfqpoint{1.683973in}{1.306174in}}%
\pgfpathlineto{\pgfqpoint{1.647108in}{1.344545in}}%
\pgfpathlineto{\pgfqpoint{1.610435in}{1.335979in}}%
\pgfpathclose%
\pgfusepath{fill}%
\end{pgfscope}%
\begin{pgfscope}%
\pgfpathrectangle{\pgfqpoint{0.150000in}{0.150000in}}{\pgfqpoint{2.700000in}{1.950000in}}%
\pgfusepath{clip}%
\pgfsetbuttcap%
\pgfsetroundjoin%
\definecolor{currentfill}{rgb}{0.779213,0.806403,0.844470}%
\pgfsetfillcolor{currentfill}%
\pgfsetlinewidth{0.000000pt}%
\definecolor{currentstroke}{rgb}{0.000000,0.000000,0.000000}%
\pgfsetstrokecolor{currentstroke}%
\pgfsetdash{}{0pt}%
\pgfpathmoveto{\pgfqpoint{1.610435in}{1.335979in}}%
\pgfpathlineto{\pgfqpoint{1.647108in}{1.344545in}}%
\pgfpathlineto{\pgfqpoint{1.610239in}{1.382921in}}%
\pgfpathlineto{\pgfqpoint{1.573463in}{1.374459in}}%
\pgfpathclose%
\pgfusepath{fill}%
\end{pgfscope}%
\begin{pgfscope}%
\pgfpathrectangle{\pgfqpoint{0.150000in}{0.150000in}}{\pgfqpoint{2.700000in}{1.950000in}}%
\pgfusepath{clip}%
\pgfsetbuttcap%
\pgfsetroundjoin%
\definecolor{currentfill}{rgb}{0.748116,0.779136,0.822564}%
\pgfsetfillcolor{currentfill}%
\pgfsetlinewidth{0.000000pt}%
\definecolor{currentstroke}{rgb}{0.000000,0.000000,0.000000}%
\pgfsetstrokecolor{currentstroke}%
\pgfsetdash{}{0pt}%
\pgfpathmoveto{\pgfqpoint{1.573463in}{1.374459in}}%
\pgfpathlineto{\pgfqpoint{1.610239in}{1.382921in}}%
\pgfpathlineto{\pgfqpoint{1.573365in}{1.421302in}}%
\pgfpathlineto{\pgfqpoint{1.536486in}{1.412944in}}%
\pgfpathclose%
\pgfusepath{fill}%
\end{pgfscope}%
\begin{pgfscope}%
\pgfpathrectangle{\pgfqpoint{0.150000in}{0.150000in}}{\pgfqpoint{2.700000in}{1.950000in}}%
\pgfusepath{clip}%
\pgfsetbuttcap%
\pgfsetroundjoin%
\definecolor{currentfill}{rgb}{0.717019,0.751869,0.800659}%
\pgfsetfillcolor{currentfill}%
\pgfsetlinewidth{0.000000pt}%
\definecolor{currentstroke}{rgb}{0.000000,0.000000,0.000000}%
\pgfsetstrokecolor{currentstroke}%
\pgfsetdash{}{0pt}%
\pgfpathmoveto{\pgfqpoint{1.536486in}{1.412944in}}%
\pgfpathlineto{\pgfqpoint{1.573365in}{1.421302in}}%
\pgfpathlineto{\pgfqpoint{1.536486in}{1.459688in}}%
\pgfpathlineto{\pgfqpoint{1.499505in}{1.451434in}}%
\pgfpathclose%
\pgfusepath{fill}%
\end{pgfscope}%
\begin{pgfscope}%
\pgfpathrectangle{\pgfqpoint{0.150000in}{0.150000in}}{\pgfqpoint{2.700000in}{1.950000in}}%
\pgfusepath{clip}%
\pgfsetbuttcap%
\pgfsetroundjoin%
\definecolor{currentfill}{rgb}{0.692142,0.730055,0.783134}%
\pgfsetfillcolor{currentfill}%
\pgfsetlinewidth{0.000000pt}%
\definecolor{currentstroke}{rgb}{0.000000,0.000000,0.000000}%
\pgfsetstrokecolor{currentstroke}%
\pgfsetdash{}{0pt}%
\pgfpathmoveto{\pgfqpoint{1.499505in}{1.451434in}}%
\pgfpathlineto{\pgfqpoint{1.536486in}{1.459688in}}%
\pgfpathlineto{\pgfqpoint{1.499603in}{1.498079in}}%
\pgfpathlineto{\pgfqpoint{1.462518in}{1.489929in}}%
\pgfpathclose%
\pgfusepath{fill}%
\end{pgfscope}%
\begin{pgfscope}%
\pgfpathrectangle{\pgfqpoint{0.150000in}{0.150000in}}{\pgfqpoint{2.700000in}{1.950000in}}%
\pgfusepath{clip}%
\pgfsetbuttcap%
\pgfsetroundjoin%
\definecolor{currentfill}{rgb}{0.661045,0.702788,0.761229}%
\pgfsetfillcolor{currentfill}%
\pgfsetlinewidth{0.000000pt}%
\definecolor{currentstroke}{rgb}{0.000000,0.000000,0.000000}%
\pgfsetstrokecolor{currentstroke}%
\pgfsetdash{}{0pt}%
\pgfpathmoveto{\pgfqpoint{1.462518in}{1.489929in}}%
\pgfpathlineto{\pgfqpoint{1.499603in}{1.498079in}}%
\pgfpathlineto{\pgfqpoint{1.462714in}{1.536476in}}%
\pgfpathlineto{\pgfqpoint{1.425527in}{1.528430in}}%
\pgfpathclose%
\pgfusepath{fill}%
\end{pgfscope}%
\begin{pgfscope}%
\pgfpathrectangle{\pgfqpoint{0.150000in}{0.150000in}}{\pgfqpoint{2.700000in}{1.950000in}}%
\pgfusepath{clip}%
\pgfsetbuttcap%
\pgfsetroundjoin%
\definecolor{currentfill}{rgb}{0.629948,0.675521,0.739323}%
\pgfsetfillcolor{currentfill}%
\pgfsetlinewidth{0.000000pt}%
\definecolor{currentstroke}{rgb}{0.000000,0.000000,0.000000}%
\pgfsetstrokecolor{currentstroke}%
\pgfsetdash{}{0pt}%
\pgfpathmoveto{\pgfqpoint{1.425527in}{1.528430in}}%
\pgfpathlineto{\pgfqpoint{1.462714in}{1.536476in}}%
\pgfpathlineto{\pgfqpoint{1.425821in}{1.574877in}}%
\pgfpathlineto{\pgfqpoint{1.388530in}{1.566935in}}%
\pgfpathclose%
\pgfusepath{fill}%
\end{pgfscope}%
\begin{pgfscope}%
\pgfpathrectangle{\pgfqpoint{0.150000in}{0.150000in}}{\pgfqpoint{2.700000in}{1.950000in}}%
\pgfusepath{clip}%
\pgfsetbuttcap%
\pgfsetroundjoin%
\definecolor{currentfill}{rgb}{0.598851,0.648254,0.717417}%
\pgfsetfillcolor{currentfill}%
\pgfsetlinewidth{0.000000pt}%
\definecolor{currentstroke}{rgb}{0.000000,0.000000,0.000000}%
\pgfsetstrokecolor{currentstroke}%
\pgfsetdash{}{0pt}%
\pgfpathmoveto{\pgfqpoint{1.388530in}{1.566935in}}%
\pgfpathlineto{\pgfqpoint{1.425821in}{1.574877in}}%
\pgfpathlineto{\pgfqpoint{1.388922in}{1.613284in}}%
\pgfpathlineto{\pgfqpoint{1.351529in}{1.605446in}}%
\pgfpathclose%
\pgfusepath{fill}%
\end{pgfscope}%
\begin{pgfscope}%
\pgfpathrectangle{\pgfqpoint{0.150000in}{0.150000in}}{\pgfqpoint{2.700000in}{1.950000in}}%
\pgfusepath{clip}%
\pgfsetbuttcap%
\pgfsetroundjoin%
\definecolor{currentfill}{rgb}{0.567754,0.620987,0.695512}%
\pgfsetfillcolor{currentfill}%
\pgfsetlinewidth{0.000000pt}%
\definecolor{currentstroke}{rgb}{0.000000,0.000000,0.000000}%
\pgfsetstrokecolor{currentstroke}%
\pgfsetdash{}{0pt}%
\pgfpathmoveto{\pgfqpoint{1.351529in}{1.605446in}}%
\pgfpathlineto{\pgfqpoint{1.388922in}{1.613284in}}%
\pgfpathlineto{\pgfqpoint{1.352019in}{1.651695in}}%
\pgfpathlineto{\pgfqpoint{1.314523in}{1.643961in}}%
\pgfpathclose%
\pgfusepath{fill}%
\end{pgfscope}%
\begin{pgfscope}%
\pgfpathrectangle{\pgfqpoint{0.150000in}{0.150000in}}{\pgfqpoint{2.700000in}{1.950000in}}%
\pgfusepath{clip}%
\pgfsetbuttcap%
\pgfsetroundjoin%
\definecolor{currentfill}{rgb}{0.536657,0.593719,0.673606}%
\pgfsetfillcolor{currentfill}%
\pgfsetlinewidth{0.000000pt}%
\definecolor{currentstroke}{rgb}{0.000000,0.000000,0.000000}%
\pgfsetstrokecolor{currentstroke}%
\pgfsetdash{}{0pt}%
\pgfpathmoveto{\pgfqpoint{1.314523in}{1.643961in}}%
\pgfpathlineto{\pgfqpoint{1.352019in}{1.651695in}}%
\pgfpathlineto{\pgfqpoint{1.315111in}{1.690112in}}%
\pgfpathlineto{\pgfqpoint{1.277512in}{1.682482in}}%
\pgfpathclose%
\pgfusepath{fill}%
\end{pgfscope}%
\begin{pgfscope}%
\pgfpathrectangle{\pgfqpoint{0.150000in}{0.150000in}}{\pgfqpoint{2.700000in}{1.950000in}}%
\pgfusepath{clip}%
\pgfsetbuttcap%
\pgfsetroundjoin%
\definecolor{currentfill}{rgb}{0.846140,0.720818,0.730744}%
\pgfsetfillcolor{currentfill}%
\pgfsetlinewidth{0.000000pt}%
\definecolor{currentstroke}{rgb}{0.000000,0.000000,0.000000}%
\pgfsetstrokecolor{currentstroke}%
\pgfsetdash{}{0pt}%
\pgfpathmoveto{\pgfqpoint{2.166093in}{0.710718in}}%
\pgfpathlineto{\pgfqpoint{2.201317in}{0.720997in}}%
\pgfpathlineto{\pgfqpoint{2.164424in}{0.759395in}}%
\pgfpathlineto{\pgfqpoint{2.129097in}{0.749220in}}%
\pgfpathclose%
\pgfusepath{fill}%
\end{pgfscope}%
\begin{pgfscope}%
\pgfpathrectangle{\pgfqpoint{0.150000in}{0.150000in}}{\pgfqpoint{2.700000in}{1.950000in}}%
\pgfusepath{clip}%
\pgfsetbuttcap%
\pgfsetroundjoin%
\definecolor{currentfill}{rgb}{0.865135,0.755285,0.763986}%
\pgfsetfillcolor{currentfill}%
\pgfsetlinewidth{0.000000pt}%
\definecolor{currentstroke}{rgb}{0.000000,0.000000,0.000000}%
\pgfsetstrokecolor{currentstroke}%
\pgfsetdash{}{0pt}%
\pgfpathmoveto{\pgfqpoint{2.129097in}{0.749220in}}%
\pgfpathlineto{\pgfqpoint{2.164424in}{0.759395in}}%
\pgfpathlineto{\pgfqpoint{2.127525in}{0.797798in}}%
\pgfpathlineto{\pgfqpoint{2.092096in}{0.787727in}}%
\pgfpathclose%
\pgfusepath{fill}%
\end{pgfscope}%
\begin{pgfscope}%
\pgfpathrectangle{\pgfqpoint{0.150000in}{0.150000in}}{\pgfqpoint{2.700000in}{1.950000in}}%
\pgfusepath{clip}%
\pgfsetbuttcap%
\pgfsetroundjoin%
\definecolor{currentfill}{rgb}{0.884130,0.789752,0.797227}%
\pgfsetfillcolor{currentfill}%
\pgfsetlinewidth{0.000000pt}%
\definecolor{currentstroke}{rgb}{0.000000,0.000000,0.000000}%
\pgfsetstrokecolor{currentstroke}%
\pgfsetdash{}{0pt}%
\pgfpathmoveto{\pgfqpoint{2.092096in}{0.787727in}}%
\pgfpathlineto{\pgfqpoint{2.127525in}{0.797798in}}%
\pgfpathlineto{\pgfqpoint{2.090622in}{0.836206in}}%
\pgfpathlineto{\pgfqpoint{2.055089in}{0.826240in}}%
\pgfpathclose%
\pgfusepath{fill}%
\end{pgfscope}%
\begin{pgfscope}%
\pgfpathrectangle{\pgfqpoint{0.150000in}{0.150000in}}{\pgfqpoint{2.700000in}{1.950000in}}%
\pgfusepath{clip}%
\pgfsetbuttcap%
\pgfsetroundjoin%
\definecolor{currentfill}{rgb}{0.903125,0.824219,0.830469}%
\pgfsetfillcolor{currentfill}%
\pgfsetlinewidth{0.000000pt}%
\definecolor{currentstroke}{rgb}{0.000000,0.000000,0.000000}%
\pgfsetstrokecolor{currentstroke}%
\pgfsetdash{}{0pt}%
\pgfpathmoveto{\pgfqpoint{2.055089in}{0.826240in}}%
\pgfpathlineto{\pgfqpoint{2.090622in}{0.836206in}}%
\pgfpathlineto{\pgfqpoint{2.053714in}{0.874620in}}%
\pgfpathlineto{\pgfqpoint{2.018078in}{0.864757in}}%
\pgfpathclose%
\pgfusepath{fill}%
\end{pgfscope}%
\begin{pgfscope}%
\pgfpathrectangle{\pgfqpoint{0.150000in}{0.150000in}}{\pgfqpoint{2.700000in}{1.950000in}}%
\pgfusepath{clip}%
\pgfsetbuttcap%
\pgfsetroundjoin%
\definecolor{currentfill}{rgb}{0.922120,0.858686,0.863710}%
\pgfsetfillcolor{currentfill}%
\pgfsetlinewidth{0.000000pt}%
\definecolor{currentstroke}{rgb}{0.000000,0.000000,0.000000}%
\pgfsetstrokecolor{currentstroke}%
\pgfsetdash{}{0pt}%
\pgfpathmoveto{\pgfqpoint{2.018078in}{0.864757in}}%
\pgfpathlineto{\pgfqpoint{2.053714in}{0.874620in}}%
\pgfpathlineto{\pgfqpoint{2.016801in}{0.913038in}}%
\pgfpathlineto{\pgfqpoint{1.981063in}{0.903280in}}%
\pgfpathclose%
\pgfusepath{fill}%
\end{pgfscope}%
\begin{pgfscope}%
\pgfpathrectangle{\pgfqpoint{0.150000in}{0.150000in}}{\pgfqpoint{2.700000in}{1.950000in}}%
\pgfusepath{clip}%
\pgfsetbuttcap%
\pgfsetroundjoin%
\definecolor{currentfill}{rgb}{0.941115,0.893153,0.896952}%
\pgfsetfillcolor{currentfill}%
\pgfsetlinewidth{0.000000pt}%
\definecolor{currentstroke}{rgb}{0.000000,0.000000,0.000000}%
\pgfsetstrokecolor{currentstroke}%
\pgfsetdash{}{0pt}%
\pgfpathmoveto{\pgfqpoint{1.981063in}{0.903280in}}%
\pgfpathlineto{\pgfqpoint{2.016801in}{0.913038in}}%
\pgfpathlineto{\pgfqpoint{1.979883in}{0.951462in}}%
\pgfpathlineto{\pgfqpoint{1.944042in}{0.941807in}}%
\pgfpathclose%
\pgfusepath{fill}%
\end{pgfscope}%
\begin{pgfscope}%
\pgfpathrectangle{\pgfqpoint{0.150000in}{0.150000in}}{\pgfqpoint{2.700000in}{1.950000in}}%
\pgfusepath{clip}%
\pgfsetbuttcap%
\pgfsetroundjoin%
\definecolor{currentfill}{rgb}{0.960110,0.927619,0.930193}%
\pgfsetfillcolor{currentfill}%
\pgfsetlinewidth{0.000000pt}%
\definecolor{currentstroke}{rgb}{0.000000,0.000000,0.000000}%
\pgfsetstrokecolor{currentstroke}%
\pgfsetdash{}{0pt}%
\pgfpathmoveto{\pgfqpoint{1.944042in}{0.941807in}}%
\pgfpathlineto{\pgfqpoint{1.979883in}{0.951462in}}%
\pgfpathlineto{\pgfqpoint{1.942961in}{0.989891in}}%
\pgfpathlineto{\pgfqpoint{1.907016in}{0.980340in}}%
\pgfpathclose%
\pgfusepath{fill}%
\end{pgfscope}%
\begin{pgfscope}%
\pgfpathrectangle{\pgfqpoint{0.150000in}{0.150000in}}{\pgfqpoint{2.700000in}{1.950000in}}%
\pgfusepath{clip}%
\pgfsetbuttcap%
\pgfsetroundjoin%
\definecolor{currentfill}{rgb}{0.975306,0.955193,0.956786}%
\pgfsetfillcolor{currentfill}%
\pgfsetlinewidth{0.000000pt}%
\definecolor{currentstroke}{rgb}{0.000000,0.000000,0.000000}%
\pgfsetstrokecolor{currentstroke}%
\pgfsetdash{}{0pt}%
\pgfpathmoveto{\pgfqpoint{1.907016in}{0.980340in}}%
\pgfpathlineto{\pgfqpoint{1.942961in}{0.989891in}}%
\pgfpathlineto{\pgfqpoint{1.906033in}{1.028325in}}%
\pgfpathlineto{\pgfqpoint{1.869985in}{1.018878in}}%
\pgfpathclose%
\pgfusepath{fill}%
\end{pgfscope}%
\begin{pgfscope}%
\pgfpathrectangle{\pgfqpoint{0.150000in}{0.150000in}}{\pgfqpoint{2.700000in}{1.950000in}}%
\pgfusepath{clip}%
\pgfsetbuttcap%
\pgfsetroundjoin%
\definecolor{currentfill}{rgb}{0.994301,0.989660,0.990028}%
\pgfsetfillcolor{currentfill}%
\pgfsetlinewidth{0.000000pt}%
\definecolor{currentstroke}{rgb}{0.000000,0.000000,0.000000}%
\pgfsetstrokecolor{currentstroke}%
\pgfsetdash{}{0pt}%
\pgfpathmoveto{\pgfqpoint{1.869985in}{1.018878in}}%
\pgfpathlineto{\pgfqpoint{1.906033in}{1.028325in}}%
\pgfpathlineto{\pgfqpoint{1.869100in}{1.066764in}}%
\pgfpathlineto{\pgfqpoint{1.832949in}{1.057421in}}%
\pgfpathclose%
\pgfusepath{fill}%
\end{pgfscope}%
\begin{pgfscope}%
\pgfpathrectangle{\pgfqpoint{0.150000in}{0.150000in}}{\pgfqpoint{2.700000in}{1.950000in}}%
\pgfusepath{clip}%
\pgfsetbuttcap%
\pgfsetroundjoin%
\definecolor{currentfill}{rgb}{0.978232,0.980913,0.984666}%
\pgfsetfillcolor{currentfill}%
\pgfsetlinewidth{0.000000pt}%
\definecolor{currentstroke}{rgb}{0.000000,0.000000,0.000000}%
\pgfsetstrokecolor{currentstroke}%
\pgfsetdash{}{0pt}%
\pgfpathmoveto{\pgfqpoint{1.832949in}{1.057421in}}%
\pgfpathlineto{\pgfqpoint{1.869100in}{1.066764in}}%
\pgfpathlineto{\pgfqpoint{1.832163in}{1.105208in}}%
\pgfpathlineto{\pgfqpoint{1.795909in}{1.095970in}}%
\pgfpathclose%
\pgfusepath{fill}%
\end{pgfscope}%
\begin{pgfscope}%
\pgfpathrectangle{\pgfqpoint{0.150000in}{0.150000in}}{\pgfqpoint{2.700000in}{1.950000in}}%
\pgfusepath{clip}%
\pgfsetbuttcap%
\pgfsetroundjoin%
\definecolor{currentfill}{rgb}{0.947135,0.953646,0.962760}%
\pgfsetfillcolor{currentfill}%
\pgfsetlinewidth{0.000000pt}%
\definecolor{currentstroke}{rgb}{0.000000,0.000000,0.000000}%
\pgfsetstrokecolor{currentstroke}%
\pgfsetdash{}{0pt}%
\pgfpathmoveto{\pgfqpoint{1.795909in}{1.095970in}}%
\pgfpathlineto{\pgfqpoint{1.832163in}{1.105208in}}%
\pgfpathlineto{\pgfqpoint{1.795221in}{1.143657in}}%
\pgfpathlineto{\pgfqpoint{1.758863in}{1.134523in}}%
\pgfpathclose%
\pgfusepath{fill}%
\end{pgfscope}%
\begin{pgfscope}%
\pgfpathrectangle{\pgfqpoint{0.150000in}{0.150000in}}{\pgfqpoint{2.700000in}{1.950000in}}%
\pgfusepath{clip}%
\pgfsetbuttcap%
\pgfsetroundjoin%
\definecolor{currentfill}{rgb}{0.916039,0.926379,0.940855}%
\pgfsetfillcolor{currentfill}%
\pgfsetlinewidth{0.000000pt}%
\definecolor{currentstroke}{rgb}{0.000000,0.000000,0.000000}%
\pgfsetstrokecolor{currentstroke}%
\pgfsetdash{}{0pt}%
\pgfpathmoveto{\pgfqpoint{1.758863in}{1.134523in}}%
\pgfpathlineto{\pgfqpoint{1.795221in}{1.143657in}}%
\pgfpathlineto{\pgfqpoint{1.758273in}{1.182111in}}%
\pgfpathlineto{\pgfqpoint{1.721813in}{1.173082in}}%
\pgfpathclose%
\pgfusepath{fill}%
\end{pgfscope}%
\begin{pgfscope}%
\pgfpathrectangle{\pgfqpoint{0.150000in}{0.150000in}}{\pgfqpoint{2.700000in}{1.950000in}}%
\pgfusepath{clip}%
\pgfsetbuttcap%
\pgfsetroundjoin%
\definecolor{currentfill}{rgb}{0.884942,0.899112,0.918949}%
\pgfsetfillcolor{currentfill}%
\pgfsetlinewidth{0.000000pt}%
\definecolor{currentstroke}{rgb}{0.000000,0.000000,0.000000}%
\pgfsetstrokecolor{currentstroke}%
\pgfsetdash{}{0pt}%
\pgfpathmoveto{\pgfqpoint{1.721813in}{1.173082in}}%
\pgfpathlineto{\pgfqpoint{1.758273in}{1.182111in}}%
\pgfpathlineto{\pgfqpoint{1.721321in}{1.220571in}}%
\pgfpathlineto{\pgfqpoint{1.684757in}{1.211645in}}%
\pgfpathclose%
\pgfusepath{fill}%
\end{pgfscope}%
\begin{pgfscope}%
\pgfpathrectangle{\pgfqpoint{0.150000in}{0.150000in}}{\pgfqpoint{2.700000in}{1.950000in}}%
\pgfusepath{clip}%
\pgfsetbuttcap%
\pgfsetroundjoin%
\definecolor{currentfill}{rgb}{0.853845,0.871844,0.897044}%
\pgfsetfillcolor{currentfill}%
\pgfsetlinewidth{0.000000pt}%
\definecolor{currentstroke}{rgb}{0.000000,0.000000,0.000000}%
\pgfsetstrokecolor{currentstroke}%
\pgfsetdash{}{0pt}%
\pgfpathmoveto{\pgfqpoint{1.684757in}{1.211645in}}%
\pgfpathlineto{\pgfqpoint{1.721321in}{1.220571in}}%
\pgfpathlineto{\pgfqpoint{1.684364in}{1.259035in}}%
\pgfpathlineto{\pgfqpoint{1.647697in}{1.250214in}}%
\pgfpathclose%
\pgfusepath{fill}%
\end{pgfscope}%
\begin{pgfscope}%
\pgfpathrectangle{\pgfqpoint{0.150000in}{0.150000in}}{\pgfqpoint{2.700000in}{1.950000in}}%
\pgfusepath{clip}%
\pgfsetbuttcap%
\pgfsetroundjoin%
\definecolor{currentfill}{rgb}{0.822748,0.844577,0.875138}%
\pgfsetfillcolor{currentfill}%
\pgfsetlinewidth{0.000000pt}%
\definecolor{currentstroke}{rgb}{0.000000,0.000000,0.000000}%
\pgfsetstrokecolor{currentstroke}%
\pgfsetdash{}{0pt}%
\pgfpathmoveto{\pgfqpoint{1.647697in}{1.250214in}}%
\pgfpathlineto{\pgfqpoint{1.684364in}{1.259035in}}%
\pgfpathlineto{\pgfqpoint{1.647402in}{1.297505in}}%
\pgfpathlineto{\pgfqpoint{1.610632in}{1.288788in}}%
\pgfpathclose%
\pgfusepath{fill}%
\end{pgfscope}%
\begin{pgfscope}%
\pgfpathrectangle{\pgfqpoint{0.150000in}{0.150000in}}{\pgfqpoint{2.700000in}{1.950000in}}%
\pgfusepath{clip}%
\pgfsetbuttcap%
\pgfsetroundjoin%
\definecolor{currentfill}{rgb}{0.797871,0.822763,0.857613}%
\pgfsetfillcolor{currentfill}%
\pgfsetlinewidth{0.000000pt}%
\definecolor{currentstroke}{rgb}{0.000000,0.000000,0.000000}%
\pgfsetstrokecolor{currentstroke}%
\pgfsetdash{}{0pt}%
\pgfpathmoveto{\pgfqpoint{1.610632in}{1.288788in}}%
\pgfpathlineto{\pgfqpoint{1.647402in}{1.297505in}}%
\pgfpathlineto{\pgfqpoint{1.610435in}{1.335979in}}%
\pgfpathlineto{\pgfqpoint{1.573562in}{1.327367in}}%
\pgfpathclose%
\pgfusepath{fill}%
\end{pgfscope}%
\begin{pgfscope}%
\pgfpathrectangle{\pgfqpoint{0.150000in}{0.150000in}}{\pgfqpoint{2.700000in}{1.950000in}}%
\pgfusepath{clip}%
\pgfsetbuttcap%
\pgfsetroundjoin%
\definecolor{currentfill}{rgb}{0.766774,0.795496,0.835708}%
\pgfsetfillcolor{currentfill}%
\pgfsetlinewidth{0.000000pt}%
\definecolor{currentstroke}{rgb}{0.000000,0.000000,0.000000}%
\pgfsetstrokecolor{currentstroke}%
\pgfsetdash{}{0pt}%
\pgfpathmoveto{\pgfqpoint{1.573562in}{1.327367in}}%
\pgfpathlineto{\pgfqpoint{1.610435in}{1.335979in}}%
\pgfpathlineto{\pgfqpoint{1.573463in}{1.374459in}}%
\pgfpathlineto{\pgfqpoint{1.536486in}{1.365951in}}%
\pgfpathclose%
\pgfusepath{fill}%
\end{pgfscope}%
\begin{pgfscope}%
\pgfpathrectangle{\pgfqpoint{0.150000in}{0.150000in}}{\pgfqpoint{2.700000in}{1.950000in}}%
\pgfusepath{clip}%
\pgfsetbuttcap%
\pgfsetroundjoin%
\definecolor{currentfill}{rgb}{0.735677,0.768229,0.813802}%
\pgfsetfillcolor{currentfill}%
\pgfsetlinewidth{0.000000pt}%
\definecolor{currentstroke}{rgb}{0.000000,0.000000,0.000000}%
\pgfsetstrokecolor{currentstroke}%
\pgfsetdash{}{0pt}%
\pgfpathmoveto{\pgfqpoint{1.536486in}{1.365951in}}%
\pgfpathlineto{\pgfqpoint{1.573463in}{1.374459in}}%
\pgfpathlineto{\pgfqpoint{1.536486in}{1.412944in}}%
\pgfpathlineto{\pgfqpoint{1.499406in}{1.404541in}}%
\pgfpathclose%
\pgfusepath{fill}%
\end{pgfscope}%
\begin{pgfscope}%
\pgfpathrectangle{\pgfqpoint{0.150000in}{0.150000in}}{\pgfqpoint{2.700000in}{1.950000in}}%
\pgfusepath{clip}%
\pgfsetbuttcap%
\pgfsetroundjoin%
\definecolor{currentfill}{rgb}{0.704580,0.740962,0.791896}%
\pgfsetfillcolor{currentfill}%
\pgfsetlinewidth{0.000000pt}%
\definecolor{currentstroke}{rgb}{0.000000,0.000000,0.000000}%
\pgfsetstrokecolor{currentstroke}%
\pgfsetdash{}{0pt}%
\pgfpathmoveto{\pgfqpoint{1.499406in}{1.404541in}}%
\pgfpathlineto{\pgfqpoint{1.536486in}{1.412944in}}%
\pgfpathlineto{\pgfqpoint{1.499505in}{1.451434in}}%
\pgfpathlineto{\pgfqpoint{1.462321in}{1.443135in}}%
\pgfpathclose%
\pgfusepath{fill}%
\end{pgfscope}%
\begin{pgfscope}%
\pgfpathrectangle{\pgfqpoint{0.150000in}{0.150000in}}{\pgfqpoint{2.700000in}{1.950000in}}%
\pgfusepath{clip}%
\pgfsetbuttcap%
\pgfsetroundjoin%
\definecolor{currentfill}{rgb}{0.673483,0.713695,0.769991}%
\pgfsetfillcolor{currentfill}%
\pgfsetlinewidth{0.000000pt}%
\definecolor{currentstroke}{rgb}{0.000000,0.000000,0.000000}%
\pgfsetstrokecolor{currentstroke}%
\pgfsetdash{}{0pt}%
\pgfpathmoveto{\pgfqpoint{1.462321in}{1.443135in}}%
\pgfpathlineto{\pgfqpoint{1.499505in}{1.451434in}}%
\pgfpathlineto{\pgfqpoint{1.462518in}{1.489929in}}%
\pgfpathlineto{\pgfqpoint{1.425231in}{1.481735in}}%
\pgfpathclose%
\pgfusepath{fill}%
\end{pgfscope}%
\begin{pgfscope}%
\pgfpathrectangle{\pgfqpoint{0.150000in}{0.150000in}}{\pgfqpoint{2.700000in}{1.950000in}}%
\pgfusepath{clip}%
\pgfsetbuttcap%
\pgfsetroundjoin%
\definecolor{currentfill}{rgb}{0.642387,0.686428,0.748085}%
\pgfsetfillcolor{currentfill}%
\pgfsetlinewidth{0.000000pt}%
\definecolor{currentstroke}{rgb}{0.000000,0.000000,0.000000}%
\pgfsetstrokecolor{currentstroke}%
\pgfsetdash{}{0pt}%
\pgfpathmoveto{\pgfqpoint{1.425231in}{1.481735in}}%
\pgfpathlineto{\pgfqpoint{1.462518in}{1.489929in}}%
\pgfpathlineto{\pgfqpoint{1.425527in}{1.528430in}}%
\pgfpathlineto{\pgfqpoint{1.388136in}{1.520340in}}%
\pgfpathclose%
\pgfusepath{fill}%
\end{pgfscope}%
\begin{pgfscope}%
\pgfpathrectangle{\pgfqpoint{0.150000in}{0.150000in}}{\pgfqpoint{2.700000in}{1.950000in}}%
\pgfusepath{clip}%
\pgfsetbuttcap%
\pgfsetroundjoin%
\definecolor{currentfill}{rgb}{0.611290,0.659161,0.726180}%
\pgfsetfillcolor{currentfill}%
\pgfsetlinewidth{0.000000pt}%
\definecolor{currentstroke}{rgb}{0.000000,0.000000,0.000000}%
\pgfsetstrokecolor{currentstroke}%
\pgfsetdash{}{0pt}%
\pgfpathmoveto{\pgfqpoint{1.388136in}{1.520340in}}%
\pgfpathlineto{\pgfqpoint{1.425527in}{1.528430in}}%
\pgfpathlineto{\pgfqpoint{1.388530in}{1.566935in}}%
\pgfpathlineto{\pgfqpoint{1.351037in}{1.558950in}}%
\pgfpathclose%
\pgfusepath{fill}%
\end{pgfscope}%
\begin{pgfscope}%
\pgfpathrectangle{\pgfqpoint{0.150000in}{0.150000in}}{\pgfqpoint{2.700000in}{1.950000in}}%
\pgfusepath{clip}%
\pgfsetbuttcap%
\pgfsetroundjoin%
\definecolor{currentfill}{rgb}{0.580193,0.631893,0.704274}%
\pgfsetfillcolor{currentfill}%
\pgfsetlinewidth{0.000000pt}%
\definecolor{currentstroke}{rgb}{0.000000,0.000000,0.000000}%
\pgfsetstrokecolor{currentstroke}%
\pgfsetdash{}{0pt}%
\pgfpathmoveto{\pgfqpoint{1.351037in}{1.558950in}}%
\pgfpathlineto{\pgfqpoint{1.388530in}{1.566935in}}%
\pgfpathlineto{\pgfqpoint{1.351529in}{1.605446in}}%
\pgfpathlineto{\pgfqpoint{1.313932in}{1.597565in}}%
\pgfpathclose%
\pgfusepath{fill}%
\end{pgfscope}%
\begin{pgfscope}%
\pgfpathrectangle{\pgfqpoint{0.150000in}{0.150000in}}{\pgfqpoint{2.700000in}{1.950000in}}%
\pgfusepath{clip}%
\pgfsetbuttcap%
\pgfsetroundjoin%
\definecolor{currentfill}{rgb}{0.555316,0.610080,0.686749}%
\pgfsetfillcolor{currentfill}%
\pgfsetlinewidth{0.000000pt}%
\definecolor{currentstroke}{rgb}{0.000000,0.000000,0.000000}%
\pgfsetstrokecolor{currentstroke}%
\pgfsetdash{}{0pt}%
\pgfpathmoveto{\pgfqpoint{1.313932in}{1.597565in}}%
\pgfpathlineto{\pgfqpoint{1.351529in}{1.605446in}}%
\pgfpathlineto{\pgfqpoint{1.314523in}{1.643961in}}%
\pgfpathlineto{\pgfqpoint{1.276822in}{1.636185in}}%
\pgfpathclose%
\pgfusepath{fill}%
\end{pgfscope}%
\begin{pgfscope}%
\pgfpathrectangle{\pgfqpoint{0.150000in}{0.150000in}}{\pgfqpoint{2.700000in}{1.950000in}}%
\pgfusepath{clip}%
\pgfsetbuttcap%
\pgfsetroundjoin%
\definecolor{currentfill}{rgb}{0.524219,0.582812,0.664844}%
\pgfsetfillcolor{currentfill}%
\pgfsetlinewidth{0.000000pt}%
\definecolor{currentstroke}{rgb}{0.000000,0.000000,0.000000}%
\pgfsetstrokecolor{currentstroke}%
\pgfsetdash{}{0pt}%
\pgfpathmoveto{\pgfqpoint{1.276822in}{1.636185in}}%
\pgfpathlineto{\pgfqpoint{1.314523in}{1.643961in}}%
\pgfpathlineto{\pgfqpoint{1.277512in}{1.682482in}}%
\pgfpathlineto{\pgfqpoint{1.239707in}{1.674811in}}%
\pgfpathclose%
\pgfusepath{fill}%
\end{pgfscope}%
\begin{pgfscope}%
\pgfpathrectangle{\pgfqpoint{0.150000in}{0.150000in}}{\pgfqpoint{2.700000in}{1.950000in}}%
\pgfusepath{clip}%
\pgfsetbuttcap%
\pgfsetroundjoin%
\definecolor{currentfill}{rgb}{0.857537,0.741498,0.750689}%
\pgfsetfillcolor{currentfill}%
\pgfsetlinewidth{0.000000pt}%
\definecolor{currentstroke}{rgb}{0.000000,0.000000,0.000000}%
\pgfsetstrokecolor{currentstroke}%
\pgfsetdash{}{0pt}%
\pgfpathmoveto{\pgfqpoint{2.130676in}{0.700383in}}%
\pgfpathlineto{\pgfqpoint{2.166093in}{0.710718in}}%
\pgfpathlineto{\pgfqpoint{2.129097in}{0.749220in}}%
\pgfpathlineto{\pgfqpoint{2.093577in}{0.738990in}}%
\pgfpathclose%
\pgfusepath{fill}%
\end{pgfscope}%
\begin{pgfscope}%
\pgfpathrectangle{\pgfqpoint{0.150000in}{0.150000in}}{\pgfqpoint{2.700000in}{1.950000in}}%
\pgfusepath{clip}%
\pgfsetbuttcap%
\pgfsetroundjoin%
\definecolor{currentfill}{rgb}{0.876532,0.775965,0.783931}%
\pgfsetfillcolor{currentfill}%
\pgfsetlinewidth{0.000000pt}%
\definecolor{currentstroke}{rgb}{0.000000,0.000000,0.000000}%
\pgfsetstrokecolor{currentstroke}%
\pgfsetdash{}{0pt}%
\pgfpathmoveto{\pgfqpoint{2.093577in}{0.738990in}}%
\pgfpathlineto{\pgfqpoint{2.129097in}{0.749220in}}%
\pgfpathlineto{\pgfqpoint{2.092096in}{0.787727in}}%
\pgfpathlineto{\pgfqpoint{2.056472in}{0.777601in}}%
\pgfpathclose%
\pgfusepath{fill}%
\end{pgfscope}%
\begin{pgfscope}%
\pgfpathrectangle{\pgfqpoint{0.150000in}{0.150000in}}{\pgfqpoint{2.700000in}{1.950000in}}%
\pgfusepath{clip}%
\pgfsetbuttcap%
\pgfsetroundjoin%
\definecolor{currentfill}{rgb}{0.891728,0.803539,0.810524}%
\pgfsetfillcolor{currentfill}%
\pgfsetlinewidth{0.000000pt}%
\definecolor{currentstroke}{rgb}{0.000000,0.000000,0.000000}%
\pgfsetstrokecolor{currentstroke}%
\pgfsetdash{}{0pt}%
\pgfpathmoveto{\pgfqpoint{2.056472in}{0.777601in}}%
\pgfpathlineto{\pgfqpoint{2.092096in}{0.787727in}}%
\pgfpathlineto{\pgfqpoint{2.055089in}{0.826240in}}%
\pgfpathlineto{\pgfqpoint{2.019362in}{0.816218in}}%
\pgfpathclose%
\pgfusepath{fill}%
\end{pgfscope}%
\begin{pgfscope}%
\pgfpathrectangle{\pgfqpoint{0.150000in}{0.150000in}}{\pgfqpoint{2.700000in}{1.950000in}}%
\pgfusepath{clip}%
\pgfsetbuttcap%
\pgfsetroundjoin%
\definecolor{currentfill}{rgb}{0.910723,0.838006,0.843765}%
\pgfsetfillcolor{currentfill}%
\pgfsetlinewidth{0.000000pt}%
\definecolor{currentstroke}{rgb}{0.000000,0.000000,0.000000}%
\pgfsetstrokecolor{currentstroke}%
\pgfsetdash{}{0pt}%
\pgfpathmoveto{\pgfqpoint{2.019362in}{0.816218in}}%
\pgfpathlineto{\pgfqpoint{2.055089in}{0.826240in}}%
\pgfpathlineto{\pgfqpoint{2.018078in}{0.864757in}}%
\pgfpathlineto{\pgfqpoint{1.982248in}{0.854840in}}%
\pgfpathclose%
\pgfusepath{fill}%
\end{pgfscope}%
\begin{pgfscope}%
\pgfpathrectangle{\pgfqpoint{0.150000in}{0.150000in}}{\pgfqpoint{2.700000in}{1.950000in}}%
\pgfusepath{clip}%
\pgfsetbuttcap%
\pgfsetroundjoin%
\definecolor{currentfill}{rgb}{0.929718,0.872472,0.877007}%
\pgfsetfillcolor{currentfill}%
\pgfsetlinewidth{0.000000pt}%
\definecolor{currentstroke}{rgb}{0.000000,0.000000,0.000000}%
\pgfsetstrokecolor{currentstroke}%
\pgfsetdash{}{0pt}%
\pgfpathmoveto{\pgfqpoint{1.982248in}{0.854840in}}%
\pgfpathlineto{\pgfqpoint{2.018078in}{0.864757in}}%
\pgfpathlineto{\pgfqpoint{1.981063in}{0.903280in}}%
\pgfpathlineto{\pgfqpoint{1.945128in}{0.893468in}}%
\pgfpathclose%
\pgfusepath{fill}%
\end{pgfscope}%
\begin{pgfscope}%
\pgfpathrectangle{\pgfqpoint{0.150000in}{0.150000in}}{\pgfqpoint{2.700000in}{1.950000in}}%
\pgfusepath{clip}%
\pgfsetbuttcap%
\pgfsetroundjoin%
\definecolor{currentfill}{rgb}{0.948713,0.906939,0.910248}%
\pgfsetfillcolor{currentfill}%
\pgfsetlinewidth{0.000000pt}%
\definecolor{currentstroke}{rgb}{0.000000,0.000000,0.000000}%
\pgfsetstrokecolor{currentstroke}%
\pgfsetdash{}{0pt}%
\pgfpathmoveto{\pgfqpoint{1.945128in}{0.893468in}}%
\pgfpathlineto{\pgfqpoint{1.981063in}{0.903280in}}%
\pgfpathlineto{\pgfqpoint{1.944042in}{0.941807in}}%
\pgfpathlineto{\pgfqpoint{1.908004in}{0.932100in}}%
\pgfpathclose%
\pgfusepath{fill}%
\end{pgfscope}%
\begin{pgfscope}%
\pgfpathrectangle{\pgfqpoint{0.150000in}{0.150000in}}{\pgfqpoint{2.700000in}{1.950000in}}%
\pgfusepath{clip}%
\pgfsetbuttcap%
\pgfsetroundjoin%
\definecolor{currentfill}{rgb}{0.967708,0.941406,0.943490}%
\pgfsetfillcolor{currentfill}%
\pgfsetlinewidth{0.000000pt}%
\definecolor{currentstroke}{rgb}{0.000000,0.000000,0.000000}%
\pgfsetstrokecolor{currentstroke}%
\pgfsetdash{}{0pt}%
\pgfpathmoveto{\pgfqpoint{1.908004in}{0.932100in}}%
\pgfpathlineto{\pgfqpoint{1.944042in}{0.941807in}}%
\pgfpathlineto{\pgfqpoint{1.907016in}{0.980340in}}%
\pgfpathlineto{\pgfqpoint{1.870875in}{0.970738in}}%
\pgfpathclose%
\pgfusepath{fill}%
\end{pgfscope}%
\begin{pgfscope}%
\pgfpathrectangle{\pgfqpoint{0.150000in}{0.150000in}}{\pgfqpoint{2.700000in}{1.950000in}}%
\pgfusepath{clip}%
\pgfsetbuttcap%
\pgfsetroundjoin%
\definecolor{currentfill}{rgb}{0.986703,0.975873,0.976731}%
\pgfsetfillcolor{currentfill}%
\pgfsetlinewidth{0.000000pt}%
\definecolor{currentstroke}{rgb}{0.000000,0.000000,0.000000}%
\pgfsetstrokecolor{currentstroke}%
\pgfsetdash{}{0pt}%
\pgfpathmoveto{\pgfqpoint{1.870875in}{0.970738in}}%
\pgfpathlineto{\pgfqpoint{1.907016in}{0.980340in}}%
\pgfpathlineto{\pgfqpoint{1.869985in}{1.018878in}}%
\pgfpathlineto{\pgfqpoint{1.833740in}{1.009380in}}%
\pgfpathclose%
\pgfusepath{fill}%
\end{pgfscope}%
\begin{pgfscope}%
\pgfpathrectangle{\pgfqpoint{0.150000in}{0.150000in}}{\pgfqpoint{2.700000in}{1.950000in}}%
\pgfusepath{clip}%
\pgfsetbuttcap%
\pgfsetroundjoin%
\definecolor{currentfill}{rgb}{0.990671,0.991820,0.993428}%
\pgfsetfillcolor{currentfill}%
\pgfsetlinewidth{0.000000pt}%
\definecolor{currentstroke}{rgb}{0.000000,0.000000,0.000000}%
\pgfsetstrokecolor{currentstroke}%
\pgfsetdash{}{0pt}%
\pgfpathmoveto{\pgfqpoint{1.833740in}{1.009380in}}%
\pgfpathlineto{\pgfqpoint{1.869985in}{1.018878in}}%
\pgfpathlineto{\pgfqpoint{1.832949in}{1.057421in}}%
\pgfpathlineto{\pgfqpoint{1.796601in}{1.048028in}}%
\pgfpathclose%
\pgfusepath{fill}%
\end{pgfscope}%
\begin{pgfscope}%
\pgfpathrectangle{\pgfqpoint{0.150000in}{0.150000in}}{\pgfqpoint{2.700000in}{1.950000in}}%
\pgfusepath{clip}%
\pgfsetbuttcap%
\pgfsetroundjoin%
\definecolor{currentfill}{rgb}{0.959574,0.964553,0.971523}%
\pgfsetfillcolor{currentfill}%
\pgfsetlinewidth{0.000000pt}%
\definecolor{currentstroke}{rgb}{0.000000,0.000000,0.000000}%
\pgfsetstrokecolor{currentstroke}%
\pgfsetdash{}{0pt}%
\pgfpathmoveto{\pgfqpoint{1.796601in}{1.048028in}}%
\pgfpathlineto{\pgfqpoint{1.832949in}{1.057421in}}%
\pgfpathlineto{\pgfqpoint{1.795909in}{1.095970in}}%
\pgfpathlineto{\pgfqpoint{1.759456in}{1.086681in}}%
\pgfpathclose%
\pgfusepath{fill}%
\end{pgfscope}%
\begin{pgfscope}%
\pgfpathrectangle{\pgfqpoint{0.150000in}{0.150000in}}{\pgfqpoint{2.700000in}{1.950000in}}%
\pgfusepath{clip}%
\pgfsetbuttcap%
\pgfsetroundjoin%
\definecolor{currentfill}{rgb}{0.934697,0.942739,0.953998}%
\pgfsetfillcolor{currentfill}%
\pgfsetlinewidth{0.000000pt}%
\definecolor{currentstroke}{rgb}{0.000000,0.000000,0.000000}%
\pgfsetstrokecolor{currentstroke}%
\pgfsetdash{}{0pt}%
\pgfpathmoveto{\pgfqpoint{1.759456in}{1.086681in}}%
\pgfpathlineto{\pgfqpoint{1.795909in}{1.095970in}}%
\pgfpathlineto{\pgfqpoint{1.758863in}{1.134523in}}%
\pgfpathlineto{\pgfqpoint{1.722307in}{1.125339in}}%
\pgfpathclose%
\pgfusepath{fill}%
\end{pgfscope}%
\begin{pgfscope}%
\pgfpathrectangle{\pgfqpoint{0.150000in}{0.150000in}}{\pgfqpoint{2.700000in}{1.950000in}}%
\pgfusepath{clip}%
\pgfsetbuttcap%
\pgfsetroundjoin%
\definecolor{currentfill}{rgb}{0.903600,0.915472,0.932093}%
\pgfsetfillcolor{currentfill}%
\pgfsetlinewidth{0.000000pt}%
\definecolor{currentstroke}{rgb}{0.000000,0.000000,0.000000}%
\pgfsetstrokecolor{currentstroke}%
\pgfsetdash{}{0pt}%
\pgfpathmoveto{\pgfqpoint{1.722307in}{1.125339in}}%
\pgfpathlineto{\pgfqpoint{1.758863in}{1.134523in}}%
\pgfpathlineto{\pgfqpoint{1.721813in}{1.173082in}}%
\pgfpathlineto{\pgfqpoint{1.685153in}{1.164003in}}%
\pgfpathclose%
\pgfusepath{fill}%
\end{pgfscope}%
\begin{pgfscope}%
\pgfpathrectangle{\pgfqpoint{0.150000in}{0.150000in}}{\pgfqpoint{2.700000in}{1.950000in}}%
\pgfusepath{clip}%
\pgfsetbuttcap%
\pgfsetroundjoin%
\definecolor{currentfill}{rgb}{0.872503,0.888205,0.910187}%
\pgfsetfillcolor{currentfill}%
\pgfsetlinewidth{0.000000pt}%
\definecolor{currentstroke}{rgb}{0.000000,0.000000,0.000000}%
\pgfsetstrokecolor{currentstroke}%
\pgfsetdash{}{0pt}%
\pgfpathmoveto{\pgfqpoint{1.685153in}{1.164003in}}%
\pgfpathlineto{\pgfqpoint{1.721813in}{1.173082in}}%
\pgfpathlineto{\pgfqpoint{1.684757in}{1.211645in}}%
\pgfpathlineto{\pgfqpoint{1.647994in}{1.202671in}}%
\pgfpathclose%
\pgfusepath{fill}%
\end{pgfscope}%
\begin{pgfscope}%
\pgfpathrectangle{\pgfqpoint{0.150000in}{0.150000in}}{\pgfqpoint{2.700000in}{1.950000in}}%
\pgfusepath{clip}%
\pgfsetbuttcap%
\pgfsetroundjoin%
\definecolor{currentfill}{rgb}{0.841406,0.860938,0.888281}%
\pgfsetfillcolor{currentfill}%
\pgfsetlinewidth{0.000000pt}%
\definecolor{currentstroke}{rgb}{0.000000,0.000000,0.000000}%
\pgfsetstrokecolor{currentstroke}%
\pgfsetdash{}{0pt}%
\pgfpathmoveto{\pgfqpoint{1.647994in}{1.202671in}}%
\pgfpathlineto{\pgfqpoint{1.684757in}{1.211645in}}%
\pgfpathlineto{\pgfqpoint{1.647697in}{1.250214in}}%
\pgfpathlineto{\pgfqpoint{1.610830in}{1.241345in}}%
\pgfpathclose%
\pgfusepath{fill}%
\end{pgfscope}%
\begin{pgfscope}%
\pgfpathrectangle{\pgfqpoint{0.150000in}{0.150000in}}{\pgfqpoint{2.700000in}{1.950000in}}%
\pgfusepath{clip}%
\pgfsetbuttcap%
\pgfsetroundjoin%
\definecolor{currentfill}{rgb}{0.810309,0.833670,0.866376}%
\pgfsetfillcolor{currentfill}%
\pgfsetlinewidth{0.000000pt}%
\definecolor{currentstroke}{rgb}{0.000000,0.000000,0.000000}%
\pgfsetstrokecolor{currentstroke}%
\pgfsetdash{}{0pt}%
\pgfpathmoveto{\pgfqpoint{1.610830in}{1.241345in}}%
\pgfpathlineto{\pgfqpoint{1.647697in}{1.250214in}}%
\pgfpathlineto{\pgfqpoint{1.610632in}{1.288788in}}%
\pgfpathlineto{\pgfqpoint{1.573661in}{1.280024in}}%
\pgfpathclose%
\pgfusepath{fill}%
\end{pgfscope}%
\begin{pgfscope}%
\pgfpathrectangle{\pgfqpoint{0.150000in}{0.150000in}}{\pgfqpoint{2.700000in}{1.950000in}}%
\pgfusepath{clip}%
\pgfsetbuttcap%
\pgfsetroundjoin%
\definecolor{currentfill}{rgb}{0.779213,0.806403,0.844470}%
\pgfsetfillcolor{currentfill}%
\pgfsetlinewidth{0.000000pt}%
\definecolor{currentstroke}{rgb}{0.000000,0.000000,0.000000}%
\pgfsetstrokecolor{currentstroke}%
\pgfsetdash{}{0pt}%
\pgfpathmoveto{\pgfqpoint{1.573661in}{1.280024in}}%
\pgfpathlineto{\pgfqpoint{1.610632in}{1.288788in}}%
\pgfpathlineto{\pgfqpoint{1.573562in}{1.327367in}}%
\pgfpathlineto{\pgfqpoint{1.536486in}{1.318708in}}%
\pgfpathclose%
\pgfusepath{fill}%
\end{pgfscope}%
\begin{pgfscope}%
\pgfpathrectangle{\pgfqpoint{0.150000in}{0.150000in}}{\pgfqpoint{2.700000in}{1.950000in}}%
\pgfusepath{clip}%
\pgfsetbuttcap%
\pgfsetroundjoin%
\definecolor{currentfill}{rgb}{0.748116,0.779136,0.822564}%
\pgfsetfillcolor{currentfill}%
\pgfsetlinewidth{0.000000pt}%
\definecolor{currentstroke}{rgb}{0.000000,0.000000,0.000000}%
\pgfsetstrokecolor{currentstroke}%
\pgfsetdash{}{0pt}%
\pgfpathmoveto{\pgfqpoint{1.536486in}{1.318708in}}%
\pgfpathlineto{\pgfqpoint{1.573562in}{1.327367in}}%
\pgfpathlineto{\pgfqpoint{1.536486in}{1.365951in}}%
\pgfpathlineto{\pgfqpoint{1.499307in}{1.357397in}}%
\pgfpathclose%
\pgfusepath{fill}%
\end{pgfscope}%
\begin{pgfscope}%
\pgfpathrectangle{\pgfqpoint{0.150000in}{0.150000in}}{\pgfqpoint{2.700000in}{1.950000in}}%
\pgfusepath{clip}%
\pgfsetbuttcap%
\pgfsetroundjoin%
\definecolor{currentfill}{rgb}{0.717019,0.751869,0.800659}%
\pgfsetfillcolor{currentfill}%
\pgfsetlinewidth{0.000000pt}%
\definecolor{currentstroke}{rgb}{0.000000,0.000000,0.000000}%
\pgfsetstrokecolor{currentstroke}%
\pgfsetdash{}{0pt}%
\pgfpathmoveto{\pgfqpoint{1.499307in}{1.357397in}}%
\pgfpathlineto{\pgfqpoint{1.536486in}{1.365951in}}%
\pgfpathlineto{\pgfqpoint{1.499406in}{1.404541in}}%
\pgfpathlineto{\pgfqpoint{1.462123in}{1.396092in}}%
\pgfpathclose%
\pgfusepath{fill}%
\end{pgfscope}%
\begin{pgfscope}%
\pgfpathrectangle{\pgfqpoint{0.150000in}{0.150000in}}{\pgfqpoint{2.700000in}{1.950000in}}%
\pgfusepath{clip}%
\pgfsetbuttcap%
\pgfsetroundjoin%
\definecolor{currentfill}{rgb}{0.692142,0.730055,0.783134}%
\pgfsetfillcolor{currentfill}%
\pgfsetlinewidth{0.000000pt}%
\definecolor{currentstroke}{rgb}{0.000000,0.000000,0.000000}%
\pgfsetstrokecolor{currentstroke}%
\pgfsetdash{}{0pt}%
\pgfpathmoveto{\pgfqpoint{1.462123in}{1.396092in}}%
\pgfpathlineto{\pgfqpoint{1.499406in}{1.404541in}}%
\pgfpathlineto{\pgfqpoint{1.462321in}{1.443135in}}%
\pgfpathlineto{\pgfqpoint{1.424934in}{1.434791in}}%
\pgfpathclose%
\pgfusepath{fill}%
\end{pgfscope}%
\begin{pgfscope}%
\pgfpathrectangle{\pgfqpoint{0.150000in}{0.150000in}}{\pgfqpoint{2.700000in}{1.950000in}}%
\pgfusepath{clip}%
\pgfsetbuttcap%
\pgfsetroundjoin%
\definecolor{currentfill}{rgb}{0.661045,0.702788,0.761229}%
\pgfsetfillcolor{currentfill}%
\pgfsetlinewidth{0.000000pt}%
\definecolor{currentstroke}{rgb}{0.000000,0.000000,0.000000}%
\pgfsetstrokecolor{currentstroke}%
\pgfsetdash{}{0pt}%
\pgfpathmoveto{\pgfqpoint{1.424934in}{1.434791in}}%
\pgfpathlineto{\pgfqpoint{1.462321in}{1.443135in}}%
\pgfpathlineto{\pgfqpoint{1.425231in}{1.481735in}}%
\pgfpathlineto{\pgfqpoint{1.387740in}{1.473496in}}%
\pgfpathclose%
\pgfusepath{fill}%
\end{pgfscope}%
\begin{pgfscope}%
\pgfpathrectangle{\pgfqpoint{0.150000in}{0.150000in}}{\pgfqpoint{2.700000in}{1.950000in}}%
\pgfusepath{clip}%
\pgfsetbuttcap%
\pgfsetroundjoin%
\definecolor{currentfill}{rgb}{0.629948,0.675521,0.739323}%
\pgfsetfillcolor{currentfill}%
\pgfsetlinewidth{0.000000pt}%
\definecolor{currentstroke}{rgb}{0.000000,0.000000,0.000000}%
\pgfsetstrokecolor{currentstroke}%
\pgfsetdash{}{0pt}%
\pgfpathmoveto{\pgfqpoint{1.387740in}{1.473496in}}%
\pgfpathlineto{\pgfqpoint{1.425231in}{1.481735in}}%
\pgfpathlineto{\pgfqpoint{1.388136in}{1.520340in}}%
\pgfpathlineto{\pgfqpoint{1.350542in}{1.512206in}}%
\pgfpathclose%
\pgfusepath{fill}%
\end{pgfscope}%
\begin{pgfscope}%
\pgfpathrectangle{\pgfqpoint{0.150000in}{0.150000in}}{\pgfqpoint{2.700000in}{1.950000in}}%
\pgfusepath{clip}%
\pgfsetbuttcap%
\pgfsetroundjoin%
\definecolor{currentfill}{rgb}{0.598851,0.648254,0.717417}%
\pgfsetfillcolor{currentfill}%
\pgfsetlinewidth{0.000000pt}%
\definecolor{currentstroke}{rgb}{0.000000,0.000000,0.000000}%
\pgfsetstrokecolor{currentstroke}%
\pgfsetdash{}{0pt}%
\pgfpathmoveto{\pgfqpoint{1.350542in}{1.512206in}}%
\pgfpathlineto{\pgfqpoint{1.388136in}{1.520340in}}%
\pgfpathlineto{\pgfqpoint{1.351037in}{1.558950in}}%
\pgfpathlineto{\pgfqpoint{1.313338in}{1.550921in}}%
\pgfpathclose%
\pgfusepath{fill}%
\end{pgfscope}%
\begin{pgfscope}%
\pgfpathrectangle{\pgfqpoint{0.150000in}{0.150000in}}{\pgfqpoint{2.700000in}{1.950000in}}%
\pgfusepath{clip}%
\pgfsetbuttcap%
\pgfsetroundjoin%
\definecolor{currentfill}{rgb}{0.567754,0.620987,0.695512}%
\pgfsetfillcolor{currentfill}%
\pgfsetlinewidth{0.000000pt}%
\definecolor{currentstroke}{rgb}{0.000000,0.000000,0.000000}%
\pgfsetstrokecolor{currentstroke}%
\pgfsetdash{}{0pt}%
\pgfpathmoveto{\pgfqpoint{1.313338in}{1.550921in}}%
\pgfpathlineto{\pgfqpoint{1.351037in}{1.558950in}}%
\pgfpathlineto{\pgfqpoint{1.313932in}{1.597565in}}%
\pgfpathlineto{\pgfqpoint{1.276129in}{1.589641in}}%
\pgfpathclose%
\pgfusepath{fill}%
\end{pgfscope}%
\begin{pgfscope}%
\pgfpathrectangle{\pgfqpoint{0.150000in}{0.150000in}}{\pgfqpoint{2.700000in}{1.950000in}}%
\pgfusepath{clip}%
\pgfsetbuttcap%
\pgfsetroundjoin%
\definecolor{currentfill}{rgb}{0.536657,0.593719,0.673606}%
\pgfsetfillcolor{currentfill}%
\pgfsetlinewidth{0.000000pt}%
\definecolor{currentstroke}{rgb}{0.000000,0.000000,0.000000}%
\pgfsetstrokecolor{currentstroke}%
\pgfsetdash{}{0pt}%
\pgfpathmoveto{\pgfqpoint{1.276129in}{1.589641in}}%
\pgfpathlineto{\pgfqpoint{1.313932in}{1.597565in}}%
\pgfpathlineto{\pgfqpoint{1.276822in}{1.636185in}}%
\pgfpathlineto{\pgfqpoint{1.238915in}{1.628366in}}%
\pgfpathclose%
\pgfusepath{fill}%
\end{pgfscope}%
\begin{pgfscope}%
\pgfpathrectangle{\pgfqpoint{0.150000in}{0.150000in}}{\pgfqpoint{2.700000in}{1.950000in}}%
\pgfusepath{clip}%
\pgfsetbuttcap%
\pgfsetroundjoin%
\definecolor{currentfill}{rgb}{0.505561,0.566452,0.651700}%
\pgfsetfillcolor{currentfill}%
\pgfsetlinewidth{0.000000pt}%
\definecolor{currentstroke}{rgb}{0.000000,0.000000,0.000000}%
\pgfsetstrokecolor{currentstroke}%
\pgfsetdash{}{0pt}%
\pgfpathmoveto{\pgfqpoint{1.238915in}{1.628366in}}%
\pgfpathlineto{\pgfqpoint{1.276822in}{1.636185in}}%
\pgfpathlineto{\pgfqpoint{1.239707in}{1.674811in}}%
\pgfpathlineto{\pgfqpoint{1.201696in}{1.667097in}}%
\pgfpathclose%
\pgfusepath{fill}%
\end{pgfscope}%
\begin{pgfscope}%
\pgfpathrectangle{\pgfqpoint{0.150000in}{0.150000in}}{\pgfqpoint{2.700000in}{1.950000in}}%
\pgfusepath{clip}%
\pgfsetbuttcap%
\pgfsetroundjoin%
\definecolor{currentfill}{rgb}{0.865135,0.755285,0.763986}%
\pgfsetfillcolor{currentfill}%
\pgfsetlinewidth{0.000000pt}%
\definecolor{currentstroke}{rgb}{0.000000,0.000000,0.000000}%
\pgfsetstrokecolor{currentstroke}%
\pgfsetdash{}{0pt}%
\pgfpathmoveto{\pgfqpoint{2.095066in}{0.689991in}}%
\pgfpathlineto{\pgfqpoint{2.130676in}{0.700383in}}%
\pgfpathlineto{\pgfqpoint{2.093577in}{0.738990in}}%
\pgfpathlineto{\pgfqpoint{2.057862in}{0.728703in}}%
\pgfpathclose%
\pgfusepath{fill}%
\end{pgfscope}%
\begin{pgfscope}%
\pgfpathrectangle{\pgfqpoint{0.150000in}{0.150000in}}{\pgfqpoint{2.700000in}{1.950000in}}%
\pgfusepath{clip}%
\pgfsetbuttcap%
\pgfsetroundjoin%
\definecolor{currentfill}{rgb}{0.884130,0.789752,0.797227}%
\pgfsetfillcolor{currentfill}%
\pgfsetlinewidth{0.000000pt}%
\definecolor{currentstroke}{rgb}{0.000000,0.000000,0.000000}%
\pgfsetstrokecolor{currentstroke}%
\pgfsetdash{}{0pt}%
\pgfpathmoveto{\pgfqpoint{2.057862in}{0.728703in}}%
\pgfpathlineto{\pgfqpoint{2.093577in}{0.738990in}}%
\pgfpathlineto{\pgfqpoint{2.056472in}{0.777601in}}%
\pgfpathlineto{\pgfqpoint{2.020653in}{0.767420in}}%
\pgfpathclose%
\pgfusepath{fill}%
\end{pgfscope}%
\begin{pgfscope}%
\pgfpathrectangle{\pgfqpoint{0.150000in}{0.150000in}}{\pgfqpoint{2.700000in}{1.950000in}}%
\pgfusepath{clip}%
\pgfsetbuttcap%
\pgfsetroundjoin%
\definecolor{currentfill}{rgb}{0.903125,0.824219,0.830469}%
\pgfsetfillcolor{currentfill}%
\pgfsetlinewidth{0.000000pt}%
\definecolor{currentstroke}{rgb}{0.000000,0.000000,0.000000}%
\pgfsetstrokecolor{currentstroke}%
\pgfsetdash{}{0pt}%
\pgfpathmoveto{\pgfqpoint{2.020653in}{0.767420in}}%
\pgfpathlineto{\pgfqpoint{2.056472in}{0.777601in}}%
\pgfpathlineto{\pgfqpoint{2.019362in}{0.816218in}}%
\pgfpathlineto{\pgfqpoint{1.983440in}{0.806142in}}%
\pgfpathclose%
\pgfusepath{fill}%
\end{pgfscope}%
\begin{pgfscope}%
\pgfpathrectangle{\pgfqpoint{0.150000in}{0.150000in}}{\pgfqpoint{2.700000in}{1.950000in}}%
\pgfusepath{clip}%
\pgfsetbuttcap%
\pgfsetroundjoin%
\definecolor{currentfill}{rgb}{0.922120,0.858686,0.863710}%
\pgfsetfillcolor{currentfill}%
\pgfsetlinewidth{0.000000pt}%
\definecolor{currentstroke}{rgb}{0.000000,0.000000,0.000000}%
\pgfsetstrokecolor{currentstroke}%
\pgfsetdash{}{0pt}%
\pgfpathmoveto{\pgfqpoint{1.983440in}{0.806142in}}%
\pgfpathlineto{\pgfqpoint{2.019362in}{0.816218in}}%
\pgfpathlineto{\pgfqpoint{1.982248in}{0.854840in}}%
\pgfpathlineto{\pgfqpoint{1.946221in}{0.844869in}}%
\pgfpathclose%
\pgfusepath{fill}%
\end{pgfscope}%
\begin{pgfscope}%
\pgfpathrectangle{\pgfqpoint{0.150000in}{0.150000in}}{\pgfqpoint{2.700000in}{1.950000in}}%
\pgfusepath{clip}%
\pgfsetbuttcap%
\pgfsetroundjoin%
\definecolor{currentfill}{rgb}{0.941115,0.893153,0.896952}%
\pgfsetfillcolor{currentfill}%
\pgfsetlinewidth{0.000000pt}%
\definecolor{currentstroke}{rgb}{0.000000,0.000000,0.000000}%
\pgfsetstrokecolor{currentstroke}%
\pgfsetdash{}{0pt}%
\pgfpathmoveto{\pgfqpoint{1.946221in}{0.844869in}}%
\pgfpathlineto{\pgfqpoint{1.982248in}{0.854840in}}%
\pgfpathlineto{\pgfqpoint{1.945128in}{0.893468in}}%
\pgfpathlineto{\pgfqpoint{1.908997in}{0.883602in}}%
\pgfpathclose%
\pgfusepath{fill}%
\end{pgfscope}%
\begin{pgfscope}%
\pgfpathrectangle{\pgfqpoint{0.150000in}{0.150000in}}{\pgfqpoint{2.700000in}{1.950000in}}%
\pgfusepath{clip}%
\pgfsetbuttcap%
\pgfsetroundjoin%
\definecolor{currentfill}{rgb}{0.960110,0.927619,0.930193}%
\pgfsetfillcolor{currentfill}%
\pgfsetlinewidth{0.000000pt}%
\definecolor{currentstroke}{rgb}{0.000000,0.000000,0.000000}%
\pgfsetstrokecolor{currentstroke}%
\pgfsetdash{}{0pt}%
\pgfpathmoveto{\pgfqpoint{1.908997in}{0.883602in}}%
\pgfpathlineto{\pgfqpoint{1.945128in}{0.893468in}}%
\pgfpathlineto{\pgfqpoint{1.908004in}{0.932100in}}%
\pgfpathlineto{\pgfqpoint{1.871769in}{0.922339in}}%
\pgfpathclose%
\pgfusepath{fill}%
\end{pgfscope}%
\begin{pgfscope}%
\pgfpathrectangle{\pgfqpoint{0.150000in}{0.150000in}}{\pgfqpoint{2.700000in}{1.950000in}}%
\pgfusepath{clip}%
\pgfsetbuttcap%
\pgfsetroundjoin%
\definecolor{currentfill}{rgb}{0.975306,0.955193,0.956786}%
\pgfsetfillcolor{currentfill}%
\pgfsetlinewidth{0.000000pt}%
\definecolor{currentstroke}{rgb}{0.000000,0.000000,0.000000}%
\pgfsetstrokecolor{currentstroke}%
\pgfsetdash{}{0pt}%
\pgfpathmoveto{\pgfqpoint{1.871769in}{0.922339in}}%
\pgfpathlineto{\pgfqpoint{1.908004in}{0.932100in}}%
\pgfpathlineto{\pgfqpoint{1.870875in}{0.970738in}}%
\pgfpathlineto{\pgfqpoint{1.834535in}{0.961082in}}%
\pgfpathclose%
\pgfusepath{fill}%
\end{pgfscope}%
\begin{pgfscope}%
\pgfpathrectangle{\pgfqpoint{0.150000in}{0.150000in}}{\pgfqpoint{2.700000in}{1.950000in}}%
\pgfusepath{clip}%
\pgfsetbuttcap%
\pgfsetroundjoin%
\definecolor{currentfill}{rgb}{0.994301,0.989660,0.990028}%
\pgfsetfillcolor{currentfill}%
\pgfsetlinewidth{0.000000pt}%
\definecolor{currentstroke}{rgb}{0.000000,0.000000,0.000000}%
\pgfsetstrokecolor{currentstroke}%
\pgfsetdash{}{0pt}%
\pgfpathmoveto{\pgfqpoint{1.834535in}{0.961082in}}%
\pgfpathlineto{\pgfqpoint{1.870875in}{0.970738in}}%
\pgfpathlineto{\pgfqpoint{1.833740in}{1.009380in}}%
\pgfpathlineto{\pgfqpoint{1.797296in}{0.999830in}}%
\pgfpathclose%
\pgfusepath{fill}%
\end{pgfscope}%
\begin{pgfscope}%
\pgfpathrectangle{\pgfqpoint{0.150000in}{0.150000in}}{\pgfqpoint{2.700000in}{1.950000in}}%
\pgfusepath{clip}%
\pgfsetbuttcap%
\pgfsetroundjoin%
\definecolor{currentfill}{rgb}{0.978232,0.980913,0.984666}%
\pgfsetfillcolor{currentfill}%
\pgfsetlinewidth{0.000000pt}%
\definecolor{currentstroke}{rgb}{0.000000,0.000000,0.000000}%
\pgfsetstrokecolor{currentstroke}%
\pgfsetdash{}{0pt}%
\pgfpathmoveto{\pgfqpoint{1.797296in}{0.999830in}}%
\pgfpathlineto{\pgfqpoint{1.833740in}{1.009380in}}%
\pgfpathlineto{\pgfqpoint{1.796601in}{1.048028in}}%
\pgfpathlineto{\pgfqpoint{1.760053in}{1.038583in}}%
\pgfpathclose%
\pgfusepath{fill}%
\end{pgfscope}%
\begin{pgfscope}%
\pgfpathrectangle{\pgfqpoint{0.150000in}{0.150000in}}{\pgfqpoint{2.700000in}{1.950000in}}%
\pgfusepath{clip}%
\pgfsetbuttcap%
\pgfsetroundjoin%
\definecolor{currentfill}{rgb}{0.947135,0.953646,0.962760}%
\pgfsetfillcolor{currentfill}%
\pgfsetlinewidth{0.000000pt}%
\definecolor{currentstroke}{rgb}{0.000000,0.000000,0.000000}%
\pgfsetstrokecolor{currentstroke}%
\pgfsetdash{}{0pt}%
\pgfpathmoveto{\pgfqpoint{1.760053in}{1.038583in}}%
\pgfpathlineto{\pgfqpoint{1.796601in}{1.048028in}}%
\pgfpathlineto{\pgfqpoint{1.759456in}{1.086681in}}%
\pgfpathlineto{\pgfqpoint{1.722804in}{1.077342in}}%
\pgfpathclose%
\pgfusepath{fill}%
\end{pgfscope}%
\begin{pgfscope}%
\pgfpathrectangle{\pgfqpoint{0.150000in}{0.150000in}}{\pgfqpoint{2.700000in}{1.950000in}}%
\pgfusepath{clip}%
\pgfsetbuttcap%
\pgfsetroundjoin%
\definecolor{currentfill}{rgb}{0.916039,0.926379,0.940855}%
\pgfsetfillcolor{currentfill}%
\pgfsetlinewidth{0.000000pt}%
\definecolor{currentstroke}{rgb}{0.000000,0.000000,0.000000}%
\pgfsetstrokecolor{currentstroke}%
\pgfsetdash{}{0pt}%
\pgfpathmoveto{\pgfqpoint{1.722804in}{1.077342in}}%
\pgfpathlineto{\pgfqpoint{1.759456in}{1.086681in}}%
\pgfpathlineto{\pgfqpoint{1.722307in}{1.125339in}}%
\pgfpathlineto{\pgfqpoint{1.685551in}{1.116105in}}%
\pgfpathclose%
\pgfusepath{fill}%
\end{pgfscope}%
\begin{pgfscope}%
\pgfpathrectangle{\pgfqpoint{0.150000in}{0.150000in}}{\pgfqpoint{2.700000in}{1.950000in}}%
\pgfusepath{clip}%
\pgfsetbuttcap%
\pgfsetroundjoin%
\definecolor{currentfill}{rgb}{0.884942,0.899112,0.918949}%
\pgfsetfillcolor{currentfill}%
\pgfsetlinewidth{0.000000pt}%
\definecolor{currentstroke}{rgb}{0.000000,0.000000,0.000000}%
\pgfsetstrokecolor{currentstroke}%
\pgfsetdash{}{0pt}%
\pgfpathmoveto{\pgfqpoint{1.685551in}{1.116105in}}%
\pgfpathlineto{\pgfqpoint{1.722307in}{1.125339in}}%
\pgfpathlineto{\pgfqpoint{1.685153in}{1.164003in}}%
\pgfpathlineto{\pgfqpoint{1.648292in}{1.154874in}}%
\pgfpathclose%
\pgfusepath{fill}%
\end{pgfscope}%
\begin{pgfscope}%
\pgfpathrectangle{\pgfqpoint{0.150000in}{0.150000in}}{\pgfqpoint{2.700000in}{1.950000in}}%
\pgfusepath{clip}%
\pgfsetbuttcap%
\pgfsetroundjoin%
\definecolor{currentfill}{rgb}{0.853845,0.871844,0.897044}%
\pgfsetfillcolor{currentfill}%
\pgfsetlinewidth{0.000000pt}%
\definecolor{currentstroke}{rgb}{0.000000,0.000000,0.000000}%
\pgfsetstrokecolor{currentstroke}%
\pgfsetdash{}{0pt}%
\pgfpathmoveto{\pgfqpoint{1.648292in}{1.154874in}}%
\pgfpathlineto{\pgfqpoint{1.685153in}{1.164003in}}%
\pgfpathlineto{\pgfqpoint{1.647994in}{1.202671in}}%
\pgfpathlineto{\pgfqpoint{1.611029in}{1.193648in}}%
\pgfpathclose%
\pgfusepath{fill}%
\end{pgfscope}%
\begin{pgfscope}%
\pgfpathrectangle{\pgfqpoint{0.150000in}{0.150000in}}{\pgfqpoint{2.700000in}{1.950000in}}%
\pgfusepath{clip}%
\pgfsetbuttcap%
\pgfsetroundjoin%
\definecolor{currentfill}{rgb}{0.822748,0.844577,0.875138}%
\pgfsetfillcolor{currentfill}%
\pgfsetlinewidth{0.000000pt}%
\definecolor{currentstroke}{rgb}{0.000000,0.000000,0.000000}%
\pgfsetstrokecolor{currentstroke}%
\pgfsetdash{}{0pt}%
\pgfpathmoveto{\pgfqpoint{1.611029in}{1.193648in}}%
\pgfpathlineto{\pgfqpoint{1.647994in}{1.202671in}}%
\pgfpathlineto{\pgfqpoint{1.610830in}{1.241345in}}%
\pgfpathlineto{\pgfqpoint{1.573760in}{1.232427in}}%
\pgfpathclose%
\pgfusepath{fill}%
\end{pgfscope}%
\begin{pgfscope}%
\pgfpathrectangle{\pgfqpoint{0.150000in}{0.150000in}}{\pgfqpoint{2.700000in}{1.950000in}}%
\pgfusepath{clip}%
\pgfsetbuttcap%
\pgfsetroundjoin%
\definecolor{currentfill}{rgb}{0.797871,0.822763,0.857613}%
\pgfsetfillcolor{currentfill}%
\pgfsetlinewidth{0.000000pt}%
\definecolor{currentstroke}{rgb}{0.000000,0.000000,0.000000}%
\pgfsetstrokecolor{currentstroke}%
\pgfsetdash{}{0pt}%
\pgfpathmoveto{\pgfqpoint{1.573760in}{1.232427in}}%
\pgfpathlineto{\pgfqpoint{1.610830in}{1.241345in}}%
\pgfpathlineto{\pgfqpoint{1.573661in}{1.280024in}}%
\pgfpathlineto{\pgfqpoint{1.536486in}{1.271212in}}%
\pgfpathclose%
\pgfusepath{fill}%
\end{pgfscope}%
\begin{pgfscope}%
\pgfpathrectangle{\pgfqpoint{0.150000in}{0.150000in}}{\pgfqpoint{2.700000in}{1.950000in}}%
\pgfusepath{clip}%
\pgfsetbuttcap%
\pgfsetroundjoin%
\definecolor{currentfill}{rgb}{0.766774,0.795496,0.835708}%
\pgfsetfillcolor{currentfill}%
\pgfsetlinewidth{0.000000pt}%
\definecolor{currentstroke}{rgb}{0.000000,0.000000,0.000000}%
\pgfsetstrokecolor{currentstroke}%
\pgfsetdash{}{0pt}%
\pgfpathmoveto{\pgfqpoint{1.536486in}{1.271212in}}%
\pgfpathlineto{\pgfqpoint{1.573661in}{1.280024in}}%
\pgfpathlineto{\pgfqpoint{1.536486in}{1.318708in}}%
\pgfpathlineto{\pgfqpoint{1.499208in}{1.310001in}}%
\pgfpathclose%
\pgfusepath{fill}%
\end{pgfscope}%
\begin{pgfscope}%
\pgfpathrectangle{\pgfqpoint{0.150000in}{0.150000in}}{\pgfqpoint{2.700000in}{1.950000in}}%
\pgfusepath{clip}%
\pgfsetbuttcap%
\pgfsetroundjoin%
\definecolor{currentfill}{rgb}{0.735677,0.768229,0.813802}%
\pgfsetfillcolor{currentfill}%
\pgfsetlinewidth{0.000000pt}%
\definecolor{currentstroke}{rgb}{0.000000,0.000000,0.000000}%
\pgfsetstrokecolor{currentstroke}%
\pgfsetdash{}{0pt}%
\pgfpathmoveto{\pgfqpoint{1.499208in}{1.310001in}}%
\pgfpathlineto{\pgfqpoint{1.536486in}{1.318708in}}%
\pgfpathlineto{\pgfqpoint{1.499307in}{1.357397in}}%
\pgfpathlineto{\pgfqpoint{1.461924in}{1.348796in}}%
\pgfpathclose%
\pgfusepath{fill}%
\end{pgfscope}%
\begin{pgfscope}%
\pgfpathrectangle{\pgfqpoint{0.150000in}{0.150000in}}{\pgfqpoint{2.700000in}{1.950000in}}%
\pgfusepath{clip}%
\pgfsetbuttcap%
\pgfsetroundjoin%
\definecolor{currentfill}{rgb}{0.704580,0.740962,0.791896}%
\pgfsetfillcolor{currentfill}%
\pgfsetlinewidth{0.000000pt}%
\definecolor{currentstroke}{rgb}{0.000000,0.000000,0.000000}%
\pgfsetstrokecolor{currentstroke}%
\pgfsetdash{}{0pt}%
\pgfpathmoveto{\pgfqpoint{1.461924in}{1.348796in}}%
\pgfpathlineto{\pgfqpoint{1.499307in}{1.357397in}}%
\pgfpathlineto{\pgfqpoint{1.462123in}{1.396092in}}%
\pgfpathlineto{\pgfqpoint{1.424636in}{1.387596in}}%
\pgfpathclose%
\pgfusepath{fill}%
\end{pgfscope}%
\begin{pgfscope}%
\pgfpathrectangle{\pgfqpoint{0.150000in}{0.150000in}}{\pgfqpoint{2.700000in}{1.950000in}}%
\pgfusepath{clip}%
\pgfsetbuttcap%
\pgfsetroundjoin%
\definecolor{currentfill}{rgb}{0.673483,0.713695,0.769991}%
\pgfsetfillcolor{currentfill}%
\pgfsetlinewidth{0.000000pt}%
\definecolor{currentstroke}{rgb}{0.000000,0.000000,0.000000}%
\pgfsetstrokecolor{currentstroke}%
\pgfsetdash{}{0pt}%
\pgfpathmoveto{\pgfqpoint{1.424636in}{1.387596in}}%
\pgfpathlineto{\pgfqpoint{1.462123in}{1.396092in}}%
\pgfpathlineto{\pgfqpoint{1.424934in}{1.434791in}}%
\pgfpathlineto{\pgfqpoint{1.387342in}{1.426401in}}%
\pgfpathclose%
\pgfusepath{fill}%
\end{pgfscope}%
\begin{pgfscope}%
\pgfpathrectangle{\pgfqpoint{0.150000in}{0.150000in}}{\pgfqpoint{2.700000in}{1.950000in}}%
\pgfusepath{clip}%
\pgfsetbuttcap%
\pgfsetroundjoin%
\definecolor{currentfill}{rgb}{0.642387,0.686428,0.748085}%
\pgfsetfillcolor{currentfill}%
\pgfsetlinewidth{0.000000pt}%
\definecolor{currentstroke}{rgb}{0.000000,0.000000,0.000000}%
\pgfsetstrokecolor{currentstroke}%
\pgfsetdash{}{0pt}%
\pgfpathmoveto{\pgfqpoint{1.387342in}{1.426401in}}%
\pgfpathlineto{\pgfqpoint{1.424934in}{1.434791in}}%
\pgfpathlineto{\pgfqpoint{1.387740in}{1.473496in}}%
\pgfpathlineto{\pgfqpoint{1.350044in}{1.465211in}}%
\pgfpathclose%
\pgfusepath{fill}%
\end{pgfscope}%
\begin{pgfscope}%
\pgfpathrectangle{\pgfqpoint{0.150000in}{0.150000in}}{\pgfqpoint{2.700000in}{1.950000in}}%
\pgfusepath{clip}%
\pgfsetbuttcap%
\pgfsetroundjoin%
\definecolor{currentfill}{rgb}{0.611290,0.659161,0.726180}%
\pgfsetfillcolor{currentfill}%
\pgfsetlinewidth{0.000000pt}%
\definecolor{currentstroke}{rgb}{0.000000,0.000000,0.000000}%
\pgfsetstrokecolor{currentstroke}%
\pgfsetdash{}{0pt}%
\pgfpathmoveto{\pgfqpoint{1.350044in}{1.465211in}}%
\pgfpathlineto{\pgfqpoint{1.387740in}{1.473496in}}%
\pgfpathlineto{\pgfqpoint{1.350542in}{1.512206in}}%
\pgfpathlineto{\pgfqpoint{1.312740in}{1.504027in}}%
\pgfpathclose%
\pgfusepath{fill}%
\end{pgfscope}%
\begin{pgfscope}%
\pgfpathrectangle{\pgfqpoint{0.150000in}{0.150000in}}{\pgfqpoint{2.700000in}{1.950000in}}%
\pgfusepath{clip}%
\pgfsetbuttcap%
\pgfsetroundjoin%
\definecolor{currentfill}{rgb}{0.580193,0.631893,0.704274}%
\pgfsetfillcolor{currentfill}%
\pgfsetlinewidth{0.000000pt}%
\definecolor{currentstroke}{rgb}{0.000000,0.000000,0.000000}%
\pgfsetstrokecolor{currentstroke}%
\pgfsetdash{}{0pt}%
\pgfpathmoveto{\pgfqpoint{1.312740in}{1.504027in}}%
\pgfpathlineto{\pgfqpoint{1.350542in}{1.512206in}}%
\pgfpathlineto{\pgfqpoint{1.313338in}{1.550921in}}%
\pgfpathlineto{\pgfqpoint{1.275432in}{1.542848in}}%
\pgfpathclose%
\pgfusepath{fill}%
\end{pgfscope}%
\begin{pgfscope}%
\pgfpathrectangle{\pgfqpoint{0.150000in}{0.150000in}}{\pgfqpoint{2.700000in}{1.950000in}}%
\pgfusepath{clip}%
\pgfsetbuttcap%
\pgfsetroundjoin%
\definecolor{currentfill}{rgb}{0.555316,0.610080,0.686749}%
\pgfsetfillcolor{currentfill}%
\pgfsetlinewidth{0.000000pt}%
\definecolor{currentstroke}{rgb}{0.000000,0.000000,0.000000}%
\pgfsetstrokecolor{currentstroke}%
\pgfsetdash{}{0pt}%
\pgfpathmoveto{\pgfqpoint{1.275432in}{1.542848in}}%
\pgfpathlineto{\pgfqpoint{1.313338in}{1.550921in}}%
\pgfpathlineto{\pgfqpoint{1.276129in}{1.589641in}}%
\pgfpathlineto{\pgfqpoint{1.238118in}{1.581674in}}%
\pgfpathclose%
\pgfusepath{fill}%
\end{pgfscope}%
\begin{pgfscope}%
\pgfpathrectangle{\pgfqpoint{0.150000in}{0.150000in}}{\pgfqpoint{2.700000in}{1.950000in}}%
\pgfusepath{clip}%
\pgfsetbuttcap%
\pgfsetroundjoin%
\definecolor{currentfill}{rgb}{0.524219,0.582812,0.664844}%
\pgfsetfillcolor{currentfill}%
\pgfsetlinewidth{0.000000pt}%
\definecolor{currentstroke}{rgb}{0.000000,0.000000,0.000000}%
\pgfsetstrokecolor{currentstroke}%
\pgfsetdash{}{0pt}%
\pgfpathmoveto{\pgfqpoint{1.238118in}{1.581674in}}%
\pgfpathlineto{\pgfqpoint{1.276129in}{1.589641in}}%
\pgfpathlineto{\pgfqpoint{1.238915in}{1.628366in}}%
\pgfpathlineto{\pgfqpoint{1.200800in}{1.620505in}}%
\pgfpathclose%
\pgfusepath{fill}%
\end{pgfscope}%
\begin{pgfscope}%
\pgfpathrectangle{\pgfqpoint{0.150000in}{0.150000in}}{\pgfqpoint{2.700000in}{1.950000in}}%
\pgfusepath{clip}%
\pgfsetbuttcap%
\pgfsetroundjoin%
\definecolor{currentfill}{rgb}{0.493122,0.555545,0.642938}%
\pgfsetfillcolor{currentfill}%
\pgfsetlinewidth{0.000000pt}%
\definecolor{currentstroke}{rgb}{0.000000,0.000000,0.000000}%
\pgfsetstrokecolor{currentstroke}%
\pgfsetdash{}{0pt}%
\pgfpathmoveto{\pgfqpoint{1.200800in}{1.620505in}}%
\pgfpathlineto{\pgfqpoint{1.238915in}{1.628366in}}%
\pgfpathlineto{\pgfqpoint{1.201696in}{1.667097in}}%
\pgfpathlineto{\pgfqpoint{1.163476in}{1.659341in}}%
\pgfpathclose%
\pgfusepath{fill}%
\end{pgfscope}%
\begin{pgfscope}%
\pgfpathrectangle{\pgfqpoint{0.150000in}{0.150000in}}{\pgfqpoint{2.700000in}{1.950000in}}%
\pgfusepath{clip}%
\pgfsetbuttcap%
\pgfsetroundjoin%
\definecolor{currentfill}{rgb}{0.876532,0.775965,0.783931}%
\pgfsetfillcolor{currentfill}%
\pgfsetlinewidth{0.000000pt}%
\definecolor{currentstroke}{rgb}{0.000000,0.000000,0.000000}%
\pgfsetstrokecolor{currentstroke}%
\pgfsetdash{}{0pt}%
\pgfpathmoveto{\pgfqpoint{2.059259in}{0.679543in}}%
\pgfpathlineto{\pgfqpoint{2.095066in}{0.689991in}}%
\pgfpathlineto{\pgfqpoint{2.057862in}{0.728703in}}%
\pgfpathlineto{\pgfqpoint{2.021951in}{0.718360in}}%
\pgfpathclose%
\pgfusepath{fill}%
\end{pgfscope}%
\begin{pgfscope}%
\pgfpathrectangle{\pgfqpoint{0.150000in}{0.150000in}}{\pgfqpoint{2.700000in}{1.950000in}}%
\pgfusepath{clip}%
\pgfsetbuttcap%
\pgfsetroundjoin%
\definecolor{currentfill}{rgb}{0.891728,0.803539,0.810524}%
\pgfsetfillcolor{currentfill}%
\pgfsetlinewidth{0.000000pt}%
\definecolor{currentstroke}{rgb}{0.000000,0.000000,0.000000}%
\pgfsetstrokecolor{currentstroke}%
\pgfsetdash{}{0pt}%
\pgfpathmoveto{\pgfqpoint{2.021951in}{0.718360in}}%
\pgfpathlineto{\pgfqpoint{2.057862in}{0.728703in}}%
\pgfpathlineto{\pgfqpoint{2.020653in}{0.767420in}}%
\pgfpathlineto{\pgfqpoint{1.984638in}{0.757183in}}%
\pgfpathclose%
\pgfusepath{fill}%
\end{pgfscope}%
\begin{pgfscope}%
\pgfpathrectangle{\pgfqpoint{0.150000in}{0.150000in}}{\pgfqpoint{2.700000in}{1.950000in}}%
\pgfusepath{clip}%
\pgfsetbuttcap%
\pgfsetroundjoin%
\definecolor{currentfill}{rgb}{0.910723,0.838006,0.843765}%
\pgfsetfillcolor{currentfill}%
\pgfsetlinewidth{0.000000pt}%
\definecolor{currentstroke}{rgb}{0.000000,0.000000,0.000000}%
\pgfsetstrokecolor{currentstroke}%
\pgfsetdash{}{0pt}%
\pgfpathmoveto{\pgfqpoint{1.984638in}{0.757183in}}%
\pgfpathlineto{\pgfqpoint{2.020653in}{0.767420in}}%
\pgfpathlineto{\pgfqpoint{1.983440in}{0.806142in}}%
\pgfpathlineto{\pgfqpoint{1.947319in}{0.796010in}}%
\pgfpathclose%
\pgfusepath{fill}%
\end{pgfscope}%
\begin{pgfscope}%
\pgfpathrectangle{\pgfqpoint{0.150000in}{0.150000in}}{\pgfqpoint{2.700000in}{1.950000in}}%
\pgfusepath{clip}%
\pgfsetbuttcap%
\pgfsetroundjoin%
\definecolor{currentfill}{rgb}{0.929718,0.872472,0.877007}%
\pgfsetfillcolor{currentfill}%
\pgfsetlinewidth{0.000000pt}%
\definecolor{currentstroke}{rgb}{0.000000,0.000000,0.000000}%
\pgfsetstrokecolor{currentstroke}%
\pgfsetdash{}{0pt}%
\pgfpathmoveto{\pgfqpoint{1.947319in}{0.796010in}}%
\pgfpathlineto{\pgfqpoint{1.983440in}{0.806142in}}%
\pgfpathlineto{\pgfqpoint{1.946221in}{0.844869in}}%
\pgfpathlineto{\pgfqpoint{1.909996in}{0.834843in}}%
\pgfpathclose%
\pgfusepath{fill}%
\end{pgfscope}%
\begin{pgfscope}%
\pgfpathrectangle{\pgfqpoint{0.150000in}{0.150000in}}{\pgfqpoint{2.700000in}{1.950000in}}%
\pgfusepath{clip}%
\pgfsetbuttcap%
\pgfsetroundjoin%
\definecolor{currentfill}{rgb}{0.948713,0.906939,0.910248}%
\pgfsetfillcolor{currentfill}%
\pgfsetlinewidth{0.000000pt}%
\definecolor{currentstroke}{rgb}{0.000000,0.000000,0.000000}%
\pgfsetstrokecolor{currentstroke}%
\pgfsetdash{}{0pt}%
\pgfpathmoveto{\pgfqpoint{1.909996in}{0.834843in}}%
\pgfpathlineto{\pgfqpoint{1.946221in}{0.844869in}}%
\pgfpathlineto{\pgfqpoint{1.908997in}{0.883602in}}%
\pgfpathlineto{\pgfqpoint{1.872668in}{0.873682in}}%
\pgfpathclose%
\pgfusepath{fill}%
\end{pgfscope}%
\begin{pgfscope}%
\pgfpathrectangle{\pgfqpoint{0.150000in}{0.150000in}}{\pgfqpoint{2.700000in}{1.950000in}}%
\pgfusepath{clip}%
\pgfsetbuttcap%
\pgfsetroundjoin%
\definecolor{currentfill}{rgb}{0.967708,0.941406,0.943490}%
\pgfsetfillcolor{currentfill}%
\pgfsetlinewidth{0.000000pt}%
\definecolor{currentstroke}{rgb}{0.000000,0.000000,0.000000}%
\pgfsetstrokecolor{currentstroke}%
\pgfsetdash{}{0pt}%
\pgfpathmoveto{\pgfqpoint{1.872668in}{0.873682in}}%
\pgfpathlineto{\pgfqpoint{1.908997in}{0.883602in}}%
\pgfpathlineto{\pgfqpoint{1.871769in}{0.922339in}}%
\pgfpathlineto{\pgfqpoint{1.835334in}{0.912525in}}%
\pgfpathclose%
\pgfusepath{fill}%
\end{pgfscope}%
\begin{pgfscope}%
\pgfpathrectangle{\pgfqpoint{0.150000in}{0.150000in}}{\pgfqpoint{2.700000in}{1.950000in}}%
\pgfusepath{clip}%
\pgfsetbuttcap%
\pgfsetroundjoin%
\definecolor{currentfill}{rgb}{0.986703,0.975873,0.976731}%
\pgfsetfillcolor{currentfill}%
\pgfsetlinewidth{0.000000pt}%
\definecolor{currentstroke}{rgb}{0.000000,0.000000,0.000000}%
\pgfsetstrokecolor{currentstroke}%
\pgfsetdash{}{0pt}%
\pgfpathmoveto{\pgfqpoint{1.835334in}{0.912525in}}%
\pgfpathlineto{\pgfqpoint{1.871769in}{0.922339in}}%
\pgfpathlineto{\pgfqpoint{1.834535in}{0.961082in}}%
\pgfpathlineto{\pgfqpoint{1.797996in}{0.951374in}}%
\pgfpathclose%
\pgfusepath{fill}%
\end{pgfscope}%
\begin{pgfscope}%
\pgfpathrectangle{\pgfqpoint{0.150000in}{0.150000in}}{\pgfqpoint{2.700000in}{1.950000in}}%
\pgfusepath{clip}%
\pgfsetbuttcap%
\pgfsetroundjoin%
\definecolor{currentfill}{rgb}{0.990671,0.991820,0.993428}%
\pgfsetfillcolor{currentfill}%
\pgfsetlinewidth{0.000000pt}%
\definecolor{currentstroke}{rgb}{0.000000,0.000000,0.000000}%
\pgfsetstrokecolor{currentstroke}%
\pgfsetdash{}{0pt}%
\pgfpathmoveto{\pgfqpoint{1.797996in}{0.951374in}}%
\pgfpathlineto{\pgfqpoint{1.834535in}{0.961082in}}%
\pgfpathlineto{\pgfqpoint{1.797296in}{0.999830in}}%
\pgfpathlineto{\pgfqpoint{1.760652in}{0.990228in}}%
\pgfpathclose%
\pgfusepath{fill}%
\end{pgfscope}%
\begin{pgfscope}%
\pgfpathrectangle{\pgfqpoint{0.150000in}{0.150000in}}{\pgfqpoint{2.700000in}{1.950000in}}%
\pgfusepath{clip}%
\pgfsetbuttcap%
\pgfsetroundjoin%
\definecolor{currentfill}{rgb}{0.959574,0.964553,0.971523}%
\pgfsetfillcolor{currentfill}%
\pgfsetlinewidth{0.000000pt}%
\definecolor{currentstroke}{rgb}{0.000000,0.000000,0.000000}%
\pgfsetstrokecolor{currentstroke}%
\pgfsetdash{}{0pt}%
\pgfpathmoveto{\pgfqpoint{1.760652in}{0.990228in}}%
\pgfpathlineto{\pgfqpoint{1.797296in}{0.999830in}}%
\pgfpathlineto{\pgfqpoint{1.760053in}{1.038583in}}%
\pgfpathlineto{\pgfqpoint{1.723304in}{1.029087in}}%
\pgfpathclose%
\pgfusepath{fill}%
\end{pgfscope}%
\begin{pgfscope}%
\pgfpathrectangle{\pgfqpoint{0.150000in}{0.150000in}}{\pgfqpoint{2.700000in}{1.950000in}}%
\pgfusepath{clip}%
\pgfsetbuttcap%
\pgfsetroundjoin%
\definecolor{currentfill}{rgb}{0.934697,0.942739,0.953998}%
\pgfsetfillcolor{currentfill}%
\pgfsetlinewidth{0.000000pt}%
\definecolor{currentstroke}{rgb}{0.000000,0.000000,0.000000}%
\pgfsetstrokecolor{currentstroke}%
\pgfsetdash{}{0pt}%
\pgfpathmoveto{\pgfqpoint{1.723304in}{1.029087in}}%
\pgfpathlineto{\pgfqpoint{1.760053in}{1.038583in}}%
\pgfpathlineto{\pgfqpoint{1.722804in}{1.077342in}}%
\pgfpathlineto{\pgfqpoint{1.685950in}{1.067951in}}%
\pgfpathclose%
\pgfusepath{fill}%
\end{pgfscope}%
\begin{pgfscope}%
\pgfpathrectangle{\pgfqpoint{0.150000in}{0.150000in}}{\pgfqpoint{2.700000in}{1.950000in}}%
\pgfusepath{clip}%
\pgfsetbuttcap%
\pgfsetroundjoin%
\definecolor{currentfill}{rgb}{0.903600,0.915472,0.932093}%
\pgfsetfillcolor{currentfill}%
\pgfsetlinewidth{0.000000pt}%
\definecolor{currentstroke}{rgb}{0.000000,0.000000,0.000000}%
\pgfsetstrokecolor{currentstroke}%
\pgfsetdash{}{0pt}%
\pgfpathmoveto{\pgfqpoint{1.685950in}{1.067951in}}%
\pgfpathlineto{\pgfqpoint{1.722804in}{1.077342in}}%
\pgfpathlineto{\pgfqpoint{1.685551in}{1.116105in}}%
\pgfpathlineto{\pgfqpoint{1.648592in}{1.106821in}}%
\pgfpathclose%
\pgfusepath{fill}%
\end{pgfscope}%
\begin{pgfscope}%
\pgfpathrectangle{\pgfqpoint{0.150000in}{0.150000in}}{\pgfqpoint{2.700000in}{1.950000in}}%
\pgfusepath{clip}%
\pgfsetbuttcap%
\pgfsetroundjoin%
\definecolor{currentfill}{rgb}{0.872503,0.888205,0.910187}%
\pgfsetfillcolor{currentfill}%
\pgfsetlinewidth{0.000000pt}%
\definecolor{currentstroke}{rgb}{0.000000,0.000000,0.000000}%
\pgfsetstrokecolor{currentstroke}%
\pgfsetdash{}{0pt}%
\pgfpathmoveto{\pgfqpoint{1.648592in}{1.106821in}}%
\pgfpathlineto{\pgfqpoint{1.685551in}{1.116105in}}%
\pgfpathlineto{\pgfqpoint{1.648292in}{1.154874in}}%
\pgfpathlineto{\pgfqpoint{1.611229in}{1.145695in}}%
\pgfpathclose%
\pgfusepath{fill}%
\end{pgfscope}%
\begin{pgfscope}%
\pgfpathrectangle{\pgfqpoint{0.150000in}{0.150000in}}{\pgfqpoint{2.700000in}{1.950000in}}%
\pgfusepath{clip}%
\pgfsetbuttcap%
\pgfsetroundjoin%
\definecolor{currentfill}{rgb}{0.841406,0.860938,0.888281}%
\pgfsetfillcolor{currentfill}%
\pgfsetlinewidth{0.000000pt}%
\definecolor{currentstroke}{rgb}{0.000000,0.000000,0.000000}%
\pgfsetstrokecolor{currentstroke}%
\pgfsetdash{}{0pt}%
\pgfpathmoveto{\pgfqpoint{1.611229in}{1.145695in}}%
\pgfpathlineto{\pgfqpoint{1.648292in}{1.154874in}}%
\pgfpathlineto{\pgfqpoint{1.611029in}{1.193648in}}%
\pgfpathlineto{\pgfqpoint{1.573860in}{1.184575in}}%
\pgfpathclose%
\pgfusepath{fill}%
\end{pgfscope}%
\begin{pgfscope}%
\pgfpathrectangle{\pgfqpoint{0.150000in}{0.150000in}}{\pgfqpoint{2.700000in}{1.950000in}}%
\pgfusepath{clip}%
\pgfsetbuttcap%
\pgfsetroundjoin%
\definecolor{currentfill}{rgb}{0.810309,0.833670,0.866376}%
\pgfsetfillcolor{currentfill}%
\pgfsetlinewidth{0.000000pt}%
\definecolor{currentstroke}{rgb}{0.000000,0.000000,0.000000}%
\pgfsetstrokecolor{currentstroke}%
\pgfsetdash{}{0pt}%
\pgfpathmoveto{\pgfqpoint{1.573860in}{1.184575in}}%
\pgfpathlineto{\pgfqpoint{1.611029in}{1.193648in}}%
\pgfpathlineto{\pgfqpoint{1.573760in}{1.232427in}}%
\pgfpathlineto{\pgfqpoint{1.536486in}{1.223460in}}%
\pgfpathclose%
\pgfusepath{fill}%
\end{pgfscope}%
\begin{pgfscope}%
\pgfpathrectangle{\pgfqpoint{0.150000in}{0.150000in}}{\pgfqpoint{2.700000in}{1.950000in}}%
\pgfusepath{clip}%
\pgfsetbuttcap%
\pgfsetroundjoin%
\definecolor{currentfill}{rgb}{0.779213,0.806403,0.844470}%
\pgfsetfillcolor{currentfill}%
\pgfsetlinewidth{0.000000pt}%
\definecolor{currentstroke}{rgb}{0.000000,0.000000,0.000000}%
\pgfsetstrokecolor{currentstroke}%
\pgfsetdash{}{0pt}%
\pgfpathmoveto{\pgfqpoint{1.536486in}{1.223460in}}%
\pgfpathlineto{\pgfqpoint{1.573760in}{1.232427in}}%
\pgfpathlineto{\pgfqpoint{1.536486in}{1.271212in}}%
\pgfpathlineto{\pgfqpoint{1.499108in}{1.262351in}}%
\pgfpathclose%
\pgfusepath{fill}%
\end{pgfscope}%
\begin{pgfscope}%
\pgfpathrectangle{\pgfqpoint{0.150000in}{0.150000in}}{\pgfqpoint{2.700000in}{1.950000in}}%
\pgfusepath{clip}%
\pgfsetbuttcap%
\pgfsetroundjoin%
\definecolor{currentfill}{rgb}{0.748116,0.779136,0.822564}%
\pgfsetfillcolor{currentfill}%
\pgfsetlinewidth{0.000000pt}%
\definecolor{currentstroke}{rgb}{0.000000,0.000000,0.000000}%
\pgfsetstrokecolor{currentstroke}%
\pgfsetdash{}{0pt}%
\pgfpathmoveto{\pgfqpoint{1.499108in}{1.262351in}}%
\pgfpathlineto{\pgfqpoint{1.536486in}{1.271212in}}%
\pgfpathlineto{\pgfqpoint{1.499208in}{1.310001in}}%
\pgfpathlineto{\pgfqpoint{1.461724in}{1.301246in}}%
\pgfpathclose%
\pgfusepath{fill}%
\end{pgfscope}%
\begin{pgfscope}%
\pgfpathrectangle{\pgfqpoint{0.150000in}{0.150000in}}{\pgfqpoint{2.700000in}{1.950000in}}%
\pgfusepath{clip}%
\pgfsetbuttcap%
\pgfsetroundjoin%
\definecolor{currentfill}{rgb}{0.717019,0.751869,0.800659}%
\pgfsetfillcolor{currentfill}%
\pgfsetlinewidth{0.000000pt}%
\definecolor{currentstroke}{rgb}{0.000000,0.000000,0.000000}%
\pgfsetstrokecolor{currentstroke}%
\pgfsetdash{}{0pt}%
\pgfpathmoveto{\pgfqpoint{1.461724in}{1.301246in}}%
\pgfpathlineto{\pgfqpoint{1.499208in}{1.310001in}}%
\pgfpathlineto{\pgfqpoint{1.461924in}{1.348796in}}%
\pgfpathlineto{\pgfqpoint{1.424336in}{1.340147in}}%
\pgfpathclose%
\pgfusepath{fill}%
\end{pgfscope}%
\begin{pgfscope}%
\pgfpathrectangle{\pgfqpoint{0.150000in}{0.150000in}}{\pgfqpoint{2.700000in}{1.950000in}}%
\pgfusepath{clip}%
\pgfsetbuttcap%
\pgfsetroundjoin%
\definecolor{currentfill}{rgb}{0.692142,0.730055,0.783134}%
\pgfsetfillcolor{currentfill}%
\pgfsetlinewidth{0.000000pt}%
\definecolor{currentstroke}{rgb}{0.000000,0.000000,0.000000}%
\pgfsetstrokecolor{currentstroke}%
\pgfsetdash{}{0pt}%
\pgfpathmoveto{\pgfqpoint{1.424336in}{1.340147in}}%
\pgfpathlineto{\pgfqpoint{1.461924in}{1.348796in}}%
\pgfpathlineto{\pgfqpoint{1.424636in}{1.387596in}}%
\pgfpathlineto{\pgfqpoint{1.386942in}{1.379053in}}%
\pgfpathclose%
\pgfusepath{fill}%
\end{pgfscope}%
\begin{pgfscope}%
\pgfpathrectangle{\pgfqpoint{0.150000in}{0.150000in}}{\pgfqpoint{2.700000in}{1.950000in}}%
\pgfusepath{clip}%
\pgfsetbuttcap%
\pgfsetroundjoin%
\definecolor{currentfill}{rgb}{0.661045,0.702788,0.761229}%
\pgfsetfillcolor{currentfill}%
\pgfsetlinewidth{0.000000pt}%
\definecolor{currentstroke}{rgb}{0.000000,0.000000,0.000000}%
\pgfsetstrokecolor{currentstroke}%
\pgfsetdash{}{0pt}%
\pgfpathmoveto{\pgfqpoint{1.386942in}{1.379053in}}%
\pgfpathlineto{\pgfqpoint{1.424636in}{1.387596in}}%
\pgfpathlineto{\pgfqpoint{1.387342in}{1.426401in}}%
\pgfpathlineto{\pgfqpoint{1.349544in}{1.417965in}}%
\pgfpathclose%
\pgfusepath{fill}%
\end{pgfscope}%
\begin{pgfscope}%
\pgfpathrectangle{\pgfqpoint{0.150000in}{0.150000in}}{\pgfqpoint{2.700000in}{1.950000in}}%
\pgfusepath{clip}%
\pgfsetbuttcap%
\pgfsetroundjoin%
\definecolor{currentfill}{rgb}{0.629948,0.675521,0.739323}%
\pgfsetfillcolor{currentfill}%
\pgfsetlinewidth{0.000000pt}%
\definecolor{currentstroke}{rgb}{0.000000,0.000000,0.000000}%
\pgfsetstrokecolor{currentstroke}%
\pgfsetdash{}{0pt}%
\pgfpathmoveto{\pgfqpoint{1.349544in}{1.417965in}}%
\pgfpathlineto{\pgfqpoint{1.387342in}{1.426401in}}%
\pgfpathlineto{\pgfqpoint{1.350044in}{1.465211in}}%
\pgfpathlineto{\pgfqpoint{1.312140in}{1.456881in}}%
\pgfpathclose%
\pgfusepath{fill}%
\end{pgfscope}%
\begin{pgfscope}%
\pgfpathrectangle{\pgfqpoint{0.150000in}{0.150000in}}{\pgfqpoint{2.700000in}{1.950000in}}%
\pgfusepath{clip}%
\pgfsetbuttcap%
\pgfsetroundjoin%
\definecolor{currentfill}{rgb}{0.598851,0.648254,0.717417}%
\pgfsetfillcolor{currentfill}%
\pgfsetlinewidth{0.000000pt}%
\definecolor{currentstroke}{rgb}{0.000000,0.000000,0.000000}%
\pgfsetstrokecolor{currentstroke}%
\pgfsetdash{}{0pt}%
\pgfpathmoveto{\pgfqpoint{1.312140in}{1.456881in}}%
\pgfpathlineto{\pgfqpoint{1.350044in}{1.465211in}}%
\pgfpathlineto{\pgfqpoint{1.312740in}{1.504027in}}%
\pgfpathlineto{\pgfqpoint{1.274731in}{1.495803in}}%
\pgfpathclose%
\pgfusepath{fill}%
\end{pgfscope}%
\begin{pgfscope}%
\pgfpathrectangle{\pgfqpoint{0.150000in}{0.150000in}}{\pgfqpoint{2.700000in}{1.950000in}}%
\pgfusepath{clip}%
\pgfsetbuttcap%
\pgfsetroundjoin%
\definecolor{currentfill}{rgb}{0.567754,0.620987,0.695512}%
\pgfsetfillcolor{currentfill}%
\pgfsetlinewidth{0.000000pt}%
\definecolor{currentstroke}{rgb}{0.000000,0.000000,0.000000}%
\pgfsetstrokecolor{currentstroke}%
\pgfsetdash{}{0pt}%
\pgfpathmoveto{\pgfqpoint{1.274731in}{1.495803in}}%
\pgfpathlineto{\pgfqpoint{1.312740in}{1.504027in}}%
\pgfpathlineto{\pgfqpoint{1.275432in}{1.542848in}}%
\pgfpathlineto{\pgfqpoint{1.237317in}{1.534730in}}%
\pgfpathclose%
\pgfusepath{fill}%
\end{pgfscope}%
\begin{pgfscope}%
\pgfpathrectangle{\pgfqpoint{0.150000in}{0.150000in}}{\pgfqpoint{2.700000in}{1.950000in}}%
\pgfusepath{clip}%
\pgfsetbuttcap%
\pgfsetroundjoin%
\definecolor{currentfill}{rgb}{0.536657,0.593719,0.673606}%
\pgfsetfillcolor{currentfill}%
\pgfsetlinewidth{0.000000pt}%
\definecolor{currentstroke}{rgb}{0.000000,0.000000,0.000000}%
\pgfsetstrokecolor{currentstroke}%
\pgfsetdash{}{0pt}%
\pgfpathmoveto{\pgfqpoint{1.237317in}{1.534730in}}%
\pgfpathlineto{\pgfqpoint{1.275432in}{1.542848in}}%
\pgfpathlineto{\pgfqpoint{1.238118in}{1.581674in}}%
\pgfpathlineto{\pgfqpoint{1.199899in}{1.573662in}}%
\pgfpathclose%
\pgfusepath{fill}%
\end{pgfscope}%
\begin{pgfscope}%
\pgfpathrectangle{\pgfqpoint{0.150000in}{0.150000in}}{\pgfqpoint{2.700000in}{1.950000in}}%
\pgfusepath{clip}%
\pgfsetbuttcap%
\pgfsetroundjoin%
\definecolor{currentfill}{rgb}{0.505561,0.566452,0.651700}%
\pgfsetfillcolor{currentfill}%
\pgfsetlinewidth{0.000000pt}%
\definecolor{currentstroke}{rgb}{0.000000,0.000000,0.000000}%
\pgfsetstrokecolor{currentstroke}%
\pgfsetdash{}{0pt}%
\pgfpathmoveto{\pgfqpoint{1.199899in}{1.573662in}}%
\pgfpathlineto{\pgfqpoint{1.238118in}{1.581674in}}%
\pgfpathlineto{\pgfqpoint{1.200800in}{1.620505in}}%
\pgfpathlineto{\pgfqpoint{1.162475in}{1.612600in}}%
\pgfpathclose%
\pgfusepath{fill}%
\end{pgfscope}%
\begin{pgfscope}%
\pgfpathrectangle{\pgfqpoint{0.150000in}{0.150000in}}{\pgfqpoint{2.700000in}{1.950000in}}%
\pgfusepath{clip}%
\pgfsetbuttcap%
\pgfsetroundjoin%
\definecolor{currentfill}{rgb}{0.474464,0.539185,0.629795}%
\pgfsetfillcolor{currentfill}%
\pgfsetlinewidth{0.000000pt}%
\definecolor{currentstroke}{rgb}{0.000000,0.000000,0.000000}%
\pgfsetstrokecolor{currentstroke}%
\pgfsetdash{}{0pt}%
\pgfpathmoveto{\pgfqpoint{1.162475in}{1.612600in}}%
\pgfpathlineto{\pgfqpoint{1.200800in}{1.620505in}}%
\pgfpathlineto{\pgfqpoint{1.163476in}{1.659341in}}%
\pgfpathlineto{\pgfqpoint{1.125046in}{1.651542in}}%
\pgfpathclose%
\pgfusepath{fill}%
\end{pgfscope}%
\begin{pgfscope}%
\pgfpathrectangle{\pgfqpoint{0.150000in}{0.150000in}}{\pgfqpoint{2.700000in}{1.950000in}}%
\pgfusepath{clip}%
\pgfsetbuttcap%
\pgfsetroundjoin%
\definecolor{currentfill}{rgb}{0.884130,0.789752,0.797227}%
\pgfsetfillcolor{currentfill}%
\pgfsetlinewidth{0.000000pt}%
\definecolor{currentstroke}{rgb}{0.000000,0.000000,0.000000}%
\pgfsetstrokecolor{currentstroke}%
\pgfsetdash{}{0pt}%
\pgfpathmoveto{\pgfqpoint{2.023256in}{0.669037in}}%
\pgfpathlineto{\pgfqpoint{2.059259in}{0.679543in}}%
\pgfpathlineto{\pgfqpoint{2.021951in}{0.718360in}}%
\pgfpathlineto{\pgfqpoint{1.985842in}{0.707960in}}%
\pgfpathclose%
\pgfusepath{fill}%
\end{pgfscope}%
\begin{pgfscope}%
\pgfpathrectangle{\pgfqpoint{0.150000in}{0.150000in}}{\pgfqpoint{2.700000in}{1.950000in}}%
\pgfusepath{clip}%
\pgfsetbuttcap%
\pgfsetroundjoin%
\definecolor{currentfill}{rgb}{0.903125,0.824219,0.830469}%
\pgfsetfillcolor{currentfill}%
\pgfsetlinewidth{0.000000pt}%
\definecolor{currentstroke}{rgb}{0.000000,0.000000,0.000000}%
\pgfsetstrokecolor{currentstroke}%
\pgfsetdash{}{0pt}%
\pgfpathmoveto{\pgfqpoint{1.985842in}{0.707960in}}%
\pgfpathlineto{\pgfqpoint{2.021951in}{0.718360in}}%
\pgfpathlineto{\pgfqpoint{1.984638in}{0.757183in}}%
\pgfpathlineto{\pgfqpoint{1.948424in}{0.746889in}}%
\pgfpathclose%
\pgfusepath{fill}%
\end{pgfscope}%
\begin{pgfscope}%
\pgfpathrectangle{\pgfqpoint{0.150000in}{0.150000in}}{\pgfqpoint{2.700000in}{1.950000in}}%
\pgfusepath{clip}%
\pgfsetbuttcap%
\pgfsetroundjoin%
\definecolor{currentfill}{rgb}{0.922120,0.858686,0.863710}%
\pgfsetfillcolor{currentfill}%
\pgfsetlinewidth{0.000000pt}%
\definecolor{currentstroke}{rgb}{0.000000,0.000000,0.000000}%
\pgfsetstrokecolor{currentstroke}%
\pgfsetdash{}{0pt}%
\pgfpathmoveto{\pgfqpoint{1.948424in}{0.746889in}}%
\pgfpathlineto{\pgfqpoint{1.984638in}{0.757183in}}%
\pgfpathlineto{\pgfqpoint{1.947319in}{0.796010in}}%
\pgfpathlineto{\pgfqpoint{1.911000in}{0.785823in}}%
\pgfpathclose%
\pgfusepath{fill}%
\end{pgfscope}%
\begin{pgfscope}%
\pgfpathrectangle{\pgfqpoint{0.150000in}{0.150000in}}{\pgfqpoint{2.700000in}{1.950000in}}%
\pgfusepath{clip}%
\pgfsetbuttcap%
\pgfsetroundjoin%
\definecolor{currentfill}{rgb}{0.941115,0.893153,0.896952}%
\pgfsetfillcolor{currentfill}%
\pgfsetlinewidth{0.000000pt}%
\definecolor{currentstroke}{rgb}{0.000000,0.000000,0.000000}%
\pgfsetstrokecolor{currentstroke}%
\pgfsetdash{}{0pt}%
\pgfpathmoveto{\pgfqpoint{1.911000in}{0.785823in}}%
\pgfpathlineto{\pgfqpoint{1.947319in}{0.796010in}}%
\pgfpathlineto{\pgfqpoint{1.909996in}{0.834843in}}%
\pgfpathlineto{\pgfqpoint{1.873571in}{0.824762in}}%
\pgfpathclose%
\pgfusepath{fill}%
\end{pgfscope}%
\begin{pgfscope}%
\pgfpathrectangle{\pgfqpoint{0.150000in}{0.150000in}}{\pgfqpoint{2.700000in}{1.950000in}}%
\pgfusepath{clip}%
\pgfsetbuttcap%
\pgfsetroundjoin%
\definecolor{currentfill}{rgb}{0.960110,0.927619,0.930193}%
\pgfsetfillcolor{currentfill}%
\pgfsetlinewidth{0.000000pt}%
\definecolor{currentstroke}{rgb}{0.000000,0.000000,0.000000}%
\pgfsetstrokecolor{currentstroke}%
\pgfsetdash{}{0pt}%
\pgfpathmoveto{\pgfqpoint{1.873571in}{0.824762in}}%
\pgfpathlineto{\pgfqpoint{1.909996in}{0.834843in}}%
\pgfpathlineto{\pgfqpoint{1.872668in}{0.873682in}}%
\pgfpathlineto{\pgfqpoint{1.836138in}{0.863707in}}%
\pgfpathclose%
\pgfusepath{fill}%
\end{pgfscope}%
\begin{pgfscope}%
\pgfpathrectangle{\pgfqpoint{0.150000in}{0.150000in}}{\pgfqpoint{2.700000in}{1.950000in}}%
\pgfusepath{clip}%
\pgfsetbuttcap%
\pgfsetroundjoin%
\definecolor{currentfill}{rgb}{0.975306,0.955193,0.956786}%
\pgfsetfillcolor{currentfill}%
\pgfsetlinewidth{0.000000pt}%
\definecolor{currentstroke}{rgb}{0.000000,0.000000,0.000000}%
\pgfsetstrokecolor{currentstroke}%
\pgfsetdash{}{0pt}%
\pgfpathmoveto{\pgfqpoint{1.836138in}{0.863707in}}%
\pgfpathlineto{\pgfqpoint{1.872668in}{0.873682in}}%
\pgfpathlineto{\pgfqpoint{1.835334in}{0.912525in}}%
\pgfpathlineto{\pgfqpoint{1.798699in}{0.902657in}}%
\pgfpathclose%
\pgfusepath{fill}%
\end{pgfscope}%
\begin{pgfscope}%
\pgfpathrectangle{\pgfqpoint{0.150000in}{0.150000in}}{\pgfqpoint{2.700000in}{1.950000in}}%
\pgfusepath{clip}%
\pgfsetbuttcap%
\pgfsetroundjoin%
\definecolor{currentfill}{rgb}{0.994301,0.989660,0.990028}%
\pgfsetfillcolor{currentfill}%
\pgfsetlinewidth{0.000000pt}%
\definecolor{currentstroke}{rgb}{0.000000,0.000000,0.000000}%
\pgfsetstrokecolor{currentstroke}%
\pgfsetdash{}{0pt}%
\pgfpathmoveto{\pgfqpoint{1.798699in}{0.902657in}}%
\pgfpathlineto{\pgfqpoint{1.835334in}{0.912525in}}%
\pgfpathlineto{\pgfqpoint{1.797996in}{0.951374in}}%
\pgfpathlineto{\pgfqpoint{1.761255in}{0.941612in}}%
\pgfpathclose%
\pgfusepath{fill}%
\end{pgfscope}%
\begin{pgfscope}%
\pgfpathrectangle{\pgfqpoint{0.150000in}{0.150000in}}{\pgfqpoint{2.700000in}{1.950000in}}%
\pgfusepath{clip}%
\pgfsetbuttcap%
\pgfsetroundjoin%
\definecolor{currentfill}{rgb}{0.978232,0.980913,0.984666}%
\pgfsetfillcolor{currentfill}%
\pgfsetlinewidth{0.000000pt}%
\definecolor{currentstroke}{rgb}{0.000000,0.000000,0.000000}%
\pgfsetstrokecolor{currentstroke}%
\pgfsetdash{}{0pt}%
\pgfpathmoveto{\pgfqpoint{1.761255in}{0.941612in}}%
\pgfpathlineto{\pgfqpoint{1.797996in}{0.951374in}}%
\pgfpathlineto{\pgfqpoint{1.760652in}{0.990228in}}%
\pgfpathlineto{\pgfqpoint{1.723806in}{0.980572in}}%
\pgfpathclose%
\pgfusepath{fill}%
\end{pgfscope}%
\begin{pgfscope}%
\pgfpathrectangle{\pgfqpoint{0.150000in}{0.150000in}}{\pgfqpoint{2.700000in}{1.950000in}}%
\pgfusepath{clip}%
\pgfsetbuttcap%
\pgfsetroundjoin%
\definecolor{currentfill}{rgb}{0.947135,0.953646,0.962760}%
\pgfsetfillcolor{currentfill}%
\pgfsetlinewidth{0.000000pt}%
\definecolor{currentstroke}{rgb}{0.000000,0.000000,0.000000}%
\pgfsetstrokecolor{currentstroke}%
\pgfsetdash{}{0pt}%
\pgfpathmoveto{\pgfqpoint{1.723806in}{0.980572in}}%
\pgfpathlineto{\pgfqpoint{1.760652in}{0.990228in}}%
\pgfpathlineto{\pgfqpoint{1.723304in}{1.029087in}}%
\pgfpathlineto{\pgfqpoint{1.686352in}{1.019538in}}%
\pgfpathclose%
\pgfusepath{fill}%
\end{pgfscope}%
\begin{pgfscope}%
\pgfpathrectangle{\pgfqpoint{0.150000in}{0.150000in}}{\pgfqpoint{2.700000in}{1.950000in}}%
\pgfusepath{clip}%
\pgfsetbuttcap%
\pgfsetroundjoin%
\definecolor{currentfill}{rgb}{0.916039,0.926379,0.940855}%
\pgfsetfillcolor{currentfill}%
\pgfsetlinewidth{0.000000pt}%
\definecolor{currentstroke}{rgb}{0.000000,0.000000,0.000000}%
\pgfsetstrokecolor{currentstroke}%
\pgfsetdash{}{0pt}%
\pgfpathmoveto{\pgfqpoint{1.686352in}{1.019538in}}%
\pgfpathlineto{\pgfqpoint{1.723304in}{1.029087in}}%
\pgfpathlineto{\pgfqpoint{1.685950in}{1.067951in}}%
\pgfpathlineto{\pgfqpoint{1.648893in}{1.058508in}}%
\pgfpathclose%
\pgfusepath{fill}%
\end{pgfscope}%
\begin{pgfscope}%
\pgfpathrectangle{\pgfqpoint{0.150000in}{0.150000in}}{\pgfqpoint{2.700000in}{1.950000in}}%
\pgfusepath{clip}%
\pgfsetbuttcap%
\pgfsetroundjoin%
\definecolor{currentfill}{rgb}{0.884942,0.899112,0.918949}%
\pgfsetfillcolor{currentfill}%
\pgfsetlinewidth{0.000000pt}%
\definecolor{currentstroke}{rgb}{0.000000,0.000000,0.000000}%
\pgfsetstrokecolor{currentstroke}%
\pgfsetdash{}{0pt}%
\pgfpathmoveto{\pgfqpoint{1.648893in}{1.058508in}}%
\pgfpathlineto{\pgfqpoint{1.685950in}{1.067951in}}%
\pgfpathlineto{\pgfqpoint{1.648592in}{1.106821in}}%
\pgfpathlineto{\pgfqpoint{1.611430in}{1.097485in}}%
\pgfpathclose%
\pgfusepath{fill}%
\end{pgfscope}%
\begin{pgfscope}%
\pgfpathrectangle{\pgfqpoint{0.150000in}{0.150000in}}{\pgfqpoint{2.700000in}{1.950000in}}%
\pgfusepath{clip}%
\pgfsetbuttcap%
\pgfsetroundjoin%
\definecolor{currentfill}{rgb}{0.853845,0.871844,0.897044}%
\pgfsetfillcolor{currentfill}%
\pgfsetlinewidth{0.000000pt}%
\definecolor{currentstroke}{rgb}{0.000000,0.000000,0.000000}%
\pgfsetstrokecolor{currentstroke}%
\pgfsetdash{}{0pt}%
\pgfpathmoveto{\pgfqpoint{1.611430in}{1.097485in}}%
\pgfpathlineto{\pgfqpoint{1.648592in}{1.106821in}}%
\pgfpathlineto{\pgfqpoint{1.611229in}{1.145695in}}%
\pgfpathlineto{\pgfqpoint{1.573961in}{1.136466in}}%
\pgfpathclose%
\pgfusepath{fill}%
\end{pgfscope}%
\begin{pgfscope}%
\pgfpathrectangle{\pgfqpoint{0.150000in}{0.150000in}}{\pgfqpoint{2.700000in}{1.950000in}}%
\pgfusepath{clip}%
\pgfsetbuttcap%
\pgfsetroundjoin%
\definecolor{currentfill}{rgb}{0.822748,0.844577,0.875138}%
\pgfsetfillcolor{currentfill}%
\pgfsetlinewidth{0.000000pt}%
\definecolor{currentstroke}{rgb}{0.000000,0.000000,0.000000}%
\pgfsetstrokecolor{currentstroke}%
\pgfsetdash{}{0pt}%
\pgfpathmoveto{\pgfqpoint{1.573961in}{1.136466in}}%
\pgfpathlineto{\pgfqpoint{1.611229in}{1.145695in}}%
\pgfpathlineto{\pgfqpoint{1.573860in}{1.184575in}}%
\pgfpathlineto{\pgfqpoint{1.536486in}{1.175452in}}%
\pgfpathclose%
\pgfusepath{fill}%
\end{pgfscope}%
\begin{pgfscope}%
\pgfpathrectangle{\pgfqpoint{0.150000in}{0.150000in}}{\pgfqpoint{2.700000in}{1.950000in}}%
\pgfusepath{clip}%
\pgfsetbuttcap%
\pgfsetroundjoin%
\definecolor{currentfill}{rgb}{0.797871,0.822763,0.857613}%
\pgfsetfillcolor{currentfill}%
\pgfsetlinewidth{0.000000pt}%
\definecolor{currentstroke}{rgb}{0.000000,0.000000,0.000000}%
\pgfsetstrokecolor{currentstroke}%
\pgfsetdash{}{0pt}%
\pgfpathmoveto{\pgfqpoint{1.536486in}{1.175452in}}%
\pgfpathlineto{\pgfqpoint{1.573860in}{1.184575in}}%
\pgfpathlineto{\pgfqpoint{1.536486in}{1.223460in}}%
\pgfpathlineto{\pgfqpoint{1.499007in}{1.214444in}}%
\pgfpathclose%
\pgfusepath{fill}%
\end{pgfscope}%
\begin{pgfscope}%
\pgfpathrectangle{\pgfqpoint{0.150000in}{0.150000in}}{\pgfqpoint{2.700000in}{1.950000in}}%
\pgfusepath{clip}%
\pgfsetbuttcap%
\pgfsetroundjoin%
\definecolor{currentfill}{rgb}{0.766774,0.795496,0.835708}%
\pgfsetfillcolor{currentfill}%
\pgfsetlinewidth{0.000000pt}%
\definecolor{currentstroke}{rgb}{0.000000,0.000000,0.000000}%
\pgfsetstrokecolor{currentstroke}%
\pgfsetdash{}{0pt}%
\pgfpathmoveto{\pgfqpoint{1.499007in}{1.214444in}}%
\pgfpathlineto{\pgfqpoint{1.536486in}{1.223460in}}%
\pgfpathlineto{\pgfqpoint{1.499108in}{1.262351in}}%
\pgfpathlineto{\pgfqpoint{1.461523in}{1.253441in}}%
\pgfpathclose%
\pgfusepath{fill}%
\end{pgfscope}%
\begin{pgfscope}%
\pgfpathrectangle{\pgfqpoint{0.150000in}{0.150000in}}{\pgfqpoint{2.700000in}{1.950000in}}%
\pgfusepath{clip}%
\pgfsetbuttcap%
\pgfsetroundjoin%
\definecolor{currentfill}{rgb}{0.735677,0.768229,0.813802}%
\pgfsetfillcolor{currentfill}%
\pgfsetlinewidth{0.000000pt}%
\definecolor{currentstroke}{rgb}{0.000000,0.000000,0.000000}%
\pgfsetstrokecolor{currentstroke}%
\pgfsetdash{}{0pt}%
\pgfpathmoveto{\pgfqpoint{1.461523in}{1.253441in}}%
\pgfpathlineto{\pgfqpoint{1.499108in}{1.262351in}}%
\pgfpathlineto{\pgfqpoint{1.461724in}{1.301246in}}%
\pgfpathlineto{\pgfqpoint{1.424034in}{1.292444in}}%
\pgfpathclose%
\pgfusepath{fill}%
\end{pgfscope}%
\begin{pgfscope}%
\pgfpathrectangle{\pgfqpoint{0.150000in}{0.150000in}}{\pgfqpoint{2.700000in}{1.950000in}}%
\pgfusepath{clip}%
\pgfsetbuttcap%
\pgfsetroundjoin%
\definecolor{currentfill}{rgb}{0.704580,0.740962,0.791896}%
\pgfsetfillcolor{currentfill}%
\pgfsetlinewidth{0.000000pt}%
\definecolor{currentstroke}{rgb}{0.000000,0.000000,0.000000}%
\pgfsetstrokecolor{currentstroke}%
\pgfsetdash{}{0pt}%
\pgfpathmoveto{\pgfqpoint{1.424034in}{1.292444in}}%
\pgfpathlineto{\pgfqpoint{1.461724in}{1.301246in}}%
\pgfpathlineto{\pgfqpoint{1.424336in}{1.340147in}}%
\pgfpathlineto{\pgfqpoint{1.386540in}{1.331451in}}%
\pgfpathclose%
\pgfusepath{fill}%
\end{pgfscope}%
\begin{pgfscope}%
\pgfpathrectangle{\pgfqpoint{0.150000in}{0.150000in}}{\pgfqpoint{2.700000in}{1.950000in}}%
\pgfusepath{clip}%
\pgfsetbuttcap%
\pgfsetroundjoin%
\definecolor{currentfill}{rgb}{0.673483,0.713695,0.769991}%
\pgfsetfillcolor{currentfill}%
\pgfsetlinewidth{0.000000pt}%
\definecolor{currentstroke}{rgb}{0.000000,0.000000,0.000000}%
\pgfsetstrokecolor{currentstroke}%
\pgfsetdash{}{0pt}%
\pgfpathmoveto{\pgfqpoint{1.386540in}{1.331451in}}%
\pgfpathlineto{\pgfqpoint{1.424336in}{1.340147in}}%
\pgfpathlineto{\pgfqpoint{1.386942in}{1.379053in}}%
\pgfpathlineto{\pgfqpoint{1.349041in}{1.370464in}}%
\pgfpathclose%
\pgfusepath{fill}%
\end{pgfscope}%
\begin{pgfscope}%
\pgfpathrectangle{\pgfqpoint{0.150000in}{0.150000in}}{\pgfqpoint{2.700000in}{1.950000in}}%
\pgfusepath{clip}%
\pgfsetbuttcap%
\pgfsetroundjoin%
\definecolor{currentfill}{rgb}{0.642387,0.686428,0.748085}%
\pgfsetfillcolor{currentfill}%
\pgfsetlinewidth{0.000000pt}%
\definecolor{currentstroke}{rgb}{0.000000,0.000000,0.000000}%
\pgfsetstrokecolor{currentstroke}%
\pgfsetdash{}{0pt}%
\pgfpathmoveto{\pgfqpoint{1.349041in}{1.370464in}}%
\pgfpathlineto{\pgfqpoint{1.386942in}{1.379053in}}%
\pgfpathlineto{\pgfqpoint{1.349544in}{1.417965in}}%
\pgfpathlineto{\pgfqpoint{1.311536in}{1.409482in}}%
\pgfpathclose%
\pgfusepath{fill}%
\end{pgfscope}%
\begin{pgfscope}%
\pgfpathrectangle{\pgfqpoint{0.150000in}{0.150000in}}{\pgfqpoint{2.700000in}{1.950000in}}%
\pgfusepath{clip}%
\pgfsetbuttcap%
\pgfsetroundjoin%
\definecolor{currentfill}{rgb}{0.611290,0.659161,0.726180}%
\pgfsetfillcolor{currentfill}%
\pgfsetlinewidth{0.000000pt}%
\definecolor{currentstroke}{rgb}{0.000000,0.000000,0.000000}%
\pgfsetstrokecolor{currentstroke}%
\pgfsetdash{}{0pt}%
\pgfpathmoveto{\pgfqpoint{1.311536in}{1.409482in}}%
\pgfpathlineto{\pgfqpoint{1.349544in}{1.417965in}}%
\pgfpathlineto{\pgfqpoint{1.312140in}{1.456881in}}%
\pgfpathlineto{\pgfqpoint{1.274027in}{1.448505in}}%
\pgfpathclose%
\pgfusepath{fill}%
\end{pgfscope}%
\begin{pgfscope}%
\pgfpathrectangle{\pgfqpoint{0.150000in}{0.150000in}}{\pgfqpoint{2.700000in}{1.950000in}}%
\pgfusepath{clip}%
\pgfsetbuttcap%
\pgfsetroundjoin%
\definecolor{currentfill}{rgb}{0.580193,0.631893,0.704274}%
\pgfsetfillcolor{currentfill}%
\pgfsetlinewidth{0.000000pt}%
\definecolor{currentstroke}{rgb}{0.000000,0.000000,0.000000}%
\pgfsetstrokecolor{currentstroke}%
\pgfsetdash{}{0pt}%
\pgfpathmoveto{\pgfqpoint{1.274027in}{1.448505in}}%
\pgfpathlineto{\pgfqpoint{1.312140in}{1.456881in}}%
\pgfpathlineto{\pgfqpoint{1.274731in}{1.495803in}}%
\pgfpathlineto{\pgfqpoint{1.236512in}{1.487534in}}%
\pgfpathclose%
\pgfusepath{fill}%
\end{pgfscope}%
\begin{pgfscope}%
\pgfpathrectangle{\pgfqpoint{0.150000in}{0.150000in}}{\pgfqpoint{2.700000in}{1.950000in}}%
\pgfusepath{clip}%
\pgfsetbuttcap%
\pgfsetroundjoin%
\definecolor{currentfill}{rgb}{0.555316,0.610080,0.686749}%
\pgfsetfillcolor{currentfill}%
\pgfsetlinewidth{0.000000pt}%
\definecolor{currentstroke}{rgb}{0.000000,0.000000,0.000000}%
\pgfsetstrokecolor{currentstroke}%
\pgfsetdash{}{0pt}%
\pgfpathmoveto{\pgfqpoint{1.236512in}{1.487534in}}%
\pgfpathlineto{\pgfqpoint{1.274731in}{1.495803in}}%
\pgfpathlineto{\pgfqpoint{1.237317in}{1.534730in}}%
\pgfpathlineto{\pgfqpoint{1.198993in}{1.526568in}}%
\pgfpathclose%
\pgfusepath{fill}%
\end{pgfscope}%
\begin{pgfscope}%
\pgfpathrectangle{\pgfqpoint{0.150000in}{0.150000in}}{\pgfqpoint{2.700000in}{1.950000in}}%
\pgfusepath{clip}%
\pgfsetbuttcap%
\pgfsetroundjoin%
\definecolor{currentfill}{rgb}{0.524219,0.582812,0.664844}%
\pgfsetfillcolor{currentfill}%
\pgfsetlinewidth{0.000000pt}%
\definecolor{currentstroke}{rgb}{0.000000,0.000000,0.000000}%
\pgfsetstrokecolor{currentstroke}%
\pgfsetdash{}{0pt}%
\pgfpathmoveto{\pgfqpoint{1.198993in}{1.526568in}}%
\pgfpathlineto{\pgfqpoint{1.237317in}{1.534730in}}%
\pgfpathlineto{\pgfqpoint{1.199899in}{1.573662in}}%
\pgfpathlineto{\pgfqpoint{1.161468in}{1.565607in}}%
\pgfpathclose%
\pgfusepath{fill}%
\end{pgfscope}%
\begin{pgfscope}%
\pgfpathrectangle{\pgfqpoint{0.150000in}{0.150000in}}{\pgfqpoint{2.700000in}{1.950000in}}%
\pgfusepath{clip}%
\pgfsetbuttcap%
\pgfsetroundjoin%
\definecolor{currentfill}{rgb}{0.493122,0.555545,0.642938}%
\pgfsetfillcolor{currentfill}%
\pgfsetlinewidth{0.000000pt}%
\definecolor{currentstroke}{rgb}{0.000000,0.000000,0.000000}%
\pgfsetstrokecolor{currentstroke}%
\pgfsetdash{}{0pt}%
\pgfpathmoveto{\pgfqpoint{1.161468in}{1.565607in}}%
\pgfpathlineto{\pgfqpoint{1.199899in}{1.573662in}}%
\pgfpathlineto{\pgfqpoint{1.162475in}{1.612600in}}%
\pgfpathlineto{\pgfqpoint{1.123939in}{1.604651in}}%
\pgfpathclose%
\pgfusepath{fill}%
\end{pgfscope}%
\begin{pgfscope}%
\pgfpathrectangle{\pgfqpoint{0.150000in}{0.150000in}}{\pgfqpoint{2.700000in}{1.950000in}}%
\pgfusepath{clip}%
\pgfsetbuttcap%
\pgfsetroundjoin%
\definecolor{currentfill}{rgb}{0.462025,0.528278,0.621032}%
\pgfsetfillcolor{currentfill}%
\pgfsetlinewidth{0.000000pt}%
\definecolor{currentstroke}{rgb}{0.000000,0.000000,0.000000}%
\pgfsetstrokecolor{currentstroke}%
\pgfsetdash{}{0pt}%
\pgfpathmoveto{\pgfqpoint{1.123939in}{1.604651in}}%
\pgfpathlineto{\pgfqpoint{1.162475in}{1.612600in}}%
\pgfpathlineto{\pgfqpoint{1.125046in}{1.651542in}}%
\pgfpathlineto{\pgfqpoint{1.086404in}{1.643701in}}%
\pgfpathclose%
\pgfusepath{fill}%
\end{pgfscope}%
\begin{pgfscope}%
\pgfpathrectangle{\pgfqpoint{0.150000in}{0.150000in}}{\pgfqpoint{2.700000in}{1.950000in}}%
\pgfusepath{clip}%
\pgfsetbuttcap%
\pgfsetroundjoin%
\definecolor{currentfill}{rgb}{0.891728,0.803539,0.810524}%
\pgfsetfillcolor{currentfill}%
\pgfsetlinewidth{0.000000pt}%
\definecolor{currentstroke}{rgb}{0.000000,0.000000,0.000000}%
\pgfsetstrokecolor{currentstroke}%
\pgfsetdash{}{0pt}%
\pgfpathmoveto{\pgfqpoint{1.987053in}{0.658472in}}%
\pgfpathlineto{\pgfqpoint{2.023256in}{0.669037in}}%
\pgfpathlineto{\pgfqpoint{1.985842in}{0.707960in}}%
\pgfpathlineto{\pgfqpoint{1.949534in}{0.697503in}}%
\pgfpathclose%
\pgfusepath{fill}%
\end{pgfscope}%
\begin{pgfscope}%
\pgfpathrectangle{\pgfqpoint{0.150000in}{0.150000in}}{\pgfqpoint{2.700000in}{1.950000in}}%
\pgfusepath{clip}%
\pgfsetbuttcap%
\pgfsetroundjoin%
\definecolor{currentfill}{rgb}{0.910723,0.838006,0.843765}%
\pgfsetfillcolor{currentfill}%
\pgfsetlinewidth{0.000000pt}%
\definecolor{currentstroke}{rgb}{0.000000,0.000000,0.000000}%
\pgfsetstrokecolor{currentstroke}%
\pgfsetdash{}{0pt}%
\pgfpathmoveto{\pgfqpoint{1.949534in}{0.697503in}}%
\pgfpathlineto{\pgfqpoint{1.985842in}{0.707960in}}%
\pgfpathlineto{\pgfqpoint{1.948424in}{0.746889in}}%
\pgfpathlineto{\pgfqpoint{1.912009in}{0.736538in}}%
\pgfpathclose%
\pgfusepath{fill}%
\end{pgfscope}%
\begin{pgfscope}%
\pgfpathrectangle{\pgfqpoint{0.150000in}{0.150000in}}{\pgfqpoint{2.700000in}{1.950000in}}%
\pgfusepath{clip}%
\pgfsetbuttcap%
\pgfsetroundjoin%
\definecolor{currentfill}{rgb}{0.929718,0.872472,0.877007}%
\pgfsetfillcolor{currentfill}%
\pgfsetlinewidth{0.000000pt}%
\definecolor{currentstroke}{rgb}{0.000000,0.000000,0.000000}%
\pgfsetstrokecolor{currentstroke}%
\pgfsetdash{}{0pt}%
\pgfpathmoveto{\pgfqpoint{1.912009in}{0.736538in}}%
\pgfpathlineto{\pgfqpoint{1.948424in}{0.746889in}}%
\pgfpathlineto{\pgfqpoint{1.911000in}{0.785823in}}%
\pgfpathlineto{\pgfqpoint{1.874480in}{0.775579in}}%
\pgfpathclose%
\pgfusepath{fill}%
\end{pgfscope}%
\begin{pgfscope}%
\pgfpathrectangle{\pgfqpoint{0.150000in}{0.150000in}}{\pgfqpoint{2.700000in}{1.950000in}}%
\pgfusepath{clip}%
\pgfsetbuttcap%
\pgfsetroundjoin%
\definecolor{currentfill}{rgb}{0.948713,0.906939,0.910248}%
\pgfsetfillcolor{currentfill}%
\pgfsetlinewidth{0.000000pt}%
\definecolor{currentstroke}{rgb}{0.000000,0.000000,0.000000}%
\pgfsetstrokecolor{currentstroke}%
\pgfsetdash{}{0pt}%
\pgfpathmoveto{\pgfqpoint{1.874480in}{0.775579in}}%
\pgfpathlineto{\pgfqpoint{1.911000in}{0.785823in}}%
\pgfpathlineto{\pgfqpoint{1.873571in}{0.824762in}}%
\pgfpathlineto{\pgfqpoint{1.836945in}{0.814626in}}%
\pgfpathclose%
\pgfusepath{fill}%
\end{pgfscope}%
\begin{pgfscope}%
\pgfpathrectangle{\pgfqpoint{0.150000in}{0.150000in}}{\pgfqpoint{2.700000in}{1.950000in}}%
\pgfusepath{clip}%
\pgfsetbuttcap%
\pgfsetroundjoin%
\definecolor{currentfill}{rgb}{0.967708,0.941406,0.943490}%
\pgfsetfillcolor{currentfill}%
\pgfsetlinewidth{0.000000pt}%
\definecolor{currentstroke}{rgb}{0.000000,0.000000,0.000000}%
\pgfsetstrokecolor{currentstroke}%
\pgfsetdash{}{0pt}%
\pgfpathmoveto{\pgfqpoint{1.836945in}{0.814626in}}%
\pgfpathlineto{\pgfqpoint{1.873571in}{0.824762in}}%
\pgfpathlineto{\pgfqpoint{1.836138in}{0.863707in}}%
\pgfpathlineto{\pgfqpoint{1.799406in}{0.853677in}}%
\pgfpathclose%
\pgfusepath{fill}%
\end{pgfscope}%
\begin{pgfscope}%
\pgfpathrectangle{\pgfqpoint{0.150000in}{0.150000in}}{\pgfqpoint{2.700000in}{1.950000in}}%
\pgfusepath{clip}%
\pgfsetbuttcap%
\pgfsetroundjoin%
\definecolor{currentfill}{rgb}{0.986703,0.975873,0.976731}%
\pgfsetfillcolor{currentfill}%
\pgfsetlinewidth{0.000000pt}%
\definecolor{currentstroke}{rgb}{0.000000,0.000000,0.000000}%
\pgfsetstrokecolor{currentstroke}%
\pgfsetdash{}{0pt}%
\pgfpathmoveto{\pgfqpoint{1.799406in}{0.853677in}}%
\pgfpathlineto{\pgfqpoint{1.836138in}{0.863707in}}%
\pgfpathlineto{\pgfqpoint{1.798699in}{0.902657in}}%
\pgfpathlineto{\pgfqpoint{1.761861in}{0.892734in}}%
\pgfpathclose%
\pgfusepath{fill}%
\end{pgfscope}%
\begin{pgfscope}%
\pgfpathrectangle{\pgfqpoint{0.150000in}{0.150000in}}{\pgfqpoint{2.700000in}{1.950000in}}%
\pgfusepath{clip}%
\pgfsetbuttcap%
\pgfsetroundjoin%
\definecolor{currentfill}{rgb}{0.990671,0.991820,0.993428}%
\pgfsetfillcolor{currentfill}%
\pgfsetlinewidth{0.000000pt}%
\definecolor{currentstroke}{rgb}{0.000000,0.000000,0.000000}%
\pgfsetstrokecolor{currentstroke}%
\pgfsetdash{}{0pt}%
\pgfpathmoveto{\pgfqpoint{1.761861in}{0.892734in}}%
\pgfpathlineto{\pgfqpoint{1.798699in}{0.902657in}}%
\pgfpathlineto{\pgfqpoint{1.761255in}{0.941612in}}%
\pgfpathlineto{\pgfqpoint{1.724311in}{0.931796in}}%
\pgfpathclose%
\pgfusepath{fill}%
\end{pgfscope}%
\begin{pgfscope}%
\pgfpathrectangle{\pgfqpoint{0.150000in}{0.150000in}}{\pgfqpoint{2.700000in}{1.950000in}}%
\pgfusepath{clip}%
\pgfsetbuttcap%
\pgfsetroundjoin%
\definecolor{currentfill}{rgb}{0.959574,0.964553,0.971523}%
\pgfsetfillcolor{currentfill}%
\pgfsetlinewidth{0.000000pt}%
\definecolor{currentstroke}{rgb}{0.000000,0.000000,0.000000}%
\pgfsetstrokecolor{currentstroke}%
\pgfsetdash{}{0pt}%
\pgfpathmoveto{\pgfqpoint{1.724311in}{0.931796in}}%
\pgfpathlineto{\pgfqpoint{1.761255in}{0.941612in}}%
\pgfpathlineto{\pgfqpoint{1.723806in}{0.980572in}}%
\pgfpathlineto{\pgfqpoint{1.686757in}{0.970863in}}%
\pgfpathclose%
\pgfusepath{fill}%
\end{pgfscope}%
\begin{pgfscope}%
\pgfpathrectangle{\pgfqpoint{0.150000in}{0.150000in}}{\pgfqpoint{2.700000in}{1.950000in}}%
\pgfusepath{clip}%
\pgfsetbuttcap%
\pgfsetroundjoin%
\definecolor{currentfill}{rgb}{0.934697,0.942739,0.953998}%
\pgfsetfillcolor{currentfill}%
\pgfsetlinewidth{0.000000pt}%
\definecolor{currentstroke}{rgb}{0.000000,0.000000,0.000000}%
\pgfsetstrokecolor{currentstroke}%
\pgfsetdash{}{0pt}%
\pgfpathmoveto{\pgfqpoint{1.686757in}{0.970863in}}%
\pgfpathlineto{\pgfqpoint{1.723806in}{0.980572in}}%
\pgfpathlineto{\pgfqpoint{1.686352in}{1.019538in}}%
\pgfpathlineto{\pgfqpoint{1.649197in}{1.009936in}}%
\pgfpathclose%
\pgfusepath{fill}%
\end{pgfscope}%
\begin{pgfscope}%
\pgfpathrectangle{\pgfqpoint{0.150000in}{0.150000in}}{\pgfqpoint{2.700000in}{1.950000in}}%
\pgfusepath{clip}%
\pgfsetbuttcap%
\pgfsetroundjoin%
\definecolor{currentfill}{rgb}{0.903600,0.915472,0.932093}%
\pgfsetfillcolor{currentfill}%
\pgfsetlinewidth{0.000000pt}%
\definecolor{currentstroke}{rgb}{0.000000,0.000000,0.000000}%
\pgfsetstrokecolor{currentstroke}%
\pgfsetdash{}{0pt}%
\pgfpathmoveto{\pgfqpoint{1.649197in}{1.009936in}}%
\pgfpathlineto{\pgfqpoint{1.686352in}{1.019538in}}%
\pgfpathlineto{\pgfqpoint{1.648893in}{1.058508in}}%
\pgfpathlineto{\pgfqpoint{1.611632in}{1.049014in}}%
\pgfpathclose%
\pgfusepath{fill}%
\end{pgfscope}%
\begin{pgfscope}%
\pgfpathrectangle{\pgfqpoint{0.150000in}{0.150000in}}{\pgfqpoint{2.700000in}{1.950000in}}%
\pgfusepath{clip}%
\pgfsetbuttcap%
\pgfsetroundjoin%
\definecolor{currentfill}{rgb}{0.872503,0.888205,0.910187}%
\pgfsetfillcolor{currentfill}%
\pgfsetlinewidth{0.000000pt}%
\definecolor{currentstroke}{rgb}{0.000000,0.000000,0.000000}%
\pgfsetstrokecolor{currentstroke}%
\pgfsetdash{}{0pt}%
\pgfpathmoveto{\pgfqpoint{1.611632in}{1.049014in}}%
\pgfpathlineto{\pgfqpoint{1.648893in}{1.058508in}}%
\pgfpathlineto{\pgfqpoint{1.611430in}{1.097485in}}%
\pgfpathlineto{\pgfqpoint{1.574062in}{1.088097in}}%
\pgfpathclose%
\pgfusepath{fill}%
\end{pgfscope}%
\begin{pgfscope}%
\pgfpathrectangle{\pgfqpoint{0.150000in}{0.150000in}}{\pgfqpoint{2.700000in}{1.950000in}}%
\pgfusepath{clip}%
\pgfsetbuttcap%
\pgfsetroundjoin%
\definecolor{currentfill}{rgb}{0.841406,0.860938,0.888281}%
\pgfsetfillcolor{currentfill}%
\pgfsetlinewidth{0.000000pt}%
\definecolor{currentstroke}{rgb}{0.000000,0.000000,0.000000}%
\pgfsetstrokecolor{currentstroke}%
\pgfsetdash{}{0pt}%
\pgfpathmoveto{\pgfqpoint{1.574062in}{1.088097in}}%
\pgfpathlineto{\pgfqpoint{1.611430in}{1.097485in}}%
\pgfpathlineto{\pgfqpoint{1.573961in}{1.136466in}}%
\pgfpathlineto{\pgfqpoint{1.536486in}{1.127185in}}%
\pgfpathclose%
\pgfusepath{fill}%
\end{pgfscope}%
\begin{pgfscope}%
\pgfpathrectangle{\pgfqpoint{0.150000in}{0.150000in}}{\pgfqpoint{2.700000in}{1.950000in}}%
\pgfusepath{clip}%
\pgfsetbuttcap%
\pgfsetroundjoin%
\definecolor{currentfill}{rgb}{0.810309,0.833670,0.866376}%
\pgfsetfillcolor{currentfill}%
\pgfsetlinewidth{0.000000pt}%
\definecolor{currentstroke}{rgb}{0.000000,0.000000,0.000000}%
\pgfsetstrokecolor{currentstroke}%
\pgfsetdash{}{0pt}%
\pgfpathmoveto{\pgfqpoint{1.536486in}{1.127185in}}%
\pgfpathlineto{\pgfqpoint{1.573961in}{1.136466in}}%
\pgfpathlineto{\pgfqpoint{1.536486in}{1.175452in}}%
\pgfpathlineto{\pgfqpoint{1.498906in}{1.166279in}}%
\pgfpathclose%
\pgfusepath{fill}%
\end{pgfscope}%
\begin{pgfscope}%
\pgfpathrectangle{\pgfqpoint{0.150000in}{0.150000in}}{\pgfqpoint{2.700000in}{1.950000in}}%
\pgfusepath{clip}%
\pgfsetbuttcap%
\pgfsetroundjoin%
\definecolor{currentfill}{rgb}{0.779213,0.806403,0.844470}%
\pgfsetfillcolor{currentfill}%
\pgfsetlinewidth{0.000000pt}%
\definecolor{currentstroke}{rgb}{0.000000,0.000000,0.000000}%
\pgfsetstrokecolor{currentstroke}%
\pgfsetdash{}{0pt}%
\pgfpathmoveto{\pgfqpoint{1.498906in}{1.166279in}}%
\pgfpathlineto{\pgfqpoint{1.536486in}{1.175452in}}%
\pgfpathlineto{\pgfqpoint{1.499007in}{1.214444in}}%
\pgfpathlineto{\pgfqpoint{1.461321in}{1.205378in}}%
\pgfpathclose%
\pgfusepath{fill}%
\end{pgfscope}%
\begin{pgfscope}%
\pgfpathrectangle{\pgfqpoint{0.150000in}{0.150000in}}{\pgfqpoint{2.700000in}{1.950000in}}%
\pgfusepath{clip}%
\pgfsetbuttcap%
\pgfsetroundjoin%
\definecolor{currentfill}{rgb}{0.748116,0.779136,0.822564}%
\pgfsetfillcolor{currentfill}%
\pgfsetlinewidth{0.000000pt}%
\definecolor{currentstroke}{rgb}{0.000000,0.000000,0.000000}%
\pgfsetstrokecolor{currentstroke}%
\pgfsetdash{}{0pt}%
\pgfpathmoveto{\pgfqpoint{1.461321in}{1.205378in}}%
\pgfpathlineto{\pgfqpoint{1.499007in}{1.214444in}}%
\pgfpathlineto{\pgfqpoint{1.461523in}{1.253441in}}%
\pgfpathlineto{\pgfqpoint{1.423731in}{1.244482in}}%
\pgfpathclose%
\pgfusepath{fill}%
\end{pgfscope}%
\begin{pgfscope}%
\pgfpathrectangle{\pgfqpoint{0.150000in}{0.150000in}}{\pgfqpoint{2.700000in}{1.950000in}}%
\pgfusepath{clip}%
\pgfsetbuttcap%
\pgfsetroundjoin%
\definecolor{currentfill}{rgb}{0.717019,0.751869,0.800659}%
\pgfsetfillcolor{currentfill}%
\pgfsetlinewidth{0.000000pt}%
\definecolor{currentstroke}{rgb}{0.000000,0.000000,0.000000}%
\pgfsetstrokecolor{currentstroke}%
\pgfsetdash{}{0pt}%
\pgfpathmoveto{\pgfqpoint{1.423731in}{1.244482in}}%
\pgfpathlineto{\pgfqpoint{1.461523in}{1.253441in}}%
\pgfpathlineto{\pgfqpoint{1.424034in}{1.292444in}}%
\pgfpathlineto{\pgfqpoint{1.386135in}{1.283592in}}%
\pgfpathclose%
\pgfusepath{fill}%
\end{pgfscope}%
\begin{pgfscope}%
\pgfpathrectangle{\pgfqpoint{0.150000in}{0.150000in}}{\pgfqpoint{2.700000in}{1.950000in}}%
\pgfusepath{clip}%
\pgfsetbuttcap%
\pgfsetroundjoin%
\definecolor{currentfill}{rgb}{0.692142,0.730055,0.783134}%
\pgfsetfillcolor{currentfill}%
\pgfsetlinewidth{0.000000pt}%
\definecolor{currentstroke}{rgb}{0.000000,0.000000,0.000000}%
\pgfsetstrokecolor{currentstroke}%
\pgfsetdash{}{0pt}%
\pgfpathmoveto{\pgfqpoint{1.386135in}{1.283592in}}%
\pgfpathlineto{\pgfqpoint{1.424034in}{1.292444in}}%
\pgfpathlineto{\pgfqpoint{1.386540in}{1.331451in}}%
\pgfpathlineto{\pgfqpoint{1.348535in}{1.322707in}}%
\pgfpathclose%
\pgfusepath{fill}%
\end{pgfscope}%
\begin{pgfscope}%
\pgfpathrectangle{\pgfqpoint{0.150000in}{0.150000in}}{\pgfqpoint{2.700000in}{1.950000in}}%
\pgfusepath{clip}%
\pgfsetbuttcap%
\pgfsetroundjoin%
\definecolor{currentfill}{rgb}{0.661045,0.702788,0.761229}%
\pgfsetfillcolor{currentfill}%
\pgfsetlinewidth{0.000000pt}%
\definecolor{currentstroke}{rgb}{0.000000,0.000000,0.000000}%
\pgfsetstrokecolor{currentstroke}%
\pgfsetdash{}{0pt}%
\pgfpathmoveto{\pgfqpoint{1.348535in}{1.322707in}}%
\pgfpathlineto{\pgfqpoint{1.386540in}{1.331451in}}%
\pgfpathlineto{\pgfqpoint{1.349041in}{1.370464in}}%
\pgfpathlineto{\pgfqpoint{1.310929in}{1.361827in}}%
\pgfpathclose%
\pgfusepath{fill}%
\end{pgfscope}%
\begin{pgfscope}%
\pgfpathrectangle{\pgfqpoint{0.150000in}{0.150000in}}{\pgfqpoint{2.700000in}{1.950000in}}%
\pgfusepath{clip}%
\pgfsetbuttcap%
\pgfsetroundjoin%
\definecolor{currentfill}{rgb}{0.629948,0.675521,0.739323}%
\pgfsetfillcolor{currentfill}%
\pgfsetlinewidth{0.000000pt}%
\definecolor{currentstroke}{rgb}{0.000000,0.000000,0.000000}%
\pgfsetstrokecolor{currentstroke}%
\pgfsetdash{}{0pt}%
\pgfpathmoveto{\pgfqpoint{1.310929in}{1.361827in}}%
\pgfpathlineto{\pgfqpoint{1.349041in}{1.370464in}}%
\pgfpathlineto{\pgfqpoint{1.311536in}{1.409482in}}%
\pgfpathlineto{\pgfqpoint{1.273319in}{1.400952in}}%
\pgfpathclose%
\pgfusepath{fill}%
\end{pgfscope}%
\begin{pgfscope}%
\pgfpathrectangle{\pgfqpoint{0.150000in}{0.150000in}}{\pgfqpoint{2.700000in}{1.950000in}}%
\pgfusepath{clip}%
\pgfsetbuttcap%
\pgfsetroundjoin%
\definecolor{currentfill}{rgb}{0.598851,0.648254,0.717417}%
\pgfsetfillcolor{currentfill}%
\pgfsetlinewidth{0.000000pt}%
\definecolor{currentstroke}{rgb}{0.000000,0.000000,0.000000}%
\pgfsetstrokecolor{currentstroke}%
\pgfsetdash{}{0pt}%
\pgfpathmoveto{\pgfqpoint{1.273319in}{1.400952in}}%
\pgfpathlineto{\pgfqpoint{1.311536in}{1.409482in}}%
\pgfpathlineto{\pgfqpoint{1.274027in}{1.448505in}}%
\pgfpathlineto{\pgfqpoint{1.235703in}{1.440083in}}%
\pgfpathclose%
\pgfusepath{fill}%
\end{pgfscope}%
\begin{pgfscope}%
\pgfpathrectangle{\pgfqpoint{0.150000in}{0.150000in}}{\pgfqpoint{2.700000in}{1.950000in}}%
\pgfusepath{clip}%
\pgfsetbuttcap%
\pgfsetroundjoin%
\definecolor{currentfill}{rgb}{0.567754,0.620987,0.695512}%
\pgfsetfillcolor{currentfill}%
\pgfsetlinewidth{0.000000pt}%
\definecolor{currentstroke}{rgb}{0.000000,0.000000,0.000000}%
\pgfsetstrokecolor{currentstroke}%
\pgfsetdash{}{0pt}%
\pgfpathmoveto{\pgfqpoint{1.235703in}{1.440083in}}%
\pgfpathlineto{\pgfqpoint{1.274027in}{1.448505in}}%
\pgfpathlineto{\pgfqpoint{1.236512in}{1.487534in}}%
\pgfpathlineto{\pgfqpoint{1.198082in}{1.479219in}}%
\pgfpathclose%
\pgfusepath{fill}%
\end{pgfscope}%
\begin{pgfscope}%
\pgfpathrectangle{\pgfqpoint{0.150000in}{0.150000in}}{\pgfqpoint{2.700000in}{1.950000in}}%
\pgfusepath{clip}%
\pgfsetbuttcap%
\pgfsetroundjoin%
\definecolor{currentfill}{rgb}{0.536657,0.593719,0.673606}%
\pgfsetfillcolor{currentfill}%
\pgfsetlinewidth{0.000000pt}%
\definecolor{currentstroke}{rgb}{0.000000,0.000000,0.000000}%
\pgfsetstrokecolor{currentstroke}%
\pgfsetdash{}{0pt}%
\pgfpathmoveto{\pgfqpoint{1.198082in}{1.479219in}}%
\pgfpathlineto{\pgfqpoint{1.236512in}{1.487534in}}%
\pgfpathlineto{\pgfqpoint{1.198993in}{1.526568in}}%
\pgfpathlineto{\pgfqpoint{1.160456in}{1.518360in}}%
\pgfpathclose%
\pgfusepath{fill}%
\end{pgfscope}%
\begin{pgfscope}%
\pgfpathrectangle{\pgfqpoint{0.150000in}{0.150000in}}{\pgfqpoint{2.700000in}{1.950000in}}%
\pgfusepath{clip}%
\pgfsetbuttcap%
\pgfsetroundjoin%
\definecolor{currentfill}{rgb}{0.505561,0.566452,0.651700}%
\pgfsetfillcolor{currentfill}%
\pgfsetlinewidth{0.000000pt}%
\definecolor{currentstroke}{rgb}{0.000000,0.000000,0.000000}%
\pgfsetstrokecolor{currentstroke}%
\pgfsetdash{}{0pt}%
\pgfpathmoveto{\pgfqpoint{1.160456in}{1.518360in}}%
\pgfpathlineto{\pgfqpoint{1.198993in}{1.526568in}}%
\pgfpathlineto{\pgfqpoint{1.161468in}{1.565607in}}%
\pgfpathlineto{\pgfqpoint{1.122825in}{1.557507in}}%
\pgfpathclose%
\pgfusepath{fill}%
\end{pgfscope}%
\begin{pgfscope}%
\pgfpathrectangle{\pgfqpoint{0.150000in}{0.150000in}}{\pgfqpoint{2.700000in}{1.950000in}}%
\pgfusepath{clip}%
\pgfsetbuttcap%
\pgfsetroundjoin%
\definecolor{currentfill}{rgb}{0.474464,0.539185,0.629795}%
\pgfsetfillcolor{currentfill}%
\pgfsetlinewidth{0.000000pt}%
\definecolor{currentstroke}{rgb}{0.000000,0.000000,0.000000}%
\pgfsetstrokecolor{currentstroke}%
\pgfsetdash{}{0pt}%
\pgfpathmoveto{\pgfqpoint{1.122825in}{1.557507in}}%
\pgfpathlineto{\pgfqpoint{1.161468in}{1.565607in}}%
\pgfpathlineto{\pgfqpoint{1.123939in}{1.604651in}}%
\pgfpathlineto{\pgfqpoint{1.085189in}{1.596659in}}%
\pgfpathclose%
\pgfusepath{fill}%
\end{pgfscope}%
\begin{pgfscope}%
\pgfpathrectangle{\pgfqpoint{0.150000in}{0.150000in}}{\pgfqpoint{2.700000in}{1.950000in}}%
\pgfusepath{clip}%
\pgfsetbuttcap%
\pgfsetroundjoin%
\definecolor{currentfill}{rgb}{0.449586,0.517371,0.612270}%
\pgfsetfillcolor{currentfill}%
\pgfsetlinewidth{0.000000pt}%
\definecolor{currentstroke}{rgb}{0.000000,0.000000,0.000000}%
\pgfsetstrokecolor{currentstroke}%
\pgfsetdash{}{0pt}%
\pgfpathmoveto{\pgfqpoint{1.085189in}{1.596659in}}%
\pgfpathlineto{\pgfqpoint{1.123939in}{1.604651in}}%
\pgfpathlineto{\pgfqpoint{1.086404in}{1.643701in}}%
\pgfpathlineto{\pgfqpoint{1.047548in}{1.635816in}}%
\pgfpathclose%
\pgfusepath{fill}%
\end{pgfscope}%
\begin{pgfscope}%
\pgfpathrectangle{\pgfqpoint{0.150000in}{0.150000in}}{\pgfqpoint{2.700000in}{1.950000in}}%
\pgfusepath{clip}%
\pgfsetbuttcap%
\pgfsetroundjoin%
\definecolor{currentfill}{rgb}{0.903125,0.824219,0.830469}%
\pgfsetfillcolor{currentfill}%
\pgfsetlinewidth{0.000000pt}%
\definecolor{currentstroke}{rgb}{0.000000,0.000000,0.000000}%
\pgfsetstrokecolor{currentstroke}%
\pgfsetdash{}{0pt}%
\pgfpathmoveto{\pgfqpoint{1.950650in}{0.647850in}}%
\pgfpathlineto{\pgfqpoint{1.987053in}{0.658472in}}%
\pgfpathlineto{\pgfqpoint{1.949534in}{0.697503in}}%
\pgfpathlineto{\pgfqpoint{1.913024in}{0.686987in}}%
\pgfpathclose%
\pgfusepath{fill}%
\end{pgfscope}%
\begin{pgfscope}%
\pgfpathrectangle{\pgfqpoint{0.150000in}{0.150000in}}{\pgfqpoint{2.700000in}{1.950000in}}%
\pgfusepath{clip}%
\pgfsetbuttcap%
\pgfsetroundjoin%
\definecolor{currentfill}{rgb}{0.922120,0.858686,0.863710}%
\pgfsetfillcolor{currentfill}%
\pgfsetlinewidth{0.000000pt}%
\definecolor{currentstroke}{rgb}{0.000000,0.000000,0.000000}%
\pgfsetstrokecolor{currentstroke}%
\pgfsetdash{}{0pt}%
\pgfpathmoveto{\pgfqpoint{1.913024in}{0.686987in}}%
\pgfpathlineto{\pgfqpoint{1.949534in}{0.697503in}}%
\pgfpathlineto{\pgfqpoint{1.912009in}{0.736538in}}%
\pgfpathlineto{\pgfqpoint{1.875393in}{0.726130in}}%
\pgfpathclose%
\pgfusepath{fill}%
\end{pgfscope}%
\begin{pgfscope}%
\pgfpathrectangle{\pgfqpoint{0.150000in}{0.150000in}}{\pgfqpoint{2.700000in}{1.950000in}}%
\pgfusepath{clip}%
\pgfsetbuttcap%
\pgfsetroundjoin%
\definecolor{currentfill}{rgb}{0.941115,0.893153,0.896952}%
\pgfsetfillcolor{currentfill}%
\pgfsetlinewidth{0.000000pt}%
\definecolor{currentstroke}{rgb}{0.000000,0.000000,0.000000}%
\pgfsetstrokecolor{currentstroke}%
\pgfsetdash{}{0pt}%
\pgfpathmoveto{\pgfqpoint{1.875393in}{0.726130in}}%
\pgfpathlineto{\pgfqpoint{1.912009in}{0.736538in}}%
\pgfpathlineto{\pgfqpoint{1.874480in}{0.775579in}}%
\pgfpathlineto{\pgfqpoint{1.837758in}{0.765279in}}%
\pgfpathclose%
\pgfusepath{fill}%
\end{pgfscope}%
\begin{pgfscope}%
\pgfpathrectangle{\pgfqpoint{0.150000in}{0.150000in}}{\pgfqpoint{2.700000in}{1.950000in}}%
\pgfusepath{clip}%
\pgfsetbuttcap%
\pgfsetroundjoin%
\definecolor{currentfill}{rgb}{0.960110,0.927619,0.930193}%
\pgfsetfillcolor{currentfill}%
\pgfsetlinewidth{0.000000pt}%
\definecolor{currentstroke}{rgb}{0.000000,0.000000,0.000000}%
\pgfsetstrokecolor{currentstroke}%
\pgfsetdash{}{0pt}%
\pgfpathmoveto{\pgfqpoint{1.837758in}{0.765279in}}%
\pgfpathlineto{\pgfqpoint{1.874480in}{0.775579in}}%
\pgfpathlineto{\pgfqpoint{1.836945in}{0.814626in}}%
\pgfpathlineto{\pgfqpoint{1.800116in}{0.804433in}}%
\pgfpathclose%
\pgfusepath{fill}%
\end{pgfscope}%
\begin{pgfscope}%
\pgfpathrectangle{\pgfqpoint{0.150000in}{0.150000in}}{\pgfqpoint{2.700000in}{1.950000in}}%
\pgfusepath{clip}%
\pgfsetbuttcap%
\pgfsetroundjoin%
\definecolor{currentfill}{rgb}{0.975306,0.955193,0.956786}%
\pgfsetfillcolor{currentfill}%
\pgfsetlinewidth{0.000000pt}%
\definecolor{currentstroke}{rgb}{0.000000,0.000000,0.000000}%
\pgfsetstrokecolor{currentstroke}%
\pgfsetdash{}{0pt}%
\pgfpathmoveto{\pgfqpoint{1.800116in}{0.804433in}}%
\pgfpathlineto{\pgfqpoint{1.836945in}{0.814626in}}%
\pgfpathlineto{\pgfqpoint{1.799406in}{0.853677in}}%
\pgfpathlineto{\pgfqpoint{1.762470in}{0.843592in}}%
\pgfpathclose%
\pgfusepath{fill}%
\end{pgfscope}%
\begin{pgfscope}%
\pgfpathrectangle{\pgfqpoint{0.150000in}{0.150000in}}{\pgfqpoint{2.700000in}{1.950000in}}%
\pgfusepath{clip}%
\pgfsetbuttcap%
\pgfsetroundjoin%
\definecolor{currentfill}{rgb}{0.994301,0.989660,0.990028}%
\pgfsetfillcolor{currentfill}%
\pgfsetlinewidth{0.000000pt}%
\definecolor{currentstroke}{rgb}{0.000000,0.000000,0.000000}%
\pgfsetstrokecolor{currentstroke}%
\pgfsetdash{}{0pt}%
\pgfpathmoveto{\pgfqpoint{1.762470in}{0.843592in}}%
\pgfpathlineto{\pgfqpoint{1.799406in}{0.853677in}}%
\pgfpathlineto{\pgfqpoint{1.761861in}{0.892734in}}%
\pgfpathlineto{\pgfqpoint{1.724819in}{0.882756in}}%
\pgfpathclose%
\pgfusepath{fill}%
\end{pgfscope}%
\begin{pgfscope}%
\pgfpathrectangle{\pgfqpoint{0.150000in}{0.150000in}}{\pgfqpoint{2.700000in}{1.950000in}}%
\pgfusepath{clip}%
\pgfsetbuttcap%
\pgfsetroundjoin%
\definecolor{currentfill}{rgb}{0.978232,0.980913,0.984666}%
\pgfsetfillcolor{currentfill}%
\pgfsetlinewidth{0.000000pt}%
\definecolor{currentstroke}{rgb}{0.000000,0.000000,0.000000}%
\pgfsetstrokecolor{currentstroke}%
\pgfsetdash{}{0pt}%
\pgfpathmoveto{\pgfqpoint{1.724819in}{0.882756in}}%
\pgfpathlineto{\pgfqpoint{1.761861in}{0.892734in}}%
\pgfpathlineto{\pgfqpoint{1.724311in}{0.931796in}}%
\pgfpathlineto{\pgfqpoint{1.687163in}{0.921926in}}%
\pgfpathclose%
\pgfusepath{fill}%
\end{pgfscope}%
\begin{pgfscope}%
\pgfpathrectangle{\pgfqpoint{0.150000in}{0.150000in}}{\pgfqpoint{2.700000in}{1.950000in}}%
\pgfusepath{clip}%
\pgfsetbuttcap%
\pgfsetroundjoin%
\definecolor{currentfill}{rgb}{0.947135,0.953646,0.962760}%
\pgfsetfillcolor{currentfill}%
\pgfsetlinewidth{0.000000pt}%
\definecolor{currentstroke}{rgb}{0.000000,0.000000,0.000000}%
\pgfsetstrokecolor{currentstroke}%
\pgfsetdash{}{0pt}%
\pgfpathmoveto{\pgfqpoint{1.687163in}{0.921926in}}%
\pgfpathlineto{\pgfqpoint{1.724311in}{0.931796in}}%
\pgfpathlineto{\pgfqpoint{1.686757in}{0.970863in}}%
\pgfpathlineto{\pgfqpoint{1.649501in}{0.961101in}}%
\pgfpathclose%
\pgfusepath{fill}%
\end{pgfscope}%
\begin{pgfscope}%
\pgfpathrectangle{\pgfqpoint{0.150000in}{0.150000in}}{\pgfqpoint{2.700000in}{1.950000in}}%
\pgfusepath{clip}%
\pgfsetbuttcap%
\pgfsetroundjoin%
\definecolor{currentfill}{rgb}{0.916039,0.926379,0.940855}%
\pgfsetfillcolor{currentfill}%
\pgfsetlinewidth{0.000000pt}%
\definecolor{currentstroke}{rgb}{0.000000,0.000000,0.000000}%
\pgfsetstrokecolor{currentstroke}%
\pgfsetdash{}{0pt}%
\pgfpathmoveto{\pgfqpoint{1.649501in}{0.961101in}}%
\pgfpathlineto{\pgfqpoint{1.686757in}{0.970863in}}%
\pgfpathlineto{\pgfqpoint{1.649197in}{1.009936in}}%
\pgfpathlineto{\pgfqpoint{1.611835in}{1.000281in}}%
\pgfpathclose%
\pgfusepath{fill}%
\end{pgfscope}%
\begin{pgfscope}%
\pgfpathrectangle{\pgfqpoint{0.150000in}{0.150000in}}{\pgfqpoint{2.700000in}{1.950000in}}%
\pgfusepath{clip}%
\pgfsetbuttcap%
\pgfsetroundjoin%
\definecolor{currentfill}{rgb}{0.884942,0.899112,0.918949}%
\pgfsetfillcolor{currentfill}%
\pgfsetlinewidth{0.000000pt}%
\definecolor{currentstroke}{rgb}{0.000000,0.000000,0.000000}%
\pgfsetstrokecolor{currentstroke}%
\pgfsetdash{}{0pt}%
\pgfpathmoveto{\pgfqpoint{1.611835in}{1.000281in}}%
\pgfpathlineto{\pgfqpoint{1.649197in}{1.009936in}}%
\pgfpathlineto{\pgfqpoint{1.611632in}{1.049014in}}%
\pgfpathlineto{\pgfqpoint{1.574163in}{1.039466in}}%
\pgfpathclose%
\pgfusepath{fill}%
\end{pgfscope}%
\begin{pgfscope}%
\pgfpathrectangle{\pgfqpoint{0.150000in}{0.150000in}}{\pgfqpoint{2.700000in}{1.950000in}}%
\pgfusepath{clip}%
\pgfsetbuttcap%
\pgfsetroundjoin%
\definecolor{currentfill}{rgb}{0.853845,0.871844,0.897044}%
\pgfsetfillcolor{currentfill}%
\pgfsetlinewidth{0.000000pt}%
\definecolor{currentstroke}{rgb}{0.000000,0.000000,0.000000}%
\pgfsetstrokecolor{currentstroke}%
\pgfsetdash{}{0pt}%
\pgfpathmoveto{\pgfqpoint{1.574163in}{1.039466in}}%
\pgfpathlineto{\pgfqpoint{1.611632in}{1.049014in}}%
\pgfpathlineto{\pgfqpoint{1.574062in}{1.088097in}}%
\pgfpathlineto{\pgfqpoint{1.536486in}{1.078657in}}%
\pgfpathclose%
\pgfusepath{fill}%
\end{pgfscope}%
\begin{pgfscope}%
\pgfpathrectangle{\pgfqpoint{0.150000in}{0.150000in}}{\pgfqpoint{2.700000in}{1.950000in}}%
\pgfusepath{clip}%
\pgfsetbuttcap%
\pgfsetroundjoin%
\definecolor{currentfill}{rgb}{0.822748,0.844577,0.875138}%
\pgfsetfillcolor{currentfill}%
\pgfsetlinewidth{0.000000pt}%
\definecolor{currentstroke}{rgb}{0.000000,0.000000,0.000000}%
\pgfsetstrokecolor{currentstroke}%
\pgfsetdash{}{0pt}%
\pgfpathmoveto{\pgfqpoint{1.536486in}{1.078657in}}%
\pgfpathlineto{\pgfqpoint{1.574062in}{1.088097in}}%
\pgfpathlineto{\pgfqpoint{1.536486in}{1.127185in}}%
\pgfpathlineto{\pgfqpoint{1.498805in}{1.117853in}}%
\pgfpathclose%
\pgfusepath{fill}%
\end{pgfscope}%
\begin{pgfscope}%
\pgfpathrectangle{\pgfqpoint{0.150000in}{0.150000in}}{\pgfqpoint{2.700000in}{1.950000in}}%
\pgfusepath{clip}%
\pgfsetbuttcap%
\pgfsetroundjoin%
\definecolor{currentfill}{rgb}{0.797871,0.822763,0.857613}%
\pgfsetfillcolor{currentfill}%
\pgfsetlinewidth{0.000000pt}%
\definecolor{currentstroke}{rgb}{0.000000,0.000000,0.000000}%
\pgfsetstrokecolor{currentstroke}%
\pgfsetdash{}{0pt}%
\pgfpathmoveto{\pgfqpoint{1.498805in}{1.117853in}}%
\pgfpathlineto{\pgfqpoint{1.536486in}{1.127185in}}%
\pgfpathlineto{\pgfqpoint{1.498906in}{1.166279in}}%
\pgfpathlineto{\pgfqpoint{1.461118in}{1.157055in}}%
\pgfpathclose%
\pgfusepath{fill}%
\end{pgfscope}%
\begin{pgfscope}%
\pgfpathrectangle{\pgfqpoint{0.150000in}{0.150000in}}{\pgfqpoint{2.700000in}{1.950000in}}%
\pgfusepath{clip}%
\pgfsetbuttcap%
\pgfsetroundjoin%
\definecolor{currentfill}{rgb}{0.766774,0.795496,0.835708}%
\pgfsetfillcolor{currentfill}%
\pgfsetlinewidth{0.000000pt}%
\definecolor{currentstroke}{rgb}{0.000000,0.000000,0.000000}%
\pgfsetstrokecolor{currentstroke}%
\pgfsetdash{}{0pt}%
\pgfpathmoveto{\pgfqpoint{1.461118in}{1.157055in}}%
\pgfpathlineto{\pgfqpoint{1.498906in}{1.166279in}}%
\pgfpathlineto{\pgfqpoint{1.461321in}{1.205378in}}%
\pgfpathlineto{\pgfqpoint{1.423426in}{1.196262in}}%
\pgfpathclose%
\pgfusepath{fill}%
\end{pgfscope}%
\begin{pgfscope}%
\pgfpathrectangle{\pgfqpoint{0.150000in}{0.150000in}}{\pgfqpoint{2.700000in}{1.950000in}}%
\pgfusepath{clip}%
\pgfsetbuttcap%
\pgfsetroundjoin%
\definecolor{currentfill}{rgb}{0.735677,0.768229,0.813802}%
\pgfsetfillcolor{currentfill}%
\pgfsetlinewidth{0.000000pt}%
\definecolor{currentstroke}{rgb}{0.000000,0.000000,0.000000}%
\pgfsetstrokecolor{currentstroke}%
\pgfsetdash{}{0pt}%
\pgfpathmoveto{\pgfqpoint{1.423426in}{1.196262in}}%
\pgfpathlineto{\pgfqpoint{1.461321in}{1.205378in}}%
\pgfpathlineto{\pgfqpoint{1.423731in}{1.244482in}}%
\pgfpathlineto{\pgfqpoint{1.385729in}{1.235474in}}%
\pgfpathclose%
\pgfusepath{fill}%
\end{pgfscope}%
\begin{pgfscope}%
\pgfpathrectangle{\pgfqpoint{0.150000in}{0.150000in}}{\pgfqpoint{2.700000in}{1.950000in}}%
\pgfusepath{clip}%
\pgfsetbuttcap%
\pgfsetroundjoin%
\definecolor{currentfill}{rgb}{0.704580,0.740962,0.791896}%
\pgfsetfillcolor{currentfill}%
\pgfsetlinewidth{0.000000pt}%
\definecolor{currentstroke}{rgb}{0.000000,0.000000,0.000000}%
\pgfsetstrokecolor{currentstroke}%
\pgfsetdash{}{0pt}%
\pgfpathmoveto{\pgfqpoint{1.385729in}{1.235474in}}%
\pgfpathlineto{\pgfqpoint{1.423731in}{1.244482in}}%
\pgfpathlineto{\pgfqpoint{1.386135in}{1.283592in}}%
\pgfpathlineto{\pgfqpoint{1.348026in}{1.274691in}}%
\pgfpathclose%
\pgfusepath{fill}%
\end{pgfscope}%
\begin{pgfscope}%
\pgfpathrectangle{\pgfqpoint{0.150000in}{0.150000in}}{\pgfqpoint{2.700000in}{1.950000in}}%
\pgfusepath{clip}%
\pgfsetbuttcap%
\pgfsetroundjoin%
\definecolor{currentfill}{rgb}{0.673483,0.713695,0.769991}%
\pgfsetfillcolor{currentfill}%
\pgfsetlinewidth{0.000000pt}%
\definecolor{currentstroke}{rgb}{0.000000,0.000000,0.000000}%
\pgfsetstrokecolor{currentstroke}%
\pgfsetdash{}{0pt}%
\pgfpathmoveto{\pgfqpoint{1.348026in}{1.274691in}}%
\pgfpathlineto{\pgfqpoint{1.386135in}{1.283592in}}%
\pgfpathlineto{\pgfqpoint{1.348535in}{1.322707in}}%
\pgfpathlineto{\pgfqpoint{1.310319in}{1.313914in}}%
\pgfpathclose%
\pgfusepath{fill}%
\end{pgfscope}%
\begin{pgfscope}%
\pgfpathrectangle{\pgfqpoint{0.150000in}{0.150000in}}{\pgfqpoint{2.700000in}{1.950000in}}%
\pgfusepath{clip}%
\pgfsetbuttcap%
\pgfsetroundjoin%
\definecolor{currentfill}{rgb}{0.642387,0.686428,0.748085}%
\pgfsetfillcolor{currentfill}%
\pgfsetlinewidth{0.000000pt}%
\definecolor{currentstroke}{rgb}{0.000000,0.000000,0.000000}%
\pgfsetstrokecolor{currentstroke}%
\pgfsetdash{}{0pt}%
\pgfpathmoveto{\pgfqpoint{1.310319in}{1.313914in}}%
\pgfpathlineto{\pgfqpoint{1.348535in}{1.322707in}}%
\pgfpathlineto{\pgfqpoint{1.310929in}{1.361827in}}%
\pgfpathlineto{\pgfqpoint{1.272606in}{1.353142in}}%
\pgfpathclose%
\pgfusepath{fill}%
\end{pgfscope}%
\begin{pgfscope}%
\pgfpathrectangle{\pgfqpoint{0.150000in}{0.150000in}}{\pgfqpoint{2.700000in}{1.950000in}}%
\pgfusepath{clip}%
\pgfsetbuttcap%
\pgfsetroundjoin%
\definecolor{currentfill}{rgb}{0.611290,0.659161,0.726180}%
\pgfsetfillcolor{currentfill}%
\pgfsetlinewidth{0.000000pt}%
\definecolor{currentstroke}{rgb}{0.000000,0.000000,0.000000}%
\pgfsetstrokecolor{currentstroke}%
\pgfsetdash{}{0pt}%
\pgfpathmoveto{\pgfqpoint{1.272606in}{1.353142in}}%
\pgfpathlineto{\pgfqpoint{1.310929in}{1.361827in}}%
\pgfpathlineto{\pgfqpoint{1.273319in}{1.400952in}}%
\pgfpathlineto{\pgfqpoint{1.234889in}{1.392375in}}%
\pgfpathclose%
\pgfusepath{fill}%
\end{pgfscope}%
\begin{pgfscope}%
\pgfpathrectangle{\pgfqpoint{0.150000in}{0.150000in}}{\pgfqpoint{2.700000in}{1.950000in}}%
\pgfusepath{clip}%
\pgfsetbuttcap%
\pgfsetroundjoin%
\definecolor{currentfill}{rgb}{0.580193,0.631893,0.704274}%
\pgfsetfillcolor{currentfill}%
\pgfsetlinewidth{0.000000pt}%
\definecolor{currentstroke}{rgb}{0.000000,0.000000,0.000000}%
\pgfsetstrokecolor{currentstroke}%
\pgfsetdash{}{0pt}%
\pgfpathmoveto{\pgfqpoint{1.234889in}{1.392375in}}%
\pgfpathlineto{\pgfqpoint{1.273319in}{1.400952in}}%
\pgfpathlineto{\pgfqpoint{1.235703in}{1.440083in}}%
\pgfpathlineto{\pgfqpoint{1.197166in}{1.431614in}}%
\pgfpathclose%
\pgfusepath{fill}%
\end{pgfscope}%
\begin{pgfscope}%
\pgfpathrectangle{\pgfqpoint{0.150000in}{0.150000in}}{\pgfqpoint{2.700000in}{1.950000in}}%
\pgfusepath{clip}%
\pgfsetbuttcap%
\pgfsetroundjoin%
\definecolor{currentfill}{rgb}{0.555316,0.610080,0.686749}%
\pgfsetfillcolor{currentfill}%
\pgfsetlinewidth{0.000000pt}%
\definecolor{currentstroke}{rgb}{0.000000,0.000000,0.000000}%
\pgfsetstrokecolor{currentstroke}%
\pgfsetdash{}{0pt}%
\pgfpathmoveto{\pgfqpoint{1.197166in}{1.431614in}}%
\pgfpathlineto{\pgfqpoint{1.235703in}{1.440083in}}%
\pgfpathlineto{\pgfqpoint{1.198082in}{1.479219in}}%
\pgfpathlineto{\pgfqpoint{1.159438in}{1.470858in}}%
\pgfpathclose%
\pgfusepath{fill}%
\end{pgfscope}%
\begin{pgfscope}%
\pgfpathrectangle{\pgfqpoint{0.150000in}{0.150000in}}{\pgfqpoint{2.700000in}{1.950000in}}%
\pgfusepath{clip}%
\pgfsetbuttcap%
\pgfsetroundjoin%
\definecolor{currentfill}{rgb}{0.524219,0.582812,0.664844}%
\pgfsetfillcolor{currentfill}%
\pgfsetlinewidth{0.000000pt}%
\definecolor{currentstroke}{rgb}{0.000000,0.000000,0.000000}%
\pgfsetstrokecolor{currentstroke}%
\pgfsetdash{}{0pt}%
\pgfpathmoveto{\pgfqpoint{1.159438in}{1.470858in}}%
\pgfpathlineto{\pgfqpoint{1.198082in}{1.479219in}}%
\pgfpathlineto{\pgfqpoint{1.160456in}{1.518360in}}%
\pgfpathlineto{\pgfqpoint{1.121705in}{1.510107in}}%
\pgfpathclose%
\pgfusepath{fill}%
\end{pgfscope}%
\begin{pgfscope}%
\pgfpathrectangle{\pgfqpoint{0.150000in}{0.150000in}}{\pgfqpoint{2.700000in}{1.950000in}}%
\pgfusepath{clip}%
\pgfsetbuttcap%
\pgfsetroundjoin%
\definecolor{currentfill}{rgb}{0.493122,0.555545,0.642938}%
\pgfsetfillcolor{currentfill}%
\pgfsetlinewidth{0.000000pt}%
\definecolor{currentstroke}{rgb}{0.000000,0.000000,0.000000}%
\pgfsetstrokecolor{currentstroke}%
\pgfsetdash{}{0pt}%
\pgfpathmoveto{\pgfqpoint{1.121705in}{1.510107in}}%
\pgfpathlineto{\pgfqpoint{1.160456in}{1.518360in}}%
\pgfpathlineto{\pgfqpoint{1.122825in}{1.557507in}}%
\pgfpathlineto{\pgfqpoint{1.083967in}{1.549362in}}%
\pgfpathclose%
\pgfusepath{fill}%
\end{pgfscope}%
\begin{pgfscope}%
\pgfpathrectangle{\pgfqpoint{0.150000in}{0.150000in}}{\pgfqpoint{2.700000in}{1.950000in}}%
\pgfusepath{clip}%
\pgfsetbuttcap%
\pgfsetroundjoin%
\definecolor{currentfill}{rgb}{0.462025,0.528278,0.621032}%
\pgfsetfillcolor{currentfill}%
\pgfsetlinewidth{0.000000pt}%
\definecolor{currentstroke}{rgb}{0.000000,0.000000,0.000000}%
\pgfsetstrokecolor{currentstroke}%
\pgfsetdash{}{0pt}%
\pgfpathmoveto{\pgfqpoint{1.083967in}{1.549362in}}%
\pgfpathlineto{\pgfqpoint{1.122825in}{1.557507in}}%
\pgfpathlineto{\pgfqpoint{1.085189in}{1.596659in}}%
\pgfpathlineto{\pgfqpoint{1.046224in}{1.588622in}}%
\pgfpathclose%
\pgfusepath{fill}%
\end{pgfscope}%
\begin{pgfscope}%
\pgfpathrectangle{\pgfqpoint{0.150000in}{0.150000in}}{\pgfqpoint{2.700000in}{1.950000in}}%
\pgfusepath{clip}%
\pgfsetbuttcap%
\pgfsetroundjoin%
\definecolor{currentfill}{rgb}{0.430928,0.501011,0.599127}%
\pgfsetfillcolor{currentfill}%
\pgfsetlinewidth{0.000000pt}%
\definecolor{currentstroke}{rgb}{0.000000,0.000000,0.000000}%
\pgfsetstrokecolor{currentstroke}%
\pgfsetdash{}{0pt}%
\pgfpathmoveto{\pgfqpoint{1.046224in}{1.588622in}}%
\pgfpathlineto{\pgfqpoint{1.085189in}{1.596659in}}%
\pgfpathlineto{\pgfqpoint{1.047548in}{1.635816in}}%
\pgfpathlineto{\pgfqpoint{1.008476in}{1.627887in}}%
\pgfpathclose%
\pgfusepath{fill}%
\end{pgfscope}%
\begin{pgfscope}%
\pgfpathrectangle{\pgfqpoint{0.150000in}{0.150000in}}{\pgfqpoint{2.700000in}{1.950000in}}%
\pgfusepath{clip}%
\pgfsetbuttcap%
\pgfsetroundjoin%
\definecolor{currentfill}{rgb}{0.910723,0.838006,0.843765}%
\pgfsetfillcolor{currentfill}%
\pgfsetlinewidth{0.000000pt}%
\definecolor{currentstroke}{rgb}{0.000000,0.000000,0.000000}%
\pgfsetstrokecolor{currentstroke}%
\pgfsetdash{}{0pt}%
\pgfpathmoveto{\pgfqpoint{1.914045in}{0.637168in}}%
\pgfpathlineto{\pgfqpoint{1.950650in}{0.647850in}}%
\pgfpathlineto{\pgfqpoint{1.913024in}{0.686987in}}%
\pgfpathlineto{\pgfqpoint{1.876312in}{0.676414in}}%
\pgfpathclose%
\pgfusepath{fill}%
\end{pgfscope}%
\begin{pgfscope}%
\pgfpathrectangle{\pgfqpoint{0.150000in}{0.150000in}}{\pgfqpoint{2.700000in}{1.950000in}}%
\pgfusepath{clip}%
\pgfsetbuttcap%
\pgfsetroundjoin%
\definecolor{currentfill}{rgb}{0.929718,0.872472,0.877007}%
\pgfsetfillcolor{currentfill}%
\pgfsetlinewidth{0.000000pt}%
\definecolor{currentstroke}{rgb}{0.000000,0.000000,0.000000}%
\pgfsetstrokecolor{currentstroke}%
\pgfsetdash{}{0pt}%
\pgfpathmoveto{\pgfqpoint{1.876312in}{0.676414in}}%
\pgfpathlineto{\pgfqpoint{1.913024in}{0.686987in}}%
\pgfpathlineto{\pgfqpoint{1.875393in}{0.726130in}}%
\pgfpathlineto{\pgfqpoint{1.838574in}{0.715665in}}%
\pgfpathclose%
\pgfusepath{fill}%
\end{pgfscope}%
\begin{pgfscope}%
\pgfpathrectangle{\pgfqpoint{0.150000in}{0.150000in}}{\pgfqpoint{2.700000in}{1.950000in}}%
\pgfusepath{clip}%
\pgfsetbuttcap%
\pgfsetroundjoin%
\definecolor{currentfill}{rgb}{0.948713,0.906939,0.910248}%
\pgfsetfillcolor{currentfill}%
\pgfsetlinewidth{0.000000pt}%
\definecolor{currentstroke}{rgb}{0.000000,0.000000,0.000000}%
\pgfsetstrokecolor{currentstroke}%
\pgfsetdash{}{0pt}%
\pgfpathmoveto{\pgfqpoint{1.838574in}{0.715665in}}%
\pgfpathlineto{\pgfqpoint{1.875393in}{0.726130in}}%
\pgfpathlineto{\pgfqpoint{1.837758in}{0.765279in}}%
\pgfpathlineto{\pgfqpoint{1.800831in}{0.754921in}}%
\pgfpathclose%
\pgfusepath{fill}%
\end{pgfscope}%
\begin{pgfscope}%
\pgfpathrectangle{\pgfqpoint{0.150000in}{0.150000in}}{\pgfqpoint{2.700000in}{1.950000in}}%
\pgfusepath{clip}%
\pgfsetbuttcap%
\pgfsetroundjoin%
\definecolor{currentfill}{rgb}{0.967708,0.941406,0.943490}%
\pgfsetfillcolor{currentfill}%
\pgfsetlinewidth{0.000000pt}%
\definecolor{currentstroke}{rgb}{0.000000,0.000000,0.000000}%
\pgfsetstrokecolor{currentstroke}%
\pgfsetdash{}{0pt}%
\pgfpathmoveto{\pgfqpoint{1.800831in}{0.754921in}}%
\pgfpathlineto{\pgfqpoint{1.837758in}{0.765279in}}%
\pgfpathlineto{\pgfqpoint{1.800116in}{0.804433in}}%
\pgfpathlineto{\pgfqpoint{1.763083in}{0.794183in}}%
\pgfpathclose%
\pgfusepath{fill}%
\end{pgfscope}%
\begin{pgfscope}%
\pgfpathrectangle{\pgfqpoint{0.150000in}{0.150000in}}{\pgfqpoint{2.700000in}{1.950000in}}%
\pgfusepath{clip}%
\pgfsetbuttcap%
\pgfsetroundjoin%
\definecolor{currentfill}{rgb}{0.986703,0.975873,0.976731}%
\pgfsetfillcolor{currentfill}%
\pgfsetlinewidth{0.000000pt}%
\definecolor{currentstroke}{rgb}{0.000000,0.000000,0.000000}%
\pgfsetstrokecolor{currentstroke}%
\pgfsetdash{}{0pt}%
\pgfpathmoveto{\pgfqpoint{1.763083in}{0.794183in}}%
\pgfpathlineto{\pgfqpoint{1.800116in}{0.804433in}}%
\pgfpathlineto{\pgfqpoint{1.762470in}{0.843592in}}%
\pgfpathlineto{\pgfqpoint{1.725330in}{0.833450in}}%
\pgfpathclose%
\pgfusepath{fill}%
\end{pgfscope}%
\begin{pgfscope}%
\pgfpathrectangle{\pgfqpoint{0.150000in}{0.150000in}}{\pgfqpoint{2.700000in}{1.950000in}}%
\pgfusepath{clip}%
\pgfsetbuttcap%
\pgfsetroundjoin%
\definecolor{currentfill}{rgb}{0.990671,0.991820,0.993428}%
\pgfsetfillcolor{currentfill}%
\pgfsetlinewidth{0.000000pt}%
\definecolor{currentstroke}{rgb}{0.000000,0.000000,0.000000}%
\pgfsetstrokecolor{currentstroke}%
\pgfsetdash{}{0pt}%
\pgfpathmoveto{\pgfqpoint{1.725330in}{0.833450in}}%
\pgfpathlineto{\pgfqpoint{1.762470in}{0.843592in}}%
\pgfpathlineto{\pgfqpoint{1.724819in}{0.882756in}}%
\pgfpathlineto{\pgfqpoint{1.687571in}{0.872723in}}%
\pgfpathclose%
\pgfusepath{fill}%
\end{pgfscope}%
\begin{pgfscope}%
\pgfpathrectangle{\pgfqpoint{0.150000in}{0.150000in}}{\pgfqpoint{2.700000in}{1.950000in}}%
\pgfusepath{clip}%
\pgfsetbuttcap%
\pgfsetroundjoin%
\definecolor{currentfill}{rgb}{0.959574,0.964553,0.971523}%
\pgfsetfillcolor{currentfill}%
\pgfsetlinewidth{0.000000pt}%
\definecolor{currentstroke}{rgb}{0.000000,0.000000,0.000000}%
\pgfsetstrokecolor{currentstroke}%
\pgfsetdash{}{0pt}%
\pgfpathmoveto{\pgfqpoint{1.687571in}{0.872723in}}%
\pgfpathlineto{\pgfqpoint{1.724819in}{0.882756in}}%
\pgfpathlineto{\pgfqpoint{1.687163in}{0.921926in}}%
\pgfpathlineto{\pgfqpoint{1.649808in}{0.912000in}}%
\pgfpathclose%
\pgfusepath{fill}%
\end{pgfscope}%
\begin{pgfscope}%
\pgfpathrectangle{\pgfqpoint{0.150000in}{0.150000in}}{\pgfqpoint{2.700000in}{1.950000in}}%
\pgfusepath{clip}%
\pgfsetbuttcap%
\pgfsetroundjoin%
\definecolor{currentfill}{rgb}{0.934697,0.942739,0.953998}%
\pgfsetfillcolor{currentfill}%
\pgfsetlinewidth{0.000000pt}%
\definecolor{currentstroke}{rgb}{0.000000,0.000000,0.000000}%
\pgfsetstrokecolor{currentstroke}%
\pgfsetdash{}{0pt}%
\pgfpathmoveto{\pgfqpoint{1.649808in}{0.912000in}}%
\pgfpathlineto{\pgfqpoint{1.687163in}{0.921926in}}%
\pgfpathlineto{\pgfqpoint{1.649501in}{0.961101in}}%
\pgfpathlineto{\pgfqpoint{1.612039in}{0.951284in}}%
\pgfpathclose%
\pgfusepath{fill}%
\end{pgfscope}%
\begin{pgfscope}%
\pgfpathrectangle{\pgfqpoint{0.150000in}{0.150000in}}{\pgfqpoint{2.700000in}{1.950000in}}%
\pgfusepath{clip}%
\pgfsetbuttcap%
\pgfsetroundjoin%
\definecolor{currentfill}{rgb}{0.903600,0.915472,0.932093}%
\pgfsetfillcolor{currentfill}%
\pgfsetlinewidth{0.000000pt}%
\definecolor{currentstroke}{rgb}{0.000000,0.000000,0.000000}%
\pgfsetstrokecolor{currentstroke}%
\pgfsetdash{}{0pt}%
\pgfpathmoveto{\pgfqpoint{1.612039in}{0.951284in}}%
\pgfpathlineto{\pgfqpoint{1.649501in}{0.961101in}}%
\pgfpathlineto{\pgfqpoint{1.611835in}{1.000281in}}%
\pgfpathlineto{\pgfqpoint{1.574265in}{0.990572in}}%
\pgfpathclose%
\pgfusepath{fill}%
\end{pgfscope}%
\begin{pgfscope}%
\pgfpathrectangle{\pgfqpoint{0.150000in}{0.150000in}}{\pgfqpoint{2.700000in}{1.950000in}}%
\pgfusepath{clip}%
\pgfsetbuttcap%
\pgfsetroundjoin%
\definecolor{currentfill}{rgb}{0.872503,0.888205,0.910187}%
\pgfsetfillcolor{currentfill}%
\pgfsetlinewidth{0.000000pt}%
\definecolor{currentstroke}{rgb}{0.000000,0.000000,0.000000}%
\pgfsetstrokecolor{currentstroke}%
\pgfsetdash{}{0pt}%
\pgfpathmoveto{\pgfqpoint{1.574265in}{0.990572in}}%
\pgfpathlineto{\pgfqpoint{1.611835in}{1.000281in}}%
\pgfpathlineto{\pgfqpoint{1.574163in}{1.039466in}}%
\pgfpathlineto{\pgfqpoint{1.536486in}{1.029866in}}%
\pgfpathclose%
\pgfusepath{fill}%
\end{pgfscope}%
\begin{pgfscope}%
\pgfpathrectangle{\pgfqpoint{0.150000in}{0.150000in}}{\pgfqpoint{2.700000in}{1.950000in}}%
\pgfusepath{clip}%
\pgfsetbuttcap%
\pgfsetroundjoin%
\definecolor{currentfill}{rgb}{0.841406,0.860938,0.888281}%
\pgfsetfillcolor{currentfill}%
\pgfsetlinewidth{0.000000pt}%
\definecolor{currentstroke}{rgb}{0.000000,0.000000,0.000000}%
\pgfsetstrokecolor{currentstroke}%
\pgfsetdash{}{0pt}%
\pgfpathmoveto{\pgfqpoint{1.536486in}{1.029866in}}%
\pgfpathlineto{\pgfqpoint{1.574163in}{1.039466in}}%
\pgfpathlineto{\pgfqpoint{1.536486in}{1.078657in}}%
\pgfpathlineto{\pgfqpoint{1.498702in}{1.069165in}}%
\pgfpathclose%
\pgfusepath{fill}%
\end{pgfscope}%
\begin{pgfscope}%
\pgfpathrectangle{\pgfqpoint{0.150000in}{0.150000in}}{\pgfqpoint{2.700000in}{1.950000in}}%
\pgfusepath{clip}%
\pgfsetbuttcap%
\pgfsetroundjoin%
\definecolor{currentfill}{rgb}{0.810309,0.833670,0.866376}%
\pgfsetfillcolor{currentfill}%
\pgfsetlinewidth{0.000000pt}%
\definecolor{currentstroke}{rgb}{0.000000,0.000000,0.000000}%
\pgfsetstrokecolor{currentstroke}%
\pgfsetdash{}{0pt}%
\pgfpathmoveto{\pgfqpoint{1.498702in}{1.069165in}}%
\pgfpathlineto{\pgfqpoint{1.536486in}{1.078657in}}%
\pgfpathlineto{\pgfqpoint{1.498805in}{1.117853in}}%
\pgfpathlineto{\pgfqpoint{1.460913in}{1.108470in}}%
\pgfpathclose%
\pgfusepath{fill}%
\end{pgfscope}%
\begin{pgfscope}%
\pgfpathrectangle{\pgfqpoint{0.150000in}{0.150000in}}{\pgfqpoint{2.700000in}{1.950000in}}%
\pgfusepath{clip}%
\pgfsetbuttcap%
\pgfsetroundjoin%
\definecolor{currentfill}{rgb}{0.779213,0.806403,0.844470}%
\pgfsetfillcolor{currentfill}%
\pgfsetlinewidth{0.000000pt}%
\definecolor{currentstroke}{rgb}{0.000000,0.000000,0.000000}%
\pgfsetstrokecolor{currentstroke}%
\pgfsetdash{}{0pt}%
\pgfpathmoveto{\pgfqpoint{1.460913in}{1.108470in}}%
\pgfpathlineto{\pgfqpoint{1.498805in}{1.117853in}}%
\pgfpathlineto{\pgfqpoint{1.461118in}{1.157055in}}%
\pgfpathlineto{\pgfqpoint{1.423119in}{1.147779in}}%
\pgfpathclose%
\pgfusepath{fill}%
\end{pgfscope}%
\begin{pgfscope}%
\pgfpathrectangle{\pgfqpoint{0.150000in}{0.150000in}}{\pgfqpoint{2.700000in}{1.950000in}}%
\pgfusepath{clip}%
\pgfsetbuttcap%
\pgfsetroundjoin%
\definecolor{currentfill}{rgb}{0.748116,0.779136,0.822564}%
\pgfsetfillcolor{currentfill}%
\pgfsetlinewidth{0.000000pt}%
\definecolor{currentstroke}{rgb}{0.000000,0.000000,0.000000}%
\pgfsetstrokecolor{currentstroke}%
\pgfsetdash{}{0pt}%
\pgfpathmoveto{\pgfqpoint{1.423119in}{1.147779in}}%
\pgfpathlineto{\pgfqpoint{1.461118in}{1.157055in}}%
\pgfpathlineto{\pgfqpoint{1.423426in}{1.196262in}}%
\pgfpathlineto{\pgfqpoint{1.385320in}{1.187095in}}%
\pgfpathclose%
\pgfusepath{fill}%
\end{pgfscope}%
\begin{pgfscope}%
\pgfpathrectangle{\pgfqpoint{0.150000in}{0.150000in}}{\pgfqpoint{2.700000in}{1.950000in}}%
\pgfusepath{clip}%
\pgfsetbuttcap%
\pgfsetroundjoin%
\definecolor{currentfill}{rgb}{0.717019,0.751869,0.800659}%
\pgfsetfillcolor{currentfill}%
\pgfsetlinewidth{0.000000pt}%
\definecolor{currentstroke}{rgb}{0.000000,0.000000,0.000000}%
\pgfsetstrokecolor{currentstroke}%
\pgfsetdash{}{0pt}%
\pgfpathmoveto{\pgfqpoint{1.385320in}{1.187095in}}%
\pgfpathlineto{\pgfqpoint{1.423426in}{1.196262in}}%
\pgfpathlineto{\pgfqpoint{1.385729in}{1.235474in}}%
\pgfpathlineto{\pgfqpoint{1.347515in}{1.226415in}}%
\pgfpathclose%
\pgfusepath{fill}%
\end{pgfscope}%
\begin{pgfscope}%
\pgfpathrectangle{\pgfqpoint{0.150000in}{0.150000in}}{\pgfqpoint{2.700000in}{1.950000in}}%
\pgfusepath{clip}%
\pgfsetbuttcap%
\pgfsetroundjoin%
\definecolor{currentfill}{rgb}{0.692142,0.730055,0.783134}%
\pgfsetfillcolor{currentfill}%
\pgfsetlinewidth{0.000000pt}%
\definecolor{currentstroke}{rgb}{0.000000,0.000000,0.000000}%
\pgfsetstrokecolor{currentstroke}%
\pgfsetdash{}{0pt}%
\pgfpathmoveto{\pgfqpoint{1.347515in}{1.226415in}}%
\pgfpathlineto{\pgfqpoint{1.385729in}{1.235474in}}%
\pgfpathlineto{\pgfqpoint{1.348026in}{1.274691in}}%
\pgfpathlineto{\pgfqpoint{1.309705in}{1.265741in}}%
\pgfpathclose%
\pgfusepath{fill}%
\end{pgfscope}%
\begin{pgfscope}%
\pgfpathrectangle{\pgfqpoint{0.150000in}{0.150000in}}{\pgfqpoint{2.700000in}{1.950000in}}%
\pgfusepath{clip}%
\pgfsetbuttcap%
\pgfsetroundjoin%
\definecolor{currentfill}{rgb}{0.661045,0.702788,0.761229}%
\pgfsetfillcolor{currentfill}%
\pgfsetlinewidth{0.000000pt}%
\definecolor{currentstroke}{rgb}{0.000000,0.000000,0.000000}%
\pgfsetstrokecolor{currentstroke}%
\pgfsetdash{}{0pt}%
\pgfpathmoveto{\pgfqpoint{1.309705in}{1.265741in}}%
\pgfpathlineto{\pgfqpoint{1.348026in}{1.274691in}}%
\pgfpathlineto{\pgfqpoint{1.310319in}{1.313914in}}%
\pgfpathlineto{\pgfqpoint{1.271891in}{1.305072in}}%
\pgfpathclose%
\pgfusepath{fill}%
\end{pgfscope}%
\begin{pgfscope}%
\pgfpathrectangle{\pgfqpoint{0.150000in}{0.150000in}}{\pgfqpoint{2.700000in}{1.950000in}}%
\pgfusepath{clip}%
\pgfsetbuttcap%
\pgfsetroundjoin%
\definecolor{currentfill}{rgb}{0.629948,0.675521,0.739323}%
\pgfsetfillcolor{currentfill}%
\pgfsetlinewidth{0.000000pt}%
\definecolor{currentstroke}{rgb}{0.000000,0.000000,0.000000}%
\pgfsetstrokecolor{currentstroke}%
\pgfsetdash{}{0pt}%
\pgfpathmoveto{\pgfqpoint{1.271891in}{1.305072in}}%
\pgfpathlineto{\pgfqpoint{1.310319in}{1.313914in}}%
\pgfpathlineto{\pgfqpoint{1.272606in}{1.353142in}}%
\pgfpathlineto{\pgfqpoint{1.234071in}{1.344409in}}%
\pgfpathclose%
\pgfusepath{fill}%
\end{pgfscope}%
\begin{pgfscope}%
\pgfpathrectangle{\pgfqpoint{0.150000in}{0.150000in}}{\pgfqpoint{2.700000in}{1.950000in}}%
\pgfusepath{clip}%
\pgfsetbuttcap%
\pgfsetroundjoin%
\definecolor{currentfill}{rgb}{0.598851,0.648254,0.717417}%
\pgfsetfillcolor{currentfill}%
\pgfsetlinewidth{0.000000pt}%
\definecolor{currentstroke}{rgb}{0.000000,0.000000,0.000000}%
\pgfsetstrokecolor{currentstroke}%
\pgfsetdash{}{0pt}%
\pgfpathmoveto{\pgfqpoint{1.234071in}{1.344409in}}%
\pgfpathlineto{\pgfqpoint{1.272606in}{1.353142in}}%
\pgfpathlineto{\pgfqpoint{1.234889in}{1.392375in}}%
\pgfpathlineto{\pgfqpoint{1.196245in}{1.383750in}}%
\pgfpathclose%
\pgfusepath{fill}%
\end{pgfscope}%
\begin{pgfscope}%
\pgfpathrectangle{\pgfqpoint{0.150000in}{0.150000in}}{\pgfqpoint{2.700000in}{1.950000in}}%
\pgfusepath{clip}%
\pgfsetbuttcap%
\pgfsetroundjoin%
\definecolor{currentfill}{rgb}{0.567754,0.620987,0.695512}%
\pgfsetfillcolor{currentfill}%
\pgfsetlinewidth{0.000000pt}%
\definecolor{currentstroke}{rgb}{0.000000,0.000000,0.000000}%
\pgfsetstrokecolor{currentstroke}%
\pgfsetdash{}{0pt}%
\pgfpathmoveto{\pgfqpoint{1.196245in}{1.383750in}}%
\pgfpathlineto{\pgfqpoint{1.234889in}{1.392375in}}%
\pgfpathlineto{\pgfqpoint{1.197166in}{1.431614in}}%
\pgfpathlineto{\pgfqpoint{1.158415in}{1.423098in}}%
\pgfpathclose%
\pgfusepath{fill}%
\end{pgfscope}%
\begin{pgfscope}%
\pgfpathrectangle{\pgfqpoint{0.150000in}{0.150000in}}{\pgfqpoint{2.700000in}{1.950000in}}%
\pgfusepath{clip}%
\pgfsetbuttcap%
\pgfsetroundjoin%
\definecolor{currentfill}{rgb}{0.536657,0.593719,0.673606}%
\pgfsetfillcolor{currentfill}%
\pgfsetlinewidth{0.000000pt}%
\definecolor{currentstroke}{rgb}{0.000000,0.000000,0.000000}%
\pgfsetstrokecolor{currentstroke}%
\pgfsetdash{}{0pt}%
\pgfpathmoveto{\pgfqpoint{1.158415in}{1.423098in}}%
\pgfpathlineto{\pgfqpoint{1.197166in}{1.431614in}}%
\pgfpathlineto{\pgfqpoint{1.159438in}{1.470858in}}%
\pgfpathlineto{\pgfqpoint{1.120580in}{1.462450in}}%
\pgfpathclose%
\pgfusepath{fill}%
\end{pgfscope}%
\begin{pgfscope}%
\pgfpathrectangle{\pgfqpoint{0.150000in}{0.150000in}}{\pgfqpoint{2.700000in}{1.950000in}}%
\pgfusepath{clip}%
\pgfsetbuttcap%
\pgfsetroundjoin%
\definecolor{currentfill}{rgb}{0.505561,0.566452,0.651700}%
\pgfsetfillcolor{currentfill}%
\pgfsetlinewidth{0.000000pt}%
\definecolor{currentstroke}{rgb}{0.000000,0.000000,0.000000}%
\pgfsetstrokecolor{currentstroke}%
\pgfsetdash{}{0pt}%
\pgfpathmoveto{\pgfqpoint{1.120580in}{1.462450in}}%
\pgfpathlineto{\pgfqpoint{1.159438in}{1.470858in}}%
\pgfpathlineto{\pgfqpoint{1.121705in}{1.510107in}}%
\pgfpathlineto{\pgfqpoint{1.082739in}{1.501808in}}%
\pgfpathclose%
\pgfusepath{fill}%
\end{pgfscope}%
\begin{pgfscope}%
\pgfpathrectangle{\pgfqpoint{0.150000in}{0.150000in}}{\pgfqpoint{2.700000in}{1.950000in}}%
\pgfusepath{clip}%
\pgfsetbuttcap%
\pgfsetroundjoin%
\definecolor{currentfill}{rgb}{0.474464,0.539185,0.629795}%
\pgfsetfillcolor{currentfill}%
\pgfsetlinewidth{0.000000pt}%
\definecolor{currentstroke}{rgb}{0.000000,0.000000,0.000000}%
\pgfsetstrokecolor{currentstroke}%
\pgfsetdash{}{0pt}%
\pgfpathmoveto{\pgfqpoint{1.082739in}{1.501808in}}%
\pgfpathlineto{\pgfqpoint{1.121705in}{1.510107in}}%
\pgfpathlineto{\pgfqpoint{1.083967in}{1.549362in}}%
\pgfpathlineto{\pgfqpoint{1.044894in}{1.541171in}}%
\pgfpathclose%
\pgfusepath{fill}%
\end{pgfscope}%
\begin{pgfscope}%
\pgfpathrectangle{\pgfqpoint{0.150000in}{0.150000in}}{\pgfqpoint{2.700000in}{1.950000in}}%
\pgfusepath{clip}%
\pgfsetbuttcap%
\pgfsetroundjoin%
\definecolor{currentfill}{rgb}{0.449586,0.517371,0.612270}%
\pgfsetfillcolor{currentfill}%
\pgfsetlinewidth{0.000000pt}%
\definecolor{currentstroke}{rgb}{0.000000,0.000000,0.000000}%
\pgfsetstrokecolor{currentstroke}%
\pgfsetdash{}{0pt}%
\pgfpathmoveto{\pgfqpoint{1.044894in}{1.541171in}}%
\pgfpathlineto{\pgfqpoint{1.083967in}{1.549362in}}%
\pgfpathlineto{\pgfqpoint{1.046224in}{1.588622in}}%
\pgfpathlineto{\pgfqpoint{1.007043in}{1.580540in}}%
\pgfpathclose%
\pgfusepath{fill}%
\end{pgfscope}%
\begin{pgfscope}%
\pgfpathrectangle{\pgfqpoint{0.150000in}{0.150000in}}{\pgfqpoint{2.700000in}{1.950000in}}%
\pgfusepath{clip}%
\pgfsetbuttcap%
\pgfsetroundjoin%
\definecolor{currentfill}{rgb}{0.418490,0.490104,0.590365}%
\pgfsetfillcolor{currentfill}%
\pgfsetlinewidth{0.000000pt}%
\definecolor{currentstroke}{rgb}{0.000000,0.000000,0.000000}%
\pgfsetstrokecolor{currentstroke}%
\pgfsetdash{}{0pt}%
\pgfpathmoveto{\pgfqpoint{1.007043in}{1.580540in}}%
\pgfpathlineto{\pgfqpoint{1.046224in}{1.588622in}}%
\pgfpathlineto{\pgfqpoint{1.008476in}{1.627887in}}%
\pgfpathlineto{\pgfqpoint{0.969187in}{1.619914in}}%
\pgfpathclose%
\pgfusepath{fill}%
\end{pgfscope}%
\begin{pgfscope}%
\pgfpathrectangle{\pgfqpoint{0.150000in}{0.150000in}}{\pgfqpoint{2.700000in}{1.950000in}}%
\pgfusepath{clip}%
\pgfsetbuttcap%
\pgfsetroundjoin%
\definecolor{currentfill}{rgb}{0.922120,0.858686,0.863710}%
\pgfsetfillcolor{currentfill}%
\pgfsetlinewidth{0.000000pt}%
\definecolor{currentstroke}{rgb}{0.000000,0.000000,0.000000}%
\pgfsetstrokecolor{currentstroke}%
\pgfsetdash{}{0pt}%
\pgfpathmoveto{\pgfqpoint{1.877235in}{0.626426in}}%
\pgfpathlineto{\pgfqpoint{1.914045in}{0.637168in}}%
\pgfpathlineto{\pgfqpoint{1.876312in}{0.676414in}}%
\pgfpathlineto{\pgfqpoint{1.839395in}{0.665781in}}%
\pgfpathclose%
\pgfusepath{fill}%
\end{pgfscope}%
\begin{pgfscope}%
\pgfpathrectangle{\pgfqpoint{0.150000in}{0.150000in}}{\pgfqpoint{2.700000in}{1.950000in}}%
\pgfusepath{clip}%
\pgfsetbuttcap%
\pgfsetroundjoin%
\definecolor{currentfill}{rgb}{0.941115,0.893153,0.896952}%
\pgfsetfillcolor{currentfill}%
\pgfsetlinewidth{0.000000pt}%
\definecolor{currentstroke}{rgb}{0.000000,0.000000,0.000000}%
\pgfsetstrokecolor{currentstroke}%
\pgfsetdash{}{0pt}%
\pgfpathmoveto{\pgfqpoint{1.839395in}{0.665781in}}%
\pgfpathlineto{\pgfqpoint{1.876312in}{0.676414in}}%
\pgfpathlineto{\pgfqpoint{1.838574in}{0.715665in}}%
\pgfpathlineto{\pgfqpoint{1.801550in}{0.705141in}}%
\pgfpathclose%
\pgfusepath{fill}%
\end{pgfscope}%
\begin{pgfscope}%
\pgfpathrectangle{\pgfqpoint{0.150000in}{0.150000in}}{\pgfqpoint{2.700000in}{1.950000in}}%
\pgfusepath{clip}%
\pgfsetbuttcap%
\pgfsetroundjoin%
\definecolor{currentfill}{rgb}{0.960110,0.927619,0.930193}%
\pgfsetfillcolor{currentfill}%
\pgfsetlinewidth{0.000000pt}%
\definecolor{currentstroke}{rgb}{0.000000,0.000000,0.000000}%
\pgfsetstrokecolor{currentstroke}%
\pgfsetdash{}{0pt}%
\pgfpathmoveto{\pgfqpoint{1.801550in}{0.705141in}}%
\pgfpathlineto{\pgfqpoint{1.838574in}{0.715665in}}%
\pgfpathlineto{\pgfqpoint{1.800831in}{0.754921in}}%
\pgfpathlineto{\pgfqpoint{1.763699in}{0.744506in}}%
\pgfpathclose%
\pgfusepath{fill}%
\end{pgfscope}%
\begin{pgfscope}%
\pgfpathrectangle{\pgfqpoint{0.150000in}{0.150000in}}{\pgfqpoint{2.700000in}{1.950000in}}%
\pgfusepath{clip}%
\pgfsetbuttcap%
\pgfsetroundjoin%
\definecolor{currentfill}{rgb}{0.975306,0.955193,0.956786}%
\pgfsetfillcolor{currentfill}%
\pgfsetlinewidth{0.000000pt}%
\definecolor{currentstroke}{rgb}{0.000000,0.000000,0.000000}%
\pgfsetstrokecolor{currentstroke}%
\pgfsetdash{}{0pt}%
\pgfpathmoveto{\pgfqpoint{1.763699in}{0.744506in}}%
\pgfpathlineto{\pgfqpoint{1.800831in}{0.754921in}}%
\pgfpathlineto{\pgfqpoint{1.763083in}{0.794183in}}%
\pgfpathlineto{\pgfqpoint{1.725843in}{0.783876in}}%
\pgfpathclose%
\pgfusepath{fill}%
\end{pgfscope}%
\begin{pgfscope}%
\pgfpathrectangle{\pgfqpoint{0.150000in}{0.150000in}}{\pgfqpoint{2.700000in}{1.950000in}}%
\pgfusepath{clip}%
\pgfsetbuttcap%
\pgfsetroundjoin%
\definecolor{currentfill}{rgb}{0.994301,0.989660,0.990028}%
\pgfsetfillcolor{currentfill}%
\pgfsetlinewidth{0.000000pt}%
\definecolor{currentstroke}{rgb}{0.000000,0.000000,0.000000}%
\pgfsetstrokecolor{currentstroke}%
\pgfsetdash{}{0pt}%
\pgfpathmoveto{\pgfqpoint{1.725843in}{0.783876in}}%
\pgfpathlineto{\pgfqpoint{1.763083in}{0.794183in}}%
\pgfpathlineto{\pgfqpoint{1.725330in}{0.833450in}}%
\pgfpathlineto{\pgfqpoint{1.687982in}{0.823252in}}%
\pgfpathclose%
\pgfusepath{fill}%
\end{pgfscope}%
\begin{pgfscope}%
\pgfpathrectangle{\pgfqpoint{0.150000in}{0.150000in}}{\pgfqpoint{2.700000in}{1.950000in}}%
\pgfusepath{clip}%
\pgfsetbuttcap%
\pgfsetroundjoin%
\definecolor{currentfill}{rgb}{0.978232,0.980913,0.984666}%
\pgfsetfillcolor{currentfill}%
\pgfsetlinewidth{0.000000pt}%
\definecolor{currentstroke}{rgb}{0.000000,0.000000,0.000000}%
\pgfsetstrokecolor{currentstroke}%
\pgfsetdash{}{0pt}%
\pgfpathmoveto{\pgfqpoint{1.687982in}{0.823252in}}%
\pgfpathlineto{\pgfqpoint{1.725330in}{0.833450in}}%
\pgfpathlineto{\pgfqpoint{1.687571in}{0.872723in}}%
\pgfpathlineto{\pgfqpoint{1.650116in}{0.862633in}}%
\pgfpathclose%
\pgfusepath{fill}%
\end{pgfscope}%
\begin{pgfscope}%
\pgfpathrectangle{\pgfqpoint{0.150000in}{0.150000in}}{\pgfqpoint{2.700000in}{1.950000in}}%
\pgfusepath{clip}%
\pgfsetbuttcap%
\pgfsetroundjoin%
\definecolor{currentfill}{rgb}{0.947135,0.953646,0.962760}%
\pgfsetfillcolor{currentfill}%
\pgfsetlinewidth{0.000000pt}%
\definecolor{currentstroke}{rgb}{0.000000,0.000000,0.000000}%
\pgfsetstrokecolor{currentstroke}%
\pgfsetdash{}{0pt}%
\pgfpathmoveto{\pgfqpoint{1.650116in}{0.862633in}}%
\pgfpathlineto{\pgfqpoint{1.687571in}{0.872723in}}%
\pgfpathlineto{\pgfqpoint{1.649808in}{0.912000in}}%
\pgfpathlineto{\pgfqpoint{1.612245in}{0.902020in}}%
\pgfpathclose%
\pgfusepath{fill}%
\end{pgfscope}%
\begin{pgfscope}%
\pgfpathrectangle{\pgfqpoint{0.150000in}{0.150000in}}{\pgfqpoint{2.700000in}{1.950000in}}%
\pgfusepath{clip}%
\pgfsetbuttcap%
\pgfsetroundjoin%
\definecolor{currentfill}{rgb}{0.916039,0.926379,0.940855}%
\pgfsetfillcolor{currentfill}%
\pgfsetlinewidth{0.000000pt}%
\definecolor{currentstroke}{rgb}{0.000000,0.000000,0.000000}%
\pgfsetstrokecolor{currentstroke}%
\pgfsetdash{}{0pt}%
\pgfpathmoveto{\pgfqpoint{1.612245in}{0.902020in}}%
\pgfpathlineto{\pgfqpoint{1.649808in}{0.912000in}}%
\pgfpathlineto{\pgfqpoint{1.612039in}{0.951284in}}%
\pgfpathlineto{\pgfqpoint{1.574368in}{0.941412in}}%
\pgfpathclose%
\pgfusepath{fill}%
\end{pgfscope}%
\begin{pgfscope}%
\pgfpathrectangle{\pgfqpoint{0.150000in}{0.150000in}}{\pgfqpoint{2.700000in}{1.950000in}}%
\pgfusepath{clip}%
\pgfsetbuttcap%
\pgfsetroundjoin%
\definecolor{currentfill}{rgb}{0.884942,0.899112,0.918949}%
\pgfsetfillcolor{currentfill}%
\pgfsetlinewidth{0.000000pt}%
\definecolor{currentstroke}{rgb}{0.000000,0.000000,0.000000}%
\pgfsetstrokecolor{currentstroke}%
\pgfsetdash{}{0pt}%
\pgfpathmoveto{\pgfqpoint{1.574368in}{0.941412in}}%
\pgfpathlineto{\pgfqpoint{1.612039in}{0.951284in}}%
\pgfpathlineto{\pgfqpoint{1.574265in}{0.990572in}}%
\pgfpathlineto{\pgfqpoint{1.536486in}{0.980809in}}%
\pgfpathclose%
\pgfusepath{fill}%
\end{pgfscope}%
\begin{pgfscope}%
\pgfpathrectangle{\pgfqpoint{0.150000in}{0.150000in}}{\pgfqpoint{2.700000in}{1.950000in}}%
\pgfusepath{clip}%
\pgfsetbuttcap%
\pgfsetroundjoin%
\definecolor{currentfill}{rgb}{0.853845,0.871844,0.897044}%
\pgfsetfillcolor{currentfill}%
\pgfsetlinewidth{0.000000pt}%
\definecolor{currentstroke}{rgb}{0.000000,0.000000,0.000000}%
\pgfsetstrokecolor{currentstroke}%
\pgfsetdash{}{0pt}%
\pgfpathmoveto{\pgfqpoint{1.536486in}{0.980809in}}%
\pgfpathlineto{\pgfqpoint{1.574265in}{0.990572in}}%
\pgfpathlineto{\pgfqpoint{1.536486in}{1.029866in}}%
\pgfpathlineto{\pgfqpoint{1.498600in}{1.020212in}}%
\pgfpathclose%
\pgfusepath{fill}%
\end{pgfscope}%
\begin{pgfscope}%
\pgfpathrectangle{\pgfqpoint{0.150000in}{0.150000in}}{\pgfqpoint{2.700000in}{1.950000in}}%
\pgfusepath{clip}%
\pgfsetbuttcap%
\pgfsetroundjoin%
\definecolor{currentfill}{rgb}{0.822748,0.844577,0.875138}%
\pgfsetfillcolor{currentfill}%
\pgfsetlinewidth{0.000000pt}%
\definecolor{currentstroke}{rgb}{0.000000,0.000000,0.000000}%
\pgfsetstrokecolor{currentstroke}%
\pgfsetdash{}{0pt}%
\pgfpathmoveto{\pgfqpoint{1.498600in}{1.020212in}}%
\pgfpathlineto{\pgfqpoint{1.536486in}{1.029866in}}%
\pgfpathlineto{\pgfqpoint{1.498702in}{1.069165in}}%
\pgfpathlineto{\pgfqpoint{1.460708in}{1.059620in}}%
\pgfpathclose%
\pgfusepath{fill}%
\end{pgfscope}%
\begin{pgfscope}%
\pgfpathrectangle{\pgfqpoint{0.150000in}{0.150000in}}{\pgfqpoint{2.700000in}{1.950000in}}%
\pgfusepath{clip}%
\pgfsetbuttcap%
\pgfsetroundjoin%
\definecolor{currentfill}{rgb}{0.797871,0.822763,0.857613}%
\pgfsetfillcolor{currentfill}%
\pgfsetlinewidth{0.000000pt}%
\definecolor{currentstroke}{rgb}{0.000000,0.000000,0.000000}%
\pgfsetstrokecolor{currentstroke}%
\pgfsetdash{}{0pt}%
\pgfpathmoveto{\pgfqpoint{1.460708in}{1.059620in}}%
\pgfpathlineto{\pgfqpoint{1.498702in}{1.069165in}}%
\pgfpathlineto{\pgfqpoint{1.460913in}{1.108470in}}%
\pgfpathlineto{\pgfqpoint{1.422811in}{1.099033in}}%
\pgfpathclose%
\pgfusepath{fill}%
\end{pgfscope}%
\begin{pgfscope}%
\pgfpathrectangle{\pgfqpoint{0.150000in}{0.150000in}}{\pgfqpoint{2.700000in}{1.950000in}}%
\pgfusepath{clip}%
\pgfsetbuttcap%
\pgfsetroundjoin%
\definecolor{currentfill}{rgb}{0.766774,0.795496,0.835708}%
\pgfsetfillcolor{currentfill}%
\pgfsetlinewidth{0.000000pt}%
\definecolor{currentstroke}{rgb}{0.000000,0.000000,0.000000}%
\pgfsetstrokecolor{currentstroke}%
\pgfsetdash{}{0pt}%
\pgfpathmoveto{\pgfqpoint{1.422811in}{1.099033in}}%
\pgfpathlineto{\pgfqpoint{1.460913in}{1.108470in}}%
\pgfpathlineto{\pgfqpoint{1.423119in}{1.147779in}}%
\pgfpathlineto{\pgfqpoint{1.384908in}{1.138452in}}%
\pgfpathclose%
\pgfusepath{fill}%
\end{pgfscope}%
\begin{pgfscope}%
\pgfpathrectangle{\pgfqpoint{0.150000in}{0.150000in}}{\pgfqpoint{2.700000in}{1.950000in}}%
\pgfusepath{clip}%
\pgfsetbuttcap%
\pgfsetroundjoin%
\definecolor{currentfill}{rgb}{0.735677,0.768229,0.813802}%
\pgfsetfillcolor{currentfill}%
\pgfsetlinewidth{0.000000pt}%
\definecolor{currentstroke}{rgb}{0.000000,0.000000,0.000000}%
\pgfsetstrokecolor{currentstroke}%
\pgfsetdash{}{0pt}%
\pgfpathmoveto{\pgfqpoint{1.384908in}{1.138452in}}%
\pgfpathlineto{\pgfqpoint{1.423119in}{1.147779in}}%
\pgfpathlineto{\pgfqpoint{1.385320in}{1.187095in}}%
\pgfpathlineto{\pgfqpoint{1.347001in}{1.177876in}}%
\pgfpathclose%
\pgfusepath{fill}%
\end{pgfscope}%
\begin{pgfscope}%
\pgfpathrectangle{\pgfqpoint{0.150000in}{0.150000in}}{\pgfqpoint{2.700000in}{1.950000in}}%
\pgfusepath{clip}%
\pgfsetbuttcap%
\pgfsetroundjoin%
\definecolor{currentfill}{rgb}{0.704580,0.740962,0.791896}%
\pgfsetfillcolor{currentfill}%
\pgfsetlinewidth{0.000000pt}%
\definecolor{currentstroke}{rgb}{0.000000,0.000000,0.000000}%
\pgfsetstrokecolor{currentstroke}%
\pgfsetdash{}{0pt}%
\pgfpathmoveto{\pgfqpoint{1.347001in}{1.177876in}}%
\pgfpathlineto{\pgfqpoint{1.385320in}{1.187095in}}%
\pgfpathlineto{\pgfqpoint{1.347515in}{1.226415in}}%
\pgfpathlineto{\pgfqpoint{1.309088in}{1.217306in}}%
\pgfpathclose%
\pgfusepath{fill}%
\end{pgfscope}%
\begin{pgfscope}%
\pgfpathrectangle{\pgfqpoint{0.150000in}{0.150000in}}{\pgfqpoint{2.700000in}{1.950000in}}%
\pgfusepath{clip}%
\pgfsetbuttcap%
\pgfsetroundjoin%
\definecolor{currentfill}{rgb}{0.673483,0.713695,0.769991}%
\pgfsetfillcolor{currentfill}%
\pgfsetlinewidth{0.000000pt}%
\definecolor{currentstroke}{rgb}{0.000000,0.000000,0.000000}%
\pgfsetstrokecolor{currentstroke}%
\pgfsetdash{}{0pt}%
\pgfpathmoveto{\pgfqpoint{1.309088in}{1.217306in}}%
\pgfpathlineto{\pgfqpoint{1.347515in}{1.226415in}}%
\pgfpathlineto{\pgfqpoint{1.309705in}{1.265741in}}%
\pgfpathlineto{\pgfqpoint{1.271171in}{1.256741in}}%
\pgfpathclose%
\pgfusepath{fill}%
\end{pgfscope}%
\begin{pgfscope}%
\pgfpathrectangle{\pgfqpoint{0.150000in}{0.150000in}}{\pgfqpoint{2.700000in}{1.950000in}}%
\pgfusepath{clip}%
\pgfsetbuttcap%
\pgfsetroundjoin%
\definecolor{currentfill}{rgb}{0.642387,0.686428,0.748085}%
\pgfsetfillcolor{currentfill}%
\pgfsetlinewidth{0.000000pt}%
\definecolor{currentstroke}{rgb}{0.000000,0.000000,0.000000}%
\pgfsetstrokecolor{currentstroke}%
\pgfsetdash{}{0pt}%
\pgfpathmoveto{\pgfqpoint{1.271171in}{1.256741in}}%
\pgfpathlineto{\pgfqpoint{1.309705in}{1.265741in}}%
\pgfpathlineto{\pgfqpoint{1.271891in}{1.305072in}}%
\pgfpathlineto{\pgfqpoint{1.233248in}{1.296181in}}%
\pgfpathclose%
\pgfusepath{fill}%
\end{pgfscope}%
\begin{pgfscope}%
\pgfpathrectangle{\pgfqpoint{0.150000in}{0.150000in}}{\pgfqpoint{2.700000in}{1.950000in}}%
\pgfusepath{clip}%
\pgfsetbuttcap%
\pgfsetroundjoin%
\definecolor{currentfill}{rgb}{0.611290,0.659161,0.726180}%
\pgfsetfillcolor{currentfill}%
\pgfsetlinewidth{0.000000pt}%
\definecolor{currentstroke}{rgb}{0.000000,0.000000,0.000000}%
\pgfsetstrokecolor{currentstroke}%
\pgfsetdash{}{0pt}%
\pgfpathmoveto{\pgfqpoint{1.233248in}{1.296181in}}%
\pgfpathlineto{\pgfqpoint{1.271891in}{1.305072in}}%
\pgfpathlineto{\pgfqpoint{1.234071in}{1.344409in}}%
\pgfpathlineto{\pgfqpoint{1.195320in}{1.335627in}}%
\pgfpathclose%
\pgfusepath{fill}%
\end{pgfscope}%
\begin{pgfscope}%
\pgfpathrectangle{\pgfqpoint{0.150000in}{0.150000in}}{\pgfqpoint{2.700000in}{1.950000in}}%
\pgfusepath{clip}%
\pgfsetbuttcap%
\pgfsetroundjoin%
\definecolor{currentfill}{rgb}{0.580193,0.631893,0.704274}%
\pgfsetfillcolor{currentfill}%
\pgfsetlinewidth{0.000000pt}%
\definecolor{currentstroke}{rgb}{0.000000,0.000000,0.000000}%
\pgfsetstrokecolor{currentstroke}%
\pgfsetdash{}{0pt}%
\pgfpathmoveto{\pgfqpoint{1.195320in}{1.335627in}}%
\pgfpathlineto{\pgfqpoint{1.234071in}{1.344409in}}%
\pgfpathlineto{\pgfqpoint{1.196245in}{1.383750in}}%
\pgfpathlineto{\pgfqpoint{1.157386in}{1.375078in}}%
\pgfpathclose%
\pgfusepath{fill}%
\end{pgfscope}%
\begin{pgfscope}%
\pgfpathrectangle{\pgfqpoint{0.150000in}{0.150000in}}{\pgfqpoint{2.700000in}{1.950000in}}%
\pgfusepath{clip}%
\pgfsetbuttcap%
\pgfsetroundjoin%
\definecolor{currentfill}{rgb}{0.555316,0.610080,0.686749}%
\pgfsetfillcolor{currentfill}%
\pgfsetlinewidth{0.000000pt}%
\definecolor{currentstroke}{rgb}{0.000000,0.000000,0.000000}%
\pgfsetstrokecolor{currentstroke}%
\pgfsetdash{}{0pt}%
\pgfpathmoveto{\pgfqpoint{1.157386in}{1.375078in}}%
\pgfpathlineto{\pgfqpoint{1.196245in}{1.383750in}}%
\pgfpathlineto{\pgfqpoint{1.158415in}{1.423098in}}%
\pgfpathlineto{\pgfqpoint{1.119448in}{1.414534in}}%
\pgfpathclose%
\pgfusepath{fill}%
\end{pgfscope}%
\begin{pgfscope}%
\pgfpathrectangle{\pgfqpoint{0.150000in}{0.150000in}}{\pgfqpoint{2.700000in}{1.950000in}}%
\pgfusepath{clip}%
\pgfsetbuttcap%
\pgfsetroundjoin%
\definecolor{currentfill}{rgb}{0.524219,0.582812,0.664844}%
\pgfsetfillcolor{currentfill}%
\pgfsetlinewidth{0.000000pt}%
\definecolor{currentstroke}{rgb}{0.000000,0.000000,0.000000}%
\pgfsetstrokecolor{currentstroke}%
\pgfsetdash{}{0pt}%
\pgfpathmoveto{\pgfqpoint{1.119448in}{1.414534in}}%
\pgfpathlineto{\pgfqpoint{1.158415in}{1.423098in}}%
\pgfpathlineto{\pgfqpoint{1.120580in}{1.462450in}}%
\pgfpathlineto{\pgfqpoint{1.081504in}{1.453996in}}%
\pgfpathclose%
\pgfusepath{fill}%
\end{pgfscope}%
\begin{pgfscope}%
\pgfpathrectangle{\pgfqpoint{0.150000in}{0.150000in}}{\pgfqpoint{2.700000in}{1.950000in}}%
\pgfusepath{clip}%
\pgfsetbuttcap%
\pgfsetroundjoin%
\definecolor{currentfill}{rgb}{0.493122,0.555545,0.642938}%
\pgfsetfillcolor{currentfill}%
\pgfsetlinewidth{0.000000pt}%
\definecolor{currentstroke}{rgb}{0.000000,0.000000,0.000000}%
\pgfsetstrokecolor{currentstroke}%
\pgfsetdash{}{0pt}%
\pgfpathmoveto{\pgfqpoint{1.081504in}{1.453996in}}%
\pgfpathlineto{\pgfqpoint{1.120580in}{1.462450in}}%
\pgfpathlineto{\pgfqpoint{1.082739in}{1.501808in}}%
\pgfpathlineto{\pgfqpoint{1.043556in}{1.493463in}}%
\pgfpathclose%
\pgfusepath{fill}%
\end{pgfscope}%
\begin{pgfscope}%
\pgfpathrectangle{\pgfqpoint{0.150000in}{0.150000in}}{\pgfqpoint{2.700000in}{1.950000in}}%
\pgfusepath{clip}%
\pgfsetbuttcap%
\pgfsetroundjoin%
\definecolor{currentfill}{rgb}{0.462025,0.528278,0.621032}%
\pgfsetfillcolor{currentfill}%
\pgfsetlinewidth{0.000000pt}%
\definecolor{currentstroke}{rgb}{0.000000,0.000000,0.000000}%
\pgfsetstrokecolor{currentstroke}%
\pgfsetdash{}{0pt}%
\pgfpathmoveto{\pgfqpoint{1.043556in}{1.493463in}}%
\pgfpathlineto{\pgfqpoint{1.082739in}{1.501808in}}%
\pgfpathlineto{\pgfqpoint{1.044894in}{1.541171in}}%
\pgfpathlineto{\pgfqpoint{1.005602in}{1.532935in}}%
\pgfpathclose%
\pgfusepath{fill}%
\end{pgfscope}%
\begin{pgfscope}%
\pgfpathrectangle{\pgfqpoint{0.150000in}{0.150000in}}{\pgfqpoint{2.700000in}{1.950000in}}%
\pgfusepath{clip}%
\pgfsetbuttcap%
\pgfsetroundjoin%
\definecolor{currentfill}{rgb}{0.430928,0.501011,0.599127}%
\pgfsetfillcolor{currentfill}%
\pgfsetlinewidth{0.000000pt}%
\definecolor{currentstroke}{rgb}{0.000000,0.000000,0.000000}%
\pgfsetstrokecolor{currentstroke}%
\pgfsetdash{}{0pt}%
\pgfpathmoveto{\pgfqpoint{1.005602in}{1.532935in}}%
\pgfpathlineto{\pgfqpoint{1.044894in}{1.541171in}}%
\pgfpathlineto{\pgfqpoint{1.007043in}{1.580540in}}%
\pgfpathlineto{\pgfqpoint{0.967643in}{1.572413in}}%
\pgfpathclose%
\pgfusepath{fill}%
\end{pgfscope}%
\begin{pgfscope}%
\pgfpathrectangle{\pgfqpoint{0.150000in}{0.150000in}}{\pgfqpoint{2.700000in}{1.950000in}}%
\pgfusepath{clip}%
\pgfsetbuttcap%
\pgfsetroundjoin%
\definecolor{currentfill}{rgb}{0.399831,0.473744,0.577221}%
\pgfsetfillcolor{currentfill}%
\pgfsetlinewidth{0.000000pt}%
\definecolor{currentstroke}{rgb}{0.000000,0.000000,0.000000}%
\pgfsetstrokecolor{currentstroke}%
\pgfsetdash{}{0pt}%
\pgfpathmoveto{\pgfqpoint{0.967643in}{1.572413in}}%
\pgfpathlineto{\pgfqpoint{1.007043in}{1.580540in}}%
\pgfpathlineto{\pgfqpoint{0.969187in}{1.619914in}}%
\pgfpathlineto{\pgfqpoint{0.929678in}{1.611897in}}%
\pgfpathclose%
\pgfusepath{fill}%
\end{pgfscope}%
\begin{pgfscope}%
\pgfpathrectangle{\pgfqpoint{0.150000in}{0.150000in}}{\pgfqpoint{2.700000in}{1.950000in}}%
\pgfusepath{clip}%
\pgfsetbuttcap%
\pgfsetroundjoin%
\definecolor{currentfill}{rgb}{0.929718,0.872472,0.877007}%
\pgfsetfillcolor{currentfill}%
\pgfsetlinewidth{0.000000pt}%
\definecolor{currentstroke}{rgb}{0.000000,0.000000,0.000000}%
\pgfsetstrokecolor{currentstroke}%
\pgfsetdash{}{0pt}%
\pgfpathmoveto{\pgfqpoint{1.840221in}{0.615625in}}%
\pgfpathlineto{\pgfqpoint{1.877235in}{0.626426in}}%
\pgfpathlineto{\pgfqpoint{1.839395in}{0.665781in}}%
\pgfpathlineto{\pgfqpoint{1.802272in}{0.655089in}}%
\pgfpathclose%
\pgfusepath{fill}%
\end{pgfscope}%
\begin{pgfscope}%
\pgfpathrectangle{\pgfqpoint{0.150000in}{0.150000in}}{\pgfqpoint{2.700000in}{1.950000in}}%
\pgfusepath{clip}%
\pgfsetbuttcap%
\pgfsetroundjoin%
\definecolor{currentfill}{rgb}{0.948713,0.906939,0.910248}%
\pgfsetfillcolor{currentfill}%
\pgfsetlinewidth{0.000000pt}%
\definecolor{currentstroke}{rgb}{0.000000,0.000000,0.000000}%
\pgfsetstrokecolor{currentstroke}%
\pgfsetdash{}{0pt}%
\pgfpathmoveto{\pgfqpoint{1.802272in}{0.655089in}}%
\pgfpathlineto{\pgfqpoint{1.839395in}{0.665781in}}%
\pgfpathlineto{\pgfqpoint{1.801550in}{0.705141in}}%
\pgfpathlineto{\pgfqpoint{1.764318in}{0.694558in}}%
\pgfpathclose%
\pgfusepath{fill}%
\end{pgfscope}%
\begin{pgfscope}%
\pgfpathrectangle{\pgfqpoint{0.150000in}{0.150000in}}{\pgfqpoint{2.700000in}{1.950000in}}%
\pgfusepath{clip}%
\pgfsetbuttcap%
\pgfsetroundjoin%
\definecolor{currentfill}{rgb}{0.967708,0.941406,0.943490}%
\pgfsetfillcolor{currentfill}%
\pgfsetlinewidth{0.000000pt}%
\definecolor{currentstroke}{rgb}{0.000000,0.000000,0.000000}%
\pgfsetstrokecolor{currentstroke}%
\pgfsetdash{}{0pt}%
\pgfpathmoveto{\pgfqpoint{1.764318in}{0.694558in}}%
\pgfpathlineto{\pgfqpoint{1.801550in}{0.705141in}}%
\pgfpathlineto{\pgfqpoint{1.763699in}{0.744506in}}%
\pgfpathlineto{\pgfqpoint{1.726359in}{0.734032in}}%
\pgfpathclose%
\pgfusepath{fill}%
\end{pgfscope}%
\begin{pgfscope}%
\pgfpathrectangle{\pgfqpoint{0.150000in}{0.150000in}}{\pgfqpoint{2.700000in}{1.950000in}}%
\pgfusepath{clip}%
\pgfsetbuttcap%
\pgfsetroundjoin%
\definecolor{currentfill}{rgb}{0.986703,0.975873,0.976731}%
\pgfsetfillcolor{currentfill}%
\pgfsetlinewidth{0.000000pt}%
\definecolor{currentstroke}{rgb}{0.000000,0.000000,0.000000}%
\pgfsetstrokecolor{currentstroke}%
\pgfsetdash{}{0pt}%
\pgfpathmoveto{\pgfqpoint{1.726359in}{0.734032in}}%
\pgfpathlineto{\pgfqpoint{1.763699in}{0.744506in}}%
\pgfpathlineto{\pgfqpoint{1.725843in}{0.783876in}}%
\pgfpathlineto{\pgfqpoint{1.688395in}{0.773512in}}%
\pgfpathclose%
\pgfusepath{fill}%
\end{pgfscope}%
\begin{pgfscope}%
\pgfpathrectangle{\pgfqpoint{0.150000in}{0.150000in}}{\pgfqpoint{2.700000in}{1.950000in}}%
\pgfusepath{clip}%
\pgfsetbuttcap%
\pgfsetroundjoin%
\definecolor{currentfill}{rgb}{0.990671,0.991820,0.993428}%
\pgfsetfillcolor{currentfill}%
\pgfsetlinewidth{0.000000pt}%
\definecolor{currentstroke}{rgb}{0.000000,0.000000,0.000000}%
\pgfsetstrokecolor{currentstroke}%
\pgfsetdash{}{0pt}%
\pgfpathmoveto{\pgfqpoint{1.688395in}{0.773512in}}%
\pgfpathlineto{\pgfqpoint{1.725843in}{0.783876in}}%
\pgfpathlineto{\pgfqpoint{1.687982in}{0.823252in}}%
\pgfpathlineto{\pgfqpoint{1.650426in}{0.812997in}}%
\pgfpathclose%
\pgfusepath{fill}%
\end{pgfscope}%
\begin{pgfscope}%
\pgfpathrectangle{\pgfqpoint{0.150000in}{0.150000in}}{\pgfqpoint{2.700000in}{1.950000in}}%
\pgfusepath{clip}%
\pgfsetbuttcap%
\pgfsetroundjoin%
\definecolor{currentfill}{rgb}{0.959574,0.964553,0.971523}%
\pgfsetfillcolor{currentfill}%
\pgfsetlinewidth{0.000000pt}%
\definecolor{currentstroke}{rgb}{0.000000,0.000000,0.000000}%
\pgfsetstrokecolor{currentstroke}%
\pgfsetdash{}{0pt}%
\pgfpathmoveto{\pgfqpoint{1.650426in}{0.812997in}}%
\pgfpathlineto{\pgfqpoint{1.687982in}{0.823252in}}%
\pgfpathlineto{\pgfqpoint{1.650116in}{0.862633in}}%
\pgfpathlineto{\pgfqpoint{1.612451in}{0.852488in}}%
\pgfpathclose%
\pgfusepath{fill}%
\end{pgfscope}%
\begin{pgfscope}%
\pgfpathrectangle{\pgfqpoint{0.150000in}{0.150000in}}{\pgfqpoint{2.700000in}{1.950000in}}%
\pgfusepath{clip}%
\pgfsetbuttcap%
\pgfsetroundjoin%
\definecolor{currentfill}{rgb}{0.934697,0.942739,0.953998}%
\pgfsetfillcolor{currentfill}%
\pgfsetlinewidth{0.000000pt}%
\definecolor{currentstroke}{rgb}{0.000000,0.000000,0.000000}%
\pgfsetstrokecolor{currentstroke}%
\pgfsetdash{}{0pt}%
\pgfpathmoveto{\pgfqpoint{1.612451in}{0.852488in}}%
\pgfpathlineto{\pgfqpoint{1.650116in}{0.862633in}}%
\pgfpathlineto{\pgfqpoint{1.612245in}{0.902020in}}%
\pgfpathlineto{\pgfqpoint{1.574471in}{0.891984in}}%
\pgfpathclose%
\pgfusepath{fill}%
\end{pgfscope}%
\begin{pgfscope}%
\pgfpathrectangle{\pgfqpoint{0.150000in}{0.150000in}}{\pgfqpoint{2.700000in}{1.950000in}}%
\pgfusepath{clip}%
\pgfsetbuttcap%
\pgfsetroundjoin%
\definecolor{currentfill}{rgb}{0.903600,0.915472,0.932093}%
\pgfsetfillcolor{currentfill}%
\pgfsetlinewidth{0.000000pt}%
\definecolor{currentstroke}{rgb}{0.000000,0.000000,0.000000}%
\pgfsetstrokecolor{currentstroke}%
\pgfsetdash{}{0pt}%
\pgfpathmoveto{\pgfqpoint{1.574471in}{0.891984in}}%
\pgfpathlineto{\pgfqpoint{1.612245in}{0.902020in}}%
\pgfpathlineto{\pgfqpoint{1.574368in}{0.941412in}}%
\pgfpathlineto{\pgfqpoint{1.536486in}{0.931485in}}%
\pgfpathclose%
\pgfusepath{fill}%
\end{pgfscope}%
\begin{pgfscope}%
\pgfpathrectangle{\pgfqpoint{0.150000in}{0.150000in}}{\pgfqpoint{2.700000in}{1.950000in}}%
\pgfusepath{clip}%
\pgfsetbuttcap%
\pgfsetroundjoin%
\definecolor{currentfill}{rgb}{0.872503,0.888205,0.910187}%
\pgfsetfillcolor{currentfill}%
\pgfsetlinewidth{0.000000pt}%
\definecolor{currentstroke}{rgb}{0.000000,0.000000,0.000000}%
\pgfsetstrokecolor{currentstroke}%
\pgfsetdash{}{0pt}%
\pgfpathmoveto{\pgfqpoint{1.536486in}{0.931485in}}%
\pgfpathlineto{\pgfqpoint{1.574368in}{0.941412in}}%
\pgfpathlineto{\pgfqpoint{1.536486in}{0.980809in}}%
\pgfpathlineto{\pgfqpoint{1.498496in}{0.970992in}}%
\pgfpathclose%
\pgfusepath{fill}%
\end{pgfscope}%
\begin{pgfscope}%
\pgfpathrectangle{\pgfqpoint{0.150000in}{0.150000in}}{\pgfqpoint{2.700000in}{1.950000in}}%
\pgfusepath{clip}%
\pgfsetbuttcap%
\pgfsetroundjoin%
\definecolor{currentfill}{rgb}{0.841406,0.860938,0.888281}%
\pgfsetfillcolor{currentfill}%
\pgfsetlinewidth{0.000000pt}%
\definecolor{currentstroke}{rgb}{0.000000,0.000000,0.000000}%
\pgfsetstrokecolor{currentstroke}%
\pgfsetdash{}{0pt}%
\pgfpathmoveto{\pgfqpoint{1.498496in}{0.970992in}}%
\pgfpathlineto{\pgfqpoint{1.536486in}{0.980809in}}%
\pgfpathlineto{\pgfqpoint{1.498600in}{1.020212in}}%
\pgfpathlineto{\pgfqpoint{1.460501in}{1.010504in}}%
\pgfpathclose%
\pgfusepath{fill}%
\end{pgfscope}%
\begin{pgfscope}%
\pgfpathrectangle{\pgfqpoint{0.150000in}{0.150000in}}{\pgfqpoint{2.700000in}{1.950000in}}%
\pgfusepath{clip}%
\pgfsetbuttcap%
\pgfsetroundjoin%
\definecolor{currentfill}{rgb}{0.810309,0.833670,0.866376}%
\pgfsetfillcolor{currentfill}%
\pgfsetlinewidth{0.000000pt}%
\definecolor{currentstroke}{rgb}{0.000000,0.000000,0.000000}%
\pgfsetstrokecolor{currentstroke}%
\pgfsetdash{}{0pt}%
\pgfpathmoveto{\pgfqpoint{1.460501in}{1.010504in}}%
\pgfpathlineto{\pgfqpoint{1.498600in}{1.020212in}}%
\pgfpathlineto{\pgfqpoint{1.460708in}{1.059620in}}%
\pgfpathlineto{\pgfqpoint{1.422501in}{1.050021in}}%
\pgfpathclose%
\pgfusepath{fill}%
\end{pgfscope}%
\begin{pgfscope}%
\pgfpathrectangle{\pgfqpoint{0.150000in}{0.150000in}}{\pgfqpoint{2.700000in}{1.950000in}}%
\pgfusepath{clip}%
\pgfsetbuttcap%
\pgfsetroundjoin%
\definecolor{currentfill}{rgb}{0.779213,0.806403,0.844470}%
\pgfsetfillcolor{currentfill}%
\pgfsetlinewidth{0.000000pt}%
\definecolor{currentstroke}{rgb}{0.000000,0.000000,0.000000}%
\pgfsetstrokecolor{currentstroke}%
\pgfsetdash{}{0pt}%
\pgfpathmoveto{\pgfqpoint{1.422501in}{1.050021in}}%
\pgfpathlineto{\pgfqpoint{1.460708in}{1.059620in}}%
\pgfpathlineto{\pgfqpoint{1.422811in}{1.099033in}}%
\pgfpathlineto{\pgfqpoint{1.384495in}{1.089544in}}%
\pgfpathclose%
\pgfusepath{fill}%
\end{pgfscope}%
\begin{pgfscope}%
\pgfpathrectangle{\pgfqpoint{0.150000in}{0.150000in}}{\pgfqpoint{2.700000in}{1.950000in}}%
\pgfusepath{clip}%
\pgfsetbuttcap%
\pgfsetroundjoin%
\definecolor{currentfill}{rgb}{0.748116,0.779136,0.822564}%
\pgfsetfillcolor{currentfill}%
\pgfsetlinewidth{0.000000pt}%
\definecolor{currentstroke}{rgb}{0.000000,0.000000,0.000000}%
\pgfsetstrokecolor{currentstroke}%
\pgfsetdash{}{0pt}%
\pgfpathmoveto{\pgfqpoint{1.384495in}{1.089544in}}%
\pgfpathlineto{\pgfqpoint{1.422811in}{1.099033in}}%
\pgfpathlineto{\pgfqpoint{1.384908in}{1.138452in}}%
\pgfpathlineto{\pgfqpoint{1.346484in}{1.129073in}}%
\pgfpathclose%
\pgfusepath{fill}%
\end{pgfscope}%
\begin{pgfscope}%
\pgfpathrectangle{\pgfqpoint{0.150000in}{0.150000in}}{\pgfqpoint{2.700000in}{1.950000in}}%
\pgfusepath{clip}%
\pgfsetbuttcap%
\pgfsetroundjoin%
\definecolor{currentfill}{rgb}{0.717019,0.751869,0.800659}%
\pgfsetfillcolor{currentfill}%
\pgfsetlinewidth{0.000000pt}%
\definecolor{currentstroke}{rgb}{0.000000,0.000000,0.000000}%
\pgfsetstrokecolor{currentstroke}%
\pgfsetdash{}{0pt}%
\pgfpathmoveto{\pgfqpoint{1.346484in}{1.129073in}}%
\pgfpathlineto{\pgfqpoint{1.384908in}{1.138452in}}%
\pgfpathlineto{\pgfqpoint{1.347001in}{1.177876in}}%
\pgfpathlineto{\pgfqpoint{1.308468in}{1.168607in}}%
\pgfpathclose%
\pgfusepath{fill}%
\end{pgfscope}%
\begin{pgfscope}%
\pgfpathrectangle{\pgfqpoint{0.150000in}{0.150000in}}{\pgfqpoint{2.700000in}{1.950000in}}%
\pgfusepath{clip}%
\pgfsetbuttcap%
\pgfsetroundjoin%
\definecolor{currentfill}{rgb}{0.692142,0.730055,0.783134}%
\pgfsetfillcolor{currentfill}%
\pgfsetlinewidth{0.000000pt}%
\definecolor{currentstroke}{rgb}{0.000000,0.000000,0.000000}%
\pgfsetstrokecolor{currentstroke}%
\pgfsetdash{}{0pt}%
\pgfpathmoveto{\pgfqpoint{1.308468in}{1.168607in}}%
\pgfpathlineto{\pgfqpoint{1.347001in}{1.177876in}}%
\pgfpathlineto{\pgfqpoint{1.309088in}{1.217306in}}%
\pgfpathlineto{\pgfqpoint{1.270447in}{1.208146in}}%
\pgfpathclose%
\pgfusepath{fill}%
\end{pgfscope}%
\begin{pgfscope}%
\pgfpathrectangle{\pgfqpoint{0.150000in}{0.150000in}}{\pgfqpoint{2.700000in}{1.950000in}}%
\pgfusepath{clip}%
\pgfsetbuttcap%
\pgfsetroundjoin%
\definecolor{currentfill}{rgb}{0.661045,0.702788,0.761229}%
\pgfsetfillcolor{currentfill}%
\pgfsetlinewidth{0.000000pt}%
\definecolor{currentstroke}{rgb}{0.000000,0.000000,0.000000}%
\pgfsetstrokecolor{currentstroke}%
\pgfsetdash{}{0pt}%
\pgfpathmoveto{\pgfqpoint{1.270447in}{1.208146in}}%
\pgfpathlineto{\pgfqpoint{1.309088in}{1.217306in}}%
\pgfpathlineto{\pgfqpoint{1.271171in}{1.256741in}}%
\pgfpathlineto{\pgfqpoint{1.232421in}{1.247690in}}%
\pgfpathclose%
\pgfusepath{fill}%
\end{pgfscope}%
\begin{pgfscope}%
\pgfpathrectangle{\pgfqpoint{0.150000in}{0.150000in}}{\pgfqpoint{2.700000in}{1.950000in}}%
\pgfusepath{clip}%
\pgfsetbuttcap%
\pgfsetroundjoin%
\definecolor{currentfill}{rgb}{0.629948,0.675521,0.739323}%
\pgfsetfillcolor{currentfill}%
\pgfsetlinewidth{0.000000pt}%
\definecolor{currentstroke}{rgb}{0.000000,0.000000,0.000000}%
\pgfsetstrokecolor{currentstroke}%
\pgfsetdash{}{0pt}%
\pgfpathmoveto{\pgfqpoint{1.232421in}{1.247690in}}%
\pgfpathlineto{\pgfqpoint{1.271171in}{1.256741in}}%
\pgfpathlineto{\pgfqpoint{1.233248in}{1.296181in}}%
\pgfpathlineto{\pgfqpoint{1.194389in}{1.287240in}}%
\pgfpathclose%
\pgfusepath{fill}%
\end{pgfscope}%
\begin{pgfscope}%
\pgfpathrectangle{\pgfqpoint{0.150000in}{0.150000in}}{\pgfqpoint{2.700000in}{1.950000in}}%
\pgfusepath{clip}%
\pgfsetbuttcap%
\pgfsetroundjoin%
\definecolor{currentfill}{rgb}{0.598851,0.648254,0.717417}%
\pgfsetfillcolor{currentfill}%
\pgfsetlinewidth{0.000000pt}%
\definecolor{currentstroke}{rgb}{0.000000,0.000000,0.000000}%
\pgfsetstrokecolor{currentstroke}%
\pgfsetdash{}{0pt}%
\pgfpathmoveto{\pgfqpoint{1.194389in}{1.287240in}}%
\pgfpathlineto{\pgfqpoint{1.233248in}{1.296181in}}%
\pgfpathlineto{\pgfqpoint{1.195320in}{1.335627in}}%
\pgfpathlineto{\pgfqpoint{1.156352in}{1.326795in}}%
\pgfpathclose%
\pgfusepath{fill}%
\end{pgfscope}%
\begin{pgfscope}%
\pgfpathrectangle{\pgfqpoint{0.150000in}{0.150000in}}{\pgfqpoint{2.700000in}{1.950000in}}%
\pgfusepath{clip}%
\pgfsetbuttcap%
\pgfsetroundjoin%
\definecolor{currentfill}{rgb}{0.567754,0.620987,0.695512}%
\pgfsetfillcolor{currentfill}%
\pgfsetlinewidth{0.000000pt}%
\definecolor{currentstroke}{rgb}{0.000000,0.000000,0.000000}%
\pgfsetstrokecolor{currentstroke}%
\pgfsetdash{}{0pt}%
\pgfpathmoveto{\pgfqpoint{1.156352in}{1.326795in}}%
\pgfpathlineto{\pgfqpoint{1.195320in}{1.335627in}}%
\pgfpathlineto{\pgfqpoint{1.157386in}{1.375078in}}%
\pgfpathlineto{\pgfqpoint{1.118310in}{1.366356in}}%
\pgfpathclose%
\pgfusepath{fill}%
\end{pgfscope}%
\begin{pgfscope}%
\pgfpathrectangle{\pgfqpoint{0.150000in}{0.150000in}}{\pgfqpoint{2.700000in}{1.950000in}}%
\pgfusepath{clip}%
\pgfsetbuttcap%
\pgfsetroundjoin%
\definecolor{currentfill}{rgb}{0.536657,0.593719,0.673606}%
\pgfsetfillcolor{currentfill}%
\pgfsetlinewidth{0.000000pt}%
\definecolor{currentstroke}{rgb}{0.000000,0.000000,0.000000}%
\pgfsetstrokecolor{currentstroke}%
\pgfsetdash{}{0pt}%
\pgfpathmoveto{\pgfqpoint{1.118310in}{1.366356in}}%
\pgfpathlineto{\pgfqpoint{1.157386in}{1.375078in}}%
\pgfpathlineto{\pgfqpoint{1.119448in}{1.414534in}}%
\pgfpathlineto{\pgfqpoint{1.080263in}{1.405922in}}%
\pgfpathclose%
\pgfusepath{fill}%
\end{pgfscope}%
\begin{pgfscope}%
\pgfpathrectangle{\pgfqpoint{0.150000in}{0.150000in}}{\pgfqpoint{2.700000in}{1.950000in}}%
\pgfusepath{clip}%
\pgfsetbuttcap%
\pgfsetroundjoin%
\definecolor{currentfill}{rgb}{0.505561,0.566452,0.651700}%
\pgfsetfillcolor{currentfill}%
\pgfsetlinewidth{0.000000pt}%
\definecolor{currentstroke}{rgb}{0.000000,0.000000,0.000000}%
\pgfsetstrokecolor{currentstroke}%
\pgfsetdash{}{0pt}%
\pgfpathmoveto{\pgfqpoint{1.080263in}{1.405922in}}%
\pgfpathlineto{\pgfqpoint{1.119448in}{1.414534in}}%
\pgfpathlineto{\pgfqpoint{1.081504in}{1.453996in}}%
\pgfpathlineto{\pgfqpoint{1.042210in}{1.445494in}}%
\pgfpathclose%
\pgfusepath{fill}%
\end{pgfscope}%
\begin{pgfscope}%
\pgfpathrectangle{\pgfqpoint{0.150000in}{0.150000in}}{\pgfqpoint{2.700000in}{1.950000in}}%
\pgfusepath{clip}%
\pgfsetbuttcap%
\pgfsetroundjoin%
\definecolor{currentfill}{rgb}{0.474464,0.539185,0.629795}%
\pgfsetfillcolor{currentfill}%
\pgfsetlinewidth{0.000000pt}%
\definecolor{currentstroke}{rgb}{0.000000,0.000000,0.000000}%
\pgfsetstrokecolor{currentstroke}%
\pgfsetdash{}{0pt}%
\pgfpathmoveto{\pgfqpoint{1.042210in}{1.445494in}}%
\pgfpathlineto{\pgfqpoint{1.081504in}{1.453996in}}%
\pgfpathlineto{\pgfqpoint{1.043556in}{1.493463in}}%
\pgfpathlineto{\pgfqpoint{1.004153in}{1.485071in}}%
\pgfpathclose%
\pgfusepath{fill}%
\end{pgfscope}%
\begin{pgfscope}%
\pgfpathrectangle{\pgfqpoint{0.150000in}{0.150000in}}{\pgfqpoint{2.700000in}{1.950000in}}%
\pgfusepath{clip}%
\pgfsetbuttcap%
\pgfsetroundjoin%
\definecolor{currentfill}{rgb}{0.449586,0.517371,0.612270}%
\pgfsetfillcolor{currentfill}%
\pgfsetlinewidth{0.000000pt}%
\definecolor{currentstroke}{rgb}{0.000000,0.000000,0.000000}%
\pgfsetstrokecolor{currentstroke}%
\pgfsetdash{}{0pt}%
\pgfpathmoveto{\pgfqpoint{1.004153in}{1.485071in}}%
\pgfpathlineto{\pgfqpoint{1.043556in}{1.493463in}}%
\pgfpathlineto{\pgfqpoint{1.005602in}{1.532935in}}%
\pgfpathlineto{\pgfqpoint{0.966090in}{1.524653in}}%
\pgfpathclose%
\pgfusepath{fill}%
\end{pgfscope}%
\begin{pgfscope}%
\pgfpathrectangle{\pgfqpoint{0.150000in}{0.150000in}}{\pgfqpoint{2.700000in}{1.950000in}}%
\pgfusepath{clip}%
\pgfsetbuttcap%
\pgfsetroundjoin%
\definecolor{currentfill}{rgb}{0.418490,0.490104,0.590365}%
\pgfsetfillcolor{currentfill}%
\pgfsetlinewidth{0.000000pt}%
\definecolor{currentstroke}{rgb}{0.000000,0.000000,0.000000}%
\pgfsetstrokecolor{currentstroke}%
\pgfsetdash{}{0pt}%
\pgfpathmoveto{\pgfqpoint{0.966090in}{1.524653in}}%
\pgfpathlineto{\pgfqpoint{1.005602in}{1.532935in}}%
\pgfpathlineto{\pgfqpoint{0.967643in}{1.572413in}}%
\pgfpathlineto{\pgfqpoint{0.928022in}{1.564241in}}%
\pgfpathclose%
\pgfusepath{fill}%
\end{pgfscope}%
\begin{pgfscope}%
\pgfpathrectangle{\pgfqpoint{0.150000in}{0.150000in}}{\pgfqpoint{2.700000in}{1.950000in}}%
\pgfusepath{clip}%
\pgfsetbuttcap%
\pgfsetroundjoin%
\definecolor{currentfill}{rgb}{0.387393,0.462837,0.568459}%
\pgfsetfillcolor{currentfill}%
\pgfsetlinewidth{0.000000pt}%
\definecolor{currentstroke}{rgb}{0.000000,0.000000,0.000000}%
\pgfsetstrokecolor{currentstroke}%
\pgfsetdash{}{0pt}%
\pgfpathmoveto{\pgfqpoint{0.928022in}{1.564241in}}%
\pgfpathlineto{\pgfqpoint{0.967643in}{1.572413in}}%
\pgfpathlineto{\pgfqpoint{0.929678in}{1.611897in}}%
\pgfpathlineto{\pgfqpoint{0.889949in}{1.603834in}}%
\pgfpathclose%
\pgfusepath{fill}%
\end{pgfscope}%
\begin{pgfscope}%
\pgfpathrectangle{\pgfqpoint{0.150000in}{0.150000in}}{\pgfqpoint{2.700000in}{1.950000in}}%
\pgfusepath{clip}%
\pgfsetbuttcap%
\pgfsetroundjoin%
\definecolor{currentfill}{rgb}{0.941115,0.893153,0.896952}%
\pgfsetfillcolor{currentfill}%
\pgfsetlinewidth{0.000000pt}%
\definecolor{currentstroke}{rgb}{0.000000,0.000000,0.000000}%
\pgfsetstrokecolor{currentstroke}%
\pgfsetdash{}{0pt}%
\pgfpathmoveto{\pgfqpoint{1.802998in}{0.604763in}}%
\pgfpathlineto{\pgfqpoint{1.840221in}{0.615625in}}%
\pgfpathlineto{\pgfqpoint{1.802272in}{0.655089in}}%
\pgfpathlineto{\pgfqpoint{1.764941in}{0.644337in}}%
\pgfpathclose%
\pgfusepath{fill}%
\end{pgfscope}%
\begin{pgfscope}%
\pgfpathrectangle{\pgfqpoint{0.150000in}{0.150000in}}{\pgfqpoint{2.700000in}{1.950000in}}%
\pgfusepath{clip}%
\pgfsetbuttcap%
\pgfsetroundjoin%
\definecolor{currentfill}{rgb}{0.960110,0.927619,0.930193}%
\pgfsetfillcolor{currentfill}%
\pgfsetlinewidth{0.000000pt}%
\definecolor{currentstroke}{rgb}{0.000000,0.000000,0.000000}%
\pgfsetstrokecolor{currentstroke}%
\pgfsetdash{}{0pt}%
\pgfpathmoveto{\pgfqpoint{1.764941in}{0.644337in}}%
\pgfpathlineto{\pgfqpoint{1.802272in}{0.655089in}}%
\pgfpathlineto{\pgfqpoint{1.764318in}{0.694558in}}%
\pgfpathlineto{\pgfqpoint{1.726878in}{0.683916in}}%
\pgfpathclose%
\pgfusepath{fill}%
\end{pgfscope}%
\begin{pgfscope}%
\pgfpathrectangle{\pgfqpoint{0.150000in}{0.150000in}}{\pgfqpoint{2.700000in}{1.950000in}}%
\pgfusepath{clip}%
\pgfsetbuttcap%
\pgfsetroundjoin%
\definecolor{currentfill}{rgb}{0.975306,0.955193,0.956786}%
\pgfsetfillcolor{currentfill}%
\pgfsetlinewidth{0.000000pt}%
\definecolor{currentstroke}{rgb}{0.000000,0.000000,0.000000}%
\pgfsetstrokecolor{currentstroke}%
\pgfsetdash{}{0pt}%
\pgfpathmoveto{\pgfqpoint{1.726878in}{0.683916in}}%
\pgfpathlineto{\pgfqpoint{1.764318in}{0.694558in}}%
\pgfpathlineto{\pgfqpoint{1.726359in}{0.734032in}}%
\pgfpathlineto{\pgfqpoint{1.688810in}{0.723500in}}%
\pgfpathclose%
\pgfusepath{fill}%
\end{pgfscope}%
\begin{pgfscope}%
\pgfpathrectangle{\pgfqpoint{0.150000in}{0.150000in}}{\pgfqpoint{2.700000in}{1.950000in}}%
\pgfusepath{clip}%
\pgfsetbuttcap%
\pgfsetroundjoin%
\definecolor{currentfill}{rgb}{0.994301,0.989660,0.990028}%
\pgfsetfillcolor{currentfill}%
\pgfsetlinewidth{0.000000pt}%
\definecolor{currentstroke}{rgb}{0.000000,0.000000,0.000000}%
\pgfsetstrokecolor{currentstroke}%
\pgfsetdash{}{0pt}%
\pgfpathmoveto{\pgfqpoint{1.688810in}{0.723500in}}%
\pgfpathlineto{\pgfqpoint{1.726359in}{0.734032in}}%
\pgfpathlineto{\pgfqpoint{1.688395in}{0.773512in}}%
\pgfpathlineto{\pgfqpoint{1.650737in}{0.763090in}}%
\pgfpathclose%
\pgfusepath{fill}%
\end{pgfscope}%
\begin{pgfscope}%
\pgfpathrectangle{\pgfqpoint{0.150000in}{0.150000in}}{\pgfqpoint{2.700000in}{1.950000in}}%
\pgfusepath{clip}%
\pgfsetbuttcap%
\pgfsetroundjoin%
\definecolor{currentfill}{rgb}{0.978232,0.980913,0.984666}%
\pgfsetfillcolor{currentfill}%
\pgfsetlinewidth{0.000000pt}%
\definecolor{currentstroke}{rgb}{0.000000,0.000000,0.000000}%
\pgfsetstrokecolor{currentstroke}%
\pgfsetdash{}{0pt}%
\pgfpathmoveto{\pgfqpoint{1.650737in}{0.763090in}}%
\pgfpathlineto{\pgfqpoint{1.688395in}{0.773512in}}%
\pgfpathlineto{\pgfqpoint{1.650426in}{0.812997in}}%
\pgfpathlineto{\pgfqpoint{1.612659in}{0.802685in}}%
\pgfpathclose%
\pgfusepath{fill}%
\end{pgfscope}%
\begin{pgfscope}%
\pgfpathrectangle{\pgfqpoint{0.150000in}{0.150000in}}{\pgfqpoint{2.700000in}{1.950000in}}%
\pgfusepath{clip}%
\pgfsetbuttcap%
\pgfsetroundjoin%
\definecolor{currentfill}{rgb}{0.947135,0.953646,0.962760}%
\pgfsetfillcolor{currentfill}%
\pgfsetlinewidth{0.000000pt}%
\definecolor{currentstroke}{rgb}{0.000000,0.000000,0.000000}%
\pgfsetstrokecolor{currentstroke}%
\pgfsetdash{}{0pt}%
\pgfpathmoveto{\pgfqpoint{1.612659in}{0.802685in}}%
\pgfpathlineto{\pgfqpoint{1.650426in}{0.812997in}}%
\pgfpathlineto{\pgfqpoint{1.612451in}{0.852488in}}%
\pgfpathlineto{\pgfqpoint{1.574575in}{0.842285in}}%
\pgfpathclose%
\pgfusepath{fill}%
\end{pgfscope}%
\begin{pgfscope}%
\pgfpathrectangle{\pgfqpoint{0.150000in}{0.150000in}}{\pgfqpoint{2.700000in}{1.950000in}}%
\pgfusepath{clip}%
\pgfsetbuttcap%
\pgfsetroundjoin%
\definecolor{currentfill}{rgb}{0.916039,0.926379,0.940855}%
\pgfsetfillcolor{currentfill}%
\pgfsetlinewidth{0.000000pt}%
\definecolor{currentstroke}{rgb}{0.000000,0.000000,0.000000}%
\pgfsetstrokecolor{currentstroke}%
\pgfsetdash{}{0pt}%
\pgfpathmoveto{\pgfqpoint{1.574575in}{0.842285in}}%
\pgfpathlineto{\pgfqpoint{1.612451in}{0.852488in}}%
\pgfpathlineto{\pgfqpoint{1.574471in}{0.891984in}}%
\pgfpathlineto{\pgfqpoint{1.536486in}{0.881891in}}%
\pgfpathclose%
\pgfusepath{fill}%
\end{pgfscope}%
\begin{pgfscope}%
\pgfpathrectangle{\pgfqpoint{0.150000in}{0.150000in}}{\pgfqpoint{2.700000in}{1.950000in}}%
\pgfusepath{clip}%
\pgfsetbuttcap%
\pgfsetroundjoin%
\definecolor{currentfill}{rgb}{0.884942,0.899112,0.918949}%
\pgfsetfillcolor{currentfill}%
\pgfsetlinewidth{0.000000pt}%
\definecolor{currentstroke}{rgb}{0.000000,0.000000,0.000000}%
\pgfsetstrokecolor{currentstroke}%
\pgfsetdash{}{0pt}%
\pgfpathmoveto{\pgfqpoint{1.536486in}{0.881891in}}%
\pgfpathlineto{\pgfqpoint{1.574471in}{0.891984in}}%
\pgfpathlineto{\pgfqpoint{1.536486in}{0.931485in}}%
\pgfpathlineto{\pgfqpoint{1.498393in}{0.921503in}}%
\pgfpathclose%
\pgfusepath{fill}%
\end{pgfscope}%
\begin{pgfscope}%
\pgfpathrectangle{\pgfqpoint{0.150000in}{0.150000in}}{\pgfqpoint{2.700000in}{1.950000in}}%
\pgfusepath{clip}%
\pgfsetbuttcap%
\pgfsetroundjoin%
\definecolor{currentfill}{rgb}{0.853845,0.871844,0.897044}%
\pgfsetfillcolor{currentfill}%
\pgfsetlinewidth{0.000000pt}%
\definecolor{currentstroke}{rgb}{0.000000,0.000000,0.000000}%
\pgfsetstrokecolor{currentstroke}%
\pgfsetdash{}{0pt}%
\pgfpathmoveto{\pgfqpoint{1.498393in}{0.921503in}}%
\pgfpathlineto{\pgfqpoint{1.536486in}{0.931485in}}%
\pgfpathlineto{\pgfqpoint{1.498496in}{0.970992in}}%
\pgfpathlineto{\pgfqpoint{1.460293in}{0.961119in}}%
\pgfpathclose%
\pgfusepath{fill}%
\end{pgfscope}%
\begin{pgfscope}%
\pgfpathrectangle{\pgfqpoint{0.150000in}{0.150000in}}{\pgfqpoint{2.700000in}{1.950000in}}%
\pgfusepath{clip}%
\pgfsetbuttcap%
\pgfsetroundjoin%
\definecolor{currentfill}{rgb}{0.822748,0.844577,0.875138}%
\pgfsetfillcolor{currentfill}%
\pgfsetlinewidth{0.000000pt}%
\definecolor{currentstroke}{rgb}{0.000000,0.000000,0.000000}%
\pgfsetstrokecolor{currentstroke}%
\pgfsetdash{}{0pt}%
\pgfpathmoveto{\pgfqpoint{1.460293in}{0.961119in}}%
\pgfpathlineto{\pgfqpoint{1.498496in}{0.970992in}}%
\pgfpathlineto{\pgfqpoint{1.460501in}{1.010504in}}%
\pgfpathlineto{\pgfqpoint{1.422189in}{1.000742in}}%
\pgfpathclose%
\pgfusepath{fill}%
\end{pgfscope}%
\begin{pgfscope}%
\pgfpathrectangle{\pgfqpoint{0.150000in}{0.150000in}}{\pgfqpoint{2.700000in}{1.950000in}}%
\pgfusepath{clip}%
\pgfsetbuttcap%
\pgfsetroundjoin%
\definecolor{currentfill}{rgb}{0.797871,0.822763,0.857613}%
\pgfsetfillcolor{currentfill}%
\pgfsetlinewidth{0.000000pt}%
\definecolor{currentstroke}{rgb}{0.000000,0.000000,0.000000}%
\pgfsetstrokecolor{currentstroke}%
\pgfsetdash{}{0pt}%
\pgfpathmoveto{\pgfqpoint{1.422189in}{1.000742in}}%
\pgfpathlineto{\pgfqpoint{1.460501in}{1.010504in}}%
\pgfpathlineto{\pgfqpoint{1.422501in}{1.050021in}}%
\pgfpathlineto{\pgfqpoint{1.384079in}{1.040369in}}%
\pgfpathclose%
\pgfusepath{fill}%
\end{pgfscope}%
\begin{pgfscope}%
\pgfpathrectangle{\pgfqpoint{0.150000in}{0.150000in}}{\pgfqpoint{2.700000in}{1.950000in}}%
\pgfusepath{clip}%
\pgfsetbuttcap%
\pgfsetroundjoin%
\definecolor{currentfill}{rgb}{0.766774,0.795496,0.835708}%
\pgfsetfillcolor{currentfill}%
\pgfsetlinewidth{0.000000pt}%
\definecolor{currentstroke}{rgb}{0.000000,0.000000,0.000000}%
\pgfsetstrokecolor{currentstroke}%
\pgfsetdash{}{0pt}%
\pgfpathmoveto{\pgfqpoint{1.384079in}{1.040369in}}%
\pgfpathlineto{\pgfqpoint{1.422501in}{1.050021in}}%
\pgfpathlineto{\pgfqpoint{1.384495in}{1.089544in}}%
\pgfpathlineto{\pgfqpoint{1.345965in}{1.080002in}}%
\pgfpathclose%
\pgfusepath{fill}%
\end{pgfscope}%
\begin{pgfscope}%
\pgfpathrectangle{\pgfqpoint{0.150000in}{0.150000in}}{\pgfqpoint{2.700000in}{1.950000in}}%
\pgfusepath{clip}%
\pgfsetbuttcap%
\pgfsetroundjoin%
\definecolor{currentfill}{rgb}{0.735677,0.768229,0.813802}%
\pgfsetfillcolor{currentfill}%
\pgfsetlinewidth{0.000000pt}%
\definecolor{currentstroke}{rgb}{0.000000,0.000000,0.000000}%
\pgfsetstrokecolor{currentstroke}%
\pgfsetdash{}{0pt}%
\pgfpathmoveto{\pgfqpoint{1.345965in}{1.080002in}}%
\pgfpathlineto{\pgfqpoint{1.384495in}{1.089544in}}%
\pgfpathlineto{\pgfqpoint{1.346484in}{1.129073in}}%
\pgfpathlineto{\pgfqpoint{1.307844in}{1.119641in}}%
\pgfpathclose%
\pgfusepath{fill}%
\end{pgfscope}%
\begin{pgfscope}%
\pgfpathrectangle{\pgfqpoint{0.150000in}{0.150000in}}{\pgfqpoint{2.700000in}{1.950000in}}%
\pgfusepath{clip}%
\pgfsetbuttcap%
\pgfsetroundjoin%
\definecolor{currentfill}{rgb}{0.704580,0.740962,0.791896}%
\pgfsetfillcolor{currentfill}%
\pgfsetlinewidth{0.000000pt}%
\definecolor{currentstroke}{rgb}{0.000000,0.000000,0.000000}%
\pgfsetstrokecolor{currentstroke}%
\pgfsetdash{}{0pt}%
\pgfpathmoveto{\pgfqpoint{1.307844in}{1.119641in}}%
\pgfpathlineto{\pgfqpoint{1.346484in}{1.129073in}}%
\pgfpathlineto{\pgfqpoint{1.308468in}{1.168607in}}%
\pgfpathlineto{\pgfqpoint{1.269719in}{1.159285in}}%
\pgfpathclose%
\pgfusepath{fill}%
\end{pgfscope}%
\begin{pgfscope}%
\pgfpathrectangle{\pgfqpoint{0.150000in}{0.150000in}}{\pgfqpoint{2.700000in}{1.950000in}}%
\pgfusepath{clip}%
\pgfsetbuttcap%
\pgfsetroundjoin%
\definecolor{currentfill}{rgb}{0.673483,0.713695,0.769991}%
\pgfsetfillcolor{currentfill}%
\pgfsetlinewidth{0.000000pt}%
\definecolor{currentstroke}{rgb}{0.000000,0.000000,0.000000}%
\pgfsetstrokecolor{currentstroke}%
\pgfsetdash{}{0pt}%
\pgfpathmoveto{\pgfqpoint{1.269719in}{1.159285in}}%
\pgfpathlineto{\pgfqpoint{1.308468in}{1.168607in}}%
\pgfpathlineto{\pgfqpoint{1.270447in}{1.208146in}}%
\pgfpathlineto{\pgfqpoint{1.231589in}{1.198934in}}%
\pgfpathclose%
\pgfusepath{fill}%
\end{pgfscope}%
\begin{pgfscope}%
\pgfpathrectangle{\pgfqpoint{0.150000in}{0.150000in}}{\pgfqpoint{2.700000in}{1.950000in}}%
\pgfusepath{clip}%
\pgfsetbuttcap%
\pgfsetroundjoin%
\definecolor{currentfill}{rgb}{0.642387,0.686428,0.748085}%
\pgfsetfillcolor{currentfill}%
\pgfsetlinewidth{0.000000pt}%
\definecolor{currentstroke}{rgb}{0.000000,0.000000,0.000000}%
\pgfsetstrokecolor{currentstroke}%
\pgfsetdash{}{0pt}%
\pgfpathmoveto{\pgfqpoint{1.231589in}{1.198934in}}%
\pgfpathlineto{\pgfqpoint{1.270447in}{1.208146in}}%
\pgfpathlineto{\pgfqpoint{1.232421in}{1.247690in}}%
\pgfpathlineto{\pgfqpoint{1.193453in}{1.238589in}}%
\pgfpathclose%
\pgfusepath{fill}%
\end{pgfscope}%
\begin{pgfscope}%
\pgfpathrectangle{\pgfqpoint{0.150000in}{0.150000in}}{\pgfqpoint{2.700000in}{1.950000in}}%
\pgfusepath{clip}%
\pgfsetbuttcap%
\pgfsetroundjoin%
\definecolor{currentfill}{rgb}{0.611290,0.659161,0.726180}%
\pgfsetfillcolor{currentfill}%
\pgfsetlinewidth{0.000000pt}%
\definecolor{currentstroke}{rgb}{0.000000,0.000000,0.000000}%
\pgfsetstrokecolor{currentstroke}%
\pgfsetdash{}{0pt}%
\pgfpathmoveto{\pgfqpoint{1.193453in}{1.238589in}}%
\pgfpathlineto{\pgfqpoint{1.232421in}{1.247690in}}%
\pgfpathlineto{\pgfqpoint{1.194389in}{1.287240in}}%
\pgfpathlineto{\pgfqpoint{1.155312in}{1.278249in}}%
\pgfpathclose%
\pgfusepath{fill}%
\end{pgfscope}%
\begin{pgfscope}%
\pgfpathrectangle{\pgfqpoint{0.150000in}{0.150000in}}{\pgfqpoint{2.700000in}{1.950000in}}%
\pgfusepath{clip}%
\pgfsetbuttcap%
\pgfsetroundjoin%
\definecolor{currentfill}{rgb}{0.580193,0.631893,0.704274}%
\pgfsetfillcolor{currentfill}%
\pgfsetlinewidth{0.000000pt}%
\definecolor{currentstroke}{rgb}{0.000000,0.000000,0.000000}%
\pgfsetstrokecolor{currentstroke}%
\pgfsetdash{}{0pt}%
\pgfpathmoveto{\pgfqpoint{1.155312in}{1.278249in}}%
\pgfpathlineto{\pgfqpoint{1.194389in}{1.287240in}}%
\pgfpathlineto{\pgfqpoint{1.156352in}{1.326795in}}%
\pgfpathlineto{\pgfqpoint{1.117166in}{1.317915in}}%
\pgfpathclose%
\pgfusepath{fill}%
\end{pgfscope}%
\begin{pgfscope}%
\pgfpathrectangle{\pgfqpoint{0.150000in}{0.150000in}}{\pgfqpoint{2.700000in}{1.950000in}}%
\pgfusepath{clip}%
\pgfsetbuttcap%
\pgfsetroundjoin%
\definecolor{currentfill}{rgb}{0.555316,0.610080,0.686749}%
\pgfsetfillcolor{currentfill}%
\pgfsetlinewidth{0.000000pt}%
\definecolor{currentstroke}{rgb}{0.000000,0.000000,0.000000}%
\pgfsetstrokecolor{currentstroke}%
\pgfsetdash{}{0pt}%
\pgfpathmoveto{\pgfqpoint{1.117166in}{1.317915in}}%
\pgfpathlineto{\pgfqpoint{1.156352in}{1.326795in}}%
\pgfpathlineto{\pgfqpoint{1.118310in}{1.366356in}}%
\pgfpathlineto{\pgfqpoint{1.079015in}{1.357586in}}%
\pgfpathclose%
\pgfusepath{fill}%
\end{pgfscope}%
\begin{pgfscope}%
\pgfpathrectangle{\pgfqpoint{0.150000in}{0.150000in}}{\pgfqpoint{2.700000in}{1.950000in}}%
\pgfusepath{clip}%
\pgfsetbuttcap%
\pgfsetroundjoin%
\definecolor{currentfill}{rgb}{0.524219,0.582812,0.664844}%
\pgfsetfillcolor{currentfill}%
\pgfsetlinewidth{0.000000pt}%
\definecolor{currentstroke}{rgb}{0.000000,0.000000,0.000000}%
\pgfsetstrokecolor{currentstroke}%
\pgfsetdash{}{0pt}%
\pgfpathmoveto{\pgfqpoint{1.079015in}{1.357586in}}%
\pgfpathlineto{\pgfqpoint{1.118310in}{1.366356in}}%
\pgfpathlineto{\pgfqpoint{1.080263in}{1.405922in}}%
\pgfpathlineto{\pgfqpoint{1.040858in}{1.397262in}}%
\pgfpathclose%
\pgfusepath{fill}%
\end{pgfscope}%
\begin{pgfscope}%
\pgfpathrectangle{\pgfqpoint{0.150000in}{0.150000in}}{\pgfqpoint{2.700000in}{1.950000in}}%
\pgfusepath{clip}%
\pgfsetbuttcap%
\pgfsetroundjoin%
\definecolor{currentfill}{rgb}{0.493122,0.555545,0.642938}%
\pgfsetfillcolor{currentfill}%
\pgfsetlinewidth{0.000000pt}%
\definecolor{currentstroke}{rgb}{0.000000,0.000000,0.000000}%
\pgfsetstrokecolor{currentstroke}%
\pgfsetdash{}{0pt}%
\pgfpathmoveto{\pgfqpoint{1.040858in}{1.397262in}}%
\pgfpathlineto{\pgfqpoint{1.080263in}{1.405922in}}%
\pgfpathlineto{\pgfqpoint{1.042210in}{1.445494in}}%
\pgfpathlineto{\pgfqpoint{1.002696in}{1.436944in}}%
\pgfpathclose%
\pgfusepath{fill}%
\end{pgfscope}%
\begin{pgfscope}%
\pgfpathrectangle{\pgfqpoint{0.150000in}{0.150000in}}{\pgfqpoint{2.700000in}{1.950000in}}%
\pgfusepath{clip}%
\pgfsetbuttcap%
\pgfsetroundjoin%
\definecolor{currentfill}{rgb}{0.462025,0.528278,0.621032}%
\pgfsetfillcolor{currentfill}%
\pgfsetlinewidth{0.000000pt}%
\definecolor{currentstroke}{rgb}{0.000000,0.000000,0.000000}%
\pgfsetstrokecolor{currentstroke}%
\pgfsetdash{}{0pt}%
\pgfpathmoveto{\pgfqpoint{1.002696in}{1.436944in}}%
\pgfpathlineto{\pgfqpoint{1.042210in}{1.445494in}}%
\pgfpathlineto{\pgfqpoint{1.004153in}{1.485071in}}%
\pgfpathlineto{\pgfqpoint{0.964529in}{1.476632in}}%
\pgfpathclose%
\pgfusepath{fill}%
\end{pgfscope}%
\begin{pgfscope}%
\pgfpathrectangle{\pgfqpoint{0.150000in}{0.150000in}}{\pgfqpoint{2.700000in}{1.950000in}}%
\pgfusepath{clip}%
\pgfsetbuttcap%
\pgfsetroundjoin%
\definecolor{currentfill}{rgb}{0.430928,0.501011,0.599127}%
\pgfsetfillcolor{currentfill}%
\pgfsetlinewidth{0.000000pt}%
\definecolor{currentstroke}{rgb}{0.000000,0.000000,0.000000}%
\pgfsetstrokecolor{currentstroke}%
\pgfsetdash{}{0pt}%
\pgfpathmoveto{\pgfqpoint{0.964529in}{1.476632in}}%
\pgfpathlineto{\pgfqpoint{1.004153in}{1.485071in}}%
\pgfpathlineto{\pgfqpoint{0.966090in}{1.524653in}}%
\pgfpathlineto{\pgfqpoint{0.926357in}{1.516325in}}%
\pgfpathclose%
\pgfusepath{fill}%
\end{pgfscope}%
\begin{pgfscope}%
\pgfpathrectangle{\pgfqpoint{0.150000in}{0.150000in}}{\pgfqpoint{2.700000in}{1.950000in}}%
\pgfusepath{clip}%
\pgfsetbuttcap%
\pgfsetroundjoin%
\definecolor{currentfill}{rgb}{0.399831,0.473744,0.577221}%
\pgfsetfillcolor{currentfill}%
\pgfsetlinewidth{0.000000pt}%
\definecolor{currentstroke}{rgb}{0.000000,0.000000,0.000000}%
\pgfsetstrokecolor{currentstroke}%
\pgfsetdash{}{0pt}%
\pgfpathmoveto{\pgfqpoint{0.926357in}{1.516325in}}%
\pgfpathlineto{\pgfqpoint{0.966090in}{1.524653in}}%
\pgfpathlineto{\pgfqpoint{0.928022in}{1.564241in}}%
\pgfpathlineto{\pgfqpoint{0.888179in}{1.556023in}}%
\pgfpathclose%
\pgfusepath{fill}%
\end{pgfscope}%
\begin{pgfscope}%
\pgfpathrectangle{\pgfqpoint{0.150000in}{0.150000in}}{\pgfqpoint{2.700000in}{1.950000in}}%
\pgfusepath{clip}%
\pgfsetbuttcap%
\pgfsetroundjoin%
\definecolor{currentfill}{rgb}{0.368735,0.446477,0.555316}%
\pgfsetfillcolor{currentfill}%
\pgfsetlinewidth{0.000000pt}%
\definecolor{currentstroke}{rgb}{0.000000,0.000000,0.000000}%
\pgfsetstrokecolor{currentstroke}%
\pgfsetdash{}{0pt}%
\pgfpathmoveto{\pgfqpoint{0.888179in}{1.556023in}}%
\pgfpathlineto{\pgfqpoint{0.928022in}{1.564241in}}%
\pgfpathlineto{\pgfqpoint{0.889949in}{1.603834in}}%
\pgfpathlineto{\pgfqpoint{0.849996in}{1.595727in}}%
\pgfpathclose%
\pgfusepath{fill}%
\end{pgfscope}%
\begin{pgfscope}%
\pgfpathrectangle{\pgfqpoint{0.150000in}{0.150000in}}{\pgfqpoint{2.700000in}{1.950000in}}%
\pgfusepath{clip}%
\pgfsetbuttcap%
\pgfsetroundjoin%
\definecolor{currentfill}{rgb}{0.948713,0.906939,0.910248}%
\pgfsetfillcolor{currentfill}%
\pgfsetlinewidth{0.000000pt}%
\definecolor{currentstroke}{rgb}{0.000000,0.000000,0.000000}%
\pgfsetstrokecolor{currentstroke}%
\pgfsetdash{}{0pt}%
\pgfpathmoveto{\pgfqpoint{1.765567in}{0.593840in}}%
\pgfpathlineto{\pgfqpoint{1.802998in}{0.604763in}}%
\pgfpathlineto{\pgfqpoint{1.764941in}{0.644337in}}%
\pgfpathlineto{\pgfqpoint{1.727400in}{0.633524in}}%
\pgfpathclose%
\pgfusepath{fill}%
\end{pgfscope}%
\begin{pgfscope}%
\pgfpathrectangle{\pgfqpoint{0.150000in}{0.150000in}}{\pgfqpoint{2.700000in}{1.950000in}}%
\pgfusepath{clip}%
\pgfsetbuttcap%
\pgfsetroundjoin%
\definecolor{currentfill}{rgb}{0.967708,0.941406,0.943490}%
\pgfsetfillcolor{currentfill}%
\pgfsetlinewidth{0.000000pt}%
\definecolor{currentstroke}{rgb}{0.000000,0.000000,0.000000}%
\pgfsetstrokecolor{currentstroke}%
\pgfsetdash{}{0pt}%
\pgfpathmoveto{\pgfqpoint{1.727400in}{0.633524in}}%
\pgfpathlineto{\pgfqpoint{1.764941in}{0.644337in}}%
\pgfpathlineto{\pgfqpoint{1.726878in}{0.683916in}}%
\pgfpathlineto{\pgfqpoint{1.689228in}{0.673214in}}%
\pgfpathclose%
\pgfusepath{fill}%
\end{pgfscope}%
\begin{pgfscope}%
\pgfpathrectangle{\pgfqpoint{0.150000in}{0.150000in}}{\pgfqpoint{2.700000in}{1.950000in}}%
\pgfusepath{clip}%
\pgfsetbuttcap%
\pgfsetroundjoin%
\definecolor{currentfill}{rgb}{0.986703,0.975873,0.976731}%
\pgfsetfillcolor{currentfill}%
\pgfsetlinewidth{0.000000pt}%
\definecolor{currentstroke}{rgb}{0.000000,0.000000,0.000000}%
\pgfsetstrokecolor{currentstroke}%
\pgfsetdash{}{0pt}%
\pgfpathmoveto{\pgfqpoint{1.689228in}{0.673214in}}%
\pgfpathlineto{\pgfqpoint{1.726878in}{0.683916in}}%
\pgfpathlineto{\pgfqpoint{1.688810in}{0.723500in}}%
\pgfpathlineto{\pgfqpoint{1.651050in}{0.712908in}}%
\pgfpathclose%
\pgfusepath{fill}%
\end{pgfscope}%
\begin{pgfscope}%
\pgfpathrectangle{\pgfqpoint{0.150000in}{0.150000in}}{\pgfqpoint{2.700000in}{1.950000in}}%
\pgfusepath{clip}%
\pgfsetbuttcap%
\pgfsetroundjoin%
\definecolor{currentfill}{rgb}{0.990671,0.991820,0.993428}%
\pgfsetfillcolor{currentfill}%
\pgfsetlinewidth{0.000000pt}%
\definecolor{currentstroke}{rgb}{0.000000,0.000000,0.000000}%
\pgfsetstrokecolor{currentstroke}%
\pgfsetdash{}{0pt}%
\pgfpathmoveto{\pgfqpoint{1.651050in}{0.712908in}}%
\pgfpathlineto{\pgfqpoint{1.688810in}{0.723500in}}%
\pgfpathlineto{\pgfqpoint{1.650737in}{0.763090in}}%
\pgfpathlineto{\pgfqpoint{1.612868in}{0.752609in}}%
\pgfpathclose%
\pgfusepath{fill}%
\end{pgfscope}%
\begin{pgfscope}%
\pgfpathrectangle{\pgfqpoint{0.150000in}{0.150000in}}{\pgfqpoint{2.700000in}{1.950000in}}%
\pgfusepath{clip}%
\pgfsetbuttcap%
\pgfsetroundjoin%
\definecolor{currentfill}{rgb}{0.959574,0.964553,0.971523}%
\pgfsetfillcolor{currentfill}%
\pgfsetlinewidth{0.000000pt}%
\definecolor{currentstroke}{rgb}{0.000000,0.000000,0.000000}%
\pgfsetstrokecolor{currentstroke}%
\pgfsetdash{}{0pt}%
\pgfpathmoveto{\pgfqpoint{1.612868in}{0.752609in}}%
\pgfpathlineto{\pgfqpoint{1.650737in}{0.763090in}}%
\pgfpathlineto{\pgfqpoint{1.612659in}{0.802685in}}%
\pgfpathlineto{\pgfqpoint{1.574680in}{0.792314in}}%
\pgfpathclose%
\pgfusepath{fill}%
\end{pgfscope}%
\begin{pgfscope}%
\pgfpathrectangle{\pgfqpoint{0.150000in}{0.150000in}}{\pgfqpoint{2.700000in}{1.950000in}}%
\pgfusepath{clip}%
\pgfsetbuttcap%
\pgfsetroundjoin%
\definecolor{currentfill}{rgb}{0.934697,0.942739,0.953998}%
\pgfsetfillcolor{currentfill}%
\pgfsetlinewidth{0.000000pt}%
\definecolor{currentstroke}{rgb}{0.000000,0.000000,0.000000}%
\pgfsetstrokecolor{currentstroke}%
\pgfsetdash{}{0pt}%
\pgfpathmoveto{\pgfqpoint{1.574680in}{0.792314in}}%
\pgfpathlineto{\pgfqpoint{1.612659in}{0.802685in}}%
\pgfpathlineto{\pgfqpoint{1.574575in}{0.842285in}}%
\pgfpathlineto{\pgfqpoint{1.536486in}{0.832025in}}%
\pgfpathclose%
\pgfusepath{fill}%
\end{pgfscope}%
\begin{pgfscope}%
\pgfpathrectangle{\pgfqpoint{0.150000in}{0.150000in}}{\pgfqpoint{2.700000in}{1.950000in}}%
\pgfusepath{clip}%
\pgfsetbuttcap%
\pgfsetroundjoin%
\definecolor{currentfill}{rgb}{0.903600,0.915472,0.932093}%
\pgfsetfillcolor{currentfill}%
\pgfsetlinewidth{0.000000pt}%
\definecolor{currentstroke}{rgb}{0.000000,0.000000,0.000000}%
\pgfsetstrokecolor{currentstroke}%
\pgfsetdash{}{0pt}%
\pgfpathmoveto{\pgfqpoint{1.536486in}{0.832025in}}%
\pgfpathlineto{\pgfqpoint{1.574575in}{0.842285in}}%
\pgfpathlineto{\pgfqpoint{1.536486in}{0.881891in}}%
\pgfpathlineto{\pgfqpoint{1.498288in}{0.871742in}}%
\pgfpathclose%
\pgfusepath{fill}%
\end{pgfscope}%
\begin{pgfscope}%
\pgfpathrectangle{\pgfqpoint{0.150000in}{0.150000in}}{\pgfqpoint{2.700000in}{1.950000in}}%
\pgfusepath{clip}%
\pgfsetbuttcap%
\pgfsetroundjoin%
\definecolor{currentfill}{rgb}{0.872503,0.888205,0.910187}%
\pgfsetfillcolor{currentfill}%
\pgfsetlinewidth{0.000000pt}%
\definecolor{currentstroke}{rgb}{0.000000,0.000000,0.000000}%
\pgfsetstrokecolor{currentstroke}%
\pgfsetdash{}{0pt}%
\pgfpathmoveto{\pgfqpoint{1.498288in}{0.871742in}}%
\pgfpathlineto{\pgfqpoint{1.536486in}{0.881891in}}%
\pgfpathlineto{\pgfqpoint{1.498393in}{0.921503in}}%
\pgfpathlineto{\pgfqpoint{1.460084in}{0.911464in}}%
\pgfpathclose%
\pgfusepath{fill}%
\end{pgfscope}%
\begin{pgfscope}%
\pgfpathrectangle{\pgfqpoint{0.150000in}{0.150000in}}{\pgfqpoint{2.700000in}{1.950000in}}%
\pgfusepath{clip}%
\pgfsetbuttcap%
\pgfsetroundjoin%
\definecolor{currentfill}{rgb}{0.841406,0.860938,0.888281}%
\pgfsetfillcolor{currentfill}%
\pgfsetlinewidth{0.000000pt}%
\definecolor{currentstroke}{rgb}{0.000000,0.000000,0.000000}%
\pgfsetstrokecolor{currentstroke}%
\pgfsetdash{}{0pt}%
\pgfpathmoveto{\pgfqpoint{1.460084in}{0.911464in}}%
\pgfpathlineto{\pgfqpoint{1.498393in}{0.921503in}}%
\pgfpathlineto{\pgfqpoint{1.460293in}{0.961119in}}%
\pgfpathlineto{\pgfqpoint{1.421876in}{0.951191in}}%
\pgfpathclose%
\pgfusepath{fill}%
\end{pgfscope}%
\begin{pgfscope}%
\pgfpathrectangle{\pgfqpoint{0.150000in}{0.150000in}}{\pgfqpoint{2.700000in}{1.950000in}}%
\pgfusepath{clip}%
\pgfsetbuttcap%
\pgfsetroundjoin%
\definecolor{currentfill}{rgb}{0.810309,0.833670,0.866376}%
\pgfsetfillcolor{currentfill}%
\pgfsetlinewidth{0.000000pt}%
\definecolor{currentstroke}{rgb}{0.000000,0.000000,0.000000}%
\pgfsetstrokecolor{currentstroke}%
\pgfsetdash{}{0pt}%
\pgfpathmoveto{\pgfqpoint{1.421876in}{0.951191in}}%
\pgfpathlineto{\pgfqpoint{1.460293in}{0.961119in}}%
\pgfpathlineto{\pgfqpoint{1.422189in}{1.000742in}}%
\pgfpathlineto{\pgfqpoint{1.383661in}{0.990924in}}%
\pgfpathclose%
\pgfusepath{fill}%
\end{pgfscope}%
\begin{pgfscope}%
\pgfpathrectangle{\pgfqpoint{0.150000in}{0.150000in}}{\pgfqpoint{2.700000in}{1.950000in}}%
\pgfusepath{clip}%
\pgfsetbuttcap%
\pgfsetroundjoin%
\definecolor{currentfill}{rgb}{0.779213,0.806403,0.844470}%
\pgfsetfillcolor{currentfill}%
\pgfsetlinewidth{0.000000pt}%
\definecolor{currentstroke}{rgb}{0.000000,0.000000,0.000000}%
\pgfsetstrokecolor{currentstroke}%
\pgfsetdash{}{0pt}%
\pgfpathmoveto{\pgfqpoint{1.383661in}{0.990924in}}%
\pgfpathlineto{\pgfqpoint{1.422189in}{1.000742in}}%
\pgfpathlineto{\pgfqpoint{1.384079in}{1.040369in}}%
\pgfpathlineto{\pgfqpoint{1.345442in}{1.030663in}}%
\pgfpathclose%
\pgfusepath{fill}%
\end{pgfscope}%
\begin{pgfscope}%
\pgfpathrectangle{\pgfqpoint{0.150000in}{0.150000in}}{\pgfqpoint{2.700000in}{1.950000in}}%
\pgfusepath{clip}%
\pgfsetbuttcap%
\pgfsetroundjoin%
\definecolor{currentfill}{rgb}{0.748116,0.779136,0.822564}%
\pgfsetfillcolor{currentfill}%
\pgfsetlinewidth{0.000000pt}%
\definecolor{currentstroke}{rgb}{0.000000,0.000000,0.000000}%
\pgfsetstrokecolor{currentstroke}%
\pgfsetdash{}{0pt}%
\pgfpathmoveto{\pgfqpoint{1.345442in}{1.030663in}}%
\pgfpathlineto{\pgfqpoint{1.384079in}{1.040369in}}%
\pgfpathlineto{\pgfqpoint{1.345965in}{1.080002in}}%
\pgfpathlineto{\pgfqpoint{1.307217in}{1.070407in}}%
\pgfpathclose%
\pgfusepath{fill}%
\end{pgfscope}%
\begin{pgfscope}%
\pgfpathrectangle{\pgfqpoint{0.150000in}{0.150000in}}{\pgfqpoint{2.700000in}{1.950000in}}%
\pgfusepath{clip}%
\pgfsetbuttcap%
\pgfsetroundjoin%
\definecolor{currentfill}{rgb}{0.717019,0.751869,0.800659}%
\pgfsetfillcolor{currentfill}%
\pgfsetlinewidth{0.000000pt}%
\definecolor{currentstroke}{rgb}{0.000000,0.000000,0.000000}%
\pgfsetstrokecolor{currentstroke}%
\pgfsetdash{}{0pt}%
\pgfpathmoveto{\pgfqpoint{1.307217in}{1.070407in}}%
\pgfpathlineto{\pgfqpoint{1.345965in}{1.080002in}}%
\pgfpathlineto{\pgfqpoint{1.307844in}{1.119641in}}%
\pgfpathlineto{\pgfqpoint{1.268988in}{1.110156in}}%
\pgfpathclose%
\pgfusepath{fill}%
\end{pgfscope}%
\begin{pgfscope}%
\pgfpathrectangle{\pgfqpoint{0.150000in}{0.150000in}}{\pgfqpoint{2.700000in}{1.950000in}}%
\pgfusepath{clip}%
\pgfsetbuttcap%
\pgfsetroundjoin%
\definecolor{currentfill}{rgb}{0.692142,0.730055,0.783134}%
\pgfsetfillcolor{currentfill}%
\pgfsetlinewidth{0.000000pt}%
\definecolor{currentstroke}{rgb}{0.000000,0.000000,0.000000}%
\pgfsetstrokecolor{currentstroke}%
\pgfsetdash{}{0pt}%
\pgfpathmoveto{\pgfqpoint{1.268988in}{1.110156in}}%
\pgfpathlineto{\pgfqpoint{1.307844in}{1.119641in}}%
\pgfpathlineto{\pgfqpoint{1.269719in}{1.159285in}}%
\pgfpathlineto{\pgfqpoint{1.230752in}{1.149911in}}%
\pgfpathclose%
\pgfusepath{fill}%
\end{pgfscope}%
\begin{pgfscope}%
\pgfpathrectangle{\pgfqpoint{0.150000in}{0.150000in}}{\pgfqpoint{2.700000in}{1.950000in}}%
\pgfusepath{clip}%
\pgfsetbuttcap%
\pgfsetroundjoin%
\definecolor{currentfill}{rgb}{0.661045,0.702788,0.761229}%
\pgfsetfillcolor{currentfill}%
\pgfsetlinewidth{0.000000pt}%
\definecolor{currentstroke}{rgb}{0.000000,0.000000,0.000000}%
\pgfsetstrokecolor{currentstroke}%
\pgfsetdash{}{0pt}%
\pgfpathmoveto{\pgfqpoint{1.230752in}{1.149911in}}%
\pgfpathlineto{\pgfqpoint{1.269719in}{1.159285in}}%
\pgfpathlineto{\pgfqpoint{1.231589in}{1.198934in}}%
\pgfpathlineto{\pgfqpoint{1.192512in}{1.189671in}}%
\pgfpathclose%
\pgfusepath{fill}%
\end{pgfscope}%
\begin{pgfscope}%
\pgfpathrectangle{\pgfqpoint{0.150000in}{0.150000in}}{\pgfqpoint{2.700000in}{1.950000in}}%
\pgfusepath{clip}%
\pgfsetbuttcap%
\pgfsetroundjoin%
\definecolor{currentfill}{rgb}{0.629948,0.675521,0.739323}%
\pgfsetfillcolor{currentfill}%
\pgfsetlinewidth{0.000000pt}%
\definecolor{currentstroke}{rgb}{0.000000,0.000000,0.000000}%
\pgfsetstrokecolor{currentstroke}%
\pgfsetdash{}{0pt}%
\pgfpathmoveto{\pgfqpoint{1.192512in}{1.189671in}}%
\pgfpathlineto{\pgfqpoint{1.231589in}{1.198934in}}%
\pgfpathlineto{\pgfqpoint{1.193453in}{1.238589in}}%
\pgfpathlineto{\pgfqpoint{1.154266in}{1.229436in}}%
\pgfpathclose%
\pgfusepath{fill}%
\end{pgfscope}%
\begin{pgfscope}%
\pgfpathrectangle{\pgfqpoint{0.150000in}{0.150000in}}{\pgfqpoint{2.700000in}{1.950000in}}%
\pgfusepath{clip}%
\pgfsetbuttcap%
\pgfsetroundjoin%
\definecolor{currentfill}{rgb}{0.598851,0.648254,0.717417}%
\pgfsetfillcolor{currentfill}%
\pgfsetlinewidth{0.000000pt}%
\definecolor{currentstroke}{rgb}{0.000000,0.000000,0.000000}%
\pgfsetstrokecolor{currentstroke}%
\pgfsetdash{}{0pt}%
\pgfpathmoveto{\pgfqpoint{1.154266in}{1.229436in}}%
\pgfpathlineto{\pgfqpoint{1.193453in}{1.238589in}}%
\pgfpathlineto{\pgfqpoint{1.155312in}{1.278249in}}%
\pgfpathlineto{\pgfqpoint{1.116015in}{1.269208in}}%
\pgfpathclose%
\pgfusepath{fill}%
\end{pgfscope}%
\begin{pgfscope}%
\pgfpathrectangle{\pgfqpoint{0.150000in}{0.150000in}}{\pgfqpoint{2.700000in}{1.950000in}}%
\pgfusepath{clip}%
\pgfsetbuttcap%
\pgfsetroundjoin%
\definecolor{currentfill}{rgb}{0.567754,0.620987,0.695512}%
\pgfsetfillcolor{currentfill}%
\pgfsetlinewidth{0.000000pt}%
\definecolor{currentstroke}{rgb}{0.000000,0.000000,0.000000}%
\pgfsetstrokecolor{currentstroke}%
\pgfsetdash{}{0pt}%
\pgfpathmoveto{\pgfqpoint{1.116015in}{1.269208in}}%
\pgfpathlineto{\pgfqpoint{1.155312in}{1.278249in}}%
\pgfpathlineto{\pgfqpoint{1.117166in}{1.317915in}}%
\pgfpathlineto{\pgfqpoint{1.077759in}{1.308984in}}%
\pgfpathclose%
\pgfusepath{fill}%
\end{pgfscope}%
\begin{pgfscope}%
\pgfpathrectangle{\pgfqpoint{0.150000in}{0.150000in}}{\pgfqpoint{2.700000in}{1.950000in}}%
\pgfusepath{clip}%
\pgfsetbuttcap%
\pgfsetroundjoin%
\definecolor{currentfill}{rgb}{0.536657,0.593719,0.673606}%
\pgfsetfillcolor{currentfill}%
\pgfsetlinewidth{0.000000pt}%
\definecolor{currentstroke}{rgb}{0.000000,0.000000,0.000000}%
\pgfsetstrokecolor{currentstroke}%
\pgfsetdash{}{0pt}%
\pgfpathmoveto{\pgfqpoint{1.077759in}{1.308984in}}%
\pgfpathlineto{\pgfqpoint{1.117166in}{1.317915in}}%
\pgfpathlineto{\pgfqpoint{1.079015in}{1.357586in}}%
\pgfpathlineto{\pgfqpoint{1.039498in}{1.348766in}}%
\pgfpathclose%
\pgfusepath{fill}%
\end{pgfscope}%
\begin{pgfscope}%
\pgfpathrectangle{\pgfqpoint{0.150000in}{0.150000in}}{\pgfqpoint{2.700000in}{1.950000in}}%
\pgfusepath{clip}%
\pgfsetbuttcap%
\pgfsetroundjoin%
\definecolor{currentfill}{rgb}{0.505561,0.566452,0.651700}%
\pgfsetfillcolor{currentfill}%
\pgfsetlinewidth{0.000000pt}%
\definecolor{currentstroke}{rgb}{0.000000,0.000000,0.000000}%
\pgfsetstrokecolor{currentstroke}%
\pgfsetdash{}{0pt}%
\pgfpathmoveto{\pgfqpoint{1.039498in}{1.348766in}}%
\pgfpathlineto{\pgfqpoint{1.079015in}{1.357586in}}%
\pgfpathlineto{\pgfqpoint{1.040858in}{1.397262in}}%
\pgfpathlineto{\pgfqpoint{1.001231in}{1.388554in}}%
\pgfpathclose%
\pgfusepath{fill}%
\end{pgfscope}%
\begin{pgfscope}%
\pgfpathrectangle{\pgfqpoint{0.150000in}{0.150000in}}{\pgfqpoint{2.700000in}{1.950000in}}%
\pgfusepath{clip}%
\pgfsetbuttcap%
\pgfsetroundjoin%
\definecolor{currentfill}{rgb}{0.474464,0.539185,0.629795}%
\pgfsetfillcolor{currentfill}%
\pgfsetlinewidth{0.000000pt}%
\definecolor{currentstroke}{rgb}{0.000000,0.000000,0.000000}%
\pgfsetstrokecolor{currentstroke}%
\pgfsetdash{}{0pt}%
\pgfpathmoveto{\pgfqpoint{1.001231in}{1.388554in}}%
\pgfpathlineto{\pgfqpoint{1.040858in}{1.397262in}}%
\pgfpathlineto{\pgfqpoint{1.002696in}{1.436944in}}%
\pgfpathlineto{\pgfqpoint{0.962959in}{1.428347in}}%
\pgfpathclose%
\pgfusepath{fill}%
\end{pgfscope}%
\begin{pgfscope}%
\pgfpathrectangle{\pgfqpoint{0.150000in}{0.150000in}}{\pgfqpoint{2.700000in}{1.950000in}}%
\pgfusepath{clip}%
\pgfsetbuttcap%
\pgfsetroundjoin%
\definecolor{currentfill}{rgb}{0.449586,0.517371,0.612270}%
\pgfsetfillcolor{currentfill}%
\pgfsetlinewidth{0.000000pt}%
\definecolor{currentstroke}{rgb}{0.000000,0.000000,0.000000}%
\pgfsetstrokecolor{currentstroke}%
\pgfsetdash{}{0pt}%
\pgfpathmoveto{\pgfqpoint{0.962959in}{1.428347in}}%
\pgfpathlineto{\pgfqpoint{1.002696in}{1.436944in}}%
\pgfpathlineto{\pgfqpoint{0.964529in}{1.476632in}}%
\pgfpathlineto{\pgfqpoint{0.924682in}{1.468145in}}%
\pgfpathclose%
\pgfusepath{fill}%
\end{pgfscope}%
\begin{pgfscope}%
\pgfpathrectangle{\pgfqpoint{0.150000in}{0.150000in}}{\pgfqpoint{2.700000in}{1.950000in}}%
\pgfusepath{clip}%
\pgfsetbuttcap%
\pgfsetroundjoin%
\definecolor{currentfill}{rgb}{0.418490,0.490104,0.590365}%
\pgfsetfillcolor{currentfill}%
\pgfsetlinewidth{0.000000pt}%
\definecolor{currentstroke}{rgb}{0.000000,0.000000,0.000000}%
\pgfsetstrokecolor{currentstroke}%
\pgfsetdash{}{0pt}%
\pgfpathmoveto{\pgfqpoint{0.924682in}{1.468145in}}%
\pgfpathlineto{\pgfqpoint{0.964529in}{1.476632in}}%
\pgfpathlineto{\pgfqpoint{0.926357in}{1.516325in}}%
\pgfpathlineto{\pgfqpoint{0.886400in}{1.507949in}}%
\pgfpathclose%
\pgfusepath{fill}%
\end{pgfscope}%
\begin{pgfscope}%
\pgfpathrectangle{\pgfqpoint{0.150000in}{0.150000in}}{\pgfqpoint{2.700000in}{1.950000in}}%
\pgfusepath{clip}%
\pgfsetbuttcap%
\pgfsetroundjoin%
\definecolor{currentfill}{rgb}{0.387393,0.462837,0.568459}%
\pgfsetfillcolor{currentfill}%
\pgfsetlinewidth{0.000000pt}%
\definecolor{currentstroke}{rgb}{0.000000,0.000000,0.000000}%
\pgfsetstrokecolor{currentstroke}%
\pgfsetdash{}{0pt}%
\pgfpathmoveto{\pgfqpoint{0.886400in}{1.507949in}}%
\pgfpathlineto{\pgfqpoint{0.926357in}{1.516325in}}%
\pgfpathlineto{\pgfqpoint{0.888179in}{1.556023in}}%
\pgfpathlineto{\pgfqpoint{0.848112in}{1.547759in}}%
\pgfpathclose%
\pgfusepath{fill}%
\end{pgfscope}%
\begin{pgfscope}%
\pgfpathrectangle{\pgfqpoint{0.150000in}{0.150000in}}{\pgfqpoint{2.700000in}{1.950000in}}%
\pgfusepath{clip}%
\pgfsetbuttcap%
\pgfsetroundjoin%
\definecolor{currentfill}{rgb}{0.356296,0.435570,0.546553}%
\pgfsetfillcolor{currentfill}%
\pgfsetlinewidth{0.000000pt}%
\definecolor{currentstroke}{rgb}{0.000000,0.000000,0.000000}%
\pgfsetstrokecolor{currentstroke}%
\pgfsetdash{}{0pt}%
\pgfpathmoveto{\pgfqpoint{0.848112in}{1.547759in}}%
\pgfpathlineto{\pgfqpoint{0.888179in}{1.556023in}}%
\pgfpathlineto{\pgfqpoint{0.849996in}{1.595727in}}%
\pgfpathlineto{\pgfqpoint{0.809819in}{1.587573in}}%
\pgfpathclose%
\pgfusepath{fill}%
\end{pgfscope}%
\begin{pgfscope}%
\pgfpathrectangle{\pgfqpoint{0.150000in}{0.150000in}}{\pgfqpoint{2.700000in}{1.950000in}}%
\pgfusepath{clip}%
\pgfsetbuttcap%
\pgfsetroundjoin%
\definecolor{currentfill}{rgb}{0.960110,0.927619,0.930193}%
\pgfsetfillcolor{currentfill}%
\pgfsetlinewidth{0.000000pt}%
\definecolor{currentstroke}{rgb}{0.000000,0.000000,0.000000}%
\pgfsetstrokecolor{currentstroke}%
\pgfsetdash{}{0pt}%
\pgfpathmoveto{\pgfqpoint{1.727925in}{0.582856in}}%
\pgfpathlineto{\pgfqpoint{1.765567in}{0.593840in}}%
\pgfpathlineto{\pgfqpoint{1.727400in}{0.633524in}}%
\pgfpathlineto{\pgfqpoint{1.689647in}{0.622651in}}%
\pgfpathclose%
\pgfusepath{fill}%
\end{pgfscope}%
\begin{pgfscope}%
\pgfpathrectangle{\pgfqpoint{0.150000in}{0.150000in}}{\pgfqpoint{2.700000in}{1.950000in}}%
\pgfusepath{clip}%
\pgfsetbuttcap%
\pgfsetroundjoin%
\definecolor{currentfill}{rgb}{0.975306,0.955193,0.956786}%
\pgfsetfillcolor{currentfill}%
\pgfsetlinewidth{0.000000pt}%
\definecolor{currentstroke}{rgb}{0.000000,0.000000,0.000000}%
\pgfsetstrokecolor{currentstroke}%
\pgfsetdash{}{0pt}%
\pgfpathmoveto{\pgfqpoint{1.689647in}{0.622651in}}%
\pgfpathlineto{\pgfqpoint{1.727400in}{0.633524in}}%
\pgfpathlineto{\pgfqpoint{1.689228in}{0.673214in}}%
\pgfpathlineto{\pgfqpoint{1.651365in}{0.662451in}}%
\pgfpathclose%
\pgfusepath{fill}%
\end{pgfscope}%
\begin{pgfscope}%
\pgfpathrectangle{\pgfqpoint{0.150000in}{0.150000in}}{\pgfqpoint{2.700000in}{1.950000in}}%
\pgfusepath{clip}%
\pgfsetbuttcap%
\pgfsetroundjoin%
\definecolor{currentfill}{rgb}{0.994301,0.989660,0.990028}%
\pgfsetfillcolor{currentfill}%
\pgfsetlinewidth{0.000000pt}%
\definecolor{currentstroke}{rgb}{0.000000,0.000000,0.000000}%
\pgfsetstrokecolor{currentstroke}%
\pgfsetdash{}{0pt}%
\pgfpathmoveto{\pgfqpoint{1.651365in}{0.662451in}}%
\pgfpathlineto{\pgfqpoint{1.689228in}{0.673214in}}%
\pgfpathlineto{\pgfqpoint{1.651050in}{0.712908in}}%
\pgfpathlineto{\pgfqpoint{1.613078in}{0.702257in}}%
\pgfpathclose%
\pgfusepath{fill}%
\end{pgfscope}%
\begin{pgfscope}%
\pgfpathrectangle{\pgfqpoint{0.150000in}{0.150000in}}{\pgfqpoint{2.700000in}{1.950000in}}%
\pgfusepath{clip}%
\pgfsetbuttcap%
\pgfsetroundjoin%
\definecolor{currentfill}{rgb}{0.978232,0.980913,0.984666}%
\pgfsetfillcolor{currentfill}%
\pgfsetlinewidth{0.000000pt}%
\definecolor{currentstroke}{rgb}{0.000000,0.000000,0.000000}%
\pgfsetstrokecolor{currentstroke}%
\pgfsetdash{}{0pt}%
\pgfpathmoveto{\pgfqpoint{1.613078in}{0.702257in}}%
\pgfpathlineto{\pgfqpoint{1.651050in}{0.712908in}}%
\pgfpathlineto{\pgfqpoint{1.612868in}{0.752609in}}%
\pgfpathlineto{\pgfqpoint{1.574785in}{0.742069in}}%
\pgfpathclose%
\pgfusepath{fill}%
\end{pgfscope}%
\begin{pgfscope}%
\pgfpathrectangle{\pgfqpoint{0.150000in}{0.150000in}}{\pgfqpoint{2.700000in}{1.950000in}}%
\pgfusepath{clip}%
\pgfsetbuttcap%
\pgfsetroundjoin%
\definecolor{currentfill}{rgb}{0.947135,0.953646,0.962760}%
\pgfsetfillcolor{currentfill}%
\pgfsetlinewidth{0.000000pt}%
\definecolor{currentstroke}{rgb}{0.000000,0.000000,0.000000}%
\pgfsetstrokecolor{currentstroke}%
\pgfsetdash{}{0pt}%
\pgfpathmoveto{\pgfqpoint{1.574785in}{0.742069in}}%
\pgfpathlineto{\pgfqpoint{1.612868in}{0.752609in}}%
\pgfpathlineto{\pgfqpoint{1.574680in}{0.792314in}}%
\pgfpathlineto{\pgfqpoint{1.536486in}{0.781885in}}%
\pgfpathclose%
\pgfusepath{fill}%
\end{pgfscope}%
\begin{pgfscope}%
\pgfpathrectangle{\pgfqpoint{0.150000in}{0.150000in}}{\pgfqpoint{2.700000in}{1.950000in}}%
\pgfusepath{clip}%
\pgfsetbuttcap%
\pgfsetroundjoin%
\definecolor{currentfill}{rgb}{0.916039,0.926379,0.940855}%
\pgfsetfillcolor{currentfill}%
\pgfsetlinewidth{0.000000pt}%
\definecolor{currentstroke}{rgb}{0.000000,0.000000,0.000000}%
\pgfsetstrokecolor{currentstroke}%
\pgfsetdash{}{0pt}%
\pgfpathmoveto{\pgfqpoint{1.536486in}{0.781885in}}%
\pgfpathlineto{\pgfqpoint{1.574680in}{0.792314in}}%
\pgfpathlineto{\pgfqpoint{1.536486in}{0.832025in}}%
\pgfpathlineto{\pgfqpoint{1.498183in}{0.821708in}}%
\pgfpathclose%
\pgfusepath{fill}%
\end{pgfscope}%
\begin{pgfscope}%
\pgfpathrectangle{\pgfqpoint{0.150000in}{0.150000in}}{\pgfqpoint{2.700000in}{1.950000in}}%
\pgfusepath{clip}%
\pgfsetbuttcap%
\pgfsetroundjoin%
\definecolor{currentfill}{rgb}{0.884942,0.899112,0.918949}%
\pgfsetfillcolor{currentfill}%
\pgfsetlinewidth{0.000000pt}%
\definecolor{currentstroke}{rgb}{0.000000,0.000000,0.000000}%
\pgfsetstrokecolor{currentstroke}%
\pgfsetdash{}{0pt}%
\pgfpathmoveto{\pgfqpoint{1.498183in}{0.821708in}}%
\pgfpathlineto{\pgfqpoint{1.536486in}{0.832025in}}%
\pgfpathlineto{\pgfqpoint{1.498288in}{0.871742in}}%
\pgfpathlineto{\pgfqpoint{1.459874in}{0.861535in}}%
\pgfpathclose%
\pgfusepath{fill}%
\end{pgfscope}%
\begin{pgfscope}%
\pgfpathrectangle{\pgfqpoint{0.150000in}{0.150000in}}{\pgfqpoint{2.700000in}{1.950000in}}%
\pgfusepath{clip}%
\pgfsetbuttcap%
\pgfsetroundjoin%
\definecolor{currentfill}{rgb}{0.853845,0.871844,0.897044}%
\pgfsetfillcolor{currentfill}%
\pgfsetlinewidth{0.000000pt}%
\definecolor{currentstroke}{rgb}{0.000000,0.000000,0.000000}%
\pgfsetstrokecolor{currentstroke}%
\pgfsetdash{}{0pt}%
\pgfpathmoveto{\pgfqpoint{1.459874in}{0.861535in}}%
\pgfpathlineto{\pgfqpoint{1.498288in}{0.871742in}}%
\pgfpathlineto{\pgfqpoint{1.460084in}{0.911464in}}%
\pgfpathlineto{\pgfqpoint{1.421560in}{0.901369in}}%
\pgfpathclose%
\pgfusepath{fill}%
\end{pgfscope}%
\begin{pgfscope}%
\pgfpathrectangle{\pgfqpoint{0.150000in}{0.150000in}}{\pgfqpoint{2.700000in}{1.950000in}}%
\pgfusepath{clip}%
\pgfsetbuttcap%
\pgfsetroundjoin%
\definecolor{currentfill}{rgb}{0.822748,0.844577,0.875138}%
\pgfsetfillcolor{currentfill}%
\pgfsetlinewidth{0.000000pt}%
\definecolor{currentstroke}{rgb}{0.000000,0.000000,0.000000}%
\pgfsetstrokecolor{currentstroke}%
\pgfsetdash{}{0pt}%
\pgfpathmoveto{\pgfqpoint{1.421560in}{0.901369in}}%
\pgfpathlineto{\pgfqpoint{1.460084in}{0.911464in}}%
\pgfpathlineto{\pgfqpoint{1.421876in}{0.951191in}}%
\pgfpathlineto{\pgfqpoint{1.383241in}{0.941207in}}%
\pgfpathclose%
\pgfusepath{fill}%
\end{pgfscope}%
\begin{pgfscope}%
\pgfpathrectangle{\pgfqpoint{0.150000in}{0.150000in}}{\pgfqpoint{2.700000in}{1.950000in}}%
\pgfusepath{clip}%
\pgfsetbuttcap%
\pgfsetroundjoin%
\definecolor{currentfill}{rgb}{0.797871,0.822763,0.857613}%
\pgfsetfillcolor{currentfill}%
\pgfsetlinewidth{0.000000pt}%
\definecolor{currentstroke}{rgb}{0.000000,0.000000,0.000000}%
\pgfsetstrokecolor{currentstroke}%
\pgfsetdash{}{0pt}%
\pgfpathmoveto{\pgfqpoint{1.383241in}{0.941207in}}%
\pgfpathlineto{\pgfqpoint{1.421876in}{0.951191in}}%
\pgfpathlineto{\pgfqpoint{1.383661in}{0.990924in}}%
\pgfpathlineto{\pgfqpoint{1.344917in}{0.981052in}}%
\pgfpathclose%
\pgfusepath{fill}%
\end{pgfscope}%
\begin{pgfscope}%
\pgfpathrectangle{\pgfqpoint{0.150000in}{0.150000in}}{\pgfqpoint{2.700000in}{1.950000in}}%
\pgfusepath{clip}%
\pgfsetbuttcap%
\pgfsetroundjoin%
\definecolor{currentfill}{rgb}{0.766774,0.795496,0.835708}%
\pgfsetfillcolor{currentfill}%
\pgfsetlinewidth{0.000000pt}%
\definecolor{currentstroke}{rgb}{0.000000,0.000000,0.000000}%
\pgfsetstrokecolor{currentstroke}%
\pgfsetdash{}{0pt}%
\pgfpathmoveto{\pgfqpoint{1.344917in}{0.981052in}}%
\pgfpathlineto{\pgfqpoint{1.383661in}{0.990924in}}%
\pgfpathlineto{\pgfqpoint{1.345442in}{1.030663in}}%
\pgfpathlineto{\pgfqpoint{1.306587in}{1.020901in}}%
\pgfpathclose%
\pgfusepath{fill}%
\end{pgfscope}%
\begin{pgfscope}%
\pgfpathrectangle{\pgfqpoint{0.150000in}{0.150000in}}{\pgfqpoint{2.700000in}{1.950000in}}%
\pgfusepath{clip}%
\pgfsetbuttcap%
\pgfsetroundjoin%
\definecolor{currentfill}{rgb}{0.735677,0.768229,0.813802}%
\pgfsetfillcolor{currentfill}%
\pgfsetlinewidth{0.000000pt}%
\definecolor{currentstroke}{rgb}{0.000000,0.000000,0.000000}%
\pgfsetstrokecolor{currentstroke}%
\pgfsetdash{}{0pt}%
\pgfpathmoveto{\pgfqpoint{1.306587in}{1.020901in}}%
\pgfpathlineto{\pgfqpoint{1.345442in}{1.030663in}}%
\pgfpathlineto{\pgfqpoint{1.307217in}{1.070407in}}%
\pgfpathlineto{\pgfqpoint{1.268252in}{1.060757in}}%
\pgfpathclose%
\pgfusepath{fill}%
\end{pgfscope}%
\begin{pgfscope}%
\pgfpathrectangle{\pgfqpoint{0.150000in}{0.150000in}}{\pgfqpoint{2.700000in}{1.950000in}}%
\pgfusepath{clip}%
\pgfsetbuttcap%
\pgfsetroundjoin%
\definecolor{currentfill}{rgb}{0.704580,0.740962,0.791896}%
\pgfsetfillcolor{currentfill}%
\pgfsetlinewidth{0.000000pt}%
\definecolor{currentstroke}{rgb}{0.000000,0.000000,0.000000}%
\pgfsetstrokecolor{currentstroke}%
\pgfsetdash{}{0pt}%
\pgfpathmoveto{\pgfqpoint{1.268252in}{1.060757in}}%
\pgfpathlineto{\pgfqpoint{1.307217in}{1.070407in}}%
\pgfpathlineto{\pgfqpoint{1.268988in}{1.110156in}}%
\pgfpathlineto{\pgfqpoint{1.229911in}{1.100617in}}%
\pgfpathclose%
\pgfusepath{fill}%
\end{pgfscope}%
\begin{pgfscope}%
\pgfpathrectangle{\pgfqpoint{0.150000in}{0.150000in}}{\pgfqpoint{2.700000in}{1.950000in}}%
\pgfusepath{clip}%
\pgfsetbuttcap%
\pgfsetroundjoin%
\definecolor{currentfill}{rgb}{0.673483,0.713695,0.769991}%
\pgfsetfillcolor{currentfill}%
\pgfsetlinewidth{0.000000pt}%
\definecolor{currentstroke}{rgb}{0.000000,0.000000,0.000000}%
\pgfsetstrokecolor{currentstroke}%
\pgfsetdash{}{0pt}%
\pgfpathmoveto{\pgfqpoint{1.229911in}{1.100617in}}%
\pgfpathlineto{\pgfqpoint{1.268988in}{1.110156in}}%
\pgfpathlineto{\pgfqpoint{1.230752in}{1.149911in}}%
\pgfpathlineto{\pgfqpoint{1.191566in}{1.140484in}}%
\pgfpathclose%
\pgfusepath{fill}%
\end{pgfscope}%
\begin{pgfscope}%
\pgfpathrectangle{\pgfqpoint{0.150000in}{0.150000in}}{\pgfqpoint{2.700000in}{1.950000in}}%
\pgfusepath{clip}%
\pgfsetbuttcap%
\pgfsetroundjoin%
\definecolor{currentfill}{rgb}{0.642387,0.686428,0.748085}%
\pgfsetfillcolor{currentfill}%
\pgfsetlinewidth{0.000000pt}%
\definecolor{currentstroke}{rgb}{0.000000,0.000000,0.000000}%
\pgfsetstrokecolor{currentstroke}%
\pgfsetdash{}{0pt}%
\pgfpathmoveto{\pgfqpoint{1.191566in}{1.140484in}}%
\pgfpathlineto{\pgfqpoint{1.230752in}{1.149911in}}%
\pgfpathlineto{\pgfqpoint{1.192512in}{1.189671in}}%
\pgfpathlineto{\pgfqpoint{1.153215in}{1.180355in}}%
\pgfpathclose%
\pgfusepath{fill}%
\end{pgfscope}%
\begin{pgfscope}%
\pgfpathrectangle{\pgfqpoint{0.150000in}{0.150000in}}{\pgfqpoint{2.700000in}{1.950000in}}%
\pgfusepath{clip}%
\pgfsetbuttcap%
\pgfsetroundjoin%
\definecolor{currentfill}{rgb}{0.611290,0.659161,0.726180}%
\pgfsetfillcolor{currentfill}%
\pgfsetlinewidth{0.000000pt}%
\definecolor{currentstroke}{rgb}{0.000000,0.000000,0.000000}%
\pgfsetstrokecolor{currentstroke}%
\pgfsetdash{}{0pt}%
\pgfpathmoveto{\pgfqpoint{1.153215in}{1.180355in}}%
\pgfpathlineto{\pgfqpoint{1.192512in}{1.189671in}}%
\pgfpathlineto{\pgfqpoint{1.154266in}{1.229436in}}%
\pgfpathlineto{\pgfqpoint{1.114859in}{1.220232in}}%
\pgfpathclose%
\pgfusepath{fill}%
\end{pgfscope}%
\begin{pgfscope}%
\pgfpathrectangle{\pgfqpoint{0.150000in}{0.150000in}}{\pgfqpoint{2.700000in}{1.950000in}}%
\pgfusepath{clip}%
\pgfsetbuttcap%
\pgfsetroundjoin%
\definecolor{currentfill}{rgb}{0.580193,0.631893,0.704274}%
\pgfsetfillcolor{currentfill}%
\pgfsetlinewidth{0.000000pt}%
\definecolor{currentstroke}{rgb}{0.000000,0.000000,0.000000}%
\pgfsetstrokecolor{currentstroke}%
\pgfsetdash{}{0pt}%
\pgfpathmoveto{\pgfqpoint{1.114859in}{1.220232in}}%
\pgfpathlineto{\pgfqpoint{1.154266in}{1.229436in}}%
\pgfpathlineto{\pgfqpoint{1.116015in}{1.269208in}}%
\pgfpathlineto{\pgfqpoint{1.076497in}{1.260115in}}%
\pgfpathclose%
\pgfusepath{fill}%
\end{pgfscope}%
\begin{pgfscope}%
\pgfpathrectangle{\pgfqpoint{0.150000in}{0.150000in}}{\pgfqpoint{2.700000in}{1.950000in}}%
\pgfusepath{clip}%
\pgfsetbuttcap%
\pgfsetroundjoin%
\definecolor{currentfill}{rgb}{0.555316,0.610080,0.686749}%
\pgfsetfillcolor{currentfill}%
\pgfsetlinewidth{0.000000pt}%
\definecolor{currentstroke}{rgb}{0.000000,0.000000,0.000000}%
\pgfsetstrokecolor{currentstroke}%
\pgfsetdash{}{0pt}%
\pgfpathmoveto{\pgfqpoint{1.076497in}{1.260115in}}%
\pgfpathlineto{\pgfqpoint{1.116015in}{1.269208in}}%
\pgfpathlineto{\pgfqpoint{1.077759in}{1.308984in}}%
\pgfpathlineto{\pgfqpoint{1.038130in}{1.300003in}}%
\pgfpathclose%
\pgfusepath{fill}%
\end{pgfscope}%
\begin{pgfscope}%
\pgfpathrectangle{\pgfqpoint{0.150000in}{0.150000in}}{\pgfqpoint{2.700000in}{1.950000in}}%
\pgfusepath{clip}%
\pgfsetbuttcap%
\pgfsetroundjoin%
\definecolor{currentfill}{rgb}{0.524219,0.582812,0.664844}%
\pgfsetfillcolor{currentfill}%
\pgfsetlinewidth{0.000000pt}%
\definecolor{currentstroke}{rgb}{0.000000,0.000000,0.000000}%
\pgfsetstrokecolor{currentstroke}%
\pgfsetdash{}{0pt}%
\pgfpathmoveto{\pgfqpoint{1.038130in}{1.300003in}}%
\pgfpathlineto{\pgfqpoint{1.077759in}{1.308984in}}%
\pgfpathlineto{\pgfqpoint{1.039498in}{1.348766in}}%
\pgfpathlineto{\pgfqpoint{0.999758in}{1.339897in}}%
\pgfpathclose%
\pgfusepath{fill}%
\end{pgfscope}%
\begin{pgfscope}%
\pgfpathrectangle{\pgfqpoint{0.150000in}{0.150000in}}{\pgfqpoint{2.700000in}{1.950000in}}%
\pgfusepath{clip}%
\pgfsetbuttcap%
\pgfsetroundjoin%
\definecolor{currentfill}{rgb}{0.493122,0.555545,0.642938}%
\pgfsetfillcolor{currentfill}%
\pgfsetlinewidth{0.000000pt}%
\definecolor{currentstroke}{rgb}{0.000000,0.000000,0.000000}%
\pgfsetstrokecolor{currentstroke}%
\pgfsetdash{}{0pt}%
\pgfpathmoveto{\pgfqpoint{0.999758in}{1.339897in}}%
\pgfpathlineto{\pgfqpoint{1.039498in}{1.348766in}}%
\pgfpathlineto{\pgfqpoint{1.001231in}{1.388554in}}%
\pgfpathlineto{\pgfqpoint{0.961381in}{1.379796in}}%
\pgfpathclose%
\pgfusepath{fill}%
\end{pgfscope}%
\begin{pgfscope}%
\pgfpathrectangle{\pgfqpoint{0.150000in}{0.150000in}}{\pgfqpoint{2.700000in}{1.950000in}}%
\pgfusepath{clip}%
\pgfsetbuttcap%
\pgfsetroundjoin%
\definecolor{currentfill}{rgb}{0.462025,0.528278,0.621032}%
\pgfsetfillcolor{currentfill}%
\pgfsetlinewidth{0.000000pt}%
\definecolor{currentstroke}{rgb}{0.000000,0.000000,0.000000}%
\pgfsetstrokecolor{currentstroke}%
\pgfsetdash{}{0pt}%
\pgfpathmoveto{\pgfqpoint{0.961381in}{1.379796in}}%
\pgfpathlineto{\pgfqpoint{1.001231in}{1.388554in}}%
\pgfpathlineto{\pgfqpoint{0.962959in}{1.428347in}}%
\pgfpathlineto{\pgfqpoint{0.922998in}{1.419701in}}%
\pgfpathclose%
\pgfusepath{fill}%
\end{pgfscope}%
\begin{pgfscope}%
\pgfpathrectangle{\pgfqpoint{0.150000in}{0.150000in}}{\pgfqpoint{2.700000in}{1.950000in}}%
\pgfusepath{clip}%
\pgfsetbuttcap%
\pgfsetroundjoin%
\definecolor{currentfill}{rgb}{0.430928,0.501011,0.599127}%
\pgfsetfillcolor{currentfill}%
\pgfsetlinewidth{0.000000pt}%
\definecolor{currentstroke}{rgb}{0.000000,0.000000,0.000000}%
\pgfsetstrokecolor{currentstroke}%
\pgfsetdash{}{0pt}%
\pgfpathmoveto{\pgfqpoint{0.922998in}{1.419701in}}%
\pgfpathlineto{\pgfqpoint{0.962959in}{1.428347in}}%
\pgfpathlineto{\pgfqpoint{0.924682in}{1.468145in}}%
\pgfpathlineto{\pgfqpoint{0.884610in}{1.459611in}}%
\pgfpathclose%
\pgfusepath{fill}%
\end{pgfscope}%
\begin{pgfscope}%
\pgfpathrectangle{\pgfqpoint{0.150000in}{0.150000in}}{\pgfqpoint{2.700000in}{1.950000in}}%
\pgfusepath{clip}%
\pgfsetbuttcap%
\pgfsetroundjoin%
\definecolor{currentfill}{rgb}{0.399831,0.473744,0.577221}%
\pgfsetfillcolor{currentfill}%
\pgfsetlinewidth{0.000000pt}%
\definecolor{currentstroke}{rgb}{0.000000,0.000000,0.000000}%
\pgfsetstrokecolor{currentstroke}%
\pgfsetdash{}{0pt}%
\pgfpathmoveto{\pgfqpoint{0.884610in}{1.459611in}}%
\pgfpathlineto{\pgfqpoint{0.924682in}{1.468145in}}%
\pgfpathlineto{\pgfqpoint{0.886400in}{1.507949in}}%
\pgfpathlineto{\pgfqpoint{0.846217in}{1.499526in}}%
\pgfpathclose%
\pgfusepath{fill}%
\end{pgfscope}%
\begin{pgfscope}%
\pgfpathrectangle{\pgfqpoint{0.150000in}{0.150000in}}{\pgfqpoint{2.700000in}{1.950000in}}%
\pgfusepath{clip}%
\pgfsetbuttcap%
\pgfsetroundjoin%
\definecolor{currentfill}{rgb}{0.368735,0.446477,0.555316}%
\pgfsetfillcolor{currentfill}%
\pgfsetlinewidth{0.000000pt}%
\definecolor{currentstroke}{rgb}{0.000000,0.000000,0.000000}%
\pgfsetstrokecolor{currentstroke}%
\pgfsetdash{}{0pt}%
\pgfpathmoveto{\pgfqpoint{0.846217in}{1.499526in}}%
\pgfpathlineto{\pgfqpoint{0.886400in}{1.507949in}}%
\pgfpathlineto{\pgfqpoint{0.848112in}{1.547759in}}%
\pgfpathlineto{\pgfqpoint{0.807819in}{1.539448in}}%
\pgfpathclose%
\pgfusepath{fill}%
\end{pgfscope}%
\begin{pgfscope}%
\pgfpathrectangle{\pgfqpoint{0.150000in}{0.150000in}}{\pgfqpoint{2.700000in}{1.950000in}}%
\pgfusepath{clip}%
\pgfsetbuttcap%
\pgfsetroundjoin%
\definecolor{currentfill}{rgb}{0.337638,0.419210,0.533410}%
\pgfsetfillcolor{currentfill}%
\pgfsetlinewidth{0.000000pt}%
\definecolor{currentstroke}{rgb}{0.000000,0.000000,0.000000}%
\pgfsetstrokecolor{currentstroke}%
\pgfsetdash{}{0pt}%
\pgfpathmoveto{\pgfqpoint{0.807819in}{1.539448in}}%
\pgfpathlineto{\pgfqpoint{0.848112in}{1.547759in}}%
\pgfpathlineto{\pgfqpoint{0.809819in}{1.587573in}}%
\pgfpathlineto{\pgfqpoint{0.769415in}{1.579374in}}%
\pgfpathclose%
\pgfusepath{fill}%
\end{pgfscope}%
\begin{pgfscope}%
\pgfpathrectangle{\pgfqpoint{0.150000in}{0.150000in}}{\pgfqpoint{2.700000in}{1.950000in}}%
\pgfusepath{clip}%
\pgfsetbuttcap%
\pgfsetroundjoin%
\definecolor{currentfill}{rgb}{0.967708,0.941406,0.943490}%
\pgfsetfillcolor{currentfill}%
\pgfsetlinewidth{0.000000pt}%
\definecolor{currentstroke}{rgb}{0.000000,0.000000,0.000000}%
\pgfsetstrokecolor{currentstroke}%
\pgfsetdash{}{0pt}%
\pgfpathmoveto{\pgfqpoint{1.690070in}{0.571810in}}%
\pgfpathlineto{\pgfqpoint{1.727925in}{0.582856in}}%
\pgfpathlineto{\pgfqpoint{1.689647in}{0.622651in}}%
\pgfpathlineto{\pgfqpoint{1.651682in}{0.611716in}}%
\pgfpathclose%
\pgfusepath{fill}%
\end{pgfscope}%
\begin{pgfscope}%
\pgfpathrectangle{\pgfqpoint{0.150000in}{0.150000in}}{\pgfqpoint{2.700000in}{1.950000in}}%
\pgfusepath{clip}%
\pgfsetbuttcap%
\pgfsetroundjoin%
\definecolor{currentfill}{rgb}{0.986703,0.975873,0.976731}%
\pgfsetfillcolor{currentfill}%
\pgfsetlinewidth{0.000000pt}%
\definecolor{currentstroke}{rgb}{0.000000,0.000000,0.000000}%
\pgfsetstrokecolor{currentstroke}%
\pgfsetdash{}{0pt}%
\pgfpathmoveto{\pgfqpoint{1.651682in}{0.611716in}}%
\pgfpathlineto{\pgfqpoint{1.689647in}{0.622651in}}%
\pgfpathlineto{\pgfqpoint{1.651365in}{0.662451in}}%
\pgfpathlineto{\pgfqpoint{1.613289in}{0.651628in}}%
\pgfpathclose%
\pgfusepath{fill}%
\end{pgfscope}%
\begin{pgfscope}%
\pgfpathrectangle{\pgfqpoint{0.150000in}{0.150000in}}{\pgfqpoint{2.700000in}{1.950000in}}%
\pgfusepath{clip}%
\pgfsetbuttcap%
\pgfsetroundjoin%
\definecolor{currentfill}{rgb}{0.990671,0.991820,0.993428}%
\pgfsetfillcolor{currentfill}%
\pgfsetlinewidth{0.000000pt}%
\definecolor{currentstroke}{rgb}{0.000000,0.000000,0.000000}%
\pgfsetstrokecolor{currentstroke}%
\pgfsetdash{}{0pt}%
\pgfpathmoveto{\pgfqpoint{1.613289in}{0.651628in}}%
\pgfpathlineto{\pgfqpoint{1.651365in}{0.662451in}}%
\pgfpathlineto{\pgfqpoint{1.613078in}{0.702257in}}%
\pgfpathlineto{\pgfqpoint{1.574890in}{0.691546in}}%
\pgfpathclose%
\pgfusepath{fill}%
\end{pgfscope}%
\begin{pgfscope}%
\pgfpathrectangle{\pgfqpoint{0.150000in}{0.150000in}}{\pgfqpoint{2.700000in}{1.950000in}}%
\pgfusepath{clip}%
\pgfsetbuttcap%
\pgfsetroundjoin%
\definecolor{currentfill}{rgb}{0.959574,0.964553,0.971523}%
\pgfsetfillcolor{currentfill}%
\pgfsetlinewidth{0.000000pt}%
\definecolor{currentstroke}{rgb}{0.000000,0.000000,0.000000}%
\pgfsetstrokecolor{currentstroke}%
\pgfsetdash{}{0pt}%
\pgfpathmoveto{\pgfqpoint{1.574890in}{0.691546in}}%
\pgfpathlineto{\pgfqpoint{1.613078in}{0.702257in}}%
\pgfpathlineto{\pgfqpoint{1.574785in}{0.742069in}}%
\pgfpathlineto{\pgfqpoint{1.536486in}{0.731469in}}%
\pgfpathclose%
\pgfusepath{fill}%
\end{pgfscope}%
\begin{pgfscope}%
\pgfpathrectangle{\pgfqpoint{0.150000in}{0.150000in}}{\pgfqpoint{2.700000in}{1.950000in}}%
\pgfusepath{clip}%
\pgfsetbuttcap%
\pgfsetroundjoin%
\definecolor{currentfill}{rgb}{0.934697,0.942739,0.953998}%
\pgfsetfillcolor{currentfill}%
\pgfsetlinewidth{0.000000pt}%
\definecolor{currentstroke}{rgb}{0.000000,0.000000,0.000000}%
\pgfsetstrokecolor{currentstroke}%
\pgfsetdash{}{0pt}%
\pgfpathmoveto{\pgfqpoint{1.536486in}{0.731469in}}%
\pgfpathlineto{\pgfqpoint{1.574785in}{0.742069in}}%
\pgfpathlineto{\pgfqpoint{1.536486in}{0.781885in}}%
\pgfpathlineto{\pgfqpoint{1.498077in}{0.771398in}}%
\pgfpathclose%
\pgfusepath{fill}%
\end{pgfscope}%
\begin{pgfscope}%
\pgfpathrectangle{\pgfqpoint{0.150000in}{0.150000in}}{\pgfqpoint{2.700000in}{1.950000in}}%
\pgfusepath{clip}%
\pgfsetbuttcap%
\pgfsetroundjoin%
\definecolor{currentfill}{rgb}{0.903600,0.915472,0.932093}%
\pgfsetfillcolor{currentfill}%
\pgfsetlinewidth{0.000000pt}%
\definecolor{currentstroke}{rgb}{0.000000,0.000000,0.000000}%
\pgfsetstrokecolor{currentstroke}%
\pgfsetdash{}{0pt}%
\pgfpathmoveto{\pgfqpoint{1.498077in}{0.771398in}}%
\pgfpathlineto{\pgfqpoint{1.536486in}{0.781885in}}%
\pgfpathlineto{\pgfqpoint{1.498183in}{0.821708in}}%
\pgfpathlineto{\pgfqpoint{1.459663in}{0.811332in}}%
\pgfpathclose%
\pgfusepath{fill}%
\end{pgfscope}%
\begin{pgfscope}%
\pgfpathrectangle{\pgfqpoint{0.150000in}{0.150000in}}{\pgfqpoint{2.700000in}{1.950000in}}%
\pgfusepath{clip}%
\pgfsetbuttcap%
\pgfsetroundjoin%
\definecolor{currentfill}{rgb}{0.872503,0.888205,0.910187}%
\pgfsetfillcolor{currentfill}%
\pgfsetlinewidth{0.000000pt}%
\definecolor{currentstroke}{rgb}{0.000000,0.000000,0.000000}%
\pgfsetstrokecolor{currentstroke}%
\pgfsetdash{}{0pt}%
\pgfpathmoveto{\pgfqpoint{1.459663in}{0.811332in}}%
\pgfpathlineto{\pgfqpoint{1.498183in}{0.821708in}}%
\pgfpathlineto{\pgfqpoint{1.459874in}{0.861535in}}%
\pgfpathlineto{\pgfqpoint{1.421244in}{0.851271in}}%
\pgfpathclose%
\pgfusepath{fill}%
\end{pgfscope}%
\begin{pgfscope}%
\pgfpathrectangle{\pgfqpoint{0.150000in}{0.150000in}}{\pgfqpoint{2.700000in}{1.950000in}}%
\pgfusepath{clip}%
\pgfsetbuttcap%
\pgfsetroundjoin%
\definecolor{currentfill}{rgb}{0.841406,0.860938,0.888281}%
\pgfsetfillcolor{currentfill}%
\pgfsetlinewidth{0.000000pt}%
\definecolor{currentstroke}{rgb}{0.000000,0.000000,0.000000}%
\pgfsetstrokecolor{currentstroke}%
\pgfsetdash{}{0pt}%
\pgfpathmoveto{\pgfqpoint{1.421244in}{0.851271in}}%
\pgfpathlineto{\pgfqpoint{1.459874in}{0.861535in}}%
\pgfpathlineto{\pgfqpoint{1.421560in}{0.901369in}}%
\pgfpathlineto{\pgfqpoint{1.382819in}{0.891216in}}%
\pgfpathclose%
\pgfusepath{fill}%
\end{pgfscope}%
\begin{pgfscope}%
\pgfpathrectangle{\pgfqpoint{0.150000in}{0.150000in}}{\pgfqpoint{2.700000in}{1.950000in}}%
\pgfusepath{clip}%
\pgfsetbuttcap%
\pgfsetroundjoin%
\definecolor{currentfill}{rgb}{0.810309,0.833670,0.866376}%
\pgfsetfillcolor{currentfill}%
\pgfsetlinewidth{0.000000pt}%
\definecolor{currentstroke}{rgb}{0.000000,0.000000,0.000000}%
\pgfsetstrokecolor{currentstroke}%
\pgfsetdash{}{0pt}%
\pgfpathmoveto{\pgfqpoint{1.382819in}{0.891216in}}%
\pgfpathlineto{\pgfqpoint{1.421560in}{0.901369in}}%
\pgfpathlineto{\pgfqpoint{1.383241in}{0.941207in}}%
\pgfpathlineto{\pgfqpoint{1.344388in}{0.931167in}}%
\pgfpathclose%
\pgfusepath{fill}%
\end{pgfscope}%
\begin{pgfscope}%
\pgfpathrectangle{\pgfqpoint{0.150000in}{0.150000in}}{\pgfqpoint{2.700000in}{1.950000in}}%
\pgfusepath{clip}%
\pgfsetbuttcap%
\pgfsetroundjoin%
\definecolor{currentfill}{rgb}{0.779213,0.806403,0.844470}%
\pgfsetfillcolor{currentfill}%
\pgfsetlinewidth{0.000000pt}%
\definecolor{currentstroke}{rgb}{0.000000,0.000000,0.000000}%
\pgfsetstrokecolor{currentstroke}%
\pgfsetdash{}{0pt}%
\pgfpathmoveto{\pgfqpoint{1.344388in}{0.931167in}}%
\pgfpathlineto{\pgfqpoint{1.383241in}{0.941207in}}%
\pgfpathlineto{\pgfqpoint{1.344917in}{0.981052in}}%
\pgfpathlineto{\pgfqpoint{1.305953in}{0.971123in}}%
\pgfpathclose%
\pgfusepath{fill}%
\end{pgfscope}%
\begin{pgfscope}%
\pgfpathrectangle{\pgfqpoint{0.150000in}{0.150000in}}{\pgfqpoint{2.700000in}{1.950000in}}%
\pgfusepath{clip}%
\pgfsetbuttcap%
\pgfsetroundjoin%
\definecolor{currentfill}{rgb}{0.748116,0.779136,0.822564}%
\pgfsetfillcolor{currentfill}%
\pgfsetlinewidth{0.000000pt}%
\definecolor{currentstroke}{rgb}{0.000000,0.000000,0.000000}%
\pgfsetstrokecolor{currentstroke}%
\pgfsetdash{}{0pt}%
\pgfpathmoveto{\pgfqpoint{1.305953in}{0.971123in}}%
\pgfpathlineto{\pgfqpoint{1.344917in}{0.981052in}}%
\pgfpathlineto{\pgfqpoint{1.306587in}{1.020901in}}%
\pgfpathlineto{\pgfqpoint{1.267512in}{1.011085in}}%
\pgfpathclose%
\pgfusepath{fill}%
\end{pgfscope}%
\begin{pgfscope}%
\pgfpathrectangle{\pgfqpoint{0.150000in}{0.150000in}}{\pgfqpoint{2.700000in}{1.950000in}}%
\pgfusepath{clip}%
\pgfsetbuttcap%
\pgfsetroundjoin%
\definecolor{currentfill}{rgb}{0.717019,0.751869,0.800659}%
\pgfsetfillcolor{currentfill}%
\pgfsetlinewidth{0.000000pt}%
\definecolor{currentstroke}{rgb}{0.000000,0.000000,0.000000}%
\pgfsetstrokecolor{currentstroke}%
\pgfsetdash{}{0pt}%
\pgfpathmoveto{\pgfqpoint{1.267512in}{1.011085in}}%
\pgfpathlineto{\pgfqpoint{1.306587in}{1.020901in}}%
\pgfpathlineto{\pgfqpoint{1.268252in}{1.060757in}}%
\pgfpathlineto{\pgfqpoint{1.229066in}{1.051052in}}%
\pgfpathclose%
\pgfusepath{fill}%
\end{pgfscope}%
\begin{pgfscope}%
\pgfpathrectangle{\pgfqpoint{0.150000in}{0.150000in}}{\pgfqpoint{2.700000in}{1.950000in}}%
\pgfusepath{clip}%
\pgfsetbuttcap%
\pgfsetroundjoin%
\definecolor{currentfill}{rgb}{0.692142,0.730055,0.783134}%
\pgfsetfillcolor{currentfill}%
\pgfsetlinewidth{0.000000pt}%
\definecolor{currentstroke}{rgb}{0.000000,0.000000,0.000000}%
\pgfsetstrokecolor{currentstroke}%
\pgfsetdash{}{0pt}%
\pgfpathmoveto{\pgfqpoint{1.229066in}{1.051052in}}%
\pgfpathlineto{\pgfqpoint{1.268252in}{1.060757in}}%
\pgfpathlineto{\pgfqpoint{1.229911in}{1.100617in}}%
\pgfpathlineto{\pgfqpoint{1.190614in}{1.091025in}}%
\pgfpathclose%
\pgfusepath{fill}%
\end{pgfscope}%
\begin{pgfscope}%
\pgfpathrectangle{\pgfqpoint{0.150000in}{0.150000in}}{\pgfqpoint{2.700000in}{1.950000in}}%
\pgfusepath{clip}%
\pgfsetbuttcap%
\pgfsetroundjoin%
\definecolor{currentfill}{rgb}{0.661045,0.702788,0.761229}%
\pgfsetfillcolor{currentfill}%
\pgfsetlinewidth{0.000000pt}%
\definecolor{currentstroke}{rgb}{0.000000,0.000000,0.000000}%
\pgfsetstrokecolor{currentstroke}%
\pgfsetdash{}{0pt}%
\pgfpathmoveto{\pgfqpoint{1.190614in}{1.091025in}}%
\pgfpathlineto{\pgfqpoint{1.229911in}{1.100617in}}%
\pgfpathlineto{\pgfqpoint{1.191566in}{1.140484in}}%
\pgfpathlineto{\pgfqpoint{1.152158in}{1.131003in}}%
\pgfpathclose%
\pgfusepath{fill}%
\end{pgfscope}%
\begin{pgfscope}%
\pgfpathrectangle{\pgfqpoint{0.150000in}{0.150000in}}{\pgfqpoint{2.700000in}{1.950000in}}%
\pgfusepath{clip}%
\pgfsetbuttcap%
\pgfsetroundjoin%
\definecolor{currentfill}{rgb}{0.629948,0.675521,0.739323}%
\pgfsetfillcolor{currentfill}%
\pgfsetlinewidth{0.000000pt}%
\definecolor{currentstroke}{rgb}{0.000000,0.000000,0.000000}%
\pgfsetstrokecolor{currentstroke}%
\pgfsetdash{}{0pt}%
\pgfpathmoveto{\pgfqpoint{1.152158in}{1.131003in}}%
\pgfpathlineto{\pgfqpoint{1.191566in}{1.140484in}}%
\pgfpathlineto{\pgfqpoint{1.153215in}{1.180355in}}%
\pgfpathlineto{\pgfqpoint{1.113695in}{1.170987in}}%
\pgfpathclose%
\pgfusepath{fill}%
\end{pgfscope}%
\begin{pgfscope}%
\pgfpathrectangle{\pgfqpoint{0.150000in}{0.150000in}}{\pgfqpoint{2.700000in}{1.950000in}}%
\pgfusepath{clip}%
\pgfsetbuttcap%
\pgfsetroundjoin%
\definecolor{currentfill}{rgb}{0.598851,0.648254,0.717417}%
\pgfsetfillcolor{currentfill}%
\pgfsetlinewidth{0.000000pt}%
\definecolor{currentstroke}{rgb}{0.000000,0.000000,0.000000}%
\pgfsetstrokecolor{currentstroke}%
\pgfsetdash{}{0pt}%
\pgfpathmoveto{\pgfqpoint{1.113695in}{1.170987in}}%
\pgfpathlineto{\pgfqpoint{1.153215in}{1.180355in}}%
\pgfpathlineto{\pgfqpoint{1.114859in}{1.220232in}}%
\pgfpathlineto{\pgfqpoint{1.075228in}{1.210976in}}%
\pgfpathclose%
\pgfusepath{fill}%
\end{pgfscope}%
\begin{pgfscope}%
\pgfpathrectangle{\pgfqpoint{0.150000in}{0.150000in}}{\pgfqpoint{2.700000in}{1.950000in}}%
\pgfusepath{clip}%
\pgfsetbuttcap%
\pgfsetroundjoin%
\definecolor{currentfill}{rgb}{0.567754,0.620987,0.695512}%
\pgfsetfillcolor{currentfill}%
\pgfsetlinewidth{0.000000pt}%
\definecolor{currentstroke}{rgb}{0.000000,0.000000,0.000000}%
\pgfsetstrokecolor{currentstroke}%
\pgfsetdash{}{0pt}%
\pgfpathmoveto{\pgfqpoint{1.075228in}{1.210976in}}%
\pgfpathlineto{\pgfqpoint{1.114859in}{1.220232in}}%
\pgfpathlineto{\pgfqpoint{1.076497in}{1.260115in}}%
\pgfpathlineto{\pgfqpoint{1.036755in}{1.250971in}}%
\pgfpathclose%
\pgfusepath{fill}%
\end{pgfscope}%
\begin{pgfscope}%
\pgfpathrectangle{\pgfqpoint{0.150000in}{0.150000in}}{\pgfqpoint{2.700000in}{1.950000in}}%
\pgfusepath{clip}%
\pgfsetbuttcap%
\pgfsetroundjoin%
\definecolor{currentfill}{rgb}{0.536657,0.593719,0.673606}%
\pgfsetfillcolor{currentfill}%
\pgfsetlinewidth{0.000000pt}%
\definecolor{currentstroke}{rgb}{0.000000,0.000000,0.000000}%
\pgfsetstrokecolor{currentstroke}%
\pgfsetdash{}{0pt}%
\pgfpathmoveto{\pgfqpoint{1.036755in}{1.250971in}}%
\pgfpathlineto{\pgfqpoint{1.076497in}{1.260115in}}%
\pgfpathlineto{\pgfqpoint{1.038130in}{1.300003in}}%
\pgfpathlineto{\pgfqpoint{0.998277in}{1.290971in}}%
\pgfpathclose%
\pgfusepath{fill}%
\end{pgfscope}%
\begin{pgfscope}%
\pgfpathrectangle{\pgfqpoint{0.150000in}{0.150000in}}{\pgfqpoint{2.700000in}{1.950000in}}%
\pgfusepath{clip}%
\pgfsetbuttcap%
\pgfsetroundjoin%
\definecolor{currentfill}{rgb}{0.505561,0.566452,0.651700}%
\pgfsetfillcolor{currentfill}%
\pgfsetlinewidth{0.000000pt}%
\definecolor{currentstroke}{rgb}{0.000000,0.000000,0.000000}%
\pgfsetstrokecolor{currentstroke}%
\pgfsetdash{}{0pt}%
\pgfpathmoveto{\pgfqpoint{0.998277in}{1.290971in}}%
\pgfpathlineto{\pgfqpoint{1.038130in}{1.300003in}}%
\pgfpathlineto{\pgfqpoint{0.999758in}{1.339897in}}%
\pgfpathlineto{\pgfqpoint{0.959794in}{1.330977in}}%
\pgfpathclose%
\pgfusepath{fill}%
\end{pgfscope}%
\begin{pgfscope}%
\pgfpathrectangle{\pgfqpoint{0.150000in}{0.150000in}}{\pgfqpoint{2.700000in}{1.950000in}}%
\pgfusepath{clip}%
\pgfsetbuttcap%
\pgfsetroundjoin%
\definecolor{currentfill}{rgb}{0.474464,0.539185,0.629795}%
\pgfsetfillcolor{currentfill}%
\pgfsetlinewidth{0.000000pt}%
\definecolor{currentstroke}{rgb}{0.000000,0.000000,0.000000}%
\pgfsetstrokecolor{currentstroke}%
\pgfsetdash{}{0pt}%
\pgfpathmoveto{\pgfqpoint{0.959794in}{1.330977in}}%
\pgfpathlineto{\pgfqpoint{0.999758in}{1.339897in}}%
\pgfpathlineto{\pgfqpoint{0.961381in}{1.379796in}}%
\pgfpathlineto{\pgfqpoint{0.921305in}{1.370989in}}%
\pgfpathclose%
\pgfusepath{fill}%
\end{pgfscope}%
\begin{pgfscope}%
\pgfpathrectangle{\pgfqpoint{0.150000in}{0.150000in}}{\pgfqpoint{2.700000in}{1.950000in}}%
\pgfusepath{clip}%
\pgfsetbuttcap%
\pgfsetroundjoin%
\definecolor{currentfill}{rgb}{0.449586,0.517371,0.612270}%
\pgfsetfillcolor{currentfill}%
\pgfsetlinewidth{0.000000pt}%
\definecolor{currentstroke}{rgb}{0.000000,0.000000,0.000000}%
\pgfsetstrokecolor{currentstroke}%
\pgfsetdash{}{0pt}%
\pgfpathmoveto{\pgfqpoint{0.921305in}{1.370989in}}%
\pgfpathlineto{\pgfqpoint{0.961381in}{1.379796in}}%
\pgfpathlineto{\pgfqpoint{0.922998in}{1.419701in}}%
\pgfpathlineto{\pgfqpoint{0.882811in}{1.411006in}}%
\pgfpathclose%
\pgfusepath{fill}%
\end{pgfscope}%
\begin{pgfscope}%
\pgfpathrectangle{\pgfqpoint{0.150000in}{0.150000in}}{\pgfqpoint{2.700000in}{1.950000in}}%
\pgfusepath{clip}%
\pgfsetbuttcap%
\pgfsetroundjoin%
\definecolor{currentfill}{rgb}{0.418490,0.490104,0.590365}%
\pgfsetfillcolor{currentfill}%
\pgfsetlinewidth{0.000000pt}%
\definecolor{currentstroke}{rgb}{0.000000,0.000000,0.000000}%
\pgfsetstrokecolor{currentstroke}%
\pgfsetdash{}{0pt}%
\pgfpathmoveto{\pgfqpoint{0.882811in}{1.411006in}}%
\pgfpathlineto{\pgfqpoint{0.922998in}{1.419701in}}%
\pgfpathlineto{\pgfqpoint{0.884610in}{1.459611in}}%
\pgfpathlineto{\pgfqpoint{0.844312in}{1.451028in}}%
\pgfpathclose%
\pgfusepath{fill}%
\end{pgfscope}%
\begin{pgfscope}%
\pgfpathrectangle{\pgfqpoint{0.150000in}{0.150000in}}{\pgfqpoint{2.700000in}{1.950000in}}%
\pgfusepath{clip}%
\pgfsetbuttcap%
\pgfsetroundjoin%
\definecolor{currentfill}{rgb}{0.387393,0.462837,0.568459}%
\pgfsetfillcolor{currentfill}%
\pgfsetlinewidth{0.000000pt}%
\definecolor{currentstroke}{rgb}{0.000000,0.000000,0.000000}%
\pgfsetstrokecolor{currentstroke}%
\pgfsetdash{}{0pt}%
\pgfpathmoveto{\pgfqpoint{0.844312in}{1.451028in}}%
\pgfpathlineto{\pgfqpoint{0.884610in}{1.459611in}}%
\pgfpathlineto{\pgfqpoint{0.846217in}{1.499526in}}%
\pgfpathlineto{\pgfqpoint{0.805807in}{1.491056in}}%
\pgfpathclose%
\pgfusepath{fill}%
\end{pgfscope}%
\begin{pgfscope}%
\pgfpathrectangle{\pgfqpoint{0.150000in}{0.150000in}}{\pgfqpoint{2.700000in}{1.950000in}}%
\pgfusepath{clip}%
\pgfsetbuttcap%
\pgfsetroundjoin%
\definecolor{currentfill}{rgb}{0.356296,0.435570,0.546553}%
\pgfsetfillcolor{currentfill}%
\pgfsetlinewidth{0.000000pt}%
\definecolor{currentstroke}{rgb}{0.000000,0.000000,0.000000}%
\pgfsetstrokecolor{currentstroke}%
\pgfsetdash{}{0pt}%
\pgfpathmoveto{\pgfqpoint{0.805807in}{1.491056in}}%
\pgfpathlineto{\pgfqpoint{0.846217in}{1.499526in}}%
\pgfpathlineto{\pgfqpoint{0.807819in}{1.539448in}}%
\pgfpathlineto{\pgfqpoint{0.767297in}{1.531090in}}%
\pgfpathclose%
\pgfusepath{fill}%
\end{pgfscope}%
\begin{pgfscope}%
\pgfpathrectangle{\pgfqpoint{0.150000in}{0.150000in}}{\pgfqpoint{2.700000in}{1.950000in}}%
\pgfusepath{clip}%
\pgfsetbuttcap%
\pgfsetroundjoin%
\definecolor{currentfill}{rgb}{0.325199,0.408303,0.524648}%
\pgfsetfillcolor{currentfill}%
\pgfsetlinewidth{0.000000pt}%
\definecolor{currentstroke}{rgb}{0.000000,0.000000,0.000000}%
\pgfsetstrokecolor{currentstroke}%
\pgfsetdash{}{0pt}%
\pgfpathmoveto{\pgfqpoint{0.767297in}{1.531090in}}%
\pgfpathlineto{\pgfqpoint{0.807819in}{1.539448in}}%
\pgfpathlineto{\pgfqpoint{0.769415in}{1.579374in}}%
\pgfpathlineto{\pgfqpoint{0.728782in}{1.571129in}}%
\pgfpathclose%
\pgfusepath{fill}%
\end{pgfscope}%
\begin{pgfscope}%
\pgfpathrectangle{\pgfqpoint{0.150000in}{0.150000in}}{\pgfqpoint{2.700000in}{1.950000in}}%
\pgfusepath{clip}%
\pgfsetbuttcap%
\pgfsetroundjoin%
\definecolor{currentfill}{rgb}{0.975306,0.955193,0.956786}%
\pgfsetfillcolor{currentfill}%
\pgfsetlinewidth{0.000000pt}%
\definecolor{currentstroke}{rgb}{0.000000,0.000000,0.000000}%
\pgfsetstrokecolor{currentstroke}%
\pgfsetdash{}{0pt}%
\pgfpathmoveto{\pgfqpoint{1.652000in}{0.560701in}}%
\pgfpathlineto{\pgfqpoint{1.690070in}{0.571810in}}%
\pgfpathlineto{\pgfqpoint{1.651682in}{0.611716in}}%
\pgfpathlineto{\pgfqpoint{1.613501in}{0.600719in}}%
\pgfpathclose%
\pgfusepath{fill}%
\end{pgfscope}%
\begin{pgfscope}%
\pgfpathrectangle{\pgfqpoint{0.150000in}{0.150000in}}{\pgfqpoint{2.700000in}{1.950000in}}%
\pgfusepath{clip}%
\pgfsetbuttcap%
\pgfsetroundjoin%
\definecolor{currentfill}{rgb}{0.994301,0.989660,0.990028}%
\pgfsetfillcolor{currentfill}%
\pgfsetlinewidth{0.000000pt}%
\definecolor{currentstroke}{rgb}{0.000000,0.000000,0.000000}%
\pgfsetstrokecolor{currentstroke}%
\pgfsetdash{}{0pt}%
\pgfpathmoveto{\pgfqpoint{1.613501in}{0.600719in}}%
\pgfpathlineto{\pgfqpoint{1.651682in}{0.611716in}}%
\pgfpathlineto{\pgfqpoint{1.613289in}{0.651628in}}%
\pgfpathlineto{\pgfqpoint{1.574996in}{0.640744in}}%
\pgfpathclose%
\pgfusepath{fill}%
\end{pgfscope}%
\begin{pgfscope}%
\pgfpathrectangle{\pgfqpoint{0.150000in}{0.150000in}}{\pgfqpoint{2.700000in}{1.950000in}}%
\pgfusepath{clip}%
\pgfsetbuttcap%
\pgfsetroundjoin%
\definecolor{currentfill}{rgb}{0.978232,0.980913,0.984666}%
\pgfsetfillcolor{currentfill}%
\pgfsetlinewidth{0.000000pt}%
\definecolor{currentstroke}{rgb}{0.000000,0.000000,0.000000}%
\pgfsetstrokecolor{currentstroke}%
\pgfsetdash{}{0pt}%
\pgfpathmoveto{\pgfqpoint{1.574996in}{0.640744in}}%
\pgfpathlineto{\pgfqpoint{1.613289in}{0.651628in}}%
\pgfpathlineto{\pgfqpoint{1.574890in}{0.691546in}}%
\pgfpathlineto{\pgfqpoint{1.536486in}{0.680774in}}%
\pgfpathclose%
\pgfusepath{fill}%
\end{pgfscope}%
\begin{pgfscope}%
\pgfpathrectangle{\pgfqpoint{0.150000in}{0.150000in}}{\pgfqpoint{2.700000in}{1.950000in}}%
\pgfusepath{clip}%
\pgfsetbuttcap%
\pgfsetroundjoin%
\definecolor{currentfill}{rgb}{0.947135,0.953646,0.962760}%
\pgfsetfillcolor{currentfill}%
\pgfsetlinewidth{0.000000pt}%
\definecolor{currentstroke}{rgb}{0.000000,0.000000,0.000000}%
\pgfsetstrokecolor{currentstroke}%
\pgfsetdash{}{0pt}%
\pgfpathmoveto{\pgfqpoint{1.536486in}{0.680774in}}%
\pgfpathlineto{\pgfqpoint{1.574890in}{0.691546in}}%
\pgfpathlineto{\pgfqpoint{1.536486in}{0.731469in}}%
\pgfpathlineto{\pgfqpoint{1.497971in}{0.720809in}}%
\pgfpathclose%
\pgfusepath{fill}%
\end{pgfscope}%
\begin{pgfscope}%
\pgfpathrectangle{\pgfqpoint{0.150000in}{0.150000in}}{\pgfqpoint{2.700000in}{1.950000in}}%
\pgfusepath{clip}%
\pgfsetbuttcap%
\pgfsetroundjoin%
\definecolor{currentfill}{rgb}{0.916039,0.926379,0.940855}%
\pgfsetfillcolor{currentfill}%
\pgfsetlinewidth{0.000000pt}%
\definecolor{currentstroke}{rgb}{0.000000,0.000000,0.000000}%
\pgfsetstrokecolor{currentstroke}%
\pgfsetdash{}{0pt}%
\pgfpathmoveto{\pgfqpoint{1.497971in}{0.720809in}}%
\pgfpathlineto{\pgfqpoint{1.536486in}{0.731469in}}%
\pgfpathlineto{\pgfqpoint{1.498077in}{0.771398in}}%
\pgfpathlineto{\pgfqpoint{1.459451in}{0.760850in}}%
\pgfpathclose%
\pgfusepath{fill}%
\end{pgfscope}%
\begin{pgfscope}%
\pgfpathrectangle{\pgfqpoint{0.150000in}{0.150000in}}{\pgfqpoint{2.700000in}{1.950000in}}%
\pgfusepath{clip}%
\pgfsetbuttcap%
\pgfsetroundjoin%
\definecolor{currentfill}{rgb}{0.884942,0.899112,0.918949}%
\pgfsetfillcolor{currentfill}%
\pgfsetlinewidth{0.000000pt}%
\definecolor{currentstroke}{rgb}{0.000000,0.000000,0.000000}%
\pgfsetstrokecolor{currentstroke}%
\pgfsetdash{}{0pt}%
\pgfpathmoveto{\pgfqpoint{1.459451in}{0.760850in}}%
\pgfpathlineto{\pgfqpoint{1.498077in}{0.771398in}}%
\pgfpathlineto{\pgfqpoint{1.459663in}{0.811332in}}%
\pgfpathlineto{\pgfqpoint{1.420925in}{0.800897in}}%
\pgfpathclose%
\pgfusepath{fill}%
\end{pgfscope}%
\begin{pgfscope}%
\pgfpathrectangle{\pgfqpoint{0.150000in}{0.150000in}}{\pgfqpoint{2.700000in}{1.950000in}}%
\pgfusepath{clip}%
\pgfsetbuttcap%
\pgfsetroundjoin%
\definecolor{currentfill}{rgb}{0.853845,0.871844,0.897044}%
\pgfsetfillcolor{currentfill}%
\pgfsetlinewidth{0.000000pt}%
\definecolor{currentstroke}{rgb}{0.000000,0.000000,0.000000}%
\pgfsetstrokecolor{currentstroke}%
\pgfsetdash{}{0pt}%
\pgfpathmoveto{\pgfqpoint{1.420925in}{0.800897in}}%
\pgfpathlineto{\pgfqpoint{1.459663in}{0.811332in}}%
\pgfpathlineto{\pgfqpoint{1.421244in}{0.851271in}}%
\pgfpathlineto{\pgfqpoint{1.382394in}{0.840949in}}%
\pgfpathclose%
\pgfusepath{fill}%
\end{pgfscope}%
\begin{pgfscope}%
\pgfpathrectangle{\pgfqpoint{0.150000in}{0.150000in}}{\pgfqpoint{2.700000in}{1.950000in}}%
\pgfusepath{clip}%
\pgfsetbuttcap%
\pgfsetroundjoin%
\definecolor{currentfill}{rgb}{0.822748,0.844577,0.875138}%
\pgfsetfillcolor{currentfill}%
\pgfsetlinewidth{0.000000pt}%
\definecolor{currentstroke}{rgb}{0.000000,0.000000,0.000000}%
\pgfsetstrokecolor{currentstroke}%
\pgfsetdash{}{0pt}%
\pgfpathmoveto{\pgfqpoint{1.382394in}{0.840949in}}%
\pgfpathlineto{\pgfqpoint{1.421244in}{0.851271in}}%
\pgfpathlineto{\pgfqpoint{1.382819in}{0.891216in}}%
\pgfpathlineto{\pgfqpoint{1.343857in}{0.881007in}}%
\pgfpathclose%
\pgfusepath{fill}%
\end{pgfscope}%
\begin{pgfscope}%
\pgfpathrectangle{\pgfqpoint{0.150000in}{0.150000in}}{\pgfqpoint{2.700000in}{1.950000in}}%
\pgfusepath{clip}%
\pgfsetbuttcap%
\pgfsetroundjoin%
\definecolor{currentfill}{rgb}{0.797871,0.822763,0.857613}%
\pgfsetfillcolor{currentfill}%
\pgfsetlinewidth{0.000000pt}%
\definecolor{currentstroke}{rgb}{0.000000,0.000000,0.000000}%
\pgfsetstrokecolor{currentstroke}%
\pgfsetdash{}{0pt}%
\pgfpathmoveto{\pgfqpoint{1.343857in}{0.881007in}}%
\pgfpathlineto{\pgfqpoint{1.382819in}{0.891216in}}%
\pgfpathlineto{\pgfqpoint{1.344388in}{0.931167in}}%
\pgfpathlineto{\pgfqpoint{1.305315in}{0.921070in}}%
\pgfpathclose%
\pgfusepath{fill}%
\end{pgfscope}%
\begin{pgfscope}%
\pgfpathrectangle{\pgfqpoint{0.150000in}{0.150000in}}{\pgfqpoint{2.700000in}{1.950000in}}%
\pgfusepath{clip}%
\pgfsetbuttcap%
\pgfsetroundjoin%
\definecolor{currentfill}{rgb}{0.766774,0.795496,0.835708}%
\pgfsetfillcolor{currentfill}%
\pgfsetlinewidth{0.000000pt}%
\definecolor{currentstroke}{rgb}{0.000000,0.000000,0.000000}%
\pgfsetstrokecolor{currentstroke}%
\pgfsetdash{}{0pt}%
\pgfpathmoveto{\pgfqpoint{1.305315in}{0.921070in}}%
\pgfpathlineto{\pgfqpoint{1.344388in}{0.931167in}}%
\pgfpathlineto{\pgfqpoint{1.305953in}{0.971123in}}%
\pgfpathlineto{\pgfqpoint{1.266768in}{0.961139in}}%
\pgfpathclose%
\pgfusepath{fill}%
\end{pgfscope}%
\begin{pgfscope}%
\pgfpathrectangle{\pgfqpoint{0.150000in}{0.150000in}}{\pgfqpoint{2.700000in}{1.950000in}}%
\pgfusepath{clip}%
\pgfsetbuttcap%
\pgfsetroundjoin%
\definecolor{currentfill}{rgb}{0.735677,0.768229,0.813802}%
\pgfsetfillcolor{currentfill}%
\pgfsetlinewidth{0.000000pt}%
\definecolor{currentstroke}{rgb}{0.000000,0.000000,0.000000}%
\pgfsetstrokecolor{currentstroke}%
\pgfsetdash{}{0pt}%
\pgfpathmoveto{\pgfqpoint{1.266768in}{0.961139in}}%
\pgfpathlineto{\pgfqpoint{1.305953in}{0.971123in}}%
\pgfpathlineto{\pgfqpoint{1.267512in}{1.011085in}}%
\pgfpathlineto{\pgfqpoint{1.228216in}{1.001213in}}%
\pgfpathclose%
\pgfusepath{fill}%
\end{pgfscope}%
\begin{pgfscope}%
\pgfpathrectangle{\pgfqpoint{0.150000in}{0.150000in}}{\pgfqpoint{2.700000in}{1.950000in}}%
\pgfusepath{clip}%
\pgfsetbuttcap%
\pgfsetroundjoin%
\definecolor{currentfill}{rgb}{0.704580,0.740962,0.791896}%
\pgfsetfillcolor{currentfill}%
\pgfsetlinewidth{0.000000pt}%
\definecolor{currentstroke}{rgb}{0.000000,0.000000,0.000000}%
\pgfsetstrokecolor{currentstroke}%
\pgfsetdash{}{0pt}%
\pgfpathmoveto{\pgfqpoint{1.228216in}{1.001213in}}%
\pgfpathlineto{\pgfqpoint{1.267512in}{1.011085in}}%
\pgfpathlineto{\pgfqpoint{1.229066in}{1.051052in}}%
\pgfpathlineto{\pgfqpoint{1.189658in}{1.041293in}}%
\pgfpathclose%
\pgfusepath{fill}%
\end{pgfscope}%
\begin{pgfscope}%
\pgfpathrectangle{\pgfqpoint{0.150000in}{0.150000in}}{\pgfqpoint{2.700000in}{1.950000in}}%
\pgfusepath{clip}%
\pgfsetbuttcap%
\pgfsetroundjoin%
\definecolor{currentfill}{rgb}{0.673483,0.713695,0.769991}%
\pgfsetfillcolor{currentfill}%
\pgfsetlinewidth{0.000000pt}%
\definecolor{currentstroke}{rgb}{0.000000,0.000000,0.000000}%
\pgfsetstrokecolor{currentstroke}%
\pgfsetdash{}{0pt}%
\pgfpathmoveto{\pgfqpoint{1.189658in}{1.041293in}}%
\pgfpathlineto{\pgfqpoint{1.229066in}{1.051052in}}%
\pgfpathlineto{\pgfqpoint{1.190614in}{1.091025in}}%
\pgfpathlineto{\pgfqpoint{1.151094in}{1.081378in}}%
\pgfpathclose%
\pgfusepath{fill}%
\end{pgfscope}%
\begin{pgfscope}%
\pgfpathrectangle{\pgfqpoint{0.150000in}{0.150000in}}{\pgfqpoint{2.700000in}{1.950000in}}%
\pgfusepath{clip}%
\pgfsetbuttcap%
\pgfsetroundjoin%
\definecolor{currentfill}{rgb}{0.642387,0.686428,0.748085}%
\pgfsetfillcolor{currentfill}%
\pgfsetlinewidth{0.000000pt}%
\definecolor{currentstroke}{rgb}{0.000000,0.000000,0.000000}%
\pgfsetstrokecolor{currentstroke}%
\pgfsetdash{}{0pt}%
\pgfpathmoveto{\pgfqpoint{1.151094in}{1.081378in}}%
\pgfpathlineto{\pgfqpoint{1.190614in}{1.091025in}}%
\pgfpathlineto{\pgfqpoint{1.152158in}{1.131003in}}%
\pgfpathlineto{\pgfqpoint{1.112526in}{1.121469in}}%
\pgfpathclose%
\pgfusepath{fill}%
\end{pgfscope}%
\begin{pgfscope}%
\pgfpathrectangle{\pgfqpoint{0.150000in}{0.150000in}}{\pgfqpoint{2.700000in}{1.950000in}}%
\pgfusepath{clip}%
\pgfsetbuttcap%
\pgfsetroundjoin%
\definecolor{currentfill}{rgb}{0.611290,0.659161,0.726180}%
\pgfsetfillcolor{currentfill}%
\pgfsetlinewidth{0.000000pt}%
\definecolor{currentstroke}{rgb}{0.000000,0.000000,0.000000}%
\pgfsetstrokecolor{currentstroke}%
\pgfsetdash{}{0pt}%
\pgfpathmoveto{\pgfqpoint{1.112526in}{1.121469in}}%
\pgfpathlineto{\pgfqpoint{1.152158in}{1.131003in}}%
\pgfpathlineto{\pgfqpoint{1.113695in}{1.170987in}}%
\pgfpathlineto{\pgfqpoint{1.073952in}{1.161566in}}%
\pgfpathclose%
\pgfusepath{fill}%
\end{pgfscope}%
\begin{pgfscope}%
\pgfpathrectangle{\pgfqpoint{0.150000in}{0.150000in}}{\pgfqpoint{2.700000in}{1.950000in}}%
\pgfusepath{clip}%
\pgfsetbuttcap%
\pgfsetroundjoin%
\definecolor{currentfill}{rgb}{0.580193,0.631893,0.704274}%
\pgfsetfillcolor{currentfill}%
\pgfsetlinewidth{0.000000pt}%
\definecolor{currentstroke}{rgb}{0.000000,0.000000,0.000000}%
\pgfsetstrokecolor{currentstroke}%
\pgfsetdash{}{0pt}%
\pgfpathmoveto{\pgfqpoint{1.073952in}{1.161566in}}%
\pgfpathlineto{\pgfqpoint{1.113695in}{1.170987in}}%
\pgfpathlineto{\pgfqpoint{1.075228in}{1.210976in}}%
\pgfpathlineto{\pgfqpoint{1.035373in}{1.201668in}}%
\pgfpathclose%
\pgfusepath{fill}%
\end{pgfscope}%
\begin{pgfscope}%
\pgfpathrectangle{\pgfqpoint{0.150000in}{0.150000in}}{\pgfqpoint{2.700000in}{1.950000in}}%
\pgfusepath{clip}%
\pgfsetbuttcap%
\pgfsetroundjoin%
\definecolor{currentfill}{rgb}{0.555316,0.610080,0.686749}%
\pgfsetfillcolor{currentfill}%
\pgfsetlinewidth{0.000000pt}%
\definecolor{currentstroke}{rgb}{0.000000,0.000000,0.000000}%
\pgfsetstrokecolor{currentstroke}%
\pgfsetdash{}{0pt}%
\pgfpathmoveto{\pgfqpoint{1.035373in}{1.201668in}}%
\pgfpathlineto{\pgfqpoint{1.075228in}{1.210976in}}%
\pgfpathlineto{\pgfqpoint{1.036755in}{1.250971in}}%
\pgfpathlineto{\pgfqpoint{0.996788in}{1.241775in}}%
\pgfpathclose%
\pgfusepath{fill}%
\end{pgfscope}%
\begin{pgfscope}%
\pgfpathrectangle{\pgfqpoint{0.150000in}{0.150000in}}{\pgfqpoint{2.700000in}{1.950000in}}%
\pgfusepath{clip}%
\pgfsetbuttcap%
\pgfsetroundjoin%
\definecolor{currentfill}{rgb}{0.524219,0.582812,0.664844}%
\pgfsetfillcolor{currentfill}%
\pgfsetlinewidth{0.000000pt}%
\definecolor{currentstroke}{rgb}{0.000000,0.000000,0.000000}%
\pgfsetstrokecolor{currentstroke}%
\pgfsetdash{}{0pt}%
\pgfpathmoveto{\pgfqpoint{0.996788in}{1.241775in}}%
\pgfpathlineto{\pgfqpoint{1.036755in}{1.250971in}}%
\pgfpathlineto{\pgfqpoint{0.998277in}{1.290971in}}%
\pgfpathlineto{\pgfqpoint{0.958198in}{1.281888in}}%
\pgfpathclose%
\pgfusepath{fill}%
\end{pgfscope}%
\begin{pgfscope}%
\pgfpathrectangle{\pgfqpoint{0.150000in}{0.150000in}}{\pgfqpoint{2.700000in}{1.950000in}}%
\pgfusepath{clip}%
\pgfsetbuttcap%
\pgfsetroundjoin%
\definecolor{currentfill}{rgb}{0.493122,0.555545,0.642938}%
\pgfsetfillcolor{currentfill}%
\pgfsetlinewidth{0.000000pt}%
\definecolor{currentstroke}{rgb}{0.000000,0.000000,0.000000}%
\pgfsetstrokecolor{currentstroke}%
\pgfsetdash{}{0pt}%
\pgfpathmoveto{\pgfqpoint{0.958198in}{1.281888in}}%
\pgfpathlineto{\pgfqpoint{0.998277in}{1.290971in}}%
\pgfpathlineto{\pgfqpoint{0.959794in}{1.330977in}}%
\pgfpathlineto{\pgfqpoint{0.919603in}{1.322007in}}%
\pgfpathclose%
\pgfusepath{fill}%
\end{pgfscope}%
\begin{pgfscope}%
\pgfpathrectangle{\pgfqpoint{0.150000in}{0.150000in}}{\pgfqpoint{2.700000in}{1.950000in}}%
\pgfusepath{clip}%
\pgfsetbuttcap%
\pgfsetroundjoin%
\definecolor{currentfill}{rgb}{0.462025,0.528278,0.621032}%
\pgfsetfillcolor{currentfill}%
\pgfsetlinewidth{0.000000pt}%
\definecolor{currentstroke}{rgb}{0.000000,0.000000,0.000000}%
\pgfsetstrokecolor{currentstroke}%
\pgfsetdash{}{0pt}%
\pgfpathmoveto{\pgfqpoint{0.919603in}{1.322007in}}%
\pgfpathlineto{\pgfqpoint{0.959794in}{1.330977in}}%
\pgfpathlineto{\pgfqpoint{0.921305in}{1.370989in}}%
\pgfpathlineto{\pgfqpoint{0.881002in}{1.362131in}}%
\pgfpathclose%
\pgfusepath{fill}%
\end{pgfscope}%
\begin{pgfscope}%
\pgfpathrectangle{\pgfqpoint{0.150000in}{0.150000in}}{\pgfqpoint{2.700000in}{1.950000in}}%
\pgfusepath{clip}%
\pgfsetbuttcap%
\pgfsetroundjoin%
\definecolor{currentfill}{rgb}{0.430928,0.501011,0.599127}%
\pgfsetfillcolor{currentfill}%
\pgfsetlinewidth{0.000000pt}%
\definecolor{currentstroke}{rgb}{0.000000,0.000000,0.000000}%
\pgfsetstrokecolor{currentstroke}%
\pgfsetdash{}{0pt}%
\pgfpathmoveto{\pgfqpoint{0.881002in}{1.362131in}}%
\pgfpathlineto{\pgfqpoint{0.921305in}{1.370989in}}%
\pgfpathlineto{\pgfqpoint{0.882811in}{1.411006in}}%
\pgfpathlineto{\pgfqpoint{0.842396in}{1.402261in}}%
\pgfpathclose%
\pgfusepath{fill}%
\end{pgfscope}%
\begin{pgfscope}%
\pgfpathrectangle{\pgfqpoint{0.150000in}{0.150000in}}{\pgfqpoint{2.700000in}{1.950000in}}%
\pgfusepath{clip}%
\pgfsetbuttcap%
\pgfsetroundjoin%
\definecolor{currentfill}{rgb}{0.399831,0.473744,0.577221}%
\pgfsetfillcolor{currentfill}%
\pgfsetlinewidth{0.000000pt}%
\definecolor{currentstroke}{rgb}{0.000000,0.000000,0.000000}%
\pgfsetstrokecolor{currentstroke}%
\pgfsetdash{}{0pt}%
\pgfpathmoveto{\pgfqpoint{0.842396in}{1.402261in}}%
\pgfpathlineto{\pgfqpoint{0.882811in}{1.411006in}}%
\pgfpathlineto{\pgfqpoint{0.844312in}{1.451028in}}%
\pgfpathlineto{\pgfqpoint{0.803785in}{1.442397in}}%
\pgfpathclose%
\pgfusepath{fill}%
\end{pgfscope}%
\begin{pgfscope}%
\pgfpathrectangle{\pgfqpoint{0.150000in}{0.150000in}}{\pgfqpoint{2.700000in}{1.950000in}}%
\pgfusepath{clip}%
\pgfsetbuttcap%
\pgfsetroundjoin%
\definecolor{currentfill}{rgb}{0.368735,0.446477,0.555316}%
\pgfsetfillcolor{currentfill}%
\pgfsetlinewidth{0.000000pt}%
\definecolor{currentstroke}{rgb}{0.000000,0.000000,0.000000}%
\pgfsetstrokecolor{currentstroke}%
\pgfsetdash{}{0pt}%
\pgfpathmoveto{\pgfqpoint{0.803785in}{1.442397in}}%
\pgfpathlineto{\pgfqpoint{0.844312in}{1.451028in}}%
\pgfpathlineto{\pgfqpoint{0.805807in}{1.491056in}}%
\pgfpathlineto{\pgfqpoint{0.765168in}{1.482538in}}%
\pgfpathclose%
\pgfusepath{fill}%
\end{pgfscope}%
\begin{pgfscope}%
\pgfpathrectangle{\pgfqpoint{0.150000in}{0.150000in}}{\pgfqpoint{2.700000in}{1.950000in}}%
\pgfusepath{clip}%
\pgfsetbuttcap%
\pgfsetroundjoin%
\definecolor{currentfill}{rgb}{0.337638,0.419210,0.533410}%
\pgfsetfillcolor{currentfill}%
\pgfsetlinewidth{0.000000pt}%
\definecolor{currentstroke}{rgb}{0.000000,0.000000,0.000000}%
\pgfsetstrokecolor{currentstroke}%
\pgfsetdash{}{0pt}%
\pgfpathmoveto{\pgfqpoint{0.765168in}{1.482538in}}%
\pgfpathlineto{\pgfqpoint{0.805807in}{1.491056in}}%
\pgfpathlineto{\pgfqpoint{0.767297in}{1.531090in}}%
\pgfpathlineto{\pgfqpoint{0.726546in}{1.522684in}}%
\pgfpathclose%
\pgfusepath{fill}%
\end{pgfscope}%
\begin{pgfscope}%
\pgfpathrectangle{\pgfqpoint{0.150000in}{0.150000in}}{\pgfqpoint{2.700000in}{1.950000in}}%
\pgfusepath{clip}%
\pgfsetbuttcap%
\pgfsetroundjoin%
\definecolor{currentfill}{rgb}{0.312760,0.397396,0.515885}%
\pgfsetfillcolor{currentfill}%
\pgfsetlinewidth{0.000000pt}%
\definecolor{currentstroke}{rgb}{0.000000,0.000000,0.000000}%
\pgfsetstrokecolor{currentstroke}%
\pgfsetdash{}{0pt}%
\pgfpathmoveto{\pgfqpoint{0.726546in}{1.522684in}}%
\pgfpathlineto{\pgfqpoint{0.767297in}{1.531090in}}%
\pgfpathlineto{\pgfqpoint{0.728782in}{1.571129in}}%
\pgfpathlineto{\pgfqpoint{0.687919in}{1.562836in}}%
\pgfpathclose%
\pgfusepath{fill}%
\end{pgfscope}%
\begin{pgfscope}%
\pgfpathrectangle{\pgfqpoint{0.150000in}{0.150000in}}{\pgfqpoint{2.700000in}{1.950000in}}%
\pgfusepath{clip}%
\pgfsetbuttcap%
\pgfsetroundjoin%
\definecolor{currentfill}{rgb}{0.986703,0.975873,0.976731}%
\pgfsetfillcolor{currentfill}%
\pgfsetlinewidth{0.000000pt}%
\definecolor{currentstroke}{rgb}{0.000000,0.000000,0.000000}%
\pgfsetstrokecolor{currentstroke}%
\pgfsetdash{}{0pt}%
\pgfpathmoveto{\pgfqpoint{1.613714in}{0.549528in}}%
\pgfpathlineto{\pgfqpoint{1.652000in}{0.560701in}}%
\pgfpathlineto{\pgfqpoint{1.613501in}{0.600719in}}%
\pgfpathlineto{\pgfqpoint{1.575103in}{0.589660in}}%
\pgfpathclose%
\pgfusepath{fill}%
\end{pgfscope}%
\begin{pgfscope}%
\pgfpathrectangle{\pgfqpoint{0.150000in}{0.150000in}}{\pgfqpoint{2.700000in}{1.950000in}}%
\pgfusepath{clip}%
\pgfsetbuttcap%
\pgfsetroundjoin%
\definecolor{currentfill}{rgb}{0.990671,0.991820,0.993428}%
\pgfsetfillcolor{currentfill}%
\pgfsetlinewidth{0.000000pt}%
\definecolor{currentstroke}{rgb}{0.000000,0.000000,0.000000}%
\pgfsetstrokecolor{currentstroke}%
\pgfsetdash{}{0pt}%
\pgfpathmoveto{\pgfqpoint{1.575103in}{0.589660in}}%
\pgfpathlineto{\pgfqpoint{1.613501in}{0.600719in}}%
\pgfpathlineto{\pgfqpoint{1.574996in}{0.640744in}}%
\pgfpathlineto{\pgfqpoint{1.536486in}{0.629798in}}%
\pgfpathclose%
\pgfusepath{fill}%
\end{pgfscope}%
\begin{pgfscope}%
\pgfpathrectangle{\pgfqpoint{0.150000in}{0.150000in}}{\pgfqpoint{2.700000in}{1.950000in}}%
\pgfusepath{clip}%
\pgfsetbuttcap%
\pgfsetroundjoin%
\definecolor{currentfill}{rgb}{0.959574,0.964553,0.971523}%
\pgfsetfillcolor{currentfill}%
\pgfsetlinewidth{0.000000pt}%
\definecolor{currentstroke}{rgb}{0.000000,0.000000,0.000000}%
\pgfsetstrokecolor{currentstroke}%
\pgfsetdash{}{0pt}%
\pgfpathmoveto{\pgfqpoint{1.536486in}{0.629798in}}%
\pgfpathlineto{\pgfqpoint{1.574996in}{0.640744in}}%
\pgfpathlineto{\pgfqpoint{1.536486in}{0.680774in}}%
\pgfpathlineto{\pgfqpoint{1.497865in}{0.669940in}}%
\pgfpathclose%
\pgfusepath{fill}%
\end{pgfscope}%
\begin{pgfscope}%
\pgfpathrectangle{\pgfqpoint{0.150000in}{0.150000in}}{\pgfqpoint{2.700000in}{1.950000in}}%
\pgfusepath{clip}%
\pgfsetbuttcap%
\pgfsetroundjoin%
\definecolor{currentfill}{rgb}{0.934697,0.942739,0.953998}%
\pgfsetfillcolor{currentfill}%
\pgfsetlinewidth{0.000000pt}%
\definecolor{currentstroke}{rgb}{0.000000,0.000000,0.000000}%
\pgfsetstrokecolor{currentstroke}%
\pgfsetdash{}{0pt}%
\pgfpathmoveto{\pgfqpoint{1.497865in}{0.669940in}}%
\pgfpathlineto{\pgfqpoint{1.536486in}{0.680774in}}%
\pgfpathlineto{\pgfqpoint{1.497971in}{0.720809in}}%
\pgfpathlineto{\pgfqpoint{1.459237in}{0.710089in}}%
\pgfpathclose%
\pgfusepath{fill}%
\end{pgfscope}%
\begin{pgfscope}%
\pgfpathrectangle{\pgfqpoint{0.150000in}{0.150000in}}{\pgfqpoint{2.700000in}{1.950000in}}%
\pgfusepath{clip}%
\pgfsetbuttcap%
\pgfsetroundjoin%
\definecolor{currentfill}{rgb}{0.903600,0.915472,0.932093}%
\pgfsetfillcolor{currentfill}%
\pgfsetlinewidth{0.000000pt}%
\definecolor{currentstroke}{rgb}{0.000000,0.000000,0.000000}%
\pgfsetstrokecolor{currentstroke}%
\pgfsetdash{}{0pt}%
\pgfpathmoveto{\pgfqpoint{1.459237in}{0.710089in}}%
\pgfpathlineto{\pgfqpoint{1.497971in}{0.720809in}}%
\pgfpathlineto{\pgfqpoint{1.459451in}{0.760850in}}%
\pgfpathlineto{\pgfqpoint{1.420605in}{0.750243in}}%
\pgfpathclose%
\pgfusepath{fill}%
\end{pgfscope}%
\begin{pgfscope}%
\pgfpathrectangle{\pgfqpoint{0.150000in}{0.150000in}}{\pgfqpoint{2.700000in}{1.950000in}}%
\pgfusepath{clip}%
\pgfsetbuttcap%
\pgfsetroundjoin%
\definecolor{currentfill}{rgb}{0.872503,0.888205,0.910187}%
\pgfsetfillcolor{currentfill}%
\pgfsetlinewidth{0.000000pt}%
\definecolor{currentstroke}{rgb}{0.000000,0.000000,0.000000}%
\pgfsetstrokecolor{currentstroke}%
\pgfsetdash{}{0pt}%
\pgfpathmoveto{\pgfqpoint{1.420605in}{0.750243in}}%
\pgfpathlineto{\pgfqpoint{1.459451in}{0.760850in}}%
\pgfpathlineto{\pgfqpoint{1.420925in}{0.800897in}}%
\pgfpathlineto{\pgfqpoint{1.381966in}{0.790403in}}%
\pgfpathclose%
\pgfusepath{fill}%
\end{pgfscope}%
\begin{pgfscope}%
\pgfpathrectangle{\pgfqpoint{0.150000in}{0.150000in}}{\pgfqpoint{2.700000in}{1.950000in}}%
\pgfusepath{clip}%
\pgfsetbuttcap%
\pgfsetroundjoin%
\definecolor{currentfill}{rgb}{0.841406,0.860938,0.888281}%
\pgfsetfillcolor{currentfill}%
\pgfsetlinewidth{0.000000pt}%
\definecolor{currentstroke}{rgb}{0.000000,0.000000,0.000000}%
\pgfsetstrokecolor{currentstroke}%
\pgfsetdash{}{0pt}%
\pgfpathmoveto{\pgfqpoint{1.381966in}{0.790403in}}%
\pgfpathlineto{\pgfqpoint{1.420925in}{0.800897in}}%
\pgfpathlineto{\pgfqpoint{1.382394in}{0.840949in}}%
\pgfpathlineto{\pgfqpoint{1.343323in}{0.830568in}}%
\pgfpathclose%
\pgfusepath{fill}%
\end{pgfscope}%
\begin{pgfscope}%
\pgfpathrectangle{\pgfqpoint{0.150000in}{0.150000in}}{\pgfqpoint{2.700000in}{1.950000in}}%
\pgfusepath{clip}%
\pgfsetbuttcap%
\pgfsetroundjoin%
\definecolor{currentfill}{rgb}{0.810309,0.833670,0.866376}%
\pgfsetfillcolor{currentfill}%
\pgfsetlinewidth{0.000000pt}%
\definecolor{currentstroke}{rgb}{0.000000,0.000000,0.000000}%
\pgfsetstrokecolor{currentstroke}%
\pgfsetdash{}{0pt}%
\pgfpathmoveto{\pgfqpoint{1.343323in}{0.830568in}}%
\pgfpathlineto{\pgfqpoint{1.382394in}{0.840949in}}%
\pgfpathlineto{\pgfqpoint{1.343857in}{0.881007in}}%
\pgfpathlineto{\pgfqpoint{1.304674in}{0.870739in}}%
\pgfpathclose%
\pgfusepath{fill}%
\end{pgfscope}%
\begin{pgfscope}%
\pgfpathrectangle{\pgfqpoint{0.150000in}{0.150000in}}{\pgfqpoint{2.700000in}{1.950000in}}%
\pgfusepath{clip}%
\pgfsetbuttcap%
\pgfsetroundjoin%
\definecolor{currentfill}{rgb}{0.779213,0.806403,0.844470}%
\pgfsetfillcolor{currentfill}%
\pgfsetlinewidth{0.000000pt}%
\definecolor{currentstroke}{rgb}{0.000000,0.000000,0.000000}%
\pgfsetstrokecolor{currentstroke}%
\pgfsetdash{}{0pt}%
\pgfpathmoveto{\pgfqpoint{1.304674in}{0.870739in}}%
\pgfpathlineto{\pgfqpoint{1.343857in}{0.881007in}}%
\pgfpathlineto{\pgfqpoint{1.305315in}{0.921070in}}%
\pgfpathlineto{\pgfqpoint{1.266020in}{0.910915in}}%
\pgfpathclose%
\pgfusepath{fill}%
\end{pgfscope}%
\begin{pgfscope}%
\pgfpathrectangle{\pgfqpoint{0.150000in}{0.150000in}}{\pgfqpoint{2.700000in}{1.950000in}}%
\pgfusepath{clip}%
\pgfsetbuttcap%
\pgfsetroundjoin%
\definecolor{currentfill}{rgb}{0.748116,0.779136,0.822564}%
\pgfsetfillcolor{currentfill}%
\pgfsetlinewidth{0.000000pt}%
\definecolor{currentstroke}{rgb}{0.000000,0.000000,0.000000}%
\pgfsetstrokecolor{currentstroke}%
\pgfsetdash{}{0pt}%
\pgfpathmoveto{\pgfqpoint{1.266020in}{0.910915in}}%
\pgfpathlineto{\pgfqpoint{1.305315in}{0.921070in}}%
\pgfpathlineto{\pgfqpoint{1.266768in}{0.961139in}}%
\pgfpathlineto{\pgfqpoint{1.227361in}{0.951097in}}%
\pgfpathclose%
\pgfusepath{fill}%
\end{pgfscope}%
\begin{pgfscope}%
\pgfpathrectangle{\pgfqpoint{0.150000in}{0.150000in}}{\pgfqpoint{2.700000in}{1.950000in}}%
\pgfusepath{clip}%
\pgfsetbuttcap%
\pgfsetroundjoin%
\definecolor{currentfill}{rgb}{0.717019,0.751869,0.800659}%
\pgfsetfillcolor{currentfill}%
\pgfsetlinewidth{0.000000pt}%
\definecolor{currentstroke}{rgb}{0.000000,0.000000,0.000000}%
\pgfsetstrokecolor{currentstroke}%
\pgfsetdash{}{0pt}%
\pgfpathmoveto{\pgfqpoint{1.227361in}{0.951097in}}%
\pgfpathlineto{\pgfqpoint{1.266768in}{0.961139in}}%
\pgfpathlineto{\pgfqpoint{1.228216in}{1.001213in}}%
\pgfpathlineto{\pgfqpoint{1.188696in}{0.991285in}}%
\pgfpathclose%
\pgfusepath{fill}%
\end{pgfscope}%
\begin{pgfscope}%
\pgfpathrectangle{\pgfqpoint{0.150000in}{0.150000in}}{\pgfqpoint{2.700000in}{1.950000in}}%
\pgfusepath{clip}%
\pgfsetbuttcap%
\pgfsetroundjoin%
\definecolor{currentfill}{rgb}{0.692142,0.730055,0.783134}%
\pgfsetfillcolor{currentfill}%
\pgfsetlinewidth{0.000000pt}%
\definecolor{currentstroke}{rgb}{0.000000,0.000000,0.000000}%
\pgfsetstrokecolor{currentstroke}%
\pgfsetdash{}{0pt}%
\pgfpathmoveto{\pgfqpoint{1.188696in}{0.991285in}}%
\pgfpathlineto{\pgfqpoint{1.228216in}{1.001213in}}%
\pgfpathlineto{\pgfqpoint{1.189658in}{1.041293in}}%
\pgfpathlineto{\pgfqpoint{1.150025in}{1.031478in}}%
\pgfpathclose%
\pgfusepath{fill}%
\end{pgfscope}%
\begin{pgfscope}%
\pgfpathrectangle{\pgfqpoint{0.150000in}{0.150000in}}{\pgfqpoint{2.700000in}{1.950000in}}%
\pgfusepath{clip}%
\pgfsetbuttcap%
\pgfsetroundjoin%
\definecolor{currentfill}{rgb}{0.661045,0.702788,0.761229}%
\pgfsetfillcolor{currentfill}%
\pgfsetlinewidth{0.000000pt}%
\definecolor{currentstroke}{rgb}{0.000000,0.000000,0.000000}%
\pgfsetstrokecolor{currentstroke}%
\pgfsetdash{}{0pt}%
\pgfpathmoveto{\pgfqpoint{1.150025in}{1.031478in}}%
\pgfpathlineto{\pgfqpoint{1.189658in}{1.041293in}}%
\pgfpathlineto{\pgfqpoint{1.151094in}{1.081378in}}%
\pgfpathlineto{\pgfqpoint{1.111350in}{1.071676in}}%
\pgfpathclose%
\pgfusepath{fill}%
\end{pgfscope}%
\begin{pgfscope}%
\pgfpathrectangle{\pgfqpoint{0.150000in}{0.150000in}}{\pgfqpoint{2.700000in}{1.950000in}}%
\pgfusepath{clip}%
\pgfsetbuttcap%
\pgfsetroundjoin%
\definecolor{currentfill}{rgb}{0.629948,0.675521,0.739323}%
\pgfsetfillcolor{currentfill}%
\pgfsetlinewidth{0.000000pt}%
\definecolor{currentstroke}{rgb}{0.000000,0.000000,0.000000}%
\pgfsetstrokecolor{currentstroke}%
\pgfsetdash{}{0pt}%
\pgfpathmoveto{\pgfqpoint{1.111350in}{1.071676in}}%
\pgfpathlineto{\pgfqpoint{1.151094in}{1.081378in}}%
\pgfpathlineto{\pgfqpoint{1.112526in}{1.121469in}}%
\pgfpathlineto{\pgfqpoint{1.072669in}{1.111881in}}%
\pgfpathclose%
\pgfusepath{fill}%
\end{pgfscope}%
\begin{pgfscope}%
\pgfpathrectangle{\pgfqpoint{0.150000in}{0.150000in}}{\pgfqpoint{2.700000in}{1.950000in}}%
\pgfusepath{clip}%
\pgfsetbuttcap%
\pgfsetroundjoin%
\definecolor{currentfill}{rgb}{0.598851,0.648254,0.717417}%
\pgfsetfillcolor{currentfill}%
\pgfsetlinewidth{0.000000pt}%
\definecolor{currentstroke}{rgb}{0.000000,0.000000,0.000000}%
\pgfsetstrokecolor{currentstroke}%
\pgfsetdash{}{0pt}%
\pgfpathmoveto{\pgfqpoint{1.072669in}{1.111881in}}%
\pgfpathlineto{\pgfqpoint{1.112526in}{1.121469in}}%
\pgfpathlineto{\pgfqpoint{1.073952in}{1.161566in}}%
\pgfpathlineto{\pgfqpoint{1.033982in}{1.152091in}}%
\pgfpathclose%
\pgfusepath{fill}%
\end{pgfscope}%
\begin{pgfscope}%
\pgfpathrectangle{\pgfqpoint{0.150000in}{0.150000in}}{\pgfqpoint{2.700000in}{1.950000in}}%
\pgfusepath{clip}%
\pgfsetbuttcap%
\pgfsetroundjoin%
\definecolor{currentfill}{rgb}{0.567754,0.620987,0.695512}%
\pgfsetfillcolor{currentfill}%
\pgfsetlinewidth{0.000000pt}%
\definecolor{currentstroke}{rgb}{0.000000,0.000000,0.000000}%
\pgfsetstrokecolor{currentstroke}%
\pgfsetdash{}{0pt}%
\pgfpathmoveto{\pgfqpoint{1.033982in}{1.152091in}}%
\pgfpathlineto{\pgfqpoint{1.073952in}{1.161566in}}%
\pgfpathlineto{\pgfqpoint{1.035373in}{1.201668in}}%
\pgfpathlineto{\pgfqpoint{0.995291in}{1.192306in}}%
\pgfpathclose%
\pgfusepath{fill}%
\end{pgfscope}%
\begin{pgfscope}%
\pgfpathrectangle{\pgfqpoint{0.150000in}{0.150000in}}{\pgfqpoint{2.700000in}{1.950000in}}%
\pgfusepath{clip}%
\pgfsetbuttcap%
\pgfsetroundjoin%
\definecolor{currentfill}{rgb}{0.536657,0.593719,0.673606}%
\pgfsetfillcolor{currentfill}%
\pgfsetlinewidth{0.000000pt}%
\definecolor{currentstroke}{rgb}{0.000000,0.000000,0.000000}%
\pgfsetstrokecolor{currentstroke}%
\pgfsetdash{}{0pt}%
\pgfpathmoveto{\pgfqpoint{0.995291in}{1.192306in}}%
\pgfpathlineto{\pgfqpoint{1.035373in}{1.201668in}}%
\pgfpathlineto{\pgfqpoint{0.996788in}{1.241775in}}%
\pgfpathlineto{\pgfqpoint{0.956594in}{1.232527in}}%
\pgfpathclose%
\pgfusepath{fill}%
\end{pgfscope}%
\begin{pgfscope}%
\pgfpathrectangle{\pgfqpoint{0.150000in}{0.150000in}}{\pgfqpoint{2.700000in}{1.950000in}}%
\pgfusepath{clip}%
\pgfsetbuttcap%
\pgfsetroundjoin%
\definecolor{currentfill}{rgb}{0.505561,0.566452,0.651700}%
\pgfsetfillcolor{currentfill}%
\pgfsetlinewidth{0.000000pt}%
\definecolor{currentstroke}{rgb}{0.000000,0.000000,0.000000}%
\pgfsetstrokecolor{currentstroke}%
\pgfsetdash{}{0pt}%
\pgfpathmoveto{\pgfqpoint{0.956594in}{1.232527in}}%
\pgfpathlineto{\pgfqpoint{0.996788in}{1.241775in}}%
\pgfpathlineto{\pgfqpoint{0.958198in}{1.281888in}}%
\pgfpathlineto{\pgfqpoint{0.917891in}{1.272754in}}%
\pgfpathclose%
\pgfusepath{fill}%
\end{pgfscope}%
\begin{pgfscope}%
\pgfpathrectangle{\pgfqpoint{0.150000in}{0.150000in}}{\pgfqpoint{2.700000in}{1.950000in}}%
\pgfusepath{clip}%
\pgfsetbuttcap%
\pgfsetroundjoin%
\definecolor{currentfill}{rgb}{0.474464,0.539185,0.629795}%
\pgfsetfillcolor{currentfill}%
\pgfsetlinewidth{0.000000pt}%
\definecolor{currentstroke}{rgb}{0.000000,0.000000,0.000000}%
\pgfsetstrokecolor{currentstroke}%
\pgfsetdash{}{0pt}%
\pgfpathmoveto{\pgfqpoint{0.917891in}{1.272754in}}%
\pgfpathlineto{\pgfqpoint{0.958198in}{1.281888in}}%
\pgfpathlineto{\pgfqpoint{0.919603in}{1.322007in}}%
\pgfpathlineto{\pgfqpoint{0.879183in}{1.312986in}}%
\pgfpathclose%
\pgfusepath{fill}%
\end{pgfscope}%
\begin{pgfscope}%
\pgfpathrectangle{\pgfqpoint{0.150000in}{0.150000in}}{\pgfqpoint{2.700000in}{1.950000in}}%
\pgfusepath{clip}%
\pgfsetbuttcap%
\pgfsetroundjoin%
\definecolor{currentfill}{rgb}{0.449586,0.517371,0.612270}%
\pgfsetfillcolor{currentfill}%
\pgfsetlinewidth{0.000000pt}%
\definecolor{currentstroke}{rgb}{0.000000,0.000000,0.000000}%
\pgfsetstrokecolor{currentstroke}%
\pgfsetdash{}{0pt}%
\pgfpathmoveto{\pgfqpoint{0.879183in}{1.312986in}}%
\pgfpathlineto{\pgfqpoint{0.919603in}{1.322007in}}%
\pgfpathlineto{\pgfqpoint{0.881002in}{1.362131in}}%
\pgfpathlineto{\pgfqpoint{0.840470in}{1.353224in}}%
\pgfpathclose%
\pgfusepath{fill}%
\end{pgfscope}%
\begin{pgfscope}%
\pgfpathrectangle{\pgfqpoint{0.150000in}{0.150000in}}{\pgfqpoint{2.700000in}{1.950000in}}%
\pgfusepath{clip}%
\pgfsetbuttcap%
\pgfsetroundjoin%
\definecolor{currentfill}{rgb}{0.418490,0.490104,0.590365}%
\pgfsetfillcolor{currentfill}%
\pgfsetlinewidth{0.000000pt}%
\definecolor{currentstroke}{rgb}{0.000000,0.000000,0.000000}%
\pgfsetstrokecolor{currentstroke}%
\pgfsetdash{}{0pt}%
\pgfpathmoveto{\pgfqpoint{0.840470in}{1.353224in}}%
\pgfpathlineto{\pgfqpoint{0.881002in}{1.362131in}}%
\pgfpathlineto{\pgfqpoint{0.842396in}{1.402261in}}%
\pgfpathlineto{\pgfqpoint{0.801751in}{1.393467in}}%
\pgfpathclose%
\pgfusepath{fill}%
\end{pgfscope}%
\begin{pgfscope}%
\pgfpathrectangle{\pgfqpoint{0.150000in}{0.150000in}}{\pgfqpoint{2.700000in}{1.950000in}}%
\pgfusepath{clip}%
\pgfsetbuttcap%
\pgfsetroundjoin%
\definecolor{currentfill}{rgb}{0.387393,0.462837,0.568459}%
\pgfsetfillcolor{currentfill}%
\pgfsetlinewidth{0.000000pt}%
\definecolor{currentstroke}{rgb}{0.000000,0.000000,0.000000}%
\pgfsetstrokecolor{currentstroke}%
\pgfsetdash{}{0pt}%
\pgfpathmoveto{\pgfqpoint{0.801751in}{1.393467in}}%
\pgfpathlineto{\pgfqpoint{0.842396in}{1.402261in}}%
\pgfpathlineto{\pgfqpoint{0.803785in}{1.442397in}}%
\pgfpathlineto{\pgfqpoint{0.763027in}{1.433716in}}%
\pgfpathclose%
\pgfusepath{fill}%
\end{pgfscope}%
\begin{pgfscope}%
\pgfpathrectangle{\pgfqpoint{0.150000in}{0.150000in}}{\pgfqpoint{2.700000in}{1.950000in}}%
\pgfusepath{clip}%
\pgfsetbuttcap%
\pgfsetroundjoin%
\definecolor{currentfill}{rgb}{0.356296,0.435570,0.546553}%
\pgfsetfillcolor{currentfill}%
\pgfsetlinewidth{0.000000pt}%
\definecolor{currentstroke}{rgb}{0.000000,0.000000,0.000000}%
\pgfsetstrokecolor{currentstroke}%
\pgfsetdash{}{0pt}%
\pgfpathmoveto{\pgfqpoint{0.763027in}{1.433716in}}%
\pgfpathlineto{\pgfqpoint{0.803785in}{1.442397in}}%
\pgfpathlineto{\pgfqpoint{0.765168in}{1.482538in}}%
\pgfpathlineto{\pgfqpoint{0.724298in}{1.473971in}}%
\pgfpathclose%
\pgfusepath{fill}%
\end{pgfscope}%
\begin{pgfscope}%
\pgfpathrectangle{\pgfqpoint{0.150000in}{0.150000in}}{\pgfqpoint{2.700000in}{1.950000in}}%
\pgfusepath{clip}%
\pgfsetbuttcap%
\pgfsetroundjoin%
\definecolor{currentfill}{rgb}{0.325199,0.408303,0.524648}%
\pgfsetfillcolor{currentfill}%
\pgfsetlinewidth{0.000000pt}%
\definecolor{currentstroke}{rgb}{0.000000,0.000000,0.000000}%
\pgfsetstrokecolor{currentstroke}%
\pgfsetdash{}{0pt}%
\pgfpathmoveto{\pgfqpoint{0.724298in}{1.473971in}}%
\pgfpathlineto{\pgfqpoint{0.765168in}{1.482538in}}%
\pgfpathlineto{\pgfqpoint{0.726546in}{1.522684in}}%
\pgfpathlineto{\pgfqpoint{0.685563in}{1.514231in}}%
\pgfpathclose%
\pgfusepath{fill}%
\end{pgfscope}%
\begin{pgfscope}%
\pgfpathrectangle{\pgfqpoint{0.150000in}{0.150000in}}{\pgfqpoint{2.700000in}{1.950000in}}%
\pgfusepath{clip}%
\pgfsetbuttcap%
\pgfsetroundjoin%
\definecolor{currentfill}{rgb}{0.294102,0.381036,0.502742}%
\pgfsetfillcolor{currentfill}%
\pgfsetlinewidth{0.000000pt}%
\definecolor{currentstroke}{rgb}{0.000000,0.000000,0.000000}%
\pgfsetstrokecolor{currentstroke}%
\pgfsetdash{}{0pt}%
\pgfpathmoveto{\pgfqpoint{0.685563in}{1.514231in}}%
\pgfpathlineto{\pgfqpoint{0.726546in}{1.522684in}}%
\pgfpathlineto{\pgfqpoint{0.687919in}{1.562836in}}%
\pgfpathlineto{\pgfqpoint{0.646823in}{1.554497in}}%
\pgfpathclose%
\pgfusepath{fill}%
\end{pgfscope}%
\begin{pgfscope}%
\pgfpathrectangle{\pgfqpoint{0.150000in}{0.150000in}}{\pgfqpoint{2.700000in}{1.950000in}}%
\pgfusepath{clip}%
\pgfsetbuttcap%
\pgfsetroundjoin%
\definecolor{currentfill}{rgb}{0.994301,0.989660,0.990028}%
\pgfsetfillcolor{currentfill}%
\pgfsetlinewidth{0.000000pt}%
\definecolor{currentstroke}{rgb}{0.000000,0.000000,0.000000}%
\pgfsetstrokecolor{currentstroke}%
\pgfsetdash{}{0pt}%
\pgfpathmoveto{\pgfqpoint{1.575210in}{0.538293in}}%
\pgfpathlineto{\pgfqpoint{1.613714in}{0.549528in}}%
\pgfpathlineto{\pgfqpoint{1.575103in}{0.589660in}}%
\pgfpathlineto{\pgfqpoint{1.536486in}{0.578538in}}%
\pgfpathclose%
\pgfusepath{fill}%
\end{pgfscope}%
\begin{pgfscope}%
\pgfpathrectangle{\pgfqpoint{0.150000in}{0.150000in}}{\pgfqpoint{2.700000in}{1.950000in}}%
\pgfusepath{clip}%
\pgfsetbuttcap%
\pgfsetroundjoin%
\definecolor{currentfill}{rgb}{0.978232,0.980913,0.984666}%
\pgfsetfillcolor{currentfill}%
\pgfsetlinewidth{0.000000pt}%
\definecolor{currentstroke}{rgb}{0.000000,0.000000,0.000000}%
\pgfsetstrokecolor{currentstroke}%
\pgfsetdash{}{0pt}%
\pgfpathmoveto{\pgfqpoint{1.536486in}{0.578538in}}%
\pgfpathlineto{\pgfqpoint{1.575103in}{0.589660in}}%
\pgfpathlineto{\pgfqpoint{1.536486in}{0.629798in}}%
\pgfpathlineto{\pgfqpoint{1.497757in}{0.618789in}}%
\pgfpathclose%
\pgfusepath{fill}%
\end{pgfscope}%
\begin{pgfscope}%
\pgfpathrectangle{\pgfqpoint{0.150000in}{0.150000in}}{\pgfqpoint{2.700000in}{1.950000in}}%
\pgfusepath{clip}%
\pgfsetbuttcap%
\pgfsetroundjoin%
\definecolor{currentfill}{rgb}{0.947135,0.953646,0.962760}%
\pgfsetfillcolor{currentfill}%
\pgfsetlinewidth{0.000000pt}%
\definecolor{currentstroke}{rgb}{0.000000,0.000000,0.000000}%
\pgfsetstrokecolor{currentstroke}%
\pgfsetdash{}{0pt}%
\pgfpathmoveto{\pgfqpoint{1.497757in}{0.618789in}}%
\pgfpathlineto{\pgfqpoint{1.536486in}{0.629798in}}%
\pgfpathlineto{\pgfqpoint{1.497865in}{0.669940in}}%
\pgfpathlineto{\pgfqpoint{1.459022in}{0.659045in}}%
\pgfpathclose%
\pgfusepath{fill}%
\end{pgfscope}%
\begin{pgfscope}%
\pgfpathrectangle{\pgfqpoint{0.150000in}{0.150000in}}{\pgfqpoint{2.700000in}{1.950000in}}%
\pgfusepath{clip}%
\pgfsetbuttcap%
\pgfsetroundjoin%
\definecolor{currentfill}{rgb}{0.916039,0.926379,0.940855}%
\pgfsetfillcolor{currentfill}%
\pgfsetlinewidth{0.000000pt}%
\definecolor{currentstroke}{rgb}{0.000000,0.000000,0.000000}%
\pgfsetstrokecolor{currentstroke}%
\pgfsetdash{}{0pt}%
\pgfpathmoveto{\pgfqpoint{1.459022in}{0.659045in}}%
\pgfpathlineto{\pgfqpoint{1.497865in}{0.669940in}}%
\pgfpathlineto{\pgfqpoint{1.459237in}{0.710089in}}%
\pgfpathlineto{\pgfqpoint{1.420282in}{0.699308in}}%
\pgfpathclose%
\pgfusepath{fill}%
\end{pgfscope}%
\begin{pgfscope}%
\pgfpathrectangle{\pgfqpoint{0.150000in}{0.150000in}}{\pgfqpoint{2.700000in}{1.950000in}}%
\pgfusepath{clip}%
\pgfsetbuttcap%
\pgfsetroundjoin%
\definecolor{currentfill}{rgb}{0.884942,0.899112,0.918949}%
\pgfsetfillcolor{currentfill}%
\pgfsetlinewidth{0.000000pt}%
\definecolor{currentstroke}{rgb}{0.000000,0.000000,0.000000}%
\pgfsetstrokecolor{currentstroke}%
\pgfsetdash{}{0pt}%
\pgfpathmoveto{\pgfqpoint{1.420282in}{0.699308in}}%
\pgfpathlineto{\pgfqpoint{1.459237in}{0.710089in}}%
\pgfpathlineto{\pgfqpoint{1.420605in}{0.750243in}}%
\pgfpathlineto{\pgfqpoint{1.381537in}{0.739575in}}%
\pgfpathclose%
\pgfusepath{fill}%
\end{pgfscope}%
\begin{pgfscope}%
\pgfpathrectangle{\pgfqpoint{0.150000in}{0.150000in}}{\pgfqpoint{2.700000in}{1.950000in}}%
\pgfusepath{clip}%
\pgfsetbuttcap%
\pgfsetroundjoin%
\definecolor{currentfill}{rgb}{0.853845,0.871844,0.897044}%
\pgfsetfillcolor{currentfill}%
\pgfsetlinewidth{0.000000pt}%
\definecolor{currentstroke}{rgb}{0.000000,0.000000,0.000000}%
\pgfsetstrokecolor{currentstroke}%
\pgfsetdash{}{0pt}%
\pgfpathmoveto{\pgfqpoint{1.381537in}{0.739575in}}%
\pgfpathlineto{\pgfqpoint{1.420605in}{0.750243in}}%
\pgfpathlineto{\pgfqpoint{1.381966in}{0.790403in}}%
\pgfpathlineto{\pgfqpoint{1.342786in}{0.779849in}}%
\pgfpathclose%
\pgfusepath{fill}%
\end{pgfscope}%
\begin{pgfscope}%
\pgfpathrectangle{\pgfqpoint{0.150000in}{0.150000in}}{\pgfqpoint{2.700000in}{1.950000in}}%
\pgfusepath{clip}%
\pgfsetbuttcap%
\pgfsetroundjoin%
\definecolor{currentfill}{rgb}{0.822748,0.844577,0.875138}%
\pgfsetfillcolor{currentfill}%
\pgfsetlinewidth{0.000000pt}%
\definecolor{currentstroke}{rgb}{0.000000,0.000000,0.000000}%
\pgfsetstrokecolor{currentstroke}%
\pgfsetdash{}{0pt}%
\pgfpathmoveto{\pgfqpoint{1.342786in}{0.779849in}}%
\pgfpathlineto{\pgfqpoint{1.381966in}{0.790403in}}%
\pgfpathlineto{\pgfqpoint{1.343323in}{0.830568in}}%
\pgfpathlineto{\pgfqpoint{1.304030in}{0.820128in}}%
\pgfpathclose%
\pgfusepath{fill}%
\end{pgfscope}%
\begin{pgfscope}%
\pgfpathrectangle{\pgfqpoint{0.150000in}{0.150000in}}{\pgfqpoint{2.700000in}{1.950000in}}%
\pgfusepath{clip}%
\pgfsetbuttcap%
\pgfsetroundjoin%
\definecolor{currentfill}{rgb}{0.797871,0.822763,0.857613}%
\pgfsetfillcolor{currentfill}%
\pgfsetlinewidth{0.000000pt}%
\definecolor{currentstroke}{rgb}{0.000000,0.000000,0.000000}%
\pgfsetstrokecolor{currentstroke}%
\pgfsetdash{}{0pt}%
\pgfpathmoveto{\pgfqpoint{1.304030in}{0.820128in}}%
\pgfpathlineto{\pgfqpoint{1.343323in}{0.830568in}}%
\pgfpathlineto{\pgfqpoint{1.304674in}{0.870739in}}%
\pgfpathlineto{\pgfqpoint{1.265268in}{0.860412in}}%
\pgfpathclose%
\pgfusepath{fill}%
\end{pgfscope}%
\begin{pgfscope}%
\pgfpathrectangle{\pgfqpoint{0.150000in}{0.150000in}}{\pgfqpoint{2.700000in}{1.950000in}}%
\pgfusepath{clip}%
\pgfsetbuttcap%
\pgfsetroundjoin%
\definecolor{currentfill}{rgb}{0.766774,0.795496,0.835708}%
\pgfsetfillcolor{currentfill}%
\pgfsetlinewidth{0.000000pt}%
\definecolor{currentstroke}{rgb}{0.000000,0.000000,0.000000}%
\pgfsetstrokecolor{currentstroke}%
\pgfsetdash{}{0pt}%
\pgfpathmoveto{\pgfqpoint{1.265268in}{0.860412in}}%
\pgfpathlineto{\pgfqpoint{1.304674in}{0.870739in}}%
\pgfpathlineto{\pgfqpoint{1.266020in}{0.910915in}}%
\pgfpathlineto{\pgfqpoint{1.226501in}{0.900702in}}%
\pgfpathclose%
\pgfusepath{fill}%
\end{pgfscope}%
\begin{pgfscope}%
\pgfpathrectangle{\pgfqpoint{0.150000in}{0.150000in}}{\pgfqpoint{2.700000in}{1.950000in}}%
\pgfusepath{clip}%
\pgfsetbuttcap%
\pgfsetroundjoin%
\definecolor{currentfill}{rgb}{0.735677,0.768229,0.813802}%
\pgfsetfillcolor{currentfill}%
\pgfsetlinewidth{0.000000pt}%
\definecolor{currentstroke}{rgb}{0.000000,0.000000,0.000000}%
\pgfsetstrokecolor{currentstroke}%
\pgfsetdash{}{0pt}%
\pgfpathmoveto{\pgfqpoint{1.226501in}{0.900702in}}%
\pgfpathlineto{\pgfqpoint{1.266020in}{0.910915in}}%
\pgfpathlineto{\pgfqpoint{1.227361in}{0.951097in}}%
\pgfpathlineto{\pgfqpoint{1.187728in}{0.940998in}}%
\pgfpathclose%
\pgfusepath{fill}%
\end{pgfscope}%
\begin{pgfscope}%
\pgfpathrectangle{\pgfqpoint{0.150000in}{0.150000in}}{\pgfqpoint{2.700000in}{1.950000in}}%
\pgfusepath{clip}%
\pgfsetbuttcap%
\pgfsetroundjoin%
\definecolor{currentfill}{rgb}{0.704580,0.740962,0.791896}%
\pgfsetfillcolor{currentfill}%
\pgfsetlinewidth{0.000000pt}%
\definecolor{currentstroke}{rgb}{0.000000,0.000000,0.000000}%
\pgfsetstrokecolor{currentstroke}%
\pgfsetdash{}{0pt}%
\pgfpathmoveto{\pgfqpoint{1.187728in}{0.940998in}}%
\pgfpathlineto{\pgfqpoint{1.227361in}{0.951097in}}%
\pgfpathlineto{\pgfqpoint{1.188696in}{0.991285in}}%
\pgfpathlineto{\pgfqpoint{1.148950in}{0.981300in}}%
\pgfpathclose%
\pgfusepath{fill}%
\end{pgfscope}%
\begin{pgfscope}%
\pgfpathrectangle{\pgfqpoint{0.150000in}{0.150000in}}{\pgfqpoint{2.700000in}{1.950000in}}%
\pgfusepath{clip}%
\pgfsetbuttcap%
\pgfsetroundjoin%
\definecolor{currentfill}{rgb}{0.673483,0.713695,0.769991}%
\pgfsetfillcolor{currentfill}%
\pgfsetlinewidth{0.000000pt}%
\definecolor{currentstroke}{rgb}{0.000000,0.000000,0.000000}%
\pgfsetstrokecolor{currentstroke}%
\pgfsetdash{}{0pt}%
\pgfpathmoveto{\pgfqpoint{1.148950in}{0.981300in}}%
\pgfpathlineto{\pgfqpoint{1.188696in}{0.991285in}}%
\pgfpathlineto{\pgfqpoint{1.150025in}{1.031478in}}%
\pgfpathlineto{\pgfqpoint{1.110167in}{1.021607in}}%
\pgfpathclose%
\pgfusepath{fill}%
\end{pgfscope}%
\begin{pgfscope}%
\pgfpathrectangle{\pgfqpoint{0.150000in}{0.150000in}}{\pgfqpoint{2.700000in}{1.950000in}}%
\pgfusepath{clip}%
\pgfsetbuttcap%
\pgfsetroundjoin%
\definecolor{currentfill}{rgb}{0.642387,0.686428,0.748085}%
\pgfsetfillcolor{currentfill}%
\pgfsetlinewidth{0.000000pt}%
\definecolor{currentstroke}{rgb}{0.000000,0.000000,0.000000}%
\pgfsetstrokecolor{currentstroke}%
\pgfsetdash{}{0pt}%
\pgfpathmoveto{\pgfqpoint{1.110167in}{1.021607in}}%
\pgfpathlineto{\pgfqpoint{1.150025in}{1.031478in}}%
\pgfpathlineto{\pgfqpoint{1.111350in}{1.071676in}}%
\pgfpathlineto{\pgfqpoint{1.071378in}{1.061919in}}%
\pgfpathclose%
\pgfusepath{fill}%
\end{pgfscope}%
\begin{pgfscope}%
\pgfpathrectangle{\pgfqpoint{0.150000in}{0.150000in}}{\pgfqpoint{2.700000in}{1.950000in}}%
\pgfusepath{clip}%
\pgfsetbuttcap%
\pgfsetroundjoin%
\definecolor{currentfill}{rgb}{0.611290,0.659161,0.726180}%
\pgfsetfillcolor{currentfill}%
\pgfsetlinewidth{0.000000pt}%
\definecolor{currentstroke}{rgb}{0.000000,0.000000,0.000000}%
\pgfsetstrokecolor{currentstroke}%
\pgfsetdash{}{0pt}%
\pgfpathmoveto{\pgfqpoint{1.071378in}{1.061919in}}%
\pgfpathlineto{\pgfqpoint{1.111350in}{1.071676in}}%
\pgfpathlineto{\pgfqpoint{1.072669in}{1.111881in}}%
\pgfpathlineto{\pgfqpoint{1.032584in}{1.102238in}}%
\pgfpathclose%
\pgfusepath{fill}%
\end{pgfscope}%
\begin{pgfscope}%
\pgfpathrectangle{\pgfqpoint{0.150000in}{0.150000in}}{\pgfqpoint{2.700000in}{1.950000in}}%
\pgfusepath{clip}%
\pgfsetbuttcap%
\pgfsetroundjoin%
\definecolor{currentfill}{rgb}{0.580193,0.631893,0.704274}%
\pgfsetfillcolor{currentfill}%
\pgfsetlinewidth{0.000000pt}%
\definecolor{currentstroke}{rgb}{0.000000,0.000000,0.000000}%
\pgfsetstrokecolor{currentstroke}%
\pgfsetdash{}{0pt}%
\pgfpathmoveto{\pgfqpoint{1.032584in}{1.102238in}}%
\pgfpathlineto{\pgfqpoint{1.072669in}{1.111881in}}%
\pgfpathlineto{\pgfqpoint{1.033982in}{1.152091in}}%
\pgfpathlineto{\pgfqpoint{0.993785in}{1.142562in}}%
\pgfpathclose%
\pgfusepath{fill}%
\end{pgfscope}%
\begin{pgfscope}%
\pgfpathrectangle{\pgfqpoint{0.150000in}{0.150000in}}{\pgfqpoint{2.700000in}{1.950000in}}%
\pgfusepath{clip}%
\pgfsetbuttcap%
\pgfsetroundjoin%
\definecolor{currentfill}{rgb}{0.555316,0.610080,0.686749}%
\pgfsetfillcolor{currentfill}%
\pgfsetlinewidth{0.000000pt}%
\definecolor{currentstroke}{rgb}{0.000000,0.000000,0.000000}%
\pgfsetstrokecolor{currentstroke}%
\pgfsetdash{}{0pt}%
\pgfpathmoveto{\pgfqpoint{0.993785in}{1.142562in}}%
\pgfpathlineto{\pgfqpoint{1.033982in}{1.152091in}}%
\pgfpathlineto{\pgfqpoint{0.995291in}{1.192306in}}%
\pgfpathlineto{\pgfqpoint{0.954980in}{1.182891in}}%
\pgfpathclose%
\pgfusepath{fill}%
\end{pgfscope}%
\begin{pgfscope}%
\pgfpathrectangle{\pgfqpoint{0.150000in}{0.150000in}}{\pgfqpoint{2.700000in}{1.950000in}}%
\pgfusepath{clip}%
\pgfsetbuttcap%
\pgfsetroundjoin%
\definecolor{currentfill}{rgb}{0.524219,0.582812,0.664844}%
\pgfsetfillcolor{currentfill}%
\pgfsetlinewidth{0.000000pt}%
\definecolor{currentstroke}{rgb}{0.000000,0.000000,0.000000}%
\pgfsetstrokecolor{currentstroke}%
\pgfsetdash{}{0pt}%
\pgfpathmoveto{\pgfqpoint{0.954980in}{1.182891in}}%
\pgfpathlineto{\pgfqpoint{0.995291in}{1.192306in}}%
\pgfpathlineto{\pgfqpoint{0.956594in}{1.232527in}}%
\pgfpathlineto{\pgfqpoint{0.916170in}{1.223226in}}%
\pgfpathclose%
\pgfusepath{fill}%
\end{pgfscope}%
\begin{pgfscope}%
\pgfpathrectangle{\pgfqpoint{0.150000in}{0.150000in}}{\pgfqpoint{2.700000in}{1.950000in}}%
\pgfusepath{clip}%
\pgfsetbuttcap%
\pgfsetroundjoin%
\definecolor{currentfill}{rgb}{0.493122,0.555545,0.642938}%
\pgfsetfillcolor{currentfill}%
\pgfsetlinewidth{0.000000pt}%
\definecolor{currentstroke}{rgb}{0.000000,0.000000,0.000000}%
\pgfsetstrokecolor{currentstroke}%
\pgfsetdash{}{0pt}%
\pgfpathmoveto{\pgfqpoint{0.916170in}{1.223226in}}%
\pgfpathlineto{\pgfqpoint{0.956594in}{1.232527in}}%
\pgfpathlineto{\pgfqpoint{0.917891in}{1.272754in}}%
\pgfpathlineto{\pgfqpoint{0.877354in}{1.263567in}}%
\pgfpathclose%
\pgfusepath{fill}%
\end{pgfscope}%
\begin{pgfscope}%
\pgfpathrectangle{\pgfqpoint{0.150000in}{0.150000in}}{\pgfqpoint{2.700000in}{1.950000in}}%
\pgfusepath{clip}%
\pgfsetbuttcap%
\pgfsetroundjoin%
\definecolor{currentfill}{rgb}{0.462025,0.528278,0.621032}%
\pgfsetfillcolor{currentfill}%
\pgfsetlinewidth{0.000000pt}%
\definecolor{currentstroke}{rgb}{0.000000,0.000000,0.000000}%
\pgfsetstrokecolor{currentstroke}%
\pgfsetdash{}{0pt}%
\pgfpathmoveto{\pgfqpoint{0.877354in}{1.263567in}}%
\pgfpathlineto{\pgfqpoint{0.917891in}{1.272754in}}%
\pgfpathlineto{\pgfqpoint{0.879183in}{1.312986in}}%
\pgfpathlineto{\pgfqpoint{0.838533in}{1.303913in}}%
\pgfpathclose%
\pgfusepath{fill}%
\end{pgfscope}%
\begin{pgfscope}%
\pgfpathrectangle{\pgfqpoint{0.150000in}{0.150000in}}{\pgfqpoint{2.700000in}{1.950000in}}%
\pgfusepath{clip}%
\pgfsetbuttcap%
\pgfsetroundjoin%
\definecolor{currentfill}{rgb}{0.430928,0.501011,0.599127}%
\pgfsetfillcolor{currentfill}%
\pgfsetlinewidth{0.000000pt}%
\definecolor{currentstroke}{rgb}{0.000000,0.000000,0.000000}%
\pgfsetstrokecolor{currentstroke}%
\pgfsetdash{}{0pt}%
\pgfpathmoveto{\pgfqpoint{0.838533in}{1.303913in}}%
\pgfpathlineto{\pgfqpoint{0.879183in}{1.312986in}}%
\pgfpathlineto{\pgfqpoint{0.840470in}{1.353224in}}%
\pgfpathlineto{\pgfqpoint{0.799706in}{1.344265in}}%
\pgfpathclose%
\pgfusepath{fill}%
\end{pgfscope}%
\begin{pgfscope}%
\pgfpathrectangle{\pgfqpoint{0.150000in}{0.150000in}}{\pgfqpoint{2.700000in}{1.950000in}}%
\pgfusepath{clip}%
\pgfsetbuttcap%
\pgfsetroundjoin%
\definecolor{currentfill}{rgb}{0.399831,0.473744,0.577221}%
\pgfsetfillcolor{currentfill}%
\pgfsetlinewidth{0.000000pt}%
\definecolor{currentstroke}{rgb}{0.000000,0.000000,0.000000}%
\pgfsetstrokecolor{currentstroke}%
\pgfsetdash{}{0pt}%
\pgfpathmoveto{\pgfqpoint{0.799706in}{1.344265in}}%
\pgfpathlineto{\pgfqpoint{0.840470in}{1.353224in}}%
\pgfpathlineto{\pgfqpoint{0.801751in}{1.393467in}}%
\pgfpathlineto{\pgfqpoint{0.760874in}{1.384623in}}%
\pgfpathclose%
\pgfusepath{fill}%
\end{pgfscope}%
\begin{pgfscope}%
\pgfpathrectangle{\pgfqpoint{0.150000in}{0.150000in}}{\pgfqpoint{2.700000in}{1.950000in}}%
\pgfusepath{clip}%
\pgfsetbuttcap%
\pgfsetroundjoin%
\definecolor{currentfill}{rgb}{0.368735,0.446477,0.555316}%
\pgfsetfillcolor{currentfill}%
\pgfsetlinewidth{0.000000pt}%
\definecolor{currentstroke}{rgb}{0.000000,0.000000,0.000000}%
\pgfsetstrokecolor{currentstroke}%
\pgfsetdash{}{0pt}%
\pgfpathmoveto{\pgfqpoint{0.760874in}{1.384623in}}%
\pgfpathlineto{\pgfqpoint{0.801751in}{1.393467in}}%
\pgfpathlineto{\pgfqpoint{0.763027in}{1.433716in}}%
\pgfpathlineto{\pgfqpoint{0.722037in}{1.424986in}}%
\pgfpathclose%
\pgfusepath{fill}%
\end{pgfscope}%
\begin{pgfscope}%
\pgfpathrectangle{\pgfqpoint{0.150000in}{0.150000in}}{\pgfqpoint{2.700000in}{1.950000in}}%
\pgfusepath{clip}%
\pgfsetbuttcap%
\pgfsetroundjoin%
\definecolor{currentfill}{rgb}{0.337638,0.419210,0.533410}%
\pgfsetfillcolor{currentfill}%
\pgfsetlinewidth{0.000000pt}%
\definecolor{currentstroke}{rgb}{0.000000,0.000000,0.000000}%
\pgfsetstrokecolor{currentstroke}%
\pgfsetdash{}{0pt}%
\pgfpathmoveto{\pgfqpoint{0.722037in}{1.424986in}}%
\pgfpathlineto{\pgfqpoint{0.763027in}{1.433716in}}%
\pgfpathlineto{\pgfqpoint{0.724298in}{1.473971in}}%
\pgfpathlineto{\pgfqpoint{0.683194in}{1.465355in}}%
\pgfpathclose%
\pgfusepath{fill}%
\end{pgfscope}%
\begin{pgfscope}%
\pgfpathrectangle{\pgfqpoint{0.150000in}{0.150000in}}{\pgfqpoint{2.700000in}{1.950000in}}%
\pgfusepath{clip}%
\pgfsetbuttcap%
\pgfsetroundjoin%
\definecolor{currentfill}{rgb}{0.312760,0.397396,0.515885}%
\pgfsetfillcolor{currentfill}%
\pgfsetlinewidth{0.000000pt}%
\definecolor{currentstroke}{rgb}{0.000000,0.000000,0.000000}%
\pgfsetstrokecolor{currentstroke}%
\pgfsetdash{}{0pt}%
\pgfpathmoveto{\pgfqpoint{0.683194in}{1.465355in}}%
\pgfpathlineto{\pgfqpoint{0.724298in}{1.473971in}}%
\pgfpathlineto{\pgfqpoint{0.685563in}{1.514231in}}%
\pgfpathlineto{\pgfqpoint{0.644346in}{1.505729in}}%
\pgfpathclose%
\pgfusepath{fill}%
\end{pgfscope}%
\begin{pgfscope}%
\pgfpathrectangle{\pgfqpoint{0.150000in}{0.150000in}}{\pgfqpoint{2.700000in}{1.950000in}}%
\pgfusepath{clip}%
\pgfsetbuttcap%
\pgfsetroundjoin%
\definecolor{currentfill}{rgb}{0.281664,0.370129,0.493980}%
\pgfsetfillcolor{currentfill}%
\pgfsetlinewidth{0.000000pt}%
\definecolor{currentstroke}{rgb}{0.000000,0.000000,0.000000}%
\pgfsetstrokecolor{currentstroke}%
\pgfsetdash{}{0pt}%
\pgfpathmoveto{\pgfqpoint{0.644346in}{1.505729in}}%
\pgfpathlineto{\pgfqpoint{0.685563in}{1.514231in}}%
\pgfpathlineto{\pgfqpoint{0.646823in}{1.554497in}}%
\pgfpathlineto{\pgfqpoint{0.605492in}{1.546110in}}%
\pgfpathclose%
\pgfusepath{fill}%
\end{pgfscope}%
\begin{pgfscope}%
\pgfpathrectangle{\pgfqpoint{0.150000in}{0.150000in}}{\pgfqpoint{2.700000in}{1.950000in}}%
\pgfusepath{clip}%
\pgfsetbuttcap%
\pgfsetroundjoin%
\definecolor{currentfill}{rgb}{0.990671,0.991820,0.993428}%
\pgfsetfillcolor{currentfill}%
\pgfsetlinewidth{0.000000pt}%
\definecolor{currentstroke}{rgb}{0.000000,0.000000,0.000000}%
\pgfsetstrokecolor{currentstroke}%
\pgfsetdash{}{0pt}%
\pgfpathmoveto{\pgfqpoint{1.536486in}{0.526993in}}%
\pgfpathlineto{\pgfqpoint{1.575210in}{0.538293in}}%
\pgfpathlineto{\pgfqpoint{1.536486in}{0.578538in}}%
\pgfpathlineto{\pgfqpoint{1.497649in}{0.567352in}}%
\pgfpathclose%
\pgfusepath{fill}%
\end{pgfscope}%
\begin{pgfscope}%
\pgfpathrectangle{\pgfqpoint{0.150000in}{0.150000in}}{\pgfqpoint{2.700000in}{1.950000in}}%
\pgfusepath{clip}%
\pgfsetbuttcap%
\pgfsetroundjoin%
\definecolor{currentfill}{rgb}{0.959574,0.964553,0.971523}%
\pgfsetfillcolor{currentfill}%
\pgfsetlinewidth{0.000000pt}%
\definecolor{currentstroke}{rgb}{0.000000,0.000000,0.000000}%
\pgfsetstrokecolor{currentstroke}%
\pgfsetdash{}{0pt}%
\pgfpathmoveto{\pgfqpoint{1.497649in}{0.567352in}}%
\pgfpathlineto{\pgfqpoint{1.536486in}{0.578538in}}%
\pgfpathlineto{\pgfqpoint{1.497757in}{0.618789in}}%
\pgfpathlineto{\pgfqpoint{1.458806in}{0.607717in}}%
\pgfpathclose%
\pgfusepath{fill}%
\end{pgfscope}%
\begin{pgfscope}%
\pgfpathrectangle{\pgfqpoint{0.150000in}{0.150000in}}{\pgfqpoint{2.700000in}{1.950000in}}%
\pgfusepath{clip}%
\pgfsetbuttcap%
\pgfsetroundjoin%
\definecolor{currentfill}{rgb}{0.934697,0.942739,0.953998}%
\pgfsetfillcolor{currentfill}%
\pgfsetlinewidth{0.000000pt}%
\definecolor{currentstroke}{rgb}{0.000000,0.000000,0.000000}%
\pgfsetstrokecolor{currentstroke}%
\pgfsetdash{}{0pt}%
\pgfpathmoveto{\pgfqpoint{1.458806in}{0.607717in}}%
\pgfpathlineto{\pgfqpoint{1.497757in}{0.618789in}}%
\pgfpathlineto{\pgfqpoint{1.459022in}{0.659045in}}%
\pgfpathlineto{\pgfqpoint{1.419958in}{0.648088in}}%
\pgfpathclose%
\pgfusepath{fill}%
\end{pgfscope}%
\begin{pgfscope}%
\pgfpathrectangle{\pgfqpoint{0.150000in}{0.150000in}}{\pgfqpoint{2.700000in}{1.950000in}}%
\pgfusepath{clip}%
\pgfsetbuttcap%
\pgfsetroundjoin%
\definecolor{currentfill}{rgb}{0.903600,0.915472,0.932093}%
\pgfsetfillcolor{currentfill}%
\pgfsetlinewidth{0.000000pt}%
\definecolor{currentstroke}{rgb}{0.000000,0.000000,0.000000}%
\pgfsetstrokecolor{currentstroke}%
\pgfsetdash{}{0pt}%
\pgfpathmoveto{\pgfqpoint{1.419958in}{0.648088in}}%
\pgfpathlineto{\pgfqpoint{1.459022in}{0.659045in}}%
\pgfpathlineto{\pgfqpoint{1.420282in}{0.699308in}}%
\pgfpathlineto{\pgfqpoint{1.381105in}{0.688465in}}%
\pgfpathclose%
\pgfusepath{fill}%
\end{pgfscope}%
\begin{pgfscope}%
\pgfpathrectangle{\pgfqpoint{0.150000in}{0.150000in}}{\pgfqpoint{2.700000in}{1.950000in}}%
\pgfusepath{clip}%
\pgfsetbuttcap%
\pgfsetroundjoin%
\definecolor{currentfill}{rgb}{0.872503,0.888205,0.910187}%
\pgfsetfillcolor{currentfill}%
\pgfsetlinewidth{0.000000pt}%
\definecolor{currentstroke}{rgb}{0.000000,0.000000,0.000000}%
\pgfsetstrokecolor{currentstroke}%
\pgfsetdash{}{0pt}%
\pgfpathmoveto{\pgfqpoint{1.381105in}{0.688465in}}%
\pgfpathlineto{\pgfqpoint{1.420282in}{0.699308in}}%
\pgfpathlineto{\pgfqpoint{1.381537in}{0.739575in}}%
\pgfpathlineto{\pgfqpoint{1.342246in}{0.728847in}}%
\pgfpathclose%
\pgfusepath{fill}%
\end{pgfscope}%
\begin{pgfscope}%
\pgfpathrectangle{\pgfqpoint{0.150000in}{0.150000in}}{\pgfqpoint{2.700000in}{1.950000in}}%
\pgfusepath{clip}%
\pgfsetbuttcap%
\pgfsetroundjoin%
\definecolor{currentfill}{rgb}{0.841406,0.860938,0.888281}%
\pgfsetfillcolor{currentfill}%
\pgfsetlinewidth{0.000000pt}%
\definecolor{currentstroke}{rgb}{0.000000,0.000000,0.000000}%
\pgfsetstrokecolor{currentstroke}%
\pgfsetdash{}{0pt}%
\pgfpathmoveto{\pgfqpoint{1.342246in}{0.728847in}}%
\pgfpathlineto{\pgfqpoint{1.381537in}{0.739575in}}%
\pgfpathlineto{\pgfqpoint{1.342786in}{0.779849in}}%
\pgfpathlineto{\pgfqpoint{1.303381in}{0.769234in}}%
\pgfpathclose%
\pgfusepath{fill}%
\end{pgfscope}%
\begin{pgfscope}%
\pgfpathrectangle{\pgfqpoint{0.150000in}{0.150000in}}{\pgfqpoint{2.700000in}{1.950000in}}%
\pgfusepath{clip}%
\pgfsetbuttcap%
\pgfsetroundjoin%
\definecolor{currentfill}{rgb}{0.810309,0.833670,0.866376}%
\pgfsetfillcolor{currentfill}%
\pgfsetlinewidth{0.000000pt}%
\definecolor{currentstroke}{rgb}{0.000000,0.000000,0.000000}%
\pgfsetstrokecolor{currentstroke}%
\pgfsetdash{}{0pt}%
\pgfpathmoveto{\pgfqpoint{1.303381in}{0.769234in}}%
\pgfpathlineto{\pgfqpoint{1.342786in}{0.779849in}}%
\pgfpathlineto{\pgfqpoint{1.304030in}{0.820128in}}%
\pgfpathlineto{\pgfqpoint{1.264512in}{0.809628in}}%
\pgfpathclose%
\pgfusepath{fill}%
\end{pgfscope}%
\begin{pgfscope}%
\pgfpathrectangle{\pgfqpoint{0.150000in}{0.150000in}}{\pgfqpoint{2.700000in}{1.950000in}}%
\pgfusepath{clip}%
\pgfsetbuttcap%
\pgfsetroundjoin%
\definecolor{currentfill}{rgb}{0.779213,0.806403,0.844470}%
\pgfsetfillcolor{currentfill}%
\pgfsetlinewidth{0.000000pt}%
\definecolor{currentstroke}{rgb}{0.000000,0.000000,0.000000}%
\pgfsetstrokecolor{currentstroke}%
\pgfsetdash{}{0pt}%
\pgfpathmoveto{\pgfqpoint{1.264512in}{0.809628in}}%
\pgfpathlineto{\pgfqpoint{1.304030in}{0.820128in}}%
\pgfpathlineto{\pgfqpoint{1.265268in}{0.860412in}}%
\pgfpathlineto{\pgfqpoint{1.225636in}{0.850027in}}%
\pgfpathclose%
\pgfusepath{fill}%
\end{pgfscope}%
\begin{pgfscope}%
\pgfpathrectangle{\pgfqpoint{0.150000in}{0.150000in}}{\pgfqpoint{2.700000in}{1.950000in}}%
\pgfusepath{clip}%
\pgfsetbuttcap%
\pgfsetroundjoin%
\definecolor{currentfill}{rgb}{0.748116,0.779136,0.822564}%
\pgfsetfillcolor{currentfill}%
\pgfsetlinewidth{0.000000pt}%
\definecolor{currentstroke}{rgb}{0.000000,0.000000,0.000000}%
\pgfsetstrokecolor{currentstroke}%
\pgfsetdash{}{0pt}%
\pgfpathmoveto{\pgfqpoint{1.225636in}{0.850027in}}%
\pgfpathlineto{\pgfqpoint{1.265268in}{0.860412in}}%
\pgfpathlineto{\pgfqpoint{1.226501in}{0.900702in}}%
\pgfpathlineto{\pgfqpoint{1.186756in}{0.890432in}}%
\pgfpathclose%
\pgfusepath{fill}%
\end{pgfscope}%
\begin{pgfscope}%
\pgfpathrectangle{\pgfqpoint{0.150000in}{0.150000in}}{\pgfqpoint{2.700000in}{1.950000in}}%
\pgfusepath{clip}%
\pgfsetbuttcap%
\pgfsetroundjoin%
\definecolor{currentfill}{rgb}{0.717019,0.751869,0.800659}%
\pgfsetfillcolor{currentfill}%
\pgfsetlinewidth{0.000000pt}%
\definecolor{currentstroke}{rgb}{0.000000,0.000000,0.000000}%
\pgfsetstrokecolor{currentstroke}%
\pgfsetdash{}{0pt}%
\pgfpathmoveto{\pgfqpoint{1.186756in}{0.890432in}}%
\pgfpathlineto{\pgfqpoint{1.226501in}{0.900702in}}%
\pgfpathlineto{\pgfqpoint{1.187728in}{0.940998in}}%
\pgfpathlineto{\pgfqpoint{1.147869in}{0.930842in}}%
\pgfpathclose%
\pgfusepath{fill}%
\end{pgfscope}%
\begin{pgfscope}%
\pgfpathrectangle{\pgfqpoint{0.150000in}{0.150000in}}{\pgfqpoint{2.700000in}{1.950000in}}%
\pgfusepath{clip}%
\pgfsetbuttcap%
\pgfsetroundjoin%
\definecolor{currentfill}{rgb}{0.692142,0.730055,0.783134}%
\pgfsetfillcolor{currentfill}%
\pgfsetlinewidth{0.000000pt}%
\definecolor{currentstroke}{rgb}{0.000000,0.000000,0.000000}%
\pgfsetstrokecolor{currentstroke}%
\pgfsetdash{}{0pt}%
\pgfpathmoveto{\pgfqpoint{1.147869in}{0.930842in}}%
\pgfpathlineto{\pgfqpoint{1.187728in}{0.940998in}}%
\pgfpathlineto{\pgfqpoint{1.148950in}{0.981300in}}%
\pgfpathlineto{\pgfqpoint{1.108978in}{0.971258in}}%
\pgfpathclose%
\pgfusepath{fill}%
\end{pgfscope}%
\begin{pgfscope}%
\pgfpathrectangle{\pgfqpoint{0.150000in}{0.150000in}}{\pgfqpoint{2.700000in}{1.950000in}}%
\pgfusepath{clip}%
\pgfsetbuttcap%
\pgfsetroundjoin%
\definecolor{currentfill}{rgb}{0.661045,0.702788,0.761229}%
\pgfsetfillcolor{currentfill}%
\pgfsetlinewidth{0.000000pt}%
\definecolor{currentstroke}{rgb}{0.000000,0.000000,0.000000}%
\pgfsetstrokecolor{currentstroke}%
\pgfsetdash{}{0pt}%
\pgfpathmoveto{\pgfqpoint{1.108978in}{0.971258in}}%
\pgfpathlineto{\pgfqpoint{1.148950in}{0.981300in}}%
\pgfpathlineto{\pgfqpoint{1.110167in}{1.021607in}}%
\pgfpathlineto{\pgfqpoint{1.070081in}{1.011679in}}%
\pgfpathclose%
\pgfusepath{fill}%
\end{pgfscope}%
\begin{pgfscope}%
\pgfpathrectangle{\pgfqpoint{0.150000in}{0.150000in}}{\pgfqpoint{2.700000in}{1.950000in}}%
\pgfusepath{clip}%
\pgfsetbuttcap%
\pgfsetroundjoin%
\definecolor{currentfill}{rgb}{0.629948,0.675521,0.739323}%
\pgfsetfillcolor{currentfill}%
\pgfsetlinewidth{0.000000pt}%
\definecolor{currentstroke}{rgb}{0.000000,0.000000,0.000000}%
\pgfsetstrokecolor{currentstroke}%
\pgfsetdash{}{0pt}%
\pgfpathmoveto{\pgfqpoint{1.070081in}{1.011679in}}%
\pgfpathlineto{\pgfqpoint{1.110167in}{1.021607in}}%
\pgfpathlineto{\pgfqpoint{1.071378in}{1.061919in}}%
\pgfpathlineto{\pgfqpoint{1.031179in}{1.052107in}}%
\pgfpathclose%
\pgfusepath{fill}%
\end{pgfscope}%
\begin{pgfscope}%
\pgfpathrectangle{\pgfqpoint{0.150000in}{0.150000in}}{\pgfqpoint{2.700000in}{1.950000in}}%
\pgfusepath{clip}%
\pgfsetbuttcap%
\pgfsetroundjoin%
\definecolor{currentfill}{rgb}{0.598851,0.648254,0.717417}%
\pgfsetfillcolor{currentfill}%
\pgfsetlinewidth{0.000000pt}%
\definecolor{currentstroke}{rgb}{0.000000,0.000000,0.000000}%
\pgfsetstrokecolor{currentstroke}%
\pgfsetdash{}{0pt}%
\pgfpathmoveto{\pgfqpoint{1.031179in}{1.052107in}}%
\pgfpathlineto{\pgfqpoint{1.071378in}{1.061919in}}%
\pgfpathlineto{\pgfqpoint{1.032584in}{1.102238in}}%
\pgfpathlineto{\pgfqpoint{0.992271in}{1.092539in}}%
\pgfpathclose%
\pgfusepath{fill}%
\end{pgfscope}%
\begin{pgfscope}%
\pgfpathrectangle{\pgfqpoint{0.150000in}{0.150000in}}{\pgfqpoint{2.700000in}{1.950000in}}%
\pgfusepath{clip}%
\pgfsetbuttcap%
\pgfsetroundjoin%
\definecolor{currentfill}{rgb}{0.567754,0.620987,0.695512}%
\pgfsetfillcolor{currentfill}%
\pgfsetlinewidth{0.000000pt}%
\definecolor{currentstroke}{rgb}{0.000000,0.000000,0.000000}%
\pgfsetstrokecolor{currentstroke}%
\pgfsetdash{}{0pt}%
\pgfpathmoveto{\pgfqpoint{0.992271in}{1.092539in}}%
\pgfpathlineto{\pgfqpoint{1.032584in}{1.102238in}}%
\pgfpathlineto{\pgfqpoint{0.993785in}{1.142562in}}%
\pgfpathlineto{\pgfqpoint{0.953357in}{1.132978in}}%
\pgfpathclose%
\pgfusepath{fill}%
\end{pgfscope}%
\begin{pgfscope}%
\pgfpathrectangle{\pgfqpoint{0.150000in}{0.150000in}}{\pgfqpoint{2.700000in}{1.950000in}}%
\pgfusepath{clip}%
\pgfsetbuttcap%
\pgfsetroundjoin%
\definecolor{currentfill}{rgb}{0.536657,0.593719,0.673606}%
\pgfsetfillcolor{currentfill}%
\pgfsetlinewidth{0.000000pt}%
\definecolor{currentstroke}{rgb}{0.000000,0.000000,0.000000}%
\pgfsetstrokecolor{currentstroke}%
\pgfsetdash{}{0pt}%
\pgfpathmoveto{\pgfqpoint{0.953357in}{1.132978in}}%
\pgfpathlineto{\pgfqpoint{0.993785in}{1.142562in}}%
\pgfpathlineto{\pgfqpoint{0.954980in}{1.182891in}}%
\pgfpathlineto{\pgfqpoint{0.914439in}{1.173422in}}%
\pgfpathclose%
\pgfusepath{fill}%
\end{pgfscope}%
\begin{pgfscope}%
\pgfpathrectangle{\pgfqpoint{0.150000in}{0.150000in}}{\pgfqpoint{2.700000in}{1.950000in}}%
\pgfusepath{clip}%
\pgfsetbuttcap%
\pgfsetroundjoin%
\definecolor{currentfill}{rgb}{0.505561,0.566452,0.651700}%
\pgfsetfillcolor{currentfill}%
\pgfsetlinewidth{0.000000pt}%
\definecolor{currentstroke}{rgb}{0.000000,0.000000,0.000000}%
\pgfsetstrokecolor{currentstroke}%
\pgfsetdash{}{0pt}%
\pgfpathmoveto{\pgfqpoint{0.914439in}{1.173422in}}%
\pgfpathlineto{\pgfqpoint{0.954980in}{1.182891in}}%
\pgfpathlineto{\pgfqpoint{0.916170in}{1.223226in}}%
\pgfpathlineto{\pgfqpoint{0.875514in}{1.213872in}}%
\pgfpathclose%
\pgfusepath{fill}%
\end{pgfscope}%
\begin{pgfscope}%
\pgfpathrectangle{\pgfqpoint{0.150000in}{0.150000in}}{\pgfqpoint{2.700000in}{1.950000in}}%
\pgfusepath{clip}%
\pgfsetbuttcap%
\pgfsetroundjoin%
\definecolor{currentfill}{rgb}{0.474464,0.539185,0.629795}%
\pgfsetfillcolor{currentfill}%
\pgfsetlinewidth{0.000000pt}%
\definecolor{currentstroke}{rgb}{0.000000,0.000000,0.000000}%
\pgfsetstrokecolor{currentstroke}%
\pgfsetdash{}{0pt}%
\pgfpathmoveto{\pgfqpoint{0.875514in}{1.213872in}}%
\pgfpathlineto{\pgfqpoint{0.916170in}{1.223226in}}%
\pgfpathlineto{\pgfqpoint{0.877354in}{1.263567in}}%
\pgfpathlineto{\pgfqpoint{0.836585in}{1.254328in}}%
\pgfpathclose%
\pgfusepath{fill}%
\end{pgfscope}%
\begin{pgfscope}%
\pgfpathrectangle{\pgfqpoint{0.150000in}{0.150000in}}{\pgfqpoint{2.700000in}{1.950000in}}%
\pgfusepath{clip}%
\pgfsetbuttcap%
\pgfsetroundjoin%
\definecolor{currentfill}{rgb}{0.449586,0.517371,0.612270}%
\pgfsetfillcolor{currentfill}%
\pgfsetlinewidth{0.000000pt}%
\definecolor{currentstroke}{rgb}{0.000000,0.000000,0.000000}%
\pgfsetstrokecolor{currentstroke}%
\pgfsetdash{}{0pt}%
\pgfpathmoveto{\pgfqpoint{0.836585in}{1.254328in}}%
\pgfpathlineto{\pgfqpoint{0.877354in}{1.263567in}}%
\pgfpathlineto{\pgfqpoint{0.838533in}{1.303913in}}%
\pgfpathlineto{\pgfqpoint{0.797650in}{1.294789in}}%
\pgfpathclose%
\pgfusepath{fill}%
\end{pgfscope}%
\begin{pgfscope}%
\pgfpathrectangle{\pgfqpoint{0.150000in}{0.150000in}}{\pgfqpoint{2.700000in}{1.950000in}}%
\pgfusepath{clip}%
\pgfsetbuttcap%
\pgfsetroundjoin%
\definecolor{currentfill}{rgb}{0.418490,0.490104,0.590365}%
\pgfsetfillcolor{currentfill}%
\pgfsetlinewidth{0.000000pt}%
\definecolor{currentstroke}{rgb}{0.000000,0.000000,0.000000}%
\pgfsetstrokecolor{currentstroke}%
\pgfsetdash{}{0pt}%
\pgfpathmoveto{\pgfqpoint{0.797650in}{1.294789in}}%
\pgfpathlineto{\pgfqpoint{0.838533in}{1.303913in}}%
\pgfpathlineto{\pgfqpoint{0.799706in}{1.344265in}}%
\pgfpathlineto{\pgfqpoint{0.758709in}{1.335255in}}%
\pgfpathclose%
\pgfusepath{fill}%
\end{pgfscope}%
\begin{pgfscope}%
\pgfpathrectangle{\pgfqpoint{0.150000in}{0.150000in}}{\pgfqpoint{2.700000in}{1.950000in}}%
\pgfusepath{clip}%
\pgfsetbuttcap%
\pgfsetroundjoin%
\definecolor{currentfill}{rgb}{0.387393,0.462837,0.568459}%
\pgfsetfillcolor{currentfill}%
\pgfsetlinewidth{0.000000pt}%
\definecolor{currentstroke}{rgb}{0.000000,0.000000,0.000000}%
\pgfsetstrokecolor{currentstroke}%
\pgfsetdash{}{0pt}%
\pgfpathmoveto{\pgfqpoint{0.758709in}{1.335255in}}%
\pgfpathlineto{\pgfqpoint{0.799706in}{1.344265in}}%
\pgfpathlineto{\pgfqpoint{0.760874in}{1.384623in}}%
\pgfpathlineto{\pgfqpoint{0.719763in}{1.375728in}}%
\pgfpathclose%
\pgfusepath{fill}%
\end{pgfscope}%
\begin{pgfscope}%
\pgfpathrectangle{\pgfqpoint{0.150000in}{0.150000in}}{\pgfqpoint{2.700000in}{1.950000in}}%
\pgfusepath{clip}%
\pgfsetbuttcap%
\pgfsetroundjoin%
\definecolor{currentfill}{rgb}{0.356296,0.435570,0.546553}%
\pgfsetfillcolor{currentfill}%
\pgfsetlinewidth{0.000000pt}%
\definecolor{currentstroke}{rgb}{0.000000,0.000000,0.000000}%
\pgfsetstrokecolor{currentstroke}%
\pgfsetdash{}{0pt}%
\pgfpathmoveto{\pgfqpoint{0.719763in}{1.375728in}}%
\pgfpathlineto{\pgfqpoint{0.760874in}{1.384623in}}%
\pgfpathlineto{\pgfqpoint{0.722037in}{1.424986in}}%
\pgfpathlineto{\pgfqpoint{0.680812in}{1.416206in}}%
\pgfpathclose%
\pgfusepath{fill}%
\end{pgfscope}%
\begin{pgfscope}%
\pgfpathrectangle{\pgfqpoint{0.150000in}{0.150000in}}{\pgfqpoint{2.700000in}{1.950000in}}%
\pgfusepath{clip}%
\pgfsetbuttcap%
\pgfsetroundjoin%
\definecolor{currentfill}{rgb}{0.325199,0.408303,0.524648}%
\pgfsetfillcolor{currentfill}%
\pgfsetlinewidth{0.000000pt}%
\definecolor{currentstroke}{rgb}{0.000000,0.000000,0.000000}%
\pgfsetstrokecolor{currentstroke}%
\pgfsetdash{}{0pt}%
\pgfpathmoveto{\pgfqpoint{0.680812in}{1.416206in}}%
\pgfpathlineto{\pgfqpoint{0.722037in}{1.424986in}}%
\pgfpathlineto{\pgfqpoint{0.683194in}{1.465355in}}%
\pgfpathlineto{\pgfqpoint{0.641855in}{1.456690in}}%
\pgfpathclose%
\pgfusepath{fill}%
\end{pgfscope}%
\begin{pgfscope}%
\pgfpathrectangle{\pgfqpoint{0.150000in}{0.150000in}}{\pgfqpoint{2.700000in}{1.950000in}}%
\pgfusepath{clip}%
\pgfsetbuttcap%
\pgfsetroundjoin%
\definecolor{currentfill}{rgb}{0.294102,0.381036,0.502742}%
\pgfsetfillcolor{currentfill}%
\pgfsetlinewidth{0.000000pt}%
\definecolor{currentstroke}{rgb}{0.000000,0.000000,0.000000}%
\pgfsetstrokecolor{currentstroke}%
\pgfsetdash{}{0pt}%
\pgfpathmoveto{\pgfqpoint{0.641855in}{1.456690in}}%
\pgfpathlineto{\pgfqpoint{0.683194in}{1.465355in}}%
\pgfpathlineto{\pgfqpoint{0.644346in}{1.505729in}}%
\pgfpathlineto{\pgfqpoint{0.602893in}{1.497179in}}%
\pgfpathclose%
\pgfusepath{fill}%
\end{pgfscope}%
\begin{pgfscope}%
\pgfpathrectangle{\pgfqpoint{0.150000in}{0.150000in}}{\pgfqpoint{2.700000in}{1.950000in}}%
\pgfusepath{clip}%
\pgfsetbuttcap%
\pgfsetroundjoin%
\definecolor{currentfill}{rgb}{0.263006,0.353768,0.480836}%
\pgfsetfillcolor{currentfill}%
\pgfsetlinewidth{0.000000pt}%
\definecolor{currentstroke}{rgb}{0.000000,0.000000,0.000000}%
\pgfsetstrokecolor{currentstroke}%
\pgfsetdash{}{0pt}%
\pgfpathmoveto{\pgfqpoint{0.602893in}{1.497179in}}%
\pgfpathlineto{\pgfqpoint{0.644346in}{1.505729in}}%
\pgfpathlineto{\pgfqpoint{0.605492in}{1.546110in}}%
\pgfpathlineto{\pgfqpoint{0.563925in}{1.537674in}}%
\pgfpathclose%
\pgfusepath{fill}%
\end{pgfscope}%
\end{pgfpicture}%
\makeatother%
\endgroup%
}
            \hfill
            \subbottom[\label{fig:parameterised-incompetent-games-b}]%
                {%% Creator: Matplotlib, PGF backend
%%
%% To include the figure in your LaTeX document, write
%%   \input{<filename>.pgf}
%%
%% Make sure the required packages are loaded in your preamble
%%   \usepackage{pgf}
%%
%% Figures using additional raster images can only be included by \input if
%% they are in the same directory as the main LaTeX file. For loading figures
%% from other directories you can use the `import` package
%%   \usepackage{import}
%% and then include the figures with
%%   \import{<path to file>}{<filename>.pgf}
%%
%% Matplotlib used the following preamble
%%   \usepackage{fontspec}
%%   \setmainfont{DejaVuSerif.ttf}[Path=C:/Users/Thomas/anaconda3/lib/site-packages/matplotlib/mpl-data/fonts/ttf/]
%%   \setsansfont{DejaVuSans.ttf}[Path=C:/Users/Thomas/anaconda3/lib/site-packages/matplotlib/mpl-data/fonts/ttf/]
%%   \setmonofont{DejaVuSansMono.ttf}[Path=C:/Users/Thomas/anaconda3/lib/site-packages/matplotlib/mpl-data/fonts/ttf/]
%%
\begingroup%
\makeatletter%
\begin{pgfpicture}%
\pgfpathrectangle{\pgfpointorigin}{\pgfqpoint{3.000000in}{2.250000in}}%
\pgfusepath{use as bounding box, clip}%
\begin{pgfscope}%
\pgfsetbuttcap%
\pgfsetmiterjoin%
\definecolor{currentfill}{rgb}{1.000000,1.000000,1.000000}%
\pgfsetfillcolor{currentfill}%
\pgfsetlinewidth{0.000000pt}%
\definecolor{currentstroke}{rgb}{1.000000,1.000000,1.000000}%
\pgfsetstrokecolor{currentstroke}%
\pgfsetdash{}{0pt}%
\pgfpathmoveto{\pgfqpoint{0.000000in}{0.000000in}}%
\pgfpathlineto{\pgfqpoint{3.000000in}{0.000000in}}%
\pgfpathlineto{\pgfqpoint{3.000000in}{2.250000in}}%
\pgfpathlineto{\pgfqpoint{0.000000in}{2.250000in}}%
\pgfpathclose%
\pgfusepath{fill}%
\end{pgfscope}%
\begin{pgfscope}%
\pgfsetbuttcap%
\pgfsetmiterjoin%
\definecolor{currentfill}{rgb}{1.000000,1.000000,1.000000}%
\pgfsetfillcolor{currentfill}%
\pgfsetlinewidth{0.000000pt}%
\definecolor{currentstroke}{rgb}{0.000000,0.000000,0.000000}%
\pgfsetstrokecolor{currentstroke}%
\pgfsetstrokeopacity{0.000000}%
\pgfsetdash{}{0pt}%
\pgfpathmoveto{\pgfqpoint{0.150000in}{0.150000in}}%
\pgfpathlineto{\pgfqpoint{2.850000in}{0.150000in}}%
\pgfpathlineto{\pgfqpoint{2.850000in}{2.100000in}}%
\pgfpathlineto{\pgfqpoint{0.150000in}{2.100000in}}%
\pgfpathclose%
\pgfusepath{fill}%
\end{pgfscope}%
\begin{pgfscope}%
\pgfsetbuttcap%
\pgfsetmiterjoin%
\definecolor{currentfill}{rgb}{0.950000,0.950000,0.950000}%
\pgfsetfillcolor{currentfill}%
\pgfsetfillopacity{0.500000}%
\pgfsetlinewidth{1.003750pt}%
\definecolor{currentstroke}{rgb}{0.950000,0.950000,0.950000}%
\pgfsetstrokecolor{currentstroke}%
\pgfsetstrokeopacity{0.500000}%
\pgfsetdash{}{0pt}%
\pgfpathmoveto{\pgfqpoint{2.573296in}{0.776948in}}%
\pgfpathlineto{\pgfqpoint{1.536486in}{1.299017in}}%
\pgfpathlineto{\pgfqpoint{1.536486in}{2.074448in}}%
\pgfpathlineto{\pgfqpoint{2.652584in}{1.554387in}}%
\pgfusepath{stroke,fill}%
\end{pgfscope}%
\begin{pgfscope}%
\pgfsetbuttcap%
\pgfsetmiterjoin%
\definecolor{currentfill}{rgb}{0.900000,0.900000,0.900000}%
\pgfsetfillcolor{currentfill}%
\pgfsetfillopacity{0.500000}%
\pgfsetlinewidth{1.003750pt}%
\definecolor{currentstroke}{rgb}{0.900000,0.900000,0.900000}%
\pgfsetstrokecolor{currentstroke}%
\pgfsetstrokeopacity{0.500000}%
\pgfsetdash{}{0pt}%
\pgfpathmoveto{\pgfqpoint{0.499677in}{0.776948in}}%
\pgfpathlineto{\pgfqpoint{1.536486in}{1.299017in}}%
\pgfpathlineto{\pgfqpoint{1.536486in}{2.074448in}}%
\pgfpathlineto{\pgfqpoint{0.420389in}{1.554387in}}%
\pgfusepath{stroke,fill}%
\end{pgfscope}%
\begin{pgfscope}%
\pgfsetbuttcap%
\pgfsetmiterjoin%
\definecolor{currentfill}{rgb}{0.925000,0.925000,0.925000}%
\pgfsetfillcolor{currentfill}%
\pgfsetfillopacity{0.500000}%
\pgfsetlinewidth{1.003750pt}%
\definecolor{currentstroke}{rgb}{0.925000,0.925000,0.925000}%
\pgfsetstrokecolor{currentstroke}%
\pgfsetstrokeopacity{0.500000}%
\pgfsetdash{}{0pt}%
\pgfpathmoveto{\pgfqpoint{1.536486in}{0.199655in}}%
\pgfpathlineto{\pgfqpoint{2.573296in}{0.776948in}}%
\pgfpathlineto{\pgfqpoint{1.536486in}{1.299017in}}%
\pgfpathlineto{\pgfqpoint{0.499677in}{0.776948in}}%
\pgfusepath{stroke,fill}%
\end{pgfscope}%
\begin{pgfscope}%
\pgfsetrectcap%
\pgfsetroundjoin%
\pgfsetlinewidth{0.803000pt}%
\definecolor{currentstroke}{rgb}{0.000000,0.000000,0.000000}%
\pgfsetstrokecolor{currentstroke}%
\pgfsetdash{}{0pt}%
\pgfpathmoveto{\pgfqpoint{2.573296in}{0.776948in}}%
\pgfpathlineto{\pgfqpoint{1.536486in}{0.199655in}}%
\pgfusepath{stroke}%
\end{pgfscope}%
\begin{pgfscope}%
\definecolor{textcolor}{rgb}{0.000000,0.000000,0.000000}%
\pgfsetstrokecolor{textcolor}%
\pgfsetfillcolor{textcolor}%
\pgftext[x=2.017747in,y=0.045475in,left,base,rotate=29.108966]{\color{textcolor}\sffamily\fontsize{8.000000}{9.600000}\selectfont Player 2 (\(\displaystyle \mu\))}%
\end{pgfscope}%
\begin{pgfscope}%
\pgfsetbuttcap%
\pgfsetroundjoin%
\pgfsetlinewidth{0.803000pt}%
\definecolor{currentstroke}{rgb}{0.690196,0.690196,0.690196}%
\pgfsetstrokecolor{currentstroke}%
\pgfsetdash{}{0pt}%
\pgfpathmoveto{\pgfqpoint{1.605722in}{0.238205in}}%
\pgfpathlineto{\pgfqpoint{0.568749in}{0.811728in}}%
\pgfpathlineto{\pgfqpoint{0.494997in}{1.589151in}}%
\pgfusepath{stroke}%
\end{pgfscope}%
\begin{pgfscope}%
\pgfsetbuttcap%
\pgfsetroundjoin%
\pgfsetlinewidth{0.803000pt}%
\definecolor{currentstroke}{rgb}{0.690196,0.690196,0.690196}%
\pgfsetstrokecolor{currentstroke}%
\pgfsetdash{}{0pt}%
\pgfpathmoveto{\pgfqpoint{1.793262in}{0.342627in}}%
\pgfpathlineto{\pgfqpoint{0.755965in}{0.905998in}}%
\pgfpathlineto{\pgfqpoint{0.697035in}{1.683294in}}%
\pgfusepath{stroke}%
\end{pgfscope}%
\begin{pgfscope}%
\pgfsetbuttcap%
\pgfsetroundjoin%
\pgfsetlinewidth{0.803000pt}%
\definecolor{currentstroke}{rgb}{0.690196,0.690196,0.690196}%
\pgfsetstrokecolor{currentstroke}%
\pgfsetdash{}{0pt}%
\pgfpathmoveto{\pgfqpoint{1.977414in}{0.445162in}}%
\pgfpathlineto{\pgfqpoint{0.939964in}{0.998647in}}%
\pgfpathlineto{\pgfqpoint{0.895342in}{1.775698in}}%
\pgfusepath{stroke}%
\end{pgfscope}%
\begin{pgfscope}%
\pgfsetbuttcap%
\pgfsetroundjoin%
\pgfsetlinewidth{0.803000pt}%
\definecolor{currentstroke}{rgb}{0.690196,0.690196,0.690196}%
\pgfsetstrokecolor{currentstroke}%
\pgfsetdash{}{0pt}%
\pgfpathmoveto{\pgfqpoint{2.158267in}{0.545861in}}%
\pgfpathlineto{\pgfqpoint{1.120829in}{1.089719in}}%
\pgfpathlineto{\pgfqpoint{1.090021in}{1.866411in}}%
\pgfusepath{stroke}%
\end{pgfscope}%
\begin{pgfscope}%
\pgfsetbuttcap%
\pgfsetroundjoin%
\pgfsetlinewidth{0.803000pt}%
\definecolor{currentstroke}{rgb}{0.690196,0.690196,0.690196}%
\pgfsetstrokecolor{currentstroke}%
\pgfsetdash{}{0pt}%
\pgfpathmoveto{\pgfqpoint{2.335912in}{0.644773in}}%
\pgfpathlineto{\pgfqpoint{1.298639in}{1.179253in}}%
\pgfpathlineto{\pgfqpoint{1.281170in}{1.955480in}}%
\pgfusepath{stroke}%
\end{pgfscope}%
\begin{pgfscope}%
\pgfsetbuttcap%
\pgfsetroundjoin%
\pgfsetlinewidth{0.803000pt}%
\definecolor{currentstroke}{rgb}{0.690196,0.690196,0.690196}%
\pgfsetstrokecolor{currentstroke}%
\pgfsetdash{}{0pt}%
\pgfpathmoveto{\pgfqpoint{2.510430in}{0.741945in}}%
\pgfpathlineto{\pgfqpoint{1.473472in}{1.267287in}}%
\pgfpathlineto{\pgfqpoint{1.468885in}{2.042948in}}%
\pgfusepath{stroke}%
\end{pgfscope}%
\begin{pgfscope}%
\pgfsetrectcap%
\pgfsetroundjoin%
\pgfsetlinewidth{0.803000pt}%
\definecolor{currentstroke}{rgb}{0.000000,0.000000,0.000000}%
\pgfsetstrokecolor{currentstroke}%
\pgfsetdash{}{0pt}%
\pgfpathmoveto{\pgfqpoint{1.596992in}{0.243033in}}%
\pgfpathlineto{\pgfqpoint{1.623203in}{0.228537in}}%
\pgfusepath{stroke}%
\end{pgfscope}%
\begin{pgfscope}%
\definecolor{textcolor}{rgb}{0.000000,0.000000,0.000000}%
\pgfsetstrokecolor{textcolor}%
\pgfsetfillcolor{textcolor}%
\pgftext[x=1.680378in,y=0.147403in,,top]{\color{textcolor}\sffamily\fontsize{6.000000}{7.200000}\selectfont \(\displaystyle 0.0\)}%
\end{pgfscope}%
\begin{pgfscope}%
\pgfsetrectcap%
\pgfsetroundjoin%
\pgfsetlinewidth{0.803000pt}%
\definecolor{currentstroke}{rgb}{0.000000,0.000000,0.000000}%
\pgfsetstrokecolor{currentstroke}%
\pgfsetdash{}{0pt}%
\pgfpathmoveto{\pgfqpoint{1.784534in}{0.347367in}}%
\pgfpathlineto{\pgfqpoint{1.810740in}{0.333134in}}%
\pgfusepath{stroke}%
\end{pgfscope}%
\begin{pgfscope}%
\definecolor{textcolor}{rgb}{0.000000,0.000000,0.000000}%
\pgfsetstrokecolor{textcolor}%
\pgfsetfillcolor{textcolor}%
\pgftext[x=1.866959in,y=0.252496in,,top]{\color{textcolor}\sffamily\fontsize{6.000000}{7.200000}\selectfont \(\displaystyle 0.2\)}%
\end{pgfscope}%
\begin{pgfscope}%
\pgfsetrectcap%
\pgfsetroundjoin%
\pgfsetlinewidth{0.803000pt}%
\definecolor{currentstroke}{rgb}{0.000000,0.000000,0.000000}%
\pgfsetstrokecolor{currentstroke}%
\pgfsetdash{}{0pt}%
\pgfpathmoveto{\pgfqpoint{1.968688in}{0.449817in}}%
\pgfpathlineto{\pgfqpoint{1.994886in}{0.435840in}}%
\pgfusepath{stroke}%
\end{pgfscope}%
\begin{pgfscope}%
\definecolor{textcolor}{rgb}{0.000000,0.000000,0.000000}%
\pgfsetstrokecolor{textcolor}%
\pgfsetfillcolor{textcolor}%
\pgftext[x=2.050175in,y=0.355693in,,top]{\color{textcolor}\sffamily\fontsize{6.000000}{7.200000}\selectfont \(\displaystyle 0.4\)}%
\end{pgfscope}%
\begin{pgfscope}%
\pgfsetrectcap%
\pgfsetroundjoin%
\pgfsetlinewidth{0.803000pt}%
\definecolor{currentstroke}{rgb}{0.000000,0.000000,0.000000}%
\pgfsetstrokecolor{currentstroke}%
\pgfsetdash{}{0pt}%
\pgfpathmoveto{\pgfqpoint{2.149546in}{0.550433in}}%
\pgfpathlineto{\pgfqpoint{2.175732in}{0.536706in}}%
\pgfusepath{stroke}%
\end{pgfscope}%
\begin{pgfscope}%
\definecolor{textcolor}{rgb}{0.000000,0.000000,0.000000}%
\pgfsetstrokecolor{textcolor}%
\pgfsetfillcolor{textcolor}%
\pgftext[x=2.230114in,y=0.457045in,,top]{\color{textcolor}\sffamily\fontsize{6.000000}{7.200000}\selectfont \(\displaystyle 0.6\)}%
\end{pgfscope}%
\begin{pgfscope}%
\pgfsetrectcap%
\pgfsetroundjoin%
\pgfsetlinewidth{0.803000pt}%
\definecolor{currentstroke}{rgb}{0.000000,0.000000,0.000000}%
\pgfsetstrokecolor{currentstroke}%
\pgfsetdash{}{0pt}%
\pgfpathmoveto{\pgfqpoint{2.327195in}{0.649264in}}%
\pgfpathlineto{\pgfqpoint{2.353366in}{0.635779in}}%
\pgfusepath{stroke}%
\end{pgfscope}%
\begin{pgfscope}%
\definecolor{textcolor}{rgb}{0.000000,0.000000,0.000000}%
\pgfsetstrokecolor{textcolor}%
\pgfsetfillcolor{textcolor}%
\pgftext[x=2.406864in,y=0.556601in,,top]{\color{textcolor}\sffamily\fontsize{6.000000}{7.200000}\selectfont \(\displaystyle 0.8\)}%
\end{pgfscope}%
\begin{pgfscope}%
\pgfsetrectcap%
\pgfsetroundjoin%
\pgfsetlinewidth{0.803000pt}%
\definecolor{currentstroke}{rgb}{0.000000,0.000000,0.000000}%
\pgfsetstrokecolor{currentstroke}%
\pgfsetdash{}{0pt}%
\pgfpathmoveto{\pgfqpoint{2.501720in}{0.746357in}}%
\pgfpathlineto{\pgfqpoint{2.527872in}{0.733108in}}%
\pgfusepath{stroke}%
\end{pgfscope}%
\begin{pgfscope}%
\definecolor{textcolor}{rgb}{0.000000,0.000000,0.000000}%
\pgfsetstrokecolor{textcolor}%
\pgfsetfillcolor{textcolor}%
\pgftext[x=2.580510in,y=0.654408in,,top]{\color{textcolor}\sffamily\fontsize{6.000000}{7.200000}\selectfont \(\displaystyle 1.0\)}%
\end{pgfscope}%
\begin{pgfscope}%
\pgfsetrectcap%
\pgfsetroundjoin%
\pgfsetlinewidth{0.803000pt}%
\definecolor{currentstroke}{rgb}{0.000000,0.000000,0.000000}%
\pgfsetstrokecolor{currentstroke}%
\pgfsetdash{}{0pt}%
\pgfpathmoveto{\pgfqpoint{0.499677in}{0.776948in}}%
\pgfpathlineto{\pgfqpoint{1.536486in}{0.199655in}}%
\pgfusepath{stroke}%
\end{pgfscope}%
\begin{pgfscope}%
\definecolor{textcolor}{rgb}{0.000000,0.000000,0.000000}%
\pgfsetstrokecolor{textcolor}%
\pgfsetfillcolor{textcolor}%
\pgftext[x=0.492803in,y=0.358631in,left,base,rotate=330.891034]{\color{textcolor}\sffamily\fontsize{8.000000}{9.600000}\selectfont Player 1 (\(\displaystyle \lambda\))}%
\end{pgfscope}%
\begin{pgfscope}%
\pgfsetbuttcap%
\pgfsetroundjoin%
\pgfsetlinewidth{0.803000pt}%
\definecolor{currentstroke}{rgb}{0.690196,0.690196,0.690196}%
\pgfsetstrokecolor{currentstroke}%
\pgfsetdash{}{0pt}%
\pgfpathmoveto{\pgfqpoint{2.577976in}{1.589151in}}%
\pgfpathlineto{\pgfqpoint{2.504223in}{0.811728in}}%
\pgfpathlineto{\pgfqpoint{1.467251in}{0.238205in}}%
\pgfusepath{stroke}%
\end{pgfscope}%
\begin{pgfscope}%
\pgfsetbuttcap%
\pgfsetroundjoin%
\pgfsetlinewidth{0.803000pt}%
\definecolor{currentstroke}{rgb}{0.690196,0.690196,0.690196}%
\pgfsetstrokecolor{currentstroke}%
\pgfsetdash{}{0pt}%
\pgfpathmoveto{\pgfqpoint{2.375938in}{1.683294in}}%
\pgfpathlineto{\pgfqpoint{2.317008in}{0.905998in}}%
\pgfpathlineto{\pgfqpoint{1.279711in}{0.342627in}}%
\pgfusepath{stroke}%
\end{pgfscope}%
\begin{pgfscope}%
\pgfsetbuttcap%
\pgfsetroundjoin%
\pgfsetlinewidth{0.803000pt}%
\definecolor{currentstroke}{rgb}{0.690196,0.690196,0.690196}%
\pgfsetstrokecolor{currentstroke}%
\pgfsetdash{}{0pt}%
\pgfpathmoveto{\pgfqpoint{2.177631in}{1.775698in}}%
\pgfpathlineto{\pgfqpoint{2.133009in}{0.998647in}}%
\pgfpathlineto{\pgfqpoint{1.095559in}{0.445162in}}%
\pgfusepath{stroke}%
\end{pgfscope}%
\begin{pgfscope}%
\pgfsetbuttcap%
\pgfsetroundjoin%
\pgfsetlinewidth{0.803000pt}%
\definecolor{currentstroke}{rgb}{0.690196,0.690196,0.690196}%
\pgfsetstrokecolor{currentstroke}%
\pgfsetdash{}{0pt}%
\pgfpathmoveto{\pgfqpoint{1.982952in}{1.866411in}}%
\pgfpathlineto{\pgfqpoint{1.952144in}{1.089719in}}%
\pgfpathlineto{\pgfqpoint{0.914705in}{0.545861in}}%
\pgfusepath{stroke}%
\end{pgfscope}%
\begin{pgfscope}%
\pgfsetbuttcap%
\pgfsetroundjoin%
\pgfsetlinewidth{0.803000pt}%
\definecolor{currentstroke}{rgb}{0.690196,0.690196,0.690196}%
\pgfsetstrokecolor{currentstroke}%
\pgfsetdash{}{0pt}%
\pgfpathmoveto{\pgfqpoint{1.791803in}{1.955480in}}%
\pgfpathlineto{\pgfqpoint{1.774334in}{1.179253in}}%
\pgfpathlineto{\pgfqpoint{0.737061in}{0.644773in}}%
\pgfusepath{stroke}%
\end{pgfscope}%
\begin{pgfscope}%
\pgfsetbuttcap%
\pgfsetroundjoin%
\pgfsetlinewidth{0.803000pt}%
\definecolor{currentstroke}{rgb}{0.690196,0.690196,0.690196}%
\pgfsetstrokecolor{currentstroke}%
\pgfsetdash{}{0pt}%
\pgfpathmoveto{\pgfqpoint{1.604088in}{2.042948in}}%
\pgfpathlineto{\pgfqpoint{1.599501in}{1.267287in}}%
\pgfpathlineto{\pgfqpoint{0.562543in}{0.741945in}}%
\pgfusepath{stroke}%
\end{pgfscope}%
\begin{pgfscope}%
\pgfsetrectcap%
\pgfsetroundjoin%
\pgfsetlinewidth{0.803000pt}%
\definecolor{currentstroke}{rgb}{0.000000,0.000000,0.000000}%
\pgfsetstrokecolor{currentstroke}%
\pgfsetdash{}{0pt}%
\pgfpathmoveto{\pgfqpoint{1.475981in}{0.243033in}}%
\pgfpathlineto{\pgfqpoint{1.449770in}{0.228537in}}%
\pgfusepath{stroke}%
\end{pgfscope}%
\begin{pgfscope}%
\definecolor{textcolor}{rgb}{0.000000,0.000000,0.000000}%
\pgfsetstrokecolor{textcolor}%
\pgfsetfillcolor{textcolor}%
\pgftext[x=1.392595in,y=0.147403in,,top]{\color{textcolor}\sffamily\fontsize{6.000000}{7.200000}\selectfont \(\displaystyle 0.0\)}%
\end{pgfscope}%
\begin{pgfscope}%
\pgfsetrectcap%
\pgfsetroundjoin%
\pgfsetlinewidth{0.803000pt}%
\definecolor{currentstroke}{rgb}{0.000000,0.000000,0.000000}%
\pgfsetstrokecolor{currentstroke}%
\pgfsetdash{}{0pt}%
\pgfpathmoveto{\pgfqpoint{1.288439in}{0.347367in}}%
\pgfpathlineto{\pgfqpoint{1.262233in}{0.333134in}}%
\pgfusepath{stroke}%
\end{pgfscope}%
\begin{pgfscope}%
\definecolor{textcolor}{rgb}{0.000000,0.000000,0.000000}%
\pgfsetstrokecolor{textcolor}%
\pgfsetfillcolor{textcolor}%
\pgftext[x=1.206013in,y=0.252496in,,top]{\color{textcolor}\sffamily\fontsize{6.000000}{7.200000}\selectfont \(\displaystyle 0.2\)}%
\end{pgfscope}%
\begin{pgfscope}%
\pgfsetrectcap%
\pgfsetroundjoin%
\pgfsetlinewidth{0.803000pt}%
\definecolor{currentstroke}{rgb}{0.000000,0.000000,0.000000}%
\pgfsetstrokecolor{currentstroke}%
\pgfsetdash{}{0pt}%
\pgfpathmoveto{\pgfqpoint{1.104285in}{0.449817in}}%
\pgfpathlineto{\pgfqpoint{1.078087in}{0.435840in}}%
\pgfusepath{stroke}%
\end{pgfscope}%
\begin{pgfscope}%
\definecolor{textcolor}{rgb}{0.000000,0.000000,0.000000}%
\pgfsetstrokecolor{textcolor}%
\pgfsetfillcolor{textcolor}%
\pgftext[x=1.022798in,y=0.355693in,,top]{\color{textcolor}\sffamily\fontsize{6.000000}{7.200000}\selectfont \(\displaystyle 0.4\)}%
\end{pgfscope}%
\begin{pgfscope}%
\pgfsetrectcap%
\pgfsetroundjoin%
\pgfsetlinewidth{0.803000pt}%
\definecolor{currentstroke}{rgb}{0.000000,0.000000,0.000000}%
\pgfsetstrokecolor{currentstroke}%
\pgfsetdash{}{0pt}%
\pgfpathmoveto{\pgfqpoint{0.923427in}{0.550433in}}%
\pgfpathlineto{\pgfqpoint{0.897241in}{0.536706in}}%
\pgfusepath{stroke}%
\end{pgfscope}%
\begin{pgfscope}%
\definecolor{textcolor}{rgb}{0.000000,0.000000,0.000000}%
\pgfsetstrokecolor{textcolor}%
\pgfsetfillcolor{textcolor}%
\pgftext[x=0.842859in,y=0.457045in,,top]{\color{textcolor}\sffamily\fontsize{6.000000}{7.200000}\selectfont \(\displaystyle 0.6\)}%
\end{pgfscope}%
\begin{pgfscope}%
\pgfsetrectcap%
\pgfsetroundjoin%
\pgfsetlinewidth{0.803000pt}%
\definecolor{currentstroke}{rgb}{0.000000,0.000000,0.000000}%
\pgfsetstrokecolor{currentstroke}%
\pgfsetdash{}{0pt}%
\pgfpathmoveto{\pgfqpoint{0.745778in}{0.649264in}}%
\pgfpathlineto{\pgfqpoint{0.719607in}{0.635779in}}%
\pgfusepath{stroke}%
\end{pgfscope}%
\begin{pgfscope}%
\definecolor{textcolor}{rgb}{0.000000,0.000000,0.000000}%
\pgfsetstrokecolor{textcolor}%
\pgfsetfillcolor{textcolor}%
\pgftext[x=0.666109in,y=0.556601in,,top]{\color{textcolor}\sffamily\fontsize{6.000000}{7.200000}\selectfont \(\displaystyle 0.8\)}%
\end{pgfscope}%
\begin{pgfscope}%
\pgfsetrectcap%
\pgfsetroundjoin%
\pgfsetlinewidth{0.803000pt}%
\definecolor{currentstroke}{rgb}{0.000000,0.000000,0.000000}%
\pgfsetstrokecolor{currentstroke}%
\pgfsetdash{}{0pt}%
\pgfpathmoveto{\pgfqpoint{0.571253in}{0.746357in}}%
\pgfpathlineto{\pgfqpoint{0.545101in}{0.733108in}}%
\pgfusepath{stroke}%
\end{pgfscope}%
\begin{pgfscope}%
\definecolor{textcolor}{rgb}{0.000000,0.000000,0.000000}%
\pgfsetstrokecolor{textcolor}%
\pgfsetfillcolor{textcolor}%
\pgftext[x=0.492463in,y=0.654408in,,top]{\color{textcolor}\sffamily\fontsize{6.000000}{7.200000}\selectfont \(\displaystyle 1.0\)}%
\end{pgfscope}%
\begin{pgfscope}%
\pgfsetrectcap%
\pgfsetroundjoin%
\pgfsetlinewidth{0.803000pt}%
\definecolor{currentstroke}{rgb}{0.000000,0.000000,0.000000}%
\pgfsetstrokecolor{currentstroke}%
\pgfsetdash{}{0pt}%
\pgfpathmoveto{\pgfqpoint{0.499677in}{0.776948in}}%
\pgfpathlineto{\pgfqpoint{0.420389in}{1.554387in}}%
\pgfusepath{stroke}%
\end{pgfscope}%
\begin{pgfscope}%
\definecolor{textcolor}{rgb}{0.000000,0.000000,0.000000}%
\pgfsetstrokecolor{textcolor}%
\pgfsetfillcolor{textcolor}%
\pgftext[x=0.041630in,y=1.401767in,left,base,rotate=275.823265]{\color{textcolor}\sffamily\fontsize{8.000000}{9.600000}\selectfont \(\displaystyle \mathsf{val}(G_{\lambda, \mu}\))}%
\end{pgfscope}%
\begin{pgfscope}%
\pgfsetbuttcap%
\pgfsetroundjoin%
\pgfsetlinewidth{0.803000pt}%
\definecolor{currentstroke}{rgb}{0.690196,0.690196,0.690196}%
\pgfsetstrokecolor{currentstroke}%
\pgfsetdash{}{0pt}%
\pgfpathmoveto{\pgfqpoint{0.498202in}{0.791413in}}%
\pgfpathlineto{\pgfqpoint{1.536486in}{1.313496in}}%
\pgfpathlineto{\pgfqpoint{2.574771in}{0.791413in}}%
\pgfusepath{stroke}%
\end{pgfscope}%
\begin{pgfscope}%
\pgfsetbuttcap%
\pgfsetroundjoin%
\pgfsetlinewidth{0.803000pt}%
\definecolor{currentstroke}{rgb}{0.690196,0.690196,0.690196}%
\pgfsetstrokecolor{currentstroke}%
\pgfsetdash{}{0pt}%
\pgfpathmoveto{\pgfqpoint{0.483706in}{0.933550in}}%
\pgfpathlineto{\pgfqpoint{1.536486in}{1.455671in}}%
\pgfpathlineto{\pgfqpoint{2.589267in}{0.933550in}}%
\pgfusepath{stroke}%
\end{pgfscope}%
\begin{pgfscope}%
\pgfsetbuttcap%
\pgfsetroundjoin%
\pgfsetlinewidth{0.803000pt}%
\definecolor{currentstroke}{rgb}{0.690196,0.690196,0.690196}%
\pgfsetstrokecolor{currentstroke}%
\pgfsetdash{}{0pt}%
\pgfpathmoveto{\pgfqpoint{0.468799in}{1.079713in}}%
\pgfpathlineto{\pgfqpoint{1.536486in}{1.601673in}}%
\pgfpathlineto{\pgfqpoint{2.604174in}{1.079713in}}%
\pgfusepath{stroke}%
\end{pgfscope}%
\begin{pgfscope}%
\pgfsetbuttcap%
\pgfsetroundjoin%
\pgfsetlinewidth{0.803000pt}%
\definecolor{currentstroke}{rgb}{0.690196,0.690196,0.690196}%
\pgfsetstrokecolor{currentstroke}%
\pgfsetdash{}{0pt}%
\pgfpathmoveto{\pgfqpoint{0.453464in}{1.230074in}}%
\pgfpathlineto{\pgfqpoint{1.536486in}{1.751661in}}%
\pgfpathlineto{\pgfqpoint{2.619509in}{1.230074in}}%
\pgfusepath{stroke}%
\end{pgfscope}%
\begin{pgfscope}%
\pgfsetbuttcap%
\pgfsetroundjoin%
\pgfsetlinewidth{0.803000pt}%
\definecolor{currentstroke}{rgb}{0.690196,0.690196,0.690196}%
\pgfsetstrokecolor{currentstroke}%
\pgfsetdash{}{0pt}%
\pgfpathmoveto{\pgfqpoint{0.437683in}{1.384818in}}%
\pgfpathlineto{\pgfqpoint{1.536486in}{1.905799in}}%
\pgfpathlineto{\pgfqpoint{2.635290in}{1.384818in}}%
\pgfusepath{stroke}%
\end{pgfscope}%
\begin{pgfscope}%
\pgfsetrectcap%
\pgfsetroundjoin%
\pgfsetlinewidth{0.803000pt}%
\definecolor{currentstroke}{rgb}{0.000000,0.000000,0.000000}%
\pgfsetstrokecolor{currentstroke}%
\pgfsetdash{}{0pt}%
\pgfpathmoveto{\pgfqpoint{0.506923in}{0.795798in}}%
\pgfpathlineto{\pgfqpoint{0.480740in}{0.782632in}}%
\pgfusepath{stroke}%
\end{pgfscope}%
\begin{pgfscope}%
\definecolor{textcolor}{rgb}{0.000000,0.000000,0.000000}%
\pgfsetstrokecolor{textcolor}%
\pgfsetfillcolor{textcolor}%
\pgftext[x=0.355298in,y=0.791413in,,top]{\color{textcolor}\sffamily\fontsize{6.000000}{7.200000}\selectfont \(\displaystyle -1.0\)}%
\end{pgfscope}%
\begin{pgfscope}%
\pgfsetrectcap%
\pgfsetroundjoin%
\pgfsetlinewidth{0.803000pt}%
\definecolor{currentstroke}{rgb}{0.000000,0.000000,0.000000}%
\pgfsetstrokecolor{currentstroke}%
\pgfsetdash{}{0pt}%
\pgfpathmoveto{\pgfqpoint{0.492554in}{0.937939in}}%
\pgfpathlineto{\pgfqpoint{0.465988in}{0.924763in}}%
\pgfusepath{stroke}%
\end{pgfscope}%
\begin{pgfscope}%
\definecolor{textcolor}{rgb}{0.000000,0.000000,0.000000}%
\pgfsetstrokecolor{textcolor}%
\pgfsetfillcolor{textcolor}%
\pgftext[x=0.338807in,y=0.933550in,,top]{\color{textcolor}\sffamily\fontsize{6.000000}{7.200000}\selectfont \(\displaystyle -0.5\)}%
\end{pgfscope}%
\begin{pgfscope}%
\pgfsetrectcap%
\pgfsetroundjoin%
\pgfsetlinewidth{0.803000pt}%
\definecolor{currentstroke}{rgb}{0.000000,0.000000,0.000000}%
\pgfsetstrokecolor{currentstroke}%
\pgfsetdash{}{0pt}%
\pgfpathmoveto{\pgfqpoint{0.477779in}{1.084103in}}%
\pgfpathlineto{\pgfqpoint{0.450818in}{1.070922in}}%
\pgfusepath{stroke}%
\end{pgfscope}%
\begin{pgfscope}%
\definecolor{textcolor}{rgb}{0.000000,0.000000,0.000000}%
\pgfsetstrokecolor{textcolor}%
\pgfsetfillcolor{textcolor}%
\pgftext[x=0.321849in,y=1.079713in,,top]{\color{textcolor}\sffamily\fontsize{6.000000}{7.200000}\selectfont \(\displaystyle 0.0\)}%
\end{pgfscope}%
\begin{pgfscope}%
\pgfsetrectcap%
\pgfsetroundjoin%
\pgfsetlinewidth{0.803000pt}%
\definecolor{currentstroke}{rgb}{0.000000,0.000000,0.000000}%
\pgfsetstrokecolor{currentstroke}%
\pgfsetdash{}{0pt}%
\pgfpathmoveto{\pgfqpoint{0.462579in}{1.234464in}}%
\pgfpathlineto{\pgfqpoint{0.435211in}{1.221284in}}%
\pgfusepath{stroke}%
\end{pgfscope}%
\begin{pgfscope}%
\definecolor{textcolor}{rgb}{0.000000,0.000000,0.000000}%
\pgfsetstrokecolor{textcolor}%
\pgfsetfillcolor{textcolor}%
\pgftext[x=0.304403in,y=1.230074in,,top]{\color{textcolor}\sffamily\fontsize{6.000000}{7.200000}\selectfont \(\displaystyle 0.5\)}%
\end{pgfscope}%
\begin{pgfscope}%
\pgfsetrectcap%
\pgfsetroundjoin%
\pgfsetlinewidth{0.803000pt}%
\definecolor{currentstroke}{rgb}{0.000000,0.000000,0.000000}%
\pgfsetstrokecolor{currentstroke}%
\pgfsetdash{}{0pt}%
\pgfpathmoveto{\pgfqpoint{0.446937in}{1.389205in}}%
\pgfpathlineto{\pgfqpoint{0.419150in}{1.376031in}}%
\pgfusepath{stroke}%
\end{pgfscope}%
\begin{pgfscope}%
\definecolor{textcolor}{rgb}{0.000000,0.000000,0.000000}%
\pgfsetstrokecolor{textcolor}%
\pgfsetfillcolor{textcolor}%
\pgftext[x=0.286449in,y=1.384818in,,top]{\color{textcolor}\sffamily\fontsize{6.000000}{7.200000}\selectfont \(\displaystyle 1.0\)}%
\end{pgfscope}%
\begin{pgfscope}%
\pgfpathrectangle{\pgfqpoint{0.150000in}{0.150000in}}{\pgfqpoint{2.700000in}{1.950000in}}%
\pgfusepath{clip}%
\pgfsetbuttcap%
\pgfsetroundjoin%
\definecolor{currentfill}{rgb}{0.990671,0.991820,0.993428}%
\pgfsetfillcolor{currentfill}%
\pgfsetlinewidth{0.000000pt}%
\definecolor{currentstroke}{rgb}{0.000000,0.000000,0.000000}%
\pgfsetstrokecolor{currentstroke}%
\pgfsetdash{}{0pt}%
\pgfpathmoveto{\pgfqpoint{1.536486in}{1.514761in}}%
\pgfpathlineto{\pgfqpoint{1.572171in}{1.508706in}}%
\pgfpathlineto{\pgfqpoint{1.536486in}{1.538112in}}%
\pgfpathlineto{\pgfqpoint{1.500685in}{1.544278in}}%
\pgfpathclose%
\pgfusepath{fill}%
\end{pgfscope}%
\begin{pgfscope}%
\pgfpathrectangle{\pgfqpoint{0.150000in}{0.150000in}}{\pgfqpoint{2.700000in}{1.950000in}}%
\pgfusepath{clip}%
\pgfsetbuttcap%
\pgfsetroundjoin%
\definecolor{currentfill}{rgb}{0.994301,0.989660,0.990028}%
\pgfsetfillcolor{currentfill}%
\pgfsetlinewidth{0.000000pt}%
\definecolor{currentstroke}{rgb}{0.000000,0.000000,0.000000}%
\pgfsetstrokecolor{currentstroke}%
\pgfsetdash{}{0pt}%
\pgfpathmoveto{\pgfqpoint{1.572334in}{1.485206in}}%
\pgfpathlineto{\pgfqpoint{1.607902in}{1.479262in}}%
\pgfpathlineto{\pgfqpoint{1.572171in}{1.508706in}}%
\pgfpathlineto{\pgfqpoint{1.536486in}{1.514761in}}%
\pgfpathclose%
\pgfusepath{fill}%
\end{pgfscope}%
\begin{pgfscope}%
\pgfpathrectangle{\pgfqpoint{0.150000in}{0.150000in}}{\pgfqpoint{2.700000in}{1.950000in}}%
\pgfusepath{clip}%
\pgfsetbuttcap%
\pgfsetroundjoin%
\definecolor{currentfill}{rgb}{0.982904,0.968980,0.970083}%
\pgfsetfillcolor{currentfill}%
\pgfsetlinewidth{0.000000pt}%
\definecolor{currentstroke}{rgb}{0.000000,0.000000,0.000000}%
\pgfsetstrokecolor{currentstroke}%
\pgfsetdash{}{0pt}%
\pgfpathmoveto{\pgfqpoint{1.608229in}{1.455612in}}%
\pgfpathlineto{\pgfqpoint{1.643680in}{1.449780in}}%
\pgfpathlineto{\pgfqpoint{1.607902in}{1.479262in}}%
\pgfpathlineto{\pgfqpoint{1.572334in}{1.485206in}}%
\pgfpathclose%
\pgfusepath{fill}%
\end{pgfscope}%
\begin{pgfscope}%
\pgfpathrectangle{\pgfqpoint{0.150000in}{0.150000in}}{\pgfqpoint{2.700000in}{1.950000in}}%
\pgfusepath{clip}%
\pgfsetbuttcap%
\pgfsetroundjoin%
\definecolor{currentfill}{rgb}{0.967708,0.941406,0.943490}%
\pgfsetfillcolor{currentfill}%
\pgfsetlinewidth{0.000000pt}%
\definecolor{currentstroke}{rgb}{0.000000,0.000000,0.000000}%
\pgfsetstrokecolor{currentstroke}%
\pgfsetdash{}{0pt}%
\pgfpathmoveto{\pgfqpoint{1.644169in}{1.425980in}}%
\pgfpathlineto{\pgfqpoint{1.679503in}{1.420260in}}%
\pgfpathlineto{\pgfqpoint{1.643680in}{1.449780in}}%
\pgfpathlineto{\pgfqpoint{1.608229in}{1.455612in}}%
\pgfpathclose%
\pgfusepath{fill}%
\end{pgfscope}%
\begin{pgfscope}%
\pgfpathrectangle{\pgfqpoint{0.150000in}{0.150000in}}{\pgfqpoint{2.700000in}{1.950000in}}%
\pgfusepath{clip}%
\pgfsetbuttcap%
\pgfsetroundjoin%
\definecolor{currentfill}{rgb}{0.956311,0.920726,0.923545}%
\pgfsetfillcolor{currentfill}%
\pgfsetlinewidth{0.000000pt}%
\definecolor{currentstroke}{rgb}{0.000000,0.000000,0.000000}%
\pgfsetstrokecolor{currentstroke}%
\pgfsetdash{}{0pt}%
\pgfpathmoveto{\pgfqpoint{1.680118in}{1.393367in}}%
\pgfpathlineto{\pgfqpoint{1.715373in}{1.390702in}}%
\pgfpathlineto{\pgfqpoint{1.679503in}{1.420260in}}%
\pgfpathlineto{\pgfqpoint{1.644169in}{1.425980in}}%
\pgfpathclose%
\pgfusepath{fill}%
\end{pgfscope}%
\begin{pgfscope}%
\pgfpathrectangle{\pgfqpoint{0.150000in}{0.150000in}}{\pgfqpoint{2.700000in}{1.950000in}}%
\pgfusepath{clip}%
\pgfsetbuttcap%
\pgfsetroundjoin%
\definecolor{currentfill}{rgb}{0.944914,0.900046,0.903600}%
\pgfsetfillcolor{currentfill}%
\pgfsetlinewidth{0.000000pt}%
\definecolor{currentstroke}{rgb}{0.000000,0.000000,0.000000}%
\pgfsetstrokecolor{currentstroke}%
\pgfsetdash{}{0pt}%
\pgfpathmoveto{\pgfqpoint{1.716142in}{1.363664in}}%
\pgfpathlineto{\pgfqpoint{1.751289in}{1.361105in}}%
\pgfpathlineto{\pgfqpoint{1.715373in}{1.390702in}}%
\pgfpathlineto{\pgfqpoint{1.680118in}{1.393367in}}%
\pgfpathclose%
\pgfusepath{fill}%
\end{pgfscope}%
\begin{pgfscope}%
\pgfpathrectangle{\pgfqpoint{0.150000in}{0.150000in}}{\pgfqpoint{2.700000in}{1.950000in}}%
\pgfusepath{clip}%
\pgfsetbuttcap%
\pgfsetroundjoin%
\definecolor{currentfill}{rgb}{0.929718,0.872472,0.877007}%
\pgfsetfillcolor{currentfill}%
\pgfsetlinewidth{0.000000pt}%
\definecolor{currentstroke}{rgb}{0.000000,0.000000,0.000000}%
\pgfsetstrokecolor{currentstroke}%
\pgfsetdash{}{0pt}%
\pgfpathmoveto{\pgfqpoint{1.752213in}{1.333923in}}%
\pgfpathlineto{\pgfqpoint{1.787251in}{1.331471in}}%
\pgfpathlineto{\pgfqpoint{1.751289in}{1.361105in}}%
\pgfpathlineto{\pgfqpoint{1.716142in}{1.363664in}}%
\pgfpathclose%
\pgfusepath{fill}%
\end{pgfscope}%
\begin{pgfscope}%
\pgfpathrectangle{\pgfqpoint{0.150000in}{0.150000in}}{\pgfqpoint{2.700000in}{1.950000in}}%
\pgfusepath{clip}%
\pgfsetbuttcap%
\pgfsetroundjoin%
\definecolor{currentfill}{rgb}{0.947135,0.953646,0.962760}%
\pgfsetfillcolor{currentfill}%
\pgfsetlinewidth{0.000000pt}%
\definecolor{currentstroke}{rgb}{0.000000,0.000000,0.000000}%
\pgfsetstrokecolor{currentstroke}%
\pgfsetdash{}{0pt}%
\pgfpathmoveto{\pgfqpoint{1.500521in}{1.520864in}}%
\pgfpathlineto{\pgfqpoint{1.536486in}{1.514761in}}%
\pgfpathlineto{\pgfqpoint{1.500685in}{1.544278in}}%
\pgfpathlineto{\pgfqpoint{1.464602in}{1.550493in}}%
\pgfpathclose%
\pgfusepath{fill}%
\end{pgfscope}%
\begin{pgfscope}%
\pgfpathrectangle{\pgfqpoint{0.150000in}{0.150000in}}{\pgfqpoint{2.700000in}{1.950000in}}%
\pgfusepath{clip}%
\pgfsetbuttcap%
\pgfsetroundjoin%
\definecolor{currentfill}{rgb}{0.918321,0.851792,0.857062}%
\pgfsetfillcolor{currentfill}%
\pgfsetlinewidth{0.000000pt}%
\definecolor{currentstroke}{rgb}{0.000000,0.000000,0.000000}%
\pgfsetstrokecolor{currentstroke}%
\pgfsetdash{}{0pt}%
\pgfpathmoveto{\pgfqpoint{1.788331in}{1.304143in}}%
\pgfpathlineto{\pgfqpoint{1.823260in}{1.301798in}}%
\pgfpathlineto{\pgfqpoint{1.787251in}{1.331471in}}%
\pgfpathlineto{\pgfqpoint{1.752213in}{1.333923in}}%
\pgfpathclose%
\pgfusepath{fill}%
\end{pgfscope}%
\begin{pgfscope}%
\pgfpathrectangle{\pgfqpoint{0.150000in}{0.150000in}}{\pgfqpoint{2.700000in}{1.950000in}}%
\pgfusepath{clip}%
\pgfsetbuttcap%
\pgfsetroundjoin%
\definecolor{currentfill}{rgb}{0.972013,0.975460,0.980285}%
\pgfsetfillcolor{currentfill}%
\pgfsetlinewidth{0.000000pt}%
\definecolor{currentstroke}{rgb}{0.000000,0.000000,0.000000}%
\pgfsetstrokecolor{currentstroke}%
\pgfsetdash{}{0pt}%
\pgfpathmoveto{\pgfqpoint{1.536486in}{1.491196in}}%
\pgfpathlineto{\pgfqpoint{1.572334in}{1.485206in}}%
\pgfpathlineto{\pgfqpoint{1.536486in}{1.514761in}}%
\pgfpathlineto{\pgfqpoint{1.500521in}{1.520864in}}%
\pgfpathclose%
\pgfusepath{fill}%
\end{pgfscope}%
\begin{pgfscope}%
\pgfpathrectangle{\pgfqpoint{0.150000in}{0.150000in}}{\pgfqpoint{2.700000in}{1.950000in}}%
\pgfusepath{clip}%
\pgfsetbuttcap%
\pgfsetroundjoin%
\definecolor{currentfill}{rgb}{0.906924,0.831112,0.837117}%
\pgfsetfillcolor{currentfill}%
\pgfsetlinewidth{0.000000pt}%
\definecolor{currentstroke}{rgb}{0.000000,0.000000,0.000000}%
\pgfsetstrokecolor{currentstroke}%
\pgfsetdash{}{0pt}%
\pgfpathmoveto{\pgfqpoint{1.824496in}{1.274325in}}%
\pgfpathlineto{\pgfqpoint{1.859316in}{1.272086in}}%
\pgfpathlineto{\pgfqpoint{1.823260in}{1.301798in}}%
\pgfpathlineto{\pgfqpoint{1.788331in}{1.304143in}}%
\pgfpathclose%
\pgfusepath{fill}%
\end{pgfscope}%
\begin{pgfscope}%
\pgfpathrectangle{\pgfqpoint{0.150000in}{0.150000in}}{\pgfqpoint{2.700000in}{1.950000in}}%
\pgfusepath{clip}%
\pgfsetbuttcap%
\pgfsetroundjoin%
\definecolor{currentfill}{rgb}{0.990671,0.991820,0.993428}%
\pgfsetfillcolor{currentfill}%
\pgfsetlinewidth{0.000000pt}%
\definecolor{currentstroke}{rgb}{0.000000,0.000000,0.000000}%
\pgfsetstrokecolor{currentstroke}%
\pgfsetdash{}{0pt}%
\pgfpathmoveto{\pgfqpoint{1.572489in}{1.458522in}}%
\pgfpathlineto{\pgfqpoint{1.608229in}{1.455612in}}%
\pgfpathlineto{\pgfqpoint{1.572334in}{1.485206in}}%
\pgfpathlineto{\pgfqpoint{1.536486in}{1.491196in}}%
\pgfpathclose%
\pgfusepath{fill}%
\end{pgfscope}%
\begin{pgfscope}%
\pgfpathrectangle{\pgfqpoint{0.150000in}{0.150000in}}{\pgfqpoint{2.700000in}{1.950000in}}%
\pgfusepath{clip}%
\pgfsetbuttcap%
\pgfsetroundjoin%
\definecolor{currentfill}{rgb}{0.895527,0.810432,0.817172}%
\pgfsetfillcolor{currentfill}%
\pgfsetlinewidth{0.000000pt}%
\definecolor{currentstroke}{rgb}{0.000000,0.000000,0.000000}%
\pgfsetstrokecolor{currentstroke}%
\pgfsetdash{}{0pt}%
\pgfpathmoveto{\pgfqpoint{1.860708in}{1.244467in}}%
\pgfpathlineto{\pgfqpoint{1.895418in}{1.242336in}}%
\pgfpathlineto{\pgfqpoint{1.859316in}{1.272086in}}%
\pgfpathlineto{\pgfqpoint{1.824496in}{1.274325in}}%
\pgfpathclose%
\pgfusepath{fill}%
\end{pgfscope}%
\begin{pgfscope}%
\pgfpathrectangle{\pgfqpoint{0.150000in}{0.150000in}}{\pgfqpoint{2.700000in}{1.950000in}}%
\pgfusepath{clip}%
\pgfsetbuttcap%
\pgfsetroundjoin%
\definecolor{currentfill}{rgb}{0.990502,0.982767,0.983379}%
\pgfsetfillcolor{currentfill}%
\pgfsetlinewidth{0.000000pt}%
\definecolor{currentstroke}{rgb}{0.000000,0.000000,0.000000}%
\pgfsetstrokecolor{currentstroke}%
\pgfsetdash{}{0pt}%
\pgfpathmoveto{\pgfqpoint{1.608538in}{1.428784in}}%
\pgfpathlineto{\pgfqpoint{1.644169in}{1.425980in}}%
\pgfpathlineto{\pgfqpoint{1.608229in}{1.455612in}}%
\pgfpathlineto{\pgfqpoint{1.572489in}{1.458522in}}%
\pgfpathclose%
\pgfusepath{fill}%
\end{pgfscope}%
\begin{pgfscope}%
\pgfpathrectangle{\pgfqpoint{0.150000in}{0.150000in}}{\pgfqpoint{2.700000in}{1.950000in}}%
\pgfusepath{clip}%
\pgfsetbuttcap%
\pgfsetroundjoin%
\definecolor{currentfill}{rgb}{0.979105,0.962086,0.963434}%
\pgfsetfillcolor{currentfill}%
\pgfsetlinewidth{0.000000pt}%
\definecolor{currentstroke}{rgb}{0.000000,0.000000,0.000000}%
\pgfsetstrokecolor{currentstroke}%
\pgfsetdash{}{0pt}%
\pgfpathmoveto{\pgfqpoint{1.644634in}{1.399006in}}%
\pgfpathlineto{\pgfqpoint{1.680118in}{1.393367in}}%
\pgfpathlineto{\pgfqpoint{1.644169in}{1.425980in}}%
\pgfpathlineto{\pgfqpoint{1.608538in}{1.428784in}}%
\pgfpathclose%
\pgfusepath{fill}%
\end{pgfscope}%
\begin{pgfscope}%
\pgfpathrectangle{\pgfqpoint{0.150000in}{0.150000in}}{\pgfqpoint{2.700000in}{1.950000in}}%
\pgfusepath{clip}%
\pgfsetbuttcap%
\pgfsetroundjoin%
\definecolor{currentfill}{rgb}{0.880331,0.782858,0.790579}%
\pgfsetfillcolor{currentfill}%
\pgfsetlinewidth{0.000000pt}%
\definecolor{currentstroke}{rgb}{0.000000,0.000000,0.000000}%
\pgfsetstrokecolor{currentstroke}%
\pgfsetdash{}{0pt}%
\pgfpathmoveto{\pgfqpoint{1.896967in}{1.214571in}}%
\pgfpathlineto{\pgfqpoint{1.931567in}{1.212547in}}%
\pgfpathlineto{\pgfqpoint{1.895418in}{1.242336in}}%
\pgfpathlineto{\pgfqpoint{1.860708in}{1.244467in}}%
\pgfpathclose%
\pgfusepath{fill}%
\end{pgfscope}%
\begin{pgfscope}%
\pgfpathrectangle{\pgfqpoint{0.150000in}{0.150000in}}{\pgfqpoint{2.700000in}{1.950000in}}%
\pgfusepath{clip}%
\pgfsetbuttcap%
\pgfsetroundjoin%
\definecolor{currentfill}{rgb}{0.967708,0.941406,0.943490}%
\pgfsetfillcolor{currentfill}%
\pgfsetlinewidth{0.000000pt}%
\definecolor{currentstroke}{rgb}{0.000000,0.000000,0.000000}%
\pgfsetstrokecolor{currentstroke}%
\pgfsetdash{}{0pt}%
\pgfpathmoveto{\pgfqpoint{1.680777in}{1.369190in}}%
\pgfpathlineto{\pgfqpoint{1.716142in}{1.363664in}}%
\pgfpathlineto{\pgfqpoint{1.680118in}{1.393367in}}%
\pgfpathlineto{\pgfqpoint{1.644634in}{1.399006in}}%
\pgfpathclose%
\pgfusepath{fill}%
\end{pgfscope}%
\begin{pgfscope}%
\pgfpathrectangle{\pgfqpoint{0.150000in}{0.150000in}}{\pgfqpoint{2.700000in}{1.950000in}}%
\pgfusepath{clip}%
\pgfsetbuttcap%
\pgfsetroundjoin%
\definecolor{currentfill}{rgb}{0.868934,0.762178,0.770634}%
\pgfsetfillcolor{currentfill}%
\pgfsetlinewidth{0.000000pt}%
\definecolor{currentstroke}{rgb}{0.000000,0.000000,0.000000}%
\pgfsetstrokecolor{currentstroke}%
\pgfsetdash{}{0pt}%
\pgfpathmoveto{\pgfqpoint{1.933273in}{1.184635in}}%
\pgfpathlineto{\pgfqpoint{1.967764in}{1.182720in}}%
\pgfpathlineto{\pgfqpoint{1.931567in}{1.212547in}}%
\pgfpathlineto{\pgfqpoint{1.896967in}{1.214571in}}%
\pgfpathclose%
\pgfusepath{fill}%
\end{pgfscope}%
\begin{pgfscope}%
\pgfpathrectangle{\pgfqpoint{0.150000in}{0.150000in}}{\pgfqpoint{2.700000in}{1.950000in}}%
\pgfusepath{clip}%
\pgfsetbuttcap%
\pgfsetroundjoin%
\definecolor{currentfill}{rgb}{0.952512,0.913833,0.916896}%
\pgfsetfillcolor{currentfill}%
\pgfsetlinewidth{0.000000pt}%
\definecolor{currentstroke}{rgb}{0.000000,0.000000,0.000000}%
\pgfsetstrokecolor{currentstroke}%
\pgfsetdash{}{0pt}%
\pgfpathmoveto{\pgfqpoint{1.716967in}{1.339335in}}%
\pgfpathlineto{\pgfqpoint{1.752213in}{1.333923in}}%
\pgfpathlineto{\pgfqpoint{1.716142in}{1.363664in}}%
\pgfpathlineto{\pgfqpoint{1.680777in}{1.369190in}}%
\pgfpathclose%
\pgfusepath{fill}%
\end{pgfscope}%
\begin{pgfscope}%
\pgfpathrectangle{\pgfqpoint{0.150000in}{0.150000in}}{\pgfqpoint{2.700000in}{1.950000in}}%
\pgfusepath{clip}%
\pgfsetbuttcap%
\pgfsetroundjoin%
\definecolor{currentfill}{rgb}{0.857537,0.741498,0.750689}%
\pgfsetfillcolor{currentfill}%
\pgfsetlinewidth{0.000000pt}%
\definecolor{currentstroke}{rgb}{0.000000,0.000000,0.000000}%
\pgfsetstrokecolor{currentstroke}%
\pgfsetdash{}{0pt}%
\pgfpathmoveto{\pgfqpoint{1.969627in}{1.154661in}}%
\pgfpathlineto{\pgfqpoint{2.004007in}{1.152854in}}%
\pgfpathlineto{\pgfqpoint{1.967764in}{1.182720in}}%
\pgfpathlineto{\pgfqpoint{1.933273in}{1.184635in}}%
\pgfpathclose%
\pgfusepath{fill}%
\end{pgfscope}%
\begin{pgfscope}%
\pgfpathrectangle{\pgfqpoint{0.150000in}{0.150000in}}{\pgfqpoint{2.700000in}{1.950000in}}%
\pgfusepath{clip}%
\pgfsetbuttcap%
\pgfsetroundjoin%
\definecolor{currentfill}{rgb}{0.909819,0.920925,0.936474}%
\pgfsetfillcolor{currentfill}%
\pgfsetlinewidth{0.000000pt}%
\definecolor{currentstroke}{rgb}{0.000000,0.000000,0.000000}%
\pgfsetstrokecolor{currentstroke}%
\pgfsetdash{}{0pt}%
\pgfpathmoveto{\pgfqpoint{1.464272in}{1.527015in}}%
\pgfpathlineto{\pgfqpoint{1.500521in}{1.520864in}}%
\pgfpathlineto{\pgfqpoint{1.464602in}{1.550493in}}%
\pgfpathlineto{\pgfqpoint{1.428235in}{1.556757in}}%
\pgfpathclose%
\pgfusepath{fill}%
\end{pgfscope}%
\begin{pgfscope}%
\pgfpathrectangle{\pgfqpoint{0.150000in}{0.150000in}}{\pgfqpoint{2.700000in}{1.950000in}}%
\pgfusepath{clip}%
\pgfsetbuttcap%
\pgfsetroundjoin%
\definecolor{currentfill}{rgb}{0.941115,0.893153,0.896952}%
\pgfsetfillcolor{currentfill}%
\pgfsetlinewidth{0.000000pt}%
\definecolor{currentstroke}{rgb}{0.000000,0.000000,0.000000}%
\pgfsetstrokecolor{currentstroke}%
\pgfsetdash{}{0pt}%
\pgfpathmoveto{\pgfqpoint{1.753145in}{1.306506in}}%
\pgfpathlineto{\pgfqpoint{1.788331in}{1.304143in}}%
\pgfpathlineto{\pgfqpoint{1.752213in}{1.333923in}}%
\pgfpathlineto{\pgfqpoint{1.716967in}{1.339335in}}%
\pgfpathclose%
\pgfusepath{fill}%
\end{pgfscope}%
\begin{pgfscope}%
\pgfpathrectangle{\pgfqpoint{0.150000in}{0.150000in}}{\pgfqpoint{2.700000in}{1.950000in}}%
\pgfusepath{clip}%
\pgfsetbuttcap%
\pgfsetroundjoin%
\definecolor{currentfill}{rgb}{0.846140,0.720818,0.730744}%
\pgfsetfillcolor{currentfill}%
\pgfsetlinewidth{0.000000pt}%
\definecolor{currentstroke}{rgb}{0.000000,0.000000,0.000000}%
\pgfsetstrokecolor{currentstroke}%
\pgfsetdash{}{0pt}%
\pgfpathmoveto{\pgfqpoint{2.006028in}{1.124648in}}%
\pgfpathlineto{\pgfqpoint{2.040297in}{1.122949in}}%
\pgfpathlineto{\pgfqpoint{2.004007in}{1.152854in}}%
\pgfpathlineto{\pgfqpoint{1.969627in}{1.154661in}}%
\pgfpathclose%
\pgfusepath{fill}%
\end{pgfscope}%
\begin{pgfscope}%
\pgfpathrectangle{\pgfqpoint{0.150000in}{0.150000in}}{\pgfqpoint{2.700000in}{1.950000in}}%
\pgfusepath{clip}%
\pgfsetbuttcap%
\pgfsetroundjoin%
\definecolor{currentfill}{rgb}{0.928477,0.937286,0.949617}%
\pgfsetfillcolor{currentfill}%
\pgfsetlinewidth{0.000000pt}%
\definecolor{currentstroke}{rgb}{0.000000,0.000000,0.000000}%
\pgfsetstrokecolor{currentstroke}%
\pgfsetdash{}{0pt}%
\pgfpathmoveto{\pgfqpoint{1.500366in}{1.494247in}}%
\pgfpathlineto{\pgfqpoint{1.536486in}{1.491196in}}%
\pgfpathlineto{\pgfqpoint{1.500521in}{1.520864in}}%
\pgfpathlineto{\pgfqpoint{1.464272in}{1.527015in}}%
\pgfpathclose%
\pgfusepath{fill}%
\end{pgfscope}%
\begin{pgfscope}%
\pgfpathrectangle{\pgfqpoint{0.150000in}{0.150000in}}{\pgfqpoint{2.700000in}{1.950000in}}%
\pgfusepath{clip}%
\pgfsetbuttcap%
\pgfsetroundjoin%
\definecolor{currentfill}{rgb}{0.925919,0.865579,0.870358}%
\pgfsetfillcolor{currentfill}%
\pgfsetlinewidth{0.000000pt}%
\definecolor{currentstroke}{rgb}{0.000000,0.000000,0.000000}%
\pgfsetstrokecolor{currentstroke}%
\pgfsetdash{}{0pt}%
\pgfpathmoveto{\pgfqpoint{1.789420in}{1.276579in}}%
\pgfpathlineto{\pgfqpoint{1.824496in}{1.274325in}}%
\pgfpathlineto{\pgfqpoint{1.788331in}{1.304143in}}%
\pgfpathlineto{\pgfqpoint{1.753145in}{1.306506in}}%
\pgfpathclose%
\pgfusepath{fill}%
\end{pgfscope}%
\begin{pgfscope}%
\pgfpathrectangle{\pgfqpoint{0.150000in}{0.150000in}}{\pgfqpoint{2.700000in}{1.950000in}}%
\pgfusepath{clip}%
\pgfsetbuttcap%
\pgfsetroundjoin%
\definecolor{currentfill}{rgb}{0.830944,0.693244,0.704151}%
\pgfsetfillcolor{currentfill}%
\pgfsetlinewidth{0.000000pt}%
\definecolor{currentstroke}{rgb}{0.000000,0.000000,0.000000}%
\pgfsetstrokecolor{currentstroke}%
\pgfsetdash{}{0pt}%
\pgfpathmoveto{\pgfqpoint{2.042338in}{1.091717in}}%
\pgfpathlineto{\pgfqpoint{2.076635in}{1.093006in}}%
\pgfpathlineto{\pgfqpoint{2.040297in}{1.122949in}}%
\pgfpathlineto{\pgfqpoint{2.006028in}{1.124648in}}%
\pgfpathclose%
\pgfusepath{fill}%
\end{pgfscope}%
\begin{pgfscope}%
\pgfpathrectangle{\pgfqpoint{0.150000in}{0.150000in}}{\pgfqpoint{2.700000in}{1.950000in}}%
\pgfusepath{clip}%
\pgfsetbuttcap%
\pgfsetroundjoin%
\definecolor{currentfill}{rgb}{0.953355,0.959099,0.967142}%
\pgfsetfillcolor{currentfill}%
\pgfsetlinewidth{0.000000pt}%
\definecolor{currentstroke}{rgb}{0.000000,0.000000,0.000000}%
\pgfsetstrokecolor{currentstroke}%
\pgfsetdash{}{0pt}%
\pgfpathmoveto{\pgfqpoint{1.536486in}{1.464433in}}%
\pgfpathlineto{\pgfqpoint{1.572489in}{1.458522in}}%
\pgfpathlineto{\pgfqpoint{1.536486in}{1.491196in}}%
\pgfpathlineto{\pgfqpoint{1.500366in}{1.494247in}}%
\pgfpathclose%
\pgfusepath{fill}%
\end{pgfscope}%
\begin{pgfscope}%
\pgfpathrectangle{\pgfqpoint{0.150000in}{0.150000in}}{\pgfqpoint{2.700000in}{1.950000in}}%
\pgfusepath{clip}%
\pgfsetbuttcap%
\pgfsetroundjoin%
\definecolor{currentfill}{rgb}{0.914522,0.844899,0.850414}%
\pgfsetfillcolor{currentfill}%
\pgfsetlinewidth{0.000000pt}%
\definecolor{currentstroke}{rgb}{0.000000,0.000000,0.000000}%
\pgfsetstrokecolor{currentstroke}%
\pgfsetdash{}{0pt}%
\pgfpathmoveto{\pgfqpoint{1.825742in}{1.246614in}}%
\pgfpathlineto{\pgfqpoint{1.860708in}{1.244467in}}%
\pgfpathlineto{\pgfqpoint{1.824496in}{1.274325in}}%
\pgfpathlineto{\pgfqpoint{1.789420in}{1.276579in}}%
\pgfpathclose%
\pgfusepath{fill}%
\end{pgfscope}%
\begin{pgfscope}%
\pgfpathrectangle{\pgfqpoint{0.150000in}{0.150000in}}{\pgfqpoint{2.700000in}{1.950000in}}%
\pgfusepath{clip}%
\pgfsetbuttcap%
\pgfsetroundjoin%
\definecolor{currentfill}{rgb}{0.819547,0.672564,0.684206}%
\pgfsetfillcolor{currentfill}%
\pgfsetlinewidth{0.000000pt}%
\definecolor{currentstroke}{rgb}{0.000000,0.000000,0.000000}%
\pgfsetstrokecolor{currentstroke}%
\pgfsetdash{}{0pt}%
\pgfpathmoveto{\pgfqpoint{2.078824in}{1.061632in}}%
\pgfpathlineto{\pgfqpoint{2.113019in}{1.063023in}}%
\pgfpathlineto{\pgfqpoint{2.076635in}{1.093006in}}%
\pgfpathlineto{\pgfqpoint{2.042338in}{1.091717in}}%
\pgfpathclose%
\pgfusepath{fill}%
\end{pgfscope}%
\begin{pgfscope}%
\pgfpathrectangle{\pgfqpoint{0.150000in}{0.150000in}}{\pgfqpoint{2.700000in}{1.950000in}}%
\pgfusepath{clip}%
\pgfsetbuttcap%
\pgfsetroundjoin%
\definecolor{currentfill}{rgb}{0.972013,0.975460,0.980285}%
\pgfsetfillcolor{currentfill}%
\pgfsetlinewidth{0.000000pt}%
\definecolor{currentstroke}{rgb}{0.000000,0.000000,0.000000}%
\pgfsetstrokecolor{currentstroke}%
\pgfsetdash{}{0pt}%
\pgfpathmoveto{\pgfqpoint{1.572655in}{1.434581in}}%
\pgfpathlineto{\pgfqpoint{1.608538in}{1.428784in}}%
\pgfpathlineto{\pgfqpoint{1.572489in}{1.458522in}}%
\pgfpathlineto{\pgfqpoint{1.536486in}{1.464433in}}%
\pgfpathclose%
\pgfusepath{fill}%
\end{pgfscope}%
\begin{pgfscope}%
\pgfpathrectangle{\pgfqpoint{0.150000in}{0.150000in}}{\pgfqpoint{2.700000in}{1.950000in}}%
\pgfusepath{clip}%
\pgfsetbuttcap%
\pgfsetroundjoin%
\definecolor{currentfill}{rgb}{0.903125,0.824219,0.830469}%
\pgfsetfillcolor{currentfill}%
\pgfsetlinewidth{0.000000pt}%
\definecolor{currentstroke}{rgb}{0.000000,0.000000,0.000000}%
\pgfsetstrokecolor{currentstroke}%
\pgfsetdash{}{0pt}%
\pgfpathmoveto{\pgfqpoint{1.862112in}{1.216609in}}%
\pgfpathlineto{\pgfqpoint{1.896967in}{1.214571in}}%
\pgfpathlineto{\pgfqpoint{1.860708in}{1.244467in}}%
\pgfpathlineto{\pgfqpoint{1.825742in}{1.246614in}}%
\pgfpathclose%
\pgfusepath{fill}%
\end{pgfscope}%
\begin{pgfscope}%
\pgfpathrectangle{\pgfqpoint{0.150000in}{0.150000in}}{\pgfqpoint{2.700000in}{1.950000in}}%
\pgfusepath{clip}%
\pgfsetbuttcap%
\pgfsetroundjoin%
\definecolor{currentfill}{rgb}{0.996890,0.997273,0.997809}%
\pgfsetfillcolor{currentfill}%
\pgfsetlinewidth{0.000000pt}%
\definecolor{currentstroke}{rgb}{0.000000,0.000000,0.000000}%
\pgfsetstrokecolor{currentstroke}%
\pgfsetdash{}{0pt}%
\pgfpathmoveto{\pgfqpoint{1.608850in}{1.401723in}}%
\pgfpathlineto{\pgfqpoint{1.644634in}{1.399006in}}%
\pgfpathlineto{\pgfqpoint{1.608538in}{1.428784in}}%
\pgfpathlineto{\pgfqpoint{1.572655in}{1.434581in}}%
\pgfpathclose%
\pgfusepath{fill}%
\end{pgfscope}%
\begin{pgfscope}%
\pgfpathrectangle{\pgfqpoint{0.150000in}{0.150000in}}{\pgfqpoint{2.700000in}{1.950000in}}%
\pgfusepath{clip}%
\pgfsetbuttcap%
\pgfsetroundjoin%
\definecolor{currentfill}{rgb}{0.804350,0.644991,0.657613}%
\pgfsetfillcolor{currentfill}%
\pgfsetlinewidth{0.000000pt}%
\definecolor{currentstroke}{rgb}{0.000000,0.000000,0.000000}%
\pgfsetstrokecolor{currentstroke}%
\pgfsetdash{}{0pt}%
\pgfpathmoveto{\pgfqpoint{2.115357in}{1.031507in}}%
\pgfpathlineto{\pgfqpoint{2.149451in}{1.033001in}}%
\pgfpathlineto{\pgfqpoint{2.113019in}{1.063023in}}%
\pgfpathlineto{\pgfqpoint{2.078824in}{1.061632in}}%
\pgfpathclose%
\pgfusepath{fill}%
\end{pgfscope}%
\begin{pgfscope}%
\pgfpathrectangle{\pgfqpoint{0.150000in}{0.150000in}}{\pgfqpoint{2.700000in}{1.950000in}}%
\pgfusepath{clip}%
\pgfsetbuttcap%
\pgfsetroundjoin%
\definecolor{currentfill}{rgb}{0.891728,0.803539,0.810524}%
\pgfsetfillcolor{currentfill}%
\pgfsetlinewidth{0.000000pt}%
\definecolor{currentstroke}{rgb}{0.000000,0.000000,0.000000}%
\pgfsetstrokecolor{currentstroke}%
\pgfsetdash{}{0pt}%
\pgfpathmoveto{\pgfqpoint{1.898429in}{1.183656in}}%
\pgfpathlineto{\pgfqpoint{1.933273in}{1.184635in}}%
\pgfpathlineto{\pgfqpoint{1.896967in}{1.214571in}}%
\pgfpathlineto{\pgfqpoint{1.862112in}{1.216609in}}%
\pgfpathclose%
\pgfusepath{fill}%
\end{pgfscope}%
\begin{pgfscope}%
\pgfpathrectangle{\pgfqpoint{0.150000in}{0.150000in}}{\pgfqpoint{2.700000in}{1.950000in}}%
\pgfusepath{clip}%
\pgfsetbuttcap%
\pgfsetroundjoin%
\definecolor{currentfill}{rgb}{0.990502,0.982767,0.983379}%
\pgfsetfillcolor{currentfill}%
\pgfsetlinewidth{0.000000pt}%
\definecolor{currentstroke}{rgb}{0.000000,0.000000,0.000000}%
\pgfsetstrokecolor{currentstroke}%
\pgfsetdash{}{0pt}%
\pgfpathmoveto{\pgfqpoint{1.645103in}{1.371799in}}%
\pgfpathlineto{\pgfqpoint{1.680777in}{1.369190in}}%
\pgfpathlineto{\pgfqpoint{1.644634in}{1.399006in}}%
\pgfpathlineto{\pgfqpoint{1.608850in}{1.401723in}}%
\pgfpathclose%
\pgfusepath{fill}%
\end{pgfscope}%
\begin{pgfscope}%
\pgfpathrectangle{\pgfqpoint{0.150000in}{0.150000in}}{\pgfqpoint{2.700000in}{1.950000in}}%
\pgfusepath{clip}%
\pgfsetbuttcap%
\pgfsetroundjoin%
\definecolor{currentfill}{rgb}{0.792953,0.624311,0.637669}%
\pgfsetfillcolor{currentfill}%
\pgfsetlinewidth{0.000000pt}%
\definecolor{currentstroke}{rgb}{0.000000,0.000000,0.000000}%
\pgfsetstrokecolor{currentstroke}%
\pgfsetdash{}{0pt}%
\pgfpathmoveto{\pgfqpoint{2.151939in}{1.001343in}}%
\pgfpathlineto{\pgfqpoint{2.185931in}{1.002940in}}%
\pgfpathlineto{\pgfqpoint{2.149451in}{1.033001in}}%
\pgfpathlineto{\pgfqpoint{2.115357in}{1.031507in}}%
\pgfpathclose%
\pgfusepath{fill}%
\end{pgfscope}%
\begin{pgfscope}%
\pgfpathrectangle{\pgfqpoint{0.150000in}{0.150000in}}{\pgfqpoint{2.700000in}{1.950000in}}%
\pgfusepath{clip}%
\pgfsetbuttcap%
\pgfsetroundjoin%
\definecolor{currentfill}{rgb}{0.876532,0.775965,0.783931}%
\pgfsetfillcolor{currentfill}%
\pgfsetlinewidth{0.000000pt}%
\definecolor{currentstroke}{rgb}{0.000000,0.000000,0.000000}%
\pgfsetstrokecolor{currentstroke}%
\pgfsetdash{}{0pt}%
\pgfpathmoveto{\pgfqpoint{1.934884in}{1.153579in}}%
\pgfpathlineto{\pgfqpoint{1.969627in}{1.154661in}}%
\pgfpathlineto{\pgfqpoint{1.933273in}{1.184635in}}%
\pgfpathlineto{\pgfqpoint{1.898429in}{1.183656in}}%
\pgfpathclose%
\pgfusepath{fill}%
\end{pgfscope}%
\begin{pgfscope}%
\pgfpathrectangle{\pgfqpoint{0.150000in}{0.150000in}}{\pgfqpoint{2.700000in}{1.950000in}}%
\pgfusepath{clip}%
\pgfsetbuttcap%
\pgfsetroundjoin%
\definecolor{currentfill}{rgb}{0.975306,0.955193,0.956786}%
\pgfsetfillcolor{currentfill}%
\pgfsetlinewidth{0.000000pt}%
\definecolor{currentstroke}{rgb}{0.000000,0.000000,0.000000}%
\pgfsetstrokecolor{currentstroke}%
\pgfsetdash{}{0pt}%
\pgfpathmoveto{\pgfqpoint{1.681403in}{1.341836in}}%
\pgfpathlineto{\pgfqpoint{1.716967in}{1.339335in}}%
\pgfpathlineto{\pgfqpoint{1.680777in}{1.369190in}}%
\pgfpathlineto{\pgfqpoint{1.645103in}{1.371799in}}%
\pgfpathclose%
\pgfusepath{fill}%
\end{pgfscope}%
\begin{pgfscope}%
\pgfpathrectangle{\pgfqpoint{0.150000in}{0.150000in}}{\pgfqpoint{2.700000in}{1.950000in}}%
\pgfusepath{clip}%
\pgfsetbuttcap%
\pgfsetroundjoin%
\definecolor{currentfill}{rgb}{0.781556,0.603631,0.617724}%
\pgfsetfillcolor{currentfill}%
\pgfsetlinewidth{0.000000pt}%
\definecolor{currentstroke}{rgb}{0.000000,0.000000,0.000000}%
\pgfsetstrokecolor{currentstroke}%
\pgfsetdash{}{0pt}%
\pgfpathmoveto{\pgfqpoint{2.188568in}{0.971139in}}%
\pgfpathlineto{\pgfqpoint{2.222458in}{0.972840in}}%
\pgfpathlineto{\pgfqpoint{2.185931in}{1.002940in}}%
\pgfpathlineto{\pgfqpoint{2.151939in}{1.001343in}}%
\pgfpathclose%
\pgfusepath{fill}%
\end{pgfscope}%
\begin{pgfscope}%
\pgfpathrectangle{\pgfqpoint{0.150000in}{0.150000in}}{\pgfqpoint{2.700000in}{1.950000in}}%
\pgfusepath{clip}%
\pgfsetbuttcap%
\pgfsetroundjoin%
\definecolor{currentfill}{rgb}{0.866284,0.882751,0.905806}%
\pgfsetfillcolor{currentfill}%
\pgfsetlinewidth{0.000000pt}%
\definecolor{currentstroke}{rgb}{0.000000,0.000000,0.000000}%
\pgfsetstrokecolor{currentstroke}%
\pgfsetdash{}{0pt}%
\pgfpathmoveto{\pgfqpoint{1.427736in}{1.533215in}}%
\pgfpathlineto{\pgfqpoint{1.464272in}{1.527015in}}%
\pgfpathlineto{\pgfqpoint{1.428235in}{1.556757in}}%
\pgfpathlineto{\pgfqpoint{1.391581in}{1.563070in}}%
\pgfpathclose%
\pgfusepath{fill}%
\end{pgfscope}%
\begin{pgfscope}%
\pgfpathrectangle{\pgfqpoint{0.150000in}{0.150000in}}{\pgfqpoint{2.700000in}{1.950000in}}%
\pgfusepath{clip}%
\pgfsetbuttcap%
\pgfsetroundjoin%
\definecolor{currentfill}{rgb}{0.865135,0.755285,0.763986}%
\pgfsetfillcolor{currentfill}%
\pgfsetlinewidth{0.000000pt}%
\definecolor{currentstroke}{rgb}{0.000000,0.000000,0.000000}%
\pgfsetstrokecolor{currentstroke}%
\pgfsetdash{}{0pt}%
\pgfpathmoveto{\pgfqpoint{1.971386in}{1.123463in}}%
\pgfpathlineto{\pgfqpoint{2.006028in}{1.124648in}}%
\pgfpathlineto{\pgfqpoint{1.969627in}{1.154661in}}%
\pgfpathlineto{\pgfqpoint{1.934884in}{1.153579in}}%
\pgfpathclose%
\pgfusepath{fill}%
\end{pgfscope}%
\begin{pgfscope}%
\pgfpathrectangle{\pgfqpoint{0.150000in}{0.150000in}}{\pgfqpoint{2.700000in}{1.950000in}}%
\pgfusepath{clip}%
\pgfsetbuttcap%
\pgfsetroundjoin%
\definecolor{currentfill}{rgb}{0.963909,0.934513,0.936841}%
\pgfsetfillcolor{currentfill}%
\pgfsetlinewidth{0.000000pt}%
\definecolor{currentstroke}{rgb}{0.000000,0.000000,0.000000}%
\pgfsetstrokecolor{currentstroke}%
\pgfsetdash{}{0pt}%
\pgfpathmoveto{\pgfqpoint{1.717701in}{1.308886in}}%
\pgfpathlineto{\pgfqpoint{1.753145in}{1.306506in}}%
\pgfpathlineto{\pgfqpoint{1.716967in}{1.339335in}}%
\pgfpathlineto{\pgfqpoint{1.681403in}{1.341836in}}%
\pgfpathclose%
\pgfusepath{fill}%
\end{pgfscope}%
\begin{pgfscope}%
\pgfpathrectangle{\pgfqpoint{0.150000in}{0.150000in}}{\pgfqpoint{2.700000in}{1.950000in}}%
\pgfusepath{clip}%
\pgfsetbuttcap%
\pgfsetroundjoin%
\definecolor{currentfill}{rgb}{0.770159,0.582950,0.597779}%
\pgfsetfillcolor{currentfill}%
\pgfsetlinewidth{0.000000pt}%
\definecolor{currentstroke}{rgb}{0.000000,0.000000,0.000000}%
\pgfsetstrokecolor{currentstroke}%
\pgfsetdash{}{0pt}%
\pgfpathmoveto{\pgfqpoint{2.225246in}{0.940896in}}%
\pgfpathlineto{\pgfqpoint{2.259033in}{0.942701in}}%
\pgfpathlineto{\pgfqpoint{2.222458in}{0.972840in}}%
\pgfpathlineto{\pgfqpoint{2.188568in}{0.971139in}}%
\pgfpathclose%
\pgfusepath{fill}%
\end{pgfscope}%
\begin{pgfscope}%
\pgfpathrectangle{\pgfqpoint{0.150000in}{0.150000in}}{\pgfqpoint{2.700000in}{1.950000in}}%
\pgfusepath{clip}%
\pgfsetbuttcap%
\pgfsetroundjoin%
\definecolor{currentfill}{rgb}{0.891161,0.904565,0.923330}%
\pgfsetfillcolor{currentfill}%
\pgfsetlinewidth{0.000000pt}%
\definecolor{currentstroke}{rgb}{0.000000,0.000000,0.000000}%
\pgfsetstrokecolor{currentstroke}%
\pgfsetdash{}{0pt}%
\pgfpathmoveto{\pgfqpoint{1.463959in}{1.500319in}}%
\pgfpathlineto{\pgfqpoint{1.500366in}{1.494247in}}%
\pgfpathlineto{\pgfqpoint{1.464272in}{1.527015in}}%
\pgfpathlineto{\pgfqpoint{1.427736in}{1.533215in}}%
\pgfpathclose%
\pgfusepath{fill}%
\end{pgfscope}%
\begin{pgfscope}%
\pgfpathrectangle{\pgfqpoint{0.150000in}{0.150000in}}{\pgfqpoint{2.700000in}{1.950000in}}%
\pgfusepath{clip}%
\pgfsetbuttcap%
\pgfsetroundjoin%
\definecolor{currentfill}{rgb}{0.948713,0.906939,0.910248}%
\pgfsetfillcolor{currentfill}%
\pgfsetlinewidth{0.000000pt}%
\definecolor{currentstroke}{rgb}{0.000000,0.000000,0.000000}%
\pgfsetstrokecolor{currentstroke}%
\pgfsetdash{}{0pt}%
\pgfpathmoveto{\pgfqpoint{1.754086in}{1.278851in}}%
\pgfpathlineto{\pgfqpoint{1.789420in}{1.276579in}}%
\pgfpathlineto{\pgfqpoint{1.753145in}{1.306506in}}%
\pgfpathlineto{\pgfqpoint{1.717701in}{1.308886in}}%
\pgfpathclose%
\pgfusepath{fill}%
\end{pgfscope}%
\begin{pgfscope}%
\pgfpathrectangle{\pgfqpoint{0.150000in}{0.150000in}}{\pgfqpoint{2.700000in}{1.950000in}}%
\pgfusepath{clip}%
\pgfsetbuttcap%
\pgfsetroundjoin%
\definecolor{currentfill}{rgb}{0.849939,0.727711,0.737393}%
\pgfsetfillcolor{currentfill}%
\pgfsetlinewidth{0.000000pt}%
\definecolor{currentstroke}{rgb}{0.000000,0.000000,0.000000}%
\pgfsetstrokecolor{currentstroke}%
\pgfsetdash{}{0pt}%
\pgfpathmoveto{\pgfqpoint{2.007937in}{1.093307in}}%
\pgfpathlineto{\pgfqpoint{2.042338in}{1.091717in}}%
\pgfpathlineto{\pgfqpoint{2.006028in}{1.124648in}}%
\pgfpathlineto{\pgfqpoint{1.971386in}{1.123463in}}%
\pgfpathclose%
\pgfusepath{fill}%
\end{pgfscope}%
\begin{pgfscope}%
\pgfpathrectangle{\pgfqpoint{0.150000in}{0.150000in}}{\pgfqpoint{2.700000in}{1.950000in}}%
\pgfusepath{clip}%
\pgfsetbuttcap%
\pgfsetroundjoin%
\definecolor{currentfill}{rgb}{0.754963,0.555377,0.571186}%
\pgfsetfillcolor{currentfill}%
\pgfsetlinewidth{0.000000pt}%
\definecolor{currentstroke}{rgb}{0.000000,0.000000,0.000000}%
\pgfsetstrokecolor{currentstroke}%
\pgfsetdash{}{0pt}%
\pgfpathmoveto{\pgfqpoint{2.261971in}{0.910613in}}%
\pgfpathlineto{\pgfqpoint{2.295656in}{0.912522in}}%
\pgfpathlineto{\pgfqpoint{2.259033in}{0.942701in}}%
\pgfpathlineto{\pgfqpoint{2.225246in}{0.940896in}}%
\pgfpathclose%
\pgfusepath{fill}%
\end{pgfscope}%
\begin{pgfscope}%
\pgfpathrectangle{\pgfqpoint{0.150000in}{0.150000in}}{\pgfqpoint{2.700000in}{1.950000in}}%
\pgfusepath{clip}%
\pgfsetbuttcap%
\pgfsetroundjoin%
\definecolor{currentfill}{rgb}{0.916039,0.926379,0.940855}%
\pgfsetfillcolor{currentfill}%
\pgfsetlinewidth{0.000000pt}%
\definecolor{currentstroke}{rgb}{0.000000,0.000000,0.000000}%
\pgfsetstrokecolor{currentstroke}%
\pgfsetdash{}{0pt}%
\pgfpathmoveto{\pgfqpoint{1.500199in}{1.470392in}}%
\pgfpathlineto{\pgfqpoint{1.536486in}{1.464433in}}%
\pgfpathlineto{\pgfqpoint{1.500366in}{1.494247in}}%
\pgfpathlineto{\pgfqpoint{1.463959in}{1.500319in}}%
\pgfpathclose%
\pgfusepath{fill}%
\end{pgfscope}%
\begin{pgfscope}%
\pgfpathrectangle{\pgfqpoint{0.150000in}{0.150000in}}{\pgfqpoint{2.700000in}{1.950000in}}%
\pgfusepath{clip}%
\pgfsetbuttcap%
\pgfsetroundjoin%
\definecolor{currentfill}{rgb}{0.838542,0.707031,0.717448}%
\pgfsetfillcolor{currentfill}%
\pgfsetlinewidth{0.000000pt}%
\definecolor{currentstroke}{rgb}{0.000000,0.000000,0.000000}%
\pgfsetstrokecolor{currentstroke}%
\pgfsetdash{}{0pt}%
\pgfpathmoveto{\pgfqpoint{2.044535in}{1.063112in}}%
\pgfpathlineto{\pgfqpoint{2.078824in}{1.061632in}}%
\pgfpathlineto{\pgfqpoint{2.042338in}{1.091717in}}%
\pgfpathlineto{\pgfqpoint{2.007937in}{1.093307in}}%
\pgfpathclose%
\pgfusepath{fill}%
\end{pgfscope}%
\begin{pgfscope}%
\pgfpathrectangle{\pgfqpoint{0.150000in}{0.150000in}}{\pgfqpoint{2.700000in}{1.950000in}}%
\pgfusepath{clip}%
\pgfsetbuttcap%
\pgfsetroundjoin%
\definecolor{currentfill}{rgb}{0.937316,0.886259,0.890303}%
\pgfsetfillcolor{currentfill}%
\pgfsetlinewidth{0.000000pt}%
\definecolor{currentstroke}{rgb}{0.000000,0.000000,0.000000}%
\pgfsetstrokecolor{currentstroke}%
\pgfsetdash{}{0pt}%
\pgfpathmoveto{\pgfqpoint{1.790519in}{1.248776in}}%
\pgfpathlineto{\pgfqpoint{1.825742in}{1.246614in}}%
\pgfpathlineto{\pgfqpoint{1.789420in}{1.276579in}}%
\pgfpathlineto{\pgfqpoint{1.754086in}{1.278851in}}%
\pgfpathclose%
\pgfusepath{fill}%
\end{pgfscope}%
\begin{pgfscope}%
\pgfpathrectangle{\pgfqpoint{0.150000in}{0.150000in}}{\pgfqpoint{2.700000in}{1.950000in}}%
\pgfusepath{clip}%
\pgfsetbuttcap%
\pgfsetroundjoin%
\definecolor{currentfill}{rgb}{0.743566,0.534697,0.551241}%
\pgfsetfillcolor{currentfill}%
\pgfsetlinewidth{0.000000pt}%
\definecolor{currentstroke}{rgb}{0.000000,0.000000,0.000000}%
\pgfsetstrokecolor{currentstroke}%
\pgfsetdash{}{0pt}%
\pgfpathmoveto{\pgfqpoint{2.298745in}{0.880291in}}%
\pgfpathlineto{\pgfqpoint{2.332326in}{0.882304in}}%
\pgfpathlineto{\pgfqpoint{2.295656in}{0.912522in}}%
\pgfpathlineto{\pgfqpoint{2.261971in}{0.910613in}}%
\pgfpathclose%
\pgfusepath{fill}%
\end{pgfscope}%
\begin{pgfscope}%
\pgfpathrectangle{\pgfqpoint{0.150000in}{0.150000in}}{\pgfqpoint{2.700000in}{1.950000in}}%
\pgfusepath{clip}%
\pgfsetbuttcap%
\pgfsetroundjoin%
\definecolor{currentfill}{rgb}{0.934697,0.942739,0.953998}%
\pgfsetfillcolor{currentfill}%
\pgfsetlinewidth{0.000000pt}%
\definecolor{currentstroke}{rgb}{0.000000,0.000000,0.000000}%
\pgfsetstrokecolor{currentstroke}%
\pgfsetdash{}{0pt}%
\pgfpathmoveto{\pgfqpoint{1.536486in}{1.437438in}}%
\pgfpathlineto{\pgfqpoint{1.572655in}{1.434581in}}%
\pgfpathlineto{\pgfqpoint{1.536486in}{1.464433in}}%
\pgfpathlineto{\pgfqpoint{1.500199in}{1.470392in}}%
\pgfpathclose%
\pgfusepath{fill}%
\end{pgfscope}%
\begin{pgfscope}%
\pgfpathrectangle{\pgfqpoint{0.150000in}{0.150000in}}{\pgfqpoint{2.700000in}{1.950000in}}%
\pgfusepath{clip}%
\pgfsetbuttcap%
\pgfsetroundjoin%
\definecolor{currentfill}{rgb}{0.827145,0.686351,0.697503}%
\pgfsetfillcolor{currentfill}%
\pgfsetlinewidth{0.000000pt}%
\definecolor{currentstroke}{rgb}{0.000000,0.000000,0.000000}%
\pgfsetstrokecolor{currentstroke}%
\pgfsetdash{}{0pt}%
\pgfpathmoveto{\pgfqpoint{2.081031in}{1.030002in}}%
\pgfpathlineto{\pgfqpoint{2.115357in}{1.031507in}}%
\pgfpathlineto{\pgfqpoint{2.078824in}{1.061632in}}%
\pgfpathlineto{\pgfqpoint{2.044535in}{1.063112in}}%
\pgfpathclose%
\pgfusepath{fill}%
\end{pgfscope}%
\begin{pgfscope}%
\pgfpathrectangle{\pgfqpoint{0.150000in}{0.150000in}}{\pgfqpoint{2.700000in}{1.950000in}}%
\pgfusepath{clip}%
\pgfsetbuttcap%
\pgfsetroundjoin%
\definecolor{currentfill}{rgb}{0.922120,0.858686,0.863710}%
\pgfsetfillcolor{currentfill}%
\pgfsetlinewidth{0.000000pt}%
\definecolor{currentstroke}{rgb}{0.000000,0.000000,0.000000}%
\pgfsetstrokecolor{currentstroke}%
\pgfsetdash{}{0pt}%
\pgfpathmoveto{\pgfqpoint{1.826919in}{1.215736in}}%
\pgfpathlineto{\pgfqpoint{1.862112in}{1.216609in}}%
\pgfpathlineto{\pgfqpoint{1.825742in}{1.246614in}}%
\pgfpathlineto{\pgfqpoint{1.790519in}{1.248776in}}%
\pgfpathclose%
\pgfusepath{fill}%
\end{pgfscope}%
\begin{pgfscope}%
\pgfpathrectangle{\pgfqpoint{0.150000in}{0.150000in}}{\pgfqpoint{2.700000in}{1.950000in}}%
\pgfusepath{clip}%
\pgfsetbuttcap%
\pgfsetroundjoin%
\definecolor{currentfill}{rgb}{0.959574,0.964553,0.971523}%
\pgfsetfillcolor{currentfill}%
\pgfsetlinewidth{0.000000pt}%
\definecolor{currentstroke}{rgb}{0.000000,0.000000,0.000000}%
\pgfsetstrokecolor{currentstroke}%
\pgfsetdash{}{0pt}%
\pgfpathmoveto{\pgfqpoint{1.572812in}{1.407438in}}%
\pgfpathlineto{\pgfqpoint{1.608850in}{1.401723in}}%
\pgfpathlineto{\pgfqpoint{1.572655in}{1.434581in}}%
\pgfpathlineto{\pgfqpoint{1.536486in}{1.437438in}}%
\pgfpathclose%
\pgfusepath{fill}%
\end{pgfscope}%
\begin{pgfscope}%
\pgfpathrectangle{\pgfqpoint{0.150000in}{0.150000in}}{\pgfqpoint{2.700000in}{1.950000in}}%
\pgfusepath{clip}%
\pgfsetbuttcap%
\pgfsetroundjoin%
\definecolor{currentfill}{rgb}{0.732169,0.514017,0.531296}%
\pgfsetfillcolor{currentfill}%
\pgfsetlinewidth{0.000000pt}%
\definecolor{currentstroke}{rgb}{0.000000,0.000000,0.000000}%
\pgfsetstrokecolor{currentstroke}%
\pgfsetdash{}{0pt}%
\pgfpathmoveto{\pgfqpoint{2.335567in}{0.849928in}}%
\pgfpathlineto{\pgfqpoint{2.369045in}{0.852047in}}%
\pgfpathlineto{\pgfqpoint{2.332326in}{0.882304in}}%
\pgfpathlineto{\pgfqpoint{2.298745in}{0.880291in}}%
\pgfpathclose%
\pgfusepath{fill}%
\end{pgfscope}%
\begin{pgfscope}%
\pgfpathrectangle{\pgfqpoint{0.150000in}{0.150000in}}{\pgfqpoint{2.700000in}{1.950000in}}%
\pgfusepath{clip}%
\pgfsetbuttcap%
\pgfsetroundjoin%
\definecolor{currentfill}{rgb}{0.910723,0.838006,0.843765}%
\pgfsetfillcolor{currentfill}%
\pgfsetlinewidth{0.000000pt}%
\definecolor{currentstroke}{rgb}{0.000000,0.000000,0.000000}%
\pgfsetstrokecolor{currentstroke}%
\pgfsetdash{}{0pt}%
\pgfpathmoveto{\pgfqpoint{1.863438in}{1.185589in}}%
\pgfpathlineto{\pgfqpoint{1.898429in}{1.183656in}}%
\pgfpathlineto{\pgfqpoint{1.862112in}{1.216609in}}%
\pgfpathlineto{\pgfqpoint{1.826919in}{1.215736in}}%
\pgfpathclose%
\pgfusepath{fill}%
\end{pgfscope}%
\begin{pgfscope}%
\pgfpathrectangle{\pgfqpoint{0.150000in}{0.150000in}}{\pgfqpoint{2.700000in}{1.950000in}}%
\pgfusepath{clip}%
\pgfsetbuttcap%
\pgfsetroundjoin%
\definecolor{currentfill}{rgb}{0.811949,0.658778,0.670910}%
\pgfsetfillcolor{currentfill}%
\pgfsetlinewidth{0.000000pt}%
\definecolor{currentstroke}{rgb}{0.000000,0.000000,0.000000}%
\pgfsetstrokecolor{currentstroke}%
\pgfsetdash{}{0pt}%
\pgfpathmoveto{\pgfqpoint{2.117715in}{0.999734in}}%
\pgfpathlineto{\pgfqpoint{2.151939in}{1.001343in}}%
\pgfpathlineto{\pgfqpoint{2.115357in}{1.031507in}}%
\pgfpathlineto{\pgfqpoint{2.081031in}{1.030002in}}%
\pgfpathclose%
\pgfusepath{fill}%
\end{pgfscope}%
\begin{pgfscope}%
\pgfpathrectangle{\pgfqpoint{0.150000in}{0.150000in}}{\pgfqpoint{2.700000in}{1.950000in}}%
\pgfusepath{clip}%
\pgfsetbuttcap%
\pgfsetroundjoin%
\definecolor{currentfill}{rgb}{0.978232,0.980913,0.984666}%
\pgfsetfillcolor{currentfill}%
\pgfsetlinewidth{0.000000pt}%
\definecolor{currentstroke}{rgb}{0.000000,0.000000,0.000000}%
\pgfsetstrokecolor{currentstroke}%
\pgfsetdash{}{0pt}%
\pgfpathmoveto{\pgfqpoint{1.609165in}{1.374426in}}%
\pgfpathlineto{\pgfqpoint{1.645103in}{1.371799in}}%
\pgfpathlineto{\pgfqpoint{1.608850in}{1.401723in}}%
\pgfpathlineto{\pgfqpoint{1.572812in}{1.407438in}}%
\pgfpathclose%
\pgfusepath{fill}%
\end{pgfscope}%
\begin{pgfscope}%
\pgfpathrectangle{\pgfqpoint{0.150000in}{0.150000in}}{\pgfqpoint{2.700000in}{1.950000in}}%
\pgfusepath{clip}%
\pgfsetbuttcap%
\pgfsetroundjoin%
\definecolor{currentfill}{rgb}{0.720772,0.493336,0.511351}%
\pgfsetfillcolor{currentfill}%
\pgfsetlinewidth{0.000000pt}%
\definecolor{currentstroke}{rgb}{0.000000,0.000000,0.000000}%
\pgfsetstrokecolor{currentstroke}%
\pgfsetdash{}{0pt}%
\pgfpathmoveto{\pgfqpoint{2.372206in}{0.816708in}}%
\pgfpathlineto{\pgfqpoint{2.405811in}{0.821750in}}%
\pgfpathlineto{\pgfqpoint{2.369045in}{0.852047in}}%
\pgfpathlineto{\pgfqpoint{2.335567in}{0.849928in}}%
\pgfpathclose%
\pgfusepath{fill}%
\end{pgfscope}%
\begin{pgfscope}%
\pgfpathrectangle{\pgfqpoint{0.150000in}{0.150000in}}{\pgfqpoint{2.700000in}{1.950000in}}%
\pgfusepath{clip}%
\pgfsetbuttcap%
\pgfsetroundjoin%
\definecolor{currentfill}{rgb}{0.895527,0.810432,0.817172}%
\pgfsetfillcolor{currentfill}%
\pgfsetlinewidth{0.000000pt}%
\definecolor{currentstroke}{rgb}{0.000000,0.000000,0.000000}%
\pgfsetstrokecolor{currentstroke}%
\pgfsetdash{}{0pt}%
\pgfpathmoveto{\pgfqpoint{1.900004in}{1.155402in}}%
\pgfpathlineto{\pgfqpoint{1.934884in}{1.153579in}}%
\pgfpathlineto{\pgfqpoint{1.898429in}{1.183656in}}%
\pgfpathlineto{\pgfqpoint{1.863438in}{1.185589in}}%
\pgfpathclose%
\pgfusepath{fill}%
\end{pgfscope}%
\begin{pgfscope}%
\pgfpathrectangle{\pgfqpoint{0.150000in}{0.150000in}}{\pgfqpoint{2.700000in}{1.950000in}}%
\pgfusepath{clip}%
\pgfsetbuttcap%
\pgfsetroundjoin%
\definecolor{currentfill}{rgb}{0.800551,0.638097,0.650965}%
\pgfsetfillcolor{currentfill}%
\pgfsetlinewidth{0.000000pt}%
\definecolor{currentstroke}{rgb}{0.000000,0.000000,0.000000}%
\pgfsetstrokecolor{currentstroke}%
\pgfsetdash{}{0pt}%
\pgfpathmoveto{\pgfqpoint{2.154447in}{0.969426in}}%
\pgfpathlineto{\pgfqpoint{2.188568in}{0.971139in}}%
\pgfpathlineto{\pgfqpoint{2.151939in}{1.001343in}}%
\pgfpathlineto{\pgfqpoint{2.117715in}{0.999734in}}%
\pgfpathclose%
\pgfusepath{fill}%
\end{pgfscope}%
\begin{pgfscope}%
\pgfpathrectangle{\pgfqpoint{0.150000in}{0.150000in}}{\pgfqpoint{2.700000in}{1.950000in}}%
\pgfusepath{clip}%
\pgfsetbuttcap%
\pgfsetroundjoin%
\definecolor{currentfill}{rgb}{0.998100,0.996553,0.996676}%
\pgfsetfillcolor{currentfill}%
\pgfsetlinewidth{0.000000pt}%
\definecolor{currentstroke}{rgb}{0.000000,0.000000,0.000000}%
\pgfsetstrokecolor{currentstroke}%
\pgfsetdash{}{0pt}%
\pgfpathmoveto{\pgfqpoint{1.645575in}{1.344354in}}%
\pgfpathlineto{\pgfqpoint{1.681403in}{1.341836in}}%
\pgfpathlineto{\pgfqpoint{1.645103in}{1.371799in}}%
\pgfpathlineto{\pgfqpoint{1.609165in}{1.374426in}}%
\pgfpathclose%
\pgfusepath{fill}%
\end{pgfscope}%
\begin{pgfscope}%
\pgfpathrectangle{\pgfqpoint{0.150000in}{0.150000in}}{\pgfqpoint{2.700000in}{1.950000in}}%
\pgfusepath{clip}%
\pgfsetbuttcap%
\pgfsetroundjoin%
\definecolor{currentfill}{rgb}{0.705576,0.465763,0.484758}%
\pgfsetfillcolor{currentfill}%
\pgfsetlinewidth{0.000000pt}%
\definecolor{currentstroke}{rgb}{0.000000,0.000000,0.000000}%
\pgfsetstrokecolor{currentstroke}%
\pgfsetdash{}{0pt}%
\pgfpathmoveto{\pgfqpoint{2.409114in}{0.786272in}}%
\pgfpathlineto{\pgfqpoint{2.442626in}{0.791413in}}%
\pgfpathlineto{\pgfqpoint{2.405811in}{0.821750in}}%
\pgfpathlineto{\pgfqpoint{2.372206in}{0.816708in}}%
\pgfpathclose%
\pgfusepath{fill}%
\end{pgfscope}%
\begin{pgfscope}%
\pgfpathrectangle{\pgfqpoint{0.150000in}{0.150000in}}{\pgfqpoint{2.700000in}{1.950000in}}%
\pgfusepath{clip}%
\pgfsetbuttcap%
\pgfsetroundjoin%
\definecolor{currentfill}{rgb}{0.884130,0.789752,0.797227}%
\pgfsetfillcolor{currentfill}%
\pgfsetlinewidth{0.000000pt}%
\definecolor{currentstroke}{rgb}{0.000000,0.000000,0.000000}%
\pgfsetstrokecolor{currentstroke}%
\pgfsetdash{}{0pt}%
\pgfpathmoveto{\pgfqpoint{1.936619in}{1.125175in}}%
\pgfpathlineto{\pgfqpoint{1.971386in}{1.123463in}}%
\pgfpathlineto{\pgfqpoint{1.934884in}{1.153579in}}%
\pgfpathlineto{\pgfqpoint{1.900004in}{1.155402in}}%
\pgfpathclose%
\pgfusepath{fill}%
\end{pgfscope}%
\begin{pgfscope}%
\pgfpathrectangle{\pgfqpoint{0.150000in}{0.150000in}}{\pgfqpoint{2.700000in}{1.950000in}}%
\pgfusepath{clip}%
\pgfsetbuttcap%
\pgfsetroundjoin%
\definecolor{currentfill}{rgb}{0.789154,0.617417,0.631020}%
\pgfsetfillcolor{currentfill}%
\pgfsetlinewidth{0.000000pt}%
\definecolor{currentstroke}{rgb}{0.000000,0.000000,0.000000}%
\pgfsetstrokecolor{currentstroke}%
\pgfsetdash{}{0pt}%
\pgfpathmoveto{\pgfqpoint{2.191227in}{0.939078in}}%
\pgfpathlineto{\pgfqpoint{2.225246in}{0.940896in}}%
\pgfpathlineto{\pgfqpoint{2.188568in}{0.971139in}}%
\pgfpathlineto{\pgfqpoint{2.154447in}{0.969426in}}%
\pgfpathclose%
\pgfusepath{fill}%
\end{pgfscope}%
\begin{pgfscope}%
\pgfpathrectangle{\pgfqpoint{0.150000in}{0.150000in}}{\pgfqpoint{2.700000in}{1.950000in}}%
\pgfusepath{clip}%
\pgfsetbuttcap%
\pgfsetroundjoin%
\definecolor{currentfill}{rgb}{0.986703,0.975873,0.976731}%
\pgfsetfillcolor{currentfill}%
\pgfsetlinewidth{0.000000pt}%
\definecolor{currentstroke}{rgb}{0.000000,0.000000,0.000000}%
\pgfsetstrokecolor{currentstroke}%
\pgfsetdash{}{0pt}%
\pgfpathmoveto{\pgfqpoint{1.682034in}{1.314243in}}%
\pgfpathlineto{\pgfqpoint{1.717701in}{1.308886in}}%
\pgfpathlineto{\pgfqpoint{1.681403in}{1.341836in}}%
\pgfpathlineto{\pgfqpoint{1.645575in}{1.344354in}}%
\pgfpathclose%
\pgfusepath{fill}%
\end{pgfscope}%
\begin{pgfscope}%
\pgfpathrectangle{\pgfqpoint{0.150000in}{0.150000in}}{\pgfqpoint{2.700000in}{1.950000in}}%
\pgfusepath{clip}%
\pgfsetbuttcap%
\pgfsetroundjoin%
\definecolor{currentfill}{rgb}{0.828968,0.850031,0.879519}%
\pgfsetfillcolor{currentfill}%
\pgfsetlinewidth{0.000000pt}%
\definecolor{currentstroke}{rgb}{0.000000,0.000000,0.000000}%
\pgfsetstrokecolor{currentstroke}%
\pgfsetdash{}{0pt}%
\pgfpathmoveto{\pgfqpoint{1.390949in}{1.536443in}}%
\pgfpathlineto{\pgfqpoint{1.427736in}{1.533215in}}%
\pgfpathlineto{\pgfqpoint{1.391581in}{1.563070in}}%
\pgfpathlineto{\pgfqpoint{1.354634in}{1.569434in}}%
\pgfpathclose%
\pgfusepath{fill}%
\end{pgfscope}%
\begin{pgfscope}%
\pgfpathrectangle{\pgfqpoint{0.150000in}{0.150000in}}{\pgfqpoint{2.700000in}{1.950000in}}%
\pgfusepath{clip}%
\pgfsetbuttcap%
\pgfsetroundjoin%
\definecolor{currentfill}{rgb}{0.872733,0.769072,0.777282}%
\pgfsetfillcolor{currentfill}%
\pgfsetlinewidth{0.000000pt}%
\definecolor{currentstroke}{rgb}{0.000000,0.000000,0.000000}%
\pgfsetstrokecolor{currentstroke}%
\pgfsetdash{}{0pt}%
\pgfpathmoveto{\pgfqpoint{1.973160in}{1.092010in}}%
\pgfpathlineto{\pgfqpoint{2.007937in}{1.093307in}}%
\pgfpathlineto{\pgfqpoint{1.971386in}{1.123463in}}%
\pgfpathlineto{\pgfqpoint{1.936619in}{1.125175in}}%
\pgfpathclose%
\pgfusepath{fill}%
\end{pgfscope}%
\begin{pgfscope}%
\pgfpathrectangle{\pgfqpoint{0.150000in}{0.150000in}}{\pgfqpoint{2.700000in}{1.950000in}}%
\pgfusepath{clip}%
\pgfsetbuttcap%
\pgfsetroundjoin%
\definecolor{currentfill}{rgb}{0.971507,0.948300,0.950138}%
\pgfsetfillcolor{currentfill}%
\pgfsetlinewidth{0.000000pt}%
\definecolor{currentstroke}{rgb}{0.000000,0.000000,0.000000}%
\pgfsetstrokecolor{currentstroke}%
\pgfsetdash{}{0pt}%
\pgfpathmoveto{\pgfqpoint{1.718490in}{1.281139in}}%
\pgfpathlineto{\pgfqpoint{1.754086in}{1.278851in}}%
\pgfpathlineto{\pgfqpoint{1.717701in}{1.308886in}}%
\pgfpathlineto{\pgfqpoint{1.682034in}{1.314243in}}%
\pgfpathclose%
\pgfusepath{fill}%
\end{pgfscope}%
\begin{pgfscope}%
\pgfpathrectangle{\pgfqpoint{0.150000in}{0.150000in}}{\pgfqpoint{2.700000in}{1.950000in}}%
\pgfusepath{clip}%
\pgfsetbuttcap%
\pgfsetroundjoin%
\definecolor{currentfill}{rgb}{0.773958,0.589844,0.604427}%
\pgfsetfillcolor{currentfill}%
\pgfsetlinewidth{0.000000pt}%
\definecolor{currentstroke}{rgb}{0.000000,0.000000,0.000000}%
\pgfsetstrokecolor{currentstroke}%
\pgfsetdash{}{0pt}%
\pgfpathmoveto{\pgfqpoint{2.228056in}{0.908691in}}%
\pgfpathlineto{\pgfqpoint{2.261971in}{0.910613in}}%
\pgfpathlineto{\pgfqpoint{2.225246in}{0.940896in}}%
\pgfpathlineto{\pgfqpoint{2.191227in}{0.939078in}}%
\pgfpathclose%
\pgfusepath{fill}%
\end{pgfscope}%
\begin{pgfscope}%
\pgfpathrectangle{\pgfqpoint{0.150000in}{0.150000in}}{\pgfqpoint{2.700000in}{1.950000in}}%
\pgfusepath{clip}%
\pgfsetbuttcap%
\pgfsetroundjoin%
\definecolor{currentfill}{rgb}{0.853845,0.871844,0.897044}%
\pgfsetfillcolor{currentfill}%
\pgfsetlinewidth{0.000000pt}%
\definecolor{currentstroke}{rgb}{0.000000,0.000000,0.000000}%
\pgfsetstrokecolor{currentstroke}%
\pgfsetdash{}{0pt}%
\pgfpathmoveto{\pgfqpoint{1.427262in}{1.506440in}}%
\pgfpathlineto{\pgfqpoint{1.463959in}{1.500319in}}%
\pgfpathlineto{\pgfqpoint{1.427736in}{1.533215in}}%
\pgfpathlineto{\pgfqpoint{1.390949in}{1.536443in}}%
\pgfpathclose%
\pgfusepath{fill}%
\end{pgfscope}%
\begin{pgfscope}%
\pgfpathrectangle{\pgfqpoint{0.150000in}{0.150000in}}{\pgfqpoint{2.700000in}{1.950000in}}%
\pgfusepath{clip}%
\pgfsetbuttcap%
\pgfsetroundjoin%
\definecolor{currentfill}{rgb}{0.857537,0.741498,0.750689}%
\pgfsetfillcolor{currentfill}%
\pgfsetlinewidth{0.000000pt}%
\definecolor{currentstroke}{rgb}{0.000000,0.000000,0.000000}%
\pgfsetstrokecolor{currentstroke}%
\pgfsetdash{}{0pt}%
\pgfpathmoveto{\pgfqpoint{2.009861in}{1.061711in}}%
\pgfpathlineto{\pgfqpoint{2.044535in}{1.063112in}}%
\pgfpathlineto{\pgfqpoint{2.007937in}{1.093307in}}%
\pgfpathlineto{\pgfqpoint{1.973160in}{1.092010in}}%
\pgfpathclose%
\pgfusepath{fill}%
\end{pgfscope}%
\begin{pgfscope}%
\pgfpathrectangle{\pgfqpoint{0.150000in}{0.150000in}}{\pgfqpoint{2.700000in}{1.950000in}}%
\pgfusepath{clip}%
\pgfsetbuttcap%
\pgfsetroundjoin%
\definecolor{currentfill}{rgb}{0.872503,0.888205,0.910187}%
\pgfsetfillcolor{currentfill}%
\pgfsetlinewidth{0.000000pt}%
\definecolor{currentstroke}{rgb}{0.000000,0.000000,0.000000}%
\pgfsetstrokecolor{currentstroke}%
\pgfsetdash{}{0pt}%
\pgfpathmoveto{\pgfqpoint{1.463642in}{1.473390in}}%
\pgfpathlineto{\pgfqpoint{1.500199in}{1.470392in}}%
\pgfpathlineto{\pgfqpoint{1.463959in}{1.500319in}}%
\pgfpathlineto{\pgfqpoint{1.427262in}{1.506440in}}%
\pgfpathclose%
\pgfusepath{fill}%
\end{pgfscope}%
\begin{pgfscope}%
\pgfpathrectangle{\pgfqpoint{0.150000in}{0.150000in}}{\pgfqpoint{2.700000in}{1.950000in}}%
\pgfusepath{clip}%
\pgfsetbuttcap%
\pgfsetroundjoin%
\definecolor{currentfill}{rgb}{0.960110,0.927619,0.930193}%
\pgfsetfillcolor{currentfill}%
\pgfsetlinewidth{0.000000pt}%
\definecolor{currentstroke}{rgb}{0.000000,0.000000,0.000000}%
\pgfsetstrokecolor{currentstroke}%
\pgfsetdash{}{0pt}%
\pgfpathmoveto{\pgfqpoint{1.755035in}{1.250955in}}%
\pgfpathlineto{\pgfqpoint{1.790519in}{1.248776in}}%
\pgfpathlineto{\pgfqpoint{1.754086in}{1.278851in}}%
\pgfpathlineto{\pgfqpoint{1.718490in}{1.281139in}}%
\pgfpathclose%
\pgfusepath{fill}%
\end{pgfscope}%
\begin{pgfscope}%
\pgfpathrectangle{\pgfqpoint{0.150000in}{0.150000in}}{\pgfqpoint{2.700000in}{1.950000in}}%
\pgfusepath{clip}%
\pgfsetbuttcap%
\pgfsetroundjoin%
\definecolor{currentfill}{rgb}{0.762561,0.569164,0.584482}%
\pgfsetfillcolor{currentfill}%
\pgfsetlinewidth{0.000000pt}%
\definecolor{currentstroke}{rgb}{0.000000,0.000000,0.000000}%
\pgfsetstrokecolor{currentstroke}%
\pgfsetdash{}{0pt}%
\pgfpathmoveto{\pgfqpoint{2.264732in}{0.875423in}}%
\pgfpathlineto{\pgfqpoint{2.298745in}{0.880291in}}%
\pgfpathlineto{\pgfqpoint{2.261971in}{0.910613in}}%
\pgfpathlineto{\pgfqpoint{2.228056in}{0.908691in}}%
\pgfpathclose%
\pgfusepath{fill}%
\end{pgfscope}%
\begin{pgfscope}%
\pgfpathrectangle{\pgfqpoint{0.150000in}{0.150000in}}{\pgfqpoint{2.700000in}{1.950000in}}%
\pgfusepath{clip}%
\pgfsetbuttcap%
\pgfsetroundjoin%
\definecolor{currentfill}{rgb}{0.846140,0.720818,0.730744}%
\pgfsetfillcolor{currentfill}%
\pgfsetlinewidth{0.000000pt}%
\definecolor{currentstroke}{rgb}{0.000000,0.000000,0.000000}%
\pgfsetstrokecolor{currentstroke}%
\pgfsetdash{}{0pt}%
\pgfpathmoveto{\pgfqpoint{2.046610in}{1.031371in}}%
\pgfpathlineto{\pgfqpoint{2.081031in}{1.030002in}}%
\pgfpathlineto{\pgfqpoint{2.044535in}{1.063112in}}%
\pgfpathlineto{\pgfqpoint{2.009861in}{1.061711in}}%
\pgfpathclose%
\pgfusepath{fill}%
\end{pgfscope}%
\begin{pgfscope}%
\pgfpathrectangle{\pgfqpoint{0.150000in}{0.150000in}}{\pgfqpoint{2.700000in}{1.950000in}}%
\pgfusepath{clip}%
\pgfsetbuttcap%
\pgfsetroundjoin%
\definecolor{currentfill}{rgb}{0.897381,0.910018,0.927711}%
\pgfsetfillcolor{currentfill}%
\pgfsetlinewidth{0.000000pt}%
\definecolor{currentstroke}{rgb}{0.000000,0.000000,0.000000}%
\pgfsetstrokecolor{currentstroke}%
\pgfsetdash{}{0pt}%
\pgfpathmoveto{\pgfqpoint{1.500040in}{1.443315in}}%
\pgfpathlineto{\pgfqpoint{1.536486in}{1.437438in}}%
\pgfpathlineto{\pgfqpoint{1.500199in}{1.470392in}}%
\pgfpathlineto{\pgfqpoint{1.463642in}{1.473390in}}%
\pgfpathclose%
\pgfusepath{fill}%
\end{pgfscope}%
\begin{pgfscope}%
\pgfpathrectangle{\pgfqpoint{0.150000in}{0.150000in}}{\pgfqpoint{2.700000in}{1.950000in}}%
\pgfusepath{clip}%
\pgfsetbuttcap%
\pgfsetroundjoin%
\definecolor{currentfill}{rgb}{0.944914,0.900046,0.903600}%
\pgfsetfillcolor{currentfill}%
\pgfsetlinewidth{0.000000pt}%
\definecolor{currentstroke}{rgb}{0.000000,0.000000,0.000000}%
\pgfsetstrokecolor{currentstroke}%
\pgfsetdash{}{0pt}%
\pgfpathmoveto{\pgfqpoint{1.791556in}{1.217793in}}%
\pgfpathlineto{\pgfqpoint{1.826919in}{1.215736in}}%
\pgfpathlineto{\pgfqpoint{1.790519in}{1.248776in}}%
\pgfpathlineto{\pgfqpoint{1.755035in}{1.250955in}}%
\pgfpathclose%
\pgfusepath{fill}%
\end{pgfscope}%
\begin{pgfscope}%
\pgfpathrectangle{\pgfqpoint{0.150000in}{0.150000in}}{\pgfqpoint{2.700000in}{1.950000in}}%
\pgfusepath{clip}%
\pgfsetbuttcap%
\pgfsetroundjoin%
\definecolor{currentfill}{rgb}{0.747365,0.541590,0.557889}%
\pgfsetfillcolor{currentfill}%
\pgfsetlinewidth{0.000000pt}%
\definecolor{currentstroke}{rgb}{0.000000,0.000000,0.000000}%
\pgfsetstrokecolor{currentstroke}%
\pgfsetdash{}{0pt}%
\pgfpathmoveto{\pgfqpoint{2.301647in}{0.844961in}}%
\pgfpathlineto{\pgfqpoint{2.335567in}{0.849928in}}%
\pgfpathlineto{\pgfqpoint{2.298745in}{0.880291in}}%
\pgfpathlineto{\pgfqpoint{2.264732in}{0.875423in}}%
\pgfpathclose%
\pgfusepath{fill}%
\end{pgfscope}%
\begin{pgfscope}%
\pgfpathrectangle{\pgfqpoint{0.150000in}{0.150000in}}{\pgfqpoint{2.700000in}{1.950000in}}%
\pgfusepath{clip}%
\pgfsetbuttcap%
\pgfsetroundjoin%
\definecolor{currentfill}{rgb}{0.830944,0.693244,0.704151}%
\pgfsetfillcolor{currentfill}%
\pgfsetlinewidth{0.000000pt}%
\definecolor{currentstroke}{rgb}{0.000000,0.000000,0.000000}%
\pgfsetstrokecolor{currentstroke}%
\pgfsetdash{}{0pt}%
\pgfpathmoveto{\pgfqpoint{2.083256in}{0.998114in}}%
\pgfpathlineto{\pgfqpoint{2.117715in}{0.999734in}}%
\pgfpathlineto{\pgfqpoint{2.081031in}{1.030002in}}%
\pgfpathlineto{\pgfqpoint{2.046610in}{1.031371in}}%
\pgfpathclose%
\pgfusepath{fill}%
\end{pgfscope}%
\begin{pgfscope}%
\pgfpathrectangle{\pgfqpoint{0.150000in}{0.150000in}}{\pgfqpoint{2.700000in}{1.950000in}}%
\pgfusepath{clip}%
\pgfsetbuttcap%
\pgfsetroundjoin%
\definecolor{currentfill}{rgb}{0.922258,0.931832,0.945236}%
\pgfsetfillcolor{currentfill}%
\pgfsetlinewidth{0.000000pt}%
\definecolor{currentstroke}{rgb}{0.000000,0.000000,0.000000}%
\pgfsetstrokecolor{currentstroke}%
\pgfsetdash{}{0pt}%
\pgfpathmoveto{\pgfqpoint{1.536486in}{1.410207in}}%
\pgfpathlineto{\pgfqpoint{1.572812in}{1.407438in}}%
\pgfpathlineto{\pgfqpoint{1.536486in}{1.437438in}}%
\pgfpathlineto{\pgfqpoint{1.500040in}{1.443315in}}%
\pgfpathclose%
\pgfusepath{fill}%
\end{pgfscope}%
\begin{pgfscope}%
\pgfpathrectangle{\pgfqpoint{0.150000in}{0.150000in}}{\pgfqpoint{2.700000in}{1.950000in}}%
\pgfusepath{clip}%
\pgfsetbuttcap%
\pgfsetroundjoin%
\definecolor{currentfill}{rgb}{0.929718,0.872472,0.877007}%
\pgfsetfillcolor{currentfill}%
\pgfsetlinewidth{0.000000pt}%
\definecolor{currentstroke}{rgb}{0.000000,0.000000,0.000000}%
\pgfsetstrokecolor{currentstroke}%
\pgfsetdash{}{0pt}%
\pgfpathmoveto{\pgfqpoint{1.828187in}{1.187535in}}%
\pgfpathlineto{\pgfqpoint{1.863438in}{1.185589in}}%
\pgfpathlineto{\pgfqpoint{1.826919in}{1.215736in}}%
\pgfpathlineto{\pgfqpoint{1.791556in}{1.217793in}}%
\pgfpathclose%
\pgfusepath{fill}%
\end{pgfscope}%
\begin{pgfscope}%
\pgfpathrectangle{\pgfqpoint{0.150000in}{0.150000in}}{\pgfqpoint{2.700000in}{1.950000in}}%
\pgfusepath{clip}%
\pgfsetbuttcap%
\pgfsetroundjoin%
\definecolor{currentfill}{rgb}{0.735968,0.520910,0.537944}%
\pgfsetfillcolor{currentfill}%
\pgfsetlinewidth{0.000000pt}%
\definecolor{currentstroke}{rgb}{0.000000,0.000000,0.000000}%
\pgfsetstrokecolor{currentstroke}%
\pgfsetdash{}{0pt}%
\pgfpathmoveto{\pgfqpoint{2.338611in}{0.814460in}}%
\pgfpathlineto{\pgfqpoint{2.372206in}{0.816708in}}%
\pgfpathlineto{\pgfqpoint{2.335567in}{0.849928in}}%
\pgfpathlineto{\pgfqpoint{2.301647in}{0.844961in}}%
\pgfpathclose%
\pgfusepath{fill}%
\end{pgfscope}%
\begin{pgfscope}%
\pgfpathrectangle{\pgfqpoint{0.150000in}{0.150000in}}{\pgfqpoint{2.700000in}{1.950000in}}%
\pgfusepath{clip}%
\pgfsetbuttcap%
\pgfsetroundjoin%
\definecolor{currentfill}{rgb}{0.819547,0.672564,0.684206}%
\pgfsetfillcolor{currentfill}%
\pgfsetlinewidth{0.000000pt}%
\definecolor{currentstroke}{rgb}{0.000000,0.000000,0.000000}%
\pgfsetstrokecolor{currentstroke}%
\pgfsetdash{}{0pt}%
\pgfpathmoveto{\pgfqpoint{2.120091in}{0.967701in}}%
\pgfpathlineto{\pgfqpoint{2.154447in}{0.969426in}}%
\pgfpathlineto{\pgfqpoint{2.117715in}{0.999734in}}%
\pgfpathlineto{\pgfqpoint{2.083256in}{0.998114in}}%
\pgfpathclose%
\pgfusepath{fill}%
\end{pgfscope}%
\begin{pgfscope}%
\pgfpathrectangle{\pgfqpoint{0.150000in}{0.150000in}}{\pgfqpoint{2.700000in}{1.950000in}}%
\pgfusepath{clip}%
\pgfsetbuttcap%
\pgfsetroundjoin%
\definecolor{currentfill}{rgb}{0.940916,0.948192,0.958379}%
\pgfsetfillcolor{currentfill}%
\pgfsetlinewidth{0.000000pt}%
\definecolor{currentstroke}{rgb}{0.000000,0.000000,0.000000}%
\pgfsetstrokecolor{currentstroke}%
\pgfsetdash{}{0pt}%
\pgfpathmoveto{\pgfqpoint{1.572970in}{1.380058in}}%
\pgfpathlineto{\pgfqpoint{1.609165in}{1.374426in}}%
\pgfpathlineto{\pgfqpoint{1.572812in}{1.407438in}}%
\pgfpathlineto{\pgfqpoint{1.536486in}{1.410207in}}%
\pgfpathclose%
\pgfusepath{fill}%
\end{pgfscope}%
\begin{pgfscope}%
\pgfpathrectangle{\pgfqpoint{0.150000in}{0.150000in}}{\pgfqpoint{2.700000in}{1.950000in}}%
\pgfusepath{clip}%
\pgfsetbuttcap%
\pgfsetroundjoin%
\definecolor{currentfill}{rgb}{0.918321,0.851792,0.857062}%
\pgfsetfillcolor{currentfill}%
\pgfsetlinewidth{0.000000pt}%
\definecolor{currentstroke}{rgb}{0.000000,0.000000,0.000000}%
\pgfsetstrokecolor{currentstroke}%
\pgfsetdash{}{0pt}%
\pgfpathmoveto{\pgfqpoint{1.864775in}{1.154315in}}%
\pgfpathlineto{\pgfqpoint{1.900004in}{1.155402in}}%
\pgfpathlineto{\pgfqpoint{1.863438in}{1.185589in}}%
\pgfpathlineto{\pgfqpoint{1.828187in}{1.187535in}}%
\pgfpathclose%
\pgfusepath{fill}%
\end{pgfscope}%
\begin{pgfscope}%
\pgfpathrectangle{\pgfqpoint{0.150000in}{0.150000in}}{\pgfqpoint{2.700000in}{1.950000in}}%
\pgfusepath{clip}%
\pgfsetbuttcap%
\pgfsetroundjoin%
\definecolor{currentfill}{rgb}{0.724571,0.500230,0.517999}%
\pgfsetfillcolor{currentfill}%
\pgfsetlinewidth{0.000000pt}%
\definecolor{currentstroke}{rgb}{0.000000,0.000000,0.000000}%
\pgfsetstrokecolor{currentstroke}%
\pgfsetdash{}{0pt}%
\pgfpathmoveto{\pgfqpoint{2.375624in}{0.783919in}}%
\pgfpathlineto{\pgfqpoint{2.409114in}{0.786272in}}%
\pgfpathlineto{\pgfqpoint{2.372206in}{0.816708in}}%
\pgfpathlineto{\pgfqpoint{2.338611in}{0.814460in}}%
\pgfpathclose%
\pgfusepath{fill}%
\end{pgfscope}%
\begin{pgfscope}%
\pgfpathrectangle{\pgfqpoint{0.150000in}{0.150000in}}{\pgfqpoint{2.700000in}{1.950000in}}%
\pgfusepath{clip}%
\pgfsetbuttcap%
\pgfsetroundjoin%
\definecolor{currentfill}{rgb}{0.965794,0.970006,0.975904}%
\pgfsetfillcolor{currentfill}%
\pgfsetlinewidth{0.000000pt}%
\definecolor{currentstroke}{rgb}{0.000000,0.000000,0.000000}%
\pgfsetstrokecolor{currentstroke}%
\pgfsetdash{}{0pt}%
\pgfpathmoveto{\pgfqpoint{1.609482in}{1.346892in}}%
\pgfpathlineto{\pgfqpoint{1.645575in}{1.344354in}}%
\pgfpathlineto{\pgfqpoint{1.609165in}{1.374426in}}%
\pgfpathlineto{\pgfqpoint{1.572970in}{1.380058in}}%
\pgfpathclose%
\pgfusepath{fill}%
\end{pgfscope}%
\begin{pgfscope}%
\pgfpathrectangle{\pgfqpoint{0.150000in}{0.150000in}}{\pgfqpoint{2.700000in}{1.950000in}}%
\pgfusepath{clip}%
\pgfsetbuttcap%
\pgfsetroundjoin%
\definecolor{currentfill}{rgb}{0.804350,0.644991,0.657613}%
\pgfsetfillcolor{currentfill}%
\pgfsetlinewidth{0.000000pt}%
\definecolor{currentstroke}{rgb}{0.000000,0.000000,0.000000}%
\pgfsetstrokecolor{currentstroke}%
\pgfsetdash{}{0pt}%
\pgfpathmoveto{\pgfqpoint{2.156975in}{0.937248in}}%
\pgfpathlineto{\pgfqpoint{2.191227in}{0.939078in}}%
\pgfpathlineto{\pgfqpoint{2.154447in}{0.969426in}}%
\pgfpathlineto{\pgfqpoint{2.120091in}{0.967701in}}%
\pgfpathclose%
\pgfusepath{fill}%
\end{pgfscope}%
\begin{pgfscope}%
\pgfpathrectangle{\pgfqpoint{0.150000in}{0.150000in}}{\pgfqpoint{2.700000in}{1.950000in}}%
\pgfusepath{clip}%
\pgfsetbuttcap%
\pgfsetroundjoin%
\definecolor{currentfill}{rgb}{0.903125,0.824219,0.830469}%
\pgfsetfillcolor{currentfill}%
\pgfsetlinewidth{0.000000pt}%
\definecolor{currentstroke}{rgb}{0.000000,0.000000,0.000000}%
\pgfsetstrokecolor{currentstroke}%
\pgfsetdash{}{0pt}%
\pgfpathmoveto{\pgfqpoint{1.901492in}{1.123984in}}%
\pgfpathlineto{\pgfqpoint{1.936619in}{1.125175in}}%
\pgfpathlineto{\pgfqpoint{1.900004in}{1.155402in}}%
\pgfpathlineto{\pgfqpoint{1.864775in}{1.154315in}}%
\pgfpathclose%
\pgfusepath{fill}%
\end{pgfscope}%
\begin{pgfscope}%
\pgfpathrectangle{\pgfqpoint{0.150000in}{0.150000in}}{\pgfqpoint{2.700000in}{1.950000in}}%
\pgfusepath{clip}%
\pgfsetbuttcap%
\pgfsetroundjoin%
\definecolor{currentfill}{rgb}{0.984452,0.986366,0.989047}%
\pgfsetfillcolor{currentfill}%
\pgfsetlinewidth{0.000000pt}%
\definecolor{currentstroke}{rgb}{0.000000,0.000000,0.000000}%
\pgfsetstrokecolor{currentstroke}%
\pgfsetdash{}{0pt}%
\pgfpathmoveto{\pgfqpoint{1.646052in}{1.316670in}}%
\pgfpathlineto{\pgfqpoint{1.682034in}{1.314243in}}%
\pgfpathlineto{\pgfqpoint{1.645575in}{1.344354in}}%
\pgfpathlineto{\pgfqpoint{1.609482in}{1.346892in}}%
\pgfpathclose%
\pgfusepath{fill}%
\end{pgfscope}%
\begin{pgfscope}%
\pgfpathrectangle{\pgfqpoint{0.150000in}{0.150000in}}{\pgfqpoint{2.700000in}{1.950000in}}%
\pgfusepath{clip}%
\pgfsetbuttcap%
\pgfsetroundjoin%
\definecolor{currentfill}{rgb}{0.792953,0.624311,0.637669}%
\pgfsetfillcolor{currentfill}%
\pgfsetlinewidth{0.000000pt}%
\definecolor{currentstroke}{rgb}{0.000000,0.000000,0.000000}%
\pgfsetstrokecolor{currentstroke}%
\pgfsetdash{}{0pt}%
\pgfpathmoveto{\pgfqpoint{2.193725in}{0.903899in}}%
\pgfpathlineto{\pgfqpoint{2.228056in}{0.908691in}}%
\pgfpathlineto{\pgfqpoint{2.191227in}{0.939078in}}%
\pgfpathlineto{\pgfqpoint{2.156975in}{0.937248in}}%
\pgfpathclose%
\pgfusepath{fill}%
\end{pgfscope}%
\begin{pgfscope}%
\pgfpathrectangle{\pgfqpoint{0.150000in}{0.150000in}}{\pgfqpoint{2.700000in}{1.950000in}}%
\pgfusepath{clip}%
\pgfsetbuttcap%
\pgfsetroundjoin%
\definecolor{currentfill}{rgb}{0.891728,0.803539,0.810524}%
\pgfsetfillcolor{currentfill}%
\pgfsetlinewidth{0.000000pt}%
\definecolor{currentstroke}{rgb}{0.000000,0.000000,0.000000}%
\pgfsetstrokecolor{currentstroke}%
\pgfsetdash{}{0pt}%
\pgfpathmoveto{\pgfqpoint{1.938257in}{1.093613in}}%
\pgfpathlineto{\pgfqpoint{1.973160in}{1.092010in}}%
\pgfpathlineto{\pgfqpoint{1.936619in}{1.125175in}}%
\pgfpathlineto{\pgfqpoint{1.901492in}{1.123984in}}%
\pgfpathclose%
\pgfusepath{fill}%
\end{pgfscope}%
\begin{pgfscope}%
\pgfpathrectangle{\pgfqpoint{0.150000in}{0.150000in}}{\pgfqpoint{2.700000in}{1.950000in}}%
\pgfusepath{clip}%
\pgfsetbuttcap%
\pgfsetroundjoin%
\definecolor{currentfill}{rgb}{0.791651,0.817310,0.853232}%
\pgfsetfillcolor{currentfill}%
\pgfsetlinewidth{0.000000pt}%
\definecolor{currentstroke}{rgb}{0.000000,0.000000,0.000000}%
\pgfsetstrokecolor{currentstroke}%
\pgfsetdash{}{0pt}%
\pgfpathmoveto{\pgfqpoint{1.353839in}{1.542728in}}%
\pgfpathlineto{\pgfqpoint{1.390949in}{1.536443in}}%
\pgfpathlineto{\pgfqpoint{1.354634in}{1.569434in}}%
\pgfpathlineto{\pgfqpoint{1.317394in}{1.575848in}}%
\pgfpathclose%
\pgfusepath{fill}%
\end{pgfscope}%
\begin{pgfscope}%
\pgfpathrectangle{\pgfqpoint{0.150000in}{0.150000in}}{\pgfqpoint{2.700000in}{1.950000in}}%
\pgfusepath{clip}%
\pgfsetbuttcap%
\pgfsetroundjoin%
\definecolor{currentfill}{rgb}{0.994301,0.989660,0.990028}%
\pgfsetfillcolor{currentfill}%
\pgfsetlinewidth{0.000000pt}%
\definecolor{currentstroke}{rgb}{0.000000,0.000000,0.000000}%
\pgfsetstrokecolor{currentstroke}%
\pgfsetdash{}{0pt}%
\pgfpathmoveto{\pgfqpoint{1.682630in}{1.283445in}}%
\pgfpathlineto{\pgfqpoint{1.718490in}{1.281139in}}%
\pgfpathlineto{\pgfqpoint{1.682034in}{1.314243in}}%
\pgfpathlineto{\pgfqpoint{1.646052in}{1.316670in}}%
\pgfpathclose%
\pgfusepath{fill}%
\end{pgfscope}%
\begin{pgfscope}%
\pgfpathrectangle{\pgfqpoint{0.150000in}{0.150000in}}{\pgfqpoint{2.700000in}{1.950000in}}%
\pgfusepath{clip}%
\pgfsetbuttcap%
\pgfsetroundjoin%
\definecolor{currentfill}{rgb}{0.777757,0.596737,0.611075}%
\pgfsetfillcolor{currentfill}%
\pgfsetlinewidth{0.000000pt}%
\definecolor{currentstroke}{rgb}{0.000000,0.000000,0.000000}%
\pgfsetstrokecolor{currentstroke}%
\pgfsetdash{}{0pt}%
\pgfpathmoveto{\pgfqpoint{2.230696in}{0.873372in}}%
\pgfpathlineto{\pgfqpoint{2.264732in}{0.875423in}}%
\pgfpathlineto{\pgfqpoint{2.228056in}{0.908691in}}%
\pgfpathlineto{\pgfqpoint{2.193725in}{0.903899in}}%
\pgfpathclose%
\pgfusepath{fill}%
\end{pgfscope}%
\begin{pgfscope}%
\pgfpathrectangle{\pgfqpoint{0.150000in}{0.150000in}}{\pgfqpoint{2.700000in}{1.950000in}}%
\pgfusepath{clip}%
\pgfsetbuttcap%
\pgfsetroundjoin%
\definecolor{currentfill}{rgb}{0.876532,0.775965,0.783931}%
\pgfsetfillcolor{currentfill}%
\pgfsetlinewidth{0.000000pt}%
\definecolor{currentstroke}{rgb}{0.000000,0.000000,0.000000}%
\pgfsetstrokecolor{currentstroke}%
\pgfsetdash{}{0pt}%
\pgfpathmoveto{\pgfqpoint{1.974949in}{1.060300in}}%
\pgfpathlineto{\pgfqpoint{2.009861in}{1.061711in}}%
\pgfpathlineto{\pgfqpoint{1.973160in}{1.092010in}}%
\pgfpathlineto{\pgfqpoint{1.938257in}{1.093613in}}%
\pgfpathclose%
\pgfusepath{fill}%
\end{pgfscope}%
\begin{pgfscope}%
\pgfpathrectangle{\pgfqpoint{0.150000in}{0.150000in}}{\pgfqpoint{2.700000in}{1.950000in}}%
\pgfusepath{clip}%
\pgfsetbuttcap%
\pgfsetroundjoin%
\definecolor{currentfill}{rgb}{0.816529,0.839124,0.870757}%
\pgfsetfillcolor{currentfill}%
\pgfsetlinewidth{0.000000pt}%
\definecolor{currentstroke}{rgb}{0.000000,0.000000,0.000000}%
\pgfsetstrokecolor{currentstroke}%
\pgfsetdash{}{0pt}%
\pgfpathmoveto{\pgfqpoint{1.390272in}{1.512609in}}%
\pgfpathlineto{\pgfqpoint{1.427262in}{1.506440in}}%
\pgfpathlineto{\pgfqpoint{1.390949in}{1.536443in}}%
\pgfpathlineto{\pgfqpoint{1.353839in}{1.542728in}}%
\pgfpathclose%
\pgfusepath{fill}%
\end{pgfscope}%
\begin{pgfscope}%
\pgfpathrectangle{\pgfqpoint{0.150000in}{0.150000in}}{\pgfqpoint{2.700000in}{1.950000in}}%
\pgfusepath{clip}%
\pgfsetbuttcap%
\pgfsetroundjoin%
\definecolor{currentfill}{rgb}{0.979105,0.962086,0.963434}%
\pgfsetfillcolor{currentfill}%
\pgfsetlinewidth{0.000000pt}%
\definecolor{currentstroke}{rgb}{0.000000,0.000000,0.000000}%
\pgfsetstrokecolor{currentstroke}%
\pgfsetdash{}{0pt}%
\pgfpathmoveto{\pgfqpoint{1.719287in}{1.253150in}}%
\pgfpathlineto{\pgfqpoint{1.755035in}{1.250955in}}%
\pgfpathlineto{\pgfqpoint{1.718490in}{1.281139in}}%
\pgfpathlineto{\pgfqpoint{1.682630in}{1.283445in}}%
\pgfpathclose%
\pgfusepath{fill}%
\end{pgfscope}%
\begin{pgfscope}%
\pgfpathrectangle{\pgfqpoint{0.150000in}{0.150000in}}{\pgfqpoint{2.700000in}{1.950000in}}%
\pgfusepath{clip}%
\pgfsetbuttcap%
\pgfsetroundjoin%
\definecolor{currentfill}{rgb}{0.766360,0.576057,0.591131}%
\pgfsetfillcolor{currentfill}%
\pgfsetlinewidth{0.000000pt}%
\definecolor{currentstroke}{rgb}{0.000000,0.000000,0.000000}%
\pgfsetstrokecolor{currentstroke}%
\pgfsetdash{}{0pt}%
\pgfpathmoveto{\pgfqpoint{2.267716in}{0.842805in}}%
\pgfpathlineto{\pgfqpoint{2.301647in}{0.844961in}}%
\pgfpathlineto{\pgfqpoint{2.264732in}{0.875423in}}%
\pgfpathlineto{\pgfqpoint{2.230696in}{0.873372in}}%
\pgfpathclose%
\pgfusepath{fill}%
\end{pgfscope}%
\begin{pgfscope}%
\pgfpathrectangle{\pgfqpoint{0.150000in}{0.150000in}}{\pgfqpoint{2.700000in}{1.950000in}}%
\pgfusepath{clip}%
\pgfsetbuttcap%
\pgfsetroundjoin%
\definecolor{currentfill}{rgb}{0.835187,0.855484,0.883900}%
\pgfsetfillcolor{currentfill}%
\pgfsetlinewidth{0.000000pt}%
\definecolor{currentstroke}{rgb}{0.000000,0.000000,0.000000}%
\pgfsetstrokecolor{currentstroke}%
\pgfsetdash{}{0pt}%
\pgfpathmoveto{\pgfqpoint{1.426784in}{1.479430in}}%
\pgfpathlineto{\pgfqpoint{1.463642in}{1.473390in}}%
\pgfpathlineto{\pgfqpoint{1.427262in}{1.506440in}}%
\pgfpathlineto{\pgfqpoint{1.390272in}{1.512609in}}%
\pgfpathclose%
\pgfusepath{fill}%
\end{pgfscope}%
\begin{pgfscope}%
\pgfpathrectangle{\pgfqpoint{0.150000in}{0.150000in}}{\pgfqpoint{2.700000in}{1.950000in}}%
\pgfusepath{clip}%
\pgfsetbuttcap%
\pgfsetroundjoin%
\definecolor{currentfill}{rgb}{0.865135,0.755285,0.763986}%
\pgfsetfillcolor{currentfill}%
\pgfsetlinewidth{0.000000pt}%
\definecolor{currentstroke}{rgb}{0.000000,0.000000,0.000000}%
\pgfsetstrokecolor{currentstroke}%
\pgfsetdash{}{0pt}%
\pgfpathmoveto{\pgfqpoint{2.011801in}{1.029855in}}%
\pgfpathlineto{\pgfqpoint{2.046610in}{1.031371in}}%
\pgfpathlineto{\pgfqpoint{2.009861in}{1.061711in}}%
\pgfpathlineto{\pgfqpoint{1.974949in}{1.060300in}}%
\pgfpathclose%
\pgfusepath{fill}%
\end{pgfscope}%
\begin{pgfscope}%
\pgfpathrectangle{\pgfqpoint{0.150000in}{0.150000in}}{\pgfqpoint{2.700000in}{1.950000in}}%
\pgfusepath{clip}%
\pgfsetbuttcap%
\pgfsetroundjoin%
\definecolor{currentfill}{rgb}{0.967708,0.941406,0.943490}%
\pgfsetfillcolor{currentfill}%
\pgfsetlinewidth{0.000000pt}%
\definecolor{currentstroke}{rgb}{0.000000,0.000000,0.000000}%
\pgfsetstrokecolor{currentstroke}%
\pgfsetdash{}{0pt}%
\pgfpathmoveto{\pgfqpoint{1.755931in}{1.219865in}}%
\pgfpathlineto{\pgfqpoint{1.791556in}{1.217793in}}%
\pgfpathlineto{\pgfqpoint{1.755035in}{1.250955in}}%
\pgfpathlineto{\pgfqpoint{1.719287in}{1.253150in}}%
\pgfpathclose%
\pgfusepath{fill}%
\end{pgfscope}%
\begin{pgfscope}%
\pgfpathrectangle{\pgfqpoint{0.150000in}{0.150000in}}{\pgfqpoint{2.700000in}{1.950000in}}%
\pgfusepath{clip}%
\pgfsetbuttcap%
\pgfsetroundjoin%
\definecolor{currentfill}{rgb}{0.860064,0.877298,0.901425}%
\pgfsetfillcolor{currentfill}%
\pgfsetlinewidth{0.000000pt}%
\definecolor{currentstroke}{rgb}{0.000000,0.000000,0.000000}%
\pgfsetstrokecolor{currentstroke}%
\pgfsetdash{}{0pt}%
\pgfpathmoveto{\pgfqpoint{1.463323in}{1.446226in}}%
\pgfpathlineto{\pgfqpoint{1.500040in}{1.443315in}}%
\pgfpathlineto{\pgfqpoint{1.463642in}{1.473390in}}%
\pgfpathlineto{\pgfqpoint{1.426784in}{1.479430in}}%
\pgfpathclose%
\pgfusepath{fill}%
\end{pgfscope}%
\begin{pgfscope}%
\pgfpathrectangle{\pgfqpoint{0.150000in}{0.150000in}}{\pgfqpoint{2.700000in}{1.950000in}}%
\pgfusepath{clip}%
\pgfsetbuttcap%
\pgfsetroundjoin%
\definecolor{currentfill}{rgb}{0.849939,0.727711,0.737393}%
\pgfsetfillcolor{currentfill}%
\pgfsetlinewidth{0.000000pt}%
\definecolor{currentstroke}{rgb}{0.000000,0.000000,0.000000}%
\pgfsetstrokecolor{currentstroke}%
\pgfsetdash{}{0pt}%
\pgfpathmoveto{\pgfqpoint{2.048560in}{0.996483in}}%
\pgfpathlineto{\pgfqpoint{2.083256in}{0.998114in}}%
\pgfpathlineto{\pgfqpoint{2.046610in}{1.031371in}}%
\pgfpathlineto{\pgfqpoint{2.011801in}{1.029855in}}%
\pgfpathclose%
\pgfusepath{fill}%
\end{pgfscope}%
\begin{pgfscope}%
\pgfpathrectangle{\pgfqpoint{0.150000in}{0.150000in}}{\pgfqpoint{2.700000in}{1.950000in}}%
\pgfusepath{clip}%
\pgfsetbuttcap%
\pgfsetroundjoin%
\definecolor{currentfill}{rgb}{0.751164,0.548483,0.564537}%
\pgfsetfillcolor{currentfill}%
\pgfsetlinewidth{0.000000pt}%
\definecolor{currentstroke}{rgb}{0.000000,0.000000,0.000000}%
\pgfsetstrokecolor{currentstroke}%
\pgfsetdash{}{0pt}%
\pgfpathmoveto{\pgfqpoint{2.304572in}{0.809362in}}%
\pgfpathlineto{\pgfqpoint{2.338611in}{0.814460in}}%
\pgfpathlineto{\pgfqpoint{2.301647in}{0.844961in}}%
\pgfpathlineto{\pgfqpoint{2.267716in}{0.842805in}}%
\pgfpathclose%
\pgfusepath{fill}%
\end{pgfscope}%
\begin{pgfscope}%
\pgfpathrectangle{\pgfqpoint{0.150000in}{0.150000in}}{\pgfqpoint{2.700000in}{1.950000in}}%
\pgfusepath{clip}%
\pgfsetbuttcap%
\pgfsetroundjoin%
\definecolor{currentfill}{rgb}{0.952512,0.913833,0.916896}%
\pgfsetfillcolor{currentfill}%
\pgfsetlinewidth{0.000000pt}%
\definecolor{currentstroke}{rgb}{0.000000,0.000000,0.000000}%
\pgfsetstrokecolor{currentstroke}%
\pgfsetdash{}{0pt}%
\pgfpathmoveto{\pgfqpoint{1.792674in}{1.189497in}}%
\pgfpathlineto{\pgfqpoint{1.828187in}{1.187535in}}%
\pgfpathlineto{\pgfqpoint{1.791556in}{1.217793in}}%
\pgfpathlineto{\pgfqpoint{1.755931in}{1.219865in}}%
\pgfpathclose%
\pgfusepath{fill}%
\end{pgfscope}%
\begin{pgfscope}%
\pgfpathrectangle{\pgfqpoint{0.150000in}{0.150000in}}{\pgfqpoint{2.700000in}{1.950000in}}%
\pgfusepath{clip}%
\pgfsetbuttcap%
\pgfsetroundjoin%
\definecolor{currentfill}{rgb}{0.884942,0.899112,0.918949}%
\pgfsetfillcolor{currentfill}%
\pgfsetlinewidth{0.000000pt}%
\definecolor{currentstroke}{rgb}{0.000000,0.000000,0.000000}%
\pgfsetstrokecolor{currentstroke}%
\pgfsetdash{}{0pt}%
\pgfpathmoveto{\pgfqpoint{1.499881in}{1.416000in}}%
\pgfpathlineto{\pgfqpoint{1.536486in}{1.410207in}}%
\pgfpathlineto{\pgfqpoint{1.500040in}{1.443315in}}%
\pgfpathlineto{\pgfqpoint{1.463323in}{1.446226in}}%
\pgfpathclose%
\pgfusepath{fill}%
\end{pgfscope}%
\begin{pgfscope}%
\pgfpathrectangle{\pgfqpoint{0.150000in}{0.150000in}}{\pgfqpoint{2.700000in}{1.950000in}}%
\pgfusepath{clip}%
\pgfsetbuttcap%
\pgfsetroundjoin%
\definecolor{currentfill}{rgb}{0.838542,0.707031,0.717448}%
\pgfsetfillcolor{currentfill}%
\pgfsetlinewidth{0.000000pt}%
\definecolor{currentstroke}{rgb}{0.000000,0.000000,0.000000}%
\pgfsetstrokecolor{currentstroke}%
\pgfsetdash{}{0pt}%
\pgfpathmoveto{\pgfqpoint{2.085499in}{0.965965in}}%
\pgfpathlineto{\pgfqpoint{2.120091in}{0.967701in}}%
\pgfpathlineto{\pgfqpoint{2.083256in}{0.998114in}}%
\pgfpathlineto{\pgfqpoint{2.048560in}{0.996483in}}%
\pgfpathclose%
\pgfusepath{fill}%
\end{pgfscope}%
\begin{pgfscope}%
\pgfpathrectangle{\pgfqpoint{0.150000in}{0.150000in}}{\pgfqpoint{2.700000in}{1.950000in}}%
\pgfusepath{clip}%
\pgfsetbuttcap%
\pgfsetroundjoin%
\definecolor{currentfill}{rgb}{0.739767,0.527803,0.544593}%
\pgfsetfillcolor{currentfill}%
\pgfsetlinewidth{0.000000pt}%
\definecolor{currentstroke}{rgb}{0.000000,0.000000,0.000000}%
\pgfsetstrokecolor{currentstroke}%
\pgfsetdash{}{0pt}%
\pgfpathmoveto{\pgfqpoint{2.341679in}{0.778721in}}%
\pgfpathlineto{\pgfqpoint{2.375624in}{0.783919in}}%
\pgfpathlineto{\pgfqpoint{2.338611in}{0.814460in}}%
\pgfpathlineto{\pgfqpoint{2.304572in}{0.809362in}}%
\pgfpathclose%
\pgfusepath{fill}%
\end{pgfscope}%
\begin{pgfscope}%
\pgfpathrectangle{\pgfqpoint{0.150000in}{0.150000in}}{\pgfqpoint{2.700000in}{1.950000in}}%
\pgfusepath{clip}%
\pgfsetbuttcap%
\pgfsetroundjoin%
\definecolor{currentfill}{rgb}{0.937316,0.886259,0.890303}%
\pgfsetfillcolor{currentfill}%
\pgfsetlinewidth{0.000000pt}%
\definecolor{currentstroke}{rgb}{0.000000,0.000000,0.000000}%
\pgfsetstrokecolor{currentstroke}%
\pgfsetdash{}{0pt}%
\pgfpathmoveto{\pgfqpoint{1.829384in}{1.156153in}}%
\pgfpathlineto{\pgfqpoint{1.864775in}{1.154315in}}%
\pgfpathlineto{\pgfqpoint{1.828187in}{1.187535in}}%
\pgfpathlineto{\pgfqpoint{1.792674in}{1.189497in}}%
\pgfpathclose%
\pgfusepath{fill}%
\end{pgfscope}%
\begin{pgfscope}%
\pgfpathrectangle{\pgfqpoint{0.150000in}{0.150000in}}{\pgfqpoint{2.700000in}{1.950000in}}%
\pgfusepath{clip}%
\pgfsetbuttcap%
\pgfsetroundjoin%
\definecolor{currentfill}{rgb}{0.903600,0.915472,0.932093}%
\pgfsetfillcolor{currentfill}%
\pgfsetlinewidth{0.000000pt}%
\definecolor{currentstroke}{rgb}{0.000000,0.000000,0.000000}%
\pgfsetstrokecolor{currentstroke}%
\pgfsetdash{}{0pt}%
\pgfpathmoveto{\pgfqpoint{1.536486in}{1.382737in}}%
\pgfpathlineto{\pgfqpoint{1.572970in}{1.380058in}}%
\pgfpathlineto{\pgfqpoint{1.536486in}{1.410207in}}%
\pgfpathlineto{\pgfqpoint{1.499881in}{1.416000in}}%
\pgfpathclose%
\pgfusepath{fill}%
\end{pgfscope}%
\begin{pgfscope}%
\pgfpathrectangle{\pgfqpoint{0.150000in}{0.150000in}}{\pgfqpoint{2.700000in}{1.950000in}}%
\pgfusepath{clip}%
\pgfsetbuttcap%
\pgfsetroundjoin%
\definecolor{currentfill}{rgb}{0.823346,0.679458,0.690855}%
\pgfsetfillcolor{currentfill}%
\pgfsetlinewidth{0.000000pt}%
\definecolor{currentstroke}{rgb}{0.000000,0.000000,0.000000}%
\pgfsetstrokecolor{currentstroke}%
\pgfsetdash{}{0pt}%
\pgfpathmoveto{\pgfqpoint{2.122324in}{0.932533in}}%
\pgfpathlineto{\pgfqpoint{2.156975in}{0.937248in}}%
\pgfpathlineto{\pgfqpoint{2.120091in}{0.967701in}}%
\pgfpathlineto{\pgfqpoint{2.085499in}{0.965965in}}%
\pgfpathclose%
\pgfusepath{fill}%
\end{pgfscope}%
\begin{pgfscope}%
\pgfpathrectangle{\pgfqpoint{0.150000in}{0.150000in}}{\pgfqpoint{2.700000in}{1.950000in}}%
\pgfusepath{clip}%
\pgfsetbuttcap%
\pgfsetroundjoin%
\definecolor{currentfill}{rgb}{0.925919,0.865579,0.870358}%
\pgfsetfillcolor{currentfill}%
\pgfsetlinewidth{0.000000pt}%
\definecolor{currentstroke}{rgb}{0.000000,0.000000,0.000000}%
\pgfsetstrokecolor{currentstroke}%
\pgfsetdash{}{0pt}%
\pgfpathmoveto{\pgfqpoint{1.866214in}{1.125711in}}%
\pgfpathlineto{\pgfqpoint{1.901492in}{1.123984in}}%
\pgfpathlineto{\pgfqpoint{1.864775in}{1.154315in}}%
\pgfpathlineto{\pgfqpoint{1.829384in}{1.156153in}}%
\pgfpathclose%
\pgfusepath{fill}%
\end{pgfscope}%
\begin{pgfscope}%
\pgfpathrectangle{\pgfqpoint{0.150000in}{0.150000in}}{\pgfqpoint{2.700000in}{1.950000in}}%
\pgfusepath{clip}%
\pgfsetbuttcap%
\pgfsetroundjoin%
\definecolor{currentfill}{rgb}{0.928477,0.937286,0.949617}%
\pgfsetfillcolor{currentfill}%
\pgfsetlinewidth{0.000000pt}%
\definecolor{currentstroke}{rgb}{0.000000,0.000000,0.000000}%
\pgfsetstrokecolor{currentstroke}%
\pgfsetdash{}{0pt}%
\pgfpathmoveto{\pgfqpoint{1.573130in}{1.352438in}}%
\pgfpathlineto{\pgfqpoint{1.609482in}{1.346892in}}%
\pgfpathlineto{\pgfqpoint{1.572970in}{1.380058in}}%
\pgfpathlineto{\pgfqpoint{1.536486in}{1.382737in}}%
\pgfpathclose%
\pgfusepath{fill}%
\end{pgfscope}%
\begin{pgfscope}%
\pgfpathrectangle{\pgfqpoint{0.150000in}{0.150000in}}{\pgfqpoint{2.700000in}{1.950000in}}%
\pgfusepath{clip}%
\pgfsetbuttcap%
\pgfsetroundjoin%
\definecolor{currentfill}{rgb}{0.808150,0.651884,0.664262}%
\pgfsetfillcolor{currentfill}%
\pgfsetlinewidth{0.000000pt}%
\definecolor{currentstroke}{rgb}{0.000000,0.000000,0.000000}%
\pgfsetstrokecolor{currentstroke}%
\pgfsetdash{}{0pt}%
\pgfpathmoveto{\pgfqpoint{2.159351in}{0.901941in}}%
\pgfpathlineto{\pgfqpoint{2.193725in}{0.903899in}}%
\pgfpathlineto{\pgfqpoint{2.156975in}{0.937248in}}%
\pgfpathlineto{\pgfqpoint{2.122324in}{0.932533in}}%
\pgfpathclose%
\pgfusepath{fill}%
\end{pgfscope}%
\begin{pgfscope}%
\pgfpathrectangle{\pgfqpoint{0.150000in}{0.150000in}}{\pgfqpoint{2.700000in}{1.950000in}}%
\pgfusepath{clip}%
\pgfsetbuttcap%
\pgfsetroundjoin%
\definecolor{currentfill}{rgb}{0.910723,0.838006,0.843765}%
\pgfsetfillcolor{currentfill}%
\pgfsetlinewidth{0.000000pt}%
\definecolor{currentstroke}{rgb}{0.000000,0.000000,0.000000}%
\pgfsetstrokecolor{currentstroke}%
\pgfsetdash{}{0pt}%
\pgfpathmoveto{\pgfqpoint{1.902991in}{1.092308in}}%
\pgfpathlineto{\pgfqpoint{1.938257in}{1.093613in}}%
\pgfpathlineto{\pgfqpoint{1.901492in}{1.123984in}}%
\pgfpathlineto{\pgfqpoint{1.866214in}{1.125711in}}%
\pgfpathclose%
\pgfusepath{fill}%
\end{pgfscope}%
\begin{pgfscope}%
\pgfpathrectangle{\pgfqpoint{0.150000in}{0.150000in}}{\pgfqpoint{2.700000in}{1.950000in}}%
\pgfusepath{clip}%
\pgfsetbuttcap%
\pgfsetroundjoin%
\definecolor{currentfill}{rgb}{0.953355,0.959099,0.967142}%
\pgfsetfillcolor{currentfill}%
\pgfsetlinewidth{0.000000pt}%
\definecolor{currentstroke}{rgb}{0.000000,0.000000,0.000000}%
\pgfsetstrokecolor{currentstroke}%
\pgfsetdash{}{0pt}%
\pgfpathmoveto{\pgfqpoint{1.609802in}{1.319115in}}%
\pgfpathlineto{\pgfqpoint{1.646052in}{1.316670in}}%
\pgfpathlineto{\pgfqpoint{1.609482in}{1.346892in}}%
\pgfpathlineto{\pgfqpoint{1.573130in}{1.352438in}}%
\pgfpathclose%
\pgfusepath{fill}%
\end{pgfscope}%
\begin{pgfscope}%
\pgfpathrectangle{\pgfqpoint{0.150000in}{0.150000in}}{\pgfqpoint{2.700000in}{1.950000in}}%
\pgfusepath{clip}%
\pgfsetbuttcap%
\pgfsetroundjoin%
\definecolor{currentfill}{rgb}{0.796752,0.631204,0.644317}%
\pgfsetfillcolor{currentfill}%
\pgfsetlinewidth{0.000000pt}%
\definecolor{currentstroke}{rgb}{0.000000,0.000000,0.000000}%
\pgfsetstrokecolor{currentstroke}%
\pgfsetdash{}{0pt}%
\pgfpathmoveto{\pgfqpoint{2.196427in}{0.871307in}}%
\pgfpathlineto{\pgfqpoint{2.230696in}{0.873372in}}%
\pgfpathlineto{\pgfqpoint{2.193725in}{0.903899in}}%
\pgfpathlineto{\pgfqpoint{2.159351in}{0.901941in}}%
\pgfpathclose%
\pgfusepath{fill}%
\end{pgfscope}%
\begin{pgfscope}%
\pgfpathrectangle{\pgfqpoint{0.150000in}{0.150000in}}{\pgfqpoint{2.700000in}{1.950000in}}%
\pgfusepath{clip}%
\pgfsetbuttcap%
\pgfsetroundjoin%
\definecolor{currentfill}{rgb}{0.895527,0.810432,0.817172}%
\pgfsetfillcolor{currentfill}%
\pgfsetlinewidth{0.000000pt}%
\definecolor{currentstroke}{rgb}{0.000000,0.000000,0.000000}%
\pgfsetstrokecolor{currentstroke}%
\pgfsetdash{}{0pt}%
\pgfpathmoveto{\pgfqpoint{1.939796in}{1.058880in}}%
\pgfpathlineto{\pgfqpoint{1.974949in}{1.060300in}}%
\pgfpathlineto{\pgfqpoint{1.938257in}{1.093613in}}%
\pgfpathlineto{\pgfqpoint{1.902991in}{1.092308in}}%
\pgfpathclose%
\pgfusepath{fill}%
\end{pgfscope}%
\begin{pgfscope}%
\pgfpathrectangle{\pgfqpoint{0.150000in}{0.150000in}}{\pgfqpoint{2.700000in}{1.950000in}}%
\pgfusepath{clip}%
\pgfsetbuttcap%
\pgfsetroundjoin%
\definecolor{currentfill}{rgb}{0.748116,0.779136,0.822564}%
\pgfsetfillcolor{currentfill}%
\pgfsetlinewidth{0.000000pt}%
\definecolor{currentstroke}{rgb}{0.000000,0.000000,0.000000}%
\pgfsetstrokecolor{currentstroke}%
\pgfsetdash{}{0pt}%
\pgfpathmoveto{\pgfqpoint{1.316432in}{1.549063in}}%
\pgfpathlineto{\pgfqpoint{1.353839in}{1.542728in}}%
\pgfpathlineto{\pgfqpoint{1.317394in}{1.575848in}}%
\pgfpathlineto{\pgfqpoint{1.279855in}{1.582314in}}%
\pgfpathclose%
\pgfusepath{fill}%
\end{pgfscope}%
\begin{pgfscope}%
\pgfpathrectangle{\pgfqpoint{0.150000in}{0.150000in}}{\pgfqpoint{2.700000in}{1.950000in}}%
\pgfusepath{clip}%
\pgfsetbuttcap%
\pgfsetroundjoin%
\definecolor{currentfill}{rgb}{0.972013,0.975460,0.980285}%
\pgfsetfillcolor{currentfill}%
\pgfsetlinewidth{0.000000pt}%
\definecolor{currentstroke}{rgb}{0.000000,0.000000,0.000000}%
\pgfsetstrokecolor{currentstroke}%
\pgfsetdash{}{0pt}%
\pgfpathmoveto{\pgfqpoint{1.646503in}{1.285767in}}%
\pgfpathlineto{\pgfqpoint{1.682630in}{1.283445in}}%
\pgfpathlineto{\pgfqpoint{1.646052in}{1.316670in}}%
\pgfpathlineto{\pgfqpoint{1.609802in}{1.319115in}}%
\pgfpathclose%
\pgfusepath{fill}%
\end{pgfscope}%
\begin{pgfscope}%
\pgfpathrectangle{\pgfqpoint{0.150000in}{0.150000in}}{\pgfqpoint{2.700000in}{1.950000in}}%
\pgfusepath{clip}%
\pgfsetbuttcap%
\pgfsetroundjoin%
\definecolor{currentfill}{rgb}{0.785355,0.610524,0.624372}%
\pgfsetfillcolor{currentfill}%
\pgfsetlinewidth{0.000000pt}%
\definecolor{currentstroke}{rgb}{0.000000,0.000000,0.000000}%
\pgfsetstrokecolor{currentstroke}%
\pgfsetdash{}{0pt}%
\pgfpathmoveto{\pgfqpoint{2.233357in}{0.837782in}}%
\pgfpathlineto{\pgfqpoint{2.267716in}{0.842805in}}%
\pgfpathlineto{\pgfqpoint{2.230696in}{0.873372in}}%
\pgfpathlineto{\pgfqpoint{2.196427in}{0.871307in}}%
\pgfpathclose%
\pgfusepath{fill}%
\end{pgfscope}%
\begin{pgfscope}%
\pgfpathrectangle{\pgfqpoint{0.150000in}{0.150000in}}{\pgfqpoint{2.700000in}{1.950000in}}%
\pgfusepath{clip}%
\pgfsetbuttcap%
\pgfsetroundjoin%
\definecolor{currentfill}{rgb}{0.884130,0.789752,0.797227}%
\pgfsetfillcolor{currentfill}%
\pgfsetlinewidth{0.000000pt}%
\definecolor{currentstroke}{rgb}{0.000000,0.000000,0.000000}%
\pgfsetstrokecolor{currentstroke}%
\pgfsetdash{}{0pt}%
\pgfpathmoveto{\pgfqpoint{1.976752in}{1.028329in}}%
\pgfpathlineto{\pgfqpoint{2.011801in}{1.029855in}}%
\pgfpathlineto{\pgfqpoint{1.974949in}{1.060300in}}%
\pgfpathlineto{\pgfqpoint{1.939796in}{1.058880in}}%
\pgfpathclose%
\pgfusepath{fill}%
\end{pgfscope}%
\begin{pgfscope}%
\pgfpathrectangle{\pgfqpoint{0.150000in}{0.150000in}}{\pgfqpoint{2.700000in}{1.950000in}}%
\pgfusepath{clip}%
\pgfsetbuttcap%
\pgfsetroundjoin%
\definecolor{currentfill}{rgb}{0.772993,0.800950,0.840089}%
\pgfsetfillcolor{currentfill}%
\pgfsetlinewidth{0.000000pt}%
\definecolor{currentstroke}{rgb}{0.000000,0.000000,0.000000}%
\pgfsetstrokecolor{currentstroke}%
\pgfsetdash{}{0pt}%
\pgfpathmoveto{\pgfqpoint{1.352986in}{1.518828in}}%
\pgfpathlineto{\pgfqpoint{1.390272in}{1.512609in}}%
\pgfpathlineto{\pgfqpoint{1.353839in}{1.542728in}}%
\pgfpathlineto{\pgfqpoint{1.316432in}{1.549063in}}%
\pgfpathclose%
\pgfusepath{fill}%
\end{pgfscope}%
\begin{pgfscope}%
\pgfpathrectangle{\pgfqpoint{0.150000in}{0.150000in}}{\pgfqpoint{2.700000in}{1.950000in}}%
\pgfusepath{clip}%
\pgfsetbuttcap%
\pgfsetroundjoin%
\definecolor{currentfill}{rgb}{0.996890,0.997273,0.997809}%
\pgfsetfillcolor{currentfill}%
\pgfsetlinewidth{0.000000pt}%
\definecolor{currentstroke}{rgb}{0.000000,0.000000,0.000000}%
\pgfsetstrokecolor{currentstroke}%
\pgfsetdash{}{0pt}%
\pgfpathmoveto{\pgfqpoint{1.683272in}{1.255361in}}%
\pgfpathlineto{\pgfqpoint{1.719287in}{1.253150in}}%
\pgfpathlineto{\pgfqpoint{1.682630in}{1.283445in}}%
\pgfpathlineto{\pgfqpoint{1.646503in}{1.285767in}}%
\pgfpathclose%
\pgfusepath{fill}%
\end{pgfscope}%
\begin{pgfscope}%
\pgfpathrectangle{\pgfqpoint{0.150000in}{0.150000in}}{\pgfqpoint{2.700000in}{1.950000in}}%
\pgfusepath{clip}%
\pgfsetbuttcap%
\pgfsetroundjoin%
\definecolor{currentfill}{rgb}{0.770159,0.582950,0.597779}%
\pgfsetfillcolor{currentfill}%
\pgfsetlinewidth{0.000000pt}%
\definecolor{currentstroke}{rgb}{0.000000,0.000000,0.000000}%
\pgfsetstrokecolor{currentstroke}%
\pgfsetdash{}{0pt}%
\pgfpathmoveto{\pgfqpoint{2.270521in}{0.807074in}}%
\pgfpathlineto{\pgfqpoint{2.304572in}{0.809362in}}%
\pgfpathlineto{\pgfqpoint{2.267716in}{0.842805in}}%
\pgfpathlineto{\pgfqpoint{2.233357in}{0.837782in}}%
\pgfpathclose%
\pgfusepath{fill}%
\end{pgfscope}%
\begin{pgfscope}%
\pgfpathrectangle{\pgfqpoint{0.150000in}{0.150000in}}{\pgfqpoint{2.700000in}{1.950000in}}%
\pgfusepath{clip}%
\pgfsetbuttcap%
\pgfsetroundjoin%
\definecolor{currentfill}{rgb}{0.868934,0.762178,0.770634}%
\pgfsetfillcolor{currentfill}%
\pgfsetlinewidth{0.000000pt}%
\definecolor{currentstroke}{rgb}{0.000000,0.000000,0.000000}%
\pgfsetstrokecolor{currentstroke}%
\pgfsetdash{}{0pt}%
\pgfpathmoveto{\pgfqpoint{2.013624in}{0.994841in}}%
\pgfpathlineto{\pgfqpoint{2.048560in}{0.996483in}}%
\pgfpathlineto{\pgfqpoint{2.011801in}{1.029855in}}%
\pgfpathlineto{\pgfqpoint{1.976752in}{1.028329in}}%
\pgfpathclose%
\pgfusepath{fill}%
\end{pgfscope}%
\begin{pgfscope}%
\pgfpathrectangle{\pgfqpoint{0.150000in}{0.150000in}}{\pgfqpoint{2.700000in}{1.950000in}}%
\pgfusepath{clip}%
\pgfsetbuttcap%
\pgfsetroundjoin%
\definecolor{currentfill}{rgb}{0.797871,0.822763,0.857613}%
\pgfsetfillcolor{currentfill}%
\pgfsetlinewidth{0.000000pt}%
\definecolor{currentstroke}{rgb}{0.000000,0.000000,0.000000}%
\pgfsetstrokecolor{currentstroke}%
\pgfsetdash{}{0pt}%
\pgfpathmoveto{\pgfqpoint{1.389630in}{1.485518in}}%
\pgfpathlineto{\pgfqpoint{1.426784in}{1.479430in}}%
\pgfpathlineto{\pgfqpoint{1.390272in}{1.512609in}}%
\pgfpathlineto{\pgfqpoint{1.352986in}{1.518828in}}%
\pgfpathclose%
\pgfusepath{fill}%
\end{pgfscope}%
\begin{pgfscope}%
\pgfpathrectangle{\pgfqpoint{0.150000in}{0.150000in}}{\pgfqpoint{2.700000in}{1.950000in}}%
\pgfusepath{clip}%
\pgfsetbuttcap%
\pgfsetroundjoin%
\definecolor{currentfill}{rgb}{0.986703,0.975873,0.976731}%
\pgfsetfillcolor{currentfill}%
\pgfsetlinewidth{0.000000pt}%
\definecolor{currentstroke}{rgb}{0.000000,0.000000,0.000000}%
\pgfsetstrokecolor{currentstroke}%
\pgfsetdash{}{0pt}%
\pgfpathmoveto{\pgfqpoint{1.720039in}{1.221953in}}%
\pgfpathlineto{\pgfqpoint{1.755931in}{1.219865in}}%
\pgfpathlineto{\pgfqpoint{1.719287in}{1.253150in}}%
\pgfpathlineto{\pgfqpoint{1.683272in}{1.255361in}}%
\pgfpathclose%
\pgfusepath{fill}%
\end{pgfscope}%
\begin{pgfscope}%
\pgfpathrectangle{\pgfqpoint{0.150000in}{0.150000in}}{\pgfqpoint{2.700000in}{1.950000in}}%
\pgfusepath{clip}%
\pgfsetbuttcap%
\pgfsetroundjoin%
\definecolor{currentfill}{rgb}{0.822748,0.844577,0.875138}%
\pgfsetfillcolor{currentfill}%
\pgfsetlinewidth{0.000000pt}%
\definecolor{currentstroke}{rgb}{0.000000,0.000000,0.000000}%
\pgfsetstrokecolor{currentstroke}%
\pgfsetdash{}{0pt}%
\pgfpathmoveto{\pgfqpoint{1.426302in}{1.452183in}}%
\pgfpathlineto{\pgfqpoint{1.463323in}{1.446226in}}%
\pgfpathlineto{\pgfqpoint{1.426784in}{1.479430in}}%
\pgfpathlineto{\pgfqpoint{1.389630in}{1.485518in}}%
\pgfpathclose%
\pgfusepath{fill}%
\end{pgfscope}%
\begin{pgfscope}%
\pgfpathrectangle{\pgfqpoint{0.150000in}{0.150000in}}{\pgfqpoint{2.700000in}{1.950000in}}%
\pgfusepath{clip}%
\pgfsetbuttcap%
\pgfsetroundjoin%
\definecolor{currentfill}{rgb}{0.754963,0.555377,0.571186}%
\pgfsetfillcolor{currentfill}%
\pgfsetlinewidth{0.000000pt}%
\definecolor{currentstroke}{rgb}{0.000000,0.000000,0.000000}%
\pgfsetstrokecolor{currentstroke}%
\pgfsetdash{}{0pt}%
\pgfpathmoveto{\pgfqpoint{2.307518in}{0.773490in}}%
\pgfpathlineto{\pgfqpoint{2.341679in}{0.778721in}}%
\pgfpathlineto{\pgfqpoint{2.304572in}{0.809362in}}%
\pgfpathlineto{\pgfqpoint{2.270521in}{0.807074in}}%
\pgfpathclose%
\pgfusepath{fill}%
\end{pgfscope}%
\begin{pgfscope}%
\pgfpathrectangle{\pgfqpoint{0.150000in}{0.150000in}}{\pgfqpoint{2.700000in}{1.950000in}}%
\pgfusepath{clip}%
\pgfsetbuttcap%
\pgfsetroundjoin%
\definecolor{currentfill}{rgb}{0.971507,0.948300,0.950138}%
\pgfsetfillcolor{currentfill}%
\pgfsetlinewidth{0.000000pt}%
\definecolor{currentstroke}{rgb}{0.000000,0.000000,0.000000}%
\pgfsetstrokecolor{currentstroke}%
\pgfsetdash{}{0pt}%
\pgfpathmoveto{\pgfqpoint{1.756834in}{1.188520in}}%
\pgfpathlineto{\pgfqpoint{1.792674in}{1.189497in}}%
\pgfpathlineto{\pgfqpoint{1.755931in}{1.219865in}}%
\pgfpathlineto{\pgfqpoint{1.720039in}{1.221953in}}%
\pgfpathclose%
\pgfusepath{fill}%
\end{pgfscope}%
\begin{pgfscope}%
\pgfpathrectangle{\pgfqpoint{0.150000in}{0.150000in}}{\pgfqpoint{2.700000in}{1.950000in}}%
\pgfusepath{clip}%
\pgfsetbuttcap%
\pgfsetroundjoin%
\definecolor{currentfill}{rgb}{0.853738,0.734605,0.744041}%
\pgfsetfillcolor{currentfill}%
\pgfsetlinewidth{0.000000pt}%
\definecolor{currentstroke}{rgb}{0.000000,0.000000,0.000000}%
\pgfsetstrokecolor{currentstroke}%
\pgfsetdash{}{0pt}%
\pgfpathmoveto{\pgfqpoint{2.050668in}{0.964216in}}%
\pgfpathlineto{\pgfqpoint{2.085499in}{0.965965in}}%
\pgfpathlineto{\pgfqpoint{2.048560in}{0.996483in}}%
\pgfpathlineto{\pgfqpoint{2.013624in}{0.994841in}}%
\pgfpathclose%
\pgfusepath{fill}%
\end{pgfscope}%
\begin{pgfscope}%
\pgfpathrectangle{\pgfqpoint{0.150000in}{0.150000in}}{\pgfqpoint{2.700000in}{1.950000in}}%
\pgfusepath{clip}%
\pgfsetbuttcap%
\pgfsetroundjoin%
\definecolor{currentfill}{rgb}{0.847626,0.866391,0.892662}%
\pgfsetfillcolor{currentfill}%
\pgfsetlinewidth{0.000000pt}%
\definecolor{currentstroke}{rgb}{0.000000,0.000000,0.000000}%
\pgfsetstrokecolor{currentstroke}%
\pgfsetdash{}{0pt}%
\pgfpathmoveto{\pgfqpoint{1.463002in}{1.418822in}}%
\pgfpathlineto{\pgfqpoint{1.499881in}{1.416000in}}%
\pgfpathlineto{\pgfqpoint{1.463323in}{1.446226in}}%
\pgfpathlineto{\pgfqpoint{1.426302in}{1.452183in}}%
\pgfpathclose%
\pgfusepath{fill}%
\end{pgfscope}%
\begin{pgfscope}%
\pgfpathrectangle{\pgfqpoint{0.150000in}{0.150000in}}{\pgfqpoint{2.700000in}{1.950000in}}%
\pgfusepath{clip}%
\pgfsetbuttcap%
\pgfsetroundjoin%
\definecolor{currentfill}{rgb}{0.960110,0.927619,0.930193}%
\pgfsetfillcolor{currentfill}%
\pgfsetlinewidth{0.000000pt}%
\definecolor{currentstroke}{rgb}{0.000000,0.000000,0.000000}%
\pgfsetstrokecolor{currentstroke}%
\pgfsetdash{}{0pt}%
\pgfpathmoveto{\pgfqpoint{1.793729in}{1.158005in}}%
\pgfpathlineto{\pgfqpoint{1.829384in}{1.156153in}}%
\pgfpathlineto{\pgfqpoint{1.792674in}{1.189497in}}%
\pgfpathlineto{\pgfqpoint{1.756834in}{1.188520in}}%
\pgfpathclose%
\pgfusepath{fill}%
\end{pgfscope}%
\begin{pgfscope}%
\pgfpathrectangle{\pgfqpoint{0.150000in}{0.150000in}}{\pgfqpoint{2.700000in}{1.950000in}}%
\pgfusepath{clip}%
\pgfsetbuttcap%
\pgfsetroundjoin%
\definecolor{currentfill}{rgb}{0.842341,0.713925,0.724096}%
\pgfsetfillcolor{currentfill}%
\pgfsetlinewidth{0.000000pt}%
\definecolor{currentstroke}{rgb}{0.000000,0.000000,0.000000}%
\pgfsetstrokecolor{currentstroke}%
\pgfsetdash{}{0pt}%
\pgfpathmoveto{\pgfqpoint{2.087607in}{0.930669in}}%
\pgfpathlineto{\pgfqpoint{2.122324in}{0.932533in}}%
\pgfpathlineto{\pgfqpoint{2.085499in}{0.965965in}}%
\pgfpathlineto{\pgfqpoint{2.050668in}{0.964216in}}%
\pgfpathclose%
\pgfusepath{fill}%
\end{pgfscope}%
\begin{pgfscope}%
\pgfpathrectangle{\pgfqpoint{0.150000in}{0.150000in}}{\pgfqpoint{2.700000in}{1.950000in}}%
\pgfusepath{clip}%
\pgfsetbuttcap%
\pgfsetroundjoin%
\definecolor{currentfill}{rgb}{0.866284,0.882751,0.905806}%
\pgfsetfillcolor{currentfill}%
\pgfsetlinewidth{0.000000pt}%
\definecolor{currentstroke}{rgb}{0.000000,0.000000,0.000000}%
\pgfsetstrokecolor{currentstroke}%
\pgfsetdash{}{0pt}%
\pgfpathmoveto{\pgfqpoint{1.499720in}{1.388446in}}%
\pgfpathlineto{\pgfqpoint{1.536486in}{1.382737in}}%
\pgfpathlineto{\pgfqpoint{1.499881in}{1.416000in}}%
\pgfpathlineto{\pgfqpoint{1.463002in}{1.418822in}}%
\pgfpathclose%
\pgfusepath{fill}%
\end{pgfscope}%
\begin{pgfscope}%
\pgfpathrectangle{\pgfqpoint{0.150000in}{0.150000in}}{\pgfqpoint{2.700000in}{1.950000in}}%
\pgfusepath{clip}%
\pgfsetbuttcap%
\pgfsetroundjoin%
\definecolor{currentfill}{rgb}{0.944914,0.900046,0.903600}%
\pgfsetfillcolor{currentfill}%
\pgfsetlinewidth{0.000000pt}%
\definecolor{currentstroke}{rgb}{0.000000,0.000000,0.000000}%
\pgfsetstrokecolor{currentstroke}%
\pgfsetdash{}{0pt}%
\pgfpathmoveto{\pgfqpoint{1.830591in}{1.124513in}}%
\pgfpathlineto{\pgfqpoint{1.866214in}{1.125711in}}%
\pgfpathlineto{\pgfqpoint{1.829384in}{1.156153in}}%
\pgfpathlineto{\pgfqpoint{1.793729in}{1.158005in}}%
\pgfpathclose%
\pgfusepath{fill}%
\end{pgfscope}%
\begin{pgfscope}%
\pgfpathrectangle{\pgfqpoint{0.150000in}{0.150000in}}{\pgfqpoint{2.700000in}{1.950000in}}%
\pgfusepath{clip}%
\pgfsetbuttcap%
\pgfsetroundjoin%
\definecolor{currentfill}{rgb}{0.827145,0.686351,0.697503}%
\pgfsetfillcolor{currentfill}%
\pgfsetlinewidth{0.000000pt}%
\definecolor{currentstroke}{rgb}{0.000000,0.000000,0.000000}%
\pgfsetstrokecolor{currentstroke}%
\pgfsetdash{}{0pt}%
\pgfpathmoveto{\pgfqpoint{2.124738in}{0.899969in}}%
\pgfpathlineto{\pgfqpoint{2.159351in}{0.901941in}}%
\pgfpathlineto{\pgfqpoint{2.122324in}{0.932533in}}%
\pgfpathlineto{\pgfqpoint{2.087607in}{0.930669in}}%
\pgfpathclose%
\pgfusepath{fill}%
\end{pgfscope}%
\begin{pgfscope}%
\pgfpathrectangle{\pgfqpoint{0.150000in}{0.150000in}}{\pgfqpoint{2.700000in}{1.950000in}}%
\pgfusepath{clip}%
\pgfsetbuttcap%
\pgfsetroundjoin%
\definecolor{currentfill}{rgb}{0.891161,0.904565,0.923330}%
\pgfsetfillcolor{currentfill}%
\pgfsetlinewidth{0.000000pt}%
\definecolor{currentstroke}{rgb}{0.000000,0.000000,0.000000}%
\pgfsetstrokecolor{currentstroke}%
\pgfsetdash{}{0pt}%
\pgfpathmoveto{\pgfqpoint{1.536486in}{1.355025in}}%
\pgfpathlineto{\pgfqpoint{1.573130in}{1.352438in}}%
\pgfpathlineto{\pgfqpoint{1.536486in}{1.382737in}}%
\pgfpathlineto{\pgfqpoint{1.499720in}{1.388446in}}%
\pgfpathclose%
\pgfusepath{fill}%
\end{pgfscope}%
\begin{pgfscope}%
\pgfpathrectangle{\pgfqpoint{0.150000in}{0.150000in}}{\pgfqpoint{2.700000in}{1.950000in}}%
\pgfusepath{clip}%
\pgfsetbuttcap%
\pgfsetroundjoin%
\definecolor{currentfill}{rgb}{0.929718,0.872472,0.877007}%
\pgfsetfillcolor{currentfill}%
\pgfsetlinewidth{0.000000pt}%
\definecolor{currentstroke}{rgb}{0.000000,0.000000,0.000000}%
\pgfsetstrokecolor{currentstroke}%
\pgfsetdash{}{0pt}%
\pgfpathmoveto{\pgfqpoint{1.867481in}{1.090994in}}%
\pgfpathlineto{\pgfqpoint{1.902991in}{1.092308in}}%
\pgfpathlineto{\pgfqpoint{1.866214in}{1.125711in}}%
\pgfpathlineto{\pgfqpoint{1.830591in}{1.124513in}}%
\pgfpathclose%
\pgfusepath{fill}%
\end{pgfscope}%
\begin{pgfscope}%
\pgfpathrectangle{\pgfqpoint{0.150000in}{0.150000in}}{\pgfqpoint{2.700000in}{1.950000in}}%
\pgfusepath{clip}%
\pgfsetbuttcap%
\pgfsetroundjoin%
\definecolor{currentfill}{rgb}{0.815748,0.665671,0.677558}%
\pgfsetfillcolor{currentfill}%
\pgfsetlinewidth{0.000000pt}%
\definecolor{currentstroke}{rgb}{0.000000,0.000000,0.000000}%
\pgfsetstrokecolor{currentstroke}%
\pgfsetdash{}{0pt}%
\pgfpathmoveto{\pgfqpoint{2.161745in}{0.866361in}}%
\pgfpathlineto{\pgfqpoint{2.196427in}{0.871307in}}%
\pgfpathlineto{\pgfqpoint{2.159351in}{0.901941in}}%
\pgfpathlineto{\pgfqpoint{2.124738in}{0.899969in}}%
\pgfpathclose%
\pgfusepath{fill}%
\end{pgfscope}%
\begin{pgfscope}%
\pgfpathrectangle{\pgfqpoint{0.150000in}{0.150000in}}{\pgfqpoint{2.700000in}{1.950000in}}%
\pgfusepath{clip}%
\pgfsetbuttcap%
\pgfsetroundjoin%
\definecolor{currentfill}{rgb}{0.916039,0.926379,0.940855}%
\pgfsetfillcolor{currentfill}%
\pgfsetlinewidth{0.000000pt}%
\definecolor{currentstroke}{rgb}{0.000000,0.000000,0.000000}%
\pgfsetstrokecolor{currentstroke}%
\pgfsetdash{}{0pt}%
\pgfpathmoveto{\pgfqpoint{1.573282in}{1.321579in}}%
\pgfpathlineto{\pgfqpoint{1.609802in}{1.319115in}}%
\pgfpathlineto{\pgfqpoint{1.573130in}{1.352438in}}%
\pgfpathlineto{\pgfqpoint{1.536486in}{1.355025in}}%
\pgfpathclose%
\pgfusepath{fill}%
\end{pgfscope}%
\begin{pgfscope}%
\pgfpathrectangle{\pgfqpoint{0.150000in}{0.150000in}}{\pgfqpoint{2.700000in}{1.950000in}}%
\pgfusepath{clip}%
\pgfsetbuttcap%
\pgfsetroundjoin%
\definecolor{currentfill}{rgb}{0.914522,0.844899,0.850414}%
\pgfsetfillcolor{currentfill}%
\pgfsetlinewidth{0.000000pt}%
\definecolor{currentstroke}{rgb}{0.000000,0.000000,0.000000}%
\pgfsetstrokecolor{currentstroke}%
\pgfsetdash{}{0pt}%
\pgfpathmoveto{\pgfqpoint{1.904503in}{1.060371in}}%
\pgfpathlineto{\pgfqpoint{1.939796in}{1.058880in}}%
\pgfpathlineto{\pgfqpoint{1.902991in}{1.092308in}}%
\pgfpathlineto{\pgfqpoint{1.867481in}{1.090994in}}%
\pgfpathclose%
\pgfusepath{fill}%
\end{pgfscope}%
\begin{pgfscope}%
\pgfpathrectangle{\pgfqpoint{0.150000in}{0.150000in}}{\pgfqpoint{2.700000in}{1.950000in}}%
\pgfusepath{clip}%
\pgfsetbuttcap%
\pgfsetroundjoin%
\definecolor{currentfill}{rgb}{0.800551,0.638097,0.650965}%
\pgfsetfillcolor{currentfill}%
\pgfsetlinewidth{0.000000pt}%
\definecolor{currentstroke}{rgb}{0.000000,0.000000,0.000000}%
\pgfsetstrokecolor{currentstroke}%
\pgfsetdash{}{0pt}%
\pgfpathmoveto{\pgfqpoint{2.198965in}{0.835586in}}%
\pgfpathlineto{\pgfqpoint{2.233357in}{0.837782in}}%
\pgfpathlineto{\pgfqpoint{2.196427in}{0.871307in}}%
\pgfpathlineto{\pgfqpoint{2.161745in}{0.866361in}}%
\pgfpathclose%
\pgfusepath{fill}%
\end{pgfscope}%
\begin{pgfscope}%
\pgfpathrectangle{\pgfqpoint{0.150000in}{0.150000in}}{\pgfqpoint{2.700000in}{1.950000in}}%
\pgfusepath{clip}%
\pgfsetbuttcap%
\pgfsetroundjoin%
\definecolor{currentfill}{rgb}{0.940916,0.948192,0.958379}%
\pgfsetfillcolor{currentfill}%
\pgfsetlinewidth{0.000000pt}%
\definecolor{currentstroke}{rgb}{0.000000,0.000000,0.000000}%
\pgfsetstrokecolor{currentstroke}%
\pgfsetdash{}{0pt}%
\pgfpathmoveto{\pgfqpoint{1.610105in}{1.288107in}}%
\pgfpathlineto{\pgfqpoint{1.646503in}{1.285767in}}%
\pgfpathlineto{\pgfqpoint{1.609802in}{1.319115in}}%
\pgfpathlineto{\pgfqpoint{1.573282in}{1.321579in}}%
\pgfpathclose%
\pgfusepath{fill}%
\end{pgfscope}%
\begin{pgfscope}%
\pgfpathrectangle{\pgfqpoint{0.150000in}{0.150000in}}{\pgfqpoint{2.700000in}{1.950000in}}%
\pgfusepath{clip}%
\pgfsetbuttcap%
\pgfsetroundjoin%
\definecolor{currentfill}{rgb}{0.710800,0.746415,0.796278}%
\pgfsetfillcolor{currentfill}%
\pgfsetlinewidth{0.000000pt}%
\definecolor{currentstroke}{rgb}{0.000000,0.000000,0.000000}%
\pgfsetstrokecolor{currentstroke}%
\pgfsetdash{}{0pt}%
\pgfpathmoveto{\pgfqpoint{1.278723in}{1.555450in}}%
\pgfpathlineto{\pgfqpoint{1.316432in}{1.549063in}}%
\pgfpathlineto{\pgfqpoint{1.279855in}{1.582314in}}%
\pgfpathlineto{\pgfqpoint{1.242014in}{1.588832in}}%
\pgfpathclose%
\pgfusepath{fill}%
\end{pgfscope}%
\begin{pgfscope}%
\pgfpathrectangle{\pgfqpoint{0.150000in}{0.150000in}}{\pgfqpoint{2.700000in}{1.950000in}}%
\pgfusepath{clip}%
\pgfsetbuttcap%
\pgfsetroundjoin%
\definecolor{currentfill}{rgb}{0.903125,0.824219,0.830469}%
\pgfsetfillcolor{currentfill}%
\pgfsetlinewidth{0.000000pt}%
\definecolor{currentstroke}{rgb}{0.000000,0.000000,0.000000}%
\pgfsetstrokecolor{currentstroke}%
\pgfsetdash{}{0pt}%
\pgfpathmoveto{\pgfqpoint{1.941461in}{1.026792in}}%
\pgfpathlineto{\pgfqpoint{1.976752in}{1.028329in}}%
\pgfpathlineto{\pgfqpoint{1.939796in}{1.058880in}}%
\pgfpathlineto{\pgfqpoint{1.904503in}{1.060371in}}%
\pgfpathclose%
\pgfusepath{fill}%
\end{pgfscope}%
\begin{pgfscope}%
\pgfpathrectangle{\pgfqpoint{0.150000in}{0.150000in}}{\pgfqpoint{2.700000in}{1.950000in}}%
\pgfusepath{clip}%
\pgfsetbuttcap%
\pgfsetroundjoin%
\definecolor{currentfill}{rgb}{0.785355,0.610524,0.624372}%
\pgfsetfillcolor{currentfill}%
\pgfsetlinewidth{0.000000pt}%
\definecolor{currentstroke}{rgb}{0.000000,0.000000,0.000000}%
\pgfsetstrokecolor{currentstroke}%
\pgfsetdash{}{0pt}%
\pgfpathmoveto{\pgfqpoint{2.236038in}{0.801919in}}%
\pgfpathlineto{\pgfqpoint{2.270521in}{0.807074in}}%
\pgfpathlineto{\pgfqpoint{2.233357in}{0.837782in}}%
\pgfpathlineto{\pgfqpoint{2.198965in}{0.835586in}}%
\pgfpathclose%
\pgfusepath{fill}%
\end{pgfscope}%
\begin{pgfscope}%
\pgfpathrectangle{\pgfqpoint{0.150000in}{0.150000in}}{\pgfqpoint{2.700000in}{1.950000in}}%
\pgfusepath{clip}%
\pgfsetbuttcap%
\pgfsetroundjoin%
\definecolor{currentfill}{rgb}{0.959574,0.964553,0.971523}%
\pgfsetfillcolor{currentfill}%
\pgfsetlinewidth{0.000000pt}%
\definecolor{currentstroke}{rgb}{0.000000,0.000000,0.000000}%
\pgfsetstrokecolor{currentstroke}%
\pgfsetdash{}{0pt}%
\pgfpathmoveto{\pgfqpoint{1.646988in}{1.257588in}}%
\pgfpathlineto{\pgfqpoint{1.683272in}{1.255361in}}%
\pgfpathlineto{\pgfqpoint{1.646503in}{1.285767in}}%
\pgfpathlineto{\pgfqpoint{1.610105in}{1.288107in}}%
\pgfpathclose%
\pgfusepath{fill}%
\end{pgfscope}%
\begin{pgfscope}%
\pgfpathrectangle{\pgfqpoint{0.150000in}{0.150000in}}{\pgfqpoint{2.700000in}{1.950000in}}%
\pgfusepath{clip}%
\pgfsetbuttcap%
\pgfsetroundjoin%
\definecolor{currentfill}{rgb}{0.735677,0.768229,0.813802}%
\pgfsetfillcolor{currentfill}%
\pgfsetlinewidth{0.000000pt}%
\definecolor{currentstroke}{rgb}{0.000000,0.000000,0.000000}%
\pgfsetstrokecolor{currentstroke}%
\pgfsetdash{}{0pt}%
\pgfpathmoveto{\pgfqpoint{1.315462in}{1.522042in}}%
\pgfpathlineto{\pgfqpoint{1.352986in}{1.518828in}}%
\pgfpathlineto{\pgfqpoint{1.316432in}{1.549063in}}%
\pgfpathlineto{\pgfqpoint{1.278723in}{1.555450in}}%
\pgfpathclose%
\pgfusepath{fill}%
\end{pgfscope}%
\begin{pgfscope}%
\pgfpathrectangle{\pgfqpoint{0.150000in}{0.150000in}}{\pgfqpoint{2.700000in}{1.950000in}}%
\pgfusepath{clip}%
\pgfsetbuttcap%
\pgfsetroundjoin%
\definecolor{currentfill}{rgb}{0.887929,0.796645,0.803876}%
\pgfsetfillcolor{currentfill}%
\pgfsetlinewidth{0.000000pt}%
\definecolor{currentstroke}{rgb}{0.000000,0.000000,0.000000}%
\pgfsetstrokecolor{currentstroke}%
\pgfsetdash{}{0pt}%
\pgfpathmoveto{\pgfqpoint{1.978447in}{0.993188in}}%
\pgfpathlineto{\pgfqpoint{2.013624in}{0.994841in}}%
\pgfpathlineto{\pgfqpoint{1.976752in}{1.028329in}}%
\pgfpathlineto{\pgfqpoint{1.941461in}{1.026792in}}%
\pgfpathclose%
\pgfusepath{fill}%
\end{pgfscope}%
\begin{pgfscope}%
\pgfpathrectangle{\pgfqpoint{0.150000in}{0.150000in}}{\pgfqpoint{2.700000in}{1.950000in}}%
\pgfusepath{clip}%
\pgfsetbuttcap%
\pgfsetroundjoin%
\definecolor{currentfill}{rgb}{0.773958,0.589844,0.604427}%
\pgfsetfillcolor{currentfill}%
\pgfsetlinewidth{0.000000pt}%
\definecolor{currentstroke}{rgb}{0.000000,0.000000,0.000000}%
\pgfsetstrokecolor{currentstroke}%
\pgfsetdash{}{0pt}%
\pgfpathmoveto{\pgfqpoint{2.273347in}{0.771069in}}%
\pgfpathlineto{\pgfqpoint{2.307518in}{0.773490in}}%
\pgfpathlineto{\pgfqpoint{2.270521in}{0.807074in}}%
\pgfpathlineto{\pgfqpoint{2.236038in}{0.801919in}}%
\pgfpathclose%
\pgfusepath{fill}%
\end{pgfscope}%
\begin{pgfscope}%
\pgfpathrectangle{\pgfqpoint{0.150000in}{0.150000in}}{\pgfqpoint{2.700000in}{1.950000in}}%
\pgfusepath{clip}%
\pgfsetbuttcap%
\pgfsetroundjoin%
\definecolor{currentfill}{rgb}{0.984452,0.986366,0.989047}%
\pgfsetfillcolor{currentfill}%
\pgfsetlinewidth{0.000000pt}%
\definecolor{currentstroke}{rgb}{0.000000,0.000000,0.000000}%
\pgfsetstrokecolor{currentstroke}%
\pgfsetdash{}{0pt}%
\pgfpathmoveto{\pgfqpoint{1.683878in}{1.224057in}}%
\pgfpathlineto{\pgfqpoint{1.720039in}{1.221953in}}%
\pgfpathlineto{\pgfqpoint{1.683272in}{1.255361in}}%
\pgfpathlineto{\pgfqpoint{1.646988in}{1.257588in}}%
\pgfpathclose%
\pgfusepath{fill}%
\end{pgfscope}%
\begin{pgfscope}%
\pgfpathrectangle{\pgfqpoint{0.150000in}{0.150000in}}{\pgfqpoint{2.700000in}{1.950000in}}%
\pgfusepath{clip}%
\pgfsetbuttcap%
\pgfsetroundjoin%
\definecolor{currentfill}{rgb}{0.760555,0.790043,0.831327}%
\pgfsetfillcolor{currentfill}%
\pgfsetlinewidth{0.000000pt}%
\definecolor{currentstroke}{rgb}{0.000000,0.000000,0.000000}%
\pgfsetstrokecolor{currentstroke}%
\pgfsetdash{}{0pt}%
\pgfpathmoveto{\pgfqpoint{1.352228in}{1.488609in}}%
\pgfpathlineto{\pgfqpoint{1.389630in}{1.485518in}}%
\pgfpathlineto{\pgfqpoint{1.352986in}{1.518828in}}%
\pgfpathlineto{\pgfqpoint{1.315462in}{1.522042in}}%
\pgfpathclose%
\pgfusepath{fill}%
\end{pgfscope}%
\begin{pgfscope}%
\pgfpathrectangle{\pgfqpoint{0.150000in}{0.150000in}}{\pgfqpoint{2.700000in}{1.950000in}}%
\pgfusepath{clip}%
\pgfsetbuttcap%
\pgfsetroundjoin%
\definecolor{currentfill}{rgb}{0.872733,0.769072,0.777282}%
\pgfsetfillcolor{currentfill}%
\pgfsetlinewidth{0.000000pt}%
\definecolor{currentstroke}{rgb}{0.000000,0.000000,0.000000}%
\pgfsetstrokecolor{currentstroke}%
\pgfsetdash{}{0pt}%
\pgfpathmoveto{\pgfqpoint{2.015595in}{0.962455in}}%
\pgfpathlineto{\pgfqpoint{2.050668in}{0.964216in}}%
\pgfpathlineto{\pgfqpoint{2.013624in}{0.994841in}}%
\pgfpathlineto{\pgfqpoint{1.978447in}{0.993188in}}%
\pgfpathclose%
\pgfusepath{fill}%
\end{pgfscope}%
\begin{pgfscope}%
\pgfpathrectangle{\pgfqpoint{0.150000in}{0.150000in}}{\pgfqpoint{2.700000in}{1.950000in}}%
\pgfusepath{clip}%
\pgfsetbuttcap%
\pgfsetroundjoin%
\definecolor{currentfill}{rgb}{0.994301,0.989660,0.990028}%
\pgfsetfillcolor{currentfill}%
\pgfsetlinewidth{0.000000pt}%
\definecolor{currentstroke}{rgb}{0.000000,0.000000,0.000000}%
\pgfsetstrokecolor{currentstroke}%
\pgfsetdash{}{0pt}%
\pgfpathmoveto{\pgfqpoint{1.720797in}{1.190499in}}%
\pgfpathlineto{\pgfqpoint{1.756834in}{1.188520in}}%
\pgfpathlineto{\pgfqpoint{1.720039in}{1.221953in}}%
\pgfpathlineto{\pgfqpoint{1.683878in}{1.224057in}}%
\pgfpathclose%
\pgfusepath{fill}%
\end{pgfscope}%
\begin{pgfscope}%
\pgfpathrectangle{\pgfqpoint{0.150000in}{0.150000in}}{\pgfqpoint{2.700000in}{1.950000in}}%
\pgfusepath{clip}%
\pgfsetbuttcap%
\pgfsetroundjoin%
\definecolor{currentfill}{rgb}{0.779213,0.806403,0.844470}%
\pgfsetfillcolor{currentfill}%
\pgfsetlinewidth{0.000000pt}%
\definecolor{currentstroke}{rgb}{0.000000,0.000000,0.000000}%
\pgfsetstrokecolor{currentstroke}%
\pgfsetdash{}{0pt}%
\pgfpathmoveto{\pgfqpoint{1.388982in}{1.458189in}}%
\pgfpathlineto{\pgfqpoint{1.426302in}{1.452183in}}%
\pgfpathlineto{\pgfqpoint{1.389630in}{1.485518in}}%
\pgfpathlineto{\pgfqpoint{1.352228in}{1.488609in}}%
\pgfpathclose%
\pgfusepath{fill}%
\end{pgfscope}%
\begin{pgfscope}%
\pgfpathrectangle{\pgfqpoint{0.150000in}{0.150000in}}{\pgfqpoint{2.700000in}{1.950000in}}%
\pgfusepath{clip}%
\pgfsetbuttcap%
\pgfsetroundjoin%
\definecolor{currentfill}{rgb}{0.861336,0.748392,0.757338}%
\pgfsetfillcolor{currentfill}%
\pgfsetlinewidth{0.000000pt}%
\definecolor{currentstroke}{rgb}{0.000000,0.000000,0.000000}%
\pgfsetstrokecolor{currentstroke}%
\pgfsetdash{}{0pt}%
\pgfpathmoveto{\pgfqpoint{2.052649in}{0.928791in}}%
\pgfpathlineto{\pgfqpoint{2.087607in}{0.930669in}}%
\pgfpathlineto{\pgfqpoint{2.050668in}{0.964216in}}%
\pgfpathlineto{\pgfqpoint{2.015595in}{0.962455in}}%
\pgfpathclose%
\pgfusepath{fill}%
\end{pgfscope}%
\begin{pgfscope}%
\pgfpathrectangle{\pgfqpoint{0.150000in}{0.150000in}}{\pgfqpoint{2.700000in}{1.950000in}}%
\pgfusepath{clip}%
\pgfsetbuttcap%
\pgfsetroundjoin%
\definecolor{currentfill}{rgb}{0.979105,0.962086,0.963434}%
\pgfsetfillcolor{currentfill}%
\pgfsetlinewidth{0.000000pt}%
\definecolor{currentstroke}{rgb}{0.000000,0.000000,0.000000}%
\pgfsetstrokecolor{currentstroke}%
\pgfsetdash{}{0pt}%
\pgfpathmoveto{\pgfqpoint{1.757745in}{1.156916in}}%
\pgfpathlineto{\pgfqpoint{1.793729in}{1.158005in}}%
\pgfpathlineto{\pgfqpoint{1.756834in}{1.188520in}}%
\pgfpathlineto{\pgfqpoint{1.720797in}{1.190499in}}%
\pgfpathclose%
\pgfusepath{fill}%
\end{pgfscope}%
\begin{pgfscope}%
\pgfpathrectangle{\pgfqpoint{0.150000in}{0.150000in}}{\pgfqpoint{2.700000in}{1.950000in}}%
\pgfusepath{clip}%
\pgfsetbuttcap%
\pgfsetroundjoin%
\definecolor{currentfill}{rgb}{0.804090,0.828217,0.861994}%
\pgfsetfillcolor{currentfill}%
\pgfsetlinewidth{0.000000pt}%
\definecolor{currentstroke}{rgb}{0.000000,0.000000,0.000000}%
\pgfsetstrokecolor{currentstroke}%
\pgfsetdash{}{0pt}%
\pgfpathmoveto{\pgfqpoint{1.425815in}{1.424696in}}%
\pgfpathlineto{\pgfqpoint{1.463002in}{1.418822in}}%
\pgfpathlineto{\pgfqpoint{1.426302in}{1.452183in}}%
\pgfpathlineto{\pgfqpoint{1.388982in}{1.458189in}}%
\pgfpathclose%
\pgfusepath{fill}%
\end{pgfscope}%
\begin{pgfscope}%
\pgfpathrectangle{\pgfqpoint{0.150000in}{0.150000in}}{\pgfqpoint{2.700000in}{1.950000in}}%
\pgfusepath{clip}%
\pgfsetbuttcap%
\pgfsetroundjoin%
\definecolor{currentfill}{rgb}{0.846140,0.720818,0.730744}%
\pgfsetfillcolor{currentfill}%
\pgfsetlinewidth{0.000000pt}%
\definecolor{currentstroke}{rgb}{0.000000,0.000000,0.000000}%
\pgfsetstrokecolor{currentstroke}%
\pgfsetdash{}{0pt}%
\pgfpathmoveto{\pgfqpoint{2.089886in}{0.897983in}}%
\pgfpathlineto{\pgfqpoint{2.124738in}{0.899969in}}%
\pgfpathlineto{\pgfqpoint{2.087607in}{0.930669in}}%
\pgfpathlineto{\pgfqpoint{2.052649in}{0.928791in}}%
\pgfpathclose%
\pgfusepath{fill}%
\end{pgfscope}%
\begin{pgfscope}%
\pgfpathrectangle{\pgfqpoint{0.150000in}{0.150000in}}{\pgfqpoint{2.700000in}{1.950000in}}%
\pgfusepath{clip}%
\pgfsetbuttcap%
\pgfsetroundjoin%
\definecolor{currentfill}{rgb}{0.963909,0.934513,0.936841}%
\pgfsetfillcolor{currentfill}%
\pgfsetlinewidth{0.000000pt}%
\definecolor{currentstroke}{rgb}{0.000000,0.000000,0.000000}%
\pgfsetstrokecolor{currentstroke}%
\pgfsetdash{}{0pt}%
\pgfpathmoveto{\pgfqpoint{1.794793in}{1.126253in}}%
\pgfpathlineto{\pgfqpoint{1.830591in}{1.124513in}}%
\pgfpathlineto{\pgfqpoint{1.793729in}{1.158005in}}%
\pgfpathlineto{\pgfqpoint{1.757745in}{1.156916in}}%
\pgfpathclose%
\pgfusepath{fill}%
\end{pgfscope}%
\begin{pgfscope}%
\pgfpathrectangle{\pgfqpoint{0.150000in}{0.150000in}}{\pgfqpoint{2.700000in}{1.950000in}}%
\pgfusepath{clip}%
\pgfsetbuttcap%
\pgfsetroundjoin%
\definecolor{currentfill}{rgb}{0.828968,0.850031,0.879519}%
\pgfsetfillcolor{currentfill}%
\pgfsetlinewidth{0.000000pt}%
\definecolor{currentstroke}{rgb}{0.000000,0.000000,0.000000}%
\pgfsetstrokecolor{currentstroke}%
\pgfsetdash{}{0pt}%
\pgfpathmoveto{\pgfqpoint{1.462677in}{1.391177in}}%
\pgfpathlineto{\pgfqpoint{1.499720in}{1.388446in}}%
\pgfpathlineto{\pgfqpoint{1.463002in}{1.418822in}}%
\pgfpathlineto{\pgfqpoint{1.425815in}{1.424696in}}%
\pgfpathclose%
\pgfusepath{fill}%
\end{pgfscope}%
\begin{pgfscope}%
\pgfpathrectangle{\pgfqpoint{0.150000in}{0.150000in}}{\pgfqpoint{2.700000in}{1.950000in}}%
\pgfusepath{clip}%
\pgfsetbuttcap%
\pgfsetroundjoin%
\definecolor{currentfill}{rgb}{0.830944,0.693244,0.704151}%
\pgfsetfillcolor{currentfill}%
\pgfsetlinewidth{0.000000pt}%
\definecolor{currentstroke}{rgb}{0.000000,0.000000,0.000000}%
\pgfsetstrokecolor{currentstroke}%
\pgfsetdash{}{0pt}%
\pgfpathmoveto{\pgfqpoint{2.127007in}{0.864258in}}%
\pgfpathlineto{\pgfqpoint{2.161745in}{0.866361in}}%
\pgfpathlineto{\pgfqpoint{2.124738in}{0.899969in}}%
\pgfpathlineto{\pgfqpoint{2.089886in}{0.897983in}}%
\pgfpathclose%
\pgfusepath{fill}%
\end{pgfscope}%
\begin{pgfscope}%
\pgfpathrectangle{\pgfqpoint{0.150000in}{0.150000in}}{\pgfqpoint{2.700000in}{1.950000in}}%
\pgfusepath{clip}%
\pgfsetbuttcap%
\pgfsetroundjoin%
\definecolor{currentfill}{rgb}{0.952512,0.913833,0.916896}%
\pgfsetfillcolor{currentfill}%
\pgfsetlinewidth{0.000000pt}%
\definecolor{currentstroke}{rgb}{0.000000,0.000000,0.000000}%
\pgfsetstrokecolor{currentstroke}%
\pgfsetdash{}{0pt}%
\pgfpathmoveto{\pgfqpoint{1.831808in}{1.092610in}}%
\pgfpathlineto{\pgfqpoint{1.867481in}{1.090994in}}%
\pgfpathlineto{\pgfqpoint{1.830591in}{1.124513in}}%
\pgfpathlineto{\pgfqpoint{1.794793in}{1.126253in}}%
\pgfpathclose%
\pgfusepath{fill}%
\end{pgfscope}%
\begin{pgfscope}%
\pgfpathrectangle{\pgfqpoint{0.150000in}{0.150000in}}{\pgfqpoint{2.700000in}{1.950000in}}%
\pgfusepath{clip}%
\pgfsetbuttcap%
\pgfsetroundjoin%
\definecolor{currentfill}{rgb}{0.853845,0.871844,0.897044}%
\pgfsetfillcolor{currentfill}%
\pgfsetlinewidth{0.000000pt}%
\definecolor{currentstroke}{rgb}{0.000000,0.000000,0.000000}%
\pgfsetstrokecolor{currentstroke}%
\pgfsetdash{}{0pt}%
\pgfpathmoveto{\pgfqpoint{1.499568in}{1.357632in}}%
\pgfpathlineto{\pgfqpoint{1.536486in}{1.355025in}}%
\pgfpathlineto{\pgfqpoint{1.499720in}{1.388446in}}%
\pgfpathlineto{\pgfqpoint{1.462677in}{1.391177in}}%
\pgfpathclose%
\pgfusepath{fill}%
\end{pgfscope}%
\begin{pgfscope}%
\pgfpathrectangle{\pgfqpoint{0.150000in}{0.150000in}}{\pgfqpoint{2.700000in}{1.950000in}}%
\pgfusepath{clip}%
\pgfsetbuttcap%
\pgfsetroundjoin%
\definecolor{currentfill}{rgb}{0.819547,0.672564,0.684206}%
\pgfsetfillcolor{currentfill}%
\pgfsetlinewidth{0.000000pt}%
\definecolor{currentstroke}{rgb}{0.000000,0.000000,0.000000}%
\pgfsetstrokecolor{currentstroke}%
\pgfsetdash{}{0pt}%
\pgfpathmoveto{\pgfqpoint{2.164157in}{0.830508in}}%
\pgfpathlineto{\pgfqpoint{2.198965in}{0.835586in}}%
\pgfpathlineto{\pgfqpoint{2.161745in}{0.866361in}}%
\pgfpathlineto{\pgfqpoint{2.127007in}{0.864258in}}%
\pgfpathclose%
\pgfusepath{fill}%
\end{pgfscope}%
\begin{pgfscope}%
\pgfpathrectangle{\pgfqpoint{0.150000in}{0.150000in}}{\pgfqpoint{2.700000in}{1.950000in}}%
\pgfusepath{clip}%
\pgfsetbuttcap%
\pgfsetroundjoin%
\definecolor{currentfill}{rgb}{0.937316,0.886259,0.890303}%
\pgfsetfillcolor{currentfill}%
\pgfsetlinewidth{0.000000pt}%
\definecolor{currentstroke}{rgb}{0.000000,0.000000,0.000000}%
\pgfsetstrokecolor{currentstroke}%
\pgfsetdash{}{0pt}%
\pgfpathmoveto{\pgfqpoint{1.868852in}{1.058940in}}%
\pgfpathlineto{\pgfqpoint{1.904503in}{1.060371in}}%
\pgfpathlineto{\pgfqpoint{1.867481in}{1.090994in}}%
\pgfpathlineto{\pgfqpoint{1.831808in}{1.092610in}}%
\pgfpathclose%
\pgfusepath{fill}%
\end{pgfscope}%
\begin{pgfscope}%
\pgfpathrectangle{\pgfqpoint{0.150000in}{0.150000in}}{\pgfqpoint{2.700000in}{1.950000in}}%
\pgfusepath{clip}%
\pgfsetbuttcap%
\pgfsetroundjoin%
\definecolor{currentfill}{rgb}{0.878722,0.893658,0.914568}%
\pgfsetfillcolor{currentfill}%
\pgfsetlinewidth{0.000000pt}%
\definecolor{currentstroke}{rgb}{0.000000,0.000000,0.000000}%
\pgfsetstrokecolor{currentstroke}%
\pgfsetdash{}{0pt}%
\pgfpathmoveto{\pgfqpoint{1.536486in}{1.324061in}}%
\pgfpathlineto{\pgfqpoint{1.573282in}{1.321579in}}%
\pgfpathlineto{\pgfqpoint{1.536486in}{1.355025in}}%
\pgfpathlineto{\pgfqpoint{1.499568in}{1.357632in}}%
\pgfpathclose%
\pgfusepath{fill}%
\end{pgfscope}%
\begin{pgfscope}%
\pgfpathrectangle{\pgfqpoint{0.150000in}{0.150000in}}{\pgfqpoint{2.700000in}{1.950000in}}%
\pgfusepath{clip}%
\pgfsetbuttcap%
\pgfsetroundjoin%
\definecolor{currentfill}{rgb}{0.903600,0.915472,0.932093}%
\pgfsetfillcolor{currentfill}%
\pgfsetlinewidth{0.000000pt}%
\definecolor{currentstroke}{rgb}{0.000000,0.000000,0.000000}%
\pgfsetstrokecolor{currentstroke}%
\pgfsetdash{}{0pt}%
\pgfpathmoveto{\pgfqpoint{1.573434in}{1.290465in}}%
\pgfpathlineto{\pgfqpoint{1.610105in}{1.288107in}}%
\pgfpathlineto{\pgfqpoint{1.573282in}{1.321579in}}%
\pgfpathlineto{\pgfqpoint{1.536486in}{1.324061in}}%
\pgfpathclose%
\pgfusepath{fill}%
\end{pgfscope}%
\begin{pgfscope}%
\pgfpathrectangle{\pgfqpoint{0.150000in}{0.150000in}}{\pgfqpoint{2.700000in}{1.950000in}}%
\pgfusepath{clip}%
\pgfsetbuttcap%
\pgfsetroundjoin%
\definecolor{currentfill}{rgb}{0.922120,0.858686,0.863710}%
\pgfsetfillcolor{currentfill}%
\pgfsetlinewidth{0.000000pt}%
\definecolor{currentstroke}{rgb}{0.000000,0.000000,0.000000}%
\pgfsetstrokecolor{currentstroke}%
\pgfsetdash{}{0pt}%
\pgfpathmoveto{\pgfqpoint{1.905924in}{1.025245in}}%
\pgfpathlineto{\pgfqpoint{1.941461in}{1.026792in}}%
\pgfpathlineto{\pgfqpoint{1.904503in}{1.060371in}}%
\pgfpathlineto{\pgfqpoint{1.868852in}{1.058940in}}%
\pgfpathclose%
\pgfusepath{fill}%
\end{pgfscope}%
\begin{pgfscope}%
\pgfpathrectangle{\pgfqpoint{0.150000in}{0.150000in}}{\pgfqpoint{2.700000in}{1.950000in}}%
\pgfusepath{clip}%
\pgfsetbuttcap%
\pgfsetroundjoin%
\definecolor{currentfill}{rgb}{0.804350,0.644991,0.657613}%
\pgfsetfillcolor{currentfill}%
\pgfsetlinewidth{0.000000pt}%
\definecolor{currentstroke}{rgb}{0.000000,0.000000,0.000000}%
\pgfsetstrokecolor{currentstroke}%
\pgfsetdash{}{0pt}%
\pgfpathmoveto{\pgfqpoint{2.201522in}{0.799590in}}%
\pgfpathlineto{\pgfqpoint{2.236038in}{0.801919in}}%
\pgfpathlineto{\pgfqpoint{2.198965in}{0.835586in}}%
\pgfpathlineto{\pgfqpoint{2.164157in}{0.830508in}}%
\pgfpathclose%
\pgfusepath{fill}%
\end{pgfscope}%
\begin{pgfscope}%
\pgfpathrectangle{\pgfqpoint{0.150000in}{0.150000in}}{\pgfqpoint{2.700000in}{1.950000in}}%
\pgfusepath{clip}%
\pgfsetbuttcap%
\pgfsetroundjoin%
\definecolor{currentfill}{rgb}{0.928477,0.937286,0.949617}%
\pgfsetfillcolor{currentfill}%
\pgfsetlinewidth{0.000000pt}%
\definecolor{currentstroke}{rgb}{0.000000,0.000000,0.000000}%
\pgfsetstrokecolor{currentstroke}%
\pgfsetdash{}{0pt}%
\pgfpathmoveto{\pgfqpoint{1.610410in}{1.256842in}}%
\pgfpathlineto{\pgfqpoint{1.646988in}{1.257588in}}%
\pgfpathlineto{\pgfqpoint{1.610105in}{1.288107in}}%
\pgfpathlineto{\pgfqpoint{1.573434in}{1.290465in}}%
\pgfpathclose%
\pgfusepath{fill}%
\end{pgfscope}%
\begin{pgfscope}%
\pgfpathrectangle{\pgfqpoint{0.150000in}{0.150000in}}{\pgfqpoint{2.700000in}{1.950000in}}%
\pgfusepath{clip}%
\pgfsetbuttcap%
\pgfsetroundjoin%
\definecolor{currentfill}{rgb}{0.906924,0.831112,0.837117}%
\pgfsetfillcolor{currentfill}%
\pgfsetlinewidth{0.000000pt}%
\definecolor{currentstroke}{rgb}{0.000000,0.000000,0.000000}%
\pgfsetstrokecolor{currentstroke}%
\pgfsetdash{}{0pt}%
\pgfpathmoveto{\pgfqpoint{1.943139in}{0.994439in}}%
\pgfpathlineto{\pgfqpoint{1.978447in}{0.993188in}}%
\pgfpathlineto{\pgfqpoint{1.941461in}{1.026792in}}%
\pgfpathlineto{\pgfqpoint{1.905924in}{1.025245in}}%
\pgfpathclose%
\pgfusepath{fill}%
\end{pgfscope}%
\begin{pgfscope}%
\pgfpathrectangle{\pgfqpoint{0.150000in}{0.150000in}}{\pgfqpoint{2.700000in}{1.950000in}}%
\pgfusepath{clip}%
\pgfsetbuttcap%
\pgfsetroundjoin%
\definecolor{currentfill}{rgb}{0.667264,0.708241,0.765610}%
\pgfsetfillcolor{currentfill}%
\pgfsetlinewidth{0.000000pt}%
\definecolor{currentstroke}{rgb}{0.000000,0.000000,0.000000}%
\pgfsetstrokecolor{currentstroke}%
\pgfsetdash{}{0pt}%
\pgfpathmoveto{\pgfqpoint{1.240710in}{1.561888in}}%
\pgfpathlineto{\pgfqpoint{1.278723in}{1.555450in}}%
\pgfpathlineto{\pgfqpoint{1.242014in}{1.588832in}}%
\pgfpathlineto{\pgfqpoint{1.203867in}{1.595402in}}%
\pgfpathclose%
\pgfusepath{fill}%
\end{pgfscope}%
\begin{pgfscope}%
\pgfpathrectangle{\pgfqpoint{0.150000in}{0.150000in}}{\pgfqpoint{2.700000in}{1.950000in}}%
\pgfusepath{clip}%
\pgfsetbuttcap%
\pgfsetroundjoin%
\definecolor{currentfill}{rgb}{0.789154,0.617417,0.631020}%
\pgfsetfillcolor{currentfill}%
\pgfsetlinewidth{0.000000pt}%
\definecolor{currentstroke}{rgb}{0.000000,0.000000,0.000000}%
\pgfsetstrokecolor{currentstroke}%
\pgfsetdash{}{0pt}%
\pgfpathmoveto{\pgfqpoint{2.238740in}{0.765779in}}%
\pgfpathlineto{\pgfqpoint{2.273347in}{0.771069in}}%
\pgfpathlineto{\pgfqpoint{2.236038in}{0.801919in}}%
\pgfpathlineto{\pgfqpoint{2.201522in}{0.799590in}}%
\pgfpathclose%
\pgfusepath{fill}%
\end{pgfscope}%
\begin{pgfscope}%
\pgfpathrectangle{\pgfqpoint{0.150000in}{0.150000in}}{\pgfqpoint{2.700000in}{1.950000in}}%
\pgfusepath{clip}%
\pgfsetbuttcap%
\pgfsetroundjoin%
\definecolor{currentfill}{rgb}{0.692142,0.730055,0.783134}%
\pgfsetfillcolor{currentfill}%
\pgfsetlinewidth{0.000000pt}%
\definecolor{currentstroke}{rgb}{0.000000,0.000000,0.000000}%
\pgfsetstrokecolor{currentstroke}%
\pgfsetdash{}{0pt}%
\pgfpathmoveto{\pgfqpoint{1.277582in}{1.528348in}}%
\pgfpathlineto{\pgfqpoint{1.315462in}{1.522042in}}%
\pgfpathlineto{\pgfqpoint{1.278723in}{1.555450in}}%
\pgfpathlineto{\pgfqpoint{1.240710in}{1.561888in}}%
\pgfpathclose%
\pgfusepath{fill}%
\end{pgfscope}%
\begin{pgfscope}%
\pgfpathrectangle{\pgfqpoint{0.150000in}{0.150000in}}{\pgfqpoint{2.700000in}{1.950000in}}%
\pgfusepath{clip}%
\pgfsetbuttcap%
\pgfsetroundjoin%
\definecolor{currentfill}{rgb}{0.953355,0.959099,0.967142}%
\pgfsetfillcolor{currentfill}%
\pgfsetlinewidth{0.000000pt}%
\definecolor{currentstroke}{rgb}{0.000000,0.000000,0.000000}%
\pgfsetstrokecolor{currentstroke}%
\pgfsetdash{}{0pt}%
\pgfpathmoveto{\pgfqpoint{1.647446in}{1.226176in}}%
\pgfpathlineto{\pgfqpoint{1.683878in}{1.224057in}}%
\pgfpathlineto{\pgfqpoint{1.646988in}{1.257588in}}%
\pgfpathlineto{\pgfqpoint{1.610410in}{1.256842in}}%
\pgfpathclose%
\pgfusepath{fill}%
\end{pgfscope}%
\begin{pgfscope}%
\pgfpathrectangle{\pgfqpoint{0.150000in}{0.150000in}}{\pgfqpoint{2.700000in}{1.950000in}}%
\pgfusepath{clip}%
\pgfsetbuttcap%
\pgfsetroundjoin%
\definecolor{currentfill}{rgb}{0.891728,0.803539,0.810524}%
\pgfsetfillcolor{currentfill}%
\pgfsetlinewidth{0.000000pt}%
\definecolor{currentstroke}{rgb}{0.000000,0.000000,0.000000}%
\pgfsetstrokecolor{currentstroke}%
\pgfsetdash{}{0pt}%
\pgfpathmoveto{\pgfqpoint{1.980279in}{0.960682in}}%
\pgfpathlineto{\pgfqpoint{2.015595in}{0.962455in}}%
\pgfpathlineto{\pgfqpoint{1.978447in}{0.993188in}}%
\pgfpathlineto{\pgfqpoint{1.943139in}{0.994439in}}%
\pgfpathclose%
\pgfusepath{fill}%
\end{pgfscope}%
\begin{pgfscope}%
\pgfpathrectangle{\pgfqpoint{0.150000in}{0.150000in}}{\pgfqpoint{2.700000in}{1.950000in}}%
\pgfusepath{clip}%
\pgfsetbuttcap%
\pgfsetroundjoin%
\definecolor{currentfill}{rgb}{0.723238,0.757322,0.805040}%
\pgfsetfillcolor{currentfill}%
\pgfsetlinewidth{0.000000pt}%
\definecolor{currentstroke}{rgb}{0.000000,0.000000,0.000000}%
\pgfsetstrokecolor{currentstroke}%
\pgfsetdash{}{0pt}%
\pgfpathmoveto{\pgfqpoint{1.314483in}{1.494782in}}%
\pgfpathlineto{\pgfqpoint{1.352228in}{1.488609in}}%
\pgfpathlineto{\pgfqpoint{1.315462in}{1.522042in}}%
\pgfpathlineto{\pgfqpoint{1.277582in}{1.528348in}}%
\pgfpathclose%
\pgfusepath{fill}%
\end{pgfscope}%
\begin{pgfscope}%
\pgfpathrectangle{\pgfqpoint{0.150000in}{0.150000in}}{\pgfqpoint{2.700000in}{1.950000in}}%
\pgfusepath{clip}%
\pgfsetbuttcap%
\pgfsetroundjoin%
\definecolor{currentfill}{rgb}{0.972013,0.975460,0.980285}%
\pgfsetfillcolor{currentfill}%
\pgfsetlinewidth{0.000000pt}%
\definecolor{currentstroke}{rgb}{0.000000,0.000000,0.000000}%
\pgfsetstrokecolor{currentstroke}%
\pgfsetdash{}{0pt}%
\pgfpathmoveto{\pgfqpoint{1.684490in}{1.192493in}}%
\pgfpathlineto{\pgfqpoint{1.720797in}{1.190499in}}%
\pgfpathlineto{\pgfqpoint{1.683878in}{1.224057in}}%
\pgfpathlineto{\pgfqpoint{1.647446in}{1.226176in}}%
\pgfpathclose%
\pgfusepath{fill}%
\end{pgfscope}%
\begin{pgfscope}%
\pgfpathrectangle{\pgfqpoint{0.150000in}{0.150000in}}{\pgfqpoint{2.700000in}{1.950000in}}%
\pgfusepath{clip}%
\pgfsetbuttcap%
\pgfsetroundjoin%
\definecolor{currentfill}{rgb}{0.880331,0.782858,0.790579}%
\pgfsetfillcolor{currentfill}%
\pgfsetlinewidth{0.000000pt}%
\definecolor{currentstroke}{rgb}{0.000000,0.000000,0.000000}%
\pgfsetstrokecolor{currentstroke}%
\pgfsetdash{}{0pt}%
\pgfpathmoveto{\pgfqpoint{2.017448in}{0.926900in}}%
\pgfpathlineto{\pgfqpoint{2.052649in}{0.928791in}}%
\pgfpathlineto{\pgfqpoint{2.015595in}{0.962455in}}%
\pgfpathlineto{\pgfqpoint{1.980279in}{0.960682in}}%
\pgfpathclose%
\pgfusepath{fill}%
\end{pgfscope}%
\begin{pgfscope}%
\pgfpathrectangle{\pgfqpoint{0.150000in}{0.150000in}}{\pgfqpoint{2.700000in}{1.950000in}}%
\pgfusepath{clip}%
\pgfsetbuttcap%
\pgfsetroundjoin%
\definecolor{currentfill}{rgb}{0.741896,0.773683,0.818183}%
\pgfsetfillcolor{currentfill}%
\pgfsetlinewidth{0.000000pt}%
\definecolor{currentstroke}{rgb}{0.000000,0.000000,0.000000}%
\pgfsetstrokecolor{currentstroke}%
\pgfsetdash{}{0pt}%
\pgfpathmoveto{\pgfqpoint{1.351412in}{1.461190in}}%
\pgfpathlineto{\pgfqpoint{1.388982in}{1.458189in}}%
\pgfpathlineto{\pgfqpoint{1.352228in}{1.488609in}}%
\pgfpathlineto{\pgfqpoint{1.314483in}{1.494782in}}%
\pgfpathclose%
\pgfusepath{fill}%
\end{pgfscope}%
\begin{pgfscope}%
\pgfpathrectangle{\pgfqpoint{0.150000in}{0.150000in}}{\pgfqpoint{2.700000in}{1.950000in}}%
\pgfusepath{clip}%
\pgfsetbuttcap%
\pgfsetroundjoin%
\definecolor{currentfill}{rgb}{0.996890,0.997273,0.997809}%
\pgfsetfillcolor{currentfill}%
\pgfsetlinewidth{0.000000pt}%
\definecolor{currentstroke}{rgb}{0.000000,0.000000,0.000000}%
\pgfsetstrokecolor{currentstroke}%
\pgfsetdash{}{0pt}%
\pgfpathmoveto{\pgfqpoint{1.721562in}{1.158784in}}%
\pgfpathlineto{\pgfqpoint{1.757745in}{1.156916in}}%
\pgfpathlineto{\pgfqpoint{1.720797in}{1.190499in}}%
\pgfpathlineto{\pgfqpoint{1.684490in}{1.192493in}}%
\pgfpathclose%
\pgfusepath{fill}%
\end{pgfscope}%
\begin{pgfscope}%
\pgfpathrectangle{\pgfqpoint{0.150000in}{0.150000in}}{\pgfqpoint{2.700000in}{1.950000in}}%
\pgfusepath{clip}%
\pgfsetbuttcap%
\pgfsetroundjoin%
\definecolor{currentfill}{rgb}{0.865135,0.755285,0.763986}%
\pgfsetfillcolor{currentfill}%
\pgfsetlinewidth{0.000000pt}%
\definecolor{currentstroke}{rgb}{0.000000,0.000000,0.000000}%
\pgfsetstrokecolor{currentstroke}%
\pgfsetdash{}{0pt}%
\pgfpathmoveto{\pgfqpoint{2.054645in}{0.893092in}}%
\pgfpathlineto{\pgfqpoint{2.089886in}{0.897983in}}%
\pgfpathlineto{\pgfqpoint{2.052649in}{0.928791in}}%
\pgfpathlineto{\pgfqpoint{2.017448in}{0.926900in}}%
\pgfpathclose%
\pgfusepath{fill}%
\end{pgfscope}%
\begin{pgfscope}%
\pgfpathrectangle{\pgfqpoint{0.150000in}{0.150000in}}{\pgfqpoint{2.700000in}{1.950000in}}%
\pgfusepath{clip}%
\pgfsetbuttcap%
\pgfsetroundjoin%
\definecolor{currentfill}{rgb}{0.766774,0.795496,0.835708}%
\pgfsetfillcolor{currentfill}%
\pgfsetlinewidth{0.000000pt}%
\definecolor{currentstroke}{rgb}{0.000000,0.000000,0.000000}%
\pgfsetstrokecolor{currentstroke}%
\pgfsetdash{}{0pt}%
\pgfpathmoveto{\pgfqpoint{1.388369in}{1.427572in}}%
\pgfpathlineto{\pgfqpoint{1.425815in}{1.424696in}}%
\pgfpathlineto{\pgfqpoint{1.388982in}{1.458189in}}%
\pgfpathlineto{\pgfqpoint{1.351412in}{1.461190in}}%
\pgfpathclose%
\pgfusepath{fill}%
\end{pgfscope}%
\begin{pgfscope}%
\pgfpathrectangle{\pgfqpoint{0.150000in}{0.150000in}}{\pgfqpoint{2.700000in}{1.950000in}}%
\pgfusepath{clip}%
\pgfsetbuttcap%
\pgfsetroundjoin%
\definecolor{currentfill}{rgb}{0.986703,0.975873,0.976731}%
\pgfsetfillcolor{currentfill}%
\pgfsetlinewidth{0.000000pt}%
\definecolor{currentstroke}{rgb}{0.000000,0.000000,0.000000}%
\pgfsetstrokecolor{currentstroke}%
\pgfsetdash{}{0pt}%
\pgfpathmoveto{\pgfqpoint{1.758663in}{1.125049in}}%
\pgfpathlineto{\pgfqpoint{1.794793in}{1.126253in}}%
\pgfpathlineto{\pgfqpoint{1.757745in}{1.156916in}}%
\pgfpathlineto{\pgfqpoint{1.721562in}{1.158784in}}%
\pgfpathclose%
\pgfusepath{fill}%
\end{pgfscope}%
\begin{pgfscope}%
\pgfpathrectangle{\pgfqpoint{0.150000in}{0.150000in}}{\pgfqpoint{2.700000in}{1.950000in}}%
\pgfusepath{clip}%
\pgfsetbuttcap%
\pgfsetroundjoin%
\definecolor{currentfill}{rgb}{0.849939,0.727711,0.737393}%
\pgfsetfillcolor{currentfill}%
\pgfsetlinewidth{0.000000pt}%
\definecolor{currentstroke}{rgb}{0.000000,0.000000,0.000000}%
\pgfsetstrokecolor{currentstroke}%
\pgfsetdash{}{0pt}%
\pgfpathmoveto{\pgfqpoint{2.092028in}{0.862141in}}%
\pgfpathlineto{\pgfqpoint{2.127007in}{0.864258in}}%
\pgfpathlineto{\pgfqpoint{2.089886in}{0.897983in}}%
\pgfpathlineto{\pgfqpoint{2.054645in}{0.893092in}}%
\pgfpathclose%
\pgfusepath{fill}%
\end{pgfscope}%
\begin{pgfscope}%
\pgfpathrectangle{\pgfqpoint{0.150000in}{0.150000in}}{\pgfqpoint{2.700000in}{1.950000in}}%
\pgfusepath{clip}%
\pgfsetbuttcap%
\pgfsetroundjoin%
\definecolor{currentfill}{rgb}{0.791651,0.817310,0.853232}%
\pgfsetfillcolor{currentfill}%
\pgfsetlinewidth{0.000000pt}%
\definecolor{currentstroke}{rgb}{0.000000,0.000000,0.000000}%
\pgfsetstrokecolor{currentstroke}%
\pgfsetdash{}{0pt}%
\pgfpathmoveto{\pgfqpoint{1.425356in}{1.393928in}}%
\pgfpathlineto{\pgfqpoint{1.462677in}{1.391177in}}%
\pgfpathlineto{\pgfqpoint{1.425815in}{1.424696in}}%
\pgfpathlineto{\pgfqpoint{1.388369in}{1.427572in}}%
\pgfpathclose%
\pgfusepath{fill}%
\end{pgfscope}%
\begin{pgfscope}%
\pgfpathrectangle{\pgfqpoint{0.150000in}{0.150000in}}{\pgfqpoint{2.700000in}{1.950000in}}%
\pgfusepath{clip}%
\pgfsetbuttcap%
\pgfsetroundjoin%
\definecolor{currentfill}{rgb}{0.971507,0.948300,0.950138}%
\pgfsetfillcolor{currentfill}%
\pgfsetlinewidth{0.000000pt}%
\definecolor{currentstroke}{rgb}{0.000000,0.000000,0.000000}%
\pgfsetstrokecolor{currentstroke}%
\pgfsetdash{}{0pt}%
\pgfpathmoveto{\pgfqpoint{1.795793in}{1.091287in}}%
\pgfpathlineto{\pgfqpoint{1.831808in}{1.092610in}}%
\pgfpathlineto{\pgfqpoint{1.794793in}{1.126253in}}%
\pgfpathlineto{\pgfqpoint{1.758663in}{1.125049in}}%
\pgfpathclose%
\pgfusepath{fill}%
\end{pgfscope}%
\begin{pgfscope}%
\pgfpathrectangle{\pgfqpoint{0.150000in}{0.150000in}}{\pgfqpoint{2.700000in}{1.950000in}}%
\pgfusepath{clip}%
\pgfsetbuttcap%
\pgfsetroundjoin%
\definecolor{currentfill}{rgb}{0.834743,0.700138,0.710800}%
\pgfsetfillcolor{currentfill}%
\pgfsetlinewidth{0.000000pt}%
\definecolor{currentstroke}{rgb}{0.000000,0.000000,0.000000}%
\pgfsetstrokecolor{currentstroke}%
\pgfsetdash{}{0pt}%
\pgfpathmoveto{\pgfqpoint{2.129294in}{0.828272in}}%
\pgfpathlineto{\pgfqpoint{2.164157in}{0.830508in}}%
\pgfpathlineto{\pgfqpoint{2.127007in}{0.864258in}}%
\pgfpathlineto{\pgfqpoint{2.092028in}{0.862141in}}%
\pgfpathclose%
\pgfusepath{fill}%
\end{pgfscope}%
\begin{pgfscope}%
\pgfpathrectangle{\pgfqpoint{0.150000in}{0.150000in}}{\pgfqpoint{2.700000in}{1.950000in}}%
\pgfusepath{clip}%
\pgfsetbuttcap%
\pgfsetroundjoin%
\definecolor{currentfill}{rgb}{0.816529,0.839124,0.870757}%
\pgfsetfillcolor{currentfill}%
\pgfsetlinewidth{0.000000pt}%
\definecolor{currentstroke}{rgb}{0.000000,0.000000,0.000000}%
\pgfsetstrokecolor{currentstroke}%
\pgfsetdash{}{0pt}%
\pgfpathmoveto{\pgfqpoint{1.462370in}{1.360258in}}%
\pgfpathlineto{\pgfqpoint{1.499568in}{1.357632in}}%
\pgfpathlineto{\pgfqpoint{1.462677in}{1.391177in}}%
\pgfpathlineto{\pgfqpoint{1.425356in}{1.393928in}}%
\pgfpathclose%
\pgfusepath{fill}%
\end{pgfscope}%
\begin{pgfscope}%
\pgfpathrectangle{\pgfqpoint{0.150000in}{0.150000in}}{\pgfqpoint{2.700000in}{1.950000in}}%
\pgfusepath{clip}%
\pgfsetbuttcap%
\pgfsetroundjoin%
\definecolor{currentfill}{rgb}{0.956311,0.920726,0.923545}%
\pgfsetfillcolor{currentfill}%
\pgfsetlinewidth{0.000000pt}%
\definecolor{currentstroke}{rgb}{0.000000,0.000000,0.000000}%
\pgfsetstrokecolor{currentstroke}%
\pgfsetdash{}{0pt}%
\pgfpathmoveto{\pgfqpoint{1.832952in}{1.057500in}}%
\pgfpathlineto{\pgfqpoint{1.868852in}{1.058940in}}%
\pgfpathlineto{\pgfqpoint{1.831808in}{1.092610in}}%
\pgfpathlineto{\pgfqpoint{1.795793in}{1.091287in}}%
\pgfpathclose%
\pgfusepath{fill}%
\end{pgfscope}%
\begin{pgfscope}%
\pgfpathrectangle{\pgfqpoint{0.150000in}{0.150000in}}{\pgfqpoint{2.700000in}{1.950000in}}%
\pgfusepath{clip}%
\pgfsetbuttcap%
\pgfsetroundjoin%
\definecolor{currentfill}{rgb}{0.841406,0.860938,0.888281}%
\pgfsetfillcolor{currentfill}%
\pgfsetlinewidth{0.000000pt}%
\definecolor{currentstroke}{rgb}{0.000000,0.000000,0.000000}%
\pgfsetstrokecolor{currentstroke}%
\pgfsetdash{}{0pt}%
\pgfpathmoveto{\pgfqpoint{1.499414in}{1.326562in}}%
\pgfpathlineto{\pgfqpoint{1.536486in}{1.324061in}}%
\pgfpathlineto{\pgfqpoint{1.499568in}{1.357632in}}%
\pgfpathlineto{\pgfqpoint{1.462370in}{1.360258in}}%
\pgfpathclose%
\pgfusepath{fill}%
\end{pgfscope}%
\begin{pgfscope}%
\pgfpathrectangle{\pgfqpoint{0.150000in}{0.150000in}}{\pgfqpoint{2.700000in}{1.950000in}}%
\pgfusepath{clip}%
\pgfsetbuttcap%
\pgfsetroundjoin%
\definecolor{currentfill}{rgb}{0.819547,0.672564,0.684206}%
\pgfsetfillcolor{currentfill}%
\pgfsetlinewidth{0.000000pt}%
\definecolor{currentstroke}{rgb}{0.000000,0.000000,0.000000}%
\pgfsetstrokecolor{currentstroke}%
\pgfsetdash{}{0pt}%
\pgfpathmoveto{\pgfqpoint{2.166588in}{0.794376in}}%
\pgfpathlineto{\pgfqpoint{2.201522in}{0.799590in}}%
\pgfpathlineto{\pgfqpoint{2.164157in}{0.830508in}}%
\pgfpathlineto{\pgfqpoint{2.129294in}{0.828272in}}%
\pgfpathclose%
\pgfusepath{fill}%
\end{pgfscope}%
\begin{pgfscope}%
\pgfpathrectangle{\pgfqpoint{0.150000in}{0.150000in}}{\pgfqpoint{2.700000in}{1.950000in}}%
\pgfusepath{clip}%
\pgfsetbuttcap%
\pgfsetroundjoin%
\definecolor{currentfill}{rgb}{0.941115,0.893153,0.896952}%
\pgfsetfillcolor{currentfill}%
\pgfsetlinewidth{0.000000pt}%
\definecolor{currentstroke}{rgb}{0.000000,0.000000,0.000000}%
\pgfsetstrokecolor{currentstroke}%
\pgfsetdash{}{0pt}%
\pgfpathmoveto{\pgfqpoint{1.870139in}{1.023686in}}%
\pgfpathlineto{\pgfqpoint{1.905924in}{1.025245in}}%
\pgfpathlineto{\pgfqpoint{1.868852in}{1.058940in}}%
\pgfpathlineto{\pgfqpoint{1.832952in}{1.057500in}}%
\pgfpathclose%
\pgfusepath{fill}%
\end{pgfscope}%
\begin{pgfscope}%
\pgfpathrectangle{\pgfqpoint{0.150000in}{0.150000in}}{\pgfqpoint{2.700000in}{1.950000in}}%
\pgfusepath{clip}%
\pgfsetbuttcap%
\pgfsetroundjoin%
\definecolor{currentfill}{rgb}{0.866284,0.882751,0.905806}%
\pgfsetfillcolor{currentfill}%
\pgfsetlinewidth{0.000000pt}%
\definecolor{currentstroke}{rgb}{0.000000,0.000000,0.000000}%
\pgfsetstrokecolor{currentstroke}%
\pgfsetdash{}{0pt}%
\pgfpathmoveto{\pgfqpoint{1.536486in}{1.292840in}}%
\pgfpathlineto{\pgfqpoint{1.573434in}{1.290465in}}%
\pgfpathlineto{\pgfqpoint{1.536486in}{1.324061in}}%
\pgfpathlineto{\pgfqpoint{1.499414in}{1.326562in}}%
\pgfpathclose%
\pgfusepath{fill}%
\end{pgfscope}%
\begin{pgfscope}%
\pgfpathrectangle{\pgfqpoint{0.150000in}{0.150000in}}{\pgfqpoint{2.700000in}{1.950000in}}%
\pgfusepath{clip}%
\pgfsetbuttcap%
\pgfsetroundjoin%
\definecolor{currentfill}{rgb}{0.804350,0.644991,0.657613}%
\pgfsetfillcolor{currentfill}%
\pgfsetlinewidth{0.000000pt}%
\definecolor{currentstroke}{rgb}{0.000000,0.000000,0.000000}%
\pgfsetstrokecolor{currentstroke}%
\pgfsetdash{}{0pt}%
\pgfpathmoveto{\pgfqpoint{2.203912in}{0.760454in}}%
\pgfpathlineto{\pgfqpoint{2.238740in}{0.765779in}}%
\pgfpathlineto{\pgfqpoint{2.201522in}{0.799590in}}%
\pgfpathlineto{\pgfqpoint{2.166588in}{0.794376in}}%
\pgfpathclose%
\pgfusepath{fill}%
\end{pgfscope}%
\begin{pgfscope}%
\pgfpathrectangle{\pgfqpoint{0.150000in}{0.150000in}}{\pgfqpoint{2.700000in}{1.950000in}}%
\pgfusepath{clip}%
\pgfsetbuttcap%
\pgfsetroundjoin%
\definecolor{currentfill}{rgb}{0.925919,0.865579,0.870358}%
\pgfsetfillcolor{currentfill}%
\pgfsetlinewidth{0.000000pt}%
\definecolor{currentstroke}{rgb}{0.000000,0.000000,0.000000}%
\pgfsetstrokecolor{currentstroke}%
\pgfsetdash{}{0pt}%
\pgfpathmoveto{\pgfqpoint{1.907460in}{0.992772in}}%
\pgfpathlineto{\pgfqpoint{1.943139in}{0.994439in}}%
\pgfpathlineto{\pgfqpoint{1.905924in}{1.025245in}}%
\pgfpathlineto{\pgfqpoint{1.870139in}{1.023686in}}%
\pgfpathclose%
\pgfusepath{fill}%
\end{pgfscope}%
\begin{pgfscope}%
\pgfpathrectangle{\pgfqpoint{0.150000in}{0.150000in}}{\pgfqpoint{2.700000in}{1.950000in}}%
\pgfusepath{clip}%
\pgfsetbuttcap%
\pgfsetroundjoin%
\definecolor{currentfill}{rgb}{0.891161,0.904565,0.923330}%
\pgfsetfillcolor{currentfill}%
\pgfsetlinewidth{0.000000pt}%
\definecolor{currentstroke}{rgb}{0.000000,0.000000,0.000000}%
\pgfsetstrokecolor{currentstroke}%
\pgfsetdash{}{0pt}%
\pgfpathmoveto{\pgfqpoint{1.573588in}{1.259092in}}%
\pgfpathlineto{\pgfqpoint{1.610410in}{1.256842in}}%
\pgfpathlineto{\pgfqpoint{1.573434in}{1.290465in}}%
\pgfpathlineto{\pgfqpoint{1.536486in}{1.292840in}}%
\pgfpathclose%
\pgfusepath{fill}%
\end{pgfscope}%
\begin{pgfscope}%
\pgfpathrectangle{\pgfqpoint{0.150000in}{0.150000in}}{\pgfqpoint{2.700000in}{1.950000in}}%
\pgfusepath{clip}%
\pgfsetbuttcap%
\pgfsetroundjoin%
\definecolor{currentfill}{rgb}{0.629948,0.675521,0.739323}%
\pgfsetfillcolor{currentfill}%
\pgfsetlinewidth{0.000000pt}%
\definecolor{currentstroke}{rgb}{0.000000,0.000000,0.000000}%
\pgfsetstrokecolor{currentstroke}%
\pgfsetdash{}{0pt}%
\pgfpathmoveto{\pgfqpoint{1.202389in}{1.568378in}}%
\pgfpathlineto{\pgfqpoint{1.240710in}{1.561888in}}%
\pgfpathlineto{\pgfqpoint{1.203867in}{1.595402in}}%
\pgfpathlineto{\pgfqpoint{1.165411in}{1.602025in}}%
\pgfpathclose%
\pgfusepath{fill}%
\end{pgfscope}%
\begin{pgfscope}%
\pgfpathrectangle{\pgfqpoint{0.150000in}{0.150000in}}{\pgfqpoint{2.700000in}{1.950000in}}%
\pgfusepath{clip}%
\pgfsetbuttcap%
\pgfsetroundjoin%
\definecolor{currentfill}{rgb}{0.910723,0.838006,0.843765}%
\pgfsetfillcolor{currentfill}%
\pgfsetlinewidth{0.000000pt}%
\definecolor{currentstroke}{rgb}{0.000000,0.000000,0.000000}%
\pgfsetstrokecolor{currentstroke}%
\pgfsetdash{}{0pt}%
\pgfpathmoveto{\pgfqpoint{1.944716in}{0.958897in}}%
\pgfpathlineto{\pgfqpoint{1.980279in}{0.960682in}}%
\pgfpathlineto{\pgfqpoint{1.943139in}{0.994439in}}%
\pgfpathlineto{\pgfqpoint{1.907460in}{0.992772in}}%
\pgfpathclose%
\pgfusepath{fill}%
\end{pgfscope}%
\begin{pgfscope}%
\pgfpathrectangle{\pgfqpoint{0.150000in}{0.150000in}}{\pgfqpoint{2.700000in}{1.950000in}}%
\pgfusepath{clip}%
\pgfsetbuttcap%
\pgfsetroundjoin%
\definecolor{currentfill}{rgb}{0.916039,0.926379,0.940855}%
\pgfsetfillcolor{currentfill}%
\pgfsetlinewidth{0.000000pt}%
\definecolor{currentstroke}{rgb}{0.000000,0.000000,0.000000}%
\pgfsetstrokecolor{currentstroke}%
\pgfsetdash{}{0pt}%
\pgfpathmoveto{\pgfqpoint{1.610718in}{1.225317in}}%
\pgfpathlineto{\pgfqpoint{1.647446in}{1.226176in}}%
\pgfpathlineto{\pgfqpoint{1.610410in}{1.256842in}}%
\pgfpathlineto{\pgfqpoint{1.573588in}{1.259092in}}%
\pgfpathclose%
\pgfusepath{fill}%
\end{pgfscope}%
\begin{pgfscope}%
\pgfpathrectangle{\pgfqpoint{0.150000in}{0.150000in}}{\pgfqpoint{2.700000in}{1.950000in}}%
\pgfusepath{clip}%
\pgfsetbuttcap%
\pgfsetroundjoin%
\definecolor{currentfill}{rgb}{0.654825,0.697335,0.756847}%
\pgfsetfillcolor{currentfill}%
\pgfsetlinewidth{0.000000pt}%
\definecolor{currentstroke}{rgb}{0.000000,0.000000,0.000000}%
\pgfsetstrokecolor{currentstroke}%
\pgfsetdash{}{0pt}%
\pgfpathmoveto{\pgfqpoint{1.239396in}{1.534705in}}%
\pgfpathlineto{\pgfqpoint{1.277582in}{1.528348in}}%
\pgfpathlineto{\pgfqpoint{1.240710in}{1.561888in}}%
\pgfpathlineto{\pgfqpoint{1.202389in}{1.568378in}}%
\pgfpathclose%
\pgfusepath{fill}%
\end{pgfscope}%
\begin{pgfscope}%
\pgfpathrectangle{\pgfqpoint{0.150000in}{0.150000in}}{\pgfqpoint{2.700000in}{1.950000in}}%
\pgfusepath{clip}%
\pgfsetbuttcap%
\pgfsetroundjoin%
\definecolor{currentfill}{rgb}{0.895527,0.810432,0.817172}%
\pgfsetfillcolor{currentfill}%
\pgfsetlinewidth{0.000000pt}%
\definecolor{currentstroke}{rgb}{0.000000,0.000000,0.000000}%
\pgfsetstrokecolor{currentstroke}%
\pgfsetdash{}{0pt}%
\pgfpathmoveto{\pgfqpoint{1.982001in}{0.924996in}}%
\pgfpathlineto{\pgfqpoint{2.017448in}{0.926900in}}%
\pgfpathlineto{\pgfqpoint{1.980279in}{0.960682in}}%
\pgfpathlineto{\pgfqpoint{1.944716in}{0.958897in}}%
\pgfpathclose%
\pgfusepath{fill}%
\end{pgfscope}%
\begin{pgfscope}%
\pgfpathrectangle{\pgfqpoint{0.150000in}{0.150000in}}{\pgfqpoint{2.700000in}{1.950000in}}%
\pgfusepath{clip}%
\pgfsetbuttcap%
\pgfsetroundjoin%
\definecolor{currentfill}{rgb}{0.940916,0.948192,0.958379}%
\pgfsetfillcolor{currentfill}%
\pgfsetlinewidth{0.000000pt}%
\definecolor{currentstroke}{rgb}{0.000000,0.000000,0.000000}%
\pgfsetstrokecolor{currentstroke}%
\pgfsetdash{}{0pt}%
\pgfpathmoveto{\pgfqpoint{1.647876in}{1.191516in}}%
\pgfpathlineto{\pgfqpoint{1.684490in}{1.192493in}}%
\pgfpathlineto{\pgfqpoint{1.647446in}{1.226176in}}%
\pgfpathlineto{\pgfqpoint{1.610718in}{1.225317in}}%
\pgfpathclose%
\pgfusepath{fill}%
\end{pgfscope}%
\begin{pgfscope}%
\pgfpathrectangle{\pgfqpoint{0.150000in}{0.150000in}}{\pgfqpoint{2.700000in}{1.950000in}}%
\pgfusepath{clip}%
\pgfsetbuttcap%
\pgfsetroundjoin%
\definecolor{currentfill}{rgb}{0.679703,0.719148,0.774372}%
\pgfsetfillcolor{currentfill}%
\pgfsetlinewidth{0.000000pt}%
\definecolor{currentstroke}{rgb}{0.000000,0.000000,0.000000}%
\pgfsetstrokecolor{currentstroke}%
\pgfsetdash{}{0pt}%
\pgfpathmoveto{\pgfqpoint{1.276431in}{1.501005in}}%
\pgfpathlineto{\pgfqpoint{1.314483in}{1.494782in}}%
\pgfpathlineto{\pgfqpoint{1.277582in}{1.528348in}}%
\pgfpathlineto{\pgfqpoint{1.239396in}{1.534705in}}%
\pgfpathclose%
\pgfusepath{fill}%
\end{pgfscope}%
\begin{pgfscope}%
\pgfpathrectangle{\pgfqpoint{0.150000in}{0.150000in}}{\pgfqpoint{2.700000in}{1.950000in}}%
\pgfusepath{clip}%
\pgfsetbuttcap%
\pgfsetroundjoin%
\definecolor{currentfill}{rgb}{0.880331,0.782858,0.790579}%
\pgfsetfillcolor{currentfill}%
\pgfsetlinewidth{0.000000pt}%
\definecolor{currentstroke}{rgb}{0.000000,0.000000,0.000000}%
\pgfsetstrokecolor{currentstroke}%
\pgfsetdash{}{0pt}%
\pgfpathmoveto{\pgfqpoint{2.019314in}{0.891069in}}%
\pgfpathlineto{\pgfqpoint{2.054645in}{0.893092in}}%
\pgfpathlineto{\pgfqpoint{2.017448in}{0.926900in}}%
\pgfpathlineto{\pgfqpoint{1.982001in}{0.924996in}}%
\pgfpathclose%
\pgfusepath{fill}%
\end{pgfscope}%
\begin{pgfscope}%
\pgfpathrectangle{\pgfqpoint{0.150000in}{0.150000in}}{\pgfqpoint{2.700000in}{1.950000in}}%
\pgfusepath{clip}%
\pgfsetbuttcap%
\pgfsetroundjoin%
\definecolor{currentfill}{rgb}{0.965794,0.970006,0.975904}%
\pgfsetfillcolor{currentfill}%
\pgfsetlinewidth{0.000000pt}%
\definecolor{currentstroke}{rgb}{0.000000,0.000000,0.000000}%
\pgfsetstrokecolor{currentstroke}%
\pgfsetdash{}{0pt}%
\pgfpathmoveto{\pgfqpoint{1.685106in}{1.160666in}}%
\pgfpathlineto{\pgfqpoint{1.721562in}{1.158784in}}%
\pgfpathlineto{\pgfqpoint{1.684490in}{1.192493in}}%
\pgfpathlineto{\pgfqpoint{1.647876in}{1.191516in}}%
\pgfpathclose%
\pgfusepath{fill}%
\end{pgfscope}%
\begin{pgfscope}%
\pgfpathrectangle{\pgfqpoint{0.150000in}{0.150000in}}{\pgfqpoint{2.700000in}{1.950000in}}%
\pgfusepath{clip}%
\pgfsetbuttcap%
\pgfsetroundjoin%
\definecolor{currentfill}{rgb}{0.704580,0.740962,0.791896}%
\pgfsetfillcolor{currentfill}%
\pgfsetlinewidth{0.000000pt}%
\definecolor{currentstroke}{rgb}{0.000000,0.000000,0.000000}%
\pgfsetstrokecolor{currentstroke}%
\pgfsetdash{}{0pt}%
\pgfpathmoveto{\pgfqpoint{1.313495in}{1.467279in}}%
\pgfpathlineto{\pgfqpoint{1.351412in}{1.461190in}}%
\pgfpathlineto{\pgfqpoint{1.314483in}{1.494782in}}%
\pgfpathlineto{\pgfqpoint{1.276431in}{1.501005in}}%
\pgfpathclose%
\pgfusepath{fill}%
\end{pgfscope}%
\begin{pgfscope}%
\pgfpathrectangle{\pgfqpoint{0.150000in}{0.150000in}}{\pgfqpoint{2.700000in}{1.950000in}}%
\pgfusepath{clip}%
\pgfsetbuttcap%
\pgfsetroundjoin%
\definecolor{currentfill}{rgb}{0.865135,0.755285,0.763986}%
\pgfsetfillcolor{currentfill}%
\pgfsetlinewidth{0.000000pt}%
\definecolor{currentstroke}{rgb}{0.000000,0.000000,0.000000}%
\pgfsetstrokecolor{currentstroke}%
\pgfsetdash{}{0pt}%
\pgfpathmoveto{\pgfqpoint{2.056657in}{0.857115in}}%
\pgfpathlineto{\pgfqpoint{2.092028in}{0.862141in}}%
\pgfpathlineto{\pgfqpoint{2.054645in}{0.893092in}}%
\pgfpathlineto{\pgfqpoint{2.019314in}{0.891069in}}%
\pgfpathclose%
\pgfusepath{fill}%
\end{pgfscope}%
\begin{pgfscope}%
\pgfpathrectangle{\pgfqpoint{0.150000in}{0.150000in}}{\pgfqpoint{2.700000in}{1.950000in}}%
\pgfusepath{clip}%
\pgfsetbuttcap%
\pgfsetroundjoin%
\definecolor{currentfill}{rgb}{0.990671,0.991820,0.993428}%
\pgfsetfillcolor{currentfill}%
\pgfsetlinewidth{0.000000pt}%
\definecolor{currentstroke}{rgb}{0.000000,0.000000,0.000000}%
\pgfsetstrokecolor{currentstroke}%
\pgfsetdash{}{0pt}%
\pgfpathmoveto{\pgfqpoint{1.722333in}{1.126804in}}%
\pgfpathlineto{\pgfqpoint{1.758663in}{1.125049in}}%
\pgfpathlineto{\pgfqpoint{1.721562in}{1.158784in}}%
\pgfpathlineto{\pgfqpoint{1.685106in}{1.160666in}}%
\pgfpathclose%
\pgfusepath{fill}%
\end{pgfscope}%
\begin{pgfscope}%
\pgfpathrectangle{\pgfqpoint{0.150000in}{0.150000in}}{\pgfqpoint{2.700000in}{1.950000in}}%
\pgfusepath{clip}%
\pgfsetbuttcap%
\pgfsetroundjoin%
\definecolor{currentfill}{rgb}{0.729458,0.762776,0.809421}%
\pgfsetfillcolor{currentfill}%
\pgfsetlinewidth{0.000000pt}%
\definecolor{currentstroke}{rgb}{0.000000,0.000000,0.000000}%
\pgfsetstrokecolor{currentstroke}%
\pgfsetdash{}{0pt}%
\pgfpathmoveto{\pgfqpoint{1.350588in}{1.433527in}}%
\pgfpathlineto{\pgfqpoint{1.388369in}{1.427572in}}%
\pgfpathlineto{\pgfqpoint{1.351412in}{1.461190in}}%
\pgfpathlineto{\pgfqpoint{1.313495in}{1.467279in}}%
\pgfpathclose%
\pgfusepath{fill}%
\end{pgfscope}%
\begin{pgfscope}%
\pgfpathrectangle{\pgfqpoint{0.150000in}{0.150000in}}{\pgfqpoint{2.700000in}{1.950000in}}%
\pgfusepath{clip}%
\pgfsetbuttcap%
\pgfsetroundjoin%
\definecolor{currentfill}{rgb}{0.853738,0.734605,0.744041}%
\pgfsetfillcolor{currentfill}%
\pgfsetlinewidth{0.000000pt}%
\definecolor{currentstroke}{rgb}{0.000000,0.000000,0.000000}%
\pgfsetstrokecolor{currentstroke}%
\pgfsetdash{}{0pt}%
\pgfpathmoveto{\pgfqpoint{2.094029in}{0.823135in}}%
\pgfpathlineto{\pgfqpoint{2.129294in}{0.828272in}}%
\pgfpathlineto{\pgfqpoint{2.092028in}{0.862141in}}%
\pgfpathlineto{\pgfqpoint{2.056657in}{0.857115in}}%
\pgfpathclose%
\pgfusepath{fill}%
\end{pgfscope}%
\begin{pgfscope}%
\pgfpathrectangle{\pgfqpoint{0.150000in}{0.150000in}}{\pgfqpoint{2.700000in}{1.950000in}}%
\pgfusepath{clip}%
\pgfsetbuttcap%
\pgfsetroundjoin%
\definecolor{currentfill}{rgb}{0.990502,0.982767,0.983379}%
\pgfsetfillcolor{currentfill}%
\pgfsetlinewidth{0.000000pt}%
\definecolor{currentstroke}{rgb}{0.000000,0.000000,0.000000}%
\pgfsetstrokecolor{currentstroke}%
\pgfsetdash{}{0pt}%
\pgfpathmoveto{\pgfqpoint{1.759589in}{1.092916in}}%
\pgfpathlineto{\pgfqpoint{1.795793in}{1.091287in}}%
\pgfpathlineto{\pgfqpoint{1.758663in}{1.125049in}}%
\pgfpathlineto{\pgfqpoint{1.722333in}{1.126804in}}%
\pgfpathclose%
\pgfusepath{fill}%
\end{pgfscope}%
\begin{pgfscope}%
\pgfpathrectangle{\pgfqpoint{0.150000in}{0.150000in}}{\pgfqpoint{2.700000in}{1.950000in}}%
\pgfusepath{clip}%
\pgfsetbuttcap%
\pgfsetroundjoin%
\definecolor{currentfill}{rgb}{0.754335,0.784589,0.826945}%
\pgfsetfillcolor{currentfill}%
\pgfsetlinewidth{0.000000pt}%
\definecolor{currentstroke}{rgb}{0.000000,0.000000,0.000000}%
\pgfsetstrokecolor{currentstroke}%
\pgfsetdash{}{0pt}%
\pgfpathmoveto{\pgfqpoint{1.387710in}{1.399749in}}%
\pgfpathlineto{\pgfqpoint{1.425356in}{1.393928in}}%
\pgfpathlineto{\pgfqpoint{1.388369in}{1.427572in}}%
\pgfpathlineto{\pgfqpoint{1.350588in}{1.433527in}}%
\pgfpathclose%
\pgfusepath{fill}%
\end{pgfscope}%
\begin{pgfscope}%
\pgfpathrectangle{\pgfqpoint{0.150000in}{0.150000in}}{\pgfqpoint{2.700000in}{1.950000in}}%
\pgfusepath{clip}%
\pgfsetbuttcap%
\pgfsetroundjoin%
\definecolor{currentfill}{rgb}{0.838542,0.707031,0.717448}%
\pgfsetfillcolor{currentfill}%
\pgfsetlinewidth{0.000000pt}%
\definecolor{currentstroke}{rgb}{0.000000,0.000000,0.000000}%
\pgfsetstrokecolor{currentstroke}%
\pgfsetdash{}{0pt}%
\pgfpathmoveto{\pgfqpoint{2.131598in}{0.792005in}}%
\pgfpathlineto{\pgfqpoint{2.166588in}{0.794376in}}%
\pgfpathlineto{\pgfqpoint{2.129294in}{0.828272in}}%
\pgfpathlineto{\pgfqpoint{2.094029in}{0.823135in}}%
\pgfpathclose%
\pgfusepath{fill}%
\end{pgfscope}%
\begin{pgfscope}%
\pgfpathrectangle{\pgfqpoint{0.150000in}{0.150000in}}{\pgfqpoint{2.700000in}{1.950000in}}%
\pgfusepath{clip}%
\pgfsetbuttcap%
\pgfsetroundjoin%
\definecolor{currentfill}{rgb}{0.975306,0.955193,0.956786}%
\pgfsetfillcolor{currentfill}%
\pgfsetlinewidth{0.000000pt}%
\definecolor{currentstroke}{rgb}{0.000000,0.000000,0.000000}%
\pgfsetstrokecolor{currentstroke}%
\pgfsetdash{}{0pt}%
\pgfpathmoveto{\pgfqpoint{1.796874in}{1.059002in}}%
\pgfpathlineto{\pgfqpoint{1.832952in}{1.057500in}}%
\pgfpathlineto{\pgfqpoint{1.795793in}{1.091287in}}%
\pgfpathlineto{\pgfqpoint{1.759589in}{1.092916in}}%
\pgfpathclose%
\pgfusepath{fill}%
\end{pgfscope}%
\begin{pgfscope}%
\pgfpathrectangle{\pgfqpoint{0.150000in}{0.150000in}}{\pgfqpoint{2.700000in}{1.950000in}}%
\pgfusepath{clip}%
\pgfsetbuttcap%
\pgfsetroundjoin%
\definecolor{currentfill}{rgb}{0.779213,0.806403,0.844470}%
\pgfsetfillcolor{currentfill}%
\pgfsetlinewidth{0.000000pt}%
\definecolor{currentstroke}{rgb}{0.000000,0.000000,0.000000}%
\pgfsetstrokecolor{currentstroke}%
\pgfsetdash{}{0pt}%
\pgfpathmoveto{\pgfqpoint{1.424861in}{1.365944in}}%
\pgfpathlineto{\pgfqpoint{1.462370in}{1.360258in}}%
\pgfpathlineto{\pgfqpoint{1.425356in}{1.393928in}}%
\pgfpathlineto{\pgfqpoint{1.387710in}{1.399749in}}%
\pgfpathclose%
\pgfusepath{fill}%
\end{pgfscope}%
\begin{pgfscope}%
\pgfpathrectangle{\pgfqpoint{0.150000in}{0.150000in}}{\pgfqpoint{2.700000in}{1.950000in}}%
\pgfusepath{clip}%
\pgfsetbuttcap%
\pgfsetroundjoin%
\definecolor{currentfill}{rgb}{0.960110,0.927619,0.930193}%
\pgfsetfillcolor{currentfill}%
\pgfsetlinewidth{0.000000pt}%
\definecolor{currentstroke}{rgb}{0.000000,0.000000,0.000000}%
\pgfsetstrokecolor{currentstroke}%
\pgfsetdash{}{0pt}%
\pgfpathmoveto{\pgfqpoint{1.834188in}{1.025061in}}%
\pgfpathlineto{\pgfqpoint{1.870139in}{1.023686in}}%
\pgfpathlineto{\pgfqpoint{1.832952in}{1.057500in}}%
\pgfpathlineto{\pgfqpoint{1.796874in}{1.059002in}}%
\pgfpathclose%
\pgfusepath{fill}%
\end{pgfscope}%
\begin{pgfscope}%
\pgfpathrectangle{\pgfqpoint{0.150000in}{0.150000in}}{\pgfqpoint{2.700000in}{1.950000in}}%
\pgfusepath{clip}%
\pgfsetbuttcap%
\pgfsetroundjoin%
\definecolor{currentfill}{rgb}{0.823346,0.679458,0.690855}%
\pgfsetfillcolor{currentfill}%
\pgfsetlinewidth{0.000000pt}%
\definecolor{currentstroke}{rgb}{0.000000,0.000000,0.000000}%
\pgfsetstrokecolor{currentstroke}%
\pgfsetdash{}{0pt}%
\pgfpathmoveto{\pgfqpoint{2.169038in}{0.757964in}}%
\pgfpathlineto{\pgfqpoint{2.203912in}{0.760454in}}%
\pgfpathlineto{\pgfqpoint{2.166588in}{0.794376in}}%
\pgfpathlineto{\pgfqpoint{2.131598in}{0.792005in}}%
\pgfpathclose%
\pgfusepath{fill}%
\end{pgfscope}%
\begin{pgfscope}%
\pgfpathrectangle{\pgfqpoint{0.150000in}{0.150000in}}{\pgfqpoint{2.700000in}{1.950000in}}%
\pgfusepath{clip}%
\pgfsetbuttcap%
\pgfsetroundjoin%
\definecolor{currentfill}{rgb}{0.810309,0.833670,0.866376}%
\pgfsetfillcolor{currentfill}%
\pgfsetlinewidth{0.000000pt}%
\definecolor{currentstroke}{rgb}{0.000000,0.000000,0.000000}%
\pgfsetstrokecolor{currentstroke}%
\pgfsetdash{}{0pt}%
\pgfpathmoveto{\pgfqpoint{1.462061in}{1.329082in}}%
\pgfpathlineto{\pgfqpoint{1.499414in}{1.326562in}}%
\pgfpathlineto{\pgfqpoint{1.462370in}{1.360258in}}%
\pgfpathlineto{\pgfqpoint{1.424861in}{1.365944in}}%
\pgfpathclose%
\pgfusepath{fill}%
\end{pgfscope}%
\begin{pgfscope}%
\pgfpathrectangle{\pgfqpoint{0.150000in}{0.150000in}}{\pgfqpoint{2.700000in}{1.950000in}}%
\pgfusepath{clip}%
\pgfsetbuttcap%
\pgfsetroundjoin%
\definecolor{currentfill}{rgb}{0.835187,0.855484,0.883900}%
\pgfsetfillcolor{currentfill}%
\pgfsetlinewidth{0.000000pt}%
\definecolor{currentstroke}{rgb}{0.000000,0.000000,0.000000}%
\pgfsetstrokecolor{currentstroke}%
\pgfsetdash{}{0pt}%
\pgfpathmoveto{\pgfqpoint{1.499259in}{1.295233in}}%
\pgfpathlineto{\pgfqpoint{1.536486in}{1.292840in}}%
\pgfpathlineto{\pgfqpoint{1.499414in}{1.326562in}}%
\pgfpathlineto{\pgfqpoint{1.462061in}{1.329082in}}%
\pgfpathclose%
\pgfusepath{fill}%
\end{pgfscope}%
\begin{pgfscope}%
\pgfpathrectangle{\pgfqpoint{0.150000in}{0.150000in}}{\pgfqpoint{2.700000in}{1.950000in}}%
\pgfusepath{clip}%
\pgfsetbuttcap%
\pgfsetroundjoin%
\definecolor{currentfill}{rgb}{0.944914,0.900046,0.903600}%
\pgfsetfillcolor{currentfill}%
\pgfsetlinewidth{0.000000pt}%
\definecolor{currentstroke}{rgb}{0.000000,0.000000,0.000000}%
\pgfsetstrokecolor{currentstroke}%
\pgfsetdash{}{0pt}%
\pgfpathmoveto{\pgfqpoint{1.871532in}{0.991093in}}%
\pgfpathlineto{\pgfqpoint{1.907460in}{0.992772in}}%
\pgfpathlineto{\pgfqpoint{1.870139in}{1.023686in}}%
\pgfpathlineto{\pgfqpoint{1.834188in}{1.025061in}}%
\pgfpathclose%
\pgfusepath{fill}%
\end{pgfscope}%
\begin{pgfscope}%
\pgfpathrectangle{\pgfqpoint{0.150000in}{0.150000in}}{\pgfqpoint{2.700000in}{1.950000in}}%
\pgfusepath{clip}%
\pgfsetbuttcap%
\pgfsetroundjoin%
\definecolor{currentfill}{rgb}{0.860064,0.877298,0.901425}%
\pgfsetfillcolor{currentfill}%
\pgfsetlinewidth{0.000000pt}%
\definecolor{currentstroke}{rgb}{0.000000,0.000000,0.000000}%
\pgfsetstrokecolor{currentstroke}%
\pgfsetdash{}{0pt}%
\pgfpathmoveto{\pgfqpoint{1.536486in}{1.261358in}}%
\pgfpathlineto{\pgfqpoint{1.573588in}{1.259092in}}%
\pgfpathlineto{\pgfqpoint{1.536486in}{1.292840in}}%
\pgfpathlineto{\pgfqpoint{1.499259in}{1.295233in}}%
\pgfpathclose%
\pgfusepath{fill}%
\end{pgfscope}%
\begin{pgfscope}%
\pgfpathrectangle{\pgfqpoint{0.150000in}{0.150000in}}{\pgfqpoint{2.700000in}{1.950000in}}%
\pgfusepath{clip}%
\pgfsetbuttcap%
\pgfsetroundjoin%
\definecolor{currentfill}{rgb}{0.929718,0.872472,0.877007}%
\pgfsetfillcolor{currentfill}%
\pgfsetlinewidth{0.000000pt}%
\definecolor{currentstroke}{rgb}{0.000000,0.000000,0.000000}%
\pgfsetstrokecolor{currentstroke}%
\pgfsetdash{}{0pt}%
\pgfpathmoveto{\pgfqpoint{1.908904in}{0.957099in}}%
\pgfpathlineto{\pgfqpoint{1.944716in}{0.958897in}}%
\pgfpathlineto{\pgfqpoint{1.907460in}{0.992772in}}%
\pgfpathlineto{\pgfqpoint{1.871532in}{0.991093in}}%
\pgfpathclose%
\pgfusepath{fill}%
\end{pgfscope}%
\begin{pgfscope}%
\pgfpathrectangle{\pgfqpoint{0.150000in}{0.150000in}}{\pgfqpoint{2.700000in}{1.950000in}}%
\pgfusepath{clip}%
\pgfsetbuttcap%
\pgfsetroundjoin%
\definecolor{currentfill}{rgb}{0.884942,0.899112,0.918949}%
\pgfsetfillcolor{currentfill}%
\pgfsetlinewidth{0.000000pt}%
\definecolor{currentstroke}{rgb}{0.000000,0.000000,0.000000}%
\pgfsetstrokecolor{currentstroke}%
\pgfsetdash{}{0pt}%
\pgfpathmoveto{\pgfqpoint{1.573743in}{1.227456in}}%
\pgfpathlineto{\pgfqpoint{1.610718in}{1.225317in}}%
\pgfpathlineto{\pgfqpoint{1.573588in}{1.259092in}}%
\pgfpathlineto{\pgfqpoint{1.536486in}{1.261358in}}%
\pgfpathclose%
\pgfusepath{fill}%
\end{pgfscope}%
\begin{pgfscope}%
\pgfpathrectangle{\pgfqpoint{0.150000in}{0.150000in}}{\pgfqpoint{2.700000in}{1.950000in}}%
\pgfusepath{clip}%
\pgfsetbuttcap%
\pgfsetroundjoin%
\definecolor{currentfill}{rgb}{0.914522,0.844899,0.850414}%
\pgfsetfillcolor{currentfill}%
\pgfsetlinewidth{0.000000pt}%
\definecolor{currentstroke}{rgb}{0.000000,0.000000,0.000000}%
\pgfsetstrokecolor{currentstroke}%
\pgfsetdash{}{0pt}%
\pgfpathmoveto{\pgfqpoint{1.946305in}{0.923079in}}%
\pgfpathlineto{\pgfqpoint{1.982001in}{0.924996in}}%
\pgfpathlineto{\pgfqpoint{1.944716in}{0.958897in}}%
\pgfpathlineto{\pgfqpoint{1.908904in}{0.957099in}}%
\pgfpathclose%
\pgfusepath{fill}%
\end{pgfscope}%
\begin{pgfscope}%
\pgfpathrectangle{\pgfqpoint{0.150000in}{0.150000in}}{\pgfqpoint{2.700000in}{1.950000in}}%
\pgfusepath{clip}%
\pgfsetbuttcap%
\pgfsetroundjoin%
\definecolor{currentfill}{rgb}{0.586412,0.637347,0.708655}%
\pgfsetfillcolor{currentfill}%
\pgfsetlinewidth{0.000000pt}%
\definecolor{currentstroke}{rgb}{0.000000,0.000000,0.000000}%
\pgfsetstrokecolor{currentstroke}%
\pgfsetdash{}{0pt}%
\pgfpathmoveto{\pgfqpoint{1.163756in}{1.574921in}}%
\pgfpathlineto{\pgfqpoint{1.202389in}{1.568378in}}%
\pgfpathlineto{\pgfqpoint{1.165411in}{1.602025in}}%
\pgfpathlineto{\pgfqpoint{1.126643in}{1.608703in}}%
\pgfpathclose%
\pgfusepath{fill}%
\end{pgfscope}%
\begin{pgfscope}%
\pgfpathrectangle{\pgfqpoint{0.150000in}{0.150000in}}{\pgfqpoint{2.700000in}{1.950000in}}%
\pgfusepath{clip}%
\pgfsetbuttcap%
\pgfsetroundjoin%
\definecolor{currentfill}{rgb}{0.909819,0.920925,0.936474}%
\pgfsetfillcolor{currentfill}%
\pgfsetlinewidth{0.000000pt}%
\definecolor{currentstroke}{rgb}{0.000000,0.000000,0.000000}%
\pgfsetstrokecolor{currentstroke}%
\pgfsetdash{}{0pt}%
\pgfpathmoveto{\pgfqpoint{1.611028in}{1.193528in}}%
\pgfpathlineto{\pgfqpoint{1.647876in}{1.191516in}}%
\pgfpathlineto{\pgfqpoint{1.610718in}{1.225317in}}%
\pgfpathlineto{\pgfqpoint{1.573743in}{1.227456in}}%
\pgfpathclose%
\pgfusepath{fill}%
\end{pgfscope}%
\begin{pgfscope}%
\pgfpathrectangle{\pgfqpoint{0.150000in}{0.150000in}}{\pgfqpoint{2.700000in}{1.950000in}}%
\pgfusepath{clip}%
\pgfsetbuttcap%
\pgfsetroundjoin%
\definecolor{currentfill}{rgb}{0.899326,0.817325,0.823820}%
\pgfsetfillcolor{currentfill}%
\pgfsetlinewidth{0.000000pt}%
\definecolor{currentstroke}{rgb}{0.000000,0.000000,0.000000}%
\pgfsetstrokecolor{currentstroke}%
\pgfsetdash{}{0pt}%
\pgfpathmoveto{\pgfqpoint{1.983736in}{0.889032in}}%
\pgfpathlineto{\pgfqpoint{2.019314in}{0.891069in}}%
\pgfpathlineto{\pgfqpoint{1.982001in}{0.924996in}}%
\pgfpathlineto{\pgfqpoint{1.946305in}{0.923079in}}%
\pgfpathclose%
\pgfusepath{fill}%
\end{pgfscope}%
\begin{pgfscope}%
\pgfpathrectangle{\pgfqpoint{0.150000in}{0.150000in}}{\pgfqpoint{2.700000in}{1.950000in}}%
\pgfusepath{clip}%
\pgfsetbuttcap%
\pgfsetroundjoin%
\definecolor{currentfill}{rgb}{0.617509,0.664614,0.730561}%
\pgfsetfillcolor{currentfill}%
\pgfsetlinewidth{0.000000pt}%
\definecolor{currentstroke}{rgb}{0.000000,0.000000,0.000000}%
\pgfsetstrokecolor{currentstroke}%
\pgfsetdash{}{0pt}%
\pgfpathmoveto{\pgfqpoint{1.200898in}{1.541113in}}%
\pgfpathlineto{\pgfqpoint{1.239396in}{1.534705in}}%
\pgfpathlineto{\pgfqpoint{1.202389in}{1.568378in}}%
\pgfpathlineto{\pgfqpoint{1.163756in}{1.574921in}}%
\pgfpathclose%
\pgfusepath{fill}%
\end{pgfscope}%
\begin{pgfscope}%
\pgfpathrectangle{\pgfqpoint{0.150000in}{0.150000in}}{\pgfqpoint{2.700000in}{1.950000in}}%
\pgfusepath{clip}%
\pgfsetbuttcap%
\pgfsetroundjoin%
\definecolor{currentfill}{rgb}{0.934697,0.942739,0.953998}%
\pgfsetfillcolor{currentfill}%
\pgfsetlinewidth{0.000000pt}%
\definecolor{currentstroke}{rgb}{0.000000,0.000000,0.000000}%
\pgfsetstrokecolor{currentstroke}%
\pgfsetdash{}{0pt}%
\pgfpathmoveto{\pgfqpoint{1.648342in}{1.159574in}}%
\pgfpathlineto{\pgfqpoint{1.685106in}{1.160666in}}%
\pgfpathlineto{\pgfqpoint{1.647876in}{1.191516in}}%
\pgfpathlineto{\pgfqpoint{1.611028in}{1.193528in}}%
\pgfpathclose%
\pgfusepath{fill}%
\end{pgfscope}%
\begin{pgfscope}%
\pgfpathrectangle{\pgfqpoint{0.150000in}{0.150000in}}{\pgfqpoint{2.700000in}{1.950000in}}%
\pgfusepath{clip}%
\pgfsetbuttcap%
\pgfsetroundjoin%
\definecolor{currentfill}{rgb}{0.884130,0.789752,0.797227}%
\pgfsetfillcolor{currentfill}%
\pgfsetlinewidth{0.000000pt}%
\definecolor{currentstroke}{rgb}{0.000000,0.000000,0.000000}%
\pgfsetstrokecolor{currentstroke}%
\pgfsetdash{}{0pt}%
\pgfpathmoveto{\pgfqpoint{2.021195in}{0.854958in}}%
\pgfpathlineto{\pgfqpoint{2.056657in}{0.857115in}}%
\pgfpathlineto{\pgfqpoint{2.019314in}{0.891069in}}%
\pgfpathlineto{\pgfqpoint{1.983736in}{0.889032in}}%
\pgfpathclose%
\pgfusepath{fill}%
\end{pgfscope}%
\begin{pgfscope}%
\pgfpathrectangle{\pgfqpoint{0.150000in}{0.150000in}}{\pgfqpoint{2.700000in}{1.950000in}}%
\pgfusepath{clip}%
\pgfsetbuttcap%
\pgfsetroundjoin%
\definecolor{currentfill}{rgb}{0.642387,0.686428,0.748085}%
\pgfsetfillcolor{currentfill}%
\pgfsetlinewidth{0.000000pt}%
\definecolor{currentstroke}{rgb}{0.000000,0.000000,0.000000}%
\pgfsetstrokecolor{currentstroke}%
\pgfsetdash{}{0pt}%
\pgfpathmoveto{\pgfqpoint{1.238154in}{1.504191in}}%
\pgfpathlineto{\pgfqpoint{1.276431in}{1.501005in}}%
\pgfpathlineto{\pgfqpoint{1.239396in}{1.534705in}}%
\pgfpathlineto{\pgfqpoint{1.200898in}{1.541113in}}%
\pgfpathclose%
\pgfusepath{fill}%
\end{pgfscope}%
\begin{pgfscope}%
\pgfpathrectangle{\pgfqpoint{0.150000in}{0.150000in}}{\pgfqpoint{2.700000in}{1.950000in}}%
\pgfusepath{clip}%
\pgfsetbuttcap%
\pgfsetroundjoin%
\definecolor{currentfill}{rgb}{0.959574,0.964553,0.971523}%
\pgfsetfillcolor{currentfill}%
\pgfsetlinewidth{0.000000pt}%
\definecolor{currentstroke}{rgb}{0.000000,0.000000,0.000000}%
\pgfsetstrokecolor{currentstroke}%
\pgfsetdash{}{0pt}%
\pgfpathmoveto{\pgfqpoint{1.685685in}{1.125593in}}%
\pgfpathlineto{\pgfqpoint{1.722333in}{1.126804in}}%
\pgfpathlineto{\pgfqpoint{1.685106in}{1.160666in}}%
\pgfpathlineto{\pgfqpoint{1.648342in}{1.159574in}}%
\pgfpathclose%
\pgfusepath{fill}%
\end{pgfscope}%
\begin{pgfscope}%
\pgfpathrectangle{\pgfqpoint{0.150000in}{0.150000in}}{\pgfqpoint{2.700000in}{1.950000in}}%
\pgfusepath{clip}%
\pgfsetbuttcap%
\pgfsetroundjoin%
\definecolor{currentfill}{rgb}{0.868934,0.762178,0.770634}%
\pgfsetfillcolor{currentfill}%
\pgfsetlinewidth{0.000000pt}%
\definecolor{currentstroke}{rgb}{0.000000,0.000000,0.000000}%
\pgfsetstrokecolor{currentstroke}%
\pgfsetdash{}{0pt}%
\pgfpathmoveto{\pgfqpoint{2.058684in}{0.820858in}}%
\pgfpathlineto{\pgfqpoint{2.094029in}{0.823135in}}%
\pgfpathlineto{\pgfqpoint{2.056657in}{0.857115in}}%
\pgfpathlineto{\pgfqpoint{2.021195in}{0.854958in}}%
\pgfpathclose%
\pgfusepath{fill}%
\end{pgfscope}%
\begin{pgfscope}%
\pgfpathrectangle{\pgfqpoint{0.150000in}{0.150000in}}{\pgfqpoint{2.700000in}{1.950000in}}%
\pgfusepath{clip}%
\pgfsetbuttcap%
\pgfsetroundjoin%
\definecolor{currentfill}{rgb}{0.667264,0.708241,0.765610}%
\pgfsetfillcolor{currentfill}%
\pgfsetlinewidth{0.000000pt}%
\definecolor{currentstroke}{rgb}{0.000000,0.000000,0.000000}%
\pgfsetstrokecolor{currentstroke}%
\pgfsetdash{}{0pt}%
\pgfpathmoveto{\pgfqpoint{1.275344in}{1.470339in}}%
\pgfpathlineto{\pgfqpoint{1.313495in}{1.467279in}}%
\pgfpathlineto{\pgfqpoint{1.276431in}{1.501005in}}%
\pgfpathlineto{\pgfqpoint{1.238154in}{1.504191in}}%
\pgfpathclose%
\pgfusepath{fill}%
\end{pgfscope}%
\begin{pgfscope}%
\pgfpathrectangle{\pgfqpoint{0.150000in}{0.150000in}}{\pgfqpoint{2.700000in}{1.950000in}}%
\pgfusepath{clip}%
\pgfsetbuttcap%
\pgfsetroundjoin%
\definecolor{currentfill}{rgb}{0.984452,0.986366,0.989047}%
\pgfsetfillcolor{currentfill}%
\pgfsetlinewidth{0.000000pt}%
\definecolor{currentstroke}{rgb}{0.000000,0.000000,0.000000}%
\pgfsetstrokecolor{currentstroke}%
\pgfsetdash{}{0pt}%
\pgfpathmoveto{\pgfqpoint{1.723058in}{1.091585in}}%
\pgfpathlineto{\pgfqpoint{1.759589in}{1.092916in}}%
\pgfpathlineto{\pgfqpoint{1.722333in}{1.126804in}}%
\pgfpathlineto{\pgfqpoint{1.685685in}{1.125593in}}%
\pgfpathclose%
\pgfusepath{fill}%
\end{pgfscope}%
\begin{pgfscope}%
\pgfpathrectangle{\pgfqpoint{0.150000in}{0.150000in}}{\pgfqpoint{2.700000in}{1.950000in}}%
\pgfusepath{clip}%
\pgfsetbuttcap%
\pgfsetroundjoin%
\definecolor{currentfill}{rgb}{0.692142,0.730055,0.783134}%
\pgfsetfillcolor{currentfill}%
\pgfsetlinewidth{0.000000pt}%
\definecolor{currentstroke}{rgb}{0.000000,0.000000,0.000000}%
\pgfsetstrokecolor{currentstroke}%
\pgfsetdash{}{0pt}%
\pgfpathmoveto{\pgfqpoint{1.312562in}{1.436460in}}%
\pgfpathlineto{\pgfqpoint{1.350588in}{1.433527in}}%
\pgfpathlineto{\pgfqpoint{1.313495in}{1.467279in}}%
\pgfpathlineto{\pgfqpoint{1.275344in}{1.470339in}}%
\pgfpathclose%
\pgfusepath{fill}%
\end{pgfscope}%
\begin{pgfscope}%
\pgfpathrectangle{\pgfqpoint{0.150000in}{0.150000in}}{\pgfqpoint{2.700000in}{1.950000in}}%
\pgfusepath{clip}%
\pgfsetbuttcap%
\pgfsetroundjoin%
\definecolor{currentfill}{rgb}{0.853738,0.734605,0.744041}%
\pgfsetfillcolor{currentfill}%
\pgfsetlinewidth{0.000000pt}%
\definecolor{currentstroke}{rgb}{0.000000,0.000000,0.000000}%
\pgfsetstrokecolor{currentstroke}%
\pgfsetdash{}{0pt}%
\pgfpathmoveto{\pgfqpoint{2.096203in}{0.786732in}}%
\pgfpathlineto{\pgfqpoint{2.131598in}{0.792005in}}%
\pgfpathlineto{\pgfqpoint{2.094029in}{0.823135in}}%
\pgfpathlineto{\pgfqpoint{2.058684in}{0.820858in}}%
\pgfpathclose%
\pgfusepath{fill}%
\end{pgfscope}%
\begin{pgfscope}%
\pgfpathrectangle{\pgfqpoint{0.150000in}{0.150000in}}{\pgfqpoint{2.700000in}{1.950000in}}%
\pgfusepath{clip}%
\pgfsetbuttcap%
\pgfsetroundjoin%
\definecolor{currentfill}{rgb}{0.994301,0.989660,0.990028}%
\pgfsetfillcolor{currentfill}%
\pgfsetlinewidth{0.000000pt}%
\definecolor{currentstroke}{rgb}{0.000000,0.000000,0.000000}%
\pgfsetstrokecolor{currentstroke}%
\pgfsetdash{}{0pt}%
\pgfpathmoveto{\pgfqpoint{1.760459in}{1.057551in}}%
\pgfpathlineto{\pgfqpoint{1.796874in}{1.059002in}}%
\pgfpathlineto{\pgfqpoint{1.759589in}{1.092916in}}%
\pgfpathlineto{\pgfqpoint{1.723058in}{1.091585in}}%
\pgfpathclose%
\pgfusepath{fill}%
\end{pgfscope}%
\begin{pgfscope}%
\pgfpathrectangle{\pgfqpoint{0.150000in}{0.150000in}}{\pgfqpoint{2.700000in}{1.950000in}}%
\pgfusepath{clip}%
\pgfsetbuttcap%
\pgfsetroundjoin%
\definecolor{currentfill}{rgb}{0.717019,0.751869,0.800659}%
\pgfsetfillcolor{currentfill}%
\pgfsetlinewidth{0.000000pt}%
\definecolor{currentstroke}{rgb}{0.000000,0.000000,0.000000}%
\pgfsetstrokecolor{currentstroke}%
\pgfsetdash{}{0pt}%
\pgfpathmoveto{\pgfqpoint{1.349810in}{1.402555in}}%
\pgfpathlineto{\pgfqpoint{1.387710in}{1.399749in}}%
\pgfpathlineto{\pgfqpoint{1.350588in}{1.433527in}}%
\pgfpathlineto{\pgfqpoint{1.312562in}{1.436460in}}%
\pgfpathclose%
\pgfusepath{fill}%
\end{pgfscope}%
\begin{pgfscope}%
\pgfpathrectangle{\pgfqpoint{0.150000in}{0.150000in}}{\pgfqpoint{2.700000in}{1.950000in}}%
\pgfusepath{clip}%
\pgfsetbuttcap%
\pgfsetroundjoin%
\definecolor{currentfill}{rgb}{0.838542,0.707031,0.717448}%
\pgfsetfillcolor{currentfill}%
\pgfsetlinewidth{0.000000pt}%
\definecolor{currentstroke}{rgb}{0.000000,0.000000,0.000000}%
\pgfsetstrokecolor{currentstroke}%
\pgfsetdash{}{0pt}%
\pgfpathmoveto{\pgfqpoint{2.133750in}{0.752578in}}%
\pgfpathlineto{\pgfqpoint{2.169038in}{0.757964in}}%
\pgfpathlineto{\pgfqpoint{2.131598in}{0.792005in}}%
\pgfpathlineto{\pgfqpoint{2.096203in}{0.786732in}}%
\pgfpathclose%
\pgfusepath{fill}%
\end{pgfscope}%
\begin{pgfscope}%
\pgfpathrectangle{\pgfqpoint{0.150000in}{0.150000in}}{\pgfqpoint{2.700000in}{1.950000in}}%
\pgfusepath{clip}%
\pgfsetbuttcap%
\pgfsetroundjoin%
\definecolor{currentfill}{rgb}{0.979105,0.962086,0.963434}%
\pgfsetfillcolor{currentfill}%
\pgfsetlinewidth{0.000000pt}%
\definecolor{currentstroke}{rgb}{0.000000,0.000000,0.000000}%
\pgfsetstrokecolor{currentstroke}%
\pgfsetdash{}{0pt}%
\pgfpathmoveto{\pgfqpoint{1.797890in}{1.023490in}}%
\pgfpathlineto{\pgfqpoint{1.834188in}{1.025061in}}%
\pgfpathlineto{\pgfqpoint{1.796874in}{1.059002in}}%
\pgfpathlineto{\pgfqpoint{1.760459in}{1.057551in}}%
\pgfpathclose%
\pgfusepath{fill}%
\end{pgfscope}%
\begin{pgfscope}%
\pgfpathrectangle{\pgfqpoint{0.150000in}{0.150000in}}{\pgfqpoint{2.700000in}{1.950000in}}%
\pgfusepath{clip}%
\pgfsetbuttcap%
\pgfsetroundjoin%
\definecolor{currentfill}{rgb}{0.741896,0.773683,0.818183}%
\pgfsetfillcolor{currentfill}%
\pgfsetlinewidth{0.000000pt}%
\definecolor{currentstroke}{rgb}{0.000000,0.000000,0.000000}%
\pgfsetstrokecolor{currentstroke}%
\pgfsetdash{}{0pt}%
\pgfpathmoveto{\pgfqpoint{1.387087in}{1.368623in}}%
\pgfpathlineto{\pgfqpoint{1.424861in}{1.365944in}}%
\pgfpathlineto{\pgfqpoint{1.387710in}{1.399749in}}%
\pgfpathlineto{\pgfqpoint{1.349810in}{1.402555in}}%
\pgfpathclose%
\pgfusepath{fill}%
\end{pgfscope}%
\begin{pgfscope}%
\pgfpathrectangle{\pgfqpoint{0.150000in}{0.150000in}}{\pgfqpoint{2.700000in}{1.950000in}}%
\pgfusepath{clip}%
\pgfsetbuttcap%
\pgfsetroundjoin%
\definecolor{currentfill}{rgb}{0.766774,0.795496,0.835708}%
\pgfsetfillcolor{currentfill}%
\pgfsetlinewidth{0.000000pt}%
\definecolor{currentstroke}{rgb}{0.000000,0.000000,0.000000}%
\pgfsetstrokecolor{currentstroke}%
\pgfsetdash{}{0pt}%
\pgfpathmoveto{\pgfqpoint{1.424393in}{1.334665in}}%
\pgfpathlineto{\pgfqpoint{1.462061in}{1.329082in}}%
\pgfpathlineto{\pgfqpoint{1.424861in}{1.365944in}}%
\pgfpathlineto{\pgfqpoint{1.387087in}{1.368623in}}%
\pgfpathclose%
\pgfusepath{fill}%
\end{pgfscope}%
\begin{pgfscope}%
\pgfpathrectangle{\pgfqpoint{0.150000in}{0.150000in}}{\pgfqpoint{2.700000in}{1.950000in}}%
\pgfusepath{clip}%
\pgfsetbuttcap%
\pgfsetroundjoin%
\definecolor{currentfill}{rgb}{0.963909,0.934513,0.936841}%
\pgfsetfillcolor{currentfill}%
\pgfsetlinewidth{0.000000pt}%
\definecolor{currentstroke}{rgb}{0.000000,0.000000,0.000000}%
\pgfsetstrokecolor{currentstroke}%
\pgfsetdash{}{0pt}%
\pgfpathmoveto{\pgfqpoint{1.835351in}{0.989403in}}%
\pgfpathlineto{\pgfqpoint{1.871532in}{0.991093in}}%
\pgfpathlineto{\pgfqpoint{1.834188in}{1.025061in}}%
\pgfpathlineto{\pgfqpoint{1.797890in}{1.023490in}}%
\pgfpathclose%
\pgfusepath{fill}%
\end{pgfscope}%
\begin{pgfscope}%
\pgfpathrectangle{\pgfqpoint{0.150000in}{0.150000in}}{\pgfqpoint{2.700000in}{1.950000in}}%
\pgfusepath{clip}%
\pgfsetbuttcap%
\pgfsetroundjoin%
\definecolor{currentfill}{rgb}{0.797871,0.822763,0.857613}%
\pgfsetfillcolor{currentfill}%
\pgfsetlinewidth{0.000000pt}%
\definecolor{currentstroke}{rgb}{0.000000,0.000000,0.000000}%
\pgfsetstrokecolor{currentstroke}%
\pgfsetdash{}{0pt}%
\pgfpathmoveto{\pgfqpoint{1.461750in}{1.297645in}}%
\pgfpathlineto{\pgfqpoint{1.499259in}{1.295233in}}%
\pgfpathlineto{\pgfqpoint{1.462061in}{1.329082in}}%
\pgfpathlineto{\pgfqpoint{1.424393in}{1.334665in}}%
\pgfpathclose%
\pgfusepath{fill}%
\end{pgfscope}%
\begin{pgfscope}%
\pgfpathrectangle{\pgfqpoint{0.150000in}{0.150000in}}{\pgfqpoint{2.700000in}{1.950000in}}%
\pgfusepath{clip}%
\pgfsetbuttcap%
\pgfsetroundjoin%
\definecolor{currentfill}{rgb}{0.948713,0.906939,0.910248}%
\pgfsetfillcolor{currentfill}%
\pgfsetlinewidth{0.000000pt}%
\definecolor{currentstroke}{rgb}{0.000000,0.000000,0.000000}%
\pgfsetstrokecolor{currentstroke}%
\pgfsetdash{}{0pt}%
\pgfpathmoveto{\pgfqpoint{1.872840in}{0.955289in}}%
\pgfpathlineto{\pgfqpoint{1.908904in}{0.957099in}}%
\pgfpathlineto{\pgfqpoint{1.871532in}{0.991093in}}%
\pgfpathlineto{\pgfqpoint{1.835351in}{0.989403in}}%
\pgfpathclose%
\pgfusepath{fill}%
\end{pgfscope}%
\begin{pgfscope}%
\pgfpathrectangle{\pgfqpoint{0.150000in}{0.150000in}}{\pgfqpoint{2.700000in}{1.950000in}}%
\pgfusepath{clip}%
\pgfsetbuttcap%
\pgfsetroundjoin%
\definecolor{currentfill}{rgb}{0.822748,0.844577,0.875138}%
\pgfsetfillcolor{currentfill}%
\pgfsetlinewidth{0.000000pt}%
\definecolor{currentstroke}{rgb}{0.000000,0.000000,0.000000}%
\pgfsetstrokecolor{currentstroke}%
\pgfsetdash{}{0pt}%
\pgfpathmoveto{\pgfqpoint{1.499103in}{1.263642in}}%
\pgfpathlineto{\pgfqpoint{1.536486in}{1.261358in}}%
\pgfpathlineto{\pgfqpoint{1.499259in}{1.295233in}}%
\pgfpathlineto{\pgfqpoint{1.461750in}{1.297645in}}%
\pgfpathclose%
\pgfusepath{fill}%
\end{pgfscope}%
\begin{pgfscope}%
\pgfpathrectangle{\pgfqpoint{0.150000in}{0.150000in}}{\pgfqpoint{2.700000in}{1.950000in}}%
\pgfusepath{clip}%
\pgfsetbuttcap%
\pgfsetroundjoin%
\definecolor{currentfill}{rgb}{0.933517,0.879366,0.883655}%
\pgfsetfillcolor{currentfill}%
\pgfsetlinewidth{0.000000pt}%
\definecolor{currentstroke}{rgb}{0.000000,0.000000,0.000000}%
\pgfsetstrokecolor{currentstroke}%
\pgfsetdash{}{0pt}%
\pgfpathmoveto{\pgfqpoint{1.910359in}{0.921148in}}%
\pgfpathlineto{\pgfqpoint{1.946305in}{0.923079in}}%
\pgfpathlineto{\pgfqpoint{1.908904in}{0.957099in}}%
\pgfpathlineto{\pgfqpoint{1.872840in}{0.955289in}}%
\pgfpathclose%
\pgfusepath{fill}%
\end{pgfscope}%
\begin{pgfscope}%
\pgfpathrectangle{\pgfqpoint{0.150000in}{0.150000in}}{\pgfqpoint{2.700000in}{1.950000in}}%
\pgfusepath{clip}%
\pgfsetbuttcap%
\pgfsetroundjoin%
\definecolor{currentfill}{rgb}{0.847626,0.866391,0.892662}%
\pgfsetfillcolor{currentfill}%
\pgfsetlinewidth{0.000000pt}%
\definecolor{currentstroke}{rgb}{0.000000,0.000000,0.000000}%
\pgfsetstrokecolor{currentstroke}%
\pgfsetdash{}{0pt}%
\pgfpathmoveto{\pgfqpoint{1.536486in}{1.229612in}}%
\pgfpathlineto{\pgfqpoint{1.573743in}{1.227456in}}%
\pgfpathlineto{\pgfqpoint{1.536486in}{1.261358in}}%
\pgfpathlineto{\pgfqpoint{1.499103in}{1.263642in}}%
\pgfpathclose%
\pgfusepath{fill}%
\end{pgfscope}%
\begin{pgfscope}%
\pgfpathrectangle{\pgfqpoint{0.150000in}{0.150000in}}{\pgfqpoint{2.700000in}{1.950000in}}%
\pgfusepath{clip}%
\pgfsetbuttcap%
\pgfsetroundjoin%
\definecolor{currentfill}{rgb}{0.918321,0.851792,0.857062}%
\pgfsetfillcolor{currentfill}%
\pgfsetlinewidth{0.000000pt}%
\definecolor{currentstroke}{rgb}{0.000000,0.000000,0.000000}%
\pgfsetstrokecolor{currentstroke}%
\pgfsetdash{}{0pt}%
\pgfpathmoveto{\pgfqpoint{1.947907in}{0.886981in}}%
\pgfpathlineto{\pgfqpoint{1.983736in}{0.889032in}}%
\pgfpathlineto{\pgfqpoint{1.946305in}{0.923079in}}%
\pgfpathlineto{\pgfqpoint{1.910359in}{0.921148in}}%
\pgfpathclose%
\pgfusepath{fill}%
\end{pgfscope}%
\begin{pgfscope}%
\pgfpathrectangle{\pgfqpoint{0.150000in}{0.150000in}}{\pgfqpoint{2.700000in}{1.950000in}}%
\pgfusepath{clip}%
\pgfsetbuttcap%
\pgfsetroundjoin%
\definecolor{currentfill}{rgb}{0.872503,0.888205,0.910187}%
\pgfsetfillcolor{currentfill}%
\pgfsetlinewidth{0.000000pt}%
\definecolor{currentstroke}{rgb}{0.000000,0.000000,0.000000}%
\pgfsetstrokecolor{currentstroke}%
\pgfsetdash{}{0pt}%
\pgfpathmoveto{\pgfqpoint{1.573899in}{1.195556in}}%
\pgfpathlineto{\pgfqpoint{1.611028in}{1.193528in}}%
\pgfpathlineto{\pgfqpoint{1.573743in}{1.227456in}}%
\pgfpathlineto{\pgfqpoint{1.536486in}{1.229612in}}%
\pgfpathclose%
\pgfusepath{fill}%
\end{pgfscope}%
\begin{pgfscope}%
\pgfpathrectangle{\pgfqpoint{0.150000in}{0.150000in}}{\pgfqpoint{2.700000in}{1.950000in}}%
\pgfusepath{clip}%
\pgfsetbuttcap%
\pgfsetroundjoin%
\definecolor{currentfill}{rgb}{0.549096,0.604626,0.682368}%
\pgfsetfillcolor{currentfill}%
\pgfsetlinewidth{0.000000pt}%
\definecolor{currentstroke}{rgb}{0.000000,0.000000,0.000000}%
\pgfsetstrokecolor{currentstroke}%
\pgfsetdash{}{0pt}%
\pgfpathmoveto{\pgfqpoint{1.124807in}{1.581518in}}%
\pgfpathlineto{\pgfqpoint{1.163756in}{1.574921in}}%
\pgfpathlineto{\pgfqpoint{1.126643in}{1.608703in}}%
\pgfpathlineto{\pgfqpoint{1.087557in}{1.615435in}}%
\pgfpathclose%
\pgfusepath{fill}%
\end{pgfscope}%
\begin{pgfscope}%
\pgfpathrectangle{\pgfqpoint{0.150000in}{0.150000in}}{\pgfqpoint{2.700000in}{1.950000in}}%
\pgfusepath{clip}%
\pgfsetbuttcap%
\pgfsetroundjoin%
\definecolor{currentfill}{rgb}{0.903125,0.824219,0.830469}%
\pgfsetfillcolor{currentfill}%
\pgfsetlinewidth{0.000000pt}%
\definecolor{currentstroke}{rgb}{0.000000,0.000000,0.000000}%
\pgfsetstrokecolor{currentstroke}%
\pgfsetdash{}{0pt}%
\pgfpathmoveto{\pgfqpoint{1.985484in}{0.852786in}}%
\pgfpathlineto{\pgfqpoint{2.021195in}{0.854958in}}%
\pgfpathlineto{\pgfqpoint{1.983736in}{0.889032in}}%
\pgfpathlineto{\pgfqpoint{1.947907in}{0.886981in}}%
\pgfpathclose%
\pgfusepath{fill}%
\end{pgfscope}%
\begin{pgfscope}%
\pgfpathrectangle{\pgfqpoint{0.150000in}{0.150000in}}{\pgfqpoint{2.700000in}{1.950000in}}%
\pgfusepath{clip}%
\pgfsetbuttcap%
\pgfsetroundjoin%
\definecolor{currentfill}{rgb}{0.897381,0.910018,0.927711}%
\pgfsetfillcolor{currentfill}%
\pgfsetlinewidth{0.000000pt}%
\definecolor{currentstroke}{rgb}{0.000000,0.000000,0.000000}%
\pgfsetstrokecolor{currentstroke}%
\pgfsetdash{}{0pt}%
\pgfpathmoveto{\pgfqpoint{1.611340in}{1.161473in}}%
\pgfpathlineto{\pgfqpoint{1.648342in}{1.159574in}}%
\pgfpathlineto{\pgfqpoint{1.611028in}{1.193528in}}%
\pgfpathlineto{\pgfqpoint{1.573899in}{1.195556in}}%
\pgfpathclose%
\pgfusepath{fill}%
\end{pgfscope}%
\begin{pgfscope}%
\pgfpathrectangle{\pgfqpoint{0.150000in}{0.150000in}}{\pgfqpoint{2.700000in}{1.950000in}}%
\pgfusepath{clip}%
\pgfsetbuttcap%
\pgfsetroundjoin%
\definecolor{currentfill}{rgb}{0.573974,0.626440,0.699893}%
\pgfsetfillcolor{currentfill}%
\pgfsetlinewidth{0.000000pt}%
\definecolor{currentstroke}{rgb}{0.000000,0.000000,0.000000}%
\pgfsetstrokecolor{currentstroke}%
\pgfsetdash{}{0pt}%
\pgfpathmoveto{\pgfqpoint{1.162192in}{1.544464in}}%
\pgfpathlineto{\pgfqpoint{1.200898in}{1.541113in}}%
\pgfpathlineto{\pgfqpoint{1.163756in}{1.574921in}}%
\pgfpathlineto{\pgfqpoint{1.124807in}{1.581518in}}%
\pgfpathclose%
\pgfusepath{fill}%
\end{pgfscope}%
\begin{pgfscope}%
\pgfpathrectangle{\pgfqpoint{0.150000in}{0.150000in}}{\pgfqpoint{2.700000in}{1.950000in}}%
\pgfusepath{clip}%
\pgfsetbuttcap%
\pgfsetroundjoin%
\definecolor{currentfill}{rgb}{0.887929,0.796645,0.803876}%
\pgfsetfillcolor{currentfill}%
\pgfsetlinewidth{0.000000pt}%
\definecolor{currentstroke}{rgb}{0.000000,0.000000,0.000000}%
\pgfsetstrokecolor{currentstroke}%
\pgfsetdash{}{0pt}%
\pgfpathmoveto{\pgfqpoint{2.022953in}{0.815663in}}%
\pgfpathlineto{\pgfqpoint{2.058684in}{0.820858in}}%
\pgfpathlineto{\pgfqpoint{2.021195in}{0.854958in}}%
\pgfpathlineto{\pgfqpoint{1.985484in}{0.852786in}}%
\pgfpathclose%
\pgfusepath{fill}%
\end{pgfscope}%
\begin{pgfscope}%
\pgfpathrectangle{\pgfqpoint{0.150000in}{0.150000in}}{\pgfqpoint{2.700000in}{1.950000in}}%
\pgfusepath{clip}%
\pgfsetbuttcap%
\pgfsetroundjoin%
\definecolor{currentfill}{rgb}{0.605070,0.653707,0.721798}%
\pgfsetfillcolor{currentfill}%
\pgfsetlinewidth{0.000000pt}%
\definecolor{currentstroke}{rgb}{0.000000,0.000000,0.000000}%
\pgfsetstrokecolor{currentstroke}%
\pgfsetdash{}{0pt}%
\pgfpathmoveto{\pgfqpoint{1.199490in}{1.510502in}}%
\pgfpathlineto{\pgfqpoint{1.238154in}{1.504191in}}%
\pgfpathlineto{\pgfqpoint{1.200898in}{1.541113in}}%
\pgfpathlineto{\pgfqpoint{1.162192in}{1.544464in}}%
\pgfpathclose%
\pgfusepath{fill}%
\end{pgfscope}%
\begin{pgfscope}%
\pgfpathrectangle{\pgfqpoint{0.150000in}{0.150000in}}{\pgfqpoint{2.700000in}{1.950000in}}%
\pgfusepath{clip}%
\pgfsetbuttcap%
\pgfsetroundjoin%
\definecolor{currentfill}{rgb}{0.922258,0.931832,0.945236}%
\pgfsetfillcolor{currentfill}%
\pgfsetlinewidth{0.000000pt}%
\definecolor{currentstroke}{rgb}{0.000000,0.000000,0.000000}%
\pgfsetstrokecolor{currentstroke}%
\pgfsetdash{}{0pt}%
\pgfpathmoveto{\pgfqpoint{1.648811in}{1.127363in}}%
\pgfpathlineto{\pgfqpoint{1.685685in}{1.125593in}}%
\pgfpathlineto{\pgfqpoint{1.648342in}{1.159574in}}%
\pgfpathlineto{\pgfqpoint{1.611340in}{1.161473in}}%
\pgfpathclose%
\pgfusepath{fill}%
\end{pgfscope}%
\begin{pgfscope}%
\pgfpathrectangle{\pgfqpoint{0.150000in}{0.150000in}}{\pgfqpoint{2.700000in}{1.950000in}}%
\pgfusepath{clip}%
\pgfsetbuttcap%
\pgfsetroundjoin%
\definecolor{currentfill}{rgb}{0.868934,0.762178,0.770634}%
\pgfsetfillcolor{currentfill}%
\pgfsetlinewidth{0.000000pt}%
\definecolor{currentstroke}{rgb}{0.000000,0.000000,0.000000}%
\pgfsetstrokecolor{currentstroke}%
\pgfsetdash{}{0pt}%
\pgfpathmoveto{\pgfqpoint{2.060579in}{0.781424in}}%
\pgfpathlineto{\pgfqpoint{2.096203in}{0.786732in}}%
\pgfpathlineto{\pgfqpoint{2.058684in}{0.820858in}}%
\pgfpathlineto{\pgfqpoint{2.022953in}{0.815663in}}%
\pgfpathclose%
\pgfusepath{fill}%
\end{pgfscope}%
\begin{pgfscope}%
\pgfpathrectangle{\pgfqpoint{0.150000in}{0.150000in}}{\pgfqpoint{2.700000in}{1.950000in}}%
\pgfusepath{clip}%
\pgfsetbuttcap%
\pgfsetroundjoin%
\definecolor{currentfill}{rgb}{0.629948,0.675521,0.739323}%
\pgfsetfillcolor{currentfill}%
\pgfsetlinewidth{0.000000pt}%
\definecolor{currentstroke}{rgb}{0.000000,0.000000,0.000000}%
\pgfsetstrokecolor{currentstroke}%
\pgfsetdash{}{0pt}%
\pgfpathmoveto{\pgfqpoint{1.236816in}{1.476514in}}%
\pgfpathlineto{\pgfqpoint{1.275344in}{1.470339in}}%
\pgfpathlineto{\pgfqpoint{1.238154in}{1.504191in}}%
\pgfpathlineto{\pgfqpoint{1.199490in}{1.510502in}}%
\pgfpathclose%
\pgfusepath{fill}%
\end{pgfscope}%
\begin{pgfscope}%
\pgfpathrectangle{\pgfqpoint{0.150000in}{0.150000in}}{\pgfqpoint{2.700000in}{1.950000in}}%
\pgfusepath{clip}%
\pgfsetbuttcap%
\pgfsetroundjoin%
\definecolor{currentfill}{rgb}{0.953355,0.959099,0.967142}%
\pgfsetfillcolor{currentfill}%
\pgfsetlinewidth{0.000000pt}%
\definecolor{currentstroke}{rgb}{0.000000,0.000000,0.000000}%
\pgfsetstrokecolor{currentstroke}%
\pgfsetdash{}{0pt}%
\pgfpathmoveto{\pgfqpoint{1.686269in}{1.090245in}}%
\pgfpathlineto{\pgfqpoint{1.723058in}{1.091585in}}%
\pgfpathlineto{\pgfqpoint{1.685685in}{1.125593in}}%
\pgfpathlineto{\pgfqpoint{1.648811in}{1.127363in}}%
\pgfpathclose%
\pgfusepath{fill}%
\end{pgfscope}%
\begin{pgfscope}%
\pgfpathrectangle{\pgfqpoint{0.150000in}{0.150000in}}{\pgfqpoint{2.700000in}{1.950000in}}%
\pgfusepath{clip}%
\pgfsetbuttcap%
\pgfsetroundjoin%
\definecolor{currentfill}{rgb}{0.853738,0.734605,0.744041}%
\pgfsetfillcolor{currentfill}%
\pgfsetlinewidth{0.000000pt}%
\definecolor{currentstroke}{rgb}{0.000000,0.000000,0.000000}%
\pgfsetstrokecolor{currentstroke}%
\pgfsetdash{}{0pt}%
\pgfpathmoveto{\pgfqpoint{2.098234in}{0.747158in}}%
\pgfpathlineto{\pgfqpoint{2.133750in}{0.752578in}}%
\pgfpathlineto{\pgfqpoint{2.096203in}{0.786732in}}%
\pgfpathlineto{\pgfqpoint{2.060579in}{0.781424in}}%
\pgfpathclose%
\pgfusepath{fill}%
\end{pgfscope}%
\begin{pgfscope}%
\pgfpathrectangle{\pgfqpoint{0.150000in}{0.150000in}}{\pgfqpoint{2.700000in}{1.950000in}}%
\pgfusepath{clip}%
\pgfsetbuttcap%
\pgfsetroundjoin%
\definecolor{currentfill}{rgb}{0.978232,0.980913,0.984666}%
\pgfsetfillcolor{currentfill}%
\pgfsetlinewidth{0.000000pt}%
\definecolor{currentstroke}{rgb}{0.000000,0.000000,0.000000}%
\pgfsetstrokecolor{currentstroke}%
\pgfsetdash{}{0pt}%
\pgfpathmoveto{\pgfqpoint{1.723788in}{1.056090in}}%
\pgfpathlineto{\pgfqpoint{1.760459in}{1.057551in}}%
\pgfpathlineto{\pgfqpoint{1.723058in}{1.091585in}}%
\pgfpathlineto{\pgfqpoint{1.686269in}{1.090245in}}%
\pgfpathclose%
\pgfusepath{fill}%
\end{pgfscope}%
\begin{pgfscope}%
\pgfpathrectangle{\pgfqpoint{0.150000in}{0.150000in}}{\pgfqpoint{2.700000in}{1.950000in}}%
\pgfusepath{clip}%
\pgfsetbuttcap%
\pgfsetroundjoin%
\definecolor{currentfill}{rgb}{0.654825,0.697335,0.756847}%
\pgfsetfillcolor{currentfill}%
\pgfsetlinewidth{0.000000pt}%
\definecolor{currentstroke}{rgb}{0.000000,0.000000,0.000000}%
\pgfsetstrokecolor{currentstroke}%
\pgfsetdash{}{0pt}%
\pgfpathmoveto{\pgfqpoint{1.274172in}{1.442498in}}%
\pgfpathlineto{\pgfqpoint{1.312562in}{1.436460in}}%
\pgfpathlineto{\pgfqpoint{1.275344in}{1.470339in}}%
\pgfpathlineto{\pgfqpoint{1.236816in}{1.476514in}}%
\pgfpathclose%
\pgfusepath{fill}%
\end{pgfscope}%
\begin{pgfscope}%
\pgfpathrectangle{\pgfqpoint{0.150000in}{0.150000in}}{\pgfqpoint{2.700000in}{1.950000in}}%
\pgfusepath{clip}%
\pgfsetbuttcap%
\pgfsetroundjoin%
\definecolor{currentfill}{rgb}{0.679703,0.719148,0.774372}%
\pgfsetfillcolor{currentfill}%
\pgfsetlinewidth{0.000000pt}%
\definecolor{currentstroke}{rgb}{0.000000,0.000000,0.000000}%
\pgfsetstrokecolor{currentstroke}%
\pgfsetdash{}{0pt}%
\pgfpathmoveto{\pgfqpoint{1.311622in}{1.405382in}}%
\pgfpathlineto{\pgfqpoint{1.349810in}{1.402555in}}%
\pgfpathlineto{\pgfqpoint{1.312562in}{1.436460in}}%
\pgfpathlineto{\pgfqpoint{1.274172in}{1.442498in}}%
\pgfpathclose%
\pgfusepath{fill}%
\end{pgfscope}%
\begin{pgfscope}%
\pgfpathrectangle{\pgfqpoint{0.150000in}{0.150000in}}{\pgfqpoint{2.700000in}{1.950000in}}%
\pgfusepath{clip}%
\pgfsetbuttcap%
\pgfsetroundjoin%
\definecolor{currentfill}{rgb}{0.998100,0.996553,0.996676}%
\pgfsetfillcolor{currentfill}%
\pgfsetlinewidth{0.000000pt}%
\definecolor{currentstroke}{rgb}{0.000000,0.000000,0.000000}%
\pgfsetstrokecolor{currentstroke}%
\pgfsetdash{}{0pt}%
\pgfpathmoveto{\pgfqpoint{1.761337in}{1.021909in}}%
\pgfpathlineto{\pgfqpoint{1.797890in}{1.023490in}}%
\pgfpathlineto{\pgfqpoint{1.760459in}{1.057551in}}%
\pgfpathlineto{\pgfqpoint{1.723788in}{1.056090in}}%
\pgfpathclose%
\pgfusepath{fill}%
\end{pgfscope}%
\begin{pgfscope}%
\pgfpathrectangle{\pgfqpoint{0.150000in}{0.150000in}}{\pgfqpoint{2.700000in}{1.950000in}}%
\pgfusepath{clip}%
\pgfsetbuttcap%
\pgfsetroundjoin%
\definecolor{currentfill}{rgb}{0.710800,0.746415,0.796278}%
\pgfsetfillcolor{currentfill}%
\pgfsetlinewidth{0.000000pt}%
\definecolor{currentstroke}{rgb}{0.000000,0.000000,0.000000}%
\pgfsetstrokecolor{currentstroke}%
\pgfsetdash{}{0pt}%
\pgfpathmoveto{\pgfqpoint{1.349026in}{1.371322in}}%
\pgfpathlineto{\pgfqpoint{1.387087in}{1.368623in}}%
\pgfpathlineto{\pgfqpoint{1.349810in}{1.402555in}}%
\pgfpathlineto{\pgfqpoint{1.311622in}{1.405382in}}%
\pgfpathclose%
\pgfusepath{fill}%
\end{pgfscope}%
\begin{pgfscope}%
\pgfpathrectangle{\pgfqpoint{0.150000in}{0.150000in}}{\pgfqpoint{2.700000in}{1.950000in}}%
\pgfusepath{clip}%
\pgfsetbuttcap%
\pgfsetroundjoin%
\definecolor{currentfill}{rgb}{0.982904,0.968980,0.970083}%
\pgfsetfillcolor{currentfill}%
\pgfsetlinewidth{0.000000pt}%
\definecolor{currentstroke}{rgb}{0.000000,0.000000,0.000000}%
\pgfsetstrokecolor{currentstroke}%
\pgfsetdash{}{0pt}%
\pgfpathmoveto{\pgfqpoint{1.798914in}{0.987700in}}%
\pgfpathlineto{\pgfqpoint{1.835351in}{0.989403in}}%
\pgfpathlineto{\pgfqpoint{1.797890in}{1.023490in}}%
\pgfpathlineto{\pgfqpoint{1.761337in}{1.021909in}}%
\pgfpathclose%
\pgfusepath{fill}%
\end{pgfscope}%
\begin{pgfscope}%
\pgfpathrectangle{\pgfqpoint{0.150000in}{0.150000in}}{\pgfqpoint{2.700000in}{1.950000in}}%
\pgfusepath{clip}%
\pgfsetbuttcap%
\pgfsetroundjoin%
\definecolor{currentfill}{rgb}{0.735677,0.768229,0.813802}%
\pgfsetfillcolor{currentfill}%
\pgfsetlinewidth{0.000000pt}%
\definecolor{currentstroke}{rgb}{0.000000,0.000000,0.000000}%
\pgfsetstrokecolor{currentstroke}%
\pgfsetdash{}{0pt}%
\pgfpathmoveto{\pgfqpoint{1.386459in}{1.337235in}}%
\pgfpathlineto{\pgfqpoint{1.424393in}{1.334665in}}%
\pgfpathlineto{\pgfqpoint{1.387087in}{1.368623in}}%
\pgfpathlineto{\pgfqpoint{1.349026in}{1.371322in}}%
\pgfpathclose%
\pgfusepath{fill}%
\end{pgfscope}%
\begin{pgfscope}%
\pgfpathrectangle{\pgfqpoint{0.150000in}{0.150000in}}{\pgfqpoint{2.700000in}{1.950000in}}%
\pgfusepath{clip}%
\pgfsetbuttcap%
\pgfsetroundjoin%
\definecolor{currentfill}{rgb}{0.967708,0.941406,0.943490}%
\pgfsetfillcolor{currentfill}%
\pgfsetlinewidth{0.000000pt}%
\definecolor{currentstroke}{rgb}{0.000000,0.000000,0.000000}%
\pgfsetstrokecolor{currentstroke}%
\pgfsetdash{}{0pt}%
\pgfpathmoveto{\pgfqpoint{1.836522in}{0.953465in}}%
\pgfpathlineto{\pgfqpoint{1.872840in}{0.955289in}}%
\pgfpathlineto{\pgfqpoint{1.835351in}{0.989403in}}%
\pgfpathlineto{\pgfqpoint{1.798914in}{0.987700in}}%
\pgfpathclose%
\pgfusepath{fill}%
\end{pgfscope}%
\begin{pgfscope}%
\pgfpathrectangle{\pgfqpoint{0.150000in}{0.150000in}}{\pgfqpoint{2.700000in}{1.950000in}}%
\pgfusepath{clip}%
\pgfsetbuttcap%
\pgfsetroundjoin%
\definecolor{currentfill}{rgb}{0.760555,0.790043,0.831327}%
\pgfsetfillcolor{currentfill}%
\pgfsetlinewidth{0.000000pt}%
\definecolor{currentstroke}{rgb}{0.000000,0.000000,0.000000}%
\pgfsetstrokecolor{currentstroke}%
\pgfsetdash{}{0pt}%
\pgfpathmoveto{\pgfqpoint{1.423954in}{1.300075in}}%
\pgfpathlineto{\pgfqpoint{1.461750in}{1.297645in}}%
\pgfpathlineto{\pgfqpoint{1.424393in}{1.334665in}}%
\pgfpathlineto{\pgfqpoint{1.386459in}{1.337235in}}%
\pgfpathclose%
\pgfusepath{fill}%
\end{pgfscope}%
\begin{pgfscope}%
\pgfpathrectangle{\pgfqpoint{0.150000in}{0.150000in}}{\pgfqpoint{2.700000in}{1.950000in}}%
\pgfusepath{clip}%
\pgfsetbuttcap%
\pgfsetroundjoin%
\definecolor{currentfill}{rgb}{0.952512,0.913833,0.916896}%
\pgfsetfillcolor{currentfill}%
\pgfsetlinewidth{0.000000pt}%
\definecolor{currentstroke}{rgb}{0.000000,0.000000,0.000000}%
\pgfsetstrokecolor{currentstroke}%
\pgfsetdash{}{0pt}%
\pgfpathmoveto{\pgfqpoint{1.874159in}{0.919204in}}%
\pgfpathlineto{\pgfqpoint{1.910359in}{0.921148in}}%
\pgfpathlineto{\pgfqpoint{1.872840in}{0.955289in}}%
\pgfpathlineto{\pgfqpoint{1.836522in}{0.953465in}}%
\pgfpathclose%
\pgfusepath{fill}%
\end{pgfscope}%
\begin{pgfscope}%
\pgfpathrectangle{\pgfqpoint{0.150000in}{0.150000in}}{\pgfqpoint{2.700000in}{1.950000in}}%
\pgfusepath{clip}%
\pgfsetbuttcap%
\pgfsetroundjoin%
\definecolor{currentfill}{rgb}{0.785432,0.811857,0.848851}%
\pgfsetfillcolor{currentfill}%
\pgfsetlinewidth{0.000000pt}%
\definecolor{currentstroke}{rgb}{0.000000,0.000000,0.000000}%
\pgfsetstrokecolor{currentstroke}%
\pgfsetdash{}{0pt}%
\pgfpathmoveto{\pgfqpoint{1.461435in}{1.265943in}}%
\pgfpathlineto{\pgfqpoint{1.499103in}{1.263642in}}%
\pgfpathlineto{\pgfqpoint{1.461750in}{1.297645in}}%
\pgfpathlineto{\pgfqpoint{1.423954in}{1.300075in}}%
\pgfpathclose%
\pgfusepath{fill}%
\end{pgfscope}%
\begin{pgfscope}%
\pgfpathrectangle{\pgfqpoint{0.150000in}{0.150000in}}{\pgfqpoint{2.700000in}{1.950000in}}%
\pgfusepath{clip}%
\pgfsetbuttcap%
\pgfsetroundjoin%
\definecolor{currentfill}{rgb}{0.937316,0.886259,0.890303}%
\pgfsetfillcolor{currentfill}%
\pgfsetlinewidth{0.000000pt}%
\definecolor{currentstroke}{rgb}{0.000000,0.000000,0.000000}%
\pgfsetstrokecolor{currentstroke}%
\pgfsetdash{}{0pt}%
\pgfpathmoveto{\pgfqpoint{1.911718in}{0.881986in}}%
\pgfpathlineto{\pgfqpoint{1.947907in}{0.886981in}}%
\pgfpathlineto{\pgfqpoint{1.910359in}{0.921148in}}%
\pgfpathlineto{\pgfqpoint{1.874159in}{0.919204in}}%
\pgfpathclose%
\pgfusepath{fill}%
\end{pgfscope}%
\begin{pgfscope}%
\pgfpathrectangle{\pgfqpoint{0.150000in}{0.150000in}}{\pgfqpoint{2.700000in}{1.950000in}}%
\pgfusepath{clip}%
\pgfsetbuttcap%
\pgfsetroundjoin%
\definecolor{currentfill}{rgb}{0.816529,0.839124,0.870757}%
\pgfsetfillcolor{currentfill}%
\pgfsetlinewidth{0.000000pt}%
\definecolor{currentstroke}{rgb}{0.000000,0.000000,0.000000}%
\pgfsetstrokecolor{currentstroke}%
\pgfsetdash{}{0pt}%
\pgfpathmoveto{\pgfqpoint{1.498946in}{1.231784in}}%
\pgfpathlineto{\pgfqpoint{1.536486in}{1.229612in}}%
\pgfpathlineto{\pgfqpoint{1.499103in}{1.263642in}}%
\pgfpathlineto{\pgfqpoint{1.461435in}{1.265943in}}%
\pgfpathclose%
\pgfusepath{fill}%
\end{pgfscope}%
\begin{pgfscope}%
\pgfpathrectangle{\pgfqpoint{0.150000in}{0.150000in}}{\pgfqpoint{2.700000in}{1.950000in}}%
\pgfusepath{clip}%
\pgfsetbuttcap%
\pgfsetroundjoin%
\definecolor{currentfill}{rgb}{0.918321,0.851792,0.857062}%
\pgfsetfillcolor{currentfill}%
\pgfsetlinewidth{0.000000pt}%
\definecolor{currentstroke}{rgb}{0.000000,0.000000,0.000000}%
\pgfsetstrokecolor{currentstroke}%
\pgfsetdash{}{0pt}%
\pgfpathmoveto{\pgfqpoint{1.949403in}{0.847679in}}%
\pgfpathlineto{\pgfqpoint{1.985484in}{0.852786in}}%
\pgfpathlineto{\pgfqpoint{1.947907in}{0.886981in}}%
\pgfpathlineto{\pgfqpoint{1.911718in}{0.881986in}}%
\pgfpathclose%
\pgfusepath{fill}%
\end{pgfscope}%
\begin{pgfscope}%
\pgfpathrectangle{\pgfqpoint{0.150000in}{0.150000in}}{\pgfqpoint{2.700000in}{1.950000in}}%
\pgfusepath{clip}%
\pgfsetbuttcap%
\pgfsetroundjoin%
\definecolor{currentfill}{rgb}{0.841406,0.860938,0.888281}%
\pgfsetfillcolor{currentfill}%
\pgfsetlinewidth{0.000000pt}%
\definecolor{currentstroke}{rgb}{0.000000,0.000000,0.000000}%
\pgfsetstrokecolor{currentstroke}%
\pgfsetdash{}{0pt}%
\pgfpathmoveto{\pgfqpoint{1.536486in}{1.197599in}}%
\pgfpathlineto{\pgfqpoint{1.573899in}{1.195556in}}%
\pgfpathlineto{\pgfqpoint{1.536486in}{1.229612in}}%
\pgfpathlineto{\pgfqpoint{1.498946in}{1.231784in}}%
\pgfpathclose%
\pgfusepath{fill}%
\end{pgfscope}%
\begin{pgfscope}%
\pgfpathrectangle{\pgfqpoint{0.150000in}{0.150000in}}{\pgfqpoint{2.700000in}{1.950000in}}%
\pgfusepath{clip}%
\pgfsetbuttcap%
\pgfsetroundjoin%
\definecolor{currentfill}{rgb}{0.903125,0.824219,0.830469}%
\pgfsetfillcolor{currentfill}%
\pgfsetlinewidth{0.000000pt}%
\definecolor{currentstroke}{rgb}{0.000000,0.000000,0.000000}%
\pgfsetstrokecolor{currentstroke}%
\pgfsetdash{}{0pt}%
\pgfpathmoveto{\pgfqpoint{1.987118in}{0.813345in}}%
\pgfpathlineto{\pgfqpoint{2.022953in}{0.815663in}}%
\pgfpathlineto{\pgfqpoint{1.985484in}{0.852786in}}%
\pgfpathlineto{\pgfqpoint{1.949403in}{0.847679in}}%
\pgfpathclose%
\pgfusepath{fill}%
\end{pgfscope}%
\begin{pgfscope}%
\pgfpathrectangle{\pgfqpoint{0.150000in}{0.150000in}}{\pgfqpoint{2.700000in}{1.950000in}}%
\pgfusepath{clip}%
\pgfsetbuttcap%
\pgfsetroundjoin%
\definecolor{currentfill}{rgb}{0.866284,0.882751,0.905806}%
\pgfsetfillcolor{currentfill}%
\pgfsetlinewidth{0.000000pt}%
\definecolor{currentstroke}{rgb}{0.000000,0.000000,0.000000}%
\pgfsetstrokecolor{currentstroke}%
\pgfsetdash{}{0pt}%
\pgfpathmoveto{\pgfqpoint{1.574046in}{1.160375in}}%
\pgfpathlineto{\pgfqpoint{1.611340in}{1.161473in}}%
\pgfpathlineto{\pgfqpoint{1.573899in}{1.195556in}}%
\pgfpathlineto{\pgfqpoint{1.536486in}{1.197599in}}%
\pgfpathclose%
\pgfusepath{fill}%
\end{pgfscope}%
\begin{pgfscope}%
\pgfpathrectangle{\pgfqpoint{0.150000in}{0.150000in}}{\pgfqpoint{2.700000in}{1.950000in}}%
\pgfusepath{clip}%
\pgfsetbuttcap%
\pgfsetroundjoin%
\definecolor{currentfill}{rgb}{0.887929,0.796645,0.803876}%
\pgfsetfillcolor{currentfill}%
\pgfsetlinewidth{0.000000pt}%
\definecolor{currentstroke}{rgb}{0.000000,0.000000,0.000000}%
\pgfsetstrokecolor{currentstroke}%
\pgfsetdash{}{0pt}%
\pgfpathmoveto{\pgfqpoint{2.024863in}{0.778984in}}%
\pgfpathlineto{\pgfqpoint{2.060579in}{0.781424in}}%
\pgfpathlineto{\pgfqpoint{2.022953in}{0.815663in}}%
\pgfpathlineto{\pgfqpoint{1.987118in}{0.813345in}}%
\pgfpathclose%
\pgfusepath{fill}%
\end{pgfscope}%
\begin{pgfscope}%
\pgfpathrectangle{\pgfqpoint{0.150000in}{0.150000in}}{\pgfqpoint{2.700000in}{1.950000in}}%
\pgfusepath{clip}%
\pgfsetbuttcap%
\pgfsetroundjoin%
\definecolor{currentfill}{rgb}{0.511780,0.571906,0.656081}%
\pgfsetfillcolor{currentfill}%
\pgfsetlinewidth{0.000000pt}%
\definecolor{currentstroke}{rgb}{0.000000,0.000000,0.000000}%
\pgfsetstrokecolor{currentstroke}%
\pgfsetdash{}{0pt}%
\pgfpathmoveto{\pgfqpoint{1.085538in}{1.588169in}}%
\pgfpathlineto{\pgfqpoint{1.124807in}{1.581518in}}%
\pgfpathlineto{\pgfqpoint{1.087557in}{1.615435in}}%
\pgfpathlineto{\pgfqpoint{1.048151in}{1.622222in}}%
\pgfpathclose%
\pgfusepath{fill}%
\end{pgfscope}%
\begin{pgfscope}%
\pgfpathrectangle{\pgfqpoint{0.150000in}{0.150000in}}{\pgfqpoint{2.700000in}{1.950000in}}%
\pgfusepath{clip}%
\pgfsetbuttcap%
\pgfsetroundjoin%
\definecolor{currentfill}{rgb}{0.891161,0.904565,0.923330}%
\pgfsetfillcolor{currentfill}%
\pgfsetlinewidth{0.000000pt}%
\definecolor{currentstroke}{rgb}{0.000000,0.000000,0.000000}%
\pgfsetstrokecolor{currentstroke}%
\pgfsetdash{}{0pt}%
\pgfpathmoveto{\pgfqpoint{1.611634in}{1.126145in}}%
\pgfpathlineto{\pgfqpoint{1.648811in}{1.127363in}}%
\pgfpathlineto{\pgfqpoint{1.611340in}{1.161473in}}%
\pgfpathlineto{\pgfqpoint{1.574046in}{1.160375in}}%
\pgfpathclose%
\pgfusepath{fill}%
\end{pgfscope}%
\begin{pgfscope}%
\pgfpathrectangle{\pgfqpoint{0.150000in}{0.150000in}}{\pgfqpoint{2.700000in}{1.950000in}}%
\pgfusepath{clip}%
\pgfsetbuttcap%
\pgfsetroundjoin%
\definecolor{currentfill}{rgb}{0.536657,0.593719,0.673606}%
\pgfsetfillcolor{currentfill}%
\pgfsetlinewidth{0.000000pt}%
\definecolor{currentstroke}{rgb}{0.000000,0.000000,0.000000}%
\pgfsetstrokecolor{currentstroke}%
\pgfsetdash{}{0pt}%
\pgfpathmoveto{\pgfqpoint{1.123072in}{1.550963in}}%
\pgfpathlineto{\pgfqpoint{1.162192in}{1.544464in}}%
\pgfpathlineto{\pgfqpoint{1.124807in}{1.581518in}}%
\pgfpathlineto{\pgfqpoint{1.085538in}{1.588169in}}%
\pgfpathclose%
\pgfusepath{fill}%
\end{pgfscope}%
\begin{pgfscope}%
\pgfpathrectangle{\pgfqpoint{0.150000in}{0.150000in}}{\pgfqpoint{2.700000in}{1.950000in}}%
\pgfusepath{clip}%
\pgfsetbuttcap%
\pgfsetroundjoin%
\definecolor{currentfill}{rgb}{0.872733,0.769072,0.777282}%
\pgfsetfillcolor{currentfill}%
\pgfsetlinewidth{0.000000pt}%
\definecolor{currentstroke}{rgb}{0.000000,0.000000,0.000000}%
\pgfsetstrokecolor{currentstroke}%
\pgfsetdash{}{0pt}%
\pgfpathmoveto{\pgfqpoint{2.062637in}{0.744596in}}%
\pgfpathlineto{\pgfqpoint{2.098234in}{0.747158in}}%
\pgfpathlineto{\pgfqpoint{2.060579in}{0.781424in}}%
\pgfpathlineto{\pgfqpoint{2.024863in}{0.778984in}}%
\pgfpathclose%
\pgfusepath{fill}%
\end{pgfscope}%
\begin{pgfscope}%
\pgfpathrectangle{\pgfqpoint{0.150000in}{0.150000in}}{\pgfqpoint{2.700000in}{1.950000in}}%
\pgfusepath{clip}%
\pgfsetbuttcap%
\pgfsetroundjoin%
\definecolor{currentfill}{rgb}{0.916039,0.926379,0.940855}%
\pgfsetfillcolor{currentfill}%
\pgfsetlinewidth{0.000000pt}%
\definecolor{currentstroke}{rgb}{0.000000,0.000000,0.000000}%
\pgfsetstrokecolor{currentstroke}%
\pgfsetdash{}{0pt}%
\pgfpathmoveto{\pgfqpoint{1.649253in}{1.091887in}}%
\pgfpathlineto{\pgfqpoint{1.686269in}{1.090245in}}%
\pgfpathlineto{\pgfqpoint{1.648811in}{1.127363in}}%
\pgfpathlineto{\pgfqpoint{1.611634in}{1.126145in}}%
\pgfpathclose%
\pgfusepath{fill}%
\end{pgfscope}%
\begin{pgfscope}%
\pgfpathrectangle{\pgfqpoint{0.150000in}{0.150000in}}{\pgfqpoint{2.700000in}{1.950000in}}%
\pgfusepath{clip}%
\pgfsetbuttcap%
\pgfsetroundjoin%
\definecolor{currentfill}{rgb}{0.561535,0.615533,0.691131}%
\pgfsetfillcolor{currentfill}%
\pgfsetlinewidth{0.000000pt}%
\definecolor{currentstroke}{rgb}{0.000000,0.000000,0.000000}%
\pgfsetstrokecolor{currentstroke}%
\pgfsetdash{}{0pt}%
\pgfpathmoveto{\pgfqpoint{1.160508in}{1.516865in}}%
\pgfpathlineto{\pgfqpoint{1.199490in}{1.510502in}}%
\pgfpathlineto{\pgfqpoint{1.162192in}{1.544464in}}%
\pgfpathlineto{\pgfqpoint{1.123072in}{1.550963in}}%
\pgfpathclose%
\pgfusepath{fill}%
\end{pgfscope}%
\begin{pgfscope}%
\pgfpathrectangle{\pgfqpoint{0.150000in}{0.150000in}}{\pgfqpoint{2.700000in}{1.950000in}}%
\pgfusepath{clip}%
\pgfsetbuttcap%
\pgfsetroundjoin%
\definecolor{currentfill}{rgb}{0.947135,0.953646,0.962760}%
\pgfsetfillcolor{currentfill}%
\pgfsetlinewidth{0.000000pt}%
\definecolor{currentstroke}{rgb}{0.000000,0.000000,0.000000}%
\pgfsetstrokecolor{currentstroke}%
\pgfsetdash{}{0pt}%
\pgfpathmoveto{\pgfqpoint{1.686900in}{1.057603in}}%
\pgfpathlineto{\pgfqpoint{1.723788in}{1.056090in}}%
\pgfpathlineto{\pgfqpoint{1.686269in}{1.090245in}}%
\pgfpathlineto{\pgfqpoint{1.649253in}{1.091887in}}%
\pgfpathclose%
\pgfusepath{fill}%
\end{pgfscope}%
\begin{pgfscope}%
\pgfpathrectangle{\pgfqpoint{0.150000in}{0.150000in}}{\pgfqpoint{2.700000in}{1.950000in}}%
\pgfusepath{clip}%
\pgfsetbuttcap%
\pgfsetroundjoin%
\definecolor{currentfill}{rgb}{0.592632,0.642800,0.713036}%
\pgfsetfillcolor{currentfill}%
\pgfsetlinewidth{0.000000pt}%
\definecolor{currentstroke}{rgb}{0.000000,0.000000,0.000000}%
\pgfsetstrokecolor{currentstroke}%
\pgfsetdash{}{0pt}%
\pgfpathmoveto{\pgfqpoint{1.198069in}{1.479633in}}%
\pgfpathlineto{\pgfqpoint{1.236816in}{1.476514in}}%
\pgfpathlineto{\pgfqpoint{1.199490in}{1.510502in}}%
\pgfpathlineto{\pgfqpoint{1.160508in}{1.516865in}}%
\pgfpathclose%
\pgfusepath{fill}%
\end{pgfscope}%
\begin{pgfscope}%
\pgfpathrectangle{\pgfqpoint{0.150000in}{0.150000in}}{\pgfqpoint{2.700000in}{1.950000in}}%
\pgfusepath{clip}%
\pgfsetbuttcap%
\pgfsetroundjoin%
\definecolor{currentfill}{rgb}{0.972013,0.975460,0.980285}%
\pgfsetfillcolor{currentfill}%
\pgfsetlinewidth{0.000000pt}%
\definecolor{currentstroke}{rgb}{0.000000,0.000000,0.000000}%
\pgfsetstrokecolor{currentstroke}%
\pgfsetdash{}{0pt}%
\pgfpathmoveto{\pgfqpoint{1.724524in}{1.020316in}}%
\pgfpathlineto{\pgfqpoint{1.761337in}{1.021909in}}%
\pgfpathlineto{\pgfqpoint{1.723788in}{1.056090in}}%
\pgfpathlineto{\pgfqpoint{1.686900in}{1.057603in}}%
\pgfpathclose%
\pgfusepath{fill}%
\end{pgfscope}%
\begin{pgfscope}%
\pgfpathrectangle{\pgfqpoint{0.150000in}{0.150000in}}{\pgfqpoint{2.700000in}{1.950000in}}%
\pgfusepath{clip}%
\pgfsetbuttcap%
\pgfsetroundjoin%
\definecolor{currentfill}{rgb}{0.617509,0.664614,0.730561}%
\pgfsetfillcolor{currentfill}%
\pgfsetlinewidth{0.000000pt}%
\definecolor{currentstroke}{rgb}{0.000000,0.000000,0.000000}%
\pgfsetstrokecolor{currentstroke}%
\pgfsetdash{}{0pt}%
\pgfpathmoveto{\pgfqpoint{1.235553in}{1.445489in}}%
\pgfpathlineto{\pgfqpoint{1.274172in}{1.442498in}}%
\pgfpathlineto{\pgfqpoint{1.236816in}{1.476514in}}%
\pgfpathlineto{\pgfqpoint{1.198069in}{1.479633in}}%
\pgfpathclose%
\pgfusepath{fill}%
\end{pgfscope}%
\begin{pgfscope}%
\pgfpathrectangle{\pgfqpoint{0.150000in}{0.150000in}}{\pgfqpoint{2.700000in}{1.950000in}}%
\pgfusepath{clip}%
\pgfsetbuttcap%
\pgfsetroundjoin%
\definecolor{currentfill}{rgb}{0.996890,0.997273,0.997809}%
\pgfsetfillcolor{currentfill}%
\pgfsetlinewidth{0.000000pt}%
\definecolor{currentstroke}{rgb}{0.000000,0.000000,0.000000}%
\pgfsetstrokecolor{currentstroke}%
\pgfsetdash{}{0pt}%
\pgfpathmoveto{\pgfqpoint{1.762221in}{0.985986in}}%
\pgfpathlineto{\pgfqpoint{1.798914in}{0.987700in}}%
\pgfpathlineto{\pgfqpoint{1.761337in}{1.021909in}}%
\pgfpathlineto{\pgfqpoint{1.724524in}{1.020316in}}%
\pgfpathclose%
\pgfusepath{fill}%
\end{pgfscope}%
\begin{pgfscope}%
\pgfpathrectangle{\pgfqpoint{0.150000in}{0.150000in}}{\pgfqpoint{2.700000in}{1.950000in}}%
\pgfusepath{clip}%
\pgfsetbuttcap%
\pgfsetroundjoin%
\definecolor{currentfill}{rgb}{0.642387,0.686428,0.748085}%
\pgfsetfillcolor{currentfill}%
\pgfsetlinewidth{0.000000pt}%
\definecolor{currentstroke}{rgb}{0.000000,0.000000,0.000000}%
\pgfsetstrokecolor{currentstroke}%
\pgfsetdash{}{0pt}%
\pgfpathmoveto{\pgfqpoint{1.273066in}{1.411318in}}%
\pgfpathlineto{\pgfqpoint{1.311622in}{1.405382in}}%
\pgfpathlineto{\pgfqpoint{1.274172in}{1.442498in}}%
\pgfpathlineto{\pgfqpoint{1.235553in}{1.445489in}}%
\pgfpathclose%
\pgfusepath{fill}%
\end{pgfscope}%
\begin{pgfscope}%
\pgfpathrectangle{\pgfqpoint{0.150000in}{0.150000in}}{\pgfqpoint{2.700000in}{1.950000in}}%
\pgfusepath{clip}%
\pgfsetbuttcap%
\pgfsetroundjoin%
\definecolor{currentfill}{rgb}{0.673483,0.713695,0.769991}%
\pgfsetfillcolor{currentfill}%
\pgfsetlinewidth{0.000000pt}%
\definecolor{currentstroke}{rgb}{0.000000,0.000000,0.000000}%
\pgfsetstrokecolor{currentstroke}%
\pgfsetdash{}{0pt}%
\pgfpathmoveto{\pgfqpoint{1.310673in}{1.374042in}}%
\pgfpathlineto{\pgfqpoint{1.349026in}{1.371322in}}%
\pgfpathlineto{\pgfqpoint{1.311622in}{1.405382in}}%
\pgfpathlineto{\pgfqpoint{1.273066in}{1.411318in}}%
\pgfpathclose%
\pgfusepath{fill}%
\end{pgfscope}%
\begin{pgfscope}%
\pgfpathrectangle{\pgfqpoint{0.150000in}{0.150000in}}{\pgfqpoint{2.700000in}{1.950000in}}%
\pgfusepath{clip}%
\pgfsetbuttcap%
\pgfsetroundjoin%
\definecolor{currentfill}{rgb}{0.986703,0.975873,0.976731}%
\pgfsetfillcolor{currentfill}%
\pgfsetlinewidth{0.000000pt}%
\definecolor{currentstroke}{rgb}{0.000000,0.000000,0.000000}%
\pgfsetstrokecolor{currentstroke}%
\pgfsetdash{}{0pt}%
\pgfpathmoveto{\pgfqpoint{1.799947in}{0.951629in}}%
\pgfpathlineto{\pgfqpoint{1.836522in}{0.953465in}}%
\pgfpathlineto{\pgfqpoint{1.798914in}{0.987700in}}%
\pgfpathlineto{\pgfqpoint{1.762221in}{0.985986in}}%
\pgfpathclose%
\pgfusepath{fill}%
\end{pgfscope}%
\begin{pgfscope}%
\pgfpathrectangle{\pgfqpoint{0.150000in}{0.150000in}}{\pgfqpoint{2.700000in}{1.950000in}}%
\pgfusepath{clip}%
\pgfsetbuttcap%
\pgfsetroundjoin%
\definecolor{currentfill}{rgb}{0.698361,0.735509,0.787515}%
\pgfsetfillcolor{currentfill}%
\pgfsetlinewidth{0.000000pt}%
\definecolor{currentstroke}{rgb}{0.000000,0.000000,0.000000}%
\pgfsetstrokecolor{currentstroke}%
\pgfsetdash{}{0pt}%
\pgfpathmoveto{\pgfqpoint{1.348234in}{1.339826in}}%
\pgfpathlineto{\pgfqpoint{1.386459in}{1.337235in}}%
\pgfpathlineto{\pgfqpoint{1.349026in}{1.371322in}}%
\pgfpathlineto{\pgfqpoint{1.310673in}{1.374042in}}%
\pgfpathclose%
\pgfusepath{fill}%
\end{pgfscope}%
\begin{pgfscope}%
\pgfpathrectangle{\pgfqpoint{0.150000in}{0.150000in}}{\pgfqpoint{2.700000in}{1.950000in}}%
\pgfusepath{clip}%
\pgfsetbuttcap%
\pgfsetroundjoin%
\definecolor{currentfill}{rgb}{0.971507,0.948300,0.950138}%
\pgfsetfillcolor{currentfill}%
\pgfsetlinewidth{0.000000pt}%
\definecolor{currentstroke}{rgb}{0.000000,0.000000,0.000000}%
\pgfsetstrokecolor{currentstroke}%
\pgfsetdash{}{0pt}%
\pgfpathmoveto{\pgfqpoint{1.837616in}{0.914298in}}%
\pgfpathlineto{\pgfqpoint{1.874159in}{0.919204in}}%
\pgfpathlineto{\pgfqpoint{1.836522in}{0.953465in}}%
\pgfpathlineto{\pgfqpoint{1.799947in}{0.951629in}}%
\pgfpathclose%
\pgfusepath{fill}%
\end{pgfscope}%
\begin{pgfscope}%
\pgfpathrectangle{\pgfqpoint{0.150000in}{0.150000in}}{\pgfqpoint{2.700000in}{1.950000in}}%
\pgfusepath{clip}%
\pgfsetbuttcap%
\pgfsetroundjoin%
\definecolor{currentfill}{rgb}{0.723238,0.757322,0.805040}%
\pgfsetfillcolor{currentfill}%
\pgfsetlinewidth{0.000000pt}%
\definecolor{currentstroke}{rgb}{0.000000,0.000000,0.000000}%
\pgfsetstrokecolor{currentstroke}%
\pgfsetdash{}{0pt}%
\pgfpathmoveto{\pgfqpoint{1.385825in}{1.305583in}}%
\pgfpathlineto{\pgfqpoint{1.423954in}{1.300075in}}%
\pgfpathlineto{\pgfqpoint{1.386459in}{1.337235in}}%
\pgfpathlineto{\pgfqpoint{1.348234in}{1.339826in}}%
\pgfpathclose%
\pgfusepath{fill}%
\end{pgfscope}%
\begin{pgfscope}%
\pgfpathrectangle{\pgfqpoint{0.150000in}{0.150000in}}{\pgfqpoint{2.700000in}{1.950000in}}%
\pgfusepath{clip}%
\pgfsetbuttcap%
\pgfsetroundjoin%
\definecolor{currentfill}{rgb}{0.952512,0.913833,0.916896}%
\pgfsetfillcolor{currentfill}%
\pgfsetlinewidth{0.000000pt}%
\definecolor{currentstroke}{rgb}{0.000000,0.000000,0.000000}%
\pgfsetstrokecolor{currentstroke}%
\pgfsetdash{}{0pt}%
\pgfpathmoveto{\pgfqpoint{1.875391in}{0.879896in}}%
\pgfpathlineto{\pgfqpoint{1.911718in}{0.881986in}}%
\pgfpathlineto{\pgfqpoint{1.874159in}{0.919204in}}%
\pgfpathlineto{\pgfqpoint{1.837616in}{0.914298in}}%
\pgfpathclose%
\pgfusepath{fill}%
\end{pgfscope}%
\begin{pgfscope}%
\pgfpathrectangle{\pgfqpoint{0.150000in}{0.150000in}}{\pgfqpoint{2.700000in}{1.950000in}}%
\pgfusepath{clip}%
\pgfsetbuttcap%
\pgfsetroundjoin%
\definecolor{currentfill}{rgb}{0.754335,0.784589,0.826945}%
\pgfsetfillcolor{currentfill}%
\pgfsetlinewidth{0.000000pt}%
\definecolor{currentstroke}{rgb}{0.000000,0.000000,0.000000}%
\pgfsetstrokecolor{currentstroke}%
\pgfsetdash{}{0pt}%
\pgfpathmoveto{\pgfqpoint{1.423479in}{1.268261in}}%
\pgfpathlineto{\pgfqpoint{1.461435in}{1.265943in}}%
\pgfpathlineto{\pgfqpoint{1.423954in}{1.300075in}}%
\pgfpathlineto{\pgfqpoint{1.385825in}{1.305583in}}%
\pgfpathclose%
\pgfusepath{fill}%
\end{pgfscope}%
\begin{pgfscope}%
\pgfpathrectangle{\pgfqpoint{0.150000in}{0.150000in}}{\pgfqpoint{2.700000in}{1.950000in}}%
\pgfusepath{clip}%
\pgfsetbuttcap%
\pgfsetroundjoin%
\definecolor{currentfill}{rgb}{0.937316,0.886259,0.890303}%
\pgfsetfillcolor{currentfill}%
\pgfsetlinewidth{0.000000pt}%
\definecolor{currentstroke}{rgb}{0.000000,0.000000,0.000000}%
\pgfsetstrokecolor{currentstroke}%
\pgfsetdash{}{0pt}%
\pgfpathmoveto{\pgfqpoint{1.913195in}{0.845466in}}%
\pgfpathlineto{\pgfqpoint{1.949403in}{0.847679in}}%
\pgfpathlineto{\pgfqpoint{1.911718in}{0.881986in}}%
\pgfpathlineto{\pgfqpoint{1.875391in}{0.879896in}}%
\pgfpathclose%
\pgfusepath{fill}%
\end{pgfscope}%
\begin{pgfscope}%
\pgfpathrectangle{\pgfqpoint{0.150000in}{0.150000in}}{\pgfqpoint{2.700000in}{1.950000in}}%
\pgfusepath{clip}%
\pgfsetbuttcap%
\pgfsetroundjoin%
\definecolor{currentfill}{rgb}{0.779213,0.806403,0.844470}%
\pgfsetfillcolor{currentfill}%
\pgfsetlinewidth{0.000000pt}%
\definecolor{currentstroke}{rgb}{0.000000,0.000000,0.000000}%
\pgfsetstrokecolor{currentstroke}%
\pgfsetdash{}{0pt}%
\pgfpathmoveto{\pgfqpoint{1.461118in}{1.233973in}}%
\pgfpathlineto{\pgfqpoint{1.498946in}{1.231784in}}%
\pgfpathlineto{\pgfqpoint{1.461435in}{1.265943in}}%
\pgfpathlineto{\pgfqpoint{1.423479in}{1.268261in}}%
\pgfpathclose%
\pgfusepath{fill}%
\end{pgfscope}%
\begin{pgfscope}%
\pgfpathrectangle{\pgfqpoint{0.150000in}{0.150000in}}{\pgfqpoint{2.700000in}{1.950000in}}%
\pgfusepath{clip}%
\pgfsetbuttcap%
\pgfsetroundjoin%
\definecolor{currentfill}{rgb}{0.922120,0.858686,0.863710}%
\pgfsetfillcolor{currentfill}%
\pgfsetlinewidth{0.000000pt}%
\definecolor{currentstroke}{rgb}{0.000000,0.000000,0.000000}%
\pgfsetstrokecolor{currentstroke}%
\pgfsetdash{}{0pt}%
\pgfpathmoveto{\pgfqpoint{1.951029in}{0.811010in}}%
\pgfpathlineto{\pgfqpoint{1.987118in}{0.813345in}}%
\pgfpathlineto{\pgfqpoint{1.949403in}{0.847679in}}%
\pgfpathlineto{\pgfqpoint{1.913195in}{0.845466in}}%
\pgfpathclose%
\pgfusepath{fill}%
\end{pgfscope}%
\begin{pgfscope}%
\pgfpathrectangle{\pgfqpoint{0.150000in}{0.150000in}}{\pgfqpoint{2.700000in}{1.950000in}}%
\pgfusepath{clip}%
\pgfsetbuttcap%
\pgfsetroundjoin%
\definecolor{currentfill}{rgb}{0.804090,0.828217,0.861994}%
\pgfsetfillcolor{currentfill}%
\pgfsetlinewidth{0.000000pt}%
\definecolor{currentstroke}{rgb}{0.000000,0.000000,0.000000}%
\pgfsetstrokecolor{currentstroke}%
\pgfsetdash{}{0pt}%
\pgfpathmoveto{\pgfqpoint{1.498798in}{1.196625in}}%
\pgfpathlineto{\pgfqpoint{1.536486in}{1.197599in}}%
\pgfpathlineto{\pgfqpoint{1.498946in}{1.231784in}}%
\pgfpathlineto{\pgfqpoint{1.461118in}{1.233973in}}%
\pgfpathclose%
\pgfusepath{fill}%
\end{pgfscope}%
\begin{pgfscope}%
\pgfpathrectangle{\pgfqpoint{0.150000in}{0.150000in}}{\pgfqpoint{2.700000in}{1.950000in}}%
\pgfusepath{clip}%
\pgfsetbuttcap%
\pgfsetroundjoin%
\definecolor{currentfill}{rgb}{0.906924,0.831112,0.837117}%
\pgfsetfillcolor{currentfill}%
\pgfsetlinewidth{0.000000pt}%
\definecolor{currentstroke}{rgb}{0.000000,0.000000,0.000000}%
\pgfsetstrokecolor{currentstroke}%
\pgfsetdash{}{0pt}%
\pgfpathmoveto{\pgfqpoint{1.988764in}{0.773615in}}%
\pgfpathlineto{\pgfqpoint{2.024863in}{0.778984in}}%
\pgfpathlineto{\pgfqpoint{1.987118in}{0.813345in}}%
\pgfpathlineto{\pgfqpoint{1.951029in}{0.811010in}}%
\pgfpathclose%
\pgfusepath{fill}%
\end{pgfscope}%
\begin{pgfscope}%
\pgfpathrectangle{\pgfqpoint{0.150000in}{0.150000in}}{\pgfqpoint{2.700000in}{1.950000in}}%
\pgfusepath{clip}%
\pgfsetbuttcap%
\pgfsetroundjoin%
\definecolor{currentfill}{rgb}{0.835187,0.855484,0.883900}%
\pgfsetfillcolor{currentfill}%
\pgfsetlinewidth{0.000000pt}%
\definecolor{currentstroke}{rgb}{0.000000,0.000000,0.000000}%
\pgfsetstrokecolor{currentstroke}%
\pgfsetdash{}{0pt}%
\pgfpathmoveto{\pgfqpoint{1.536486in}{1.162291in}}%
\pgfpathlineto{\pgfqpoint{1.574046in}{1.160375in}}%
\pgfpathlineto{\pgfqpoint{1.536486in}{1.197599in}}%
\pgfpathlineto{\pgfqpoint{1.498798in}{1.196625in}}%
\pgfpathclose%
\pgfusepath{fill}%
\end{pgfscope}%
\begin{pgfscope}%
\pgfpathrectangle{\pgfqpoint{0.150000in}{0.150000in}}{\pgfqpoint{2.700000in}{1.950000in}}%
\pgfusepath{clip}%
\pgfsetbuttcap%
\pgfsetroundjoin%
\definecolor{currentfill}{rgb}{0.887929,0.796645,0.803876}%
\pgfsetfillcolor{currentfill}%
\pgfsetlinewidth{0.000000pt}%
\definecolor{currentstroke}{rgb}{0.000000,0.000000,0.000000}%
\pgfsetstrokecolor{currentstroke}%
\pgfsetdash{}{0pt}%
\pgfpathmoveto{\pgfqpoint{2.026647in}{0.739112in}}%
\pgfpathlineto{\pgfqpoint{2.062637in}{0.744596in}}%
\pgfpathlineto{\pgfqpoint{2.024863in}{0.778984in}}%
\pgfpathlineto{\pgfqpoint{1.988764in}{0.773615in}}%
\pgfpathclose%
\pgfusepath{fill}%
\end{pgfscope}%
\begin{pgfscope}%
\pgfpathrectangle{\pgfqpoint{0.150000in}{0.150000in}}{\pgfqpoint{2.700000in}{1.950000in}}%
\pgfusepath{clip}%
\pgfsetbuttcap%
\pgfsetroundjoin%
\definecolor{currentfill}{rgb}{0.860064,0.877298,0.901425}%
\pgfsetfillcolor{currentfill}%
\pgfsetlinewidth{0.000000pt}%
\definecolor{currentstroke}{rgb}{0.000000,0.000000,0.000000}%
\pgfsetstrokecolor{currentstroke}%
\pgfsetdash{}{0pt}%
\pgfpathmoveto{\pgfqpoint{1.574204in}{1.127930in}}%
\pgfpathlineto{\pgfqpoint{1.611634in}{1.126145in}}%
\pgfpathlineto{\pgfqpoint{1.574046in}{1.160375in}}%
\pgfpathlineto{\pgfqpoint{1.536486in}{1.162291in}}%
\pgfpathclose%
\pgfusepath{fill}%
\end{pgfscope}%
\begin{pgfscope}%
\pgfpathrectangle{\pgfqpoint{0.150000in}{0.150000in}}{\pgfqpoint{2.700000in}{1.950000in}}%
\pgfusepath{clip}%
\pgfsetbuttcap%
\pgfsetroundjoin%
\definecolor{currentfill}{rgb}{0.468244,0.533732,0.625414}%
\pgfsetfillcolor{currentfill}%
\pgfsetlinewidth{0.000000pt}%
\definecolor{currentstroke}{rgb}{0.000000,0.000000,0.000000}%
\pgfsetstrokecolor{currentstroke}%
\pgfsetdash{}{0pt}%
\pgfpathmoveto{\pgfqpoint{1.045945in}{1.594874in}}%
\pgfpathlineto{\pgfqpoint{1.085538in}{1.588169in}}%
\pgfpathlineto{\pgfqpoint{1.048151in}{1.622222in}}%
\pgfpathlineto{\pgfqpoint{1.008420in}{1.629066in}}%
\pgfpathclose%
\pgfusepath{fill}%
\end{pgfscope}%
\begin{pgfscope}%
\pgfpathrectangle{\pgfqpoint{0.150000in}{0.150000in}}{\pgfqpoint{2.700000in}{1.950000in}}%
\pgfusepath{clip}%
\pgfsetbuttcap%
\pgfsetroundjoin%
\definecolor{currentfill}{rgb}{0.884942,0.899112,0.918949}%
\pgfsetfillcolor{currentfill}%
\pgfsetlinewidth{0.000000pt}%
\definecolor{currentstroke}{rgb}{0.000000,0.000000,0.000000}%
\pgfsetstrokecolor{currentstroke}%
\pgfsetdash{}{0pt}%
\pgfpathmoveto{\pgfqpoint{1.611931in}{1.090538in}}%
\pgfpathlineto{\pgfqpoint{1.649253in}{1.091887in}}%
\pgfpathlineto{\pgfqpoint{1.611634in}{1.126145in}}%
\pgfpathlineto{\pgfqpoint{1.574204in}{1.127930in}}%
\pgfpathclose%
\pgfusepath{fill}%
\end{pgfscope}%
\begin{pgfscope}%
\pgfpathrectangle{\pgfqpoint{0.150000in}{0.150000in}}{\pgfqpoint{2.700000in}{1.950000in}}%
\pgfusepath{clip}%
\pgfsetbuttcap%
\pgfsetroundjoin%
\definecolor{currentfill}{rgb}{0.493122,0.555545,0.642938}%
\pgfsetfillcolor{currentfill}%
\pgfsetlinewidth{0.000000pt}%
\definecolor{currentstroke}{rgb}{0.000000,0.000000,0.000000}%
\pgfsetstrokecolor{currentstroke}%
\pgfsetdash{}{0pt}%
\pgfpathmoveto{\pgfqpoint{1.083630in}{1.557516in}}%
\pgfpathlineto{\pgfqpoint{1.123072in}{1.550963in}}%
\pgfpathlineto{\pgfqpoint{1.085538in}{1.588169in}}%
\pgfpathlineto{\pgfqpoint{1.045945in}{1.594874in}}%
\pgfpathclose%
\pgfusepath{fill}%
\end{pgfscope}%
\begin{pgfscope}%
\pgfpathrectangle{\pgfqpoint{0.150000in}{0.150000in}}{\pgfqpoint{2.700000in}{1.950000in}}%
\pgfusepath{clip}%
\pgfsetbuttcap%
\pgfsetroundjoin%
\definecolor{currentfill}{rgb}{0.909819,0.920925,0.936474}%
\pgfsetfillcolor{currentfill}%
\pgfsetlinewidth{0.000000pt}%
\definecolor{currentstroke}{rgb}{0.000000,0.000000,0.000000}%
\pgfsetstrokecolor{currentstroke}%
\pgfsetdash{}{0pt}%
\pgfpathmoveto{\pgfqpoint{1.649697in}{1.056131in}}%
\pgfpathlineto{\pgfqpoint{1.686900in}{1.057603in}}%
\pgfpathlineto{\pgfqpoint{1.649253in}{1.091887in}}%
\pgfpathlineto{\pgfqpoint{1.611931in}{1.090538in}}%
\pgfpathclose%
\pgfusepath{fill}%
\end{pgfscope}%
\begin{pgfscope}%
\pgfpathrectangle{\pgfqpoint{0.150000in}{0.150000in}}{\pgfqpoint{2.700000in}{1.950000in}}%
\pgfusepath{clip}%
\pgfsetbuttcap%
\pgfsetroundjoin%
\definecolor{currentfill}{rgb}{0.524219,0.582812,0.664844}%
\pgfsetfillcolor{currentfill}%
\pgfsetlinewidth{0.000000pt}%
\definecolor{currentstroke}{rgb}{0.000000,0.000000,0.000000}%
\pgfsetstrokecolor{currentstroke}%
\pgfsetdash{}{0pt}%
\pgfpathmoveto{\pgfqpoint{1.121204in}{1.523279in}}%
\pgfpathlineto{\pgfqpoint{1.160508in}{1.516865in}}%
\pgfpathlineto{\pgfqpoint{1.123072in}{1.550963in}}%
\pgfpathlineto{\pgfqpoint{1.083630in}{1.557516in}}%
\pgfpathclose%
\pgfusepath{fill}%
\end{pgfscope}%
\begin{pgfscope}%
\pgfpathrectangle{\pgfqpoint{0.150000in}{0.150000in}}{\pgfqpoint{2.700000in}{1.950000in}}%
\pgfusepath{clip}%
\pgfsetbuttcap%
\pgfsetroundjoin%
\definecolor{currentfill}{rgb}{0.940916,0.948192,0.958379}%
\pgfsetfillcolor{currentfill}%
\pgfsetlinewidth{0.000000pt}%
\definecolor{currentstroke}{rgb}{0.000000,0.000000,0.000000}%
\pgfsetstrokecolor{currentstroke}%
\pgfsetdash{}{0pt}%
\pgfpathmoveto{\pgfqpoint{1.687451in}{1.018712in}}%
\pgfpathlineto{\pgfqpoint{1.724524in}{1.020316in}}%
\pgfpathlineto{\pgfqpoint{1.686900in}{1.057603in}}%
\pgfpathlineto{\pgfqpoint{1.649697in}{1.056131in}}%
\pgfpathclose%
\pgfusepath{fill}%
\end{pgfscope}%
\begin{pgfscope}%
\pgfpathrectangle{\pgfqpoint{0.150000in}{0.150000in}}{\pgfqpoint{2.700000in}{1.950000in}}%
\pgfusepath{clip}%
\pgfsetbuttcap%
\pgfsetroundjoin%
\definecolor{currentfill}{rgb}{0.549096,0.604626,0.682368}%
\pgfsetfillcolor{currentfill}%
\pgfsetlinewidth{0.000000pt}%
\definecolor{currentstroke}{rgb}{0.000000,0.000000,0.000000}%
\pgfsetstrokecolor{currentstroke}%
\pgfsetdash{}{0pt}%
\pgfpathmoveto{\pgfqpoint{1.158916in}{1.485895in}}%
\pgfpathlineto{\pgfqpoint{1.198069in}{1.479633in}}%
\pgfpathlineto{\pgfqpoint{1.160508in}{1.516865in}}%
\pgfpathlineto{\pgfqpoint{1.121204in}{1.523279in}}%
\pgfpathclose%
\pgfusepath{fill}%
\end{pgfscope}%
\begin{pgfscope}%
\pgfpathrectangle{\pgfqpoint{0.150000in}{0.150000in}}{\pgfqpoint{2.700000in}{1.950000in}}%
\pgfusepath{clip}%
\pgfsetbuttcap%
\pgfsetroundjoin%
\definecolor{currentfill}{rgb}{0.965794,0.970006,0.975904}%
\pgfsetfillcolor{currentfill}%
\pgfsetlinewidth{0.000000pt}%
\definecolor{currentstroke}{rgb}{0.000000,0.000000,0.000000}%
\pgfsetstrokecolor{currentstroke}%
\pgfsetdash{}{0pt}%
\pgfpathmoveto{\pgfqpoint{1.725266in}{0.984260in}}%
\pgfpathlineto{\pgfqpoint{1.762221in}{0.985986in}}%
\pgfpathlineto{\pgfqpoint{1.724524in}{1.020316in}}%
\pgfpathlineto{\pgfqpoint{1.687451in}{1.018712in}}%
\pgfpathclose%
\pgfusepath{fill}%
\end{pgfscope}%
\begin{pgfscope}%
\pgfpathrectangle{\pgfqpoint{0.150000in}{0.150000in}}{\pgfqpoint{2.700000in}{1.950000in}}%
\pgfusepath{clip}%
\pgfsetbuttcap%
\pgfsetroundjoin%
\definecolor{currentfill}{rgb}{0.580193,0.631893,0.704274}%
\pgfsetfillcolor{currentfill}%
\pgfsetlinewidth{0.000000pt}%
\definecolor{currentstroke}{rgb}{0.000000,0.000000,0.000000}%
\pgfsetstrokecolor{currentstroke}%
\pgfsetdash{}{0pt}%
\pgfpathmoveto{\pgfqpoint{1.196539in}{1.451613in}}%
\pgfpathlineto{\pgfqpoint{1.235553in}{1.445489in}}%
\pgfpathlineto{\pgfqpoint{1.198069in}{1.479633in}}%
\pgfpathlineto{\pgfqpoint{1.158916in}{1.485895in}}%
\pgfpathclose%
\pgfusepath{fill}%
\end{pgfscope}%
\begin{pgfscope}%
\pgfpathrectangle{\pgfqpoint{0.150000in}{0.150000in}}{\pgfqpoint{2.700000in}{1.950000in}}%
\pgfusepath{clip}%
\pgfsetbuttcap%
\pgfsetroundjoin%
\definecolor{currentfill}{rgb}{0.990671,0.991820,0.993428}%
\pgfsetfillcolor{currentfill}%
\pgfsetlinewidth{0.000000pt}%
\definecolor{currentstroke}{rgb}{0.000000,0.000000,0.000000}%
\pgfsetstrokecolor{currentstroke}%
\pgfsetdash{}{0pt}%
\pgfpathmoveto{\pgfqpoint{1.763112in}{0.949780in}}%
\pgfpathlineto{\pgfqpoint{1.799947in}{0.951629in}}%
\pgfpathlineto{\pgfqpoint{1.762221in}{0.985986in}}%
\pgfpathlineto{\pgfqpoint{1.725266in}{0.984260in}}%
\pgfpathclose%
\pgfusepath{fill}%
\end{pgfscope}%
\begin{pgfscope}%
\pgfpathrectangle{\pgfqpoint{0.150000in}{0.150000in}}{\pgfqpoint{2.700000in}{1.950000in}}%
\pgfusepath{clip}%
\pgfsetbuttcap%
\pgfsetroundjoin%
\definecolor{currentfill}{rgb}{0.605070,0.653707,0.721798}%
\pgfsetfillcolor{currentfill}%
\pgfsetlinewidth{0.000000pt}%
\definecolor{currentstroke}{rgb}{0.000000,0.000000,0.000000}%
\pgfsetstrokecolor{currentstroke}%
\pgfsetdash{}{0pt}%
\pgfpathmoveto{\pgfqpoint{1.234278in}{1.414202in}}%
\pgfpathlineto{\pgfqpoint{1.273066in}{1.411318in}}%
\pgfpathlineto{\pgfqpoint{1.235553in}{1.445489in}}%
\pgfpathlineto{\pgfqpoint{1.196539in}{1.451613in}}%
\pgfpathclose%
\pgfusepath{fill}%
\end{pgfscope}%
\begin{pgfscope}%
\pgfpathrectangle{\pgfqpoint{0.150000in}{0.150000in}}{\pgfqpoint{2.700000in}{1.950000in}}%
\pgfusepath{clip}%
\pgfsetbuttcap%
\pgfsetroundjoin%
\definecolor{currentfill}{rgb}{0.986703,0.975873,0.976731}%
\pgfsetfillcolor{currentfill}%
\pgfsetlinewidth{0.000000pt}%
\definecolor{currentstroke}{rgb}{0.000000,0.000000,0.000000}%
\pgfsetstrokecolor{currentstroke}%
\pgfsetdash{}{0pt}%
\pgfpathmoveto{\pgfqpoint{1.800911in}{0.912316in}}%
\pgfpathlineto{\pgfqpoint{1.837616in}{0.914298in}}%
\pgfpathlineto{\pgfqpoint{1.799947in}{0.951629in}}%
\pgfpathlineto{\pgfqpoint{1.763112in}{0.949780in}}%
\pgfpathclose%
\pgfusepath{fill}%
\end{pgfscope}%
\begin{pgfscope}%
\pgfpathrectangle{\pgfqpoint{0.150000in}{0.150000in}}{\pgfqpoint{2.700000in}{1.950000in}}%
\pgfusepath{clip}%
\pgfsetbuttcap%
\pgfsetroundjoin%
\definecolor{currentfill}{rgb}{0.636167,0.680974,0.743704}%
\pgfsetfillcolor{currentfill}%
\pgfsetlinewidth{0.000000pt}%
\definecolor{currentstroke}{rgb}{0.000000,0.000000,0.000000}%
\pgfsetstrokecolor{currentstroke}%
\pgfsetdash{}{0pt}%
\pgfpathmoveto{\pgfqpoint{1.271950in}{1.379874in}}%
\pgfpathlineto{\pgfqpoint{1.310673in}{1.374042in}}%
\pgfpathlineto{\pgfqpoint{1.273066in}{1.411318in}}%
\pgfpathlineto{\pgfqpoint{1.234278in}{1.414202in}}%
\pgfpathclose%
\pgfusepath{fill}%
\end{pgfscope}%
\begin{pgfscope}%
\pgfpathrectangle{\pgfqpoint{0.150000in}{0.150000in}}{\pgfqpoint{2.700000in}{1.950000in}}%
\pgfusepath{clip}%
\pgfsetbuttcap%
\pgfsetroundjoin%
\definecolor{currentfill}{rgb}{0.971507,0.948300,0.950138}%
\pgfsetfillcolor{currentfill}%
\pgfsetlinewidth{0.000000pt}%
\definecolor{currentstroke}{rgb}{0.000000,0.000000,0.000000}%
\pgfsetstrokecolor{currentstroke}%
\pgfsetdash{}{0pt}%
\pgfpathmoveto{\pgfqpoint{1.838805in}{0.877790in}}%
\pgfpathlineto{\pgfqpoint{1.875391in}{0.879896in}}%
\pgfpathlineto{\pgfqpoint{1.837616in}{0.914298in}}%
\pgfpathlineto{\pgfqpoint{1.800911in}{0.912316in}}%
\pgfpathclose%
\pgfusepath{fill}%
\end{pgfscope}%
\begin{pgfscope}%
\pgfpathrectangle{\pgfqpoint{0.150000in}{0.150000in}}{\pgfqpoint{2.700000in}{1.950000in}}%
\pgfusepath{clip}%
\pgfsetbuttcap%
\pgfsetroundjoin%
\definecolor{currentfill}{rgb}{0.661045,0.702788,0.761229}%
\pgfsetfillcolor{currentfill}%
\pgfsetlinewidth{0.000000pt}%
\definecolor{currentstroke}{rgb}{0.000000,0.000000,0.000000}%
\pgfsetstrokecolor{currentstroke}%
\pgfsetdash{}{0pt}%
\pgfpathmoveto{\pgfqpoint{1.309716in}{1.342436in}}%
\pgfpathlineto{\pgfqpoint{1.348234in}{1.339826in}}%
\pgfpathlineto{\pgfqpoint{1.310673in}{1.374042in}}%
\pgfpathlineto{\pgfqpoint{1.271950in}{1.379874in}}%
\pgfpathclose%
\pgfusepath{fill}%
\end{pgfscope}%
\begin{pgfscope}%
\pgfpathrectangle{\pgfqpoint{0.150000in}{0.150000in}}{\pgfqpoint{2.700000in}{1.950000in}}%
\pgfusepath{clip}%
\pgfsetbuttcap%
\pgfsetroundjoin%
\definecolor{currentfill}{rgb}{0.956311,0.920726,0.923545}%
\pgfsetfillcolor{currentfill}%
\pgfsetlinewidth{0.000000pt}%
\definecolor{currentstroke}{rgb}{0.000000,0.000000,0.000000}%
\pgfsetstrokecolor{currentstroke}%
\pgfsetdash{}{0pt}%
\pgfpathmoveto{\pgfqpoint{1.876632in}{0.840300in}}%
\pgfpathlineto{\pgfqpoint{1.913195in}{0.845466in}}%
\pgfpathlineto{\pgfqpoint{1.875391in}{0.879896in}}%
\pgfpathlineto{\pgfqpoint{1.838805in}{0.877790in}}%
\pgfpathclose%
\pgfusepath{fill}%
\end{pgfscope}%
\begin{pgfscope}%
\pgfpathrectangle{\pgfqpoint{0.150000in}{0.150000in}}{\pgfqpoint{2.700000in}{1.950000in}}%
\pgfusepath{clip}%
\pgfsetbuttcap%
\pgfsetroundjoin%
\definecolor{currentfill}{rgb}{0.685922,0.724602,0.778753}%
\pgfsetfillcolor{currentfill}%
\pgfsetlinewidth{0.000000pt}%
\definecolor{currentstroke}{rgb}{0.000000,0.000000,0.000000}%
\pgfsetstrokecolor{currentstroke}%
\pgfsetdash{}{0pt}%
\pgfpathmoveto{\pgfqpoint{1.347437in}{1.308062in}}%
\pgfpathlineto{\pgfqpoint{1.385825in}{1.305583in}}%
\pgfpathlineto{\pgfqpoint{1.348234in}{1.339826in}}%
\pgfpathlineto{\pgfqpoint{1.309716in}{1.342436in}}%
\pgfpathclose%
\pgfusepath{fill}%
\end{pgfscope}%
\begin{pgfscope}%
\pgfpathrectangle{\pgfqpoint{0.150000in}{0.150000in}}{\pgfqpoint{2.700000in}{1.950000in}}%
\pgfusepath{clip}%
\pgfsetbuttcap%
\pgfsetroundjoin%
\definecolor{currentfill}{rgb}{0.717019,0.751869,0.800659}%
\pgfsetfillcolor{currentfill}%
\pgfsetlinewidth{0.000000pt}%
\definecolor{currentstroke}{rgb}{0.000000,0.000000,0.000000}%
\pgfsetstrokecolor{currentstroke}%
\pgfsetdash{}{0pt}%
\pgfpathmoveto{\pgfqpoint{1.385230in}{1.270598in}}%
\pgfpathlineto{\pgfqpoint{1.423479in}{1.268261in}}%
\pgfpathlineto{\pgfqpoint{1.385825in}{1.305583in}}%
\pgfpathlineto{\pgfqpoint{1.347437in}{1.308062in}}%
\pgfpathclose%
\pgfusepath{fill}%
\end{pgfscope}%
\begin{pgfscope}%
\pgfpathrectangle{\pgfqpoint{0.150000in}{0.150000in}}{\pgfqpoint{2.700000in}{1.950000in}}%
\pgfusepath{clip}%
\pgfsetbuttcap%
\pgfsetroundjoin%
\definecolor{currentfill}{rgb}{0.937316,0.886259,0.890303}%
\pgfsetfillcolor{currentfill}%
\pgfsetlinewidth{0.000000pt}%
\definecolor{currentstroke}{rgb}{0.000000,0.000000,0.000000}%
\pgfsetstrokecolor{currentstroke}%
\pgfsetdash{}{0pt}%
\pgfpathmoveto{\pgfqpoint{1.914575in}{0.805728in}}%
\pgfpathlineto{\pgfqpoint{1.951029in}{0.811010in}}%
\pgfpathlineto{\pgfqpoint{1.913195in}{0.845466in}}%
\pgfpathlineto{\pgfqpoint{1.876632in}{0.840300in}}%
\pgfpathclose%
\pgfusepath{fill}%
\end{pgfscope}%
\begin{pgfscope}%
\pgfpathrectangle{\pgfqpoint{0.150000in}{0.150000in}}{\pgfqpoint{2.700000in}{1.950000in}}%
\pgfusepath{clip}%
\pgfsetbuttcap%
\pgfsetroundjoin%
\definecolor{currentfill}{rgb}{0.741896,0.773683,0.818183}%
\pgfsetfillcolor{currentfill}%
\pgfsetlinewidth{0.000000pt}%
\definecolor{currentstroke}{rgb}{0.000000,0.000000,0.000000}%
\pgfsetstrokecolor{currentstroke}%
\pgfsetdash{}{0pt}%
\pgfpathmoveto{\pgfqpoint{1.422999in}{1.236179in}}%
\pgfpathlineto{\pgfqpoint{1.461118in}{1.233973in}}%
\pgfpathlineto{\pgfqpoint{1.423479in}{1.268261in}}%
\pgfpathlineto{\pgfqpoint{1.385230in}{1.270598in}}%
\pgfpathclose%
\pgfusepath{fill}%
\end{pgfscope}%
\begin{pgfscope}%
\pgfpathrectangle{\pgfqpoint{0.150000in}{0.150000in}}{\pgfqpoint{2.700000in}{1.950000in}}%
\pgfusepath{clip}%
\pgfsetbuttcap%
\pgfsetroundjoin%
\definecolor{currentfill}{rgb}{0.922120,0.858686,0.863710}%
\pgfsetfillcolor{currentfill}%
\pgfsetlinewidth{0.000000pt}%
\definecolor{currentstroke}{rgb}{0.000000,0.000000,0.000000}%
\pgfsetstrokecolor{currentstroke}%
\pgfsetdash{}{0pt}%
\pgfpathmoveto{\pgfqpoint{1.952549in}{0.771130in}}%
\pgfpathlineto{\pgfqpoint{1.988764in}{0.773615in}}%
\pgfpathlineto{\pgfqpoint{1.951029in}{0.811010in}}%
\pgfpathlineto{\pgfqpoint{1.914575in}{0.805728in}}%
\pgfpathclose%
\pgfusepath{fill}%
\end{pgfscope}%
\begin{pgfscope}%
\pgfpathrectangle{\pgfqpoint{0.150000in}{0.150000in}}{\pgfqpoint{2.700000in}{1.950000in}}%
\pgfusepath{clip}%
\pgfsetbuttcap%
\pgfsetroundjoin%
\definecolor{currentfill}{rgb}{0.772993,0.800950,0.840089}%
\pgfsetfillcolor{currentfill}%
\pgfsetlinewidth{0.000000pt}%
\definecolor{currentstroke}{rgb}{0.000000,0.000000,0.000000}%
\pgfsetstrokecolor{currentstroke}%
\pgfsetdash{}{0pt}%
\pgfpathmoveto{\pgfqpoint{1.460820in}{1.198688in}}%
\pgfpathlineto{\pgfqpoint{1.498798in}{1.196625in}}%
\pgfpathlineto{\pgfqpoint{1.461118in}{1.233973in}}%
\pgfpathlineto{\pgfqpoint{1.422999in}{1.236179in}}%
\pgfpathclose%
\pgfusepath{fill}%
\end{pgfscope}%
\begin{pgfscope}%
\pgfpathrectangle{\pgfqpoint{0.150000in}{0.150000in}}{\pgfqpoint{2.700000in}{1.950000in}}%
\pgfusepath{clip}%
\pgfsetbuttcap%
\pgfsetroundjoin%
\definecolor{currentfill}{rgb}{0.906924,0.831112,0.837117}%
\pgfsetfillcolor{currentfill}%
\pgfsetlinewidth{0.000000pt}%
\definecolor{currentstroke}{rgb}{0.000000,0.000000,0.000000}%
\pgfsetstrokecolor{currentstroke}%
\pgfsetdash{}{0pt}%
\pgfpathmoveto{\pgfqpoint{1.990422in}{0.733594in}}%
\pgfpathlineto{\pgfqpoint{2.026647in}{0.739112in}}%
\pgfpathlineto{\pgfqpoint{1.988764in}{0.773615in}}%
\pgfpathlineto{\pgfqpoint{1.952549in}{0.771130in}}%
\pgfpathclose%
\pgfusepath{fill}%
\end{pgfscope}%
\begin{pgfscope}%
\pgfpathrectangle{\pgfqpoint{0.150000in}{0.150000in}}{\pgfqpoint{2.700000in}{1.950000in}}%
\pgfusepath{clip}%
\pgfsetbuttcap%
\pgfsetroundjoin%
\definecolor{currentfill}{rgb}{0.797871,0.822763,0.857613}%
\pgfsetfillcolor{currentfill}%
\pgfsetlinewidth{0.000000pt}%
\definecolor{currentstroke}{rgb}{0.000000,0.000000,0.000000}%
\pgfsetstrokecolor{currentstroke}%
\pgfsetdash{}{0pt}%
\pgfpathmoveto{\pgfqpoint{1.498638in}{1.164222in}}%
\pgfpathlineto{\pgfqpoint{1.536486in}{1.162291in}}%
\pgfpathlineto{\pgfqpoint{1.498798in}{1.196625in}}%
\pgfpathlineto{\pgfqpoint{1.460820in}{1.198688in}}%
\pgfpathclose%
\pgfusepath{fill}%
\end{pgfscope}%
\begin{pgfscope}%
\pgfpathrectangle{\pgfqpoint{0.150000in}{0.150000in}}{\pgfqpoint{2.700000in}{1.950000in}}%
\pgfusepath{clip}%
\pgfsetbuttcap%
\pgfsetroundjoin%
\definecolor{currentfill}{rgb}{0.822748,0.844577,0.875138}%
\pgfsetfillcolor{currentfill}%
\pgfsetlinewidth{0.000000pt}%
\definecolor{currentstroke}{rgb}{0.000000,0.000000,0.000000}%
\pgfsetstrokecolor{currentstroke}%
\pgfsetdash{}{0pt}%
\pgfpathmoveto{\pgfqpoint{1.536486in}{1.126705in}}%
\pgfpathlineto{\pgfqpoint{1.574204in}{1.127930in}}%
\pgfpathlineto{\pgfqpoint{1.536486in}{1.162291in}}%
\pgfpathlineto{\pgfqpoint{1.498638in}{1.164222in}}%
\pgfpathclose%
\pgfusepath{fill}%
\end{pgfscope}%
\begin{pgfscope}%
\pgfpathrectangle{\pgfqpoint{0.150000in}{0.150000in}}{\pgfqpoint{2.700000in}{1.950000in}}%
\pgfusepath{clip}%
\pgfsetbuttcap%
\pgfsetroundjoin%
\definecolor{currentfill}{rgb}{0.853845,0.871844,0.897044}%
\pgfsetfillcolor{currentfill}%
\pgfsetlinewidth{0.000000pt}%
\definecolor{currentstroke}{rgb}{0.000000,0.000000,0.000000}%
\pgfsetstrokecolor{currentstroke}%
\pgfsetdash{}{0pt}%
\pgfpathmoveto{\pgfqpoint{1.574354in}{1.092194in}}%
\pgfpathlineto{\pgfqpoint{1.611931in}{1.090538in}}%
\pgfpathlineto{\pgfqpoint{1.574204in}{1.127930in}}%
\pgfpathlineto{\pgfqpoint{1.536486in}{1.126705in}}%
\pgfpathclose%
\pgfusepath{fill}%
\end{pgfscope}%
\begin{pgfscope}%
\pgfpathrectangle{\pgfqpoint{0.150000in}{0.150000in}}{\pgfqpoint{2.700000in}{1.950000in}}%
\pgfusepath{clip}%
\pgfsetbuttcap%
\pgfsetroundjoin%
\definecolor{currentfill}{rgb}{0.878722,0.893658,0.914568}%
\pgfsetfillcolor{currentfill}%
\pgfsetlinewidth{0.000000pt}%
\definecolor{currentstroke}{rgb}{0.000000,0.000000,0.000000}%
\pgfsetstrokecolor{currentstroke}%
\pgfsetdash{}{0pt}%
\pgfpathmoveto{\pgfqpoint{1.612229in}{1.054649in}}%
\pgfpathlineto{\pgfqpoint{1.649697in}{1.056131in}}%
\pgfpathlineto{\pgfqpoint{1.611931in}{1.090538in}}%
\pgfpathlineto{\pgfqpoint{1.574354in}{1.092194in}}%
\pgfpathclose%
\pgfusepath{fill}%
\end{pgfscope}%
\begin{pgfscope}%
\pgfpathrectangle{\pgfqpoint{0.150000in}{0.150000in}}{\pgfqpoint{2.700000in}{1.950000in}}%
\pgfusepath{clip}%
\pgfsetbuttcap%
\pgfsetroundjoin%
\definecolor{currentfill}{rgb}{0.430928,0.501011,0.599127}%
\pgfsetfillcolor{currentfill}%
\pgfsetlinewidth{0.000000pt}%
\definecolor{currentstroke}{rgb}{0.000000,0.000000,0.000000}%
\pgfsetstrokecolor{currentstroke}%
\pgfsetdash{}{0pt}%
\pgfpathmoveto{\pgfqpoint{1.006177in}{1.598472in}}%
\pgfpathlineto{\pgfqpoint{1.045945in}{1.594874in}}%
\pgfpathlineto{\pgfqpoint{1.008420in}{1.629066in}}%
\pgfpathlineto{\pgfqpoint{0.968360in}{1.635966in}}%
\pgfpathclose%
\pgfusepath{fill}%
\end{pgfscope}%
\begin{pgfscope}%
\pgfpathrectangle{\pgfqpoint{0.150000in}{0.150000in}}{\pgfqpoint{2.700000in}{1.950000in}}%
\pgfusepath{clip}%
\pgfsetbuttcap%
\pgfsetroundjoin%
\definecolor{currentfill}{rgb}{0.909819,0.920925,0.936474}%
\pgfsetfillcolor{currentfill}%
\pgfsetlinewidth{0.000000pt}%
\definecolor{currentstroke}{rgb}{0.000000,0.000000,0.000000}%
\pgfsetstrokecolor{currentstroke}%
\pgfsetdash{}{0pt}%
\pgfpathmoveto{\pgfqpoint{1.650145in}{1.020092in}}%
\pgfpathlineto{\pgfqpoint{1.687451in}{1.018712in}}%
\pgfpathlineto{\pgfqpoint{1.649697in}{1.056131in}}%
\pgfpathlineto{\pgfqpoint{1.612229in}{1.054649in}}%
\pgfpathclose%
\pgfusepath{fill}%
\end{pgfscope}%
\begin{pgfscope}%
\pgfpathrectangle{\pgfqpoint{0.150000in}{0.150000in}}{\pgfqpoint{2.700000in}{1.950000in}}%
\pgfusepath{clip}%
\pgfsetbuttcap%
\pgfsetroundjoin%
\definecolor{currentfill}{rgb}{0.455806,0.522825,0.616651}%
\pgfsetfillcolor{currentfill}%
\pgfsetlinewidth{0.000000pt}%
\definecolor{currentstroke}{rgb}{0.000000,0.000000,0.000000}%
\pgfsetstrokecolor{currentstroke}%
\pgfsetdash{}{0pt}%
\pgfpathmoveto{\pgfqpoint{1.043861in}{1.564124in}}%
\pgfpathlineto{\pgfqpoint{1.083630in}{1.557516in}}%
\pgfpathlineto{\pgfqpoint{1.045945in}{1.594874in}}%
\pgfpathlineto{\pgfqpoint{1.006177in}{1.598472in}}%
\pgfpathclose%
\pgfusepath{fill}%
\end{pgfscope}%
\begin{pgfscope}%
\pgfpathrectangle{\pgfqpoint{0.150000in}{0.150000in}}{\pgfqpoint{2.700000in}{1.950000in}}%
\pgfusepath{clip}%
\pgfsetbuttcap%
\pgfsetroundjoin%
\definecolor{currentfill}{rgb}{0.934697,0.942739,0.953998}%
\pgfsetfillcolor{currentfill}%
\pgfsetlinewidth{0.000000pt}%
\definecolor{currentstroke}{rgb}{0.000000,0.000000,0.000000}%
\pgfsetstrokecolor{currentstroke}%
\pgfsetdash{}{0pt}%
\pgfpathmoveto{\pgfqpoint{1.688048in}{0.982521in}}%
\pgfpathlineto{\pgfqpoint{1.725266in}{0.984260in}}%
\pgfpathlineto{\pgfqpoint{1.687451in}{1.018712in}}%
\pgfpathlineto{\pgfqpoint{1.650145in}{1.020092in}}%
\pgfpathclose%
\pgfusepath{fill}%
\end{pgfscope}%
\begin{pgfscope}%
\pgfpathrectangle{\pgfqpoint{0.150000in}{0.150000in}}{\pgfqpoint{2.700000in}{1.950000in}}%
\pgfusepath{clip}%
\pgfsetbuttcap%
\pgfsetroundjoin%
\definecolor{currentfill}{rgb}{0.486903,0.550092,0.638557}%
\pgfsetfillcolor{currentfill}%
\pgfsetlinewidth{0.000000pt}%
\definecolor{currentstroke}{rgb}{0.000000,0.000000,0.000000}%
\pgfsetstrokecolor{currentstroke}%
\pgfsetdash{}{0pt}%
\pgfpathmoveto{\pgfqpoint{1.081706in}{1.526603in}}%
\pgfpathlineto{\pgfqpoint{1.121204in}{1.523279in}}%
\pgfpathlineto{\pgfqpoint{1.083630in}{1.557516in}}%
\pgfpathlineto{\pgfqpoint{1.043861in}{1.564124in}}%
\pgfpathclose%
\pgfusepath{fill}%
\end{pgfscope}%
\begin{pgfscope}%
\pgfpathrectangle{\pgfqpoint{0.150000in}{0.150000in}}{\pgfqpoint{2.700000in}{1.950000in}}%
\pgfusepath{clip}%
\pgfsetbuttcap%
\pgfsetroundjoin%
\definecolor{currentfill}{rgb}{0.511780,0.571906,0.656081}%
\pgfsetfillcolor{currentfill}%
\pgfsetlinewidth{0.000000pt}%
\definecolor{currentstroke}{rgb}{0.000000,0.000000,0.000000}%
\pgfsetstrokecolor{currentstroke}%
\pgfsetdash{}{0pt}%
\pgfpathmoveto{\pgfqpoint{1.119439in}{1.492209in}}%
\pgfpathlineto{\pgfqpoint{1.158916in}{1.485895in}}%
\pgfpathlineto{\pgfqpoint{1.121204in}{1.523279in}}%
\pgfpathlineto{\pgfqpoint{1.081706in}{1.526603in}}%
\pgfpathclose%
\pgfusepath{fill}%
\end{pgfscope}%
\begin{pgfscope}%
\pgfpathrectangle{\pgfqpoint{0.150000in}{0.150000in}}{\pgfqpoint{2.700000in}{1.950000in}}%
\pgfusepath{clip}%
\pgfsetbuttcap%
\pgfsetroundjoin%
\definecolor{currentfill}{rgb}{0.959574,0.964553,0.971523}%
\pgfsetfillcolor{currentfill}%
\pgfsetlinewidth{0.000000pt}%
\definecolor{currentstroke}{rgb}{0.000000,0.000000,0.000000}%
\pgfsetstrokecolor{currentstroke}%
\pgfsetdash{}{0pt}%
\pgfpathmoveto{\pgfqpoint{1.726014in}{0.947918in}}%
\pgfpathlineto{\pgfqpoint{1.763112in}{0.949780in}}%
\pgfpathlineto{\pgfqpoint{1.725266in}{0.984260in}}%
\pgfpathlineto{\pgfqpoint{1.688048in}{0.982521in}}%
\pgfpathclose%
\pgfusepath{fill}%
\end{pgfscope}%
\begin{pgfscope}%
\pgfpathrectangle{\pgfqpoint{0.150000in}{0.150000in}}{\pgfqpoint{2.700000in}{1.950000in}}%
\pgfusepath{clip}%
\pgfsetbuttcap%
\pgfsetroundjoin%
\definecolor{currentfill}{rgb}{0.542877,0.599173,0.677987}%
\pgfsetfillcolor{currentfill}%
\pgfsetlinewidth{0.000000pt}%
\definecolor{currentstroke}{rgb}{0.000000,0.000000,0.000000}%
\pgfsetstrokecolor{currentstroke}%
\pgfsetdash{}{0pt}%
\pgfpathmoveto{\pgfqpoint{1.157312in}{1.454662in}}%
\pgfpathlineto{\pgfqpoint{1.196539in}{1.451613in}}%
\pgfpathlineto{\pgfqpoint{1.158916in}{1.485895in}}%
\pgfpathlineto{\pgfqpoint{1.119439in}{1.492209in}}%
\pgfpathclose%
\pgfusepath{fill}%
\end{pgfscope}%
\begin{pgfscope}%
\pgfpathrectangle{\pgfqpoint{0.150000in}{0.150000in}}{\pgfqpoint{2.700000in}{1.950000in}}%
\pgfusepath{clip}%
\pgfsetbuttcap%
\pgfsetroundjoin%
\definecolor{currentfill}{rgb}{0.990671,0.991820,0.993428}%
\pgfsetfillcolor{currentfill}%
\pgfsetlinewidth{0.000000pt}%
\definecolor{currentstroke}{rgb}{0.000000,0.000000,0.000000}%
\pgfsetstrokecolor{currentstroke}%
\pgfsetdash{}{0pt}%
\pgfpathmoveto{\pgfqpoint{1.763944in}{0.910320in}}%
\pgfpathlineto{\pgfqpoint{1.800911in}{0.912316in}}%
\pgfpathlineto{\pgfqpoint{1.763112in}{0.949780in}}%
\pgfpathlineto{\pgfqpoint{1.726014in}{0.947918in}}%
\pgfpathclose%
\pgfusepath{fill}%
\end{pgfscope}%
\begin{pgfscope}%
\pgfpathrectangle{\pgfqpoint{0.150000in}{0.150000in}}{\pgfqpoint{2.700000in}{1.950000in}}%
\pgfusepath{clip}%
\pgfsetbuttcap%
\pgfsetroundjoin%
\definecolor{currentfill}{rgb}{0.567754,0.620987,0.695512}%
\pgfsetfillcolor{currentfill}%
\pgfsetlinewidth{0.000000pt}%
\definecolor{currentstroke}{rgb}{0.000000,0.000000,0.000000}%
\pgfsetstrokecolor{currentstroke}%
\pgfsetdash{}{0pt}%
\pgfpathmoveto{\pgfqpoint{1.195192in}{1.417107in}}%
\pgfpathlineto{\pgfqpoint{1.234278in}{1.414202in}}%
\pgfpathlineto{\pgfqpoint{1.196539in}{1.451613in}}%
\pgfpathlineto{\pgfqpoint{1.157312in}{1.454662in}}%
\pgfpathclose%
\pgfusepath{fill}%
\end{pgfscope}%
\begin{pgfscope}%
\pgfpathrectangle{\pgfqpoint{0.150000in}{0.150000in}}{\pgfqpoint{2.700000in}{1.950000in}}%
\pgfusepath{clip}%
\pgfsetbuttcap%
\pgfsetroundjoin%
\definecolor{currentfill}{rgb}{0.990502,0.982767,0.983379}%
\pgfsetfillcolor{currentfill}%
\pgfsetlinewidth{0.000000pt}%
\definecolor{currentstroke}{rgb}{0.000000,0.000000,0.000000}%
\pgfsetstrokecolor{currentstroke}%
\pgfsetdash{}{0pt}%
\pgfpathmoveto{\pgfqpoint{1.801959in}{0.875670in}}%
\pgfpathlineto{\pgfqpoint{1.838805in}{0.877790in}}%
\pgfpathlineto{\pgfqpoint{1.800911in}{0.912316in}}%
\pgfpathlineto{\pgfqpoint{1.763944in}{0.910320in}}%
\pgfpathclose%
\pgfusepath{fill}%
\end{pgfscope}%
\begin{pgfscope}%
\pgfpathrectangle{\pgfqpoint{0.150000in}{0.150000in}}{\pgfqpoint{2.700000in}{1.950000in}}%
\pgfusepath{clip}%
\pgfsetbuttcap%
\pgfsetroundjoin%
\definecolor{currentfill}{rgb}{0.598851,0.648254,0.717417}%
\pgfsetfillcolor{currentfill}%
\pgfsetlinewidth{0.000000pt}%
\definecolor{currentstroke}{rgb}{0.000000,0.000000,0.000000}%
\pgfsetstrokecolor{currentstroke}%
\pgfsetdash{}{0pt}%
\pgfpathmoveto{\pgfqpoint{1.232993in}{1.382648in}}%
\pgfpathlineto{\pgfqpoint{1.271950in}{1.379874in}}%
\pgfpathlineto{\pgfqpoint{1.234278in}{1.414202in}}%
\pgfpathlineto{\pgfqpoint{1.195192in}{1.417107in}}%
\pgfpathclose%
\pgfusepath{fill}%
\end{pgfscope}%
\begin{pgfscope}%
\pgfpathrectangle{\pgfqpoint{0.150000in}{0.150000in}}{\pgfqpoint{2.700000in}{1.950000in}}%
\pgfusepath{clip}%
\pgfsetbuttcap%
\pgfsetroundjoin%
\definecolor{currentfill}{rgb}{0.971507,0.948300,0.950138}%
\pgfsetfillcolor{currentfill}%
\pgfsetlinewidth{0.000000pt}%
\definecolor{currentstroke}{rgb}{0.000000,0.000000,0.000000}%
\pgfsetstrokecolor{currentstroke}%
\pgfsetdash{}{0pt}%
\pgfpathmoveto{\pgfqpoint{1.839916in}{0.838046in}}%
\pgfpathlineto{\pgfqpoint{1.876632in}{0.840300in}}%
\pgfpathlineto{\pgfqpoint{1.838805in}{0.877790in}}%
\pgfpathlineto{\pgfqpoint{1.801959in}{0.875670in}}%
\pgfpathclose%
\pgfusepath{fill}%
\end{pgfscope}%
\begin{pgfscope}%
\pgfpathrectangle{\pgfqpoint{0.150000in}{0.150000in}}{\pgfqpoint{2.700000in}{1.950000in}}%
\pgfusepath{clip}%
\pgfsetbuttcap%
\pgfsetroundjoin%
\definecolor{currentfill}{rgb}{0.623729,0.670067,0.734942}%
\pgfsetfillcolor{currentfill}%
\pgfsetlinewidth{0.000000pt}%
\definecolor{currentstroke}{rgb}{0.000000,0.000000,0.000000}%
\pgfsetstrokecolor{currentstroke}%
\pgfsetdash{}{0pt}%
\pgfpathmoveto{\pgfqpoint{1.270901in}{1.345066in}}%
\pgfpathlineto{\pgfqpoint{1.309716in}{1.342436in}}%
\pgfpathlineto{\pgfqpoint{1.271950in}{1.379874in}}%
\pgfpathlineto{\pgfqpoint{1.232993in}{1.382648in}}%
\pgfpathclose%
\pgfusepath{fill}%
\end{pgfscope}%
\begin{pgfscope}%
\pgfpathrectangle{\pgfqpoint{0.150000in}{0.150000in}}{\pgfqpoint{2.700000in}{1.950000in}}%
\pgfusepath{clip}%
\pgfsetbuttcap%
\pgfsetroundjoin%
\definecolor{currentfill}{rgb}{0.956311,0.920726,0.923545}%
\pgfsetfillcolor{currentfill}%
\pgfsetlinewidth{0.000000pt}%
\definecolor{currentstroke}{rgb}{0.000000,0.000000,0.000000}%
\pgfsetstrokecolor{currentstroke}%
\pgfsetdash{}{0pt}%
\pgfpathmoveto{\pgfqpoint{1.877981in}{0.803350in}}%
\pgfpathlineto{\pgfqpoint{1.914575in}{0.805728in}}%
\pgfpathlineto{\pgfqpoint{1.876632in}{0.840300in}}%
\pgfpathlineto{\pgfqpoint{1.839916in}{0.838046in}}%
\pgfpathclose%
\pgfusepath{fill}%
\end{pgfscope}%
\begin{pgfscope}%
\pgfpathrectangle{\pgfqpoint{0.150000in}{0.150000in}}{\pgfqpoint{2.700000in}{1.950000in}}%
\pgfusepath{clip}%
\pgfsetbuttcap%
\pgfsetroundjoin%
\definecolor{currentfill}{rgb}{0.654825,0.697335,0.756847}%
\pgfsetfillcolor{currentfill}%
\pgfsetlinewidth{0.000000pt}%
\definecolor{currentstroke}{rgb}{0.000000,0.000000,0.000000}%
\pgfsetstrokecolor{currentstroke}%
\pgfsetdash{}{0pt}%
\pgfpathmoveto{\pgfqpoint{1.308751in}{1.310561in}}%
\pgfpathlineto{\pgfqpoint{1.347437in}{1.308062in}}%
\pgfpathlineto{\pgfqpoint{1.309716in}{1.342436in}}%
\pgfpathlineto{\pgfqpoint{1.270901in}{1.345066in}}%
\pgfpathclose%
\pgfusepath{fill}%
\end{pgfscope}%
\begin{pgfscope}%
\pgfpathrectangle{\pgfqpoint{0.150000in}{0.150000in}}{\pgfqpoint{2.700000in}{1.950000in}}%
\pgfusepath{clip}%
\pgfsetbuttcap%
\pgfsetroundjoin%
\definecolor{currentfill}{rgb}{0.937316,0.886259,0.890303}%
\pgfsetfillcolor{currentfill}%
\pgfsetlinewidth{0.000000pt}%
\definecolor{currentstroke}{rgb}{0.000000,0.000000,0.000000}%
\pgfsetstrokecolor{currentstroke}%
\pgfsetdash{}{0pt}%
\pgfpathmoveto{\pgfqpoint{1.915966in}{0.765698in}}%
\pgfpathlineto{\pgfqpoint{1.952549in}{0.771130in}}%
\pgfpathlineto{\pgfqpoint{1.914575in}{0.805728in}}%
\pgfpathlineto{\pgfqpoint{1.877981in}{0.803350in}}%
\pgfpathclose%
\pgfusepath{fill}%
\end{pgfscope}%
\begin{pgfscope}%
\pgfpathrectangle{\pgfqpoint{0.150000in}{0.150000in}}{\pgfqpoint{2.700000in}{1.950000in}}%
\pgfusepath{clip}%
\pgfsetbuttcap%
\pgfsetroundjoin%
\definecolor{currentfill}{rgb}{0.679703,0.719148,0.774372}%
\pgfsetfillcolor{currentfill}%
\pgfsetlinewidth{0.000000pt}%
\definecolor{currentstroke}{rgb}{0.000000,0.000000,0.000000}%
\pgfsetstrokecolor{currentstroke}%
\pgfsetdash{}{0pt}%
\pgfpathmoveto{\pgfqpoint{1.346687in}{1.272953in}}%
\pgfpathlineto{\pgfqpoint{1.385230in}{1.270598in}}%
\pgfpathlineto{\pgfqpoint{1.347437in}{1.308062in}}%
\pgfpathlineto{\pgfqpoint{1.308751in}{1.310561in}}%
\pgfpathclose%
\pgfusepath{fill}%
\end{pgfscope}%
\begin{pgfscope}%
\pgfpathrectangle{\pgfqpoint{0.150000in}{0.150000in}}{\pgfqpoint{2.700000in}{1.950000in}}%
\pgfusepath{clip}%
\pgfsetbuttcap%
\pgfsetroundjoin%
\definecolor{currentfill}{rgb}{0.922120,0.858686,0.863710}%
\pgfsetfillcolor{currentfill}%
\pgfsetlinewidth{0.000000pt}%
\definecolor{currentstroke}{rgb}{0.000000,0.000000,0.000000}%
\pgfsetstrokecolor{currentstroke}%
\pgfsetdash{}{0pt}%
\pgfpathmoveto{\pgfqpoint{1.954079in}{0.730957in}}%
\pgfpathlineto{\pgfqpoint{1.990422in}{0.733594in}}%
\pgfpathlineto{\pgfqpoint{1.952549in}{0.771130in}}%
\pgfpathlineto{\pgfqpoint{1.915966in}{0.765698in}}%
\pgfpathclose%
\pgfusepath{fill}%
\end{pgfscope}%
\begin{pgfscope}%
\pgfpathrectangle{\pgfqpoint{0.150000in}{0.150000in}}{\pgfqpoint{2.700000in}{1.950000in}}%
\pgfusepath{clip}%
\pgfsetbuttcap%
\pgfsetroundjoin%
\definecolor{currentfill}{rgb}{0.710800,0.746415,0.796278}%
\pgfsetfillcolor{currentfill}%
\pgfsetlinewidth{0.000000pt}%
\definecolor{currentstroke}{rgb}{0.000000,0.000000,0.000000}%
\pgfsetstrokecolor{currentstroke}%
\pgfsetdash{}{0pt}%
\pgfpathmoveto{\pgfqpoint{1.384630in}{1.235336in}}%
\pgfpathlineto{\pgfqpoint{1.422999in}{1.236179in}}%
\pgfpathlineto{\pgfqpoint{1.385230in}{1.270598in}}%
\pgfpathlineto{\pgfqpoint{1.346687in}{1.272953in}}%
\pgfpathclose%
\pgfusepath{fill}%
\end{pgfscope}%
\begin{pgfscope}%
\pgfpathrectangle{\pgfqpoint{0.150000in}{0.150000in}}{\pgfqpoint{2.700000in}{1.950000in}}%
\pgfusepath{clip}%
\pgfsetbuttcap%
\pgfsetroundjoin%
\definecolor{currentfill}{rgb}{0.735677,0.768229,0.813802}%
\pgfsetfillcolor{currentfill}%
\pgfsetlinewidth{0.000000pt}%
\definecolor{currentstroke}{rgb}{0.000000,0.000000,0.000000}%
\pgfsetstrokecolor{currentstroke}%
\pgfsetdash{}{0pt}%
\pgfpathmoveto{\pgfqpoint{1.422549in}{1.200766in}}%
\pgfpathlineto{\pgfqpoint{1.460820in}{1.198688in}}%
\pgfpathlineto{\pgfqpoint{1.422999in}{1.236179in}}%
\pgfpathlineto{\pgfqpoint{1.384630in}{1.235336in}}%
\pgfpathclose%
\pgfusepath{fill}%
\end{pgfscope}%
\begin{pgfscope}%
\pgfpathrectangle{\pgfqpoint{0.150000in}{0.150000in}}{\pgfqpoint{2.700000in}{1.950000in}}%
\pgfusepath{clip}%
\pgfsetbuttcap%
\pgfsetroundjoin%
\definecolor{currentfill}{rgb}{0.766774,0.795496,0.835708}%
\pgfsetfillcolor{currentfill}%
\pgfsetlinewidth{0.000000pt}%
\definecolor{currentstroke}{rgb}{0.000000,0.000000,0.000000}%
\pgfsetstrokecolor{currentstroke}%
\pgfsetdash{}{0pt}%
\pgfpathmoveto{\pgfqpoint{1.460520in}{1.163122in}}%
\pgfpathlineto{\pgfqpoint{1.498638in}{1.164222in}}%
\pgfpathlineto{\pgfqpoint{1.460820in}{1.198688in}}%
\pgfpathlineto{\pgfqpoint{1.422549in}{1.200766in}}%
\pgfpathclose%
\pgfusepath{fill}%
\end{pgfscope}%
\begin{pgfscope}%
\pgfpathrectangle{\pgfqpoint{0.150000in}{0.150000in}}{\pgfqpoint{2.700000in}{1.950000in}}%
\pgfusepath{clip}%
\pgfsetbuttcap%
\pgfsetroundjoin%
\definecolor{currentfill}{rgb}{0.791651,0.817310,0.853232}%
\pgfsetfillcolor{currentfill}%
\pgfsetlinewidth{0.000000pt}%
\definecolor{currentstroke}{rgb}{0.000000,0.000000,0.000000}%
\pgfsetstrokecolor{currentstroke}%
\pgfsetdash{}{0pt}%
\pgfpathmoveto{\pgfqpoint{1.498488in}{1.128506in}}%
\pgfpathlineto{\pgfqpoint{1.536486in}{1.126705in}}%
\pgfpathlineto{\pgfqpoint{1.498638in}{1.164222in}}%
\pgfpathlineto{\pgfqpoint{1.460520in}{1.163122in}}%
\pgfpathclose%
\pgfusepath{fill}%
\end{pgfscope}%
\begin{pgfscope}%
\pgfpathrectangle{\pgfqpoint{0.150000in}{0.150000in}}{\pgfqpoint{2.700000in}{1.950000in}}%
\pgfusepath{clip}%
\pgfsetbuttcap%
\pgfsetroundjoin%
\definecolor{currentfill}{rgb}{0.822748,0.844577,0.875138}%
\pgfsetfillcolor{currentfill}%
\pgfsetlinewidth{0.000000pt}%
\definecolor{currentstroke}{rgb}{0.000000,0.000000,0.000000}%
\pgfsetstrokecolor{currentstroke}%
\pgfsetdash{}{0pt}%
\pgfpathmoveto{\pgfqpoint{1.536486in}{1.090835in}}%
\pgfpathlineto{\pgfqpoint{1.574354in}{1.092194in}}%
\pgfpathlineto{\pgfqpoint{1.536486in}{1.126705in}}%
\pgfpathlineto{\pgfqpoint{1.498488in}{1.128506in}}%
\pgfpathclose%
\pgfusepath{fill}%
\end{pgfscope}%
\begin{pgfscope}%
\pgfpathrectangle{\pgfqpoint{0.150000in}{0.150000in}}{\pgfqpoint{2.700000in}{1.950000in}}%
\pgfusepath{clip}%
\pgfsetbuttcap%
\pgfsetroundjoin%
\definecolor{currentfill}{rgb}{0.847626,0.866391,0.892662}%
\pgfsetfillcolor{currentfill}%
\pgfsetlinewidth{0.000000pt}%
\definecolor{currentstroke}{rgb}{0.000000,0.000000,0.000000}%
\pgfsetstrokecolor{currentstroke}%
\pgfsetdash{}{0pt}%
\pgfpathmoveto{\pgfqpoint{1.574493in}{1.053157in}}%
\pgfpathlineto{\pgfqpoint{1.612229in}{1.054649in}}%
\pgfpathlineto{\pgfqpoint{1.574354in}{1.092194in}}%
\pgfpathlineto{\pgfqpoint{1.536486in}{1.090835in}}%
\pgfpathclose%
\pgfusepath{fill}%
\end{pgfscope}%
\begin{pgfscope}%
\pgfpathrectangle{\pgfqpoint{0.150000in}{0.150000in}}{\pgfqpoint{2.700000in}{1.950000in}}%
\pgfusepath{clip}%
\pgfsetbuttcap%
\pgfsetroundjoin%
\definecolor{currentfill}{rgb}{0.878722,0.893658,0.914568}%
\pgfsetfillcolor{currentfill}%
\pgfsetlinewidth{0.000000pt}%
\definecolor{currentstroke}{rgb}{0.000000,0.000000,0.000000}%
\pgfsetstrokecolor{currentstroke}%
\pgfsetdash{}{0pt}%
\pgfpathmoveto{\pgfqpoint{1.612530in}{1.018475in}}%
\pgfpathlineto{\pgfqpoint{1.650145in}{1.020092in}}%
\pgfpathlineto{\pgfqpoint{1.612229in}{1.054649in}}%
\pgfpathlineto{\pgfqpoint{1.574493in}{1.053157in}}%
\pgfpathclose%
\pgfusepath{fill}%
\end{pgfscope}%
\begin{pgfscope}%
\pgfpathrectangle{\pgfqpoint{0.150000in}{0.150000in}}{\pgfqpoint{2.700000in}{1.950000in}}%
\pgfusepath{clip}%
\pgfsetbuttcap%
\pgfsetroundjoin%
\definecolor{currentfill}{rgb}{0.387393,0.462837,0.568459}%
\pgfsetfillcolor{currentfill}%
\pgfsetlinewidth{0.000000pt}%
\definecolor{currentstroke}{rgb}{0.000000,0.000000,0.000000}%
\pgfsetstrokecolor{currentstroke}%
\pgfsetdash{}{0pt}%
\pgfpathmoveto{\pgfqpoint{0.965937in}{1.605274in}}%
\pgfpathlineto{\pgfqpoint{1.006177in}{1.598472in}}%
\pgfpathlineto{\pgfqpoint{0.968360in}{1.635966in}}%
\pgfpathlineto{\pgfqpoint{0.927967in}{1.642923in}}%
\pgfpathclose%
\pgfusepath{fill}%
\end{pgfscope}%
\begin{pgfscope}%
\pgfpathrectangle{\pgfqpoint{0.150000in}{0.150000in}}{\pgfqpoint{2.700000in}{1.950000in}}%
\pgfusepath{clip}%
\pgfsetbuttcap%
\pgfsetroundjoin%
\definecolor{currentfill}{rgb}{0.903600,0.915472,0.932093}%
\pgfsetfillcolor{currentfill}%
\pgfsetlinewidth{0.000000pt}%
\definecolor{currentstroke}{rgb}{0.000000,0.000000,0.000000}%
\pgfsetstrokecolor{currentstroke}%
\pgfsetdash{}{0pt}%
\pgfpathmoveto{\pgfqpoint{1.650564in}{0.980770in}}%
\pgfpathlineto{\pgfqpoint{1.688048in}{0.982521in}}%
\pgfpathlineto{\pgfqpoint{1.650145in}{1.020092in}}%
\pgfpathlineto{\pgfqpoint{1.612530in}{1.018475in}}%
\pgfpathclose%
\pgfusepath{fill}%
\end{pgfscope}%
\begin{pgfscope}%
\pgfpathrectangle{\pgfqpoint{0.150000in}{0.150000in}}{\pgfqpoint{2.700000in}{1.950000in}}%
\pgfusepath{clip}%
\pgfsetbuttcap%
\pgfsetroundjoin%
\definecolor{currentfill}{rgb}{0.418490,0.490104,0.590365}%
\pgfsetfillcolor{currentfill}%
\pgfsetlinewidth{0.000000pt}%
\definecolor{currentstroke}{rgb}{0.000000,0.000000,0.000000}%
\pgfsetstrokecolor{currentstroke}%
\pgfsetdash{}{0pt}%
\pgfpathmoveto{\pgfqpoint{1.003762in}{1.570786in}}%
\pgfpathlineto{\pgfqpoint{1.043861in}{1.564124in}}%
\pgfpathlineto{\pgfqpoint{1.006177in}{1.598472in}}%
\pgfpathlineto{\pgfqpoint{0.965937in}{1.605274in}}%
\pgfpathclose%
\pgfusepath{fill}%
\end{pgfscope}%
\begin{pgfscope}%
\pgfpathrectangle{\pgfqpoint{0.150000in}{0.150000in}}{\pgfqpoint{2.700000in}{1.950000in}}%
\pgfusepath{clip}%
\pgfsetbuttcap%
\pgfsetroundjoin%
\definecolor{currentfill}{rgb}{0.934697,0.942739,0.953998}%
\pgfsetfillcolor{currentfill}%
\pgfsetlinewidth{0.000000pt}%
\definecolor{currentstroke}{rgb}{0.000000,0.000000,0.000000}%
\pgfsetstrokecolor{currentstroke}%
\pgfsetdash{}{0pt}%
\pgfpathmoveto{\pgfqpoint{1.688651in}{0.946042in}}%
\pgfpathlineto{\pgfqpoint{1.726014in}{0.947918in}}%
\pgfpathlineto{\pgfqpoint{1.688048in}{0.982521in}}%
\pgfpathlineto{\pgfqpoint{1.650564in}{0.980770in}}%
\pgfpathclose%
\pgfusepath{fill}%
\end{pgfscope}%
\begin{pgfscope}%
\pgfpathrectangle{\pgfqpoint{0.150000in}{0.150000in}}{\pgfqpoint{2.700000in}{1.950000in}}%
\pgfusepath{clip}%
\pgfsetbuttcap%
\pgfsetroundjoin%
\definecolor{currentfill}{rgb}{0.443367,0.511918,0.607889}%
\pgfsetfillcolor{currentfill}%
\pgfsetlinewidth{0.000000pt}%
\definecolor{currentstroke}{rgb}{0.000000,0.000000,0.000000}%
\pgfsetstrokecolor{currentstroke}%
\pgfsetdash{}{0pt}%
\pgfpathmoveto{\pgfqpoint{1.041760in}{1.533110in}}%
\pgfpathlineto{\pgfqpoint{1.081706in}{1.526603in}}%
\pgfpathlineto{\pgfqpoint{1.043861in}{1.564124in}}%
\pgfpathlineto{\pgfqpoint{1.003762in}{1.570786in}}%
\pgfpathclose%
\pgfusepath{fill}%
\end{pgfscope}%
\begin{pgfscope}%
\pgfpathrectangle{\pgfqpoint{0.150000in}{0.150000in}}{\pgfqpoint{2.700000in}{1.950000in}}%
\pgfusepath{clip}%
\pgfsetbuttcap%
\pgfsetroundjoin%
\definecolor{currentfill}{rgb}{0.959574,0.964553,0.971523}%
\pgfsetfillcolor{currentfill}%
\pgfsetlinewidth{0.000000pt}%
\definecolor{currentstroke}{rgb}{0.000000,0.000000,0.000000}%
\pgfsetstrokecolor{currentstroke}%
\pgfsetdash{}{0pt}%
\pgfpathmoveto{\pgfqpoint{1.726713in}{0.908309in}}%
\pgfpathlineto{\pgfqpoint{1.763944in}{0.910320in}}%
\pgfpathlineto{\pgfqpoint{1.726014in}{0.947918in}}%
\pgfpathlineto{\pgfqpoint{1.688651in}{0.946042in}}%
\pgfpathclose%
\pgfusepath{fill}%
\end{pgfscope}%
\begin{pgfscope}%
\pgfpathrectangle{\pgfqpoint{0.150000in}{0.150000in}}{\pgfqpoint{2.700000in}{1.950000in}}%
\pgfusepath{clip}%
\pgfsetbuttcap%
\pgfsetroundjoin%
\definecolor{currentfill}{rgb}{0.474464,0.539185,0.629795}%
\pgfsetfillcolor{currentfill}%
\pgfsetlinewidth{0.000000pt}%
\definecolor{currentstroke}{rgb}{0.000000,0.000000,0.000000}%
\pgfsetstrokecolor{currentstroke}%
\pgfsetdash{}{0pt}%
\pgfpathmoveto{\pgfqpoint{1.079766in}{1.495427in}}%
\pgfpathlineto{\pgfqpoint{1.119439in}{1.492209in}}%
\pgfpathlineto{\pgfqpoint{1.081706in}{1.526603in}}%
\pgfpathlineto{\pgfqpoint{1.041760in}{1.533110in}}%
\pgfpathclose%
\pgfusepath{fill}%
\end{pgfscope}%
\begin{pgfscope}%
\pgfpathrectangle{\pgfqpoint{0.150000in}{0.150000in}}{\pgfqpoint{2.700000in}{1.950000in}}%
\pgfusepath{clip}%
\pgfsetbuttcap%
\pgfsetroundjoin%
\definecolor{currentfill}{rgb}{0.990671,0.991820,0.993428}%
\pgfsetfillcolor{currentfill}%
\pgfsetlinewidth{0.000000pt}%
\definecolor{currentstroke}{rgb}{0.000000,0.000000,0.000000}%
\pgfsetstrokecolor{currentstroke}%
\pgfsetdash{}{0pt}%
\pgfpathmoveto{\pgfqpoint{1.764783in}{0.870568in}}%
\pgfpathlineto{\pgfqpoint{1.801959in}{0.875670in}}%
\pgfpathlineto{\pgfqpoint{1.763944in}{0.910320in}}%
\pgfpathlineto{\pgfqpoint{1.726713in}{0.908309in}}%
\pgfpathclose%
\pgfusepath{fill}%
\end{pgfscope}%
\begin{pgfscope}%
\pgfpathrectangle{\pgfqpoint{0.150000in}{0.150000in}}{\pgfqpoint{2.700000in}{1.950000in}}%
\pgfusepath{clip}%
\pgfsetbuttcap%
\pgfsetroundjoin%
\definecolor{currentfill}{rgb}{0.505561,0.566452,0.651700}%
\pgfsetfillcolor{currentfill}%
\pgfsetlinewidth{0.000000pt}%
\definecolor{currentstroke}{rgb}{0.000000,0.000000,0.000000}%
\pgfsetstrokecolor{currentstroke}%
\pgfsetdash{}{0pt}%
\pgfpathmoveto{\pgfqpoint{1.117780in}{1.457735in}}%
\pgfpathlineto{\pgfqpoint{1.157312in}{1.454662in}}%
\pgfpathlineto{\pgfqpoint{1.119439in}{1.492209in}}%
\pgfpathlineto{\pgfqpoint{1.079766in}{1.495427in}}%
\pgfpathclose%
\pgfusepath{fill}%
\end{pgfscope}%
\begin{pgfscope}%
\pgfpathrectangle{\pgfqpoint{0.150000in}{0.150000in}}{\pgfqpoint{2.700000in}{1.950000in}}%
\pgfusepath{clip}%
\pgfsetbuttcap%
\pgfsetroundjoin%
\definecolor{currentfill}{rgb}{0.990502,0.982767,0.983379}%
\pgfsetfillcolor{currentfill}%
\pgfsetlinewidth{0.000000pt}%
\definecolor{currentstroke}{rgb}{0.000000,0.000000,0.000000}%
\pgfsetstrokecolor{currentstroke}%
\pgfsetdash{}{0pt}%
\pgfpathmoveto{\pgfqpoint{1.802938in}{0.835775in}}%
\pgfpathlineto{\pgfqpoint{1.839916in}{0.838046in}}%
\pgfpathlineto{\pgfqpoint{1.801959in}{0.875670in}}%
\pgfpathlineto{\pgfqpoint{1.764783in}{0.870568in}}%
\pgfpathclose%
\pgfusepath{fill}%
\end{pgfscope}%
\begin{pgfscope}%
\pgfpathrectangle{\pgfqpoint{0.150000in}{0.150000in}}{\pgfqpoint{2.700000in}{1.950000in}}%
\pgfusepath{clip}%
\pgfsetbuttcap%
\pgfsetroundjoin%
\definecolor{currentfill}{rgb}{0.530438,0.588266,0.669225}%
\pgfsetfillcolor{currentfill}%
\pgfsetlinewidth{0.000000pt}%
\definecolor{currentstroke}{rgb}{0.000000,0.000000,0.000000}%
\pgfsetstrokecolor{currentstroke}%
\pgfsetdash{}{0pt}%
\pgfpathmoveto{\pgfqpoint{1.155693in}{1.423163in}}%
\pgfpathlineto{\pgfqpoint{1.195192in}{1.417107in}}%
\pgfpathlineto{\pgfqpoint{1.157312in}{1.454662in}}%
\pgfpathlineto{\pgfqpoint{1.117780in}{1.457735in}}%
\pgfpathclose%
\pgfusepath{fill}%
\end{pgfscope}%
\begin{pgfscope}%
\pgfpathrectangle{\pgfqpoint{0.150000in}{0.150000in}}{\pgfqpoint{2.700000in}{1.950000in}}%
\pgfusepath{clip}%
\pgfsetbuttcap%
\pgfsetroundjoin%
\definecolor{currentfill}{rgb}{0.971507,0.948300,0.950138}%
\pgfsetfillcolor{currentfill}%
\pgfsetlinewidth{0.000000pt}%
\definecolor{currentstroke}{rgb}{0.000000,0.000000,0.000000}%
\pgfsetstrokecolor{currentstroke}%
\pgfsetdash{}{0pt}%
\pgfpathmoveto{\pgfqpoint{1.841036in}{0.798007in}}%
\pgfpathlineto{\pgfqpoint{1.877981in}{0.803350in}}%
\pgfpathlineto{\pgfqpoint{1.839916in}{0.838046in}}%
\pgfpathlineto{\pgfqpoint{1.802938in}{0.835775in}}%
\pgfpathclose%
\pgfusepath{fill}%
\end{pgfscope}%
\begin{pgfscope}%
\pgfpathrectangle{\pgfqpoint{0.150000in}{0.150000in}}{\pgfqpoint{2.700000in}{1.950000in}}%
\pgfusepath{clip}%
\pgfsetbuttcap%
\pgfsetroundjoin%
\definecolor{currentfill}{rgb}{0.561535,0.615533,0.691131}%
\pgfsetfillcolor{currentfill}%
\pgfsetlinewidth{0.000000pt}%
\definecolor{currentstroke}{rgb}{0.000000,0.000000,0.000000}%
\pgfsetstrokecolor{currentstroke}%
\pgfsetdash{}{0pt}%
\pgfpathmoveto{\pgfqpoint{1.193735in}{1.385444in}}%
\pgfpathlineto{\pgfqpoint{1.232993in}{1.382648in}}%
\pgfpathlineto{\pgfqpoint{1.195192in}{1.417107in}}%
\pgfpathlineto{\pgfqpoint{1.155693in}{1.423163in}}%
\pgfpathclose%
\pgfusepath{fill}%
\end{pgfscope}%
\begin{pgfscope}%
\pgfpathrectangle{\pgfqpoint{0.150000in}{0.150000in}}{\pgfqpoint{2.700000in}{1.950000in}}%
\pgfusepath{clip}%
\pgfsetbuttcap%
\pgfsetroundjoin%
\definecolor{currentfill}{rgb}{0.956311,0.920726,0.923545}%
\pgfsetfillcolor{currentfill}%
\pgfsetlinewidth{0.000000pt}%
\definecolor{currentstroke}{rgb}{0.000000,0.000000,0.000000}%
\pgfsetstrokecolor{currentstroke}%
\pgfsetdash{}{0pt}%
\pgfpathmoveto{\pgfqpoint{1.879241in}{0.763168in}}%
\pgfpathlineto{\pgfqpoint{1.915966in}{0.765698in}}%
\pgfpathlineto{\pgfqpoint{1.877981in}{0.803350in}}%
\pgfpathlineto{\pgfqpoint{1.841036in}{0.798007in}}%
\pgfpathclose%
\pgfusepath{fill}%
\end{pgfscope}%
\begin{pgfscope}%
\pgfpathrectangle{\pgfqpoint{0.150000in}{0.150000in}}{\pgfqpoint{2.700000in}{1.950000in}}%
\pgfusepath{clip}%
\pgfsetbuttcap%
\pgfsetroundjoin%
\definecolor{currentfill}{rgb}{0.586412,0.637347,0.708655}%
\pgfsetfillcolor{currentfill}%
\pgfsetlinewidth{0.000000pt}%
\definecolor{currentstroke}{rgb}{0.000000,0.000000,0.000000}%
\pgfsetstrokecolor{currentstroke}%
\pgfsetdash{}{0pt}%
\pgfpathmoveto{\pgfqpoint{1.231785in}{1.347717in}}%
\pgfpathlineto{\pgfqpoint{1.270901in}{1.345066in}}%
\pgfpathlineto{\pgfqpoint{1.232993in}{1.382648in}}%
\pgfpathlineto{\pgfqpoint{1.193735in}{1.385444in}}%
\pgfpathclose%
\pgfusepath{fill}%
\end{pgfscope}%
\begin{pgfscope}%
\pgfpathrectangle{\pgfqpoint{0.150000in}{0.150000in}}{\pgfqpoint{2.700000in}{1.950000in}}%
\pgfusepath{clip}%
\pgfsetbuttcap%
\pgfsetroundjoin%
\definecolor{currentfill}{rgb}{0.937316,0.886259,0.890303}%
\pgfsetfillcolor{currentfill}%
\pgfsetlinewidth{0.000000pt}%
\definecolor{currentstroke}{rgb}{0.000000,0.000000,0.000000}%
\pgfsetstrokecolor{currentstroke}%
\pgfsetdash{}{0pt}%
\pgfpathmoveto{\pgfqpoint{1.917366in}{0.725373in}}%
\pgfpathlineto{\pgfqpoint{1.954079in}{0.730957in}}%
\pgfpathlineto{\pgfqpoint{1.915966in}{0.765698in}}%
\pgfpathlineto{\pgfqpoint{1.879241in}{0.763168in}}%
\pgfpathclose%
\pgfusepath{fill}%
\end{pgfscope}%
\begin{pgfscope}%
\pgfpathrectangle{\pgfqpoint{0.150000in}{0.150000in}}{\pgfqpoint{2.700000in}{1.950000in}}%
\pgfusepath{clip}%
\pgfsetbuttcap%
\pgfsetroundjoin%
\definecolor{currentfill}{rgb}{0.617509,0.664614,0.730561}%
\pgfsetfillcolor{currentfill}%
\pgfsetlinewidth{0.000000pt}%
\definecolor{currentstroke}{rgb}{0.000000,0.000000,0.000000}%
\pgfsetstrokecolor{currentstroke}%
\pgfsetdash{}{0pt}%
\pgfpathmoveto{\pgfqpoint{1.269767in}{1.313079in}}%
\pgfpathlineto{\pgfqpoint{1.308751in}{1.310561in}}%
\pgfpathlineto{\pgfqpoint{1.270901in}{1.345066in}}%
\pgfpathlineto{\pgfqpoint{1.231785in}{1.347717in}}%
\pgfpathclose%
\pgfusepath{fill}%
\end{pgfscope}%
\begin{pgfscope}%
\pgfpathrectangle{\pgfqpoint{0.150000in}{0.150000in}}{\pgfqpoint{2.700000in}{1.950000in}}%
\pgfusepath{clip}%
\pgfsetbuttcap%
\pgfsetroundjoin%
\definecolor{currentfill}{rgb}{0.642387,0.686428,0.748085}%
\pgfsetfillcolor{currentfill}%
\pgfsetlinewidth{0.000000pt}%
\definecolor{currentstroke}{rgb}{0.000000,0.000000,0.000000}%
\pgfsetstrokecolor{currentstroke}%
\pgfsetdash{}{0pt}%
\pgfpathmoveto{\pgfqpoint{1.307845in}{1.275325in}}%
\pgfpathlineto{\pgfqpoint{1.346687in}{1.272953in}}%
\pgfpathlineto{\pgfqpoint{1.308751in}{1.310561in}}%
\pgfpathlineto{\pgfqpoint{1.269767in}{1.313079in}}%
\pgfpathclose%
\pgfusepath{fill}%
\end{pgfscope}%
\begin{pgfscope}%
\pgfpathrectangle{\pgfqpoint{0.150000in}{0.150000in}}{\pgfqpoint{2.700000in}{1.950000in}}%
\pgfusepath{clip}%
\pgfsetbuttcap%
\pgfsetroundjoin%
\definecolor{currentfill}{rgb}{0.673483,0.713695,0.769991}%
\pgfsetfillcolor{currentfill}%
\pgfsetlinewidth{0.000000pt}%
\definecolor{currentstroke}{rgb}{0.000000,0.000000,0.000000}%
\pgfsetstrokecolor{currentstroke}%
\pgfsetdash{}{0pt}%
\pgfpathmoveto{\pgfqpoint{1.345931in}{1.237563in}}%
\pgfpathlineto{\pgfqpoint{1.384630in}{1.235336in}}%
\pgfpathlineto{\pgfqpoint{1.346687in}{1.272953in}}%
\pgfpathlineto{\pgfqpoint{1.307845in}{1.275325in}}%
\pgfpathclose%
\pgfusepath{fill}%
\end{pgfscope}%
\begin{pgfscope}%
\pgfpathrectangle{\pgfqpoint{0.150000in}{0.150000in}}{\pgfqpoint{2.700000in}{1.950000in}}%
\pgfusepath{clip}%
\pgfsetbuttcap%
\pgfsetroundjoin%
\definecolor{currentfill}{rgb}{0.704580,0.740962,0.791896}%
\pgfsetfillcolor{currentfill}%
\pgfsetlinewidth{0.000000pt}%
\definecolor{currentstroke}{rgb}{0.000000,0.000000,0.000000}%
\pgfsetstrokecolor{currentstroke}%
\pgfsetdash{}{0pt}%
\pgfpathmoveto{\pgfqpoint{1.383981in}{1.202860in}}%
\pgfpathlineto{\pgfqpoint{1.422549in}{1.200766in}}%
\pgfpathlineto{\pgfqpoint{1.384630in}{1.235336in}}%
\pgfpathlineto{\pgfqpoint{1.345931in}{1.237563in}}%
\pgfpathclose%
\pgfusepath{fill}%
\end{pgfscope}%
\begin{pgfscope}%
\pgfpathrectangle{\pgfqpoint{0.150000in}{0.150000in}}{\pgfqpoint{2.700000in}{1.950000in}}%
\pgfusepath{clip}%
\pgfsetbuttcap%
\pgfsetroundjoin%
\definecolor{currentfill}{rgb}{0.729458,0.762776,0.809421}%
\pgfsetfillcolor{currentfill}%
\pgfsetlinewidth{0.000000pt}%
\definecolor{currentstroke}{rgb}{0.000000,0.000000,0.000000}%
\pgfsetstrokecolor{currentstroke}%
\pgfsetdash{}{0pt}%
\pgfpathmoveto{\pgfqpoint{1.422095in}{1.165071in}}%
\pgfpathlineto{\pgfqpoint{1.460520in}{1.163122in}}%
\pgfpathlineto{\pgfqpoint{1.422549in}{1.200766in}}%
\pgfpathlineto{\pgfqpoint{1.383981in}{1.202860in}}%
\pgfpathclose%
\pgfusepath{fill}%
\end{pgfscope}%
\begin{pgfscope}%
\pgfpathrectangle{\pgfqpoint{0.150000in}{0.150000in}}{\pgfqpoint{2.700000in}{1.950000in}}%
\pgfusepath{clip}%
\pgfsetbuttcap%
\pgfsetroundjoin%
\definecolor{currentfill}{rgb}{0.760555,0.790043,0.831327}%
\pgfsetfillcolor{currentfill}%
\pgfsetlinewidth{0.000000pt}%
\definecolor{currentstroke}{rgb}{0.000000,0.000000,0.000000}%
\pgfsetstrokecolor{currentstroke}%
\pgfsetdash{}{0pt}%
\pgfpathmoveto{\pgfqpoint{1.460217in}{1.127273in}}%
\pgfpathlineto{\pgfqpoint{1.498488in}{1.128506in}}%
\pgfpathlineto{\pgfqpoint{1.460520in}{1.163122in}}%
\pgfpathlineto{\pgfqpoint{1.422095in}{1.165071in}}%
\pgfpathclose%
\pgfusepath{fill}%
\end{pgfscope}%
\begin{pgfscope}%
\pgfpathrectangle{\pgfqpoint{0.150000in}{0.150000in}}{\pgfqpoint{2.700000in}{1.950000in}}%
\pgfusepath{clip}%
\pgfsetbuttcap%
\pgfsetroundjoin%
\definecolor{currentfill}{rgb}{0.785432,0.811857,0.848851}%
\pgfsetfillcolor{currentfill}%
\pgfsetlinewidth{0.000000pt}%
\definecolor{currentstroke}{rgb}{0.000000,0.000000,0.000000}%
\pgfsetstrokecolor{currentstroke}%
\pgfsetdash{}{0pt}%
\pgfpathmoveto{\pgfqpoint{1.498348in}{1.089467in}}%
\pgfpathlineto{\pgfqpoint{1.536486in}{1.090835in}}%
\pgfpathlineto{\pgfqpoint{1.498488in}{1.128506in}}%
\pgfpathlineto{\pgfqpoint{1.460217in}{1.127273in}}%
\pgfpathclose%
\pgfusepath{fill}%
\end{pgfscope}%
\begin{pgfscope}%
\pgfpathrectangle{\pgfqpoint{0.150000in}{0.150000in}}{\pgfqpoint{2.700000in}{1.950000in}}%
\pgfusepath{clip}%
\pgfsetbuttcap%
\pgfsetroundjoin%
\definecolor{currentfill}{rgb}{0.816529,0.839124,0.870757}%
\pgfsetfillcolor{currentfill}%
\pgfsetlinewidth{0.000000pt}%
\definecolor{currentstroke}{rgb}{0.000000,0.000000,0.000000}%
\pgfsetstrokecolor{currentstroke}%
\pgfsetdash{}{0pt}%
\pgfpathmoveto{\pgfqpoint{1.536486in}{1.054680in}}%
\pgfpathlineto{\pgfqpoint{1.574493in}{1.053157in}}%
\pgfpathlineto{\pgfqpoint{1.536486in}{1.090835in}}%
\pgfpathlineto{\pgfqpoint{1.498348in}{1.089467in}}%
\pgfpathclose%
\pgfusepath{fill}%
\end{pgfscope}%
\begin{pgfscope}%
\pgfpathrectangle{\pgfqpoint{0.150000in}{0.150000in}}{\pgfqpoint{2.700000in}{1.950000in}}%
\pgfusepath{clip}%
\pgfsetbuttcap%
\pgfsetroundjoin%
\definecolor{currentfill}{rgb}{0.847626,0.866391,0.892662}%
\pgfsetfillcolor{currentfill}%
\pgfsetlinewidth{0.000000pt}%
\definecolor{currentstroke}{rgb}{0.000000,0.000000,0.000000}%
\pgfsetstrokecolor{currentstroke}%
\pgfsetdash{}{0pt}%
\pgfpathmoveto{\pgfqpoint{1.574645in}{1.016847in}}%
\pgfpathlineto{\pgfqpoint{1.612530in}{1.018475in}}%
\pgfpathlineto{\pgfqpoint{1.574493in}{1.053157in}}%
\pgfpathlineto{\pgfqpoint{1.536486in}{1.054680in}}%
\pgfpathclose%
\pgfusepath{fill}%
\end{pgfscope}%
\begin{pgfscope}%
\pgfpathrectangle{\pgfqpoint{0.150000in}{0.150000in}}{\pgfqpoint{2.700000in}{1.950000in}}%
\pgfusepath{clip}%
\pgfsetbuttcap%
\pgfsetroundjoin%
\definecolor{currentfill}{rgb}{0.872503,0.888205,0.910187}%
\pgfsetfillcolor{currentfill}%
\pgfsetlinewidth{0.000000pt}%
\definecolor{currentstroke}{rgb}{0.000000,0.000000,0.000000}%
\pgfsetstrokecolor{currentstroke}%
\pgfsetdash{}{0pt}%
\pgfpathmoveto{\pgfqpoint{1.612811in}{0.979006in}}%
\pgfpathlineto{\pgfqpoint{1.650564in}{0.980770in}}%
\pgfpathlineto{\pgfqpoint{1.612530in}{1.018475in}}%
\pgfpathlineto{\pgfqpoint{1.574645in}{1.016847in}}%
\pgfpathclose%
\pgfusepath{fill}%
\end{pgfscope}%
\begin{pgfscope}%
\pgfpathrectangle{\pgfqpoint{0.150000in}{0.150000in}}{\pgfqpoint{2.700000in}{1.950000in}}%
\pgfusepath{clip}%
\pgfsetbuttcap%
\pgfsetroundjoin%
\definecolor{currentfill}{rgb}{0.903600,0.915472,0.932093}%
\pgfsetfillcolor{currentfill}%
\pgfsetlinewidth{0.000000pt}%
\definecolor{currentstroke}{rgb}{0.000000,0.000000,0.000000}%
\pgfsetstrokecolor{currentstroke}%
\pgfsetdash{}{0pt}%
\pgfpathmoveto{\pgfqpoint{1.651019in}{0.944153in}}%
\pgfpathlineto{\pgfqpoint{1.688651in}{0.946042in}}%
\pgfpathlineto{\pgfqpoint{1.650564in}{0.980770in}}%
\pgfpathlineto{\pgfqpoint{1.612811in}{0.979006in}}%
\pgfpathclose%
\pgfusepath{fill}%
\end{pgfscope}%
\begin{pgfscope}%
\pgfpathrectangle{\pgfqpoint{0.150000in}{0.150000in}}{\pgfqpoint{2.700000in}{1.950000in}}%
\pgfusepath{clip}%
\pgfsetbuttcap%
\pgfsetroundjoin%
\definecolor{currentfill}{rgb}{0.928477,0.937286,0.949617}%
\pgfsetfillcolor{currentfill}%
\pgfsetlinewidth{0.000000pt}%
\definecolor{currentstroke}{rgb}{0.000000,0.000000,0.000000}%
\pgfsetstrokecolor{currentstroke}%
\pgfsetdash{}{0pt}%
\pgfpathmoveto{\pgfqpoint{1.689214in}{0.906284in}}%
\pgfpathlineto{\pgfqpoint{1.726713in}{0.908309in}}%
\pgfpathlineto{\pgfqpoint{1.688651in}{0.946042in}}%
\pgfpathlineto{\pgfqpoint{1.651019in}{0.944153in}}%
\pgfpathclose%
\pgfusepath{fill}%
\end{pgfscope}%
\begin{pgfscope}%
\pgfpathrectangle{\pgfqpoint{0.150000in}{0.150000in}}{\pgfqpoint{2.700000in}{1.950000in}}%
\pgfusepath{clip}%
\pgfsetbuttcap%
\pgfsetroundjoin%
\definecolor{currentfill}{rgb}{0.350077,0.430116,0.542172}%
\pgfsetfillcolor{currentfill}%
\pgfsetlinewidth{0.000000pt}%
\definecolor{currentstroke}{rgb}{0.000000,0.000000,0.000000}%
\pgfsetstrokecolor{currentstroke}%
\pgfsetdash{}{0pt}%
\pgfpathmoveto{\pgfqpoint{0.925362in}{1.612133in}}%
\pgfpathlineto{\pgfqpoint{0.965937in}{1.605274in}}%
\pgfpathlineto{\pgfqpoint{0.927967in}{1.642923in}}%
\pgfpathlineto{\pgfqpoint{0.887237in}{1.649938in}}%
\pgfpathclose%
\pgfusepath{fill}%
\end{pgfscope}%
\begin{pgfscope}%
\pgfpathrectangle{\pgfqpoint{0.150000in}{0.150000in}}{\pgfqpoint{2.700000in}{1.950000in}}%
\pgfusepath{clip}%
\pgfsetbuttcap%
\pgfsetroundjoin%
\definecolor{currentfill}{rgb}{0.959574,0.964553,0.971523}%
\pgfsetfillcolor{currentfill}%
\pgfsetlinewidth{0.000000pt}%
\definecolor{currentstroke}{rgb}{0.000000,0.000000,0.000000}%
\pgfsetstrokecolor{currentstroke}%
\pgfsetdash{}{0pt}%
\pgfpathmoveto{\pgfqpoint{1.727416in}{0.868408in}}%
\pgfpathlineto{\pgfqpoint{1.764783in}{0.870568in}}%
\pgfpathlineto{\pgfqpoint{1.726713in}{0.908309in}}%
\pgfpathlineto{\pgfqpoint{1.689214in}{0.906284in}}%
\pgfpathclose%
\pgfusepath{fill}%
\end{pgfscope}%
\begin{pgfscope}%
\pgfpathrectangle{\pgfqpoint{0.150000in}{0.150000in}}{\pgfqpoint{2.700000in}{1.950000in}}%
\pgfusepath{clip}%
\pgfsetbuttcap%
\pgfsetroundjoin%
\definecolor{currentfill}{rgb}{0.381173,0.457384,0.564078}%
\pgfsetfillcolor{currentfill}%
\pgfsetlinewidth{0.000000pt}%
\definecolor{currentstroke}{rgb}{0.000000,0.000000,0.000000}%
\pgfsetstrokecolor{currentstroke}%
\pgfsetdash{}{0pt}%
\pgfpathmoveto{\pgfqpoint{0.963494in}{1.574321in}}%
\pgfpathlineto{\pgfqpoint{1.003762in}{1.570786in}}%
\pgfpathlineto{\pgfqpoint{0.965937in}{1.605274in}}%
\pgfpathlineto{\pgfqpoint{0.925362in}{1.612133in}}%
\pgfpathclose%
\pgfusepath{fill}%
\end{pgfscope}%
\begin{pgfscope}%
\pgfpathrectangle{\pgfqpoint{0.150000in}{0.150000in}}{\pgfqpoint{2.700000in}{1.950000in}}%
\pgfusepath{clip}%
\pgfsetbuttcap%
\pgfsetroundjoin%
\definecolor{currentfill}{rgb}{0.990671,0.991820,0.993428}%
\pgfsetfillcolor{currentfill}%
\pgfsetlinewidth{0.000000pt}%
\definecolor{currentstroke}{rgb}{0.000000,0.000000,0.000000}%
\pgfsetstrokecolor{currentstroke}%
\pgfsetdash{}{0pt}%
\pgfpathmoveto{\pgfqpoint{1.765627in}{0.830523in}}%
\pgfpathlineto{\pgfqpoint{1.802938in}{0.835775in}}%
\pgfpathlineto{\pgfqpoint{1.764783in}{0.870568in}}%
\pgfpathlineto{\pgfqpoint{1.727416in}{0.868408in}}%
\pgfpathclose%
\pgfusepath{fill}%
\end{pgfscope}%
\begin{pgfscope}%
\pgfpathrectangle{\pgfqpoint{0.150000in}{0.150000in}}{\pgfqpoint{2.700000in}{1.950000in}}%
\pgfusepath{clip}%
\pgfsetbuttcap%
\pgfsetroundjoin%
\definecolor{currentfill}{rgb}{0.406051,0.479197,0.581602}%
\pgfsetfillcolor{currentfill}%
\pgfsetlinewidth{0.000000pt}%
\definecolor{currentstroke}{rgb}{0.000000,0.000000,0.000000}%
\pgfsetstrokecolor{currentstroke}%
\pgfsetdash{}{0pt}%
\pgfpathmoveto{\pgfqpoint{1.001480in}{1.539672in}}%
\pgfpathlineto{\pgfqpoint{1.041760in}{1.533110in}}%
\pgfpathlineto{\pgfqpoint{1.003762in}{1.570786in}}%
\pgfpathlineto{\pgfqpoint{0.963494in}{1.574321in}}%
\pgfpathclose%
\pgfusepath{fill}%
\end{pgfscope}%
\begin{pgfscope}%
\pgfpathrectangle{\pgfqpoint{0.150000in}{0.150000in}}{\pgfqpoint{2.700000in}{1.950000in}}%
\pgfusepath{clip}%
\pgfsetbuttcap%
\pgfsetroundjoin%
\definecolor{currentfill}{rgb}{0.990502,0.982767,0.983379}%
\pgfsetfillcolor{currentfill}%
\pgfsetlinewidth{0.000000pt}%
\definecolor{currentstroke}{rgb}{0.000000,0.000000,0.000000}%
\pgfsetstrokecolor{currentstroke}%
\pgfsetdash{}{0pt}%
\pgfpathmoveto{\pgfqpoint{1.803924in}{0.795585in}}%
\pgfpathlineto{\pgfqpoint{1.841036in}{0.798007in}}%
\pgfpathlineto{\pgfqpoint{1.802938in}{0.835775in}}%
\pgfpathlineto{\pgfqpoint{1.765627in}{0.830523in}}%
\pgfpathclose%
\pgfusepath{fill}%
\end{pgfscope}%
\begin{pgfscope}%
\pgfpathrectangle{\pgfqpoint{0.150000in}{0.150000in}}{\pgfqpoint{2.700000in}{1.950000in}}%
\pgfusepath{clip}%
\pgfsetbuttcap%
\pgfsetroundjoin%
\definecolor{currentfill}{rgb}{0.437148,0.506464,0.603508}%
\pgfsetfillcolor{currentfill}%
\pgfsetlinewidth{0.000000pt}%
\definecolor{currentstroke}{rgb}{0.000000,0.000000,0.000000}%
\pgfsetstrokecolor{currentstroke}%
\pgfsetdash{}{0pt}%
\pgfpathmoveto{\pgfqpoint{1.039640in}{1.501832in}}%
\pgfpathlineto{\pgfqpoint{1.079766in}{1.495427in}}%
\pgfpathlineto{\pgfqpoint{1.041760in}{1.533110in}}%
\pgfpathlineto{\pgfqpoint{1.001480in}{1.539672in}}%
\pgfpathclose%
\pgfusepath{fill}%
\end{pgfscope}%
\begin{pgfscope}%
\pgfpathrectangle{\pgfqpoint{0.150000in}{0.150000in}}{\pgfqpoint{2.700000in}{1.950000in}}%
\pgfusepath{clip}%
\pgfsetbuttcap%
\pgfsetroundjoin%
\definecolor{currentfill}{rgb}{0.468244,0.533732,0.625414}%
\pgfsetfillcolor{currentfill}%
\pgfsetlinewidth{0.000000pt}%
\definecolor{currentstroke}{rgb}{0.000000,0.000000,0.000000}%
\pgfsetstrokecolor{currentstroke}%
\pgfsetdash{}{0pt}%
\pgfpathmoveto{\pgfqpoint{1.077809in}{1.463983in}}%
\pgfpathlineto{\pgfqpoint{1.117780in}{1.457735in}}%
\pgfpathlineto{\pgfqpoint{1.079766in}{1.495427in}}%
\pgfpathlineto{\pgfqpoint{1.039640in}{1.501832in}}%
\pgfpathclose%
\pgfusepath{fill}%
\end{pgfscope}%
\begin{pgfscope}%
\pgfpathrectangle{\pgfqpoint{0.150000in}{0.150000in}}{\pgfqpoint{2.700000in}{1.950000in}}%
\pgfusepath{clip}%
\pgfsetbuttcap%
\pgfsetroundjoin%
\definecolor{currentfill}{rgb}{0.971507,0.948300,0.950138}%
\pgfsetfillcolor{currentfill}%
\pgfsetlinewidth{0.000000pt}%
\definecolor{currentstroke}{rgb}{0.000000,0.000000,0.000000}%
\pgfsetstrokecolor{currentstroke}%
\pgfsetdash{}{0pt}%
\pgfpathmoveto{\pgfqpoint{1.842163in}{0.757673in}}%
\pgfpathlineto{\pgfqpoint{1.879241in}{0.763168in}}%
\pgfpathlineto{\pgfqpoint{1.841036in}{0.798007in}}%
\pgfpathlineto{\pgfqpoint{1.803924in}{0.795585in}}%
\pgfpathclose%
\pgfusepath{fill}%
\end{pgfscope}%
\begin{pgfscope}%
\pgfpathrectangle{\pgfqpoint{0.150000in}{0.150000in}}{\pgfqpoint{2.700000in}{1.950000in}}%
\pgfusepath{clip}%
\pgfsetbuttcap%
\pgfsetroundjoin%
\definecolor{currentfill}{rgb}{0.493122,0.555545,0.642938}%
\pgfsetfillcolor{currentfill}%
\pgfsetlinewidth{0.000000pt}%
\definecolor{currentstroke}{rgb}{0.000000,0.000000,0.000000}%
\pgfsetstrokecolor{currentstroke}%
\pgfsetdash{}{0pt}%
\pgfpathmoveto{\pgfqpoint{1.115986in}{1.426127in}}%
\pgfpathlineto{\pgfqpoint{1.155693in}{1.423163in}}%
\pgfpathlineto{\pgfqpoint{1.117780in}{1.457735in}}%
\pgfpathlineto{\pgfqpoint{1.077809in}{1.463983in}}%
\pgfpathclose%
\pgfusepath{fill}%
\end{pgfscope}%
\begin{pgfscope}%
\pgfpathrectangle{\pgfqpoint{0.150000in}{0.150000in}}{\pgfqpoint{2.700000in}{1.950000in}}%
\pgfusepath{clip}%
\pgfsetbuttcap%
\pgfsetroundjoin%
\definecolor{currentfill}{rgb}{0.956311,0.920726,0.923545}%
\pgfsetfillcolor{currentfill}%
\pgfsetlinewidth{0.000000pt}%
\definecolor{currentstroke}{rgb}{0.000000,0.000000,0.000000}%
\pgfsetstrokecolor{currentstroke}%
\pgfsetdash{}{0pt}%
\pgfpathmoveto{\pgfqpoint{1.880411in}{0.719752in}}%
\pgfpathlineto{\pgfqpoint{1.917366in}{0.725373in}}%
\pgfpathlineto{\pgfqpoint{1.879241in}{0.763168in}}%
\pgfpathlineto{\pgfqpoint{1.842163in}{0.757673in}}%
\pgfpathclose%
\pgfusepath{fill}%
\end{pgfscope}%
\begin{pgfscope}%
\pgfpathrectangle{\pgfqpoint{0.150000in}{0.150000in}}{\pgfqpoint{2.700000in}{1.950000in}}%
\pgfusepath{clip}%
\pgfsetbuttcap%
\pgfsetroundjoin%
\definecolor{currentfill}{rgb}{0.524219,0.582812,0.664844}%
\pgfsetfillcolor{currentfill}%
\pgfsetlinewidth{0.000000pt}%
\definecolor{currentstroke}{rgb}{0.000000,0.000000,0.000000}%
\pgfsetstrokecolor{currentstroke}%
\pgfsetdash{}{0pt}%
\pgfpathmoveto{\pgfqpoint{1.154172in}{1.388262in}}%
\pgfpathlineto{\pgfqpoint{1.193735in}{1.385444in}}%
\pgfpathlineto{\pgfqpoint{1.155693in}{1.423163in}}%
\pgfpathlineto{\pgfqpoint{1.115986in}{1.426127in}}%
\pgfpathclose%
\pgfusepath{fill}%
\end{pgfscope}%
\begin{pgfscope}%
\pgfpathrectangle{\pgfqpoint{0.150000in}{0.150000in}}{\pgfqpoint{2.700000in}{1.950000in}}%
\pgfusepath{clip}%
\pgfsetbuttcap%
\pgfsetroundjoin%
\definecolor{currentfill}{rgb}{0.555316,0.610080,0.686749}%
\pgfsetfillcolor{currentfill}%
\pgfsetlinewidth{0.000000pt}%
\definecolor{currentstroke}{rgb}{0.000000,0.000000,0.000000}%
\pgfsetstrokecolor{currentstroke}%
\pgfsetdash{}{0pt}%
\pgfpathmoveto{\pgfqpoint{1.192365in}{1.350388in}}%
\pgfpathlineto{\pgfqpoint{1.231785in}{1.347717in}}%
\pgfpathlineto{\pgfqpoint{1.193735in}{1.385444in}}%
\pgfpathlineto{\pgfqpoint{1.154172in}{1.388262in}}%
\pgfpathclose%
\pgfusepath{fill}%
\end{pgfscope}%
\begin{pgfscope}%
\pgfpathrectangle{\pgfqpoint{0.150000in}{0.150000in}}{\pgfqpoint{2.700000in}{1.950000in}}%
\pgfusepath{clip}%
\pgfsetbuttcap%
\pgfsetroundjoin%
\definecolor{currentfill}{rgb}{0.580193,0.631893,0.704274}%
\pgfsetfillcolor{currentfill}%
\pgfsetlinewidth{0.000000pt}%
\definecolor{currentstroke}{rgb}{0.000000,0.000000,0.000000}%
\pgfsetstrokecolor{currentstroke}%
\pgfsetdash{}{0pt}%
\pgfpathmoveto{\pgfqpoint{1.230479in}{1.315617in}}%
\pgfpathlineto{\pgfqpoint{1.269767in}{1.313079in}}%
\pgfpathlineto{\pgfqpoint{1.231785in}{1.347717in}}%
\pgfpathlineto{\pgfqpoint{1.192365in}{1.350388in}}%
\pgfpathclose%
\pgfusepath{fill}%
\end{pgfscope}%
\begin{pgfscope}%
\pgfpathrectangle{\pgfqpoint{0.150000in}{0.150000in}}{\pgfqpoint{2.700000in}{1.950000in}}%
\pgfusepath{clip}%
\pgfsetbuttcap%
\pgfsetroundjoin%
\definecolor{currentfill}{rgb}{0.611290,0.659161,0.726180}%
\pgfsetfillcolor{currentfill}%
\pgfsetlinewidth{0.000000pt}%
\definecolor{currentstroke}{rgb}{0.000000,0.000000,0.000000}%
\pgfsetstrokecolor{currentstroke}%
\pgfsetdash{}{0pt}%
\pgfpathmoveto{\pgfqpoint{1.268700in}{1.277717in}}%
\pgfpathlineto{\pgfqpoint{1.307845in}{1.275325in}}%
\pgfpathlineto{\pgfqpoint{1.269767in}{1.313079in}}%
\pgfpathlineto{\pgfqpoint{1.230479in}{1.315617in}}%
\pgfpathclose%
\pgfusepath{fill}%
\end{pgfscope}%
\begin{pgfscope}%
\pgfpathrectangle{\pgfqpoint{0.150000in}{0.150000in}}{\pgfqpoint{2.700000in}{1.950000in}}%
\pgfusepath{clip}%
\pgfsetbuttcap%
\pgfsetroundjoin%
\definecolor{currentfill}{rgb}{0.642387,0.686428,0.748085}%
\pgfsetfillcolor{currentfill}%
\pgfsetlinewidth{0.000000pt}%
\definecolor{currentstroke}{rgb}{0.000000,0.000000,0.000000}%
\pgfsetstrokecolor{currentstroke}%
\pgfsetdash{}{0pt}%
\pgfpathmoveto{\pgfqpoint{1.306930in}{1.239808in}}%
\pgfpathlineto{\pgfqpoint{1.345931in}{1.237563in}}%
\pgfpathlineto{\pgfqpoint{1.307845in}{1.275325in}}%
\pgfpathlineto{\pgfqpoint{1.268700in}{1.277717in}}%
\pgfpathclose%
\pgfusepath{fill}%
\end{pgfscope}%
\begin{pgfscope}%
\pgfpathrectangle{\pgfqpoint{0.150000in}{0.150000in}}{\pgfqpoint{2.700000in}{1.950000in}}%
\pgfusepath{clip}%
\pgfsetbuttcap%
\pgfsetroundjoin%
\definecolor{currentfill}{rgb}{0.667264,0.708241,0.765610}%
\pgfsetfillcolor{currentfill}%
\pgfsetlinewidth{0.000000pt}%
\definecolor{currentstroke}{rgb}{0.000000,0.000000,0.000000}%
\pgfsetstrokecolor{currentstroke}%
\pgfsetdash{}{0pt}%
\pgfpathmoveto{\pgfqpoint{1.345169in}{1.201891in}}%
\pgfpathlineto{\pgfqpoint{1.383981in}{1.202860in}}%
\pgfpathlineto{\pgfqpoint{1.345931in}{1.237563in}}%
\pgfpathlineto{\pgfqpoint{1.306930in}{1.239808in}}%
\pgfpathclose%
\pgfusepath{fill}%
\end{pgfscope}%
\begin{pgfscope}%
\pgfpathrectangle{\pgfqpoint{0.150000in}{0.150000in}}{\pgfqpoint{2.700000in}{1.950000in}}%
\pgfusepath{clip}%
\pgfsetbuttcap%
\pgfsetroundjoin%
\definecolor{currentfill}{rgb}{0.698361,0.735509,0.787515}%
\pgfsetfillcolor{currentfill}%
\pgfsetlinewidth{0.000000pt}%
\definecolor{currentstroke}{rgb}{0.000000,0.000000,0.000000}%
\pgfsetstrokecolor{currentstroke}%
\pgfsetdash{}{0pt}%
\pgfpathmoveto{\pgfqpoint{1.383416in}{1.163965in}}%
\pgfpathlineto{\pgfqpoint{1.422095in}{1.165071in}}%
\pgfpathlineto{\pgfqpoint{1.383981in}{1.202860in}}%
\pgfpathlineto{\pgfqpoint{1.345169in}{1.201891in}}%
\pgfpathclose%
\pgfusepath{fill}%
\end{pgfscope}%
\begin{pgfscope}%
\pgfpathrectangle{\pgfqpoint{0.150000in}{0.150000in}}{\pgfqpoint{2.700000in}{1.950000in}}%
\pgfusepath{clip}%
\pgfsetbuttcap%
\pgfsetroundjoin%
\definecolor{currentfill}{rgb}{0.729458,0.762776,0.809421}%
\pgfsetfillcolor{currentfill}%
\pgfsetlinewidth{0.000000pt}%
\definecolor{currentstroke}{rgb}{0.000000,0.000000,0.000000}%
\pgfsetstrokecolor{currentstroke}%
\pgfsetdash{}{0pt}%
\pgfpathmoveto{\pgfqpoint{1.421637in}{1.129090in}}%
\pgfpathlineto{\pgfqpoint{1.460217in}{1.127273in}}%
\pgfpathlineto{\pgfqpoint{1.422095in}{1.165071in}}%
\pgfpathlineto{\pgfqpoint{1.383416in}{1.163965in}}%
\pgfpathclose%
\pgfusepath{fill}%
\end{pgfscope}%
\begin{pgfscope}%
\pgfpathrectangle{\pgfqpoint{0.150000in}{0.150000in}}{\pgfqpoint{2.700000in}{1.950000in}}%
\pgfusepath{clip}%
\pgfsetbuttcap%
\pgfsetroundjoin%
\definecolor{currentfill}{rgb}{0.754335,0.784589,0.826945}%
\pgfsetfillcolor{currentfill}%
\pgfsetlinewidth{0.000000pt}%
\definecolor{currentstroke}{rgb}{0.000000,0.000000,0.000000}%
\pgfsetstrokecolor{currentstroke}%
\pgfsetdash{}{0pt}%
\pgfpathmoveto{\pgfqpoint{1.459912in}{1.091137in}}%
\pgfpathlineto{\pgfqpoint{1.498348in}{1.089467in}}%
\pgfpathlineto{\pgfqpoint{1.460217in}{1.127273in}}%
\pgfpathlineto{\pgfqpoint{1.421637in}{1.129090in}}%
\pgfpathclose%
\pgfusepath{fill}%
\end{pgfscope}%
\begin{pgfscope}%
\pgfpathrectangle{\pgfqpoint{0.150000in}{0.150000in}}{\pgfqpoint{2.700000in}{1.950000in}}%
\pgfusepath{clip}%
\pgfsetbuttcap%
\pgfsetroundjoin%
\definecolor{currentfill}{rgb}{0.785432,0.811857,0.848851}%
\pgfsetfillcolor{currentfill}%
\pgfsetlinewidth{0.000000pt}%
\definecolor{currentstroke}{rgb}{0.000000,0.000000,0.000000}%
\pgfsetstrokecolor{currentstroke}%
\pgfsetdash{}{0pt}%
\pgfpathmoveto{\pgfqpoint{1.498195in}{1.053176in}}%
\pgfpathlineto{\pgfqpoint{1.536486in}{1.054680in}}%
\pgfpathlineto{\pgfqpoint{1.498348in}{1.089467in}}%
\pgfpathlineto{\pgfqpoint{1.459912in}{1.091137in}}%
\pgfpathclose%
\pgfusepath{fill}%
\end{pgfscope}%
\begin{pgfscope}%
\pgfpathrectangle{\pgfqpoint{0.150000in}{0.150000in}}{\pgfqpoint{2.700000in}{1.950000in}}%
\pgfusepath{clip}%
\pgfsetbuttcap%
\pgfsetroundjoin%
\definecolor{currentfill}{rgb}{0.816529,0.839124,0.870757}%
\pgfsetfillcolor{currentfill}%
\pgfsetlinewidth{0.000000pt}%
\definecolor{currentstroke}{rgb}{0.000000,0.000000,0.000000}%
\pgfsetstrokecolor{currentstroke}%
\pgfsetdash{}{0pt}%
\pgfpathmoveto{\pgfqpoint{1.536486in}{1.015207in}}%
\pgfpathlineto{\pgfqpoint{1.574645in}{1.016847in}}%
\pgfpathlineto{\pgfqpoint{1.536486in}{1.054680in}}%
\pgfpathlineto{\pgfqpoint{1.498195in}{1.053176in}}%
\pgfpathclose%
\pgfusepath{fill}%
\end{pgfscope}%
\begin{pgfscope}%
\pgfpathrectangle{\pgfqpoint{0.150000in}{0.150000in}}{\pgfqpoint{2.700000in}{1.950000in}}%
\pgfusepath{clip}%
\pgfsetbuttcap%
\pgfsetroundjoin%
\definecolor{currentfill}{rgb}{0.841406,0.860938,0.888281}%
\pgfsetfillcolor{currentfill}%
\pgfsetlinewidth{0.000000pt}%
\definecolor{currentstroke}{rgb}{0.000000,0.000000,0.000000}%
\pgfsetstrokecolor{currentstroke}%
\pgfsetdash{}{0pt}%
\pgfpathmoveto{\pgfqpoint{1.574786in}{0.977229in}}%
\pgfpathlineto{\pgfqpoint{1.612811in}{0.979006in}}%
\pgfpathlineto{\pgfqpoint{1.574645in}{1.016847in}}%
\pgfpathlineto{\pgfqpoint{1.536486in}{1.015207in}}%
\pgfpathclose%
\pgfusepath{fill}%
\end{pgfscope}%
\begin{pgfscope}%
\pgfpathrectangle{\pgfqpoint{0.150000in}{0.150000in}}{\pgfqpoint{2.700000in}{1.950000in}}%
\pgfusepath{clip}%
\pgfsetbuttcap%
\pgfsetroundjoin%
\definecolor{currentfill}{rgb}{0.872503,0.888205,0.910187}%
\pgfsetfillcolor{currentfill}%
\pgfsetlinewidth{0.000000pt}%
\definecolor{currentstroke}{rgb}{0.000000,0.000000,0.000000}%
\pgfsetstrokecolor{currentstroke}%
\pgfsetdash{}{0pt}%
\pgfpathmoveto{\pgfqpoint{1.613117in}{0.942250in}}%
\pgfpathlineto{\pgfqpoint{1.651019in}{0.944153in}}%
\pgfpathlineto{\pgfqpoint{1.612811in}{0.979006in}}%
\pgfpathlineto{\pgfqpoint{1.574786in}{0.977229in}}%
\pgfpathclose%
\pgfusepath{fill}%
\end{pgfscope}%
\begin{pgfscope}%
\pgfpathrectangle{\pgfqpoint{0.150000in}{0.150000in}}{\pgfqpoint{2.700000in}{1.950000in}}%
\pgfusepath{clip}%
\pgfsetbuttcap%
\pgfsetroundjoin%
\definecolor{currentfill}{rgb}{0.897381,0.910018,0.927711}%
\pgfsetfillcolor{currentfill}%
\pgfsetlinewidth{0.000000pt}%
\definecolor{currentstroke}{rgb}{0.000000,0.000000,0.000000}%
\pgfsetstrokecolor{currentstroke}%
\pgfsetdash{}{0pt}%
\pgfpathmoveto{\pgfqpoint{1.651445in}{0.904245in}}%
\pgfpathlineto{\pgfqpoint{1.689214in}{0.906284in}}%
\pgfpathlineto{\pgfqpoint{1.651019in}{0.944153in}}%
\pgfpathlineto{\pgfqpoint{1.613117in}{0.942250in}}%
\pgfpathclose%
\pgfusepath{fill}%
\end{pgfscope}%
\begin{pgfscope}%
\pgfpathrectangle{\pgfqpoint{0.150000in}{0.150000in}}{\pgfqpoint{2.700000in}{1.950000in}}%
\pgfusepath{clip}%
\pgfsetbuttcap%
\pgfsetroundjoin%
\definecolor{currentfill}{rgb}{0.928477,0.937286,0.949617}%
\pgfsetfillcolor{currentfill}%
\pgfsetlinewidth{0.000000pt}%
\definecolor{currentstroke}{rgb}{0.000000,0.000000,0.000000}%
\pgfsetstrokecolor{currentstroke}%
\pgfsetdash{}{0pt}%
\pgfpathmoveto{\pgfqpoint{1.689781in}{0.866231in}}%
\pgfpathlineto{\pgfqpoint{1.727416in}{0.868408in}}%
\pgfpathlineto{\pgfqpoint{1.689214in}{0.906284in}}%
\pgfpathlineto{\pgfqpoint{1.651445in}{0.904245in}}%
\pgfpathclose%
\pgfusepath{fill}%
\end{pgfscope}%
\begin{pgfscope}%
\pgfpathrectangle{\pgfqpoint{0.150000in}{0.150000in}}{\pgfqpoint{2.700000in}{1.950000in}}%
\pgfusepath{clip}%
\pgfsetbuttcap%
\pgfsetroundjoin%
\definecolor{currentfill}{rgb}{0.959574,0.964553,0.971523}%
\pgfsetfillcolor{currentfill}%
\pgfsetlinewidth{0.000000pt}%
\definecolor{currentstroke}{rgb}{0.000000,0.000000,0.000000}%
\pgfsetstrokecolor{currentstroke}%
\pgfsetdash{}{0pt}%
\pgfpathmoveto{\pgfqpoint{1.728126in}{0.828210in}}%
\pgfpathlineto{\pgfqpoint{1.765627in}{0.830523in}}%
\pgfpathlineto{\pgfqpoint{1.727416in}{0.868408in}}%
\pgfpathlineto{\pgfqpoint{1.689781in}{0.866231in}}%
\pgfpathclose%
\pgfusepath{fill}%
\end{pgfscope}%
\begin{pgfscope}%
\pgfpathrectangle{\pgfqpoint{0.150000in}{0.150000in}}{\pgfqpoint{2.700000in}{1.950000in}}%
\pgfusepath{clip}%
\pgfsetbuttcap%
\pgfsetroundjoin%
\definecolor{currentfill}{rgb}{0.990671,0.991820,0.993428}%
\pgfsetfillcolor{currentfill}%
\pgfsetlinewidth{0.000000pt}%
\definecolor{currentstroke}{rgb}{0.000000,0.000000,0.000000}%
\pgfsetstrokecolor{currentstroke}%
\pgfsetdash{}{0pt}%
\pgfpathmoveto{\pgfqpoint{1.766479in}{0.790180in}}%
\pgfpathlineto{\pgfqpoint{1.803924in}{0.795585in}}%
\pgfpathlineto{\pgfqpoint{1.765627in}{0.830523in}}%
\pgfpathlineto{\pgfqpoint{1.728126in}{0.828210in}}%
\pgfpathclose%
\pgfusepath{fill}%
\end{pgfscope}%
\begin{pgfscope}%
\pgfpathrectangle{\pgfqpoint{0.150000in}{0.150000in}}{\pgfqpoint{2.700000in}{1.950000in}}%
\pgfusepath{clip}%
\pgfsetbuttcap%
\pgfsetroundjoin%
\definecolor{currentfill}{rgb}{0.990502,0.982767,0.983379}%
\pgfsetfillcolor{currentfill}%
\pgfsetlinewidth{0.000000pt}%
\definecolor{currentstroke}{rgb}{0.000000,0.000000,0.000000}%
\pgfsetstrokecolor{currentstroke}%
\pgfsetdash{}{0pt}%
\pgfpathmoveto{\pgfqpoint{1.804840in}{0.752141in}}%
\pgfpathlineto{\pgfqpoint{1.842163in}{0.757673in}}%
\pgfpathlineto{\pgfqpoint{1.803924in}{0.795585in}}%
\pgfpathlineto{\pgfqpoint{1.766479in}{0.790180in}}%
\pgfpathclose%
\pgfusepath{fill}%
\end{pgfscope}%
\begin{pgfscope}%
\pgfpathrectangle{\pgfqpoint{0.150000in}{0.150000in}}{\pgfqpoint{2.700000in}{1.950000in}}%
\pgfusepath{clip}%
\pgfsetbuttcap%
\pgfsetroundjoin%
\definecolor{currentfill}{rgb}{0.312760,0.397396,0.515885}%
\pgfsetfillcolor{currentfill}%
\pgfsetlinewidth{0.000000pt}%
\definecolor{currentstroke}{rgb}{0.000000,0.000000,0.000000}%
\pgfsetstrokecolor{currentstroke}%
\pgfsetdash{}{0pt}%
\pgfpathmoveto{\pgfqpoint{0.884446in}{1.619050in}}%
\pgfpathlineto{\pgfqpoint{0.925362in}{1.612133in}}%
\pgfpathlineto{\pgfqpoint{0.887237in}{1.649938in}}%
\pgfpathlineto{\pgfqpoint{0.846166in}{1.657012in}}%
\pgfpathclose%
\pgfusepath{fill}%
\end{pgfscope}%
\begin{pgfscope}%
\pgfpathrectangle{\pgfqpoint{0.150000in}{0.150000in}}{\pgfqpoint{2.700000in}{1.950000in}}%
\pgfusepath{clip}%
\pgfsetbuttcap%
\pgfsetroundjoin%
\definecolor{currentfill}{rgb}{0.337638,0.419210,0.533410}%
\pgfsetfillcolor{currentfill}%
\pgfsetlinewidth{0.000000pt}%
\definecolor{currentstroke}{rgb}{0.000000,0.000000,0.000000}%
\pgfsetstrokecolor{currentstroke}%
\pgfsetdash{}{0pt}%
\pgfpathmoveto{\pgfqpoint{0.922734in}{1.581079in}}%
\pgfpathlineto{\pgfqpoint{0.963494in}{1.574321in}}%
\pgfpathlineto{\pgfqpoint{0.925362in}{1.612133in}}%
\pgfpathlineto{\pgfqpoint{0.884446in}{1.619050in}}%
\pgfpathclose%
\pgfusepath{fill}%
\end{pgfscope}%
\begin{pgfscope}%
\pgfpathrectangle{\pgfqpoint{0.150000in}{0.150000in}}{\pgfqpoint{2.700000in}{1.950000in}}%
\pgfusepath{clip}%
\pgfsetbuttcap%
\pgfsetroundjoin%
\definecolor{currentfill}{rgb}{0.971507,0.948300,0.950138}%
\pgfsetfillcolor{currentfill}%
\pgfsetlinewidth{0.000000pt}%
\definecolor{currentstroke}{rgb}{0.000000,0.000000,0.000000}%
\pgfsetstrokecolor{currentstroke}%
\pgfsetdash{}{0pt}%
\pgfpathmoveto{\pgfqpoint{1.843299in}{0.717038in}}%
\pgfpathlineto{\pgfqpoint{1.880411in}{0.719752in}}%
\pgfpathlineto{\pgfqpoint{1.842163in}{0.757673in}}%
\pgfpathlineto{\pgfqpoint{1.804840in}{0.752141in}}%
\pgfpathclose%
\pgfusepath{fill}%
\end{pgfscope}%
\begin{pgfscope}%
\pgfpathrectangle{\pgfqpoint{0.150000in}{0.150000in}}{\pgfqpoint{2.700000in}{1.950000in}}%
\pgfusepath{clip}%
\pgfsetbuttcap%
\pgfsetroundjoin%
\definecolor{currentfill}{rgb}{0.368735,0.446477,0.555316}%
\pgfsetfillcolor{currentfill}%
\pgfsetlinewidth{0.000000pt}%
\definecolor{currentstroke}{rgb}{0.000000,0.000000,0.000000}%
\pgfsetstrokecolor{currentstroke}%
\pgfsetdash{}{0pt}%
\pgfpathmoveto{\pgfqpoint{0.961030in}{1.543100in}}%
\pgfpathlineto{\pgfqpoint{1.001480in}{1.539672in}}%
\pgfpathlineto{\pgfqpoint{0.963494in}{1.574321in}}%
\pgfpathlineto{\pgfqpoint{0.922734in}{1.581079in}}%
\pgfpathclose%
\pgfusepath{fill}%
\end{pgfscope}%
\begin{pgfscope}%
\pgfpathrectangle{\pgfqpoint{0.150000in}{0.150000in}}{\pgfqpoint{2.700000in}{1.950000in}}%
\pgfusepath{clip}%
\pgfsetbuttcap%
\pgfsetroundjoin%
\definecolor{currentfill}{rgb}{0.399831,0.473744,0.577221}%
\pgfsetfillcolor{currentfill}%
\pgfsetlinewidth{0.000000pt}%
\definecolor{currentstroke}{rgb}{0.000000,0.000000,0.000000}%
\pgfsetstrokecolor{currentstroke}%
\pgfsetdash{}{0pt}%
\pgfpathmoveto{\pgfqpoint{0.999335in}{1.505113in}}%
\pgfpathlineto{\pgfqpoint{1.039640in}{1.501832in}}%
\pgfpathlineto{\pgfqpoint{1.001480in}{1.539672in}}%
\pgfpathlineto{\pgfqpoint{0.961030in}{1.543100in}}%
\pgfpathclose%
\pgfusepath{fill}%
\end{pgfscope}%
\begin{pgfscope}%
\pgfpathrectangle{\pgfqpoint{0.150000in}{0.150000in}}{\pgfqpoint{2.700000in}{1.950000in}}%
\pgfusepath{clip}%
\pgfsetbuttcap%
\pgfsetroundjoin%
\definecolor{currentfill}{rgb}{0.430928,0.501011,0.599127}%
\pgfsetfillcolor{currentfill}%
\pgfsetlinewidth{0.000000pt}%
\definecolor{currentstroke}{rgb}{0.000000,0.000000,0.000000}%
\pgfsetstrokecolor{currentstroke}%
\pgfsetdash{}{0pt}%
\pgfpathmoveto{\pgfqpoint{1.037648in}{1.467117in}}%
\pgfpathlineto{\pgfqpoint{1.077809in}{1.463983in}}%
\pgfpathlineto{\pgfqpoint{1.039640in}{1.501832in}}%
\pgfpathlineto{\pgfqpoint{0.999335in}{1.505113in}}%
\pgfpathclose%
\pgfusepath{fill}%
\end{pgfscope}%
\begin{pgfscope}%
\pgfpathrectangle{\pgfqpoint{0.150000in}{0.150000in}}{\pgfqpoint{2.700000in}{1.950000in}}%
\pgfusepath{clip}%
\pgfsetbuttcap%
\pgfsetroundjoin%
\definecolor{currentfill}{rgb}{0.455806,0.522825,0.616651}%
\pgfsetfillcolor{currentfill}%
\pgfsetlinewidth{0.000000pt}%
\definecolor{currentstroke}{rgb}{0.000000,0.000000,0.000000}%
\pgfsetstrokecolor{currentstroke}%
\pgfsetdash{}{0pt}%
\pgfpathmoveto{\pgfqpoint{1.075970in}{1.429114in}}%
\pgfpathlineto{\pgfqpoint{1.115986in}{1.426127in}}%
\pgfpathlineto{\pgfqpoint{1.077809in}{1.463983in}}%
\pgfpathlineto{\pgfqpoint{1.037648in}{1.467117in}}%
\pgfpathclose%
\pgfusepath{fill}%
\end{pgfscope}%
\begin{pgfscope}%
\pgfpathrectangle{\pgfqpoint{0.150000in}{0.150000in}}{\pgfqpoint{2.700000in}{1.950000in}}%
\pgfusepath{clip}%
\pgfsetbuttcap%
\pgfsetroundjoin%
\definecolor{currentfill}{rgb}{0.486903,0.550092,0.638557}%
\pgfsetfillcolor{currentfill}%
\pgfsetlinewidth{0.000000pt}%
\definecolor{currentstroke}{rgb}{0.000000,0.000000,0.000000}%
\pgfsetstrokecolor{currentstroke}%
\pgfsetdash{}{0pt}%
\pgfpathmoveto{\pgfqpoint{1.114177in}{1.394246in}}%
\pgfpathlineto{\pgfqpoint{1.154172in}{1.388262in}}%
\pgfpathlineto{\pgfqpoint{1.115986in}{1.426127in}}%
\pgfpathlineto{\pgfqpoint{1.075970in}{1.429114in}}%
\pgfpathclose%
\pgfusepath{fill}%
\end{pgfscope}%
\begin{pgfscope}%
\pgfpathrectangle{\pgfqpoint{0.150000in}{0.150000in}}{\pgfqpoint{2.700000in}{1.950000in}}%
\pgfusepath{clip}%
\pgfsetbuttcap%
\pgfsetroundjoin%
\definecolor{currentfill}{rgb}{0.517999,0.577359,0.660463}%
\pgfsetfillcolor{currentfill}%
\pgfsetlinewidth{0.000000pt}%
\definecolor{currentstroke}{rgb}{0.000000,0.000000,0.000000}%
\pgfsetstrokecolor{currentstroke}%
\pgfsetdash{}{0pt}%
\pgfpathmoveto{\pgfqpoint{1.152526in}{1.356214in}}%
\pgfpathlineto{\pgfqpoint{1.192365in}{1.350388in}}%
\pgfpathlineto{\pgfqpoint{1.154172in}{1.388262in}}%
\pgfpathlineto{\pgfqpoint{1.114177in}{1.394246in}}%
\pgfpathclose%
\pgfusepath{fill}%
\end{pgfscope}%
\begin{pgfscope}%
\pgfpathrectangle{\pgfqpoint{0.150000in}{0.150000in}}{\pgfqpoint{2.700000in}{1.950000in}}%
\pgfusepath{clip}%
\pgfsetbuttcap%
\pgfsetroundjoin%
\definecolor{currentfill}{rgb}{0.542877,0.599173,0.677987}%
\pgfsetfillcolor{currentfill}%
\pgfsetlinewidth{0.000000pt}%
\definecolor{currentstroke}{rgb}{0.000000,0.000000,0.000000}%
\pgfsetstrokecolor{currentstroke}%
\pgfsetdash{}{0pt}%
\pgfpathmoveto{\pgfqpoint{1.190884in}{1.318175in}}%
\pgfpathlineto{\pgfqpoint{1.230479in}{1.315617in}}%
\pgfpathlineto{\pgfqpoint{1.192365in}{1.350388in}}%
\pgfpathlineto{\pgfqpoint{1.152526in}{1.356214in}}%
\pgfpathclose%
\pgfusepath{fill}%
\end{pgfscope}%
\begin{pgfscope}%
\pgfpathrectangle{\pgfqpoint{0.150000in}{0.150000in}}{\pgfqpoint{2.700000in}{1.950000in}}%
\pgfusepath{clip}%
\pgfsetbuttcap%
\pgfsetroundjoin%
\definecolor{currentfill}{rgb}{0.573974,0.626440,0.699893}%
\pgfsetfillcolor{currentfill}%
\pgfsetlinewidth{0.000000pt}%
\definecolor{currentstroke}{rgb}{0.000000,0.000000,0.000000}%
\pgfsetstrokecolor{currentstroke}%
\pgfsetdash{}{0pt}%
\pgfpathmoveto{\pgfqpoint{1.229251in}{1.280126in}}%
\pgfpathlineto{\pgfqpoint{1.268700in}{1.277717in}}%
\pgfpathlineto{\pgfqpoint{1.230479in}{1.315617in}}%
\pgfpathlineto{\pgfqpoint{1.190884in}{1.318175in}}%
\pgfpathclose%
\pgfusepath{fill}%
\end{pgfscope}%
\begin{pgfscope}%
\pgfpathrectangle{\pgfqpoint{0.150000in}{0.150000in}}{\pgfqpoint{2.700000in}{1.950000in}}%
\pgfusepath{clip}%
\pgfsetbuttcap%
\pgfsetroundjoin%
\definecolor{currentfill}{rgb}{0.605070,0.653707,0.721798}%
\pgfsetfillcolor{currentfill}%
\pgfsetlinewidth{0.000000pt}%
\definecolor{currentstroke}{rgb}{0.000000,0.000000,0.000000}%
\pgfsetstrokecolor{currentstroke}%
\pgfsetdash{}{0pt}%
\pgfpathmoveto{\pgfqpoint{1.267626in}{1.242070in}}%
\pgfpathlineto{\pgfqpoint{1.306930in}{1.239808in}}%
\pgfpathlineto{\pgfqpoint{1.268700in}{1.277717in}}%
\pgfpathlineto{\pgfqpoint{1.229251in}{1.280126in}}%
\pgfpathclose%
\pgfusepath{fill}%
\end{pgfscope}%
\begin{pgfscope}%
\pgfpathrectangle{\pgfqpoint{0.150000in}{0.150000in}}{\pgfqpoint{2.700000in}{1.950000in}}%
\pgfusepath{clip}%
\pgfsetbuttcap%
\pgfsetroundjoin%
\definecolor{currentfill}{rgb}{0.636167,0.680974,0.743704}%
\pgfsetfillcolor{currentfill}%
\pgfsetlinewidth{0.000000pt}%
\definecolor{currentstroke}{rgb}{0.000000,0.000000,0.000000}%
\pgfsetstrokecolor{currentstroke}%
\pgfsetdash{}{0pt}%
\pgfpathmoveto{\pgfqpoint{1.306009in}{1.204005in}}%
\pgfpathlineto{\pgfqpoint{1.345169in}{1.201891in}}%
\pgfpathlineto{\pgfqpoint{1.306930in}{1.239808in}}%
\pgfpathlineto{\pgfqpoint{1.267626in}{1.242070in}}%
\pgfpathclose%
\pgfusepath{fill}%
\end{pgfscope}%
\begin{pgfscope}%
\pgfpathrectangle{\pgfqpoint{0.150000in}{0.150000in}}{\pgfqpoint{2.700000in}{1.950000in}}%
\pgfusepath{clip}%
\pgfsetbuttcap%
\pgfsetroundjoin%
\definecolor{currentfill}{rgb}{0.667264,0.708241,0.765610}%
\pgfsetfillcolor{currentfill}%
\pgfsetlinewidth{0.000000pt}%
\definecolor{currentstroke}{rgb}{0.000000,0.000000,0.000000}%
\pgfsetstrokecolor{currentstroke}%
\pgfsetdash{}{0pt}%
\pgfpathmoveto{\pgfqpoint{1.344401in}{1.165932in}}%
\pgfpathlineto{\pgfqpoint{1.383416in}{1.163965in}}%
\pgfpathlineto{\pgfqpoint{1.345169in}{1.201891in}}%
\pgfpathlineto{\pgfqpoint{1.306009in}{1.204005in}}%
\pgfpathclose%
\pgfusepath{fill}%
\end{pgfscope}%
\begin{pgfscope}%
\pgfpathrectangle{\pgfqpoint{0.150000in}{0.150000in}}{\pgfqpoint{2.700000in}{1.950000in}}%
\pgfusepath{clip}%
\pgfsetbuttcap%
\pgfsetroundjoin%
\definecolor{currentfill}{rgb}{0.692142,0.730055,0.783134}%
\pgfsetfillcolor{currentfill}%
\pgfsetlinewidth{0.000000pt}%
\definecolor{currentstroke}{rgb}{0.000000,0.000000,0.000000}%
\pgfsetstrokecolor{currentstroke}%
\pgfsetdash{}{0pt}%
\pgfpathmoveto{\pgfqpoint{1.382801in}{1.127850in}}%
\pgfpathlineto{\pgfqpoint{1.421637in}{1.129090in}}%
\pgfpathlineto{\pgfqpoint{1.383416in}{1.163965in}}%
\pgfpathlineto{\pgfqpoint{1.344401in}{1.165932in}}%
\pgfpathclose%
\pgfusepath{fill}%
\end{pgfscope}%
\begin{pgfscope}%
\pgfpathrectangle{\pgfqpoint{0.150000in}{0.150000in}}{\pgfqpoint{2.700000in}{1.950000in}}%
\pgfusepath{clip}%
\pgfsetbuttcap%
\pgfsetroundjoin%
\definecolor{currentfill}{rgb}{0.723238,0.757322,0.805040}%
\pgfsetfillcolor{currentfill}%
\pgfsetlinewidth{0.000000pt}%
\definecolor{currentstroke}{rgb}{0.000000,0.000000,0.000000}%
\pgfsetstrokecolor{currentstroke}%
\pgfsetdash{}{0pt}%
\pgfpathmoveto{\pgfqpoint{1.421210in}{1.089760in}}%
\pgfpathlineto{\pgfqpoint{1.459912in}{1.091137in}}%
\pgfpathlineto{\pgfqpoint{1.421637in}{1.129090in}}%
\pgfpathlineto{\pgfqpoint{1.382801in}{1.127850in}}%
\pgfpathclose%
\pgfusepath{fill}%
\end{pgfscope}%
\begin{pgfscope}%
\pgfpathrectangle{\pgfqpoint{0.150000in}{0.150000in}}{\pgfqpoint{2.700000in}{1.950000in}}%
\pgfusepath{clip}%
\pgfsetbuttcap%
\pgfsetroundjoin%
\definecolor{currentfill}{rgb}{0.754335,0.784589,0.826945}%
\pgfsetfillcolor{currentfill}%
\pgfsetlinewidth{0.000000pt}%
\definecolor{currentstroke}{rgb}{0.000000,0.000000,0.000000}%
\pgfsetstrokecolor{currentstroke}%
\pgfsetdash{}{0pt}%
\pgfpathmoveto{\pgfqpoint{1.459627in}{1.051662in}}%
\pgfpathlineto{\pgfqpoint{1.498195in}{1.053176in}}%
\pgfpathlineto{\pgfqpoint{1.459912in}{1.091137in}}%
\pgfpathlineto{\pgfqpoint{1.421210in}{1.089760in}}%
\pgfpathclose%
\pgfusepath{fill}%
\end{pgfscope}%
\begin{pgfscope}%
\pgfpathrectangle{\pgfqpoint{0.150000in}{0.150000in}}{\pgfqpoint{2.700000in}{1.950000in}}%
\pgfusepath{clip}%
\pgfsetbuttcap%
\pgfsetroundjoin%
\definecolor{currentfill}{rgb}{0.785432,0.811857,0.848851}%
\pgfsetfillcolor{currentfill}%
\pgfsetlinewidth{0.000000pt}%
\definecolor{currentstroke}{rgb}{0.000000,0.000000,0.000000}%
\pgfsetstrokecolor{currentstroke}%
\pgfsetdash{}{0pt}%
\pgfpathmoveto{\pgfqpoint{1.498052in}{1.013555in}}%
\pgfpathlineto{\pgfqpoint{1.536486in}{1.015207in}}%
\pgfpathlineto{\pgfqpoint{1.498195in}{1.053176in}}%
\pgfpathlineto{\pgfqpoint{1.459627in}{1.051662in}}%
\pgfpathclose%
\pgfusepath{fill}%
\end{pgfscope}%
\begin{pgfscope}%
\pgfpathrectangle{\pgfqpoint{0.150000in}{0.150000in}}{\pgfqpoint{2.700000in}{1.950000in}}%
\pgfusepath{clip}%
\pgfsetbuttcap%
\pgfsetroundjoin%
\definecolor{currentfill}{rgb}{0.816529,0.839124,0.870757}%
\pgfsetfillcolor{currentfill}%
\pgfsetlinewidth{0.000000pt}%
\definecolor{currentstroke}{rgb}{0.000000,0.000000,0.000000}%
\pgfsetstrokecolor{currentstroke}%
\pgfsetdash{}{0pt}%
\pgfpathmoveto{\pgfqpoint{1.536486in}{0.975440in}}%
\pgfpathlineto{\pgfqpoint{1.574786in}{0.977229in}}%
\pgfpathlineto{\pgfqpoint{1.536486in}{1.015207in}}%
\pgfpathlineto{\pgfqpoint{1.498052in}{1.013555in}}%
\pgfpathclose%
\pgfusepath{fill}%
\end{pgfscope}%
\begin{pgfscope}%
\pgfpathrectangle{\pgfqpoint{0.150000in}{0.150000in}}{\pgfqpoint{2.700000in}{1.950000in}}%
\pgfusepath{clip}%
\pgfsetbuttcap%
\pgfsetroundjoin%
\definecolor{currentfill}{rgb}{0.486903,0.550092,0.638557}%
\pgfsetfillcolor{currentfill}%
\pgfsetlinewidth{0.000000pt}%
\definecolor{currentstroke}{rgb}{0.000000,0.000000,0.000000}%
\pgfsetstrokecolor{currentstroke}%
\pgfsetdash{}{0pt}%
\pgfpathmoveto{\pgfqpoint{0.615519in}{1.349949in}}%
\pgfpathlineto{\pgfqpoint{0.653475in}{1.394246in}}%
\pgfpathlineto{\pgfqpoint{0.615453in}{1.429114in}}%
\pgfpathlineto{\pgfqpoint{0.577530in}{1.384818in}}%
\pgfpathclose%
\pgfusepath{fill}%
\end{pgfscope}%
\begin{pgfscope}%
\pgfpathrectangle{\pgfqpoint{0.150000in}{0.150000in}}{\pgfqpoint{2.700000in}{1.950000in}}%
\pgfusepath{clip}%
\pgfsetbuttcap%
\pgfsetroundjoin%
\definecolor{currentfill}{rgb}{0.841406,0.860938,0.888281}%
\pgfsetfillcolor{currentfill}%
\pgfsetlinewidth{0.000000pt}%
\definecolor{currentstroke}{rgb}{0.000000,0.000000,0.000000}%
\pgfsetstrokecolor{currentstroke}%
\pgfsetdash{}{0pt}%
\pgfpathmoveto{\pgfqpoint{1.574940in}{0.940333in}}%
\pgfpathlineto{\pgfqpoint{1.613117in}{0.942250in}}%
\pgfpathlineto{\pgfqpoint{1.574786in}{0.977229in}}%
\pgfpathlineto{\pgfqpoint{1.536486in}{0.975440in}}%
\pgfpathclose%
\pgfusepath{fill}%
\end{pgfscope}%
\begin{pgfscope}%
\pgfpathrectangle{\pgfqpoint{0.150000in}{0.150000in}}{\pgfqpoint{2.700000in}{1.950000in}}%
\pgfusepath{clip}%
\pgfsetbuttcap%
\pgfsetroundjoin%
\definecolor{currentfill}{rgb}{0.517999,0.577359,0.660463}%
\pgfsetfillcolor{currentfill}%
\pgfsetlinewidth{0.000000pt}%
\definecolor{currentstroke}{rgb}{0.000000,0.000000,0.000000}%
\pgfsetstrokecolor{currentstroke}%
\pgfsetdash{}{0pt}%
\pgfpathmoveto{\pgfqpoint{0.653796in}{1.311930in}}%
\pgfpathlineto{\pgfqpoint{0.691774in}{1.356214in}}%
\pgfpathlineto{\pgfqpoint{0.653475in}{1.394246in}}%
\pgfpathlineto{\pgfqpoint{0.615519in}{1.349949in}}%
\pgfpathclose%
\pgfusepath{fill}%
\end{pgfscope}%
\begin{pgfscope}%
\pgfpathrectangle{\pgfqpoint{0.150000in}{0.150000in}}{\pgfqpoint{2.700000in}{1.950000in}}%
\pgfusepath{clip}%
\pgfsetbuttcap%
\pgfsetroundjoin%
\definecolor{currentfill}{rgb}{0.866284,0.882751,0.905806}%
\pgfsetfillcolor{currentfill}%
\pgfsetlinewidth{0.000000pt}%
\definecolor{currentstroke}{rgb}{0.000000,0.000000,0.000000}%
\pgfsetstrokecolor{currentstroke}%
\pgfsetdash{}{0pt}%
\pgfpathmoveto{\pgfqpoint{1.613402in}{0.902191in}}%
\pgfpathlineto{\pgfqpoint{1.651445in}{0.904245in}}%
\pgfpathlineto{\pgfqpoint{1.613117in}{0.942250in}}%
\pgfpathlineto{\pgfqpoint{1.574940in}{0.940333in}}%
\pgfpathclose%
\pgfusepath{fill}%
\end{pgfscope}%
\begin{pgfscope}%
\pgfpathrectangle{\pgfqpoint{0.150000in}{0.150000in}}{\pgfqpoint{2.700000in}{1.950000in}}%
\pgfusepath{clip}%
\pgfsetbuttcap%
\pgfsetroundjoin%
\definecolor{currentfill}{rgb}{0.449586,0.517371,0.612270}%
\pgfsetfillcolor{currentfill}%
\pgfsetlinewidth{0.000000pt}%
\definecolor{currentstroke}{rgb}{0.000000,0.000000,0.000000}%
\pgfsetstrokecolor{currentstroke}%
\pgfsetdash{}{0pt}%
\pgfpathmoveto{\pgfqpoint{0.653475in}{1.394246in}}%
\pgfpathlineto{\pgfqpoint{0.691713in}{1.435426in}}%
\pgfpathlineto{\pgfqpoint{0.653411in}{1.473451in}}%
\pgfpathlineto{\pgfqpoint{0.615453in}{1.429114in}}%
\pgfpathclose%
\pgfusepath{fill}%
\end{pgfscope}%
\begin{pgfscope}%
\pgfpathrectangle{\pgfqpoint{0.150000in}{0.150000in}}{\pgfqpoint{2.700000in}{1.950000in}}%
\pgfusepath{clip}%
\pgfsetbuttcap%
\pgfsetroundjoin%
\definecolor{currentfill}{rgb}{0.897381,0.910018,0.927711}%
\pgfsetfillcolor{currentfill}%
\pgfsetlinewidth{0.000000pt}%
\definecolor{currentstroke}{rgb}{0.000000,0.000000,0.000000}%
\pgfsetstrokecolor{currentstroke}%
\pgfsetdash{}{0pt}%
\pgfpathmoveto{\pgfqpoint{1.651873in}{0.864040in}}%
\pgfpathlineto{\pgfqpoint{1.689781in}{0.866231in}}%
\pgfpathlineto{\pgfqpoint{1.651445in}{0.904245in}}%
\pgfpathlineto{\pgfqpoint{1.613402in}{0.902191in}}%
\pgfpathclose%
\pgfusepath{fill}%
\end{pgfscope}%
\begin{pgfscope}%
\pgfpathrectangle{\pgfqpoint{0.150000in}{0.150000in}}{\pgfqpoint{2.700000in}{1.950000in}}%
\pgfusepath{clip}%
\pgfsetbuttcap%
\pgfsetroundjoin%
\definecolor{currentfill}{rgb}{0.542877,0.599173,0.677987}%
\pgfsetfillcolor{currentfill}%
\pgfsetlinewidth{0.000000pt}%
\definecolor{currentstroke}{rgb}{0.000000,0.000000,0.000000}%
\pgfsetstrokecolor{currentstroke}%
\pgfsetdash{}{0pt}%
\pgfpathmoveto{\pgfqpoint{0.691835in}{1.277014in}}%
\pgfpathlineto{\pgfqpoint{0.729846in}{1.321299in}}%
\pgfpathlineto{\pgfqpoint{0.691774in}{1.356214in}}%
\pgfpathlineto{\pgfqpoint{0.653796in}{1.311930in}}%
\pgfpathclose%
\pgfusepath{fill}%
\end{pgfscope}%
\begin{pgfscope}%
\pgfpathrectangle{\pgfqpoint{0.150000in}{0.150000in}}{\pgfqpoint{2.700000in}{1.950000in}}%
\pgfusepath{clip}%
\pgfsetbuttcap%
\pgfsetroundjoin%
\definecolor{currentfill}{rgb}{0.928477,0.937286,0.949617}%
\pgfsetfillcolor{currentfill}%
\pgfsetlinewidth{0.000000pt}%
\definecolor{currentstroke}{rgb}{0.000000,0.000000,0.000000}%
\pgfsetstrokecolor{currentstroke}%
\pgfsetdash{}{0pt}%
\pgfpathmoveto{\pgfqpoint{1.690352in}{0.825880in}}%
\pgfpathlineto{\pgfqpoint{1.728126in}{0.828210in}}%
\pgfpathlineto{\pgfqpoint{1.689781in}{0.866231in}}%
\pgfpathlineto{\pgfqpoint{1.651873in}{0.864040in}}%
\pgfpathclose%
\pgfusepath{fill}%
\end{pgfscope}%
\begin{pgfscope}%
\pgfpathrectangle{\pgfqpoint{0.150000in}{0.150000in}}{\pgfqpoint{2.700000in}{1.950000in}}%
\pgfusepath{clip}%
\pgfsetbuttcap%
\pgfsetroundjoin%
\definecolor{currentfill}{rgb}{0.959574,0.964553,0.971523}%
\pgfsetfillcolor{currentfill}%
\pgfsetlinewidth{0.000000pt}%
\definecolor{currentstroke}{rgb}{0.000000,0.000000,0.000000}%
\pgfsetstrokecolor{currentstroke}%
\pgfsetdash{}{0pt}%
\pgfpathmoveto{\pgfqpoint{1.728840in}{0.787712in}}%
\pgfpathlineto{\pgfqpoint{1.766479in}{0.790180in}}%
\pgfpathlineto{\pgfqpoint{1.728126in}{0.828210in}}%
\pgfpathlineto{\pgfqpoint{1.690352in}{0.825880in}}%
\pgfpathclose%
\pgfusepath{fill}%
\end{pgfscope}%
\begin{pgfscope}%
\pgfpathrectangle{\pgfqpoint{0.150000in}{0.150000in}}{\pgfqpoint{2.700000in}{1.950000in}}%
\pgfusepath{clip}%
\pgfsetbuttcap%
\pgfsetroundjoin%
\definecolor{currentfill}{rgb}{0.474464,0.539185,0.629795}%
\pgfsetfillcolor{currentfill}%
\pgfsetlinewidth{0.000000pt}%
\definecolor{currentstroke}{rgb}{0.000000,0.000000,0.000000}%
\pgfsetstrokecolor{currentstroke}%
\pgfsetdash{}{0pt}%
\pgfpathmoveto{\pgfqpoint{0.691774in}{1.356214in}}%
\pgfpathlineto{\pgfqpoint{0.729788in}{1.400540in}}%
\pgfpathlineto{\pgfqpoint{0.691713in}{1.435426in}}%
\pgfpathlineto{\pgfqpoint{0.653475in}{1.394246in}}%
\pgfpathclose%
\pgfusepath{fill}%
\end{pgfscope}%
\begin{pgfscope}%
\pgfpathrectangle{\pgfqpoint{0.150000in}{0.150000in}}{\pgfqpoint{2.700000in}{1.950000in}}%
\pgfusepath{clip}%
\pgfsetbuttcap%
\pgfsetroundjoin%
\definecolor{currentfill}{rgb}{0.990671,0.991820,0.993428}%
\pgfsetfillcolor{currentfill}%
\pgfsetlinewidth{0.000000pt}%
\definecolor{currentstroke}{rgb}{0.000000,0.000000,0.000000}%
\pgfsetstrokecolor{currentstroke}%
\pgfsetdash{}{0pt}%
\pgfpathmoveto{\pgfqpoint{1.767336in}{0.749536in}}%
\pgfpathlineto{\pgfqpoint{1.804840in}{0.752141in}}%
\pgfpathlineto{\pgfqpoint{1.766479in}{0.790180in}}%
\pgfpathlineto{\pgfqpoint{1.728840in}{0.787712in}}%
\pgfpathclose%
\pgfusepath{fill}%
\end{pgfscope}%
\begin{pgfscope}%
\pgfpathrectangle{\pgfqpoint{0.150000in}{0.150000in}}{\pgfqpoint{2.700000in}{1.950000in}}%
\pgfusepath{clip}%
\pgfsetbuttcap%
\pgfsetroundjoin%
\definecolor{currentfill}{rgb}{0.412270,0.484651,0.585983}%
\pgfsetfillcolor{currentfill}%
\pgfsetlinewidth{0.000000pt}%
\definecolor{currentstroke}{rgb}{0.000000,0.000000,0.000000}%
\pgfsetstrokecolor{currentstroke}%
\pgfsetdash{}{0pt}%
\pgfpathmoveto{\pgfqpoint{0.691713in}{1.435426in}}%
\pgfpathlineto{\pgfqpoint{0.729730in}{1.479793in}}%
\pgfpathlineto{\pgfqpoint{0.691406in}{1.517831in}}%
\pgfpathlineto{\pgfqpoint{0.653411in}{1.473451in}}%
\pgfpathclose%
\pgfusepath{fill}%
\end{pgfscope}%
\begin{pgfscope}%
\pgfpathrectangle{\pgfqpoint{0.150000in}{0.150000in}}{\pgfqpoint{2.700000in}{1.950000in}}%
\pgfusepath{clip}%
\pgfsetbuttcap%
\pgfsetroundjoin%
\definecolor{currentfill}{rgb}{0.567754,0.620987,0.695512}%
\pgfsetfillcolor{currentfill}%
\pgfsetlinewidth{0.000000pt}%
\definecolor{currentstroke}{rgb}{0.000000,0.000000,0.000000}%
\pgfsetstrokecolor{currentstroke}%
\pgfsetdash{}{0pt}%
\pgfpathmoveto{\pgfqpoint{0.729904in}{1.242070in}}%
\pgfpathlineto{\pgfqpoint{0.768173in}{1.283241in}}%
\pgfpathlineto{\pgfqpoint{0.729846in}{1.321299in}}%
\pgfpathlineto{\pgfqpoint{0.691835in}{1.277014in}}%
\pgfpathclose%
\pgfusepath{fill}%
\end{pgfscope}%
\begin{pgfscope}%
\pgfpathrectangle{\pgfqpoint{0.150000in}{0.150000in}}{\pgfqpoint{2.700000in}{1.950000in}}%
\pgfusepath{clip}%
\pgfsetbuttcap%
\pgfsetroundjoin%
\definecolor{currentfill}{rgb}{0.505561,0.566452,0.651700}%
\pgfsetfillcolor{currentfill}%
\pgfsetlinewidth{0.000000pt}%
\definecolor{currentstroke}{rgb}{0.000000,0.000000,0.000000}%
\pgfsetstrokecolor{currentstroke}%
\pgfsetdash{}{0pt}%
\pgfpathmoveto{\pgfqpoint{0.729846in}{1.321299in}}%
\pgfpathlineto{\pgfqpoint{0.768118in}{1.362487in}}%
\pgfpathlineto{\pgfqpoint{0.729788in}{1.400540in}}%
\pgfpathlineto{\pgfqpoint{0.691774in}{1.356214in}}%
\pgfpathclose%
\pgfusepath{fill}%
\end{pgfscope}%
\begin{pgfscope}%
\pgfpathrectangle{\pgfqpoint{0.150000in}{0.150000in}}{\pgfqpoint{2.700000in}{1.950000in}}%
\pgfusepath{clip}%
\pgfsetbuttcap%
\pgfsetroundjoin%
\definecolor{currentfill}{rgb}{0.986703,0.975873,0.976731}%
\pgfsetfillcolor{currentfill}%
\pgfsetlinewidth{0.000000pt}%
\definecolor{currentstroke}{rgb}{0.000000,0.000000,0.000000}%
\pgfsetstrokecolor{currentstroke}%
\pgfsetdash{}{0pt}%
\pgfpathmoveto{\pgfqpoint{1.805841in}{0.711351in}}%
\pgfpathlineto{\pgfqpoint{1.843299in}{0.717038in}}%
\pgfpathlineto{\pgfqpoint{1.804840in}{0.752141in}}%
\pgfpathlineto{\pgfqpoint{1.767336in}{0.749536in}}%
\pgfpathclose%
\pgfusepath{fill}%
\end{pgfscope}%
\begin{pgfscope}%
\pgfpathrectangle{\pgfqpoint{0.150000in}{0.150000in}}{\pgfqpoint{2.700000in}{1.950000in}}%
\pgfusepath{clip}%
\pgfsetbuttcap%
\pgfsetroundjoin%
\definecolor{currentfill}{rgb}{0.281664,0.370129,0.493980}%
\pgfsetfillcolor{currentfill}%
\pgfsetlinewidth{0.000000pt}%
\definecolor{currentstroke}{rgb}{0.000000,0.000000,0.000000}%
\pgfsetstrokecolor{currentstroke}%
\pgfsetdash{}{0pt}%
\pgfpathmoveto{\pgfqpoint{0.843997in}{1.613145in}}%
\pgfpathlineto{\pgfqpoint{0.884446in}{1.619050in}}%
\pgfpathlineto{\pgfqpoint{0.846166in}{1.657012in}}%
\pgfpathlineto{\pgfqpoint{0.805605in}{1.651223in}}%
\pgfpathclose%
\pgfusepath{fill}%
\end{pgfscope}%
\begin{pgfscope}%
\pgfpathrectangle{\pgfqpoint{0.150000in}{0.150000in}}{\pgfqpoint{2.700000in}{1.950000in}}%
\pgfusepath{clip}%
\pgfsetbuttcap%
\pgfsetroundjoin%
\definecolor{currentfill}{rgb}{0.437148,0.506464,0.603508}%
\pgfsetfillcolor{currentfill}%
\pgfsetlinewidth{0.000000pt}%
\definecolor{currentstroke}{rgb}{0.000000,0.000000,0.000000}%
\pgfsetstrokecolor{currentstroke}%
\pgfsetdash{}{0pt}%
\pgfpathmoveto{\pgfqpoint{0.729788in}{1.400540in}}%
\pgfpathlineto{\pgfqpoint{0.767838in}{1.444908in}}%
\pgfpathlineto{\pgfqpoint{0.729730in}{1.479793in}}%
\pgfpathlineto{\pgfqpoint{0.691713in}{1.435426in}}%
\pgfpathclose%
\pgfusepath{fill}%
\end{pgfscope}%
\begin{pgfscope}%
\pgfpathrectangle{\pgfqpoint{0.150000in}{0.150000in}}{\pgfqpoint{2.700000in}{1.950000in}}%
\pgfusepath{clip}%
\pgfsetbuttcap%
\pgfsetroundjoin%
\definecolor{currentfill}{rgb}{0.312760,0.397396,0.515885}%
\pgfsetfillcolor{currentfill}%
\pgfsetlinewidth{0.000000pt}%
\definecolor{currentstroke}{rgb}{0.000000,0.000000,0.000000}%
\pgfsetstrokecolor{currentstroke}%
\pgfsetdash{}{0pt}%
\pgfpathmoveto{\pgfqpoint{0.882396in}{1.575059in}}%
\pgfpathlineto{\pgfqpoint{0.922734in}{1.581079in}}%
\pgfpathlineto{\pgfqpoint{0.884446in}{1.619050in}}%
\pgfpathlineto{\pgfqpoint{0.843997in}{1.613145in}}%
\pgfpathclose%
\pgfusepath{fill}%
\end{pgfscope}%
\begin{pgfscope}%
\pgfpathrectangle{\pgfqpoint{0.150000in}{0.150000in}}{\pgfqpoint{2.700000in}{1.950000in}}%
\pgfusepath{clip}%
\pgfsetbuttcap%
\pgfsetroundjoin%
\definecolor{currentfill}{rgb}{0.598851,0.648254,0.717417}%
\pgfsetfillcolor{currentfill}%
\pgfsetlinewidth{0.000000pt}%
\definecolor{currentstroke}{rgb}{0.000000,0.000000,0.000000}%
\pgfsetstrokecolor{currentstroke}%
\pgfsetdash{}{0pt}%
\pgfpathmoveto{\pgfqpoint{0.768228in}{1.204005in}}%
\pgfpathlineto{\pgfqpoint{0.806295in}{1.248278in}}%
\pgfpathlineto{\pgfqpoint{0.768173in}{1.283241in}}%
\pgfpathlineto{\pgfqpoint{0.729904in}{1.242070in}}%
\pgfpathclose%
\pgfusepath{fill}%
\end{pgfscope}%
\begin{pgfscope}%
\pgfpathrectangle{\pgfqpoint{0.150000in}{0.150000in}}{\pgfqpoint{2.700000in}{1.950000in}}%
\pgfusepath{clip}%
\pgfsetbuttcap%
\pgfsetroundjoin%
\definecolor{currentfill}{rgb}{0.337638,0.419210,0.533410}%
\pgfsetfillcolor{currentfill}%
\pgfsetlinewidth{0.000000pt}%
\definecolor{currentstroke}{rgb}{0.000000,0.000000,0.000000}%
\pgfsetstrokecolor{currentstroke}%
\pgfsetdash{}{0pt}%
\pgfpathmoveto{\pgfqpoint{0.920804in}{1.536964in}}%
\pgfpathlineto{\pgfqpoint{0.961030in}{1.543100in}}%
\pgfpathlineto{\pgfqpoint{0.922734in}{1.581079in}}%
\pgfpathlineto{\pgfqpoint{0.882396in}{1.575059in}}%
\pgfpathclose%
\pgfusepath{fill}%
\end{pgfscope}%
\begin{pgfscope}%
\pgfpathrectangle{\pgfqpoint{0.150000in}{0.150000in}}{\pgfqpoint{2.700000in}{1.950000in}}%
\pgfusepath{clip}%
\pgfsetbuttcap%
\pgfsetroundjoin%
\definecolor{currentfill}{rgb}{0.530438,0.588266,0.669225}%
\pgfsetfillcolor{currentfill}%
\pgfsetlinewidth{0.000000pt}%
\definecolor{currentstroke}{rgb}{0.000000,0.000000,0.000000}%
\pgfsetstrokecolor{currentstroke}%
\pgfsetdash{}{0pt}%
\pgfpathmoveto{\pgfqpoint{0.768173in}{1.283241in}}%
\pgfpathlineto{\pgfqpoint{0.806243in}{1.327555in}}%
\pgfpathlineto{\pgfqpoint{0.768118in}{1.362487in}}%
\pgfpathlineto{\pgfqpoint{0.729846in}{1.321299in}}%
\pgfpathclose%
\pgfusepath{fill}%
\end{pgfscope}%
\begin{pgfscope}%
\pgfpathrectangle{\pgfqpoint{0.150000in}{0.150000in}}{\pgfqpoint{2.700000in}{1.950000in}}%
\pgfusepath{clip}%
\pgfsetbuttcap%
\pgfsetroundjoin%
\definecolor{currentfill}{rgb}{0.368735,0.446477,0.555316}%
\pgfsetfillcolor{currentfill}%
\pgfsetlinewidth{0.000000pt}%
\definecolor{currentstroke}{rgb}{0.000000,0.000000,0.000000}%
\pgfsetstrokecolor{currentstroke}%
\pgfsetdash{}{0pt}%
\pgfpathmoveto{\pgfqpoint{0.729730in}{1.479793in}}%
\pgfpathlineto{\pgfqpoint{0.767783in}{1.524201in}}%
\pgfpathlineto{\pgfqpoint{0.729436in}{1.562253in}}%
\pgfpathlineto{\pgfqpoint{0.691406in}{1.517831in}}%
\pgfpathclose%
\pgfusepath{fill}%
\end{pgfscope}%
\begin{pgfscope}%
\pgfpathrectangle{\pgfqpoint{0.150000in}{0.150000in}}{\pgfqpoint{2.700000in}{1.950000in}}%
\pgfusepath{clip}%
\pgfsetbuttcap%
\pgfsetroundjoin%
\definecolor{currentfill}{rgb}{0.368735,0.446477,0.555316}%
\pgfsetfillcolor{currentfill}%
\pgfsetlinewidth{0.000000pt}%
\definecolor{currentstroke}{rgb}{0.000000,0.000000,0.000000}%
\pgfsetstrokecolor{currentstroke}%
\pgfsetdash{}{0pt}%
\pgfpathmoveto{\pgfqpoint{0.959221in}{1.498861in}}%
\pgfpathlineto{\pgfqpoint{0.999335in}{1.505113in}}%
\pgfpathlineto{\pgfqpoint{0.961030in}{1.543100in}}%
\pgfpathlineto{\pgfqpoint{0.920804in}{1.536964in}}%
\pgfpathclose%
\pgfusepath{fill}%
\end{pgfscope}%
\begin{pgfscope}%
\pgfpathrectangle{\pgfqpoint{0.150000in}{0.150000in}}{\pgfqpoint{2.700000in}{1.950000in}}%
\pgfusepath{clip}%
\pgfsetbuttcap%
\pgfsetroundjoin%
\definecolor{currentfill}{rgb}{0.468244,0.533732,0.625414}%
\pgfsetfillcolor{currentfill}%
\pgfsetlinewidth{0.000000pt}%
\definecolor{currentstroke}{rgb}{0.000000,0.000000,0.000000}%
\pgfsetstrokecolor{currentstroke}%
\pgfsetdash{}{0pt}%
\pgfpathmoveto{\pgfqpoint{0.768118in}{1.362487in}}%
\pgfpathlineto{\pgfqpoint{0.806190in}{1.406842in}}%
\pgfpathlineto{\pgfqpoint{0.767838in}{1.444908in}}%
\pgfpathlineto{\pgfqpoint{0.729788in}{1.400540in}}%
\pgfpathclose%
\pgfusepath{fill}%
\end{pgfscope}%
\begin{pgfscope}%
\pgfpathrectangle{\pgfqpoint{0.150000in}{0.150000in}}{\pgfqpoint{2.700000in}{1.950000in}}%
\pgfusepath{clip}%
\pgfsetbuttcap%
\pgfsetroundjoin%
\definecolor{currentfill}{rgb}{0.623729,0.670067,0.734942}%
\pgfsetfillcolor{currentfill}%
\pgfsetlinewidth{0.000000pt}%
\definecolor{currentstroke}{rgb}{0.000000,0.000000,0.000000}%
\pgfsetstrokecolor{currentstroke}%
\pgfsetdash{}{0pt}%
\pgfpathmoveto{\pgfqpoint{0.806348in}{1.169014in}}%
\pgfpathlineto{\pgfqpoint{0.844650in}{1.210192in}}%
\pgfpathlineto{\pgfqpoint{0.806295in}{1.248278in}}%
\pgfpathlineto{\pgfqpoint{0.768228in}{1.204005in}}%
\pgfpathclose%
\pgfusepath{fill}%
\end{pgfscope}%
\begin{pgfscope}%
\pgfpathrectangle{\pgfqpoint{0.150000in}{0.150000in}}{\pgfqpoint{2.700000in}{1.950000in}}%
\pgfusepath{clip}%
\pgfsetbuttcap%
\pgfsetroundjoin%
\definecolor{currentfill}{rgb}{0.399831,0.473744,0.577221}%
\pgfsetfillcolor{currentfill}%
\pgfsetlinewidth{0.000000pt}%
\definecolor{currentstroke}{rgb}{0.000000,0.000000,0.000000}%
\pgfsetstrokecolor{currentstroke}%
\pgfsetdash{}{0pt}%
\pgfpathmoveto{\pgfqpoint{0.997646in}{1.460749in}}%
\pgfpathlineto{\pgfqpoint{1.037648in}{1.467117in}}%
\pgfpathlineto{\pgfqpoint{0.999335in}{1.505113in}}%
\pgfpathlineto{\pgfqpoint{0.959221in}{1.498861in}}%
\pgfpathclose%
\pgfusepath{fill}%
\end{pgfscope}%
\begin{pgfscope}%
\pgfpathrectangle{\pgfqpoint{0.150000in}{0.150000in}}{\pgfqpoint{2.700000in}{1.950000in}}%
\pgfusepath{clip}%
\pgfsetbuttcap%
\pgfsetroundjoin%
\definecolor{currentfill}{rgb}{0.561535,0.615533,0.691131}%
\pgfsetfillcolor{currentfill}%
\pgfsetlinewidth{0.000000pt}%
\definecolor{currentstroke}{rgb}{0.000000,0.000000,0.000000}%
\pgfsetstrokecolor{currentstroke}%
\pgfsetdash{}{0pt}%
\pgfpathmoveto{\pgfqpoint{0.806295in}{1.248278in}}%
\pgfpathlineto{\pgfqpoint{0.844600in}{1.289474in}}%
\pgfpathlineto{\pgfqpoint{0.806243in}{1.327555in}}%
\pgfpathlineto{\pgfqpoint{0.768173in}{1.283241in}}%
\pgfpathclose%
\pgfusepath{fill}%
\end{pgfscope}%
\begin{pgfscope}%
\pgfpathrectangle{\pgfqpoint{0.150000in}{0.150000in}}{\pgfqpoint{2.700000in}{1.950000in}}%
\pgfusepath{clip}%
\pgfsetbuttcap%
\pgfsetroundjoin%
\definecolor{currentfill}{rgb}{0.399831,0.473744,0.577221}%
\pgfsetfillcolor{currentfill}%
\pgfsetlinewidth{0.000000pt}%
\definecolor{currentstroke}{rgb}{0.000000,0.000000,0.000000}%
\pgfsetstrokecolor{currentstroke}%
\pgfsetdash{}{0pt}%
\pgfpathmoveto{\pgfqpoint{0.767838in}{1.444908in}}%
\pgfpathlineto{\pgfqpoint{0.806137in}{1.486141in}}%
\pgfpathlineto{\pgfqpoint{0.767783in}{1.524201in}}%
\pgfpathlineto{\pgfqpoint{0.729730in}{1.479793in}}%
\pgfpathclose%
\pgfusepath{fill}%
\end{pgfscope}%
\begin{pgfscope}%
\pgfpathrectangle{\pgfqpoint{0.150000in}{0.150000in}}{\pgfqpoint{2.700000in}{1.950000in}}%
\pgfusepath{clip}%
\pgfsetbuttcap%
\pgfsetroundjoin%
\definecolor{currentfill}{rgb}{0.493122,0.555545,0.642938}%
\pgfsetfillcolor{currentfill}%
\pgfsetlinewidth{0.000000pt}%
\definecolor{currentstroke}{rgb}{0.000000,0.000000,0.000000}%
\pgfsetstrokecolor{currentstroke}%
\pgfsetdash{}{0pt}%
\pgfpathmoveto{\pgfqpoint{0.806243in}{1.327555in}}%
\pgfpathlineto{\pgfqpoint{0.844551in}{1.368768in}}%
\pgfpathlineto{\pgfqpoint{0.806190in}{1.406842in}}%
\pgfpathlineto{\pgfqpoint{0.768118in}{1.362487in}}%
\pgfpathclose%
\pgfusepath{fill}%
\end{pgfscope}%
\begin{pgfscope}%
\pgfpathrectangle{\pgfqpoint{0.150000in}{0.150000in}}{\pgfqpoint{2.700000in}{1.950000in}}%
\pgfusepath{clip}%
\pgfsetbuttcap%
\pgfsetroundjoin%
\definecolor{currentfill}{rgb}{0.430928,0.501011,0.599127}%
\pgfsetfillcolor{currentfill}%
\pgfsetlinewidth{0.000000pt}%
\definecolor{currentstroke}{rgb}{0.000000,0.000000,0.000000}%
\pgfsetstrokecolor{currentstroke}%
\pgfsetdash{}{0pt}%
\pgfpathmoveto{\pgfqpoint{1.036079in}{1.422629in}}%
\pgfpathlineto{\pgfqpoint{1.075970in}{1.429114in}}%
\pgfpathlineto{\pgfqpoint{1.037648in}{1.467117in}}%
\pgfpathlineto{\pgfqpoint{0.997646in}{1.460749in}}%
\pgfpathclose%
\pgfusepath{fill}%
\end{pgfscope}%
\begin{pgfscope}%
\pgfpathrectangle{\pgfqpoint{0.150000in}{0.150000in}}{\pgfqpoint{2.700000in}{1.950000in}}%
\pgfusepath{clip}%
\pgfsetbuttcap%
\pgfsetroundjoin%
\definecolor{currentfill}{rgb}{0.462025,0.528278,0.621032}%
\pgfsetfillcolor{currentfill}%
\pgfsetlinewidth{0.000000pt}%
\definecolor{currentstroke}{rgb}{0.000000,0.000000,0.000000}%
\pgfsetstrokecolor{currentstroke}%
\pgfsetdash{}{0pt}%
\pgfpathmoveto{\pgfqpoint{1.074521in}{1.384501in}}%
\pgfpathlineto{\pgfqpoint{1.114177in}{1.394246in}}%
\pgfpathlineto{\pgfqpoint{1.075970in}{1.429114in}}%
\pgfpathlineto{\pgfqpoint{1.036079in}{1.422629in}}%
\pgfpathclose%
\pgfusepath{fill}%
\end{pgfscope}%
\begin{pgfscope}%
\pgfpathrectangle{\pgfqpoint{0.150000in}{0.150000in}}{\pgfqpoint{2.700000in}{1.950000in}}%
\pgfusepath{clip}%
\pgfsetbuttcap%
\pgfsetroundjoin%
\definecolor{currentfill}{rgb}{0.586412,0.637347,0.708655}%
\pgfsetfillcolor{currentfill}%
\pgfsetlinewidth{0.000000pt}%
\definecolor{currentstroke}{rgb}{0.000000,0.000000,0.000000}%
\pgfsetstrokecolor{currentstroke}%
\pgfsetdash{}{0pt}%
\pgfpathmoveto{\pgfqpoint{0.844650in}{1.210192in}}%
\pgfpathlineto{\pgfqpoint{0.882776in}{1.254494in}}%
\pgfpathlineto{\pgfqpoint{0.844600in}{1.289474in}}%
\pgfpathlineto{\pgfqpoint{0.806295in}{1.248278in}}%
\pgfpathclose%
\pgfusepath{fill}%
\end{pgfscope}%
\begin{pgfscope}%
\pgfpathrectangle{\pgfqpoint{0.150000in}{0.150000in}}{\pgfqpoint{2.700000in}{1.950000in}}%
\pgfusepath{clip}%
\pgfsetbuttcap%
\pgfsetroundjoin%
\definecolor{currentfill}{rgb}{0.654825,0.697335,0.756847}%
\pgfsetfillcolor{currentfill}%
\pgfsetlinewidth{0.000000pt}%
\definecolor{currentstroke}{rgb}{0.000000,0.000000,0.000000}%
\pgfsetstrokecolor{currentstroke}%
\pgfsetdash{}{0pt}%
\pgfpathmoveto{\pgfqpoint{0.844498in}{1.133994in}}%
\pgfpathlineto{\pgfqpoint{0.882823in}{1.175183in}}%
\pgfpathlineto{\pgfqpoint{0.844650in}{1.210192in}}%
\pgfpathlineto{\pgfqpoint{0.806348in}{1.169014in}}%
\pgfpathclose%
\pgfusepath{fill}%
\end{pgfscope}%
\begin{pgfscope}%
\pgfpathrectangle{\pgfqpoint{0.150000in}{0.150000in}}{\pgfqpoint{2.700000in}{1.950000in}}%
\pgfusepath{clip}%
\pgfsetbuttcap%
\pgfsetroundjoin%
\definecolor{currentfill}{rgb}{0.424709,0.495558,0.594746}%
\pgfsetfillcolor{currentfill}%
\pgfsetlinewidth{0.000000pt}%
\definecolor{currentstroke}{rgb}{0.000000,0.000000,0.000000}%
\pgfsetstrokecolor{currentstroke}%
\pgfsetdash{}{0pt}%
\pgfpathmoveto{\pgfqpoint{0.806190in}{1.406842in}}%
\pgfpathlineto{\pgfqpoint{0.844299in}{1.451239in}}%
\pgfpathlineto{\pgfqpoint{0.806137in}{1.486141in}}%
\pgfpathlineto{\pgfqpoint{0.767838in}{1.444908in}}%
\pgfpathclose%
\pgfusepath{fill}%
\end{pgfscope}%
\begin{pgfscope}%
\pgfpathrectangle{\pgfqpoint{0.150000in}{0.150000in}}{\pgfqpoint{2.700000in}{1.950000in}}%
\pgfusepath{clip}%
\pgfsetbuttcap%
\pgfsetroundjoin%
\definecolor{currentfill}{rgb}{0.331419,0.413756,0.529029}%
\pgfsetfillcolor{currentfill}%
\pgfsetlinewidth{0.000000pt}%
\definecolor{currentstroke}{rgb}{0.000000,0.000000,0.000000}%
\pgfsetstrokecolor{currentstroke}%
\pgfsetdash{}{0pt}%
\pgfpathmoveto{\pgfqpoint{0.767783in}{1.524201in}}%
\pgfpathlineto{\pgfqpoint{0.805872in}{1.568652in}}%
\pgfpathlineto{\pgfqpoint{0.767503in}{1.606717in}}%
\pgfpathlineto{\pgfqpoint{0.729436in}{1.562253in}}%
\pgfpathclose%
\pgfusepath{fill}%
\end{pgfscope}%
\begin{pgfscope}%
\pgfpathrectangle{\pgfqpoint{0.150000in}{0.150000in}}{\pgfqpoint{2.700000in}{1.950000in}}%
\pgfusepath{clip}%
\pgfsetbuttcap%
\pgfsetroundjoin%
\definecolor{currentfill}{rgb}{0.524219,0.582812,0.664844}%
\pgfsetfillcolor{currentfill}%
\pgfsetlinewidth{0.000000pt}%
\definecolor{currentstroke}{rgb}{0.000000,0.000000,0.000000}%
\pgfsetstrokecolor{currentstroke}%
\pgfsetdash{}{0pt}%
\pgfpathmoveto{\pgfqpoint{0.844600in}{1.289474in}}%
\pgfpathlineto{\pgfqpoint{0.882729in}{1.333817in}}%
\pgfpathlineto{\pgfqpoint{0.844551in}{1.368768in}}%
\pgfpathlineto{\pgfqpoint{0.806243in}{1.327555in}}%
\pgfpathclose%
\pgfusepath{fill}%
\end{pgfscope}%
\begin{pgfscope}%
\pgfpathrectangle{\pgfqpoint{0.150000in}{0.150000in}}{\pgfqpoint{2.700000in}{1.950000in}}%
\pgfusepath{clip}%
\pgfsetbuttcap%
\pgfsetroundjoin%
\definecolor{currentfill}{rgb}{0.362515,0.441023,0.550934}%
\pgfsetfillcolor{currentfill}%
\pgfsetlinewidth{0.000000pt}%
\definecolor{currentstroke}{rgb}{0.000000,0.000000,0.000000}%
\pgfsetstrokecolor{currentstroke}%
\pgfsetdash{}{0pt}%
\pgfpathmoveto{\pgfqpoint{0.806137in}{1.486141in}}%
\pgfpathlineto{\pgfqpoint{0.844249in}{1.530579in}}%
\pgfpathlineto{\pgfqpoint{0.805872in}{1.568652in}}%
\pgfpathlineto{\pgfqpoint{0.767783in}{1.524201in}}%
\pgfpathclose%
\pgfusepath{fill}%
\end{pgfscope}%
\begin{pgfscope}%
\pgfpathrectangle{\pgfqpoint{0.150000in}{0.150000in}}{\pgfqpoint{2.700000in}{1.950000in}}%
\pgfusepath{clip}%
\pgfsetbuttcap%
\pgfsetroundjoin%
\definecolor{currentfill}{rgb}{0.486903,0.550092,0.638557}%
\pgfsetfillcolor{currentfill}%
\pgfsetlinewidth{0.000000pt}%
\definecolor{currentstroke}{rgb}{0.000000,0.000000,0.000000}%
\pgfsetstrokecolor{currentstroke}%
\pgfsetdash{}{0pt}%
\pgfpathmoveto{\pgfqpoint{1.112972in}{1.346364in}}%
\pgfpathlineto{\pgfqpoint{1.152526in}{1.356214in}}%
\pgfpathlineto{\pgfqpoint{1.114177in}{1.394246in}}%
\pgfpathlineto{\pgfqpoint{1.074521in}{1.384501in}}%
\pgfpathclose%
\pgfusepath{fill}%
\end{pgfscope}%
\begin{pgfscope}%
\pgfpathrectangle{\pgfqpoint{0.150000in}{0.150000in}}{\pgfqpoint{2.700000in}{1.950000in}}%
\pgfusepath{clip}%
\pgfsetbuttcap%
\pgfsetroundjoin%
\definecolor{currentfill}{rgb}{0.455806,0.522825,0.616651}%
\pgfsetfillcolor{currentfill}%
\pgfsetlinewidth{0.000000pt}%
\definecolor{currentstroke}{rgb}{0.000000,0.000000,0.000000}%
\pgfsetstrokecolor{currentstroke}%
\pgfsetdash{}{0pt}%
\pgfpathmoveto{\pgfqpoint{0.844551in}{1.368768in}}%
\pgfpathlineto{\pgfqpoint{0.882681in}{1.413152in}}%
\pgfpathlineto{\pgfqpoint{0.844299in}{1.451239in}}%
\pgfpathlineto{\pgfqpoint{0.806190in}{1.406842in}}%
\pgfpathclose%
\pgfusepath{fill}%
\end{pgfscope}%
\begin{pgfscope}%
\pgfpathrectangle{\pgfqpoint{0.150000in}{0.150000in}}{\pgfqpoint{2.700000in}{1.950000in}}%
\pgfusepath{clip}%
\pgfsetbuttcap%
\pgfsetroundjoin%
\definecolor{currentfill}{rgb}{0.617509,0.664614,0.730561}%
\pgfsetfillcolor{currentfill}%
\pgfsetlinewidth{0.000000pt}%
\definecolor{currentstroke}{rgb}{0.000000,0.000000,0.000000}%
\pgfsetstrokecolor{currentstroke}%
\pgfsetdash{}{0pt}%
\pgfpathmoveto{\pgfqpoint{0.882823in}{1.175183in}}%
\pgfpathlineto{\pgfqpoint{0.921161in}{1.216386in}}%
\pgfpathlineto{\pgfqpoint{0.882776in}{1.254494in}}%
\pgfpathlineto{\pgfqpoint{0.844650in}{1.210192in}}%
\pgfpathclose%
\pgfusepath{fill}%
\end{pgfscope}%
\begin{pgfscope}%
\pgfpathrectangle{\pgfqpoint{0.150000in}{0.150000in}}{\pgfqpoint{2.700000in}{1.950000in}}%
\pgfusepath{clip}%
\pgfsetbuttcap%
\pgfsetroundjoin%
\definecolor{currentfill}{rgb}{0.679703,0.719148,0.774372}%
\pgfsetfillcolor{currentfill}%
\pgfsetlinewidth{0.000000pt}%
\definecolor{currentstroke}{rgb}{0.000000,0.000000,0.000000}%
\pgfsetstrokecolor{currentstroke}%
\pgfsetdash{}{0pt}%
\pgfpathmoveto{\pgfqpoint{0.882870in}{1.095882in}}%
\pgfpathlineto{\pgfqpoint{0.921206in}{1.137069in}}%
\pgfpathlineto{\pgfqpoint{0.882823in}{1.175183in}}%
\pgfpathlineto{\pgfqpoint{0.844498in}{1.133994in}}%
\pgfpathclose%
\pgfusepath{fill}%
\end{pgfscope}%
\begin{pgfscope}%
\pgfpathrectangle{\pgfqpoint{0.150000in}{0.150000in}}{\pgfqpoint{2.700000in}{1.950000in}}%
\pgfusepath{clip}%
\pgfsetbuttcap%
\pgfsetroundjoin%
\definecolor{currentfill}{rgb}{0.517999,0.577359,0.660463}%
\pgfsetfillcolor{currentfill}%
\pgfsetlinewidth{0.000000pt}%
\definecolor{currentstroke}{rgb}{0.000000,0.000000,0.000000}%
\pgfsetstrokecolor{currentstroke}%
\pgfsetdash{}{0pt}%
\pgfpathmoveto{\pgfqpoint{1.151431in}{1.308219in}}%
\pgfpathlineto{\pgfqpoint{1.190884in}{1.318175in}}%
\pgfpathlineto{\pgfqpoint{1.152526in}{1.356214in}}%
\pgfpathlineto{\pgfqpoint{1.112972in}{1.346364in}}%
\pgfpathclose%
\pgfusepath{fill}%
\end{pgfscope}%
\begin{pgfscope}%
\pgfpathrectangle{\pgfqpoint{0.150000in}{0.150000in}}{\pgfqpoint{2.700000in}{1.950000in}}%
\pgfusepath{clip}%
\pgfsetbuttcap%
\pgfsetroundjoin%
\definecolor{currentfill}{rgb}{0.549096,0.604626,0.682368}%
\pgfsetfillcolor{currentfill}%
\pgfsetlinewidth{0.000000pt}%
\definecolor{currentstroke}{rgb}{0.000000,0.000000,0.000000}%
\pgfsetstrokecolor{currentstroke}%
\pgfsetdash{}{0pt}%
\pgfpathmoveto{\pgfqpoint{0.882776in}{1.254494in}}%
\pgfpathlineto{\pgfqpoint{0.921117in}{1.295715in}}%
\pgfpathlineto{\pgfqpoint{0.882729in}{1.333817in}}%
\pgfpathlineto{\pgfqpoint{0.844600in}{1.289474in}}%
\pgfpathclose%
\pgfusepath{fill}%
\end{pgfscope}%
\begin{pgfscope}%
\pgfpathrectangle{\pgfqpoint{0.150000in}{0.150000in}}{\pgfqpoint{2.700000in}{1.950000in}}%
\pgfusepath{clip}%
\pgfsetbuttcap%
\pgfsetroundjoin%
\definecolor{currentfill}{rgb}{0.387393,0.462837,0.568459}%
\pgfsetfillcolor{currentfill}%
\pgfsetlinewidth{0.000000pt}%
\definecolor{currentstroke}{rgb}{0.000000,0.000000,0.000000}%
\pgfsetstrokecolor{currentstroke}%
\pgfsetdash{}{0pt}%
\pgfpathmoveto{\pgfqpoint{0.844299in}{1.451239in}}%
\pgfpathlineto{\pgfqpoint{0.882634in}{1.492497in}}%
\pgfpathlineto{\pgfqpoint{0.844249in}{1.530579in}}%
\pgfpathlineto{\pgfqpoint{0.806137in}{1.486141in}}%
\pgfpathclose%
\pgfusepath{fill}%
\end{pgfscope}%
\begin{pgfscope}%
\pgfpathrectangle{\pgfqpoint{0.150000in}{0.150000in}}{\pgfqpoint{2.700000in}{1.950000in}}%
\pgfusepath{clip}%
\pgfsetbuttcap%
\pgfsetroundjoin%
\definecolor{currentfill}{rgb}{0.486903,0.550092,0.638557}%
\pgfsetfillcolor{currentfill}%
\pgfsetlinewidth{0.000000pt}%
\definecolor{currentstroke}{rgb}{0.000000,0.000000,0.000000}%
\pgfsetstrokecolor{currentstroke}%
\pgfsetdash{}{0pt}%
\pgfpathmoveto{\pgfqpoint{0.882729in}{1.333817in}}%
\pgfpathlineto{\pgfqpoint{0.921073in}{1.375056in}}%
\pgfpathlineto{\pgfqpoint{0.882681in}{1.413152in}}%
\pgfpathlineto{\pgfqpoint{0.844551in}{1.368768in}}%
\pgfpathclose%
\pgfusepath{fill}%
\end{pgfscope}%
\begin{pgfscope}%
\pgfpathrectangle{\pgfqpoint{0.150000in}{0.150000in}}{\pgfqpoint{2.700000in}{1.950000in}}%
\pgfusepath{clip}%
\pgfsetbuttcap%
\pgfsetroundjoin%
\definecolor{currentfill}{rgb}{0.549096,0.604626,0.682368}%
\pgfsetfillcolor{currentfill}%
\pgfsetlinewidth{0.000000pt}%
\definecolor{currentstroke}{rgb}{0.000000,0.000000,0.000000}%
\pgfsetstrokecolor{currentstroke}%
\pgfsetdash{}{0pt}%
\pgfpathmoveto{\pgfqpoint{1.189898in}{1.270065in}}%
\pgfpathlineto{\pgfqpoint{1.229251in}{1.280126in}}%
\pgfpathlineto{\pgfqpoint{1.190884in}{1.318175in}}%
\pgfpathlineto{\pgfqpoint{1.151431in}{1.308219in}}%
\pgfpathclose%
\pgfusepath{fill}%
\end{pgfscope}%
\begin{pgfscope}%
\pgfpathrectangle{\pgfqpoint{0.150000in}{0.150000in}}{\pgfqpoint{2.700000in}{1.950000in}}%
\pgfusepath{clip}%
\pgfsetbuttcap%
\pgfsetroundjoin%
\definecolor{currentfill}{rgb}{0.580193,0.631893,0.704274}%
\pgfsetfillcolor{currentfill}%
\pgfsetlinewidth{0.000000pt}%
\definecolor{currentstroke}{rgb}{0.000000,0.000000,0.000000}%
\pgfsetstrokecolor{currentstroke}%
\pgfsetdash{}{0pt}%
\pgfpathmoveto{\pgfqpoint{0.921161in}{1.216386in}}%
\pgfpathlineto{\pgfqpoint{0.959514in}{1.257605in}}%
\pgfpathlineto{\pgfqpoint{0.921117in}{1.295715in}}%
\pgfpathlineto{\pgfqpoint{0.882776in}{1.254494in}}%
\pgfpathclose%
\pgfusepath{fill}%
\end{pgfscope}%
\begin{pgfscope}%
\pgfpathrectangle{\pgfqpoint{0.150000in}{0.150000in}}{\pgfqpoint{2.700000in}{1.950000in}}%
\pgfusepath{clip}%
\pgfsetbuttcap%
\pgfsetroundjoin%
\definecolor{currentfill}{rgb}{0.642387,0.686428,0.748085}%
\pgfsetfillcolor{currentfill}%
\pgfsetlinewidth{0.000000pt}%
\definecolor{currentstroke}{rgb}{0.000000,0.000000,0.000000}%
\pgfsetstrokecolor{currentstroke}%
\pgfsetdash{}{0pt}%
\pgfpathmoveto{\pgfqpoint{0.921206in}{1.137069in}}%
\pgfpathlineto{\pgfqpoint{0.959387in}{1.181359in}}%
\pgfpathlineto{\pgfqpoint{0.921161in}{1.216386in}}%
\pgfpathlineto{\pgfqpoint{0.882823in}{1.175183in}}%
\pgfpathclose%
\pgfusepath{fill}%
\end{pgfscope}%
\begin{pgfscope}%
\pgfpathrectangle{\pgfqpoint{0.150000in}{0.150000in}}{\pgfqpoint{2.700000in}{1.950000in}}%
\pgfusepath{clip}%
\pgfsetbuttcap%
\pgfsetroundjoin%
\definecolor{currentfill}{rgb}{0.580193,0.631893,0.704274}%
\pgfsetfillcolor{currentfill}%
\pgfsetlinewidth{0.000000pt}%
\definecolor{currentstroke}{rgb}{0.000000,0.000000,0.000000}%
\pgfsetstrokecolor{currentstroke}%
\pgfsetdash{}{0pt}%
\pgfpathmoveto{\pgfqpoint{1.228374in}{1.231903in}}%
\pgfpathlineto{\pgfqpoint{1.267626in}{1.242070in}}%
\pgfpathlineto{\pgfqpoint{1.229251in}{1.280126in}}%
\pgfpathlineto{\pgfqpoint{1.189898in}{1.270065in}}%
\pgfpathclose%
\pgfusepath{fill}%
\end{pgfscope}%
\begin{pgfscope}%
\pgfpathrectangle{\pgfqpoint{0.150000in}{0.150000in}}{\pgfqpoint{2.700000in}{1.950000in}}%
\pgfusepath{clip}%
\pgfsetbuttcap%
\pgfsetroundjoin%
\definecolor{currentfill}{rgb}{0.710800,0.746415,0.796278}%
\pgfsetfillcolor{currentfill}%
\pgfsetlinewidth{0.000000pt}%
\definecolor{currentstroke}{rgb}{0.000000,0.000000,0.000000}%
\pgfsetstrokecolor{currentstroke}%
\pgfsetdash{}{0pt}%
\pgfpathmoveto{\pgfqpoint{0.921070in}{1.060816in}}%
\pgfpathlineto{\pgfqpoint{0.959429in}{1.102012in}}%
\pgfpathlineto{\pgfqpoint{0.921206in}{1.137069in}}%
\pgfpathlineto{\pgfqpoint{0.882870in}{1.095882in}}%
\pgfpathclose%
\pgfusepath{fill}%
\end{pgfscope}%
\begin{pgfscope}%
\pgfpathrectangle{\pgfqpoint{0.150000in}{0.150000in}}{\pgfqpoint{2.700000in}{1.950000in}}%
\pgfusepath{clip}%
\pgfsetbuttcap%
\pgfsetroundjoin%
\definecolor{currentfill}{rgb}{0.287883,0.375582,0.498361}%
\pgfsetfillcolor{currentfill}%
\pgfsetlinewidth{0.000000pt}%
\definecolor{currentstroke}{rgb}{0.000000,0.000000,0.000000}%
\pgfsetstrokecolor{currentstroke}%
\pgfsetdash{}{0pt}%
\pgfpathmoveto{\pgfqpoint{0.805872in}{1.568652in}}%
\pgfpathlineto{\pgfqpoint{0.843997in}{1.613145in}}%
\pgfpathlineto{\pgfqpoint{0.805605in}{1.651223in}}%
\pgfpathlineto{\pgfqpoint{0.767503in}{1.606717in}}%
\pgfpathclose%
\pgfusepath{fill}%
\end{pgfscope}%
\begin{pgfscope}%
\pgfpathrectangle{\pgfqpoint{0.150000in}{0.150000in}}{\pgfqpoint{2.700000in}{1.950000in}}%
\pgfusepath{clip}%
\pgfsetbuttcap%
\pgfsetroundjoin%
\definecolor{currentfill}{rgb}{0.511780,0.571906,0.656081}%
\pgfsetfillcolor{currentfill}%
\pgfsetlinewidth{0.000000pt}%
\definecolor{currentstroke}{rgb}{0.000000,0.000000,0.000000}%
\pgfsetstrokecolor{currentstroke}%
\pgfsetdash{}{0pt}%
\pgfpathmoveto{\pgfqpoint{0.921117in}{1.295715in}}%
\pgfpathlineto{\pgfqpoint{0.959473in}{1.336951in}}%
\pgfpathlineto{\pgfqpoint{0.921073in}{1.375056in}}%
\pgfpathlineto{\pgfqpoint{0.882729in}{1.333817in}}%
\pgfpathclose%
\pgfusepath{fill}%
\end{pgfscope}%
\begin{pgfscope}%
\pgfpathrectangle{\pgfqpoint{0.150000in}{0.150000in}}{\pgfqpoint{2.700000in}{1.950000in}}%
\pgfusepath{clip}%
\pgfsetbuttcap%
\pgfsetroundjoin%
\definecolor{currentfill}{rgb}{0.418490,0.490104,0.590365}%
\pgfsetfillcolor{currentfill}%
\pgfsetlinewidth{0.000000pt}%
\definecolor{currentstroke}{rgb}{0.000000,0.000000,0.000000}%
\pgfsetstrokecolor{currentstroke}%
\pgfsetdash{}{0pt}%
\pgfpathmoveto{\pgfqpoint{0.882681in}{1.413152in}}%
\pgfpathlineto{\pgfqpoint{0.920849in}{1.457578in}}%
\pgfpathlineto{\pgfqpoint{0.882634in}{1.492497in}}%
\pgfpathlineto{\pgfqpoint{0.844299in}{1.451239in}}%
\pgfpathclose%
\pgfusepath{fill}%
\end{pgfscope}%
\begin{pgfscope}%
\pgfpathrectangle{\pgfqpoint{0.150000in}{0.150000in}}{\pgfqpoint{2.700000in}{1.950000in}}%
\pgfusepath{clip}%
\pgfsetbuttcap%
\pgfsetroundjoin%
\definecolor{currentfill}{rgb}{0.611290,0.659161,0.726180}%
\pgfsetfillcolor{currentfill}%
\pgfsetlinewidth{0.000000pt}%
\definecolor{currentstroke}{rgb}{0.000000,0.000000,0.000000}%
\pgfsetstrokecolor{currentstroke}%
\pgfsetdash{}{0pt}%
\pgfpathmoveto{\pgfqpoint{1.266858in}{1.193733in}}%
\pgfpathlineto{\pgfqpoint{1.306009in}{1.204005in}}%
\pgfpathlineto{\pgfqpoint{1.267626in}{1.242070in}}%
\pgfpathlineto{\pgfqpoint{1.228374in}{1.231903in}}%
\pgfpathclose%
\pgfusepath{fill}%
\end{pgfscope}%
\begin{pgfscope}%
\pgfpathrectangle{\pgfqpoint{0.150000in}{0.150000in}}{\pgfqpoint{2.700000in}{1.950000in}}%
\pgfusepath{clip}%
\pgfsetbuttcap%
\pgfsetroundjoin%
\definecolor{currentfill}{rgb}{0.318980,0.402849,0.520267}%
\pgfsetfillcolor{currentfill}%
\pgfsetlinewidth{0.000000pt}%
\definecolor{currentstroke}{rgb}{0.000000,0.000000,0.000000}%
\pgfsetstrokecolor{currentstroke}%
\pgfsetdash{}{0pt}%
\pgfpathmoveto{\pgfqpoint{0.844249in}{1.530579in}}%
\pgfpathlineto{\pgfqpoint{0.882396in}{1.575059in}}%
\pgfpathlineto{\pgfqpoint{0.843997in}{1.613145in}}%
\pgfpathlineto{\pgfqpoint{0.805872in}{1.568652in}}%
\pgfpathclose%
\pgfusepath{fill}%
\end{pgfscope}%
\begin{pgfscope}%
\pgfpathrectangle{\pgfqpoint{0.150000in}{0.150000in}}{\pgfqpoint{2.700000in}{1.950000in}}%
\pgfusepath{clip}%
\pgfsetbuttcap%
\pgfsetroundjoin%
\definecolor{currentfill}{rgb}{0.642387,0.686428,0.748085}%
\pgfsetfillcolor{currentfill}%
\pgfsetlinewidth{0.000000pt}%
\definecolor{currentstroke}{rgb}{0.000000,0.000000,0.000000}%
\pgfsetstrokecolor{currentstroke}%
\pgfsetdash{}{0pt}%
\pgfpathmoveto{\pgfqpoint{1.305351in}{1.155554in}}%
\pgfpathlineto{\pgfqpoint{1.344401in}{1.165932in}}%
\pgfpathlineto{\pgfqpoint{1.306009in}{1.204005in}}%
\pgfpathlineto{\pgfqpoint{1.266858in}{1.193733in}}%
\pgfpathclose%
\pgfusepath{fill}%
\end{pgfscope}%
\begin{pgfscope}%
\pgfpathrectangle{\pgfqpoint{0.150000in}{0.150000in}}{\pgfqpoint{2.700000in}{1.950000in}}%
\pgfusepath{clip}%
\pgfsetbuttcap%
\pgfsetroundjoin%
\definecolor{currentfill}{rgb}{0.350077,0.430116,0.542172}%
\pgfsetfillcolor{currentfill}%
\pgfsetlinewidth{0.000000pt}%
\definecolor{currentstroke}{rgb}{0.000000,0.000000,0.000000}%
\pgfsetstrokecolor{currentstroke}%
\pgfsetdash{}{0pt}%
\pgfpathmoveto{\pgfqpoint{0.882634in}{1.492497in}}%
\pgfpathlineto{\pgfqpoint{0.920804in}{1.536964in}}%
\pgfpathlineto{\pgfqpoint{0.882396in}{1.575059in}}%
\pgfpathlineto{\pgfqpoint{0.844249in}{1.530579in}}%
\pgfpathclose%
\pgfusepath{fill}%
\end{pgfscope}%
\begin{pgfscope}%
\pgfpathrectangle{\pgfqpoint{0.150000in}{0.150000in}}{\pgfqpoint{2.700000in}{1.950000in}}%
\pgfusepath{clip}%
\pgfsetbuttcap%
\pgfsetroundjoin%
\definecolor{currentfill}{rgb}{0.542877,0.599173,0.677987}%
\pgfsetfillcolor{currentfill}%
\pgfsetlinewidth{0.000000pt}%
\definecolor{currentstroke}{rgb}{0.000000,0.000000,0.000000}%
\pgfsetstrokecolor{currentstroke}%
\pgfsetdash{}{0pt}%
\pgfpathmoveto{\pgfqpoint{0.959514in}{1.257605in}}%
\pgfpathlineto{\pgfqpoint{0.997723in}{1.301963in}}%
\pgfpathlineto{\pgfqpoint{0.959473in}{1.336951in}}%
\pgfpathlineto{\pgfqpoint{0.921117in}{1.295715in}}%
\pgfpathclose%
\pgfusepath{fill}%
\end{pgfscope}%
\begin{pgfscope}%
\pgfpathrectangle{\pgfqpoint{0.150000in}{0.150000in}}{\pgfqpoint{2.700000in}{1.950000in}}%
\pgfusepath{clip}%
\pgfsetbuttcap%
\pgfsetroundjoin%
\definecolor{currentfill}{rgb}{0.605070,0.653707,0.721798}%
\pgfsetfillcolor{currentfill}%
\pgfsetlinewidth{0.000000pt}%
\definecolor{currentstroke}{rgb}{0.000000,0.000000,0.000000}%
\pgfsetstrokecolor{currentstroke}%
\pgfsetdash{}{0pt}%
\pgfpathmoveto{\pgfqpoint{0.959387in}{1.181359in}}%
\pgfpathlineto{\pgfqpoint{0.997762in}{1.222588in}}%
\pgfpathlineto{\pgfqpoint{0.959514in}{1.257605in}}%
\pgfpathlineto{\pgfqpoint{0.921161in}{1.216386in}}%
\pgfpathclose%
\pgfusepath{fill}%
\end{pgfscope}%
\begin{pgfscope}%
\pgfpathrectangle{\pgfqpoint{0.150000in}{0.150000in}}{\pgfqpoint{2.700000in}{1.950000in}}%
\pgfusepath{clip}%
\pgfsetbuttcap%
\pgfsetroundjoin%
\definecolor{currentfill}{rgb}{0.673483,0.713695,0.769991}%
\pgfsetfillcolor{currentfill}%
\pgfsetlinewidth{0.000000pt}%
\definecolor{currentstroke}{rgb}{0.000000,0.000000,0.000000}%
\pgfsetstrokecolor{currentstroke}%
\pgfsetdash{}{0pt}%
\pgfpathmoveto{\pgfqpoint{0.959429in}{1.102012in}}%
\pgfpathlineto{\pgfqpoint{0.997801in}{1.143223in}}%
\pgfpathlineto{\pgfqpoint{0.959387in}{1.181359in}}%
\pgfpathlineto{\pgfqpoint{0.921206in}{1.137069in}}%
\pgfpathclose%
\pgfusepath{fill}%
\end{pgfscope}%
\begin{pgfscope}%
\pgfpathrectangle{\pgfqpoint{0.150000in}{0.150000in}}{\pgfqpoint{2.700000in}{1.950000in}}%
\pgfusepath{clip}%
\pgfsetbuttcap%
\pgfsetroundjoin%
\definecolor{currentfill}{rgb}{0.443367,0.511918,0.607889}%
\pgfsetfillcolor{currentfill}%
\pgfsetlinewidth{0.000000pt}%
\definecolor{currentstroke}{rgb}{0.000000,0.000000,0.000000}%
\pgfsetstrokecolor{currentstroke}%
\pgfsetdash{}{0pt}%
\pgfpathmoveto{\pgfqpoint{0.921073in}{1.375056in}}%
\pgfpathlineto{\pgfqpoint{0.959262in}{1.419468in}}%
\pgfpathlineto{\pgfqpoint{0.920849in}{1.457578in}}%
\pgfpathlineto{\pgfqpoint{0.882681in}{1.413152in}}%
\pgfpathclose%
\pgfusepath{fill}%
\end{pgfscope}%
\begin{pgfscope}%
\pgfpathrectangle{\pgfqpoint{0.150000in}{0.150000in}}{\pgfqpoint{2.700000in}{1.950000in}}%
\pgfusepath{clip}%
\pgfsetbuttcap%
\pgfsetroundjoin%
\definecolor{currentfill}{rgb}{0.667264,0.708241,0.765610}%
\pgfsetfillcolor{currentfill}%
\pgfsetlinewidth{0.000000pt}%
\definecolor{currentstroke}{rgb}{0.000000,0.000000,0.000000}%
\pgfsetstrokecolor{currentstroke}%
\pgfsetdash{}{0pt}%
\pgfpathmoveto{\pgfqpoint{1.343852in}{1.117367in}}%
\pgfpathlineto{\pgfqpoint{1.382801in}{1.127850in}}%
\pgfpathlineto{\pgfqpoint{1.344401in}{1.165932in}}%
\pgfpathlineto{\pgfqpoint{1.305351in}{1.155554in}}%
\pgfpathclose%
\pgfusepath{fill}%
\end{pgfscope}%
\begin{pgfscope}%
\pgfpathrectangle{\pgfqpoint{0.150000in}{0.150000in}}{\pgfqpoint{2.700000in}{1.950000in}}%
\pgfusepath{clip}%
\pgfsetbuttcap%
\pgfsetroundjoin%
\definecolor{currentfill}{rgb}{0.474464,0.539185,0.629795}%
\pgfsetfillcolor{currentfill}%
\pgfsetlinewidth{0.000000pt}%
\definecolor{currentstroke}{rgb}{0.000000,0.000000,0.000000}%
\pgfsetstrokecolor{currentstroke}%
\pgfsetdash{}{0pt}%
\pgfpathmoveto{\pgfqpoint{0.959473in}{1.336951in}}%
\pgfpathlineto{\pgfqpoint{0.997685in}{1.381351in}}%
\pgfpathlineto{\pgfqpoint{0.959262in}{1.419468in}}%
\pgfpathlineto{\pgfqpoint{0.921073in}{1.375056in}}%
\pgfpathclose%
\pgfusepath{fill}%
\end{pgfscope}%
\begin{pgfscope}%
\pgfpathrectangle{\pgfqpoint{0.150000in}{0.150000in}}{\pgfqpoint{2.700000in}{1.950000in}}%
\pgfusepath{clip}%
\pgfsetbuttcap%
\pgfsetroundjoin%
\definecolor{currentfill}{rgb}{0.735677,0.768229,0.813802}%
\pgfsetfillcolor{currentfill}%
\pgfsetlinewidth{0.000000pt}%
\definecolor{currentstroke}{rgb}{0.000000,0.000000,0.000000}%
\pgfsetstrokecolor{currentstroke}%
\pgfsetdash{}{0pt}%
\pgfpathmoveto{\pgfqpoint{0.959301in}{1.025720in}}%
\pgfpathlineto{\pgfqpoint{0.997682in}{1.066927in}}%
\pgfpathlineto{\pgfqpoint{0.959429in}{1.102012in}}%
\pgfpathlineto{\pgfqpoint{0.921070in}{1.060816in}}%
\pgfpathclose%
\pgfusepath{fill}%
\end{pgfscope}%
\begin{pgfscope}%
\pgfpathrectangle{\pgfqpoint{0.150000in}{0.150000in}}{\pgfqpoint{2.700000in}{1.950000in}}%
\pgfusepath{clip}%
\pgfsetbuttcap%
\pgfsetroundjoin%
\definecolor{currentfill}{rgb}{0.381173,0.457384,0.564078}%
\pgfsetfillcolor{currentfill}%
\pgfsetlinewidth{0.000000pt}%
\definecolor{currentstroke}{rgb}{0.000000,0.000000,0.000000}%
\pgfsetstrokecolor{currentstroke}%
\pgfsetdash{}{0pt}%
\pgfpathmoveto{\pgfqpoint{0.920849in}{1.457578in}}%
\pgfpathlineto{\pgfqpoint{0.959221in}{1.498861in}}%
\pgfpathlineto{\pgfqpoint{0.920804in}{1.536964in}}%
\pgfpathlineto{\pgfqpoint{0.882634in}{1.492497in}}%
\pgfpathclose%
\pgfusepath{fill}%
\end{pgfscope}%
\begin{pgfscope}%
\pgfpathrectangle{\pgfqpoint{0.150000in}{0.150000in}}{\pgfqpoint{2.700000in}{1.950000in}}%
\pgfusepath{clip}%
\pgfsetbuttcap%
\pgfsetroundjoin%
\definecolor{currentfill}{rgb}{0.698361,0.735509,0.787515}%
\pgfsetfillcolor{currentfill}%
\pgfsetlinewidth{0.000000pt}%
\definecolor{currentstroke}{rgb}{0.000000,0.000000,0.000000}%
\pgfsetstrokecolor{currentstroke}%
\pgfsetdash{}{0pt}%
\pgfpathmoveto{\pgfqpoint{1.382362in}{1.079171in}}%
\pgfpathlineto{\pgfqpoint{1.421210in}{1.089760in}}%
\pgfpathlineto{\pgfqpoint{1.382801in}{1.127850in}}%
\pgfpathlineto{\pgfqpoint{1.343852in}{1.117367in}}%
\pgfpathclose%
\pgfusepath{fill}%
\end{pgfscope}%
\begin{pgfscope}%
\pgfpathrectangle{\pgfqpoint{0.150000in}{0.150000in}}{\pgfqpoint{2.700000in}{1.950000in}}%
\pgfusepath{clip}%
\pgfsetbuttcap%
\pgfsetroundjoin%
\definecolor{currentfill}{rgb}{0.567754,0.620987,0.695512}%
\pgfsetfillcolor{currentfill}%
\pgfsetlinewidth{0.000000pt}%
\definecolor{currentstroke}{rgb}{0.000000,0.000000,0.000000}%
\pgfsetstrokecolor{currentstroke}%
\pgfsetdash{}{0pt}%
\pgfpathmoveto{\pgfqpoint{0.997762in}{1.222588in}}%
\pgfpathlineto{\pgfqpoint{1.036151in}{1.263831in}}%
\pgfpathlineto{\pgfqpoint{0.997723in}{1.301963in}}%
\pgfpathlineto{\pgfqpoint{0.959514in}{1.257605in}}%
\pgfpathclose%
\pgfusepath{fill}%
\end{pgfscope}%
\begin{pgfscope}%
\pgfpathrectangle{\pgfqpoint{0.150000in}{0.150000in}}{\pgfqpoint{2.700000in}{1.950000in}}%
\pgfusepath{clip}%
\pgfsetbuttcap%
\pgfsetroundjoin%
\definecolor{currentfill}{rgb}{0.636167,0.680974,0.743704}%
\pgfsetfillcolor{currentfill}%
\pgfsetlinewidth{0.000000pt}%
\definecolor{currentstroke}{rgb}{0.000000,0.000000,0.000000}%
\pgfsetstrokecolor{currentstroke}%
\pgfsetdash{}{0pt}%
\pgfpathmoveto{\pgfqpoint{0.997801in}{1.143223in}}%
\pgfpathlineto{\pgfqpoint{1.036187in}{1.184450in}}%
\pgfpathlineto{\pgfqpoint{0.997762in}{1.222588in}}%
\pgfpathlineto{\pgfqpoint{0.959387in}{1.181359in}}%
\pgfpathclose%
\pgfusepath{fill}%
\end{pgfscope}%
\begin{pgfscope}%
\pgfpathrectangle{\pgfqpoint{0.150000in}{0.150000in}}{\pgfqpoint{2.700000in}{1.950000in}}%
\pgfusepath{clip}%
\pgfsetbuttcap%
\pgfsetroundjoin%
\definecolor{currentfill}{rgb}{0.729458,0.762776,0.809421}%
\pgfsetfillcolor{currentfill}%
\pgfsetlinewidth{0.000000pt}%
\definecolor{currentstroke}{rgb}{0.000000,0.000000,0.000000}%
\pgfsetstrokecolor{currentstroke}%
\pgfsetdash{}{0pt}%
\pgfpathmoveto{\pgfqpoint{1.420880in}{1.040967in}}%
\pgfpathlineto{\pgfqpoint{1.459627in}{1.051662in}}%
\pgfpathlineto{\pgfqpoint{1.421210in}{1.089760in}}%
\pgfpathlineto{\pgfqpoint{1.382362in}{1.079171in}}%
\pgfpathclose%
\pgfusepath{fill}%
\end{pgfscope}%
\begin{pgfscope}%
\pgfpathrectangle{\pgfqpoint{0.150000in}{0.150000in}}{\pgfqpoint{2.700000in}{1.950000in}}%
\pgfusepath{clip}%
\pgfsetbuttcap%
\pgfsetroundjoin%
\definecolor{currentfill}{rgb}{0.505561,0.566452,0.651700}%
\pgfsetfillcolor{currentfill}%
\pgfsetlinewidth{0.000000pt}%
\definecolor{currentstroke}{rgb}{0.000000,0.000000,0.000000}%
\pgfsetstrokecolor{currentstroke}%
\pgfsetdash{}{0pt}%
\pgfpathmoveto{\pgfqpoint{0.997723in}{1.301963in}}%
\pgfpathlineto{\pgfqpoint{1.036115in}{1.343225in}}%
\pgfpathlineto{\pgfqpoint{0.997685in}{1.381351in}}%
\pgfpathlineto{\pgfqpoint{0.959473in}{1.336951in}}%
\pgfpathclose%
\pgfusepath{fill}%
\end{pgfscope}%
\begin{pgfscope}%
\pgfpathrectangle{\pgfqpoint{0.150000in}{0.150000in}}{\pgfqpoint{2.700000in}{1.950000in}}%
\pgfusepath{clip}%
\pgfsetbuttcap%
\pgfsetroundjoin%
\definecolor{currentfill}{rgb}{0.698361,0.735509,0.787515}%
\pgfsetfillcolor{currentfill}%
\pgfsetlinewidth{0.000000pt}%
\definecolor{currentstroke}{rgb}{0.000000,0.000000,0.000000}%
\pgfsetstrokecolor{currentstroke}%
\pgfsetdash{}{0pt}%
\pgfpathmoveto{\pgfqpoint{0.997682in}{1.066927in}}%
\pgfpathlineto{\pgfqpoint{1.036077in}{1.108148in}}%
\pgfpathlineto{\pgfqpoint{0.997801in}{1.143223in}}%
\pgfpathlineto{\pgfqpoint{0.959429in}{1.102012in}}%
\pgfpathclose%
\pgfusepath{fill}%
\end{pgfscope}%
\begin{pgfscope}%
\pgfpathrectangle{\pgfqpoint{0.150000in}{0.150000in}}{\pgfqpoint{2.700000in}{1.950000in}}%
\pgfusepath{clip}%
\pgfsetbuttcap%
\pgfsetroundjoin%
\definecolor{currentfill}{rgb}{0.406051,0.479197,0.581602}%
\pgfsetfillcolor{currentfill}%
\pgfsetlinewidth{0.000000pt}%
\definecolor{currentstroke}{rgb}{0.000000,0.000000,0.000000}%
\pgfsetstrokecolor{currentstroke}%
\pgfsetdash{}{0pt}%
\pgfpathmoveto{\pgfqpoint{0.959262in}{1.419468in}}%
\pgfpathlineto{\pgfqpoint{0.997646in}{1.460749in}}%
\pgfpathlineto{\pgfqpoint{0.959221in}{1.498861in}}%
\pgfpathlineto{\pgfqpoint{0.920849in}{1.457578in}}%
\pgfpathclose%
\pgfusepath{fill}%
\end{pgfscope}%
\begin{pgfscope}%
\pgfpathrectangle{\pgfqpoint{0.150000in}{0.150000in}}{\pgfqpoint{2.700000in}{1.950000in}}%
\pgfusepath{clip}%
\pgfsetbuttcap%
\pgfsetroundjoin%
\definecolor{currentfill}{rgb}{0.760555,0.790043,0.831327}%
\pgfsetfillcolor{currentfill}%
\pgfsetlinewidth{0.000000pt}%
\definecolor{currentstroke}{rgb}{0.000000,0.000000,0.000000}%
\pgfsetstrokecolor{currentstroke}%
\pgfsetdash{}{0pt}%
\pgfpathmoveto{\pgfqpoint{0.997721in}{0.987562in}}%
\pgfpathlineto{\pgfqpoint{1.036113in}{1.028766in}}%
\pgfpathlineto{\pgfqpoint{0.997682in}{1.066927in}}%
\pgfpathlineto{\pgfqpoint{0.959301in}{1.025720in}}%
\pgfpathclose%
\pgfusepath{fill}%
\end{pgfscope}%
\begin{pgfscope}%
\pgfpathrectangle{\pgfqpoint{0.150000in}{0.150000in}}{\pgfqpoint{2.700000in}{1.950000in}}%
\pgfusepath{clip}%
\pgfsetbuttcap%
\pgfsetroundjoin%
\definecolor{currentfill}{rgb}{0.760555,0.790043,0.831327}%
\pgfsetfillcolor{currentfill}%
\pgfsetlinewidth{0.000000pt}%
\definecolor{currentstroke}{rgb}{0.000000,0.000000,0.000000}%
\pgfsetstrokecolor{currentstroke}%
\pgfsetdash{}{0pt}%
\pgfpathmoveto{\pgfqpoint{1.459407in}{1.002754in}}%
\pgfpathlineto{\pgfqpoint{1.498052in}{1.013555in}}%
\pgfpathlineto{\pgfqpoint{1.459627in}{1.051662in}}%
\pgfpathlineto{\pgfqpoint{1.420880in}{1.040967in}}%
\pgfpathclose%
\pgfusepath{fill}%
\end{pgfscope}%
\begin{pgfscope}%
\pgfpathrectangle{\pgfqpoint{0.150000in}{0.150000in}}{\pgfqpoint{2.700000in}{1.950000in}}%
\pgfusepath{clip}%
\pgfsetbuttcap%
\pgfsetroundjoin%
\definecolor{currentfill}{rgb}{0.598851,0.648254,0.717417}%
\pgfsetfillcolor{currentfill}%
\pgfsetlinewidth{0.000000pt}%
\definecolor{currentstroke}{rgb}{0.000000,0.000000,0.000000}%
\pgfsetstrokecolor{currentstroke}%
\pgfsetdash{}{0pt}%
\pgfpathmoveto{\pgfqpoint{1.036187in}{1.184450in}}%
\pgfpathlineto{\pgfqpoint{1.074588in}{1.225691in}}%
\pgfpathlineto{\pgfqpoint{1.036151in}{1.263831in}}%
\pgfpathlineto{\pgfqpoint{0.997762in}{1.222588in}}%
\pgfpathclose%
\pgfusepath{fill}%
\end{pgfscope}%
\begin{pgfscope}%
\pgfpathrectangle{\pgfqpoint{0.150000in}{0.150000in}}{\pgfqpoint{2.700000in}{1.950000in}}%
\pgfusepath{clip}%
\pgfsetbuttcap%
\pgfsetroundjoin%
\definecolor{currentfill}{rgb}{0.661045,0.702788,0.761229}%
\pgfsetfillcolor{currentfill}%
\pgfsetlinewidth{0.000000pt}%
\definecolor{currentstroke}{rgb}{0.000000,0.000000,0.000000}%
\pgfsetstrokecolor{currentstroke}%
\pgfsetdash{}{0pt}%
\pgfpathmoveto{\pgfqpoint{1.036077in}{1.108148in}}%
\pgfpathlineto{\pgfqpoint{1.074621in}{1.146303in}}%
\pgfpathlineto{\pgfqpoint{1.036187in}{1.184450in}}%
\pgfpathlineto{\pgfqpoint{0.997801in}{1.143223in}}%
\pgfpathclose%
\pgfusepath{fill}%
\end{pgfscope}%
\begin{pgfscope}%
\pgfpathrectangle{\pgfqpoint{0.150000in}{0.150000in}}{\pgfqpoint{2.700000in}{1.950000in}}%
\pgfusepath{clip}%
\pgfsetbuttcap%
\pgfsetroundjoin%
\definecolor{currentfill}{rgb}{0.437148,0.506464,0.603508}%
\pgfsetfillcolor{currentfill}%
\pgfsetlinewidth{0.000000pt}%
\definecolor{currentstroke}{rgb}{0.000000,0.000000,0.000000}%
\pgfsetstrokecolor{currentstroke}%
\pgfsetdash{}{0pt}%
\pgfpathmoveto{\pgfqpoint{0.997685in}{1.381351in}}%
\pgfpathlineto{\pgfqpoint{1.036079in}{1.422629in}}%
\pgfpathlineto{\pgfqpoint{0.997646in}{1.460749in}}%
\pgfpathlineto{\pgfqpoint{0.959262in}{1.419468in}}%
\pgfpathclose%
\pgfusepath{fill}%
\end{pgfscope}%
\begin{pgfscope}%
\pgfpathrectangle{\pgfqpoint{0.150000in}{0.150000in}}{\pgfqpoint{2.700000in}{1.950000in}}%
\pgfusepath{clip}%
\pgfsetbuttcap%
\pgfsetroundjoin%
\definecolor{currentfill}{rgb}{0.536657,0.593719,0.673606}%
\pgfsetfillcolor{currentfill}%
\pgfsetlinewidth{0.000000pt}%
\definecolor{currentstroke}{rgb}{0.000000,0.000000,0.000000}%
\pgfsetstrokecolor{currentstroke}%
\pgfsetdash{}{0pt}%
\pgfpathmoveto{\pgfqpoint{1.036151in}{1.263831in}}%
\pgfpathlineto{\pgfqpoint{1.074555in}{1.305090in}}%
\pgfpathlineto{\pgfqpoint{1.036115in}{1.343225in}}%
\pgfpathlineto{\pgfqpoint{0.997723in}{1.301963in}}%
\pgfpathclose%
\pgfusepath{fill}%
\end{pgfscope}%
\begin{pgfscope}%
\pgfpathrectangle{\pgfqpoint{0.150000in}{0.150000in}}{\pgfqpoint{2.700000in}{1.950000in}}%
\pgfusepath{clip}%
\pgfsetbuttcap%
\pgfsetroundjoin%
\definecolor{currentfill}{rgb}{0.791651,0.817310,0.853232}%
\pgfsetfillcolor{currentfill}%
\pgfsetlinewidth{0.000000pt}%
\definecolor{currentstroke}{rgb}{0.000000,0.000000,0.000000}%
\pgfsetstrokecolor{currentstroke}%
\pgfsetdash{}{0pt}%
\pgfpathmoveto{\pgfqpoint{1.497943in}{0.964533in}}%
\pgfpathlineto{\pgfqpoint{1.536486in}{0.975440in}}%
\pgfpathlineto{\pgfqpoint{1.498052in}{1.013555in}}%
\pgfpathlineto{\pgfqpoint{1.459407in}{1.002754in}}%
\pgfpathclose%
\pgfusepath{fill}%
\end{pgfscope}%
\begin{pgfscope}%
\pgfpathrectangle{\pgfqpoint{0.150000in}{0.150000in}}{\pgfqpoint{2.700000in}{1.950000in}}%
\pgfusepath{clip}%
\pgfsetbuttcap%
\pgfsetroundjoin%
\definecolor{currentfill}{rgb}{0.629948,0.675521,0.739323}%
\pgfsetfillcolor{currentfill}%
\pgfsetlinewidth{0.000000pt}%
\definecolor{currentstroke}{rgb}{0.000000,0.000000,0.000000}%
\pgfsetstrokecolor{currentstroke}%
\pgfsetdash{}{0pt}%
\pgfpathmoveto{\pgfqpoint{1.074621in}{1.146303in}}%
\pgfpathlineto{\pgfqpoint{1.113033in}{1.187542in}}%
\pgfpathlineto{\pgfqpoint{1.074588in}{1.225691in}}%
\pgfpathlineto{\pgfqpoint{1.036187in}{1.184450in}}%
\pgfpathclose%
\pgfusepath{fill}%
\end{pgfscope}%
\begin{pgfscope}%
\pgfpathrectangle{\pgfqpoint{0.150000in}{0.150000in}}{\pgfqpoint{2.700000in}{1.950000in}}%
\pgfusepath{clip}%
\pgfsetbuttcap%
\pgfsetroundjoin%
\definecolor{currentfill}{rgb}{0.468244,0.533732,0.625414}%
\pgfsetfillcolor{currentfill}%
\pgfsetlinewidth{0.000000pt}%
\definecolor{currentstroke}{rgb}{0.000000,0.000000,0.000000}%
\pgfsetstrokecolor{currentstroke}%
\pgfsetdash{}{0pt}%
\pgfpathmoveto{\pgfqpoint{1.036115in}{1.343225in}}%
\pgfpathlineto{\pgfqpoint{1.074521in}{1.384501in}}%
\pgfpathlineto{\pgfqpoint{1.036079in}{1.422629in}}%
\pgfpathlineto{\pgfqpoint{0.997685in}{1.381351in}}%
\pgfpathclose%
\pgfusepath{fill}%
\end{pgfscope}%
\begin{pgfscope}%
\pgfpathrectangle{\pgfqpoint{0.150000in}{0.150000in}}{\pgfqpoint{2.700000in}{1.950000in}}%
\pgfusepath{clip}%
\pgfsetbuttcap%
\pgfsetroundjoin%
\definecolor{currentfill}{rgb}{0.723238,0.757322,0.805040}%
\pgfsetfillcolor{currentfill}%
\pgfsetlinewidth{0.000000pt}%
\definecolor{currentstroke}{rgb}{0.000000,0.000000,0.000000}%
\pgfsetstrokecolor{currentstroke}%
\pgfsetdash{}{0pt}%
\pgfpathmoveto{\pgfqpoint{1.036113in}{1.028766in}}%
\pgfpathlineto{\pgfqpoint{1.074519in}{1.069985in}}%
\pgfpathlineto{\pgfqpoint{1.036077in}{1.108148in}}%
\pgfpathlineto{\pgfqpoint{0.997682in}{1.066927in}}%
\pgfpathclose%
\pgfusepath{fill}%
\end{pgfscope}%
\begin{pgfscope}%
\pgfpathrectangle{\pgfqpoint{0.150000in}{0.150000in}}{\pgfqpoint{2.700000in}{1.950000in}}%
\pgfusepath{clip}%
\pgfsetbuttcap%
\pgfsetroundjoin%
\definecolor{currentfill}{rgb}{0.561535,0.615533,0.691131}%
\pgfsetfillcolor{currentfill}%
\pgfsetlinewidth{0.000000pt}%
\definecolor{currentstroke}{rgb}{0.000000,0.000000,0.000000}%
\pgfsetstrokecolor{currentstroke}%
\pgfsetdash{}{0pt}%
\pgfpathmoveto{\pgfqpoint{1.074588in}{1.225691in}}%
\pgfpathlineto{\pgfqpoint{1.113002in}{1.266948in}}%
\pgfpathlineto{\pgfqpoint{1.074555in}{1.305090in}}%
\pgfpathlineto{\pgfqpoint{1.036151in}{1.263831in}}%
\pgfpathclose%
\pgfusepath{fill}%
\end{pgfscope}%
\begin{pgfscope}%
\pgfpathrectangle{\pgfqpoint{0.150000in}{0.150000in}}{\pgfqpoint{2.700000in}{1.950000in}}%
\pgfusepath{clip}%
\pgfsetbuttcap%
\pgfsetroundjoin%
\definecolor{currentfill}{rgb}{0.785432,0.811857,0.848851}%
\pgfsetfillcolor{currentfill}%
\pgfsetlinewidth{0.000000pt}%
\definecolor{currentstroke}{rgb}{0.000000,0.000000,0.000000}%
\pgfsetstrokecolor{currentstroke}%
\pgfsetdash{}{0pt}%
\pgfpathmoveto{\pgfqpoint{1.036003in}{0.952419in}}%
\pgfpathlineto{\pgfqpoint{1.074417in}{0.993634in}}%
\pgfpathlineto{\pgfqpoint{1.036113in}{1.028766in}}%
\pgfpathlineto{\pgfqpoint{0.997721in}{0.987562in}}%
\pgfpathclose%
\pgfusepath{fill}%
\end{pgfscope}%
\begin{pgfscope}%
\pgfpathrectangle{\pgfqpoint{0.150000in}{0.150000in}}{\pgfqpoint{2.700000in}{1.950000in}}%
\pgfusepath{clip}%
\pgfsetbuttcap%
\pgfsetroundjoin%
\definecolor{currentfill}{rgb}{0.499341,0.560999,0.647319}%
\pgfsetfillcolor{currentfill}%
\pgfsetlinewidth{0.000000pt}%
\definecolor{currentstroke}{rgb}{0.000000,0.000000,0.000000}%
\pgfsetstrokecolor{currentstroke}%
\pgfsetdash{}{0pt}%
\pgfpathmoveto{\pgfqpoint{1.074555in}{1.305090in}}%
\pgfpathlineto{\pgfqpoint{1.112972in}{1.346364in}}%
\pgfpathlineto{\pgfqpoint{1.074521in}{1.384501in}}%
\pgfpathlineto{\pgfqpoint{1.036115in}{1.343225in}}%
\pgfpathclose%
\pgfusepath{fill}%
\end{pgfscope}%
\begin{pgfscope}%
\pgfpathrectangle{\pgfqpoint{0.150000in}{0.150000in}}{\pgfqpoint{2.700000in}{1.950000in}}%
\pgfusepath{clip}%
\pgfsetbuttcap%
\pgfsetroundjoin%
\definecolor{currentfill}{rgb}{0.692142,0.730055,0.783134}%
\pgfsetfillcolor{currentfill}%
\pgfsetlinewidth{0.000000pt}%
\definecolor{currentstroke}{rgb}{0.000000,0.000000,0.000000}%
\pgfsetstrokecolor{currentstroke}%
\pgfsetdash{}{0pt}%
\pgfpathmoveto{\pgfqpoint{1.074519in}{1.069985in}}%
\pgfpathlineto{\pgfqpoint{1.112939in}{1.111219in}}%
\pgfpathlineto{\pgfqpoint{1.074621in}{1.146303in}}%
\pgfpathlineto{\pgfqpoint{1.036077in}{1.108148in}}%
\pgfpathclose%
\pgfusepath{fill}%
\end{pgfscope}%
\begin{pgfscope}%
\pgfpathrectangle{\pgfqpoint{0.150000in}{0.150000in}}{\pgfqpoint{2.700000in}{1.950000in}}%
\pgfusepath{clip}%
\pgfsetbuttcap%
\pgfsetroundjoin%
\definecolor{currentfill}{rgb}{0.592632,0.642800,0.713036}%
\pgfsetfillcolor{currentfill}%
\pgfsetlinewidth{0.000000pt}%
\definecolor{currentstroke}{rgb}{0.000000,0.000000,0.000000}%
\pgfsetstrokecolor{currentstroke}%
\pgfsetdash{}{0pt}%
\pgfpathmoveto{\pgfqpoint{1.113033in}{1.187542in}}%
\pgfpathlineto{\pgfqpoint{1.151458in}{1.228796in}}%
\pgfpathlineto{\pgfqpoint{1.113002in}{1.266948in}}%
\pgfpathlineto{\pgfqpoint{1.074588in}{1.225691in}}%
\pgfpathclose%
\pgfusepath{fill}%
\end{pgfscope}%
\begin{pgfscope}%
\pgfpathrectangle{\pgfqpoint{0.150000in}{0.150000in}}{\pgfqpoint{2.700000in}{1.950000in}}%
\pgfusepath{clip}%
\pgfsetbuttcap%
\pgfsetroundjoin%
\definecolor{currentfill}{rgb}{0.754335,0.784589,0.826945}%
\pgfsetfillcolor{currentfill}%
\pgfsetlinewidth{0.000000pt}%
\definecolor{currentstroke}{rgb}{0.000000,0.000000,0.000000}%
\pgfsetstrokecolor{currentstroke}%
\pgfsetdash{}{0pt}%
\pgfpathmoveto{\pgfqpoint{1.074417in}{0.993634in}}%
\pgfpathlineto{\pgfqpoint{1.112970in}{1.031814in}}%
\pgfpathlineto{\pgfqpoint{1.074519in}{1.069985in}}%
\pgfpathlineto{\pgfqpoint{1.036113in}{1.028766in}}%
\pgfpathclose%
\pgfusepath{fill}%
\end{pgfscope}%
\begin{pgfscope}%
\pgfpathrectangle{\pgfqpoint{0.150000in}{0.150000in}}{\pgfqpoint{2.700000in}{1.950000in}}%
\pgfusepath{clip}%
\pgfsetbuttcap%
\pgfsetroundjoin%
\definecolor{currentfill}{rgb}{0.816529,0.839124,0.870757}%
\pgfsetfillcolor{currentfill}%
\pgfsetlinewidth{0.000000pt}%
\definecolor{currentstroke}{rgb}{0.000000,0.000000,0.000000}%
\pgfsetstrokecolor{currentstroke}%
\pgfsetdash{}{0pt}%
\pgfpathmoveto{\pgfqpoint{1.536486in}{0.929326in}}%
\pgfpathlineto{\pgfqpoint{1.574940in}{0.940333in}}%
\pgfpathlineto{\pgfqpoint{1.536486in}{0.975440in}}%
\pgfpathlineto{\pgfqpoint{1.497943in}{0.964533in}}%
\pgfpathclose%
\pgfusepath{fill}%
\end{pgfscope}%
\begin{pgfscope}%
\pgfpathrectangle{\pgfqpoint{0.150000in}{0.150000in}}{\pgfqpoint{2.700000in}{1.950000in}}%
\pgfusepath{clip}%
\pgfsetbuttcap%
\pgfsetroundjoin%
\definecolor{currentfill}{rgb}{0.530438,0.588266,0.669225}%
\pgfsetfillcolor{currentfill}%
\pgfsetlinewidth{0.000000pt}%
\definecolor{currentstroke}{rgb}{0.000000,0.000000,0.000000}%
\pgfsetstrokecolor{currentstroke}%
\pgfsetdash{}{0pt}%
\pgfpathmoveto{\pgfqpoint{1.113002in}{1.266948in}}%
\pgfpathlineto{\pgfqpoint{1.151431in}{1.308219in}}%
\pgfpathlineto{\pgfqpoint{1.112972in}{1.346364in}}%
\pgfpathlineto{\pgfqpoint{1.074555in}{1.305090in}}%
\pgfpathclose%
\pgfusepath{fill}%
\end{pgfscope}%
\begin{pgfscope}%
\pgfpathrectangle{\pgfqpoint{0.150000in}{0.150000in}}{\pgfqpoint{2.700000in}{1.950000in}}%
\pgfusepath{clip}%
\pgfsetbuttcap%
\pgfsetroundjoin%
\definecolor{currentfill}{rgb}{0.654825,0.697335,0.756847}%
\pgfsetfillcolor{currentfill}%
\pgfsetlinewidth{0.000000pt}%
\definecolor{currentstroke}{rgb}{0.000000,0.000000,0.000000}%
\pgfsetstrokecolor{currentstroke}%
\pgfsetdash{}{0pt}%
\pgfpathmoveto{\pgfqpoint{1.112939in}{1.111219in}}%
\pgfpathlineto{\pgfqpoint{1.151373in}{1.152469in}}%
\pgfpathlineto{\pgfqpoint{1.113033in}{1.187542in}}%
\pgfpathlineto{\pgfqpoint{1.074621in}{1.146303in}}%
\pgfpathclose%
\pgfusepath{fill}%
\end{pgfscope}%
\begin{pgfscope}%
\pgfpathrectangle{\pgfqpoint{0.150000in}{0.150000in}}{\pgfqpoint{2.700000in}{1.950000in}}%
\pgfusepath{clip}%
\pgfsetbuttcap%
\pgfsetroundjoin%
\definecolor{currentfill}{rgb}{0.723238,0.757322,0.805040}%
\pgfsetfillcolor{currentfill}%
\pgfsetlinewidth{0.000000pt}%
\definecolor{currentstroke}{rgb}{0.000000,0.000000,0.000000}%
\pgfsetstrokecolor{currentstroke}%
\pgfsetdash{}{0pt}%
\pgfpathmoveto{\pgfqpoint{1.112970in}{1.031814in}}%
\pgfpathlineto{\pgfqpoint{1.151401in}{1.073045in}}%
\pgfpathlineto{\pgfqpoint{1.112939in}{1.111219in}}%
\pgfpathlineto{\pgfqpoint{1.074519in}{1.069985in}}%
\pgfpathclose%
\pgfusepath{fill}%
\end{pgfscope}%
\begin{pgfscope}%
\pgfpathrectangle{\pgfqpoint{0.150000in}{0.150000in}}{\pgfqpoint{2.700000in}{1.950000in}}%
\pgfusepath{clip}%
\pgfsetbuttcap%
\pgfsetroundjoin%
\definecolor{currentfill}{rgb}{0.561535,0.615533,0.691131}%
\pgfsetfillcolor{currentfill}%
\pgfsetlinewidth{0.000000pt}%
\definecolor{currentstroke}{rgb}{0.000000,0.000000,0.000000}%
\pgfsetstrokecolor{currentstroke}%
\pgfsetdash{}{0pt}%
\pgfpathmoveto{\pgfqpoint{1.151458in}{1.228796in}}%
\pgfpathlineto{\pgfqpoint{1.189898in}{1.270065in}}%
\pgfpathlineto{\pgfqpoint{1.151431in}{1.308219in}}%
\pgfpathlineto{\pgfqpoint{1.113002in}{1.266948in}}%
\pgfpathclose%
\pgfusepath{fill}%
\end{pgfscope}%
\begin{pgfscope}%
\pgfpathrectangle{\pgfqpoint{0.150000in}{0.150000in}}{\pgfqpoint{2.700000in}{1.950000in}}%
\pgfusepath{clip}%
\pgfsetbuttcap%
\pgfsetroundjoin%
\definecolor{currentfill}{rgb}{0.623729,0.670067,0.734942}%
\pgfsetfillcolor{currentfill}%
\pgfsetlinewidth{0.000000pt}%
\definecolor{currentstroke}{rgb}{0.000000,0.000000,0.000000}%
\pgfsetstrokecolor{currentstroke}%
\pgfsetdash{}{0pt}%
\pgfpathmoveto{\pgfqpoint{1.151373in}{1.152469in}}%
\pgfpathlineto{\pgfqpoint{1.189923in}{1.190637in}}%
\pgfpathlineto{\pgfqpoint{1.151458in}{1.228796in}}%
\pgfpathlineto{\pgfqpoint{1.113033in}{1.187542in}}%
\pgfpathclose%
\pgfusepath{fill}%
\end{pgfscope}%
\begin{pgfscope}%
\pgfpathrectangle{\pgfqpoint{0.150000in}{0.150000in}}{\pgfqpoint{2.700000in}{1.950000in}}%
\pgfusepath{clip}%
\pgfsetbuttcap%
\pgfsetroundjoin%
\definecolor{currentfill}{rgb}{0.816529,0.839124,0.870757}%
\pgfsetfillcolor{currentfill}%
\pgfsetlinewidth{0.000000pt}%
\definecolor{currentstroke}{rgb}{0.000000,0.000000,0.000000}%
\pgfsetstrokecolor{currentstroke}%
\pgfsetdash{}{0pt}%
\pgfpathmoveto{\pgfqpoint{1.074315in}{0.917248in}}%
\pgfpathlineto{\pgfqpoint{1.112876in}{0.955445in}}%
\pgfpathlineto{\pgfqpoint{1.074417in}{0.993634in}}%
\pgfpathlineto{\pgfqpoint{1.036003in}{0.952419in}}%
\pgfpathclose%
\pgfusepath{fill}%
\end{pgfscope}%
\begin{pgfscope}%
\pgfpathrectangle{\pgfqpoint{0.150000in}{0.150000in}}{\pgfqpoint{2.700000in}{1.950000in}}%
\pgfusepath{clip}%
\pgfsetbuttcap%
\pgfsetroundjoin%
\definecolor{currentfill}{rgb}{0.586412,0.637347,0.708655}%
\pgfsetfillcolor{currentfill}%
\pgfsetlinewidth{0.000000pt}%
\definecolor{currentstroke}{rgb}{0.000000,0.000000,0.000000}%
\pgfsetstrokecolor{currentstroke}%
\pgfsetdash{}{0pt}%
\pgfpathmoveto{\pgfqpoint{1.189923in}{1.190637in}}%
\pgfpathlineto{\pgfqpoint{1.228374in}{1.231903in}}%
\pgfpathlineto{\pgfqpoint{1.189898in}{1.270065in}}%
\pgfpathlineto{\pgfqpoint{1.151458in}{1.228796in}}%
\pgfpathclose%
\pgfusepath{fill}%
\end{pgfscope}%
\begin{pgfscope}%
\pgfpathrectangle{\pgfqpoint{0.150000in}{0.150000in}}{\pgfqpoint{2.700000in}{1.950000in}}%
\pgfusepath{clip}%
\pgfsetbuttcap%
\pgfsetroundjoin%
\definecolor{currentfill}{rgb}{0.779213,0.806403,0.844470}%
\pgfsetfillcolor{currentfill}%
\pgfsetlinewidth{0.000000pt}%
\definecolor{currentstroke}{rgb}{0.000000,0.000000,0.000000}%
\pgfsetstrokecolor{currentstroke}%
\pgfsetdash{}{0pt}%
\pgfpathmoveto{\pgfqpoint{1.112876in}{0.955445in}}%
\pgfpathlineto{\pgfqpoint{1.151316in}{0.996672in}}%
\pgfpathlineto{\pgfqpoint{1.112970in}{1.031814in}}%
\pgfpathlineto{\pgfqpoint{1.074417in}{0.993634in}}%
\pgfpathclose%
\pgfusepath{fill}%
\end{pgfscope}%
\begin{pgfscope}%
\pgfpathrectangle{\pgfqpoint{0.150000in}{0.150000in}}{\pgfqpoint{2.700000in}{1.950000in}}%
\pgfusepath{clip}%
\pgfsetbuttcap%
\pgfsetroundjoin%
\definecolor{currentfill}{rgb}{0.847626,0.866391,0.892662}%
\pgfsetfillcolor{currentfill}%
\pgfsetlinewidth{0.000000pt}%
\definecolor{currentstroke}{rgb}{0.000000,0.000000,0.000000}%
\pgfsetstrokecolor{currentstroke}%
\pgfsetdash{}{0pt}%
\pgfpathmoveto{\pgfqpoint{1.575050in}{0.891077in}}%
\pgfpathlineto{\pgfqpoint{1.613402in}{0.902191in}}%
\pgfpathlineto{\pgfqpoint{1.574940in}{0.940333in}}%
\pgfpathlineto{\pgfqpoint{1.536486in}{0.929326in}}%
\pgfpathclose%
\pgfusepath{fill}%
\end{pgfscope}%
\begin{pgfscope}%
\pgfpathrectangle{\pgfqpoint{0.150000in}{0.150000in}}{\pgfqpoint{2.700000in}{1.950000in}}%
\pgfusepath{clip}%
\pgfsetbuttcap%
\pgfsetroundjoin%
\definecolor{currentfill}{rgb}{0.685922,0.724602,0.778753}%
\pgfsetfillcolor{currentfill}%
\pgfsetlinewidth{0.000000pt}%
\definecolor{currentstroke}{rgb}{0.000000,0.000000,0.000000}%
\pgfsetstrokecolor{currentstroke}%
\pgfsetdash{}{0pt}%
\pgfpathmoveto{\pgfqpoint{1.151401in}{1.073045in}}%
\pgfpathlineto{\pgfqpoint{1.189846in}{1.114292in}}%
\pgfpathlineto{\pgfqpoint{1.151373in}{1.152469in}}%
\pgfpathlineto{\pgfqpoint{1.112939in}{1.111219in}}%
\pgfpathclose%
\pgfusepath{fill}%
\end{pgfscope}%
\begin{pgfscope}%
\pgfpathrectangle{\pgfqpoint{0.150000in}{0.150000in}}{\pgfqpoint{2.700000in}{1.950000in}}%
\pgfusepath{clip}%
\pgfsetbuttcap%
\pgfsetroundjoin%
\definecolor{currentfill}{rgb}{0.748116,0.779136,0.822564}%
\pgfsetfillcolor{currentfill}%
\pgfsetlinewidth{0.000000pt}%
\definecolor{currentstroke}{rgb}{0.000000,0.000000,0.000000}%
\pgfsetstrokecolor{currentstroke}%
\pgfsetdash{}{0pt}%
\pgfpathmoveto{\pgfqpoint{1.151316in}{0.996672in}}%
\pgfpathlineto{\pgfqpoint{1.189770in}{1.037914in}}%
\pgfpathlineto{\pgfqpoint{1.151401in}{1.073045in}}%
\pgfpathlineto{\pgfqpoint{1.112970in}{1.031814in}}%
\pgfpathclose%
\pgfusepath{fill}%
\end{pgfscope}%
\begin{pgfscope}%
\pgfpathrectangle{\pgfqpoint{0.150000in}{0.150000in}}{\pgfqpoint{2.700000in}{1.950000in}}%
\pgfusepath{clip}%
\pgfsetbuttcap%
\pgfsetroundjoin%
\definecolor{currentfill}{rgb}{0.878722,0.893658,0.914568}%
\pgfsetfillcolor{currentfill}%
\pgfsetlinewidth{0.000000pt}%
\definecolor{currentstroke}{rgb}{0.000000,0.000000,0.000000}%
\pgfsetstrokecolor{currentstroke}%
\pgfsetdash{}{0pt}%
\pgfpathmoveto{\pgfqpoint{1.613622in}{0.852820in}}%
\pgfpathlineto{\pgfqpoint{1.651873in}{0.864040in}}%
\pgfpathlineto{\pgfqpoint{1.613402in}{0.902191in}}%
\pgfpathlineto{\pgfqpoint{1.575050in}{0.891077in}}%
\pgfpathclose%
\pgfusepath{fill}%
\end{pgfscope}%
\begin{pgfscope}%
\pgfpathrectangle{\pgfqpoint{0.150000in}{0.150000in}}{\pgfqpoint{2.700000in}{1.950000in}}%
\pgfusepath{clip}%
\pgfsetbuttcap%
\pgfsetroundjoin%
\definecolor{currentfill}{rgb}{0.648606,0.691881,0.752466}%
\pgfsetfillcolor{currentfill}%
\pgfsetlinewidth{0.000000pt}%
\definecolor{currentstroke}{rgb}{0.000000,0.000000,0.000000}%
\pgfsetstrokecolor{currentstroke}%
\pgfsetdash{}{0pt}%
\pgfpathmoveto{\pgfqpoint{1.189846in}{1.114292in}}%
\pgfpathlineto{\pgfqpoint{1.228306in}{1.155554in}}%
\pgfpathlineto{\pgfqpoint{1.189923in}{1.190637in}}%
\pgfpathlineto{\pgfqpoint{1.151373in}{1.152469in}}%
\pgfpathclose%
\pgfusepath{fill}%
\end{pgfscope}%
\begin{pgfscope}%
\pgfpathrectangle{\pgfqpoint{0.150000in}{0.150000in}}{\pgfqpoint{2.700000in}{1.950000in}}%
\pgfusepath{clip}%
\pgfsetbuttcap%
\pgfsetroundjoin%
\definecolor{currentfill}{rgb}{0.617509,0.664614,0.730561}%
\pgfsetfillcolor{currentfill}%
\pgfsetlinewidth{0.000000pt}%
\definecolor{currentstroke}{rgb}{0.000000,0.000000,0.000000}%
\pgfsetstrokecolor{currentstroke}%
\pgfsetdash{}{0pt}%
\pgfpathmoveto{\pgfqpoint{1.228306in}{1.155554in}}%
\pgfpathlineto{\pgfqpoint{1.266858in}{1.193733in}}%
\pgfpathlineto{\pgfqpoint{1.228374in}{1.231903in}}%
\pgfpathlineto{\pgfqpoint{1.189923in}{1.190637in}}%
\pgfpathclose%
\pgfusepath{fill}%
\end{pgfscope}%
\begin{pgfscope}%
\pgfpathrectangle{\pgfqpoint{0.150000in}{0.150000in}}{\pgfqpoint{2.700000in}{1.950000in}}%
\pgfusepath{clip}%
\pgfsetbuttcap%
\pgfsetroundjoin%
\definecolor{currentfill}{rgb}{0.717019,0.751869,0.800659}%
\pgfsetfillcolor{currentfill}%
\pgfsetlinewidth{0.000000pt}%
\definecolor{currentstroke}{rgb}{0.000000,0.000000,0.000000}%
\pgfsetstrokecolor{currentstroke}%
\pgfsetdash{}{0pt}%
\pgfpathmoveto{\pgfqpoint{1.189770in}{1.037914in}}%
\pgfpathlineto{\pgfqpoint{1.228328in}{1.076107in}}%
\pgfpathlineto{\pgfqpoint{1.189846in}{1.114292in}}%
\pgfpathlineto{\pgfqpoint{1.151401in}{1.073045in}}%
\pgfpathclose%
\pgfusepath{fill}%
\end{pgfscope}%
\begin{pgfscope}%
\pgfpathrectangle{\pgfqpoint{0.150000in}{0.150000in}}{\pgfqpoint{2.700000in}{1.950000in}}%
\pgfusepath{clip}%
\pgfsetbuttcap%
\pgfsetroundjoin%
\definecolor{currentfill}{rgb}{0.841406,0.860938,0.888281}%
\pgfsetfillcolor{currentfill}%
\pgfsetlinewidth{0.000000pt}%
\definecolor{currentstroke}{rgb}{0.000000,0.000000,0.000000}%
\pgfsetstrokecolor{currentstroke}%
\pgfsetdash{}{0pt}%
\pgfpathmoveto{\pgfqpoint{1.112783in}{0.879043in}}%
\pgfpathlineto{\pgfqpoint{1.151231in}{0.920265in}}%
\pgfpathlineto{\pgfqpoint{1.112876in}{0.955445in}}%
\pgfpathlineto{\pgfqpoint{1.074315in}{0.917248in}}%
\pgfpathclose%
\pgfusepath{fill}%
\end{pgfscope}%
\begin{pgfscope}%
\pgfpathrectangle{\pgfqpoint{0.150000in}{0.150000in}}{\pgfqpoint{2.700000in}{1.950000in}}%
\pgfusepath{clip}%
\pgfsetbuttcap%
\pgfsetroundjoin%
\definecolor{currentfill}{rgb}{0.909819,0.920925,0.936474}%
\pgfsetfillcolor{currentfill}%
\pgfsetlinewidth{0.000000pt}%
\definecolor{currentstroke}{rgb}{0.000000,0.000000,0.000000}%
\pgfsetstrokecolor{currentstroke}%
\pgfsetdash{}{0pt}%
\pgfpathmoveto{\pgfqpoint{1.652203in}{0.814554in}}%
\pgfpathlineto{\pgfqpoint{1.690352in}{0.825880in}}%
\pgfpathlineto{\pgfqpoint{1.651873in}{0.864040in}}%
\pgfpathlineto{\pgfqpoint{1.613622in}{0.852820in}}%
\pgfpathclose%
\pgfusepath{fill}%
\end{pgfscope}%
\begin{pgfscope}%
\pgfpathrectangle{\pgfqpoint{0.150000in}{0.150000in}}{\pgfqpoint{2.700000in}{1.950000in}}%
\pgfusepath{clip}%
\pgfsetbuttcap%
\pgfsetroundjoin%
\definecolor{currentfill}{rgb}{0.810309,0.833670,0.866376}%
\pgfsetfillcolor{currentfill}%
\pgfsetlinewidth{0.000000pt}%
\definecolor{currentstroke}{rgb}{0.000000,0.000000,0.000000}%
\pgfsetstrokecolor{currentstroke}%
\pgfsetdash{}{0pt}%
\pgfpathmoveto{\pgfqpoint{1.151231in}{0.920265in}}%
\pgfpathlineto{\pgfqpoint{1.189795in}{0.958473in}}%
\pgfpathlineto{\pgfqpoint{1.151316in}{0.996672in}}%
\pgfpathlineto{\pgfqpoint{1.112876in}{0.955445in}}%
\pgfpathclose%
\pgfusepath{fill}%
\end{pgfscope}%
\begin{pgfscope}%
\pgfpathrectangle{\pgfqpoint{0.150000in}{0.150000in}}{\pgfqpoint{2.700000in}{1.950000in}}%
\pgfusepath{clip}%
\pgfsetbuttcap%
\pgfsetroundjoin%
\definecolor{currentfill}{rgb}{0.679703,0.719148,0.774372}%
\pgfsetfillcolor{currentfill}%
\pgfsetlinewidth{0.000000pt}%
\definecolor{currentstroke}{rgb}{0.000000,0.000000,0.000000}%
\pgfsetstrokecolor{currentstroke}%
\pgfsetdash{}{0pt}%
\pgfpathmoveto{\pgfqpoint{1.228328in}{1.076107in}}%
\pgfpathlineto{\pgfqpoint{1.266799in}{1.117367in}}%
\pgfpathlineto{\pgfqpoint{1.228306in}{1.155554in}}%
\pgfpathlineto{\pgfqpoint{1.189846in}{1.114292in}}%
\pgfpathclose%
\pgfusepath{fill}%
\end{pgfscope}%
\begin{pgfscope}%
\pgfpathrectangle{\pgfqpoint{0.150000in}{0.150000in}}{\pgfqpoint{2.700000in}{1.950000in}}%
\pgfusepath{clip}%
\pgfsetbuttcap%
\pgfsetroundjoin%
\definecolor{currentfill}{rgb}{0.934697,0.942739,0.953998}%
\pgfsetfillcolor{currentfill}%
\pgfsetlinewidth{0.000000pt}%
\definecolor{currentstroke}{rgb}{0.000000,0.000000,0.000000}%
\pgfsetstrokecolor{currentstroke}%
\pgfsetdash{}{0pt}%
\pgfpathmoveto{\pgfqpoint{1.690792in}{0.776280in}}%
\pgfpathlineto{\pgfqpoint{1.728840in}{0.787712in}}%
\pgfpathlineto{\pgfqpoint{1.690352in}{0.825880in}}%
\pgfpathlineto{\pgfqpoint{1.652203in}{0.814554in}}%
\pgfpathclose%
\pgfusepath{fill}%
\end{pgfscope}%
\begin{pgfscope}%
\pgfpathrectangle{\pgfqpoint{0.150000in}{0.150000in}}{\pgfqpoint{2.700000in}{1.950000in}}%
\pgfusepath{clip}%
\pgfsetbuttcap%
\pgfsetroundjoin%
\definecolor{currentfill}{rgb}{0.648606,0.691881,0.752466}%
\pgfsetfillcolor{currentfill}%
\pgfsetlinewidth{0.000000pt}%
\definecolor{currentstroke}{rgb}{0.000000,0.000000,0.000000}%
\pgfsetstrokecolor{currentstroke}%
\pgfsetdash{}{0pt}%
\pgfpathmoveto{\pgfqpoint{1.266799in}{1.117367in}}%
\pgfpathlineto{\pgfqpoint{1.305351in}{1.155554in}}%
\pgfpathlineto{\pgfqpoint{1.266858in}{1.193733in}}%
\pgfpathlineto{\pgfqpoint{1.228306in}{1.155554in}}%
\pgfpathclose%
\pgfusepath{fill}%
\end{pgfscope}%
\begin{pgfscope}%
\pgfpathrectangle{\pgfqpoint{0.150000in}{0.150000in}}{\pgfqpoint{2.700000in}{1.950000in}}%
\pgfusepath{clip}%
\pgfsetbuttcap%
\pgfsetroundjoin%
\definecolor{currentfill}{rgb}{0.772993,0.800950,0.840089}%
\pgfsetfillcolor{currentfill}%
\pgfsetlinewidth{0.000000pt}%
\definecolor{currentstroke}{rgb}{0.000000,0.000000,0.000000}%
\pgfsetstrokecolor{currentstroke}%
\pgfsetdash{}{0pt}%
\pgfpathmoveto{\pgfqpoint{1.189795in}{0.958473in}}%
\pgfpathlineto{\pgfqpoint{1.228260in}{0.999712in}}%
\pgfpathlineto{\pgfqpoint{1.189770in}{1.037914in}}%
\pgfpathlineto{\pgfqpoint{1.151316in}{0.996672in}}%
\pgfpathclose%
\pgfusepath{fill}%
\end{pgfscope}%
\begin{pgfscope}%
\pgfpathrectangle{\pgfqpoint{0.150000in}{0.150000in}}{\pgfqpoint{2.700000in}{1.950000in}}%
\pgfusepath{clip}%
\pgfsetbuttcap%
\pgfsetroundjoin%
\definecolor{currentfill}{rgb}{0.741896,0.773683,0.818183}%
\pgfsetfillcolor{currentfill}%
\pgfsetlinewidth{0.000000pt}%
\definecolor{currentstroke}{rgb}{0.000000,0.000000,0.000000}%
\pgfsetstrokecolor{currentstroke}%
\pgfsetdash{}{0pt}%
\pgfpathmoveto{\pgfqpoint{1.228260in}{0.999712in}}%
\pgfpathlineto{\pgfqpoint{1.266818in}{1.037914in}}%
\pgfpathlineto{\pgfqpoint{1.228328in}{1.076107in}}%
\pgfpathlineto{\pgfqpoint{1.189770in}{1.037914in}}%
\pgfpathclose%
\pgfusepath{fill}%
\end{pgfscope}%
\begin{pgfscope}%
\pgfpathrectangle{\pgfqpoint{0.150000in}{0.150000in}}{\pgfqpoint{2.700000in}{1.950000in}}%
\pgfusepath{clip}%
\pgfsetbuttcap%
\pgfsetroundjoin%
\definecolor{currentfill}{rgb}{0.710800,0.746415,0.796278}%
\pgfsetfillcolor{currentfill}%
\pgfsetlinewidth{0.000000pt}%
\definecolor{currentstroke}{rgb}{0.000000,0.000000,0.000000}%
\pgfsetstrokecolor{currentstroke}%
\pgfsetdash{}{0pt}%
\pgfpathmoveto{\pgfqpoint{1.266818in}{1.037914in}}%
\pgfpathlineto{\pgfqpoint{1.305300in}{1.079171in}}%
\pgfpathlineto{\pgfqpoint{1.266799in}{1.117367in}}%
\pgfpathlineto{\pgfqpoint{1.228328in}{1.076107in}}%
\pgfpathclose%
\pgfusepath{fill}%
\end{pgfscope}%
\begin{pgfscope}%
\pgfpathrectangle{\pgfqpoint{0.150000in}{0.150000in}}{\pgfqpoint{2.700000in}{1.950000in}}%
\pgfusepath{clip}%
\pgfsetbuttcap%
\pgfsetroundjoin%
\definecolor{currentfill}{rgb}{0.965794,0.970006,0.975904}%
\pgfsetfillcolor{currentfill}%
\pgfsetlinewidth{0.000000pt}%
\definecolor{currentstroke}{rgb}{0.000000,0.000000,0.000000}%
\pgfsetstrokecolor{currentstroke}%
\pgfsetdash{}{0pt}%
\pgfpathmoveto{\pgfqpoint{1.729390in}{0.737998in}}%
\pgfpathlineto{\pgfqpoint{1.767336in}{0.749536in}}%
\pgfpathlineto{\pgfqpoint{1.728840in}{0.787712in}}%
\pgfpathlineto{\pgfqpoint{1.690792in}{0.776280in}}%
\pgfpathclose%
\pgfusepath{fill}%
\end{pgfscope}%
\begin{pgfscope}%
\pgfpathrectangle{\pgfqpoint{0.150000in}{0.150000in}}{\pgfqpoint{2.700000in}{1.950000in}}%
\pgfusepath{clip}%
\pgfsetbuttcap%
\pgfsetroundjoin%
\definecolor{currentfill}{rgb}{0.679703,0.719148,0.774372}%
\pgfsetfillcolor{currentfill}%
\pgfsetlinewidth{0.000000pt}%
\definecolor{currentstroke}{rgb}{0.000000,0.000000,0.000000}%
\pgfsetstrokecolor{currentstroke}%
\pgfsetdash{}{0pt}%
\pgfpathmoveto{\pgfqpoint{1.305300in}{1.079171in}}%
\pgfpathlineto{\pgfqpoint{1.343852in}{1.117367in}}%
\pgfpathlineto{\pgfqpoint{1.305351in}{1.155554in}}%
\pgfpathlineto{\pgfqpoint{1.266799in}{1.117367in}}%
\pgfpathclose%
\pgfusepath{fill}%
\end{pgfscope}%
\begin{pgfscope}%
\pgfpathrectangle{\pgfqpoint{0.150000in}{0.150000in}}{\pgfqpoint{2.700000in}{1.950000in}}%
\pgfusepath{clip}%
\pgfsetbuttcap%
\pgfsetroundjoin%
\definecolor{currentfill}{rgb}{0.872503,0.888205,0.910187}%
\pgfsetfillcolor{currentfill}%
\pgfsetlinewidth{0.000000pt}%
\definecolor{currentstroke}{rgb}{0.000000,0.000000,0.000000}%
\pgfsetstrokecolor{currentstroke}%
\pgfsetdash{}{0pt}%
\pgfpathmoveto{\pgfqpoint{1.151146in}{0.843824in}}%
\pgfpathlineto{\pgfqpoint{1.189719in}{0.882049in}}%
\pgfpathlineto{\pgfqpoint{1.151231in}{0.920265in}}%
\pgfpathlineto{\pgfqpoint{1.112783in}{0.879043in}}%
\pgfpathclose%
\pgfusepath{fill}%
\end{pgfscope}%
\begin{pgfscope}%
\pgfpathrectangle{\pgfqpoint{0.150000in}{0.150000in}}{\pgfqpoint{2.700000in}{1.950000in}}%
\pgfusepath{clip}%
\pgfsetbuttcap%
\pgfsetroundjoin%
\definecolor{currentfill}{rgb}{0.835187,0.855484,0.883900}%
\pgfsetfillcolor{currentfill}%
\pgfsetlinewidth{0.000000pt}%
\definecolor{currentstroke}{rgb}{0.000000,0.000000,0.000000}%
\pgfsetstrokecolor{currentstroke}%
\pgfsetdash{}{0pt}%
\pgfpathmoveto{\pgfqpoint{1.189719in}{0.882049in}}%
\pgfpathlineto{\pgfqpoint{1.228192in}{0.923284in}}%
\pgfpathlineto{\pgfqpoint{1.189795in}{0.958473in}}%
\pgfpathlineto{\pgfqpoint{1.151231in}{0.920265in}}%
\pgfpathclose%
\pgfusepath{fill}%
\end{pgfscope}%
\begin{pgfscope}%
\pgfpathrectangle{\pgfqpoint{0.150000in}{0.150000in}}{\pgfqpoint{2.700000in}{1.950000in}}%
\pgfusepath{clip}%
\pgfsetbuttcap%
\pgfsetroundjoin%
\definecolor{currentfill}{rgb}{0.804090,0.828217,0.861994}%
\pgfsetfillcolor{currentfill}%
\pgfsetlinewidth{0.000000pt}%
\definecolor{currentstroke}{rgb}{0.000000,0.000000,0.000000}%
\pgfsetstrokecolor{currentstroke}%
\pgfsetdash{}{0pt}%
\pgfpathmoveto{\pgfqpoint{1.228192in}{0.923284in}}%
\pgfpathlineto{\pgfqpoint{1.266759in}{0.961502in}}%
\pgfpathlineto{\pgfqpoint{1.228260in}{0.999712in}}%
\pgfpathlineto{\pgfqpoint{1.189795in}{0.958473in}}%
\pgfpathclose%
\pgfusepath{fill}%
\end{pgfscope}%
\begin{pgfscope}%
\pgfpathrectangle{\pgfqpoint{0.150000in}{0.150000in}}{\pgfqpoint{2.700000in}{1.950000in}}%
\pgfusepath{clip}%
\pgfsetbuttcap%
\pgfsetroundjoin%
\definecolor{currentfill}{rgb}{0.996890,0.997273,0.997809}%
\pgfsetfillcolor{currentfill}%
\pgfsetlinewidth{0.000000pt}%
\definecolor{currentstroke}{rgb}{0.000000,0.000000,0.000000}%
\pgfsetstrokecolor{currentstroke}%
\pgfsetdash{}{0pt}%
\pgfpathmoveto{\pgfqpoint{1.767996in}{0.699707in}}%
\pgfpathlineto{\pgfqpoint{1.805841in}{0.711351in}}%
\pgfpathlineto{\pgfqpoint{1.767336in}{0.749536in}}%
\pgfpathlineto{\pgfqpoint{1.729390in}{0.737998in}}%
\pgfpathclose%
\pgfusepath{fill}%
\end{pgfscope}%
\begin{pgfscope}%
\pgfpathrectangle{\pgfqpoint{0.150000in}{0.150000in}}{\pgfqpoint{2.700000in}{1.950000in}}%
\pgfusepath{clip}%
\pgfsetbuttcap%
\pgfsetroundjoin%
\definecolor{currentfill}{rgb}{0.710800,0.746415,0.796278}%
\pgfsetfillcolor{currentfill}%
\pgfsetlinewidth{0.000000pt}%
\definecolor{currentstroke}{rgb}{0.000000,0.000000,0.000000}%
\pgfsetstrokecolor{currentstroke}%
\pgfsetdash{}{0pt}%
\pgfpathmoveto{\pgfqpoint{1.343810in}{1.040967in}}%
\pgfpathlineto{\pgfqpoint{1.382362in}{1.079171in}}%
\pgfpathlineto{\pgfqpoint{1.343852in}{1.117367in}}%
\pgfpathlineto{\pgfqpoint{1.305300in}{1.079171in}}%
\pgfpathclose%
\pgfusepath{fill}%
\end{pgfscope}%
\begin{pgfscope}%
\pgfpathrectangle{\pgfqpoint{0.150000in}{0.150000in}}{\pgfqpoint{2.700000in}{1.950000in}}%
\pgfusepath{clip}%
\pgfsetbuttcap%
\pgfsetroundjoin%
\definecolor{currentfill}{rgb}{0.772993,0.800950,0.840089}%
\pgfsetfillcolor{currentfill}%
\pgfsetlinewidth{0.000000pt}%
\definecolor{currentstroke}{rgb}{0.000000,0.000000,0.000000}%
\pgfsetstrokecolor{currentstroke}%
\pgfsetdash{}{0pt}%
\pgfpathmoveto{\pgfqpoint{1.266759in}{0.961502in}}%
\pgfpathlineto{\pgfqpoint{1.305249in}{1.002754in}}%
\pgfpathlineto{\pgfqpoint{1.266818in}{1.037914in}}%
\pgfpathlineto{\pgfqpoint{1.228260in}{0.999712in}}%
\pgfpathclose%
\pgfusepath{fill}%
\end{pgfscope}%
\begin{pgfscope}%
\pgfpathrectangle{\pgfqpoint{0.150000in}{0.150000in}}{\pgfqpoint{2.700000in}{1.950000in}}%
\pgfusepath{clip}%
\pgfsetbuttcap%
\pgfsetroundjoin%
\definecolor{currentfill}{rgb}{0.741896,0.773683,0.818183}%
\pgfsetfillcolor{currentfill}%
\pgfsetlinewidth{0.000000pt}%
\definecolor{currentstroke}{rgb}{0.000000,0.000000,0.000000}%
\pgfsetstrokecolor{currentstroke}%
\pgfsetdash{}{0pt}%
\pgfpathmoveto{\pgfqpoint{1.305249in}{1.002754in}}%
\pgfpathlineto{\pgfqpoint{1.343810in}{1.040967in}}%
\pgfpathlineto{\pgfqpoint{1.305300in}{1.079171in}}%
\pgfpathlineto{\pgfqpoint{1.266818in}{1.037914in}}%
\pgfpathclose%
\pgfusepath{fill}%
\end{pgfscope}%
\begin{pgfscope}%
\pgfpathrectangle{\pgfqpoint{0.150000in}{0.150000in}}{\pgfqpoint{2.700000in}{1.950000in}}%
\pgfusepath{clip}%
\pgfsetbuttcap%
\pgfsetroundjoin%
\definecolor{currentfill}{rgb}{0.735677,0.768229,0.813802}%
\pgfsetfillcolor{currentfill}%
\pgfsetlinewidth{0.000000pt}%
\definecolor{currentstroke}{rgb}{0.000000,0.000000,0.000000}%
\pgfsetstrokecolor{currentstroke}%
\pgfsetdash{}{0pt}%
\pgfpathmoveto{\pgfqpoint{1.382328in}{1.002754in}}%
\pgfpathlineto{\pgfqpoint{1.420880in}{1.040967in}}%
\pgfpathlineto{\pgfqpoint{1.382362in}{1.079171in}}%
\pgfpathlineto{\pgfqpoint{1.343810in}{1.040967in}}%
\pgfpathclose%
\pgfusepath{fill}%
\end{pgfscope}%
\begin{pgfscope}%
\pgfpathrectangle{\pgfqpoint{0.150000in}{0.150000in}}{\pgfqpoint{2.700000in}{1.950000in}}%
\pgfusepath{clip}%
\pgfsetbuttcap%
\pgfsetroundjoin%
\definecolor{currentfill}{rgb}{0.866284,0.882751,0.905806}%
\pgfsetfillcolor{currentfill}%
\pgfsetlinewidth{0.000000pt}%
\definecolor{currentstroke}{rgb}{0.000000,0.000000,0.000000}%
\pgfsetstrokecolor{currentstroke}%
\pgfsetdash{}{0pt}%
\pgfpathmoveto{\pgfqpoint{1.228124in}{0.846821in}}%
\pgfpathlineto{\pgfqpoint{1.266699in}{0.885057in}}%
\pgfpathlineto{\pgfqpoint{1.228192in}{0.923284in}}%
\pgfpathlineto{\pgfqpoint{1.189719in}{0.882049in}}%
\pgfpathclose%
\pgfusepath{fill}%
\end{pgfscope}%
\begin{pgfscope}%
\pgfpathrectangle{\pgfqpoint{0.150000in}{0.150000in}}{\pgfqpoint{2.700000in}{1.950000in}}%
\pgfusepath{clip}%
\pgfsetbuttcap%
\pgfsetroundjoin%
\definecolor{currentfill}{rgb}{0.766774,0.795496,0.835708}%
\pgfsetfillcolor{currentfill}%
\pgfsetlinewidth{0.000000pt}%
\definecolor{currentstroke}{rgb}{0.000000,0.000000,0.000000}%
\pgfsetstrokecolor{currentstroke}%
\pgfsetdash{}{0pt}%
\pgfpathmoveto{\pgfqpoint{1.420855in}{0.964533in}}%
\pgfpathlineto{\pgfqpoint{1.459407in}{1.002754in}}%
\pgfpathlineto{\pgfqpoint{1.420880in}{1.040967in}}%
\pgfpathlineto{\pgfqpoint{1.382328in}{1.002754in}}%
\pgfpathclose%
\pgfusepath{fill}%
\end{pgfscope}%
\begin{pgfscope}%
\pgfpathrectangle{\pgfqpoint{0.150000in}{0.150000in}}{\pgfqpoint{2.700000in}{1.950000in}}%
\pgfusepath{clip}%
\pgfsetbuttcap%
\pgfsetroundjoin%
\definecolor{currentfill}{rgb}{0.766774,0.795496,0.835708}%
\pgfsetfillcolor{currentfill}%
\pgfsetlinewidth{0.000000pt}%
\definecolor{currentstroke}{rgb}{0.000000,0.000000,0.000000}%
\pgfsetstrokecolor{currentstroke}%
\pgfsetdash{}{0pt}%
\pgfpathmoveto{\pgfqpoint{1.343767in}{0.964533in}}%
\pgfpathlineto{\pgfqpoint{1.382328in}{1.002754in}}%
\pgfpathlineto{\pgfqpoint{1.343810in}{1.040967in}}%
\pgfpathlineto{\pgfqpoint{1.305249in}{1.002754in}}%
\pgfpathclose%
\pgfusepath{fill}%
\end{pgfscope}%
\begin{pgfscope}%
\pgfpathrectangle{\pgfqpoint{0.150000in}{0.150000in}}{\pgfqpoint{2.700000in}{1.950000in}}%
\pgfusepath{clip}%
\pgfsetbuttcap%
\pgfsetroundjoin%
\definecolor{currentfill}{rgb}{0.835187,0.855484,0.883900}%
\pgfsetfillcolor{currentfill}%
\pgfsetlinewidth{0.000000pt}%
\definecolor{currentstroke}{rgb}{0.000000,0.000000,0.000000}%
\pgfsetstrokecolor{currentstroke}%
\pgfsetdash{}{0pt}%
\pgfpathmoveto{\pgfqpoint{1.266699in}{0.885057in}}%
\pgfpathlineto{\pgfqpoint{1.305198in}{0.926304in}}%
\pgfpathlineto{\pgfqpoint{1.266759in}{0.961502in}}%
\pgfpathlineto{\pgfqpoint{1.228192in}{0.923284in}}%
\pgfpathclose%
\pgfusepath{fill}%
\end{pgfscope}%
\begin{pgfscope}%
\pgfpathrectangle{\pgfqpoint{0.150000in}{0.150000in}}{\pgfqpoint{2.700000in}{1.950000in}}%
\pgfusepath{clip}%
\pgfsetbuttcap%
\pgfsetroundjoin%
\definecolor{currentfill}{rgb}{0.897381,0.910018,0.927711}%
\pgfsetfillcolor{currentfill}%
\pgfsetlinewidth{0.000000pt}%
\definecolor{currentstroke}{rgb}{0.000000,0.000000,0.000000}%
\pgfsetstrokecolor{currentstroke}%
\pgfsetdash{}{0pt}%
\pgfpathmoveto{\pgfqpoint{1.189540in}{0.808577in}}%
\pgfpathlineto{\pgfqpoint{1.228124in}{0.846821in}}%
\pgfpathlineto{\pgfqpoint{1.189719in}{0.882049in}}%
\pgfpathlineto{\pgfqpoint{1.151146in}{0.843824in}}%
\pgfpathclose%
\pgfusepath{fill}%
\end{pgfscope}%
\begin{pgfscope}%
\pgfpathrectangle{\pgfqpoint{0.150000in}{0.150000in}}{\pgfqpoint{2.700000in}{1.950000in}}%
\pgfusepath{clip}%
\pgfsetbuttcap%
\pgfsetroundjoin%
\definecolor{currentfill}{rgb}{0.797871,0.822763,0.857613}%
\pgfsetfillcolor{currentfill}%
\pgfsetlinewidth{0.000000pt}%
\definecolor{currentstroke}{rgb}{0.000000,0.000000,0.000000}%
\pgfsetstrokecolor{currentstroke}%
\pgfsetdash{}{0pt}%
\pgfpathmoveto{\pgfqpoint{1.305198in}{0.926304in}}%
\pgfpathlineto{\pgfqpoint{1.343767in}{0.964533in}}%
\pgfpathlineto{\pgfqpoint{1.305249in}{1.002754in}}%
\pgfpathlineto{\pgfqpoint{1.266759in}{0.961502in}}%
\pgfpathclose%
\pgfusepath{fill}%
\end{pgfscope}%
\begin{pgfscope}%
\pgfpathrectangle{\pgfqpoint{0.150000in}{0.150000in}}{\pgfqpoint{2.700000in}{1.950000in}}%
\pgfusepath{clip}%
\pgfsetbuttcap%
\pgfsetroundjoin%
\definecolor{currentfill}{rgb}{0.797871,0.822763,0.857613}%
\pgfsetfillcolor{currentfill}%
\pgfsetlinewidth{0.000000pt}%
\definecolor{currentstroke}{rgb}{0.000000,0.000000,0.000000}%
\pgfsetstrokecolor{currentstroke}%
\pgfsetdash{}{0pt}%
\pgfpathmoveto{\pgfqpoint{1.459390in}{0.926304in}}%
\pgfpathlineto{\pgfqpoint{1.497943in}{0.964533in}}%
\pgfpathlineto{\pgfqpoint{1.459407in}{1.002754in}}%
\pgfpathlineto{\pgfqpoint{1.420855in}{0.964533in}}%
\pgfpathclose%
\pgfusepath{fill}%
\end{pgfscope}%
\begin{pgfscope}%
\pgfpathrectangle{\pgfqpoint{0.150000in}{0.150000in}}{\pgfqpoint{2.700000in}{1.950000in}}%
\pgfusepath{clip}%
\pgfsetbuttcap%
\pgfsetroundjoin%
\definecolor{currentfill}{rgb}{0.797871,0.822763,0.857613}%
\pgfsetfillcolor{currentfill}%
\pgfsetlinewidth{0.000000pt}%
\definecolor{currentstroke}{rgb}{0.000000,0.000000,0.000000}%
\pgfsetstrokecolor{currentstroke}%
\pgfsetdash{}{0pt}%
\pgfpathmoveto{\pgfqpoint{1.382294in}{0.926304in}}%
\pgfpathlineto{\pgfqpoint{1.420855in}{0.964533in}}%
\pgfpathlineto{\pgfqpoint{1.382328in}{1.002754in}}%
\pgfpathlineto{\pgfqpoint{1.343767in}{0.964533in}}%
\pgfpathclose%
\pgfusepath{fill}%
\end{pgfscope}%
\begin{pgfscope}%
\pgfpathrectangle{\pgfqpoint{0.150000in}{0.150000in}}{\pgfqpoint{2.700000in}{1.950000in}}%
\pgfusepath{clip}%
\pgfsetbuttcap%
\pgfsetroundjoin%
\definecolor{currentfill}{rgb}{0.828968,0.850031,0.879519}%
\pgfsetfillcolor{currentfill}%
\pgfsetlinewidth{0.000000pt}%
\definecolor{currentstroke}{rgb}{0.000000,0.000000,0.000000}%
\pgfsetstrokecolor{currentstroke}%
\pgfsetdash{}{0pt}%
\pgfpathmoveto{\pgfqpoint{1.420829in}{0.888066in}}%
\pgfpathlineto{\pgfqpoint{1.459390in}{0.926304in}}%
\pgfpathlineto{\pgfqpoint{1.420855in}{0.964533in}}%
\pgfpathlineto{\pgfqpoint{1.382294in}{0.926304in}}%
\pgfpathclose%
\pgfusepath{fill}%
\end{pgfscope}%
\begin{pgfscope}%
\pgfpathrectangle{\pgfqpoint{0.150000in}{0.150000in}}{\pgfqpoint{2.700000in}{1.950000in}}%
\pgfusepath{clip}%
\pgfsetbuttcap%
\pgfsetroundjoin%
\definecolor{currentfill}{rgb}{0.828968,0.850031,0.879519}%
\pgfsetfillcolor{currentfill}%
\pgfsetlinewidth{0.000000pt}%
\definecolor{currentstroke}{rgb}{0.000000,0.000000,0.000000}%
\pgfsetstrokecolor{currentstroke}%
\pgfsetdash{}{0pt}%
\pgfpathmoveto{\pgfqpoint{1.343725in}{0.888066in}}%
\pgfpathlineto{\pgfqpoint{1.382294in}{0.926304in}}%
\pgfpathlineto{\pgfqpoint{1.343767in}{0.964533in}}%
\pgfpathlineto{\pgfqpoint{1.305198in}{0.926304in}}%
\pgfpathclose%
\pgfusepath{fill}%
\end{pgfscope}%
\begin{pgfscope}%
\pgfpathrectangle{\pgfqpoint{0.150000in}{0.150000in}}{\pgfqpoint{2.700000in}{1.950000in}}%
\pgfusepath{clip}%
\pgfsetbuttcap%
\pgfsetroundjoin%
\definecolor{currentfill}{rgb}{0.860064,0.877298,0.901425}%
\pgfsetfillcolor{currentfill}%
\pgfsetlinewidth{0.000000pt}%
\definecolor{currentstroke}{rgb}{0.000000,0.000000,0.000000}%
\pgfsetstrokecolor{currentstroke}%
\pgfsetdash{}{0pt}%
\pgfpathmoveto{\pgfqpoint{1.305147in}{0.849820in}}%
\pgfpathlineto{\pgfqpoint{1.343725in}{0.888066in}}%
\pgfpathlineto{\pgfqpoint{1.305198in}{0.926304in}}%
\pgfpathlineto{\pgfqpoint{1.266699in}{0.885057in}}%
\pgfpathclose%
\pgfusepath{fill}%
\end{pgfscope}%
\begin{pgfscope}%
\pgfpathrectangle{\pgfqpoint{0.150000in}{0.150000in}}{\pgfqpoint{2.700000in}{1.950000in}}%
\pgfusepath{clip}%
\pgfsetbuttcap%
\pgfsetroundjoin%
\definecolor{currentfill}{rgb}{0.891161,0.904565,0.923330}%
\pgfsetfillcolor{currentfill}%
\pgfsetlinewidth{0.000000pt}%
\definecolor{currentstroke}{rgb}{0.000000,0.000000,0.000000}%
\pgfsetstrokecolor{currentstroke}%
\pgfsetdash{}{0pt}%
\pgfpathmoveto{\pgfqpoint{1.266560in}{0.811565in}}%
\pgfpathlineto{\pgfqpoint{1.305147in}{0.849820in}}%
\pgfpathlineto{\pgfqpoint{1.266699in}{0.885057in}}%
\pgfpathlineto{\pgfqpoint{1.228124in}{0.846821in}}%
\pgfpathclose%
\pgfusepath{fill}%
\end{pgfscope}%
\begin{pgfscope}%
\pgfpathrectangle{\pgfqpoint{0.150000in}{0.150000in}}{\pgfqpoint{2.700000in}{1.950000in}}%
\pgfusepath{clip}%
\pgfsetbuttcap%
\pgfsetroundjoin%
\definecolor{currentfill}{rgb}{0.860064,0.877298,0.901425}%
\pgfsetfillcolor{currentfill}%
\pgfsetlinewidth{0.000000pt}%
\definecolor{currentstroke}{rgb}{0.000000,0.000000,0.000000}%
\pgfsetstrokecolor{currentstroke}%
\pgfsetdash{}{0pt}%
\pgfpathmoveto{\pgfqpoint{1.382260in}{0.849820in}}%
\pgfpathlineto{\pgfqpoint{1.420829in}{0.888066in}}%
\pgfpathlineto{\pgfqpoint{1.382294in}{0.926304in}}%
\pgfpathlineto{\pgfqpoint{1.343725in}{0.888066in}}%
\pgfpathclose%
\pgfusepath{fill}%
\end{pgfscope}%
\begin{pgfscope}%
\pgfpathrectangle{\pgfqpoint{0.150000in}{0.150000in}}{\pgfqpoint{2.700000in}{1.950000in}}%
\pgfusepath{clip}%
\pgfsetbuttcap%
\pgfsetroundjoin%
\definecolor{currentfill}{rgb}{0.922258,0.931832,0.945236}%
\pgfsetfillcolor{currentfill}%
\pgfsetlinewidth{0.000000pt}%
\definecolor{currentstroke}{rgb}{0.000000,0.000000,0.000000}%
\pgfsetstrokecolor{currentstroke}%
\pgfsetdash{}{0pt}%
\pgfpathmoveto{\pgfqpoint{1.228056in}{0.770325in}}%
\pgfpathlineto{\pgfqpoint{1.266560in}{0.811565in}}%
\pgfpathlineto{\pgfqpoint{1.228124in}{0.846821in}}%
\pgfpathlineto{\pgfqpoint{1.189540in}{0.808577in}}%
\pgfpathclose%
\pgfusepath{fill}%
\end{pgfscope}%
\begin{pgfscope}%
\pgfpathrectangle{\pgfqpoint{0.150000in}{0.150000in}}{\pgfqpoint{2.700000in}{1.950000in}}%
\pgfusepath{clip}%
\pgfsetbuttcap%
\pgfsetroundjoin%
\definecolor{currentfill}{rgb}{0.828968,0.850031,0.879519}%
\pgfsetfillcolor{currentfill}%
\pgfsetlinewidth{0.000000pt}%
\definecolor{currentstroke}{rgb}{0.000000,0.000000,0.000000}%
\pgfsetstrokecolor{currentstroke}%
\pgfsetdash{}{0pt}%
\pgfpathmoveto{\pgfqpoint{1.497923in}{0.891077in}}%
\pgfpathlineto{\pgfqpoint{1.536486in}{0.929326in}}%
\pgfpathlineto{\pgfqpoint{1.497943in}{0.964533in}}%
\pgfpathlineto{\pgfqpoint{1.459390in}{0.926304in}}%
\pgfpathclose%
\pgfusepath{fill}%
\end{pgfscope}%
\begin{pgfscope}%
\pgfpathrectangle{\pgfqpoint{0.150000in}{0.150000in}}{\pgfqpoint{2.700000in}{1.950000in}}%
\pgfusepath{clip}%
\pgfsetbuttcap%
\pgfsetroundjoin%
\definecolor{currentfill}{rgb}{0.891161,0.904565,0.923330}%
\pgfsetfillcolor{currentfill}%
\pgfsetlinewidth{0.000000pt}%
\definecolor{currentstroke}{rgb}{0.000000,0.000000,0.000000}%
\pgfsetstrokecolor{currentstroke}%
\pgfsetdash{}{0pt}%
\pgfpathmoveto{\pgfqpoint{1.343682in}{0.811565in}}%
\pgfpathlineto{\pgfqpoint{1.382260in}{0.849820in}}%
\pgfpathlineto{\pgfqpoint{1.343725in}{0.888066in}}%
\pgfpathlineto{\pgfqpoint{1.305147in}{0.849820in}}%
\pgfpathclose%
\pgfusepath{fill}%
\end{pgfscope}%
\begin{pgfscope}%
\pgfpathrectangle{\pgfqpoint{0.150000in}{0.150000in}}{\pgfqpoint{2.700000in}{1.950000in}}%
\pgfusepath{clip}%
\pgfsetbuttcap%
\pgfsetroundjoin%
\definecolor{currentfill}{rgb}{0.853845,0.871844,0.897044}%
\pgfsetfillcolor{currentfill}%
\pgfsetlinewidth{0.000000pt}%
\definecolor{currentstroke}{rgb}{0.000000,0.000000,0.000000}%
\pgfsetstrokecolor{currentstroke}%
\pgfsetdash{}{0pt}%
\pgfpathmoveto{\pgfqpoint{1.459351in}{0.852820in}}%
\pgfpathlineto{\pgfqpoint{1.497923in}{0.891077in}}%
\pgfpathlineto{\pgfqpoint{1.459390in}{0.926304in}}%
\pgfpathlineto{\pgfqpoint{1.420829in}{0.888066in}}%
\pgfpathclose%
\pgfusepath{fill}%
\end{pgfscope}%
\begin{pgfscope}%
\pgfpathrectangle{\pgfqpoint{0.150000in}{0.150000in}}{\pgfqpoint{2.700000in}{1.950000in}}%
\pgfusepath{clip}%
\pgfsetbuttcap%
\pgfsetroundjoin%
\definecolor{currentfill}{rgb}{0.922258,0.931832,0.945236}%
\pgfsetfillcolor{currentfill}%
\pgfsetlinewidth{0.000000pt}%
\definecolor{currentstroke}{rgb}{0.000000,0.000000,0.000000}%
\pgfsetstrokecolor{currentstroke}%
\pgfsetdash{}{0pt}%
\pgfpathmoveto{\pgfqpoint{1.305096in}{0.773302in}}%
\pgfpathlineto{\pgfqpoint{1.343682in}{0.811565in}}%
\pgfpathlineto{\pgfqpoint{1.305147in}{0.849820in}}%
\pgfpathlineto{\pgfqpoint{1.266560in}{0.811565in}}%
\pgfpathclose%
\pgfusepath{fill}%
\end{pgfscope}%
\begin{pgfscope}%
\pgfpathrectangle{\pgfqpoint{0.150000in}{0.150000in}}{\pgfqpoint{2.700000in}{1.950000in}}%
\pgfusepath{clip}%
\pgfsetbuttcap%
\pgfsetroundjoin%
\definecolor{currentfill}{rgb}{0.884942,0.899112,0.918949}%
\pgfsetfillcolor{currentfill}%
\pgfsetlinewidth{0.000000pt}%
\definecolor{currentstroke}{rgb}{0.000000,0.000000,0.000000}%
\pgfsetstrokecolor{currentstroke}%
\pgfsetdash{}{0pt}%
\pgfpathmoveto{\pgfqpoint{1.420770in}{0.814554in}}%
\pgfpathlineto{\pgfqpoint{1.459351in}{0.852820in}}%
\pgfpathlineto{\pgfqpoint{1.420829in}{0.888066in}}%
\pgfpathlineto{\pgfqpoint{1.382260in}{0.849820in}}%
\pgfpathclose%
\pgfusepath{fill}%
\end{pgfscope}%
\begin{pgfscope}%
\pgfpathrectangle{\pgfqpoint{0.150000in}{0.150000in}}{\pgfqpoint{2.700000in}{1.950000in}}%
\pgfusepath{clip}%
\pgfsetbuttcap%
\pgfsetroundjoin%
\definecolor{currentfill}{rgb}{0.953355,0.959099,0.967142}%
\pgfsetfillcolor{currentfill}%
\pgfsetlinewidth{0.000000pt}%
\definecolor{currentstroke}{rgb}{0.000000,0.000000,0.000000}%
\pgfsetstrokecolor{currentstroke}%
\pgfsetdash{}{0pt}%
\pgfpathmoveto{\pgfqpoint{1.266501in}{0.735030in}}%
\pgfpathlineto{\pgfqpoint{1.305096in}{0.773302in}}%
\pgfpathlineto{\pgfqpoint{1.266560in}{0.811565in}}%
\pgfpathlineto{\pgfqpoint{1.228056in}{0.770325in}}%
\pgfpathclose%
\pgfusepath{fill}%
\end{pgfscope}%
\begin{pgfscope}%
\pgfpathrectangle{\pgfqpoint{0.150000in}{0.150000in}}{\pgfqpoint{2.700000in}{1.950000in}}%
\pgfusepath{clip}%
\pgfsetbuttcap%
\pgfsetroundjoin%
\definecolor{currentfill}{rgb}{0.853845,0.871844,0.897044}%
\pgfsetfillcolor{currentfill}%
\pgfsetlinewidth{0.000000pt}%
\definecolor{currentstroke}{rgb}{0.000000,0.000000,0.000000}%
\pgfsetstrokecolor{currentstroke}%
\pgfsetdash{}{0pt}%
\pgfpathmoveto{\pgfqpoint{1.536486in}{0.852820in}}%
\pgfpathlineto{\pgfqpoint{1.575050in}{0.891077in}}%
\pgfpathlineto{\pgfqpoint{1.536486in}{0.929326in}}%
\pgfpathlineto{\pgfqpoint{1.497923in}{0.891077in}}%
\pgfpathclose%
\pgfusepath{fill}%
\end{pgfscope}%
\begin{pgfscope}%
\pgfpathrectangle{\pgfqpoint{0.150000in}{0.150000in}}{\pgfqpoint{2.700000in}{1.950000in}}%
\pgfusepath{clip}%
\pgfsetbuttcap%
\pgfsetroundjoin%
\definecolor{currentfill}{rgb}{0.916039,0.926379,0.940855}%
\pgfsetfillcolor{currentfill}%
\pgfsetlinewidth{0.000000pt}%
\definecolor{currentstroke}{rgb}{0.000000,0.000000,0.000000}%
\pgfsetstrokecolor{currentstroke}%
\pgfsetdash{}{0pt}%
\pgfpathmoveto{\pgfqpoint{1.382181in}{0.776280in}}%
\pgfpathlineto{\pgfqpoint{1.420770in}{0.814554in}}%
\pgfpathlineto{\pgfqpoint{1.382260in}{0.849820in}}%
\pgfpathlineto{\pgfqpoint{1.343682in}{0.811565in}}%
\pgfpathclose%
\pgfusepath{fill}%
\end{pgfscope}%
\begin{pgfscope}%
\pgfpathrectangle{\pgfqpoint{0.150000in}{0.150000in}}{\pgfqpoint{2.700000in}{1.950000in}}%
\pgfusepath{clip}%
\pgfsetbuttcap%
\pgfsetroundjoin%
\definecolor{currentfill}{rgb}{0.884942,0.899112,0.918949}%
\pgfsetfillcolor{currentfill}%
\pgfsetlinewidth{0.000000pt}%
\definecolor{currentstroke}{rgb}{0.000000,0.000000,0.000000}%
\pgfsetstrokecolor{currentstroke}%
\pgfsetdash{}{0pt}%
\pgfpathmoveto{\pgfqpoint{1.575059in}{0.814554in}}%
\pgfpathlineto{\pgfqpoint{1.613622in}{0.852820in}}%
\pgfpathlineto{\pgfqpoint{1.575050in}{0.891077in}}%
\pgfpathlineto{\pgfqpoint{1.536486in}{0.852820in}}%
\pgfpathclose%
\pgfusepath{fill}%
\end{pgfscope}%
\begin{pgfscope}%
\pgfpathrectangle{\pgfqpoint{0.150000in}{0.150000in}}{\pgfqpoint{2.700000in}{1.950000in}}%
\pgfusepath{clip}%
\pgfsetbuttcap%
\pgfsetroundjoin%
\definecolor{currentfill}{rgb}{0.884942,0.899112,0.918949}%
\pgfsetfillcolor{currentfill}%
\pgfsetlinewidth{0.000000pt}%
\definecolor{currentstroke}{rgb}{0.000000,0.000000,0.000000}%
\pgfsetstrokecolor{currentstroke}%
\pgfsetdash{}{0pt}%
\pgfpathmoveto{\pgfqpoint{1.497914in}{0.814554in}}%
\pgfpathlineto{\pgfqpoint{1.536486in}{0.852820in}}%
\pgfpathlineto{\pgfqpoint{1.497923in}{0.891077in}}%
\pgfpathlineto{\pgfqpoint{1.459351in}{0.852820in}}%
\pgfpathclose%
\pgfusepath{fill}%
\end{pgfscope}%
\begin{pgfscope}%
\pgfpathrectangle{\pgfqpoint{0.150000in}{0.150000in}}{\pgfqpoint{2.700000in}{1.950000in}}%
\pgfusepath{clip}%
\pgfsetbuttcap%
\pgfsetroundjoin%
\definecolor{currentfill}{rgb}{0.947135,0.953646,0.962760}%
\pgfsetfillcolor{currentfill}%
\pgfsetlinewidth{0.000000pt}%
\definecolor{currentstroke}{rgb}{0.000000,0.000000,0.000000}%
\pgfsetstrokecolor{currentstroke}%
\pgfsetdash{}{0pt}%
\pgfpathmoveto{\pgfqpoint{1.343583in}{0.737998in}}%
\pgfpathlineto{\pgfqpoint{1.382181in}{0.776280in}}%
\pgfpathlineto{\pgfqpoint{1.343682in}{0.811565in}}%
\pgfpathlineto{\pgfqpoint{1.305096in}{0.773302in}}%
\pgfpathclose%
\pgfusepath{fill}%
\end{pgfscope}%
\begin{pgfscope}%
\pgfpathrectangle{\pgfqpoint{0.150000in}{0.150000in}}{\pgfqpoint{2.700000in}{1.950000in}}%
\pgfusepath{clip}%
\pgfsetbuttcap%
\pgfsetroundjoin%
\definecolor{currentfill}{rgb}{0.916039,0.926379,0.940855}%
\pgfsetfillcolor{currentfill}%
\pgfsetlinewidth{0.000000pt}%
\definecolor{currentstroke}{rgb}{0.000000,0.000000,0.000000}%
\pgfsetstrokecolor{currentstroke}%
\pgfsetdash{}{0pt}%
\pgfpathmoveto{\pgfqpoint{1.613639in}{0.776280in}}%
\pgfpathlineto{\pgfqpoint{1.652203in}{0.814554in}}%
\pgfpathlineto{\pgfqpoint{1.613622in}{0.852820in}}%
\pgfpathlineto{\pgfqpoint{1.575059in}{0.814554in}}%
\pgfpathclose%
\pgfusepath{fill}%
\end{pgfscope}%
\begin{pgfscope}%
\pgfpathrectangle{\pgfqpoint{0.150000in}{0.150000in}}{\pgfqpoint{2.700000in}{1.950000in}}%
\pgfusepath{clip}%
\pgfsetbuttcap%
\pgfsetroundjoin%
\definecolor{currentfill}{rgb}{0.916039,0.926379,0.940855}%
\pgfsetfillcolor{currentfill}%
\pgfsetlinewidth{0.000000pt}%
\definecolor{currentstroke}{rgb}{0.000000,0.000000,0.000000}%
\pgfsetstrokecolor{currentstroke}%
\pgfsetdash{}{0pt}%
\pgfpathmoveto{\pgfqpoint{1.536486in}{0.776280in}}%
\pgfpathlineto{\pgfqpoint{1.575059in}{0.814554in}}%
\pgfpathlineto{\pgfqpoint{1.536486in}{0.852820in}}%
\pgfpathlineto{\pgfqpoint{1.497914in}{0.814554in}}%
\pgfpathclose%
\pgfusepath{fill}%
\end{pgfscope}%
\begin{pgfscope}%
\pgfpathrectangle{\pgfqpoint{0.150000in}{0.150000in}}{\pgfqpoint{2.700000in}{1.950000in}}%
\pgfusepath{clip}%
\pgfsetbuttcap%
\pgfsetroundjoin%
\definecolor{currentfill}{rgb}{0.916039,0.926379,0.940855}%
\pgfsetfillcolor{currentfill}%
\pgfsetlinewidth{0.000000pt}%
\definecolor{currentstroke}{rgb}{0.000000,0.000000,0.000000}%
\pgfsetstrokecolor{currentstroke}%
\pgfsetdash{}{0pt}%
\pgfpathmoveto{\pgfqpoint{1.459334in}{0.776280in}}%
\pgfpathlineto{\pgfqpoint{1.497914in}{0.814554in}}%
\pgfpathlineto{\pgfqpoint{1.459351in}{0.852820in}}%
\pgfpathlineto{\pgfqpoint{1.420770in}{0.814554in}}%
\pgfpathclose%
\pgfusepath{fill}%
\end{pgfscope}%
\begin{pgfscope}%
\pgfpathrectangle{\pgfqpoint{0.150000in}{0.150000in}}{\pgfqpoint{2.700000in}{1.950000in}}%
\pgfusepath{clip}%
\pgfsetbuttcap%
\pgfsetroundjoin%
\definecolor{currentfill}{rgb}{0.978232,0.980913,0.984666}%
\pgfsetfillcolor{currentfill}%
\pgfsetlinewidth{0.000000pt}%
\definecolor{currentstroke}{rgb}{0.000000,0.000000,0.000000}%
\pgfsetstrokecolor{currentstroke}%
\pgfsetdash{}{0pt}%
\pgfpathmoveto{\pgfqpoint{1.304977in}{0.699707in}}%
\pgfpathlineto{\pgfqpoint{1.343583in}{0.737998in}}%
\pgfpathlineto{\pgfqpoint{1.305096in}{0.773302in}}%
\pgfpathlineto{\pgfqpoint{1.266501in}{0.735030in}}%
\pgfpathclose%
\pgfusepath{fill}%
\end{pgfscope}%
\begin{pgfscope}%
\pgfpathrectangle{\pgfqpoint{0.150000in}{0.150000in}}{\pgfqpoint{2.700000in}{1.950000in}}%
\pgfusepath{clip}%
\pgfsetbuttcap%
\pgfsetroundjoin%
\definecolor{currentfill}{rgb}{0.947135,0.953646,0.962760}%
\pgfsetfillcolor{currentfill}%
\pgfsetlinewidth{0.000000pt}%
\definecolor{currentstroke}{rgb}{0.000000,0.000000,0.000000}%
\pgfsetstrokecolor{currentstroke}%
\pgfsetdash{}{0pt}%
\pgfpathmoveto{\pgfqpoint{1.652229in}{0.737998in}}%
\pgfpathlineto{\pgfqpoint{1.690792in}{0.776280in}}%
\pgfpathlineto{\pgfqpoint{1.652203in}{0.814554in}}%
\pgfpathlineto{\pgfqpoint{1.613639in}{0.776280in}}%
\pgfpathclose%
\pgfusepath{fill}%
\end{pgfscope}%
\begin{pgfscope}%
\pgfpathrectangle{\pgfqpoint{0.150000in}{0.150000in}}{\pgfqpoint{2.700000in}{1.950000in}}%
\pgfusepath{clip}%
\pgfsetbuttcap%
\pgfsetroundjoin%
\definecolor{currentfill}{rgb}{0.947135,0.953646,0.962760}%
\pgfsetfillcolor{currentfill}%
\pgfsetlinewidth{0.000000pt}%
\definecolor{currentstroke}{rgb}{0.000000,0.000000,0.000000}%
\pgfsetstrokecolor{currentstroke}%
\pgfsetdash{}{0pt}%
\pgfpathmoveto{\pgfqpoint{1.575067in}{0.737998in}}%
\pgfpathlineto{\pgfqpoint{1.613639in}{0.776280in}}%
\pgfpathlineto{\pgfqpoint{1.575059in}{0.814554in}}%
\pgfpathlineto{\pgfqpoint{1.536486in}{0.776280in}}%
\pgfpathclose%
\pgfusepath{fill}%
\end{pgfscope}%
\begin{pgfscope}%
\pgfpathrectangle{\pgfqpoint{0.150000in}{0.150000in}}{\pgfqpoint{2.700000in}{1.950000in}}%
\pgfusepath{clip}%
\pgfsetbuttcap%
\pgfsetroundjoin%
\definecolor{currentfill}{rgb}{0.947135,0.953646,0.962760}%
\pgfsetfillcolor{currentfill}%
\pgfsetlinewidth{0.000000pt}%
\definecolor{currentstroke}{rgb}{0.000000,0.000000,0.000000}%
\pgfsetstrokecolor{currentstroke}%
\pgfsetdash{}{0pt}%
\pgfpathmoveto{\pgfqpoint{1.497906in}{0.737998in}}%
\pgfpathlineto{\pgfqpoint{1.536486in}{0.776280in}}%
\pgfpathlineto{\pgfqpoint{1.497914in}{0.814554in}}%
\pgfpathlineto{\pgfqpoint{1.459334in}{0.776280in}}%
\pgfpathclose%
\pgfusepath{fill}%
\end{pgfscope}%
\begin{pgfscope}%
\pgfpathrectangle{\pgfqpoint{0.150000in}{0.150000in}}{\pgfqpoint{2.700000in}{1.950000in}}%
\pgfusepath{clip}%
\pgfsetbuttcap%
\pgfsetroundjoin%
\definecolor{currentfill}{rgb}{0.947135,0.953646,0.962760}%
\pgfsetfillcolor{currentfill}%
\pgfsetlinewidth{0.000000pt}%
\definecolor{currentstroke}{rgb}{0.000000,0.000000,0.000000}%
\pgfsetstrokecolor{currentstroke}%
\pgfsetdash{}{0pt}%
\pgfpathmoveto{\pgfqpoint{1.420744in}{0.737998in}}%
\pgfpathlineto{\pgfqpoint{1.459334in}{0.776280in}}%
\pgfpathlineto{\pgfqpoint{1.420770in}{0.814554in}}%
\pgfpathlineto{\pgfqpoint{1.382181in}{0.776280in}}%
\pgfpathclose%
\pgfusepath{fill}%
\end{pgfscope}%
\begin{pgfscope}%
\pgfpathrectangle{\pgfqpoint{0.150000in}{0.150000in}}{\pgfqpoint{2.700000in}{1.950000in}}%
\pgfusepath{clip}%
\pgfsetbuttcap%
\pgfsetroundjoin%
\definecolor{currentfill}{rgb}{0.972013,0.975460,0.980285}%
\pgfsetfillcolor{currentfill}%
\pgfsetlinewidth{0.000000pt}%
\definecolor{currentstroke}{rgb}{0.000000,0.000000,0.000000}%
\pgfsetstrokecolor{currentstroke}%
\pgfsetdash{}{0pt}%
\pgfpathmoveto{\pgfqpoint{1.690826in}{0.699707in}}%
\pgfpathlineto{\pgfqpoint{1.729390in}{0.737998in}}%
\pgfpathlineto{\pgfqpoint{1.690792in}{0.776280in}}%
\pgfpathlineto{\pgfqpoint{1.652229in}{0.737998in}}%
\pgfpathclose%
\pgfusepath{fill}%
\end{pgfscope}%
\begin{pgfscope}%
\pgfpathrectangle{\pgfqpoint{0.150000in}{0.150000in}}{\pgfqpoint{2.700000in}{1.950000in}}%
\pgfusepath{clip}%
\pgfsetbuttcap%
\pgfsetroundjoin%
\definecolor{currentfill}{rgb}{0.972013,0.975460,0.980285}%
\pgfsetfillcolor{currentfill}%
\pgfsetlinewidth{0.000000pt}%
\definecolor{currentstroke}{rgb}{0.000000,0.000000,0.000000}%
\pgfsetstrokecolor{currentstroke}%
\pgfsetdash{}{0pt}%
\pgfpathmoveto{\pgfqpoint{1.382147in}{0.699707in}}%
\pgfpathlineto{\pgfqpoint{1.420744in}{0.737998in}}%
\pgfpathlineto{\pgfqpoint{1.382181in}{0.776280in}}%
\pgfpathlineto{\pgfqpoint{1.343583in}{0.737998in}}%
\pgfpathclose%
\pgfusepath{fill}%
\end{pgfscope}%
\begin{pgfscope}%
\pgfpathrectangle{\pgfqpoint{0.150000in}{0.150000in}}{\pgfqpoint{2.700000in}{1.950000in}}%
\pgfusepath{clip}%
\pgfsetbuttcap%
\pgfsetroundjoin%
\definecolor{currentfill}{rgb}{0.972013,0.975460,0.980285}%
\pgfsetfillcolor{currentfill}%
\pgfsetlinewidth{0.000000pt}%
\definecolor{currentstroke}{rgb}{0.000000,0.000000,0.000000}%
\pgfsetstrokecolor{currentstroke}%
\pgfsetdash{}{0pt}%
\pgfpathmoveto{\pgfqpoint{1.613679in}{0.702665in}}%
\pgfpathlineto{\pgfqpoint{1.652229in}{0.737998in}}%
\pgfpathlineto{\pgfqpoint{1.613639in}{0.776280in}}%
\pgfpathlineto{\pgfqpoint{1.575067in}{0.737998in}}%
\pgfpathclose%
\pgfusepath{fill}%
\end{pgfscope}%
\begin{pgfscope}%
\pgfpathrectangle{\pgfqpoint{0.150000in}{0.150000in}}{\pgfqpoint{2.700000in}{1.950000in}}%
\pgfusepath{clip}%
\pgfsetbuttcap%
\pgfsetroundjoin%
\definecolor{currentfill}{rgb}{0.972013,0.975460,0.980285}%
\pgfsetfillcolor{currentfill}%
\pgfsetlinewidth{0.000000pt}%
\definecolor{currentstroke}{rgb}{0.000000,0.000000,0.000000}%
\pgfsetstrokecolor{currentstroke}%
\pgfsetdash{}{0pt}%
\pgfpathmoveto{\pgfqpoint{1.536486in}{0.702665in}}%
\pgfpathlineto{\pgfqpoint{1.575067in}{0.737998in}}%
\pgfpathlineto{\pgfqpoint{1.536486in}{0.776280in}}%
\pgfpathlineto{\pgfqpoint{1.497906in}{0.737998in}}%
\pgfpathclose%
\pgfusepath{fill}%
\end{pgfscope}%
\begin{pgfscope}%
\pgfpathrectangle{\pgfqpoint{0.150000in}{0.150000in}}{\pgfqpoint{2.700000in}{1.950000in}}%
\pgfusepath{clip}%
\pgfsetbuttcap%
\pgfsetroundjoin%
\definecolor{currentfill}{rgb}{0.972013,0.975460,0.980285}%
\pgfsetfillcolor{currentfill}%
\pgfsetlinewidth{0.000000pt}%
\definecolor{currentstroke}{rgb}{0.000000,0.000000,0.000000}%
\pgfsetstrokecolor{currentstroke}%
\pgfsetdash{}{0pt}%
\pgfpathmoveto{\pgfqpoint{1.459294in}{0.702665in}}%
\pgfpathlineto{\pgfqpoint{1.497906in}{0.737998in}}%
\pgfpathlineto{\pgfqpoint{1.459334in}{0.776280in}}%
\pgfpathlineto{\pgfqpoint{1.420744in}{0.737998in}}%
\pgfpathclose%
\pgfusepath{fill}%
\end{pgfscope}%
\begin{pgfscope}%
\pgfpathrectangle{\pgfqpoint{0.150000in}{0.150000in}}{\pgfqpoint{2.700000in}{1.950000in}}%
\pgfusepath{clip}%
\pgfsetbuttcap%
\pgfsetroundjoin%
\definecolor{currentfill}{rgb}{0.998100,0.996553,0.996676}%
\pgfsetfillcolor{currentfill}%
\pgfsetlinewidth{0.000000pt}%
\definecolor{currentstroke}{rgb}{0.000000,0.000000,0.000000}%
\pgfsetstrokecolor{currentstroke}%
\pgfsetdash{}{0pt}%
\pgfpathmoveto{\pgfqpoint{1.729433in}{0.661407in}}%
\pgfpathlineto{\pgfqpoint{1.767996in}{0.699707in}}%
\pgfpathlineto{\pgfqpoint{1.729390in}{0.737998in}}%
\pgfpathlineto{\pgfqpoint{1.690826in}{0.699707in}}%
\pgfpathclose%
\pgfusepath{fill}%
\end{pgfscope}%
\begin{pgfscope}%
\pgfpathrectangle{\pgfqpoint{0.150000in}{0.150000in}}{\pgfqpoint{2.700000in}{1.950000in}}%
\pgfusepath{clip}%
\pgfsetbuttcap%
\pgfsetroundjoin%
\definecolor{currentfill}{rgb}{0.998100,0.996553,0.996676}%
\pgfsetfillcolor{currentfill}%
\pgfsetlinewidth{0.000000pt}%
\definecolor{currentstroke}{rgb}{0.000000,0.000000,0.000000}%
\pgfsetstrokecolor{currentstroke}%
\pgfsetdash{}{0pt}%
\pgfpathmoveto{\pgfqpoint{1.343540in}{0.661407in}}%
\pgfpathlineto{\pgfqpoint{1.382147in}{0.699707in}}%
\pgfpathlineto{\pgfqpoint{1.343583in}{0.737998in}}%
\pgfpathlineto{\pgfqpoint{1.304977in}{0.699707in}}%
\pgfpathclose%
\pgfusepath{fill}%
\end{pgfscope}%
\begin{pgfscope}%
\pgfpathrectangle{\pgfqpoint{0.150000in}{0.150000in}}{\pgfqpoint{2.700000in}{1.950000in}}%
\pgfusepath{clip}%
\pgfsetbuttcap%
\pgfsetroundjoin%
\definecolor{currentfill}{rgb}{0.998100,0.996553,0.996676}%
\pgfsetfillcolor{currentfill}%
\pgfsetlinewidth{0.000000pt}%
\definecolor{currentstroke}{rgb}{0.000000,0.000000,0.000000}%
\pgfsetstrokecolor{currentstroke}%
\pgfsetdash{}{0pt}%
\pgfpathmoveto{\pgfqpoint{1.652288in}{0.664355in}}%
\pgfpathlineto{\pgfqpoint{1.690826in}{0.699707in}}%
\pgfpathlineto{\pgfqpoint{1.652229in}{0.737998in}}%
\pgfpathlineto{\pgfqpoint{1.613679in}{0.702665in}}%
\pgfpathclose%
\pgfusepath{fill}%
\end{pgfscope}%
\begin{pgfscope}%
\pgfpathrectangle{\pgfqpoint{0.150000in}{0.150000in}}{\pgfqpoint{2.700000in}{1.950000in}}%
\pgfusepath{clip}%
\pgfsetbuttcap%
\pgfsetroundjoin%
\definecolor{currentfill}{rgb}{0.998100,0.996553,0.996676}%
\pgfsetfillcolor{currentfill}%
\pgfsetlinewidth{0.000000pt}%
\definecolor{currentstroke}{rgb}{0.000000,0.000000,0.000000}%
\pgfsetstrokecolor{currentstroke}%
\pgfsetdash{}{0pt}%
\pgfpathmoveto{\pgfqpoint{1.420685in}{0.664355in}}%
\pgfpathlineto{\pgfqpoint{1.459294in}{0.702665in}}%
\pgfpathlineto{\pgfqpoint{1.420744in}{0.737998in}}%
\pgfpathlineto{\pgfqpoint{1.382147in}{0.699707in}}%
\pgfpathclose%
\pgfusepath{fill}%
\end{pgfscope}%
\begin{pgfscope}%
\pgfpathrectangle{\pgfqpoint{0.150000in}{0.150000in}}{\pgfqpoint{2.700000in}{1.950000in}}%
\pgfusepath{clip}%
\pgfsetbuttcap%
\pgfsetroundjoin%
\definecolor{currentfill}{rgb}{0.998100,0.996553,0.996676}%
\pgfsetfillcolor{currentfill}%
\pgfsetlinewidth{0.000000pt}%
\definecolor{currentstroke}{rgb}{0.000000,0.000000,0.000000}%
\pgfsetstrokecolor{currentstroke}%
\pgfsetdash{}{0pt}%
\pgfpathmoveto{\pgfqpoint{1.575087in}{0.664355in}}%
\pgfpathlineto{\pgfqpoint{1.613679in}{0.702665in}}%
\pgfpathlineto{\pgfqpoint{1.575067in}{0.737998in}}%
\pgfpathlineto{\pgfqpoint{1.536486in}{0.702665in}}%
\pgfpathclose%
\pgfusepath{fill}%
\end{pgfscope}%
\begin{pgfscope}%
\pgfpathrectangle{\pgfqpoint{0.150000in}{0.150000in}}{\pgfqpoint{2.700000in}{1.950000in}}%
\pgfusepath{clip}%
\pgfsetbuttcap%
\pgfsetroundjoin%
\definecolor{currentfill}{rgb}{0.998100,0.996553,0.996676}%
\pgfsetfillcolor{currentfill}%
\pgfsetlinewidth{0.000000pt}%
\definecolor{currentstroke}{rgb}{0.000000,0.000000,0.000000}%
\pgfsetstrokecolor{currentstroke}%
\pgfsetdash{}{0pt}%
\pgfpathmoveto{\pgfqpoint{1.497886in}{0.664355in}}%
\pgfpathlineto{\pgfqpoint{1.536486in}{0.702665in}}%
\pgfpathlineto{\pgfqpoint{1.497906in}{0.737998in}}%
\pgfpathlineto{\pgfqpoint{1.459294in}{0.702665in}}%
\pgfpathclose%
\pgfusepath{fill}%
\end{pgfscope}%
\begin{pgfscope}%
\pgfpathrectangle{\pgfqpoint{0.150000in}{0.150000in}}{\pgfqpoint{2.700000in}{1.950000in}}%
\pgfusepath{clip}%
\pgfsetbuttcap%
\pgfsetroundjoin%
\definecolor{currentfill}{rgb}{0.979105,0.962086,0.963434}%
\pgfsetfillcolor{currentfill}%
\pgfsetlinewidth{0.000000pt}%
\definecolor{currentstroke}{rgb}{0.000000,0.000000,0.000000}%
\pgfsetstrokecolor{currentstroke}%
\pgfsetdash{}{0pt}%
\pgfpathmoveto{\pgfqpoint{1.690906in}{0.626036in}}%
\pgfpathlineto{\pgfqpoint{1.729433in}{0.661407in}}%
\pgfpathlineto{\pgfqpoint{1.690826in}{0.699707in}}%
\pgfpathlineto{\pgfqpoint{1.652288in}{0.664355in}}%
\pgfpathclose%
\pgfusepath{fill}%
\end{pgfscope}%
\begin{pgfscope}%
\pgfpathrectangle{\pgfqpoint{0.150000in}{0.150000in}}{\pgfqpoint{2.700000in}{1.950000in}}%
\pgfusepath{clip}%
\pgfsetbuttcap%
\pgfsetroundjoin%
\definecolor{currentfill}{rgb}{0.979105,0.962086,0.963434}%
\pgfsetfillcolor{currentfill}%
\pgfsetlinewidth{0.000000pt}%
\definecolor{currentstroke}{rgb}{0.000000,0.000000,0.000000}%
\pgfsetstrokecolor{currentstroke}%
\pgfsetdash{}{0pt}%
\pgfpathmoveto{\pgfqpoint{1.382067in}{0.626036in}}%
\pgfpathlineto{\pgfqpoint{1.420685in}{0.664355in}}%
\pgfpathlineto{\pgfqpoint{1.382147in}{0.699707in}}%
\pgfpathlineto{\pgfqpoint{1.343540in}{0.661407in}}%
\pgfpathclose%
\pgfusepath{fill}%
\end{pgfscope}%
\begin{pgfscope}%
\pgfpathrectangle{\pgfqpoint{0.150000in}{0.150000in}}{\pgfqpoint{2.700000in}{1.950000in}}%
\pgfusepath{clip}%
\pgfsetbuttcap%
\pgfsetroundjoin%
\definecolor{currentfill}{rgb}{0.982904,0.968980,0.970083}%
\pgfsetfillcolor{currentfill}%
\pgfsetlinewidth{0.000000pt}%
\definecolor{currentstroke}{rgb}{0.000000,0.000000,0.000000}%
\pgfsetstrokecolor{currentstroke}%
\pgfsetdash{}{0pt}%
\pgfpathmoveto{\pgfqpoint{1.613696in}{0.626036in}}%
\pgfpathlineto{\pgfqpoint{1.652288in}{0.664355in}}%
\pgfpathlineto{\pgfqpoint{1.613679in}{0.702665in}}%
\pgfpathlineto{\pgfqpoint{1.575087in}{0.664355in}}%
\pgfpathclose%
\pgfusepath{fill}%
\end{pgfscope}%
\begin{pgfscope}%
\pgfpathrectangle{\pgfqpoint{0.150000in}{0.150000in}}{\pgfqpoint{2.700000in}{1.950000in}}%
\pgfusepath{clip}%
\pgfsetbuttcap%
\pgfsetroundjoin%
\definecolor{currentfill}{rgb}{0.982904,0.968980,0.970083}%
\pgfsetfillcolor{currentfill}%
\pgfsetlinewidth{0.000000pt}%
\definecolor{currentstroke}{rgb}{0.000000,0.000000,0.000000}%
\pgfsetstrokecolor{currentstroke}%
\pgfsetdash{}{0pt}%
\pgfpathmoveto{\pgfqpoint{1.536486in}{0.626036in}}%
\pgfpathlineto{\pgfqpoint{1.575087in}{0.664355in}}%
\pgfpathlineto{\pgfqpoint{1.536486in}{0.702665in}}%
\pgfpathlineto{\pgfqpoint{1.497886in}{0.664355in}}%
\pgfpathclose%
\pgfusepath{fill}%
\end{pgfscope}%
\begin{pgfscope}%
\pgfpathrectangle{\pgfqpoint{0.150000in}{0.150000in}}{\pgfqpoint{2.700000in}{1.950000in}}%
\pgfusepath{clip}%
\pgfsetbuttcap%
\pgfsetroundjoin%
\definecolor{currentfill}{rgb}{0.982904,0.968980,0.970083}%
\pgfsetfillcolor{currentfill}%
\pgfsetlinewidth{0.000000pt}%
\definecolor{currentstroke}{rgb}{0.000000,0.000000,0.000000}%
\pgfsetstrokecolor{currentstroke}%
\pgfsetdash{}{0pt}%
\pgfpathmoveto{\pgfqpoint{1.459277in}{0.626036in}}%
\pgfpathlineto{\pgfqpoint{1.497886in}{0.664355in}}%
\pgfpathlineto{\pgfqpoint{1.459294in}{0.702665in}}%
\pgfpathlineto{\pgfqpoint{1.420685in}{0.664355in}}%
\pgfpathclose%
\pgfusepath{fill}%
\end{pgfscope}%
\begin{pgfscope}%
\pgfpathrectangle{\pgfqpoint{0.150000in}{0.150000in}}{\pgfqpoint{2.700000in}{1.950000in}}%
\pgfusepath{clip}%
\pgfsetbuttcap%
\pgfsetroundjoin%
\definecolor{currentfill}{rgb}{0.963909,0.934513,0.936841}%
\pgfsetfillcolor{currentfill}%
\pgfsetlinewidth{0.000000pt}%
\definecolor{currentstroke}{rgb}{0.000000,0.000000,0.000000}%
\pgfsetstrokecolor{currentstroke}%
\pgfsetdash{}{0pt}%
\pgfpathmoveto{\pgfqpoint{1.652348in}{0.590636in}}%
\pgfpathlineto{\pgfqpoint{1.690906in}{0.626036in}}%
\pgfpathlineto{\pgfqpoint{1.652288in}{0.664355in}}%
\pgfpathlineto{\pgfqpoint{1.613696in}{0.626036in}}%
\pgfpathclose%
\pgfusepath{fill}%
\end{pgfscope}%
\begin{pgfscope}%
\pgfpathrectangle{\pgfqpoint{0.150000in}{0.150000in}}{\pgfqpoint{2.700000in}{1.950000in}}%
\pgfusepath{clip}%
\pgfsetbuttcap%
\pgfsetroundjoin%
\definecolor{currentfill}{rgb}{0.963909,0.934513,0.936841}%
\pgfsetfillcolor{currentfill}%
\pgfsetlinewidth{0.000000pt}%
\definecolor{currentstroke}{rgb}{0.000000,0.000000,0.000000}%
\pgfsetstrokecolor{currentstroke}%
\pgfsetdash{}{0pt}%
\pgfpathmoveto{\pgfqpoint{1.575107in}{0.590636in}}%
\pgfpathlineto{\pgfqpoint{1.613696in}{0.626036in}}%
\pgfpathlineto{\pgfqpoint{1.575087in}{0.664355in}}%
\pgfpathlineto{\pgfqpoint{1.536486in}{0.626036in}}%
\pgfpathclose%
\pgfusepath{fill}%
\end{pgfscope}%
\begin{pgfscope}%
\pgfpathrectangle{\pgfqpoint{0.150000in}{0.150000in}}{\pgfqpoint{2.700000in}{1.950000in}}%
\pgfusepath{clip}%
\pgfsetbuttcap%
\pgfsetroundjoin%
\definecolor{currentfill}{rgb}{0.963909,0.934513,0.936841}%
\pgfsetfillcolor{currentfill}%
\pgfsetlinewidth{0.000000pt}%
\definecolor{currentstroke}{rgb}{0.000000,0.000000,0.000000}%
\pgfsetstrokecolor{currentstroke}%
\pgfsetdash{}{0pt}%
\pgfpathmoveto{\pgfqpoint{1.497866in}{0.590636in}}%
\pgfpathlineto{\pgfqpoint{1.536486in}{0.626036in}}%
\pgfpathlineto{\pgfqpoint{1.497886in}{0.664355in}}%
\pgfpathlineto{\pgfqpoint{1.459277in}{0.626036in}}%
\pgfpathclose%
\pgfusepath{fill}%
\end{pgfscope}%
\begin{pgfscope}%
\pgfpathrectangle{\pgfqpoint{0.150000in}{0.150000in}}{\pgfqpoint{2.700000in}{1.950000in}}%
\pgfusepath{clip}%
\pgfsetbuttcap%
\pgfsetroundjoin%
\definecolor{currentfill}{rgb}{0.963909,0.934513,0.936841}%
\pgfsetfillcolor{currentfill}%
\pgfsetlinewidth{0.000000pt}%
\definecolor{currentstroke}{rgb}{0.000000,0.000000,0.000000}%
\pgfsetstrokecolor{currentstroke}%
\pgfsetdash{}{0pt}%
\pgfpathmoveto{\pgfqpoint{1.420625in}{0.590636in}}%
\pgfpathlineto{\pgfqpoint{1.459277in}{0.626036in}}%
\pgfpathlineto{\pgfqpoint{1.420685in}{0.664355in}}%
\pgfpathlineto{\pgfqpoint{1.382067in}{0.626036in}}%
\pgfpathclose%
\pgfusepath{fill}%
\end{pgfscope}%
\begin{pgfscope}%
\pgfpathrectangle{\pgfqpoint{0.150000in}{0.150000in}}{\pgfqpoint{2.700000in}{1.950000in}}%
\pgfusepath{clip}%
\pgfsetbuttcap%
\pgfsetroundjoin%
\definecolor{currentfill}{rgb}{0.948713,0.906939,0.910248}%
\pgfsetfillcolor{currentfill}%
\pgfsetlinewidth{0.000000pt}%
\definecolor{currentstroke}{rgb}{0.000000,0.000000,0.000000}%
\pgfsetstrokecolor{currentstroke}%
\pgfsetdash{}{0pt}%
\pgfpathmoveto{\pgfqpoint{1.613736in}{0.552290in}}%
\pgfpathlineto{\pgfqpoint{1.652348in}{0.590636in}}%
\pgfpathlineto{\pgfqpoint{1.613696in}{0.626036in}}%
\pgfpathlineto{\pgfqpoint{1.575107in}{0.590636in}}%
\pgfpathclose%
\pgfusepath{fill}%
\end{pgfscope}%
\begin{pgfscope}%
\pgfpathrectangle{\pgfqpoint{0.150000in}{0.150000in}}{\pgfqpoint{2.700000in}{1.950000in}}%
\pgfusepath{clip}%
\pgfsetbuttcap%
\pgfsetroundjoin%
\definecolor{currentfill}{rgb}{0.948713,0.906939,0.910248}%
\pgfsetfillcolor{currentfill}%
\pgfsetlinewidth{0.000000pt}%
\definecolor{currentstroke}{rgb}{0.000000,0.000000,0.000000}%
\pgfsetstrokecolor{currentstroke}%
\pgfsetdash{}{0pt}%
\pgfpathmoveto{\pgfqpoint{1.536486in}{0.552290in}}%
\pgfpathlineto{\pgfqpoint{1.575107in}{0.590636in}}%
\pgfpathlineto{\pgfqpoint{1.536486in}{0.626036in}}%
\pgfpathlineto{\pgfqpoint{1.497866in}{0.590636in}}%
\pgfpathclose%
\pgfusepath{fill}%
\end{pgfscope}%
\begin{pgfscope}%
\pgfpathrectangle{\pgfqpoint{0.150000in}{0.150000in}}{\pgfqpoint{2.700000in}{1.950000in}}%
\pgfusepath{clip}%
\pgfsetbuttcap%
\pgfsetroundjoin%
\definecolor{currentfill}{rgb}{0.948713,0.906939,0.910248}%
\pgfsetfillcolor{currentfill}%
\pgfsetlinewidth{0.000000pt}%
\definecolor{currentstroke}{rgb}{0.000000,0.000000,0.000000}%
\pgfsetstrokecolor{currentstroke}%
\pgfsetdash{}{0pt}%
\pgfpathmoveto{\pgfqpoint{1.459237in}{0.552290in}}%
\pgfpathlineto{\pgfqpoint{1.497866in}{0.590636in}}%
\pgfpathlineto{\pgfqpoint{1.459277in}{0.626036in}}%
\pgfpathlineto{\pgfqpoint{1.420625in}{0.590636in}}%
\pgfpathclose%
\pgfusepath{fill}%
\end{pgfscope}%
\begin{pgfscope}%
\pgfpathrectangle{\pgfqpoint{0.150000in}{0.150000in}}{\pgfqpoint{2.700000in}{1.950000in}}%
\pgfusepath{clip}%
\pgfsetbuttcap%
\pgfsetroundjoin%
\definecolor{currentfill}{rgb}{0.929718,0.872472,0.877007}%
\pgfsetfillcolor{currentfill}%
\pgfsetlinewidth{0.000000pt}%
\definecolor{currentstroke}{rgb}{0.000000,0.000000,0.000000}%
\pgfsetstrokecolor{currentstroke}%
\pgfsetdash{}{0pt}%
\pgfpathmoveto{\pgfqpoint{1.575127in}{0.516842in}}%
\pgfpathlineto{\pgfqpoint{1.613736in}{0.552290in}}%
\pgfpathlineto{\pgfqpoint{1.575107in}{0.590636in}}%
\pgfpathlineto{\pgfqpoint{1.536486in}{0.552290in}}%
\pgfpathclose%
\pgfusepath{fill}%
\end{pgfscope}%
\begin{pgfscope}%
\pgfpathrectangle{\pgfqpoint{0.150000in}{0.150000in}}{\pgfqpoint{2.700000in}{1.950000in}}%
\pgfusepath{clip}%
\pgfsetbuttcap%
\pgfsetroundjoin%
\definecolor{currentfill}{rgb}{0.929718,0.872472,0.877007}%
\pgfsetfillcolor{currentfill}%
\pgfsetlinewidth{0.000000pt}%
\definecolor{currentstroke}{rgb}{0.000000,0.000000,0.000000}%
\pgfsetstrokecolor{currentstroke}%
\pgfsetdash{}{0pt}%
\pgfpathmoveto{\pgfqpoint{1.497846in}{0.516842in}}%
\pgfpathlineto{\pgfqpoint{1.536486in}{0.552290in}}%
\pgfpathlineto{\pgfqpoint{1.497866in}{0.590636in}}%
\pgfpathlineto{\pgfqpoint{1.459237in}{0.552290in}}%
\pgfpathclose%
\pgfusepath{fill}%
\end{pgfscope}%
\begin{pgfscope}%
\pgfpathrectangle{\pgfqpoint{0.150000in}{0.150000in}}{\pgfqpoint{2.700000in}{1.950000in}}%
\pgfusepath{clip}%
\pgfsetbuttcap%
\pgfsetroundjoin%
\definecolor{currentfill}{rgb}{0.914522,0.844899,0.850414}%
\pgfsetfillcolor{currentfill}%
\pgfsetlinewidth{0.000000pt}%
\definecolor{currentstroke}{rgb}{0.000000,0.000000,0.000000}%
\pgfsetstrokecolor{currentstroke}%
\pgfsetdash{}{0pt}%
\pgfpathmoveto{\pgfqpoint{1.536486in}{0.481365in}}%
\pgfpathlineto{\pgfqpoint{1.575127in}{0.516842in}}%
\pgfpathlineto{\pgfqpoint{1.536486in}{0.552290in}}%
\pgfpathlineto{\pgfqpoint{1.497846in}{0.516842in}}%
\pgfpathclose%
\pgfusepath{fill}%
\end{pgfscope}%
\end{pgfpicture}%
\makeatother%
\endgroup%
}

            \subbottom[\label{fig:parameterised-incompetent-games-c}]%
                {%% Creator: Matplotlib, PGF backend
%%
%% To include the figure in your LaTeX document, write
%%   \input{<filename>.pgf}
%%
%% Make sure the required packages are loaded in your preamble
%%   \usepackage{pgf}
%%
%% Figures using additional raster images can only be included by \input if
%% they are in the same directory as the main LaTeX file. For loading figures
%% from other directories you can use the `import` package
%%   \usepackage{import}
%% and then include the figures with
%%   \import{<path to file>}{<filename>.pgf}
%%
%% Matplotlib used the following preamble
%%   \usepackage{fontspec}
%%   \setmainfont{DejaVuSerif.ttf}[Path=C:/Users/Thomas/anaconda3/lib/site-packages/matplotlib/mpl-data/fonts/ttf/]
%%   \setsansfont{DejaVuSans.ttf}[Path=C:/Users/Thomas/anaconda3/lib/site-packages/matplotlib/mpl-data/fonts/ttf/]
%%   \setmonofont{DejaVuSansMono.ttf}[Path=C:/Users/Thomas/anaconda3/lib/site-packages/matplotlib/mpl-data/fonts/ttf/]
%%
\begingroup%
\makeatletter%
\begin{pgfpicture}%
\pgfpathrectangle{\pgfpointorigin}{\pgfqpoint{3.000000in}{2.250000in}}%
\pgfusepath{use as bounding box, clip}%
\begin{pgfscope}%
\pgfsetbuttcap%
\pgfsetmiterjoin%
\definecolor{currentfill}{rgb}{1.000000,1.000000,1.000000}%
\pgfsetfillcolor{currentfill}%
\pgfsetlinewidth{0.000000pt}%
\definecolor{currentstroke}{rgb}{1.000000,1.000000,1.000000}%
\pgfsetstrokecolor{currentstroke}%
\pgfsetdash{}{0pt}%
\pgfpathmoveto{\pgfqpoint{0.000000in}{0.000000in}}%
\pgfpathlineto{\pgfqpoint{3.000000in}{0.000000in}}%
\pgfpathlineto{\pgfqpoint{3.000000in}{2.250000in}}%
\pgfpathlineto{\pgfqpoint{0.000000in}{2.250000in}}%
\pgfpathclose%
\pgfusepath{fill}%
\end{pgfscope}%
\begin{pgfscope}%
\pgfsetbuttcap%
\pgfsetmiterjoin%
\definecolor{currentfill}{rgb}{1.000000,1.000000,1.000000}%
\pgfsetfillcolor{currentfill}%
\pgfsetlinewidth{0.000000pt}%
\definecolor{currentstroke}{rgb}{0.000000,0.000000,0.000000}%
\pgfsetstrokecolor{currentstroke}%
\pgfsetstrokeopacity{0.000000}%
\pgfsetdash{}{0pt}%
\pgfpathmoveto{\pgfqpoint{0.150000in}{0.150000in}}%
\pgfpathlineto{\pgfqpoint{2.850000in}{0.150000in}}%
\pgfpathlineto{\pgfqpoint{2.850000in}{2.100000in}}%
\pgfpathlineto{\pgfqpoint{0.150000in}{2.100000in}}%
\pgfpathclose%
\pgfusepath{fill}%
\end{pgfscope}%
\begin{pgfscope}%
\pgfsetbuttcap%
\pgfsetmiterjoin%
\definecolor{currentfill}{rgb}{0.950000,0.950000,0.950000}%
\pgfsetfillcolor{currentfill}%
\pgfsetfillopacity{0.500000}%
\pgfsetlinewidth{1.003750pt}%
\definecolor{currentstroke}{rgb}{0.950000,0.950000,0.950000}%
\pgfsetstrokecolor{currentstroke}%
\pgfsetstrokeopacity{0.500000}%
\pgfsetdash{}{0pt}%
\pgfpathmoveto{\pgfqpoint{2.573296in}{0.776948in}}%
\pgfpathlineto{\pgfqpoint{1.536486in}{1.299017in}}%
\pgfpathlineto{\pgfqpoint{1.536486in}{2.074448in}}%
\pgfpathlineto{\pgfqpoint{2.652584in}{1.554387in}}%
\pgfusepath{stroke,fill}%
\end{pgfscope}%
\begin{pgfscope}%
\pgfsetbuttcap%
\pgfsetmiterjoin%
\definecolor{currentfill}{rgb}{0.900000,0.900000,0.900000}%
\pgfsetfillcolor{currentfill}%
\pgfsetfillopacity{0.500000}%
\pgfsetlinewidth{1.003750pt}%
\definecolor{currentstroke}{rgb}{0.900000,0.900000,0.900000}%
\pgfsetstrokecolor{currentstroke}%
\pgfsetstrokeopacity{0.500000}%
\pgfsetdash{}{0pt}%
\pgfpathmoveto{\pgfqpoint{0.499677in}{0.776948in}}%
\pgfpathlineto{\pgfqpoint{1.536486in}{1.299017in}}%
\pgfpathlineto{\pgfqpoint{1.536486in}{2.074448in}}%
\pgfpathlineto{\pgfqpoint{0.420389in}{1.554387in}}%
\pgfusepath{stroke,fill}%
\end{pgfscope}%
\begin{pgfscope}%
\pgfsetbuttcap%
\pgfsetmiterjoin%
\definecolor{currentfill}{rgb}{0.925000,0.925000,0.925000}%
\pgfsetfillcolor{currentfill}%
\pgfsetfillopacity{0.500000}%
\pgfsetlinewidth{1.003750pt}%
\definecolor{currentstroke}{rgb}{0.925000,0.925000,0.925000}%
\pgfsetstrokecolor{currentstroke}%
\pgfsetstrokeopacity{0.500000}%
\pgfsetdash{}{0pt}%
\pgfpathmoveto{\pgfqpoint{1.536486in}{0.199655in}}%
\pgfpathlineto{\pgfqpoint{2.573296in}{0.776948in}}%
\pgfpathlineto{\pgfqpoint{1.536486in}{1.299017in}}%
\pgfpathlineto{\pgfqpoint{0.499677in}{0.776948in}}%
\pgfusepath{stroke,fill}%
\end{pgfscope}%
\begin{pgfscope}%
\pgfsetrectcap%
\pgfsetroundjoin%
\pgfsetlinewidth{0.803000pt}%
\definecolor{currentstroke}{rgb}{0.000000,0.000000,0.000000}%
\pgfsetstrokecolor{currentstroke}%
\pgfsetdash{}{0pt}%
\pgfpathmoveto{\pgfqpoint{2.573296in}{0.776948in}}%
\pgfpathlineto{\pgfqpoint{1.536486in}{0.199655in}}%
\pgfusepath{stroke}%
\end{pgfscope}%
\begin{pgfscope}%
\definecolor{textcolor}{rgb}{0.000000,0.000000,0.000000}%
\pgfsetstrokecolor{textcolor}%
\pgfsetfillcolor{textcolor}%
\pgftext[x=2.017747in,y=0.045475in,left,base,rotate=29.108966]{\color{textcolor}\sffamily\fontsize{8.000000}{9.600000}\selectfont Player 2 (\(\displaystyle \mu\))}%
\end{pgfscope}%
\begin{pgfscope}%
\pgfsetbuttcap%
\pgfsetroundjoin%
\pgfsetlinewidth{0.803000pt}%
\definecolor{currentstroke}{rgb}{0.690196,0.690196,0.690196}%
\pgfsetstrokecolor{currentstroke}%
\pgfsetdash{}{0pt}%
\pgfpathmoveto{\pgfqpoint{1.605722in}{0.238205in}}%
\pgfpathlineto{\pgfqpoint{0.568749in}{0.811728in}}%
\pgfpathlineto{\pgfqpoint{0.494997in}{1.589151in}}%
\pgfusepath{stroke}%
\end{pgfscope}%
\begin{pgfscope}%
\pgfsetbuttcap%
\pgfsetroundjoin%
\pgfsetlinewidth{0.803000pt}%
\definecolor{currentstroke}{rgb}{0.690196,0.690196,0.690196}%
\pgfsetstrokecolor{currentstroke}%
\pgfsetdash{}{0pt}%
\pgfpathmoveto{\pgfqpoint{1.793262in}{0.342627in}}%
\pgfpathlineto{\pgfqpoint{0.755965in}{0.905998in}}%
\pgfpathlineto{\pgfqpoint{0.697035in}{1.683294in}}%
\pgfusepath{stroke}%
\end{pgfscope}%
\begin{pgfscope}%
\pgfsetbuttcap%
\pgfsetroundjoin%
\pgfsetlinewidth{0.803000pt}%
\definecolor{currentstroke}{rgb}{0.690196,0.690196,0.690196}%
\pgfsetstrokecolor{currentstroke}%
\pgfsetdash{}{0pt}%
\pgfpathmoveto{\pgfqpoint{1.977414in}{0.445162in}}%
\pgfpathlineto{\pgfqpoint{0.939964in}{0.998647in}}%
\pgfpathlineto{\pgfqpoint{0.895342in}{1.775698in}}%
\pgfusepath{stroke}%
\end{pgfscope}%
\begin{pgfscope}%
\pgfsetbuttcap%
\pgfsetroundjoin%
\pgfsetlinewidth{0.803000pt}%
\definecolor{currentstroke}{rgb}{0.690196,0.690196,0.690196}%
\pgfsetstrokecolor{currentstroke}%
\pgfsetdash{}{0pt}%
\pgfpathmoveto{\pgfqpoint{2.158267in}{0.545861in}}%
\pgfpathlineto{\pgfqpoint{1.120829in}{1.089719in}}%
\pgfpathlineto{\pgfqpoint{1.090021in}{1.866411in}}%
\pgfusepath{stroke}%
\end{pgfscope}%
\begin{pgfscope}%
\pgfsetbuttcap%
\pgfsetroundjoin%
\pgfsetlinewidth{0.803000pt}%
\definecolor{currentstroke}{rgb}{0.690196,0.690196,0.690196}%
\pgfsetstrokecolor{currentstroke}%
\pgfsetdash{}{0pt}%
\pgfpathmoveto{\pgfqpoint{2.335912in}{0.644773in}}%
\pgfpathlineto{\pgfqpoint{1.298639in}{1.179253in}}%
\pgfpathlineto{\pgfqpoint{1.281170in}{1.955480in}}%
\pgfusepath{stroke}%
\end{pgfscope}%
\begin{pgfscope}%
\pgfsetbuttcap%
\pgfsetroundjoin%
\pgfsetlinewidth{0.803000pt}%
\definecolor{currentstroke}{rgb}{0.690196,0.690196,0.690196}%
\pgfsetstrokecolor{currentstroke}%
\pgfsetdash{}{0pt}%
\pgfpathmoveto{\pgfqpoint{2.510430in}{0.741945in}}%
\pgfpathlineto{\pgfqpoint{1.473472in}{1.267287in}}%
\pgfpathlineto{\pgfqpoint{1.468885in}{2.042948in}}%
\pgfusepath{stroke}%
\end{pgfscope}%
\begin{pgfscope}%
\pgfsetrectcap%
\pgfsetroundjoin%
\pgfsetlinewidth{0.803000pt}%
\definecolor{currentstroke}{rgb}{0.000000,0.000000,0.000000}%
\pgfsetstrokecolor{currentstroke}%
\pgfsetdash{}{0pt}%
\pgfpathmoveto{\pgfqpoint{1.596992in}{0.243033in}}%
\pgfpathlineto{\pgfqpoint{1.623203in}{0.228537in}}%
\pgfusepath{stroke}%
\end{pgfscope}%
\begin{pgfscope}%
\definecolor{textcolor}{rgb}{0.000000,0.000000,0.000000}%
\pgfsetstrokecolor{textcolor}%
\pgfsetfillcolor{textcolor}%
\pgftext[x=1.680378in,y=0.147403in,,top]{\color{textcolor}\sffamily\fontsize{6.000000}{7.200000}\selectfont \(\displaystyle 0.0\)}%
\end{pgfscope}%
\begin{pgfscope}%
\pgfsetrectcap%
\pgfsetroundjoin%
\pgfsetlinewidth{0.803000pt}%
\definecolor{currentstroke}{rgb}{0.000000,0.000000,0.000000}%
\pgfsetstrokecolor{currentstroke}%
\pgfsetdash{}{0pt}%
\pgfpathmoveto{\pgfqpoint{1.784534in}{0.347367in}}%
\pgfpathlineto{\pgfqpoint{1.810740in}{0.333134in}}%
\pgfusepath{stroke}%
\end{pgfscope}%
\begin{pgfscope}%
\definecolor{textcolor}{rgb}{0.000000,0.000000,0.000000}%
\pgfsetstrokecolor{textcolor}%
\pgfsetfillcolor{textcolor}%
\pgftext[x=1.866959in,y=0.252496in,,top]{\color{textcolor}\sffamily\fontsize{6.000000}{7.200000}\selectfont \(\displaystyle 0.2\)}%
\end{pgfscope}%
\begin{pgfscope}%
\pgfsetrectcap%
\pgfsetroundjoin%
\pgfsetlinewidth{0.803000pt}%
\definecolor{currentstroke}{rgb}{0.000000,0.000000,0.000000}%
\pgfsetstrokecolor{currentstroke}%
\pgfsetdash{}{0pt}%
\pgfpathmoveto{\pgfqpoint{1.968688in}{0.449817in}}%
\pgfpathlineto{\pgfqpoint{1.994886in}{0.435840in}}%
\pgfusepath{stroke}%
\end{pgfscope}%
\begin{pgfscope}%
\definecolor{textcolor}{rgb}{0.000000,0.000000,0.000000}%
\pgfsetstrokecolor{textcolor}%
\pgfsetfillcolor{textcolor}%
\pgftext[x=2.050175in,y=0.355693in,,top]{\color{textcolor}\sffamily\fontsize{6.000000}{7.200000}\selectfont \(\displaystyle 0.4\)}%
\end{pgfscope}%
\begin{pgfscope}%
\pgfsetrectcap%
\pgfsetroundjoin%
\pgfsetlinewidth{0.803000pt}%
\definecolor{currentstroke}{rgb}{0.000000,0.000000,0.000000}%
\pgfsetstrokecolor{currentstroke}%
\pgfsetdash{}{0pt}%
\pgfpathmoveto{\pgfqpoint{2.149546in}{0.550433in}}%
\pgfpathlineto{\pgfqpoint{2.175732in}{0.536706in}}%
\pgfusepath{stroke}%
\end{pgfscope}%
\begin{pgfscope}%
\definecolor{textcolor}{rgb}{0.000000,0.000000,0.000000}%
\pgfsetstrokecolor{textcolor}%
\pgfsetfillcolor{textcolor}%
\pgftext[x=2.230114in,y=0.457045in,,top]{\color{textcolor}\sffamily\fontsize{6.000000}{7.200000}\selectfont \(\displaystyle 0.6\)}%
\end{pgfscope}%
\begin{pgfscope}%
\pgfsetrectcap%
\pgfsetroundjoin%
\pgfsetlinewidth{0.803000pt}%
\definecolor{currentstroke}{rgb}{0.000000,0.000000,0.000000}%
\pgfsetstrokecolor{currentstroke}%
\pgfsetdash{}{0pt}%
\pgfpathmoveto{\pgfqpoint{2.327195in}{0.649264in}}%
\pgfpathlineto{\pgfqpoint{2.353366in}{0.635779in}}%
\pgfusepath{stroke}%
\end{pgfscope}%
\begin{pgfscope}%
\definecolor{textcolor}{rgb}{0.000000,0.000000,0.000000}%
\pgfsetstrokecolor{textcolor}%
\pgfsetfillcolor{textcolor}%
\pgftext[x=2.406864in,y=0.556601in,,top]{\color{textcolor}\sffamily\fontsize{6.000000}{7.200000}\selectfont \(\displaystyle 0.8\)}%
\end{pgfscope}%
\begin{pgfscope}%
\pgfsetrectcap%
\pgfsetroundjoin%
\pgfsetlinewidth{0.803000pt}%
\definecolor{currentstroke}{rgb}{0.000000,0.000000,0.000000}%
\pgfsetstrokecolor{currentstroke}%
\pgfsetdash{}{0pt}%
\pgfpathmoveto{\pgfqpoint{2.501720in}{0.746357in}}%
\pgfpathlineto{\pgfqpoint{2.527872in}{0.733108in}}%
\pgfusepath{stroke}%
\end{pgfscope}%
\begin{pgfscope}%
\definecolor{textcolor}{rgb}{0.000000,0.000000,0.000000}%
\pgfsetstrokecolor{textcolor}%
\pgfsetfillcolor{textcolor}%
\pgftext[x=2.580510in,y=0.654408in,,top]{\color{textcolor}\sffamily\fontsize{6.000000}{7.200000}\selectfont \(\displaystyle 1.0\)}%
\end{pgfscope}%
\begin{pgfscope}%
\pgfsetrectcap%
\pgfsetroundjoin%
\pgfsetlinewidth{0.803000pt}%
\definecolor{currentstroke}{rgb}{0.000000,0.000000,0.000000}%
\pgfsetstrokecolor{currentstroke}%
\pgfsetdash{}{0pt}%
\pgfpathmoveto{\pgfqpoint{0.499677in}{0.776948in}}%
\pgfpathlineto{\pgfqpoint{1.536486in}{0.199655in}}%
\pgfusepath{stroke}%
\end{pgfscope}%
\begin{pgfscope}%
\definecolor{textcolor}{rgb}{0.000000,0.000000,0.000000}%
\pgfsetstrokecolor{textcolor}%
\pgfsetfillcolor{textcolor}%
\pgftext[x=0.492803in,y=0.358631in,left,base,rotate=330.891034]{\color{textcolor}\sffamily\fontsize{8.000000}{9.600000}\selectfont Player 1 (\(\displaystyle \lambda\))}%
\end{pgfscope}%
\begin{pgfscope}%
\pgfsetbuttcap%
\pgfsetroundjoin%
\pgfsetlinewidth{0.803000pt}%
\definecolor{currentstroke}{rgb}{0.690196,0.690196,0.690196}%
\pgfsetstrokecolor{currentstroke}%
\pgfsetdash{}{0pt}%
\pgfpathmoveto{\pgfqpoint{2.577976in}{1.589151in}}%
\pgfpathlineto{\pgfqpoint{2.504223in}{0.811728in}}%
\pgfpathlineto{\pgfqpoint{1.467251in}{0.238205in}}%
\pgfusepath{stroke}%
\end{pgfscope}%
\begin{pgfscope}%
\pgfsetbuttcap%
\pgfsetroundjoin%
\pgfsetlinewidth{0.803000pt}%
\definecolor{currentstroke}{rgb}{0.690196,0.690196,0.690196}%
\pgfsetstrokecolor{currentstroke}%
\pgfsetdash{}{0pt}%
\pgfpathmoveto{\pgfqpoint{2.375938in}{1.683294in}}%
\pgfpathlineto{\pgfqpoint{2.317008in}{0.905998in}}%
\pgfpathlineto{\pgfqpoint{1.279711in}{0.342627in}}%
\pgfusepath{stroke}%
\end{pgfscope}%
\begin{pgfscope}%
\pgfsetbuttcap%
\pgfsetroundjoin%
\pgfsetlinewidth{0.803000pt}%
\definecolor{currentstroke}{rgb}{0.690196,0.690196,0.690196}%
\pgfsetstrokecolor{currentstroke}%
\pgfsetdash{}{0pt}%
\pgfpathmoveto{\pgfqpoint{2.177631in}{1.775698in}}%
\pgfpathlineto{\pgfqpoint{2.133009in}{0.998647in}}%
\pgfpathlineto{\pgfqpoint{1.095559in}{0.445162in}}%
\pgfusepath{stroke}%
\end{pgfscope}%
\begin{pgfscope}%
\pgfsetbuttcap%
\pgfsetroundjoin%
\pgfsetlinewidth{0.803000pt}%
\definecolor{currentstroke}{rgb}{0.690196,0.690196,0.690196}%
\pgfsetstrokecolor{currentstroke}%
\pgfsetdash{}{0pt}%
\pgfpathmoveto{\pgfqpoint{1.982952in}{1.866411in}}%
\pgfpathlineto{\pgfqpoint{1.952144in}{1.089719in}}%
\pgfpathlineto{\pgfqpoint{0.914705in}{0.545861in}}%
\pgfusepath{stroke}%
\end{pgfscope}%
\begin{pgfscope}%
\pgfsetbuttcap%
\pgfsetroundjoin%
\pgfsetlinewidth{0.803000pt}%
\definecolor{currentstroke}{rgb}{0.690196,0.690196,0.690196}%
\pgfsetstrokecolor{currentstroke}%
\pgfsetdash{}{0pt}%
\pgfpathmoveto{\pgfqpoint{1.791803in}{1.955480in}}%
\pgfpathlineto{\pgfqpoint{1.774334in}{1.179253in}}%
\pgfpathlineto{\pgfqpoint{0.737061in}{0.644773in}}%
\pgfusepath{stroke}%
\end{pgfscope}%
\begin{pgfscope}%
\pgfsetbuttcap%
\pgfsetroundjoin%
\pgfsetlinewidth{0.803000pt}%
\definecolor{currentstroke}{rgb}{0.690196,0.690196,0.690196}%
\pgfsetstrokecolor{currentstroke}%
\pgfsetdash{}{0pt}%
\pgfpathmoveto{\pgfqpoint{1.604088in}{2.042948in}}%
\pgfpathlineto{\pgfqpoint{1.599501in}{1.267287in}}%
\pgfpathlineto{\pgfqpoint{0.562543in}{0.741945in}}%
\pgfusepath{stroke}%
\end{pgfscope}%
\begin{pgfscope}%
\pgfsetrectcap%
\pgfsetroundjoin%
\pgfsetlinewidth{0.803000pt}%
\definecolor{currentstroke}{rgb}{0.000000,0.000000,0.000000}%
\pgfsetstrokecolor{currentstroke}%
\pgfsetdash{}{0pt}%
\pgfpathmoveto{\pgfqpoint{1.475981in}{0.243033in}}%
\pgfpathlineto{\pgfqpoint{1.449770in}{0.228537in}}%
\pgfusepath{stroke}%
\end{pgfscope}%
\begin{pgfscope}%
\definecolor{textcolor}{rgb}{0.000000,0.000000,0.000000}%
\pgfsetstrokecolor{textcolor}%
\pgfsetfillcolor{textcolor}%
\pgftext[x=1.392595in,y=0.147403in,,top]{\color{textcolor}\sffamily\fontsize{6.000000}{7.200000}\selectfont \(\displaystyle 0.0\)}%
\end{pgfscope}%
\begin{pgfscope}%
\pgfsetrectcap%
\pgfsetroundjoin%
\pgfsetlinewidth{0.803000pt}%
\definecolor{currentstroke}{rgb}{0.000000,0.000000,0.000000}%
\pgfsetstrokecolor{currentstroke}%
\pgfsetdash{}{0pt}%
\pgfpathmoveto{\pgfqpoint{1.288439in}{0.347367in}}%
\pgfpathlineto{\pgfqpoint{1.262233in}{0.333134in}}%
\pgfusepath{stroke}%
\end{pgfscope}%
\begin{pgfscope}%
\definecolor{textcolor}{rgb}{0.000000,0.000000,0.000000}%
\pgfsetstrokecolor{textcolor}%
\pgfsetfillcolor{textcolor}%
\pgftext[x=1.206013in,y=0.252496in,,top]{\color{textcolor}\sffamily\fontsize{6.000000}{7.200000}\selectfont \(\displaystyle 0.2\)}%
\end{pgfscope}%
\begin{pgfscope}%
\pgfsetrectcap%
\pgfsetroundjoin%
\pgfsetlinewidth{0.803000pt}%
\definecolor{currentstroke}{rgb}{0.000000,0.000000,0.000000}%
\pgfsetstrokecolor{currentstroke}%
\pgfsetdash{}{0pt}%
\pgfpathmoveto{\pgfqpoint{1.104285in}{0.449817in}}%
\pgfpathlineto{\pgfqpoint{1.078087in}{0.435840in}}%
\pgfusepath{stroke}%
\end{pgfscope}%
\begin{pgfscope}%
\definecolor{textcolor}{rgb}{0.000000,0.000000,0.000000}%
\pgfsetstrokecolor{textcolor}%
\pgfsetfillcolor{textcolor}%
\pgftext[x=1.022798in,y=0.355693in,,top]{\color{textcolor}\sffamily\fontsize{6.000000}{7.200000}\selectfont \(\displaystyle 0.4\)}%
\end{pgfscope}%
\begin{pgfscope}%
\pgfsetrectcap%
\pgfsetroundjoin%
\pgfsetlinewidth{0.803000pt}%
\definecolor{currentstroke}{rgb}{0.000000,0.000000,0.000000}%
\pgfsetstrokecolor{currentstroke}%
\pgfsetdash{}{0pt}%
\pgfpathmoveto{\pgfqpoint{0.923427in}{0.550433in}}%
\pgfpathlineto{\pgfqpoint{0.897241in}{0.536706in}}%
\pgfusepath{stroke}%
\end{pgfscope}%
\begin{pgfscope}%
\definecolor{textcolor}{rgb}{0.000000,0.000000,0.000000}%
\pgfsetstrokecolor{textcolor}%
\pgfsetfillcolor{textcolor}%
\pgftext[x=0.842859in,y=0.457045in,,top]{\color{textcolor}\sffamily\fontsize{6.000000}{7.200000}\selectfont \(\displaystyle 0.6\)}%
\end{pgfscope}%
\begin{pgfscope}%
\pgfsetrectcap%
\pgfsetroundjoin%
\pgfsetlinewidth{0.803000pt}%
\definecolor{currentstroke}{rgb}{0.000000,0.000000,0.000000}%
\pgfsetstrokecolor{currentstroke}%
\pgfsetdash{}{0pt}%
\pgfpathmoveto{\pgfqpoint{0.745778in}{0.649264in}}%
\pgfpathlineto{\pgfqpoint{0.719607in}{0.635779in}}%
\pgfusepath{stroke}%
\end{pgfscope}%
\begin{pgfscope}%
\definecolor{textcolor}{rgb}{0.000000,0.000000,0.000000}%
\pgfsetstrokecolor{textcolor}%
\pgfsetfillcolor{textcolor}%
\pgftext[x=0.666109in,y=0.556601in,,top]{\color{textcolor}\sffamily\fontsize{6.000000}{7.200000}\selectfont \(\displaystyle 0.8\)}%
\end{pgfscope}%
\begin{pgfscope}%
\pgfsetrectcap%
\pgfsetroundjoin%
\pgfsetlinewidth{0.803000pt}%
\definecolor{currentstroke}{rgb}{0.000000,0.000000,0.000000}%
\pgfsetstrokecolor{currentstroke}%
\pgfsetdash{}{0pt}%
\pgfpathmoveto{\pgfqpoint{0.571253in}{0.746357in}}%
\pgfpathlineto{\pgfqpoint{0.545101in}{0.733108in}}%
\pgfusepath{stroke}%
\end{pgfscope}%
\begin{pgfscope}%
\definecolor{textcolor}{rgb}{0.000000,0.000000,0.000000}%
\pgfsetstrokecolor{textcolor}%
\pgfsetfillcolor{textcolor}%
\pgftext[x=0.492463in,y=0.654408in,,top]{\color{textcolor}\sffamily\fontsize{6.000000}{7.200000}\selectfont \(\displaystyle 1.0\)}%
\end{pgfscope}%
\begin{pgfscope}%
\pgfsetrectcap%
\pgfsetroundjoin%
\pgfsetlinewidth{0.803000pt}%
\definecolor{currentstroke}{rgb}{0.000000,0.000000,0.000000}%
\pgfsetstrokecolor{currentstroke}%
\pgfsetdash{}{0pt}%
\pgfpathmoveto{\pgfqpoint{0.499677in}{0.776948in}}%
\pgfpathlineto{\pgfqpoint{0.420389in}{1.554387in}}%
\pgfusepath{stroke}%
\end{pgfscope}%
\begin{pgfscope}%
\definecolor{textcolor}{rgb}{0.000000,0.000000,0.000000}%
\pgfsetstrokecolor{textcolor}%
\pgfsetfillcolor{textcolor}%
\pgftext[x=0.041630in,y=1.401767in,left,base,rotate=275.823265]{\color{textcolor}\sffamily\fontsize{8.000000}{9.600000}\selectfont \(\displaystyle \mathsf{val}(G_{\lambda, \mu}\))}%
\end{pgfscope}%
\begin{pgfscope}%
\pgfsetbuttcap%
\pgfsetroundjoin%
\pgfsetlinewidth{0.803000pt}%
\definecolor{currentstroke}{rgb}{0.690196,0.690196,0.690196}%
\pgfsetstrokecolor{currentstroke}%
\pgfsetdash{}{0pt}%
\pgfpathmoveto{\pgfqpoint{0.492066in}{0.851576in}}%
\pgfpathlineto{\pgfqpoint{1.536486in}{1.373698in}}%
\pgfpathlineto{\pgfqpoint{2.580907in}{0.851576in}}%
\pgfusepath{stroke}%
\end{pgfscope}%
\begin{pgfscope}%
\pgfsetbuttcap%
\pgfsetroundjoin%
\pgfsetlinewidth{0.803000pt}%
\definecolor{currentstroke}{rgb}{0.690196,0.690196,0.690196}%
\pgfsetstrokecolor{currentstroke}%
\pgfsetdash{}{0pt}%
\pgfpathmoveto{\pgfqpoint{0.476404in}{1.005144in}}%
\pgfpathlineto{\pgfqpoint{1.536486in}{1.527211in}}%
\pgfpathlineto{\pgfqpoint{2.596569in}{1.005144in}}%
\pgfusepath{stroke}%
\end{pgfscope}%
\begin{pgfscope}%
\pgfsetbuttcap%
\pgfsetroundjoin%
\pgfsetlinewidth{0.803000pt}%
\definecolor{currentstroke}{rgb}{0.690196,0.690196,0.690196}%
\pgfsetstrokecolor{currentstroke}%
\pgfsetdash{}{0pt}%
\pgfpathmoveto{\pgfqpoint{0.460266in}{1.163387in}}%
\pgfpathlineto{\pgfqpoint{1.536486in}{1.685165in}}%
\pgfpathlineto{\pgfqpoint{2.612707in}{1.163387in}}%
\pgfusepath{stroke}%
\end{pgfscope}%
\begin{pgfscope}%
\pgfsetbuttcap%
\pgfsetroundjoin%
\pgfsetlinewidth{0.803000pt}%
\definecolor{currentstroke}{rgb}{0.690196,0.690196,0.690196}%
\pgfsetstrokecolor{currentstroke}%
\pgfsetdash{}{0pt}%
\pgfpathmoveto{\pgfqpoint{0.443628in}{1.326522in}}%
\pgfpathlineto{\pgfqpoint{1.536486in}{1.847758in}}%
\pgfpathlineto{\pgfqpoint{2.629345in}{1.326522in}}%
\pgfusepath{stroke}%
\end{pgfscope}%
\begin{pgfscope}%
\pgfsetbuttcap%
\pgfsetroundjoin%
\pgfsetlinewidth{0.803000pt}%
\definecolor{currentstroke}{rgb}{0.690196,0.690196,0.690196}%
\pgfsetstrokecolor{currentstroke}%
\pgfsetdash{}{0pt}%
\pgfpathmoveto{\pgfqpoint{0.426468in}{1.494781in}}%
\pgfpathlineto{\pgfqpoint{1.536486in}{2.015196in}}%
\pgfpathlineto{\pgfqpoint{2.646505in}{1.494781in}}%
\pgfusepath{stroke}%
\end{pgfscope}%
\begin{pgfscope}%
\pgfsetrectcap%
\pgfsetroundjoin%
\pgfsetlinewidth{0.803000pt}%
\definecolor{currentstroke}{rgb}{0.000000,0.000000,0.000000}%
\pgfsetstrokecolor{currentstroke}%
\pgfsetdash{}{0pt}%
\pgfpathmoveto{\pgfqpoint{0.500841in}{0.855963in}}%
\pgfpathlineto{\pgfqpoint{0.474496in}{0.842793in}}%
\pgfusepath{stroke}%
\end{pgfscope}%
\begin{pgfscope}%
\definecolor{textcolor}{rgb}{0.000000,0.000000,0.000000}%
\pgfsetstrokecolor{textcolor}%
\pgfsetfillcolor{textcolor}%
\pgftext[x=0.348318in,y=0.851576in,,top]{\color{textcolor}\sffamily\fontsize{6.000000}{7.200000}\selectfont \(\displaystyle -0.4\)}%
\end{pgfscope}%
\begin{pgfscope}%
\pgfsetrectcap%
\pgfsetroundjoin%
\pgfsetlinewidth{0.803000pt}%
\definecolor{currentstroke}{rgb}{0.000000,0.000000,0.000000}%
\pgfsetstrokecolor{currentstroke}%
\pgfsetdash{}{0pt}%
\pgfpathmoveto{\pgfqpoint{0.485317in}{1.009533in}}%
\pgfpathlineto{\pgfqpoint{0.458557in}{0.996354in}}%
\pgfusepath{stroke}%
\end{pgfscope}%
\begin{pgfscope}%
\definecolor{textcolor}{rgb}{0.000000,0.000000,0.000000}%
\pgfsetstrokecolor{textcolor}%
\pgfsetfillcolor{textcolor}%
\pgftext[x=0.330501in,y=1.005144in,,top]{\color{textcolor}\sffamily\fontsize{6.000000}{7.200000}\selectfont \(\displaystyle -0.2\)}%
\end{pgfscope}%
\begin{pgfscope}%
\pgfsetrectcap%
\pgfsetroundjoin%
\pgfsetlinewidth{0.803000pt}%
\definecolor{currentstroke}{rgb}{0.000000,0.000000,0.000000}%
\pgfsetstrokecolor{currentstroke}%
\pgfsetdash{}{0pt}%
\pgfpathmoveto{\pgfqpoint{0.469321in}{1.167777in}}%
\pgfpathlineto{\pgfqpoint{0.442133in}{1.154595in}}%
\pgfusepath{stroke}%
\end{pgfscope}%
\begin{pgfscope}%
\definecolor{textcolor}{rgb}{0.000000,0.000000,0.000000}%
\pgfsetstrokecolor{textcolor}%
\pgfsetfillcolor{textcolor}%
\pgftext[x=0.312141in,y=1.163387in,,top]{\color{textcolor}\sffamily\fontsize{6.000000}{7.200000}\selectfont \(\displaystyle 0.0\)}%
\end{pgfscope}%
\begin{pgfscope}%
\pgfsetrectcap%
\pgfsetroundjoin%
\pgfsetlinewidth{0.803000pt}%
\definecolor{currentstroke}{rgb}{0.000000,0.000000,0.000000}%
\pgfsetstrokecolor{currentstroke}%
\pgfsetdash{}{0pt}%
\pgfpathmoveto{\pgfqpoint{0.452830in}{1.330911in}}%
\pgfpathlineto{\pgfqpoint{0.425201in}{1.317733in}}%
\pgfusepath{stroke}%
\end{pgfscope}%
\begin{pgfscope}%
\definecolor{textcolor}{rgb}{0.000000,0.000000,0.000000}%
\pgfsetstrokecolor{textcolor}%
\pgfsetfillcolor{textcolor}%
\pgftext[x=0.293213in,y=1.326522in,,top]{\color{textcolor}\sffamily\fontsize{6.000000}{7.200000}\selectfont \(\displaystyle 0.2\)}%
\end{pgfscope}%
\begin{pgfscope}%
\pgfsetrectcap%
\pgfsetroundjoin%
\pgfsetlinewidth{0.803000pt}%
\definecolor{currentstroke}{rgb}{0.000000,0.000000,0.000000}%
\pgfsetstrokecolor{currentstroke}%
\pgfsetdash{}{0pt}%
\pgfpathmoveto{\pgfqpoint{0.435822in}{1.499166in}}%
\pgfpathlineto{\pgfqpoint{0.407736in}{1.485999in}}%
\pgfusepath{stroke}%
\end{pgfscope}%
\begin{pgfscope}%
\definecolor{textcolor}{rgb}{0.000000,0.000000,0.000000}%
\pgfsetstrokecolor{textcolor}%
\pgfsetfillcolor{textcolor}%
\pgftext[x=0.273691in,y=1.494781in,,top]{\color{textcolor}\sffamily\fontsize{6.000000}{7.200000}\selectfont \(\displaystyle 0.4\)}%
\end{pgfscope}%
\begin{pgfscope}%
\pgfpathrectangle{\pgfqpoint{0.150000in}{0.150000in}}{\pgfqpoint{2.700000in}{1.950000in}}%
\pgfusepath{clip}%
\pgfsetbuttcap%
\pgfsetroundjoin%
\definecolor{currentfill}{rgb}{0.899326,0.817325,0.823820}%
\pgfsetfillcolor{currentfill}%
\pgfsetlinewidth{0.000000pt}%
\definecolor{currentstroke}{rgb}{0.000000,0.000000,0.000000}%
\pgfsetstrokecolor{currentstroke}%
\pgfsetdash{}{0pt}%
\pgfpathmoveto{\pgfqpoint{1.536486in}{1.491069in}}%
\pgfpathlineto{\pgfqpoint{1.572171in}{1.508706in}}%
\pgfpathlineto{\pgfqpoint{1.536486in}{1.526281in}}%
\pgfpathlineto{\pgfqpoint{1.500801in}{1.508706in}}%
\pgfpathclose%
\pgfusepath{fill}%
\end{pgfscope}%
\begin{pgfscope}%
\pgfpathrectangle{\pgfqpoint{0.150000in}{0.150000in}}{\pgfqpoint{2.700000in}{1.950000in}}%
\pgfusepath{clip}%
\pgfsetbuttcap%
\pgfsetroundjoin%
\definecolor{currentfill}{rgb}{0.899326,0.817325,0.823820}%
\pgfsetfillcolor{currentfill}%
\pgfsetlinewidth{0.000000pt}%
\definecolor{currentstroke}{rgb}{0.000000,0.000000,0.000000}%
\pgfsetstrokecolor{currentstroke}%
\pgfsetdash{}{0pt}%
\pgfpathmoveto{\pgfqpoint{1.572295in}{1.473371in}}%
\pgfpathlineto{\pgfqpoint{1.607980in}{1.491069in}}%
\pgfpathlineto{\pgfqpoint{1.572171in}{1.508706in}}%
\pgfpathlineto{\pgfqpoint{1.536486in}{1.491069in}}%
\pgfpathclose%
\pgfusepath{fill}%
\end{pgfscope}%
\begin{pgfscope}%
\pgfpathrectangle{\pgfqpoint{0.150000in}{0.150000in}}{\pgfqpoint{2.700000in}{1.950000in}}%
\pgfusepath{clip}%
\pgfsetbuttcap%
\pgfsetroundjoin%
\definecolor{currentfill}{rgb}{0.899326,0.817325,0.823820}%
\pgfsetfillcolor{currentfill}%
\pgfsetlinewidth{0.000000pt}%
\definecolor{currentstroke}{rgb}{0.000000,0.000000,0.000000}%
\pgfsetstrokecolor{currentstroke}%
\pgfsetdash{}{0pt}%
\pgfpathmoveto{\pgfqpoint{1.500678in}{1.473371in}}%
\pgfpathlineto{\pgfqpoint{1.536486in}{1.491069in}}%
\pgfpathlineto{\pgfqpoint{1.500801in}{1.508706in}}%
\pgfpathlineto{\pgfqpoint{1.464993in}{1.491069in}}%
\pgfpathclose%
\pgfusepath{fill}%
\end{pgfscope}%
\begin{pgfscope}%
\pgfpathrectangle{\pgfqpoint{0.150000in}{0.150000in}}{\pgfqpoint{2.700000in}{1.950000in}}%
\pgfusepath{clip}%
\pgfsetbuttcap%
\pgfsetroundjoin%
\definecolor{currentfill}{rgb}{0.899326,0.817325,0.823820}%
\pgfsetfillcolor{currentfill}%
\pgfsetlinewidth{0.000000pt}%
\definecolor{currentstroke}{rgb}{0.000000,0.000000,0.000000}%
\pgfsetstrokecolor{currentstroke}%
\pgfsetdash{}{0pt}%
\pgfpathmoveto{\pgfqpoint{1.608229in}{1.455612in}}%
\pgfpathlineto{\pgfqpoint{1.643913in}{1.473371in}}%
\pgfpathlineto{\pgfqpoint{1.607980in}{1.491069in}}%
\pgfpathlineto{\pgfqpoint{1.572295in}{1.473371in}}%
\pgfpathclose%
\pgfusepath{fill}%
\end{pgfscope}%
\begin{pgfscope}%
\pgfpathrectangle{\pgfqpoint{0.150000in}{0.150000in}}{\pgfqpoint{2.700000in}{1.950000in}}%
\pgfusepath{clip}%
\pgfsetbuttcap%
\pgfsetroundjoin%
\definecolor{currentfill}{rgb}{0.899326,0.817325,0.823820}%
\pgfsetfillcolor{currentfill}%
\pgfsetlinewidth{0.000000pt}%
\definecolor{currentstroke}{rgb}{0.000000,0.000000,0.000000}%
\pgfsetstrokecolor{currentstroke}%
\pgfsetdash{}{0pt}%
\pgfpathmoveto{\pgfqpoint{1.536486in}{1.455612in}}%
\pgfpathlineto{\pgfqpoint{1.572295in}{1.473371in}}%
\pgfpathlineto{\pgfqpoint{1.536486in}{1.491069in}}%
\pgfpathlineto{\pgfqpoint{1.500678in}{1.473371in}}%
\pgfpathclose%
\pgfusepath{fill}%
\end{pgfscope}%
\begin{pgfscope}%
\pgfpathrectangle{\pgfqpoint{0.150000in}{0.150000in}}{\pgfqpoint{2.700000in}{1.950000in}}%
\pgfusepath{clip}%
\pgfsetbuttcap%
\pgfsetroundjoin%
\definecolor{currentfill}{rgb}{0.899326,0.817325,0.823820}%
\pgfsetfillcolor{currentfill}%
\pgfsetlinewidth{0.000000pt}%
\definecolor{currentstroke}{rgb}{0.000000,0.000000,0.000000}%
\pgfsetstrokecolor{currentstroke}%
\pgfsetdash{}{0pt}%
\pgfpathmoveto{\pgfqpoint{1.464744in}{1.455612in}}%
\pgfpathlineto{\pgfqpoint{1.500678in}{1.473371in}}%
\pgfpathlineto{\pgfqpoint{1.464993in}{1.491069in}}%
\pgfpathlineto{\pgfqpoint{1.429060in}{1.473371in}}%
\pgfpathclose%
\pgfusepath{fill}%
\end{pgfscope}%
\begin{pgfscope}%
\pgfpathrectangle{\pgfqpoint{0.150000in}{0.150000in}}{\pgfqpoint{2.700000in}{1.950000in}}%
\pgfusepath{clip}%
\pgfsetbuttcap%
\pgfsetroundjoin%
\definecolor{currentfill}{rgb}{0.899326,0.817325,0.823820}%
\pgfsetfillcolor{currentfill}%
\pgfsetlinewidth{0.000000pt}%
\definecolor{currentstroke}{rgb}{0.000000,0.000000,0.000000}%
\pgfsetstrokecolor{currentstroke}%
\pgfsetdash{}{0pt}%
\pgfpathmoveto{\pgfqpoint{1.644287in}{1.437791in}}%
\pgfpathlineto{\pgfqpoint{1.679971in}{1.455612in}}%
\pgfpathlineto{\pgfqpoint{1.643913in}{1.473371in}}%
\pgfpathlineto{\pgfqpoint{1.608229in}{1.455612in}}%
\pgfpathclose%
\pgfusepath{fill}%
\end{pgfscope}%
\begin{pgfscope}%
\pgfpathrectangle{\pgfqpoint{0.150000in}{0.150000in}}{\pgfqpoint{2.700000in}{1.950000in}}%
\pgfusepath{clip}%
\pgfsetbuttcap%
\pgfsetroundjoin%
\definecolor{currentfill}{rgb}{0.899326,0.817325,0.823820}%
\pgfsetfillcolor{currentfill}%
\pgfsetlinewidth{0.000000pt}%
\definecolor{currentstroke}{rgb}{0.000000,0.000000,0.000000}%
\pgfsetstrokecolor{currentstroke}%
\pgfsetdash{}{0pt}%
\pgfpathmoveto{\pgfqpoint{1.572420in}{1.437791in}}%
\pgfpathlineto{\pgfqpoint{1.608229in}{1.455612in}}%
\pgfpathlineto{\pgfqpoint{1.572295in}{1.473371in}}%
\pgfpathlineto{\pgfqpoint{1.536486in}{1.455612in}}%
\pgfpathclose%
\pgfusepath{fill}%
\end{pgfscope}%
\begin{pgfscope}%
\pgfpathrectangle{\pgfqpoint{0.150000in}{0.150000in}}{\pgfqpoint{2.700000in}{1.950000in}}%
\pgfusepath{clip}%
\pgfsetbuttcap%
\pgfsetroundjoin%
\definecolor{currentfill}{rgb}{0.899326,0.817325,0.823820}%
\pgfsetfillcolor{currentfill}%
\pgfsetlinewidth{0.000000pt}%
\definecolor{currentstroke}{rgb}{0.000000,0.000000,0.000000}%
\pgfsetstrokecolor{currentstroke}%
\pgfsetdash{}{0pt}%
\pgfpathmoveto{\pgfqpoint{1.500553in}{1.437791in}}%
\pgfpathlineto{\pgfqpoint{1.536486in}{1.455612in}}%
\pgfpathlineto{\pgfqpoint{1.500678in}{1.473371in}}%
\pgfpathlineto{\pgfqpoint{1.464744in}{1.455612in}}%
\pgfpathclose%
\pgfusepath{fill}%
\end{pgfscope}%
\begin{pgfscope}%
\pgfpathrectangle{\pgfqpoint{0.150000in}{0.150000in}}{\pgfqpoint{2.700000in}{1.950000in}}%
\pgfusepath{clip}%
\pgfsetbuttcap%
\pgfsetroundjoin%
\definecolor{currentfill}{rgb}{0.899326,0.817325,0.823820}%
\pgfsetfillcolor{currentfill}%
\pgfsetlinewidth{0.000000pt}%
\definecolor{currentstroke}{rgb}{0.000000,0.000000,0.000000}%
\pgfsetstrokecolor{currentstroke}%
\pgfsetdash{}{0pt}%
\pgfpathmoveto{\pgfqpoint{1.428686in}{1.437791in}}%
\pgfpathlineto{\pgfqpoint{1.464744in}{1.455612in}}%
\pgfpathlineto{\pgfqpoint{1.429060in}{1.473371in}}%
\pgfpathlineto{\pgfqpoint{1.393002in}{1.455612in}}%
\pgfpathclose%
\pgfusepath{fill}%
\end{pgfscope}%
\begin{pgfscope}%
\pgfpathrectangle{\pgfqpoint{0.150000in}{0.150000in}}{\pgfqpoint{2.700000in}{1.950000in}}%
\pgfusepath{clip}%
\pgfsetbuttcap%
\pgfsetroundjoin%
\definecolor{currentfill}{rgb}{0.899326,0.817325,0.823820}%
\pgfsetfillcolor{currentfill}%
\pgfsetlinewidth{0.000000pt}%
\definecolor{currentstroke}{rgb}{0.000000,0.000000,0.000000}%
\pgfsetstrokecolor{currentstroke}%
\pgfsetdash{}{0pt}%
\pgfpathmoveto{\pgfqpoint{1.680471in}{1.419908in}}%
\pgfpathlineto{\pgfqpoint{1.716154in}{1.437791in}}%
\pgfpathlineto{\pgfqpoint{1.679971in}{1.455612in}}%
\pgfpathlineto{\pgfqpoint{1.644287in}{1.437791in}}%
\pgfpathclose%
\pgfusepath{fill}%
\end{pgfscope}%
\begin{pgfscope}%
\pgfpathrectangle{\pgfqpoint{0.150000in}{0.150000in}}{\pgfqpoint{2.700000in}{1.950000in}}%
\pgfusepath{clip}%
\pgfsetbuttcap%
\pgfsetroundjoin%
\definecolor{currentfill}{rgb}{0.899326,0.817325,0.823820}%
\pgfsetfillcolor{currentfill}%
\pgfsetlinewidth{0.000000pt}%
\definecolor{currentstroke}{rgb}{0.000000,0.000000,0.000000}%
\pgfsetstrokecolor{currentstroke}%
\pgfsetdash{}{0pt}%
\pgfpathmoveto{\pgfqpoint{1.608479in}{1.419908in}}%
\pgfpathlineto{\pgfqpoint{1.644287in}{1.437791in}}%
\pgfpathlineto{\pgfqpoint{1.608229in}{1.455612in}}%
\pgfpathlineto{\pgfqpoint{1.572420in}{1.437791in}}%
\pgfpathclose%
\pgfusepath{fill}%
\end{pgfscope}%
\begin{pgfscope}%
\pgfpathrectangle{\pgfqpoint{0.150000in}{0.150000in}}{\pgfqpoint{2.700000in}{1.950000in}}%
\pgfusepath{clip}%
\pgfsetbuttcap%
\pgfsetroundjoin%
\definecolor{currentfill}{rgb}{0.899326,0.817325,0.823820}%
\pgfsetfillcolor{currentfill}%
\pgfsetlinewidth{0.000000pt}%
\definecolor{currentstroke}{rgb}{0.000000,0.000000,0.000000}%
\pgfsetstrokecolor{currentstroke}%
\pgfsetdash{}{0pt}%
\pgfpathmoveto{\pgfqpoint{1.536486in}{1.419908in}}%
\pgfpathlineto{\pgfqpoint{1.572420in}{1.437791in}}%
\pgfpathlineto{\pgfqpoint{1.536486in}{1.455612in}}%
\pgfpathlineto{\pgfqpoint{1.500553in}{1.437791in}}%
\pgfpathclose%
\pgfusepath{fill}%
\end{pgfscope}%
\begin{pgfscope}%
\pgfpathrectangle{\pgfqpoint{0.150000in}{0.150000in}}{\pgfqpoint{2.700000in}{1.950000in}}%
\pgfusepath{clip}%
\pgfsetbuttcap%
\pgfsetroundjoin%
\definecolor{currentfill}{rgb}{0.899326,0.817325,0.823820}%
\pgfsetfillcolor{currentfill}%
\pgfsetlinewidth{0.000000pt}%
\definecolor{currentstroke}{rgb}{0.000000,0.000000,0.000000}%
\pgfsetstrokecolor{currentstroke}%
\pgfsetdash{}{0pt}%
\pgfpathmoveto{\pgfqpoint{1.464494in}{1.419908in}}%
\pgfpathlineto{\pgfqpoint{1.500553in}{1.437791in}}%
\pgfpathlineto{\pgfqpoint{1.464744in}{1.455612in}}%
\pgfpathlineto{\pgfqpoint{1.428686in}{1.437791in}}%
\pgfpathclose%
\pgfusepath{fill}%
\end{pgfscope}%
\begin{pgfscope}%
\pgfpathrectangle{\pgfqpoint{0.150000in}{0.150000in}}{\pgfqpoint{2.700000in}{1.950000in}}%
\pgfusepath{clip}%
\pgfsetbuttcap%
\pgfsetroundjoin%
\definecolor{currentfill}{rgb}{0.899326,0.817325,0.823820}%
\pgfsetfillcolor{currentfill}%
\pgfsetlinewidth{0.000000pt}%
\definecolor{currentstroke}{rgb}{0.000000,0.000000,0.000000}%
\pgfsetstrokecolor{currentstroke}%
\pgfsetdash{}{0pt}%
\pgfpathmoveto{\pgfqpoint{1.392502in}{1.419908in}}%
\pgfpathlineto{\pgfqpoint{1.428686in}{1.437791in}}%
\pgfpathlineto{\pgfqpoint{1.393002in}{1.455612in}}%
\pgfpathlineto{\pgfqpoint{1.356819in}{1.437791in}}%
\pgfpathclose%
\pgfusepath{fill}%
\end{pgfscope}%
\begin{pgfscope}%
\pgfpathrectangle{\pgfqpoint{0.150000in}{0.150000in}}{\pgfqpoint{2.700000in}{1.950000in}}%
\pgfusepath{clip}%
\pgfsetbuttcap%
\pgfsetroundjoin%
\definecolor{currentfill}{rgb}{0.625797,0.321002,0.345144}%
\pgfsetfillcolor{currentfill}%
\pgfsetlinewidth{0.000000pt}%
\definecolor{currentstroke}{rgb}{0.000000,0.000000,0.000000}%
\pgfsetstrokecolor{currentstroke}%
\pgfsetdash{}{0pt}%
\pgfpathmoveto{\pgfqpoint{2.184644in}{0.908781in}}%
\pgfpathlineto{\pgfqpoint{2.217972in}{0.905069in}}%
\pgfpathlineto{\pgfqpoint{2.182860in}{0.953795in}}%
\pgfpathlineto{\pgfqpoint{2.149354in}{0.957703in}}%
\pgfpathclose%
\pgfusepath{fill}%
\end{pgfscope}%
\begin{pgfscope}%
\pgfpathrectangle{\pgfqpoint{0.150000in}{0.150000in}}{\pgfqpoint{2.700000in}{1.950000in}}%
\pgfusepath{clip}%
\pgfsetbuttcap%
\pgfsetroundjoin%
\definecolor{currentfill}{rgb}{0.899326,0.817325,0.823820}%
\pgfsetfillcolor{currentfill}%
\pgfsetlinewidth{0.000000pt}%
\definecolor{currentstroke}{rgb}{0.000000,0.000000,0.000000}%
\pgfsetstrokecolor{currentstroke}%
\pgfsetdash{}{0pt}%
\pgfpathmoveto{\pgfqpoint{1.716782in}{1.401962in}}%
\pgfpathlineto{\pgfqpoint{1.752463in}{1.419908in}}%
\pgfpathlineto{\pgfqpoint{1.716154in}{1.437791in}}%
\pgfpathlineto{\pgfqpoint{1.680471in}{1.419908in}}%
\pgfpathclose%
\pgfusepath{fill}%
\end{pgfscope}%
\begin{pgfscope}%
\pgfpathrectangle{\pgfqpoint{0.150000in}{0.150000in}}{\pgfqpoint{2.700000in}{1.950000in}}%
\pgfusepath{clip}%
\pgfsetbuttcap%
\pgfsetroundjoin%
\definecolor{currentfill}{rgb}{0.899326,0.817325,0.823820}%
\pgfsetfillcolor{currentfill}%
\pgfsetlinewidth{0.000000pt}%
\definecolor{currentstroke}{rgb}{0.000000,0.000000,0.000000}%
\pgfsetstrokecolor{currentstroke}%
\pgfsetdash{}{0pt}%
\pgfpathmoveto{\pgfqpoint{1.644664in}{1.401962in}}%
\pgfpathlineto{\pgfqpoint{1.680471in}{1.419908in}}%
\pgfpathlineto{\pgfqpoint{1.644287in}{1.437791in}}%
\pgfpathlineto{\pgfqpoint{1.608479in}{1.419908in}}%
\pgfpathclose%
\pgfusepath{fill}%
\end{pgfscope}%
\begin{pgfscope}%
\pgfpathrectangle{\pgfqpoint{0.150000in}{0.150000in}}{\pgfqpoint{2.700000in}{1.950000in}}%
\pgfusepath{clip}%
\pgfsetbuttcap%
\pgfsetroundjoin%
\definecolor{currentfill}{rgb}{0.899326,0.817325,0.823820}%
\pgfsetfillcolor{currentfill}%
\pgfsetlinewidth{0.000000pt}%
\definecolor{currentstroke}{rgb}{0.000000,0.000000,0.000000}%
\pgfsetstrokecolor{currentstroke}%
\pgfsetdash{}{0pt}%
\pgfpathmoveto{\pgfqpoint{1.572546in}{1.401962in}}%
\pgfpathlineto{\pgfqpoint{1.608479in}{1.419908in}}%
\pgfpathlineto{\pgfqpoint{1.572420in}{1.437791in}}%
\pgfpathlineto{\pgfqpoint{1.536486in}{1.419908in}}%
\pgfpathclose%
\pgfusepath{fill}%
\end{pgfscope}%
\begin{pgfscope}%
\pgfpathrectangle{\pgfqpoint{0.150000in}{0.150000in}}{\pgfqpoint{2.700000in}{1.950000in}}%
\pgfusepath{clip}%
\pgfsetbuttcap%
\pgfsetroundjoin%
\definecolor{currentfill}{rgb}{0.899326,0.817325,0.823820}%
\pgfsetfillcolor{currentfill}%
\pgfsetlinewidth{0.000000pt}%
\definecolor{currentstroke}{rgb}{0.000000,0.000000,0.000000}%
\pgfsetstrokecolor{currentstroke}%
\pgfsetdash{}{0pt}%
\pgfpathmoveto{\pgfqpoint{1.500427in}{1.401962in}}%
\pgfpathlineto{\pgfqpoint{1.536486in}{1.419908in}}%
\pgfpathlineto{\pgfqpoint{1.500553in}{1.437791in}}%
\pgfpathlineto{\pgfqpoint{1.464494in}{1.419908in}}%
\pgfpathclose%
\pgfusepath{fill}%
\end{pgfscope}%
\begin{pgfscope}%
\pgfpathrectangle{\pgfqpoint{0.150000in}{0.150000in}}{\pgfqpoint{2.700000in}{1.950000in}}%
\pgfusepath{clip}%
\pgfsetbuttcap%
\pgfsetroundjoin%
\definecolor{currentfill}{rgb}{0.899326,0.817325,0.823820}%
\pgfsetfillcolor{currentfill}%
\pgfsetlinewidth{0.000000pt}%
\definecolor{currentstroke}{rgb}{0.000000,0.000000,0.000000}%
\pgfsetstrokecolor{currentstroke}%
\pgfsetdash{}{0pt}%
\pgfpathmoveto{\pgfqpoint{1.428309in}{1.401962in}}%
\pgfpathlineto{\pgfqpoint{1.464494in}{1.419908in}}%
\pgfpathlineto{\pgfqpoint{1.428686in}{1.437791in}}%
\pgfpathlineto{\pgfqpoint{1.392502in}{1.419908in}}%
\pgfpathclose%
\pgfusepath{fill}%
\end{pgfscope}%
\begin{pgfscope}%
\pgfpathrectangle{\pgfqpoint{0.150000in}{0.150000in}}{\pgfqpoint{2.700000in}{1.950000in}}%
\pgfusepath{clip}%
\pgfsetbuttcap%
\pgfsetroundjoin%
\definecolor{currentfill}{rgb}{0.899326,0.817325,0.823820}%
\pgfsetfillcolor{currentfill}%
\pgfsetlinewidth{0.000000pt}%
\definecolor{currentstroke}{rgb}{0.000000,0.000000,0.000000}%
\pgfsetstrokecolor{currentstroke}%
\pgfsetdash{}{0pt}%
\pgfpathmoveto{\pgfqpoint{1.356191in}{1.401962in}}%
\pgfpathlineto{\pgfqpoint{1.392502in}{1.419908in}}%
\pgfpathlineto{\pgfqpoint{1.356819in}{1.437791in}}%
\pgfpathlineto{\pgfqpoint{1.320509in}{1.419908in}}%
\pgfpathclose%
\pgfusepath{fill}%
\end{pgfscope}%
\begin{pgfscope}%
\pgfpathrectangle{\pgfqpoint{0.150000in}{0.150000in}}{\pgfqpoint{2.700000in}{1.950000in}}%
\pgfusepath{clip}%
\pgfsetbuttcap%
\pgfsetroundjoin%
\definecolor{currentfill}{rgb}{0.675184,0.410616,0.431572}%
\pgfsetfillcolor{currentfill}%
\pgfsetlinewidth{0.000000pt}%
\definecolor{currentstroke}{rgb}{0.000000,0.000000,0.000000}%
\pgfsetstrokecolor{currentstroke}%
\pgfsetdash{}{0pt}%
\pgfpathmoveto{\pgfqpoint{2.149354in}{0.957703in}}%
\pgfpathlineto{\pgfqpoint{2.182860in}{0.953795in}}%
\pgfpathlineto{\pgfqpoint{2.149451in}{1.033001in}}%
\pgfpathlineto{\pgfqpoint{2.115251in}{1.029598in}}%
\pgfpathclose%
\pgfusepath{fill}%
\end{pgfscope}%
\begin{pgfscope}%
\pgfpathrectangle{\pgfqpoint{0.150000in}{0.150000in}}{\pgfqpoint{2.700000in}{1.950000in}}%
\pgfusepath{clip}%
\pgfsetbuttcap%
\pgfsetroundjoin%
\definecolor{currentfill}{rgb}{0.899326,0.817325,0.823820}%
\pgfsetfillcolor{currentfill}%
\pgfsetlinewidth{0.000000pt}%
\definecolor{currentstroke}{rgb}{0.000000,0.000000,0.000000}%
\pgfsetstrokecolor{currentstroke}%
\pgfsetdash{}{0pt}%
\pgfpathmoveto{\pgfqpoint{1.753219in}{1.383953in}}%
\pgfpathlineto{\pgfqpoint{1.788900in}{1.401962in}}%
\pgfpathlineto{\pgfqpoint{1.752463in}{1.419908in}}%
\pgfpathlineto{\pgfqpoint{1.716782in}{1.401962in}}%
\pgfpathclose%
\pgfusepath{fill}%
\end{pgfscope}%
\begin{pgfscope}%
\pgfpathrectangle{\pgfqpoint{0.150000in}{0.150000in}}{\pgfqpoint{2.700000in}{1.950000in}}%
\pgfusepath{clip}%
\pgfsetbuttcap%
\pgfsetroundjoin%
\definecolor{currentfill}{rgb}{0.899326,0.817325,0.823820}%
\pgfsetfillcolor{currentfill}%
\pgfsetlinewidth{0.000000pt}%
\definecolor{currentstroke}{rgb}{0.000000,0.000000,0.000000}%
\pgfsetstrokecolor{currentstroke}%
\pgfsetdash{}{0pt}%
\pgfpathmoveto{\pgfqpoint{1.680975in}{1.383953in}}%
\pgfpathlineto{\pgfqpoint{1.716782in}{1.401962in}}%
\pgfpathlineto{\pgfqpoint{1.680471in}{1.419908in}}%
\pgfpathlineto{\pgfqpoint{1.644664in}{1.401962in}}%
\pgfpathclose%
\pgfusepath{fill}%
\end{pgfscope}%
\begin{pgfscope}%
\pgfpathrectangle{\pgfqpoint{0.150000in}{0.150000in}}{\pgfqpoint{2.700000in}{1.950000in}}%
\pgfusepath{clip}%
\pgfsetbuttcap%
\pgfsetroundjoin%
\definecolor{currentfill}{rgb}{0.899326,0.817325,0.823820}%
\pgfsetfillcolor{currentfill}%
\pgfsetlinewidth{0.000000pt}%
\definecolor{currentstroke}{rgb}{0.000000,0.000000,0.000000}%
\pgfsetstrokecolor{currentstroke}%
\pgfsetdash{}{0pt}%
\pgfpathmoveto{\pgfqpoint{1.608731in}{1.383953in}}%
\pgfpathlineto{\pgfqpoint{1.644664in}{1.401962in}}%
\pgfpathlineto{\pgfqpoint{1.608479in}{1.419908in}}%
\pgfpathlineto{\pgfqpoint{1.572546in}{1.401962in}}%
\pgfpathclose%
\pgfusepath{fill}%
\end{pgfscope}%
\begin{pgfscope}%
\pgfpathrectangle{\pgfqpoint{0.150000in}{0.150000in}}{\pgfqpoint{2.700000in}{1.950000in}}%
\pgfusepath{clip}%
\pgfsetbuttcap%
\pgfsetroundjoin%
\definecolor{currentfill}{rgb}{0.899326,0.817325,0.823820}%
\pgfsetfillcolor{currentfill}%
\pgfsetlinewidth{0.000000pt}%
\definecolor{currentstroke}{rgb}{0.000000,0.000000,0.000000}%
\pgfsetstrokecolor{currentstroke}%
\pgfsetdash{}{0pt}%
\pgfpathmoveto{\pgfqpoint{1.536486in}{1.383953in}}%
\pgfpathlineto{\pgfqpoint{1.572546in}{1.401962in}}%
\pgfpathlineto{\pgfqpoint{1.536486in}{1.419908in}}%
\pgfpathlineto{\pgfqpoint{1.500427in}{1.401962in}}%
\pgfpathclose%
\pgfusepath{fill}%
\end{pgfscope}%
\begin{pgfscope}%
\pgfpathrectangle{\pgfqpoint{0.150000in}{0.150000in}}{\pgfqpoint{2.700000in}{1.950000in}}%
\pgfusepath{clip}%
\pgfsetbuttcap%
\pgfsetroundjoin%
\definecolor{currentfill}{rgb}{0.899326,0.817325,0.823820}%
\pgfsetfillcolor{currentfill}%
\pgfsetlinewidth{0.000000pt}%
\definecolor{currentstroke}{rgb}{0.000000,0.000000,0.000000}%
\pgfsetstrokecolor{currentstroke}%
\pgfsetdash{}{0pt}%
\pgfpathmoveto{\pgfqpoint{1.464242in}{1.383953in}}%
\pgfpathlineto{\pgfqpoint{1.500427in}{1.401962in}}%
\pgfpathlineto{\pgfqpoint{1.464494in}{1.419908in}}%
\pgfpathlineto{\pgfqpoint{1.428309in}{1.401962in}}%
\pgfpathclose%
\pgfusepath{fill}%
\end{pgfscope}%
\begin{pgfscope}%
\pgfpathrectangle{\pgfqpoint{0.150000in}{0.150000in}}{\pgfqpoint{2.700000in}{1.950000in}}%
\pgfusepath{clip}%
\pgfsetbuttcap%
\pgfsetroundjoin%
\definecolor{currentfill}{rgb}{0.899326,0.817325,0.823820}%
\pgfsetfillcolor{currentfill}%
\pgfsetlinewidth{0.000000pt}%
\definecolor{currentstroke}{rgb}{0.000000,0.000000,0.000000}%
\pgfsetstrokecolor{currentstroke}%
\pgfsetdash{}{0pt}%
\pgfpathmoveto{\pgfqpoint{1.391998in}{1.383953in}}%
\pgfpathlineto{\pgfqpoint{1.428309in}{1.401962in}}%
\pgfpathlineto{\pgfqpoint{1.392502in}{1.419908in}}%
\pgfpathlineto{\pgfqpoint{1.356191in}{1.401962in}}%
\pgfpathclose%
\pgfusepath{fill}%
\end{pgfscope}%
\begin{pgfscope}%
\pgfpathrectangle{\pgfqpoint{0.150000in}{0.150000in}}{\pgfqpoint{2.700000in}{1.950000in}}%
\pgfusepath{clip}%
\pgfsetbuttcap%
\pgfsetroundjoin%
\definecolor{currentfill}{rgb}{0.899326,0.817325,0.823820}%
\pgfsetfillcolor{currentfill}%
\pgfsetlinewidth{0.000000pt}%
\definecolor{currentstroke}{rgb}{0.000000,0.000000,0.000000}%
\pgfsetstrokecolor{currentstroke}%
\pgfsetdash{}{0pt}%
\pgfpathmoveto{\pgfqpoint{1.319754in}{1.383953in}}%
\pgfpathlineto{\pgfqpoint{1.356191in}{1.401962in}}%
\pgfpathlineto{\pgfqpoint{1.320509in}{1.419908in}}%
\pgfpathlineto{\pgfqpoint{1.284073in}{1.401962in}}%
\pgfpathclose%
\pgfusepath{fill}%
\end{pgfscope}%
\begin{pgfscope}%
\pgfpathrectangle{\pgfqpoint{0.150000in}{0.150000in}}{\pgfqpoint{2.700000in}{1.950000in}}%
\pgfusepath{clip}%
\pgfsetbuttcap%
\pgfsetroundjoin%
\definecolor{currentfill}{rgb}{0.640993,0.348575,0.371737}%
\pgfsetfillcolor{currentfill}%
\pgfsetlinewidth{0.000000pt}%
\definecolor{currentstroke}{rgb}{0.000000,0.000000,0.000000}%
\pgfsetstrokecolor{currentstroke}%
\pgfsetdash{}{0pt}%
\pgfpathmoveto{\pgfqpoint{2.225372in}{0.942793in}}%
\pgfpathlineto{\pgfqpoint{2.259298in}{0.946484in}}%
\pgfpathlineto{\pgfqpoint{2.217972in}{0.905069in}}%
\pgfpathlineto{\pgfqpoint{2.184644in}{0.908781in}}%
\pgfpathclose%
\pgfusepath{fill}%
\end{pgfscope}%
\begin{pgfscope}%
\pgfpathrectangle{\pgfqpoint{0.150000in}{0.150000in}}{\pgfqpoint{2.700000in}{1.950000in}}%
\pgfusepath{clip}%
\pgfsetbuttcap%
\pgfsetroundjoin%
\definecolor{currentfill}{rgb}{0.648591,0.362362,0.385034}%
\pgfsetfillcolor{currentfill}%
\pgfsetlinewidth{0.000000pt}%
\definecolor{currentstroke}{rgb}{0.000000,0.000000,0.000000}%
\pgfsetstrokecolor{currentstroke}%
\pgfsetdash{}{0pt}%
\pgfpathmoveto{\pgfqpoint{2.151052in}{0.912522in}}%
\pgfpathlineto{\pgfqpoint{2.184644in}{0.908781in}}%
\pgfpathlineto{\pgfqpoint{2.149354in}{0.957703in}}%
\pgfpathlineto{\pgfqpoint{2.115159in}{0.954058in}}%
\pgfpathclose%
\pgfusepath{fill}%
\end{pgfscope}%
\begin{pgfscope}%
\pgfpathrectangle{\pgfqpoint{0.150000in}{0.150000in}}{\pgfqpoint{2.700000in}{1.950000in}}%
\pgfusepath{clip}%
\pgfsetbuttcap%
\pgfsetroundjoin%
\definecolor{currentfill}{rgb}{0.899326,0.817325,0.823820}%
\pgfsetfillcolor{currentfill}%
\pgfsetlinewidth{0.000000pt}%
\definecolor{currentstroke}{rgb}{0.000000,0.000000,0.000000}%
\pgfsetstrokecolor{currentstroke}%
\pgfsetdash{}{0pt}%
\pgfpathmoveto{\pgfqpoint{1.789785in}{1.365882in}}%
\pgfpathlineto{\pgfqpoint{1.825464in}{1.383953in}}%
\pgfpathlineto{\pgfqpoint{1.788900in}{1.401962in}}%
\pgfpathlineto{\pgfqpoint{1.753219in}{1.383953in}}%
\pgfpathclose%
\pgfusepath{fill}%
\end{pgfscope}%
\begin{pgfscope}%
\pgfpathrectangle{\pgfqpoint{0.150000in}{0.150000in}}{\pgfqpoint{2.700000in}{1.950000in}}%
\pgfusepath{clip}%
\pgfsetbuttcap%
\pgfsetroundjoin%
\definecolor{currentfill}{rgb}{0.899326,0.817325,0.823820}%
\pgfsetfillcolor{currentfill}%
\pgfsetlinewidth{0.000000pt}%
\definecolor{currentstroke}{rgb}{0.000000,0.000000,0.000000}%
\pgfsetstrokecolor{currentstroke}%
\pgfsetdash{}{0pt}%
\pgfpathmoveto{\pgfqpoint{1.717414in}{1.365882in}}%
\pgfpathlineto{\pgfqpoint{1.753219in}{1.383953in}}%
\pgfpathlineto{\pgfqpoint{1.716782in}{1.401962in}}%
\pgfpathlineto{\pgfqpoint{1.680975in}{1.383953in}}%
\pgfpathclose%
\pgfusepath{fill}%
\end{pgfscope}%
\begin{pgfscope}%
\pgfpathrectangle{\pgfqpoint{0.150000in}{0.150000in}}{\pgfqpoint{2.700000in}{1.950000in}}%
\pgfusepath{clip}%
\pgfsetbuttcap%
\pgfsetroundjoin%
\definecolor{currentfill}{rgb}{0.899326,0.817325,0.823820}%
\pgfsetfillcolor{currentfill}%
\pgfsetlinewidth{0.000000pt}%
\definecolor{currentstroke}{rgb}{0.000000,0.000000,0.000000}%
\pgfsetstrokecolor{currentstroke}%
\pgfsetdash{}{0pt}%
\pgfpathmoveto{\pgfqpoint{1.645043in}{1.365882in}}%
\pgfpathlineto{\pgfqpoint{1.680975in}{1.383953in}}%
\pgfpathlineto{\pgfqpoint{1.644664in}{1.401962in}}%
\pgfpathlineto{\pgfqpoint{1.608731in}{1.383953in}}%
\pgfpathclose%
\pgfusepath{fill}%
\end{pgfscope}%
\begin{pgfscope}%
\pgfpathrectangle{\pgfqpoint{0.150000in}{0.150000in}}{\pgfqpoint{2.700000in}{1.950000in}}%
\pgfusepath{clip}%
\pgfsetbuttcap%
\pgfsetroundjoin%
\definecolor{currentfill}{rgb}{0.899326,0.817325,0.823820}%
\pgfsetfillcolor{currentfill}%
\pgfsetlinewidth{0.000000pt}%
\definecolor{currentstroke}{rgb}{0.000000,0.000000,0.000000}%
\pgfsetstrokecolor{currentstroke}%
\pgfsetdash{}{0pt}%
\pgfpathmoveto{\pgfqpoint{1.572672in}{1.365882in}}%
\pgfpathlineto{\pgfqpoint{1.608731in}{1.383953in}}%
\pgfpathlineto{\pgfqpoint{1.572546in}{1.401962in}}%
\pgfpathlineto{\pgfqpoint{1.536486in}{1.383953in}}%
\pgfpathclose%
\pgfusepath{fill}%
\end{pgfscope}%
\begin{pgfscope}%
\pgfpathrectangle{\pgfqpoint{0.150000in}{0.150000in}}{\pgfqpoint{2.700000in}{1.950000in}}%
\pgfusepath{clip}%
\pgfsetbuttcap%
\pgfsetroundjoin%
\definecolor{currentfill}{rgb}{0.899326,0.817325,0.823820}%
\pgfsetfillcolor{currentfill}%
\pgfsetlinewidth{0.000000pt}%
\definecolor{currentstroke}{rgb}{0.000000,0.000000,0.000000}%
\pgfsetstrokecolor{currentstroke}%
\pgfsetdash{}{0pt}%
\pgfpathmoveto{\pgfqpoint{1.500301in}{1.365882in}}%
\pgfpathlineto{\pgfqpoint{1.536486in}{1.383953in}}%
\pgfpathlineto{\pgfqpoint{1.500427in}{1.401962in}}%
\pgfpathlineto{\pgfqpoint{1.464242in}{1.383953in}}%
\pgfpathclose%
\pgfusepath{fill}%
\end{pgfscope}%
\begin{pgfscope}%
\pgfpathrectangle{\pgfqpoint{0.150000in}{0.150000in}}{\pgfqpoint{2.700000in}{1.950000in}}%
\pgfusepath{clip}%
\pgfsetbuttcap%
\pgfsetroundjoin%
\definecolor{currentfill}{rgb}{0.899326,0.817325,0.823820}%
\pgfsetfillcolor{currentfill}%
\pgfsetlinewidth{0.000000pt}%
\definecolor{currentstroke}{rgb}{0.000000,0.000000,0.000000}%
\pgfsetstrokecolor{currentstroke}%
\pgfsetdash{}{0pt}%
\pgfpathmoveto{\pgfqpoint{1.427930in}{1.365882in}}%
\pgfpathlineto{\pgfqpoint{1.464242in}{1.383953in}}%
\pgfpathlineto{\pgfqpoint{1.428309in}{1.401962in}}%
\pgfpathlineto{\pgfqpoint{1.391998in}{1.383953in}}%
\pgfpathclose%
\pgfusepath{fill}%
\end{pgfscope}%
\begin{pgfscope}%
\pgfpathrectangle{\pgfqpoint{0.150000in}{0.150000in}}{\pgfqpoint{2.700000in}{1.950000in}}%
\pgfusepath{clip}%
\pgfsetbuttcap%
\pgfsetroundjoin%
\definecolor{currentfill}{rgb}{0.899326,0.817325,0.823820}%
\pgfsetfillcolor{currentfill}%
\pgfsetlinewidth{0.000000pt}%
\definecolor{currentstroke}{rgb}{0.000000,0.000000,0.000000}%
\pgfsetstrokecolor{currentstroke}%
\pgfsetdash{}{0pt}%
\pgfpathmoveto{\pgfqpoint{1.355559in}{1.365882in}}%
\pgfpathlineto{\pgfqpoint{1.391998in}{1.383953in}}%
\pgfpathlineto{\pgfqpoint{1.356191in}{1.401962in}}%
\pgfpathlineto{\pgfqpoint{1.319754in}{1.383953in}}%
\pgfpathclose%
\pgfusepath{fill}%
\end{pgfscope}%
\begin{pgfscope}%
\pgfpathrectangle{\pgfqpoint{0.150000in}{0.150000in}}{\pgfqpoint{2.700000in}{1.950000in}}%
\pgfusepath{clip}%
\pgfsetbuttcap%
\pgfsetroundjoin%
\definecolor{currentfill}{rgb}{0.899326,0.817325,0.823820}%
\pgfsetfillcolor{currentfill}%
\pgfsetlinewidth{0.000000pt}%
\definecolor{currentstroke}{rgb}{0.000000,0.000000,0.000000}%
\pgfsetstrokecolor{currentstroke}%
\pgfsetdash{}{0pt}%
\pgfpathmoveto{\pgfqpoint{1.283188in}{1.365882in}}%
\pgfpathlineto{\pgfqpoint{1.319754in}{1.383953in}}%
\pgfpathlineto{\pgfqpoint{1.284073in}{1.401962in}}%
\pgfpathlineto{\pgfqpoint{1.247509in}{1.383953in}}%
\pgfpathclose%
\pgfusepath{fill}%
\end{pgfscope}%
\begin{pgfscope}%
\pgfpathrectangle{\pgfqpoint{0.150000in}{0.150000in}}{\pgfqpoint{2.700000in}{1.950000in}}%
\pgfusepath{clip}%
\pgfsetbuttcap%
\pgfsetroundjoin%
\definecolor{currentfill}{rgb}{0.694179,0.445083,0.464813}%
\pgfsetfillcolor{currentfill}%
\pgfsetlinewidth{0.000000pt}%
\definecolor{currentstroke}{rgb}{0.000000,0.000000,0.000000}%
\pgfsetstrokecolor{currentstroke}%
\pgfsetdash{}{0pt}%
\pgfpathmoveto{\pgfqpoint{2.115159in}{0.954058in}}%
\pgfpathlineto{\pgfqpoint{2.149354in}{0.957703in}}%
\pgfpathlineto{\pgfqpoint{2.115251in}{1.029598in}}%
\pgfpathlineto{\pgfqpoint{2.080830in}{1.026172in}}%
\pgfpathclose%
\pgfusepath{fill}%
\end{pgfscope}%
\begin{pgfscope}%
\pgfpathrectangle{\pgfqpoint{0.150000in}{0.150000in}}{\pgfqpoint{2.700000in}{1.950000in}}%
\pgfusepath{clip}%
\pgfsetbuttcap%
\pgfsetroundjoin%
\definecolor{currentfill}{rgb}{0.739767,0.527803,0.544593}%
\pgfsetfillcolor{currentfill}%
\pgfsetlinewidth{0.000000pt}%
\definecolor{currentstroke}{rgb}{0.000000,0.000000,0.000000}%
\pgfsetstrokecolor{currentstroke}%
\pgfsetdash{}{0pt}%
\pgfpathmoveto{\pgfqpoint{2.115251in}{1.029598in}}%
\pgfpathlineto{\pgfqpoint{2.149451in}{1.033001in}}%
\pgfpathlineto{\pgfqpoint{2.115768in}{1.112860in}}%
\pgfpathlineto{\pgfqpoint{2.080917in}{1.101979in}}%
\pgfpathclose%
\pgfusepath{fill}%
\end{pgfscope}%
\begin{pgfscope}%
\pgfpathrectangle{\pgfqpoint{0.150000in}{0.150000in}}{\pgfqpoint{2.700000in}{1.950000in}}%
\pgfusepath{clip}%
\pgfsetbuttcap%
\pgfsetroundjoin%
\definecolor{currentfill}{rgb}{0.899326,0.817325,0.823820}%
\pgfsetfillcolor{currentfill}%
\pgfsetlinewidth{0.000000pt}%
\definecolor{currentstroke}{rgb}{0.000000,0.000000,0.000000}%
\pgfsetstrokecolor{currentstroke}%
\pgfsetdash{}{0pt}%
\pgfpathmoveto{\pgfqpoint{1.826479in}{1.347746in}}%
\pgfpathlineto{\pgfqpoint{1.862156in}{1.365882in}}%
\pgfpathlineto{\pgfqpoint{1.825464in}{1.383953in}}%
\pgfpathlineto{\pgfqpoint{1.789785in}{1.365882in}}%
\pgfpathclose%
\pgfusepath{fill}%
\end{pgfscope}%
\begin{pgfscope}%
\pgfpathrectangle{\pgfqpoint{0.150000in}{0.150000in}}{\pgfqpoint{2.700000in}{1.950000in}}%
\pgfusepath{clip}%
\pgfsetbuttcap%
\pgfsetroundjoin%
\definecolor{currentfill}{rgb}{0.899326,0.817325,0.823820}%
\pgfsetfillcolor{currentfill}%
\pgfsetlinewidth{0.000000pt}%
\definecolor{currentstroke}{rgb}{0.000000,0.000000,0.000000}%
\pgfsetstrokecolor{currentstroke}%
\pgfsetdash{}{0pt}%
\pgfpathmoveto{\pgfqpoint{1.753981in}{1.347746in}}%
\pgfpathlineto{\pgfqpoint{1.789785in}{1.365882in}}%
\pgfpathlineto{\pgfqpoint{1.753219in}{1.383953in}}%
\pgfpathlineto{\pgfqpoint{1.717414in}{1.365882in}}%
\pgfpathclose%
\pgfusepath{fill}%
\end{pgfscope}%
\begin{pgfscope}%
\pgfpathrectangle{\pgfqpoint{0.150000in}{0.150000in}}{\pgfqpoint{2.700000in}{1.950000in}}%
\pgfusepath{clip}%
\pgfsetbuttcap%
\pgfsetroundjoin%
\definecolor{currentfill}{rgb}{0.899326,0.817325,0.823820}%
\pgfsetfillcolor{currentfill}%
\pgfsetlinewidth{0.000000pt}%
\definecolor{currentstroke}{rgb}{0.000000,0.000000,0.000000}%
\pgfsetstrokecolor{currentstroke}%
\pgfsetdash{}{0pt}%
\pgfpathmoveto{\pgfqpoint{1.681483in}{1.347746in}}%
\pgfpathlineto{\pgfqpoint{1.717414in}{1.365882in}}%
\pgfpathlineto{\pgfqpoint{1.680975in}{1.383953in}}%
\pgfpathlineto{\pgfqpoint{1.645043in}{1.365882in}}%
\pgfpathclose%
\pgfusepath{fill}%
\end{pgfscope}%
\begin{pgfscope}%
\pgfpathrectangle{\pgfqpoint{0.150000in}{0.150000in}}{\pgfqpoint{2.700000in}{1.950000in}}%
\pgfusepath{clip}%
\pgfsetbuttcap%
\pgfsetroundjoin%
\definecolor{currentfill}{rgb}{0.899326,0.817325,0.823820}%
\pgfsetfillcolor{currentfill}%
\pgfsetlinewidth{0.000000pt}%
\definecolor{currentstroke}{rgb}{0.000000,0.000000,0.000000}%
\pgfsetstrokecolor{currentstroke}%
\pgfsetdash{}{0pt}%
\pgfpathmoveto{\pgfqpoint{1.608985in}{1.347746in}}%
\pgfpathlineto{\pgfqpoint{1.645043in}{1.365882in}}%
\pgfpathlineto{\pgfqpoint{1.608731in}{1.383953in}}%
\pgfpathlineto{\pgfqpoint{1.572672in}{1.365882in}}%
\pgfpathclose%
\pgfusepath{fill}%
\end{pgfscope}%
\begin{pgfscope}%
\pgfpathrectangle{\pgfqpoint{0.150000in}{0.150000in}}{\pgfqpoint{2.700000in}{1.950000in}}%
\pgfusepath{clip}%
\pgfsetbuttcap%
\pgfsetroundjoin%
\definecolor{currentfill}{rgb}{0.899326,0.817325,0.823820}%
\pgfsetfillcolor{currentfill}%
\pgfsetlinewidth{0.000000pt}%
\definecolor{currentstroke}{rgb}{0.000000,0.000000,0.000000}%
\pgfsetstrokecolor{currentstroke}%
\pgfsetdash{}{0pt}%
\pgfpathmoveto{\pgfqpoint{1.536486in}{1.347746in}}%
\pgfpathlineto{\pgfqpoint{1.572672in}{1.365882in}}%
\pgfpathlineto{\pgfqpoint{1.536486in}{1.383953in}}%
\pgfpathlineto{\pgfqpoint{1.500301in}{1.365882in}}%
\pgfpathclose%
\pgfusepath{fill}%
\end{pgfscope}%
\begin{pgfscope}%
\pgfpathrectangle{\pgfqpoint{0.150000in}{0.150000in}}{\pgfqpoint{2.700000in}{1.950000in}}%
\pgfusepath{clip}%
\pgfsetbuttcap%
\pgfsetroundjoin%
\definecolor{currentfill}{rgb}{0.899326,0.817325,0.823820}%
\pgfsetfillcolor{currentfill}%
\pgfsetlinewidth{0.000000pt}%
\definecolor{currentstroke}{rgb}{0.000000,0.000000,0.000000}%
\pgfsetstrokecolor{currentstroke}%
\pgfsetdash{}{0pt}%
\pgfpathmoveto{\pgfqpoint{1.463988in}{1.347746in}}%
\pgfpathlineto{\pgfqpoint{1.500301in}{1.365882in}}%
\pgfpathlineto{\pgfqpoint{1.464242in}{1.383953in}}%
\pgfpathlineto{\pgfqpoint{1.427930in}{1.365882in}}%
\pgfpathclose%
\pgfusepath{fill}%
\end{pgfscope}%
\begin{pgfscope}%
\pgfpathrectangle{\pgfqpoint{0.150000in}{0.150000in}}{\pgfqpoint{2.700000in}{1.950000in}}%
\pgfusepath{clip}%
\pgfsetbuttcap%
\pgfsetroundjoin%
\definecolor{currentfill}{rgb}{0.899326,0.817325,0.823820}%
\pgfsetfillcolor{currentfill}%
\pgfsetlinewidth{0.000000pt}%
\definecolor{currentstroke}{rgb}{0.000000,0.000000,0.000000}%
\pgfsetstrokecolor{currentstroke}%
\pgfsetdash{}{0pt}%
\pgfpathmoveto{\pgfqpoint{1.391490in}{1.347746in}}%
\pgfpathlineto{\pgfqpoint{1.427930in}{1.365882in}}%
\pgfpathlineto{\pgfqpoint{1.391998in}{1.383953in}}%
\pgfpathlineto{\pgfqpoint{1.355559in}{1.365882in}}%
\pgfpathclose%
\pgfusepath{fill}%
\end{pgfscope}%
\begin{pgfscope}%
\pgfpathrectangle{\pgfqpoint{0.150000in}{0.150000in}}{\pgfqpoint{2.700000in}{1.950000in}}%
\pgfusepath{clip}%
\pgfsetbuttcap%
\pgfsetroundjoin%
\definecolor{currentfill}{rgb}{0.899326,0.817325,0.823820}%
\pgfsetfillcolor{currentfill}%
\pgfsetlinewidth{0.000000pt}%
\definecolor{currentstroke}{rgb}{0.000000,0.000000,0.000000}%
\pgfsetstrokecolor{currentstroke}%
\pgfsetdash{}{0pt}%
\pgfpathmoveto{\pgfqpoint{1.318992in}{1.347746in}}%
\pgfpathlineto{\pgfqpoint{1.355559in}{1.365882in}}%
\pgfpathlineto{\pgfqpoint{1.319754in}{1.383953in}}%
\pgfpathlineto{\pgfqpoint{1.283188in}{1.365882in}}%
\pgfpathclose%
\pgfusepath{fill}%
\end{pgfscope}%
\begin{pgfscope}%
\pgfpathrectangle{\pgfqpoint{0.150000in}{0.150000in}}{\pgfqpoint{2.700000in}{1.950000in}}%
\pgfusepath{clip}%
\pgfsetbuttcap%
\pgfsetroundjoin%
\definecolor{currentfill}{rgb}{0.899326,0.817325,0.823820}%
\pgfsetfillcolor{currentfill}%
\pgfsetlinewidth{0.000000pt}%
\definecolor{currentstroke}{rgb}{0.000000,0.000000,0.000000}%
\pgfsetstrokecolor{currentstroke}%
\pgfsetdash{}{0pt}%
\pgfpathmoveto{\pgfqpoint{1.246494in}{1.347746in}}%
\pgfpathlineto{\pgfqpoint{1.283188in}{1.365882in}}%
\pgfpathlineto{\pgfqpoint{1.247509in}{1.383953in}}%
\pgfpathlineto{\pgfqpoint{1.210817in}{1.365882in}}%
\pgfpathclose%
\pgfusepath{fill}%
\end{pgfscope}%
\begin{pgfscope}%
\pgfpathrectangle{\pgfqpoint{0.150000in}{0.150000in}}{\pgfqpoint{2.700000in}{1.950000in}}%
\pgfusepath{clip}%
\pgfsetbuttcap%
\pgfsetroundjoin%
\definecolor{currentfill}{rgb}{0.663787,0.389936,0.411627}%
\pgfsetfillcolor{currentfill}%
\pgfsetlinewidth{0.000000pt}%
\definecolor{currentstroke}{rgb}{0.000000,0.000000,0.000000}%
\pgfsetstrokecolor{currentstroke}%
\pgfsetdash{}{0pt}%
\pgfpathmoveto{\pgfqpoint{2.191227in}{0.939078in}}%
\pgfpathlineto{\pgfqpoint{2.225372in}{0.942793in}}%
\pgfpathlineto{\pgfqpoint{2.184644in}{0.908781in}}%
\pgfpathlineto{\pgfqpoint{2.151052in}{0.912522in}}%
\pgfpathclose%
\pgfusepath{fill}%
\end{pgfscope}%
\begin{pgfscope}%
\pgfpathrectangle{\pgfqpoint{0.150000in}{0.150000in}}{\pgfqpoint{2.700000in}{1.950000in}}%
\pgfusepath{clip}%
\pgfsetbuttcap%
\pgfsetroundjoin%
\definecolor{currentfill}{rgb}{0.671385,0.403722,0.424923}%
\pgfsetfillcolor{currentfill}%
\pgfsetlinewidth{0.000000pt}%
\definecolor{currentstroke}{rgb}{0.000000,0.000000,0.000000}%
\pgfsetstrokecolor{currentstroke}%
\pgfsetdash{}{0pt}%
\pgfpathmoveto{\pgfqpoint{2.117194in}{0.916294in}}%
\pgfpathlineto{\pgfqpoint{2.151052in}{0.912522in}}%
\pgfpathlineto{\pgfqpoint{2.115159in}{0.954058in}}%
\pgfpathlineto{\pgfqpoint{2.081144in}{0.957995in}}%
\pgfpathclose%
\pgfusepath{fill}%
\end{pgfscope}%
\begin{pgfscope}%
\pgfpathrectangle{\pgfqpoint{0.150000in}{0.150000in}}{\pgfqpoint{2.700000in}{1.950000in}}%
\pgfusepath{clip}%
\pgfsetbuttcap%
\pgfsetroundjoin%
\definecolor{currentfill}{rgb}{0.899326,0.817325,0.823820}%
\pgfsetfillcolor{currentfill}%
\pgfsetlinewidth{0.000000pt}%
\definecolor{currentstroke}{rgb}{0.000000,0.000000,0.000000}%
\pgfsetstrokecolor{currentstroke}%
\pgfsetdash{}{0pt}%
\pgfpathmoveto{\pgfqpoint{1.863302in}{1.329547in}}%
\pgfpathlineto{\pgfqpoint{1.898977in}{1.347746in}}%
\pgfpathlineto{\pgfqpoint{1.862156in}{1.365882in}}%
\pgfpathlineto{\pgfqpoint{1.826479in}{1.347746in}}%
\pgfpathclose%
\pgfusepath{fill}%
\end{pgfscope}%
\begin{pgfscope}%
\pgfpathrectangle{\pgfqpoint{0.150000in}{0.150000in}}{\pgfqpoint{2.700000in}{1.950000in}}%
\pgfusepath{clip}%
\pgfsetbuttcap%
\pgfsetroundjoin%
\definecolor{currentfill}{rgb}{0.899326,0.817325,0.823820}%
\pgfsetfillcolor{currentfill}%
\pgfsetlinewidth{0.000000pt}%
\definecolor{currentstroke}{rgb}{0.000000,0.000000,0.000000}%
\pgfsetstrokecolor{currentstroke}%
\pgfsetdash{}{0pt}%
\pgfpathmoveto{\pgfqpoint{1.790676in}{1.329547in}}%
\pgfpathlineto{\pgfqpoint{1.826479in}{1.347746in}}%
\pgfpathlineto{\pgfqpoint{1.789785in}{1.365882in}}%
\pgfpathlineto{\pgfqpoint{1.753981in}{1.347746in}}%
\pgfpathclose%
\pgfusepath{fill}%
\end{pgfscope}%
\begin{pgfscope}%
\pgfpathrectangle{\pgfqpoint{0.150000in}{0.150000in}}{\pgfqpoint{2.700000in}{1.950000in}}%
\pgfusepath{clip}%
\pgfsetbuttcap%
\pgfsetroundjoin%
\definecolor{currentfill}{rgb}{0.899326,0.817325,0.823820}%
\pgfsetfillcolor{currentfill}%
\pgfsetlinewidth{0.000000pt}%
\definecolor{currentstroke}{rgb}{0.000000,0.000000,0.000000}%
\pgfsetstrokecolor{currentstroke}%
\pgfsetdash{}{0pt}%
\pgfpathmoveto{\pgfqpoint{1.718051in}{1.329547in}}%
\pgfpathlineto{\pgfqpoint{1.753981in}{1.347746in}}%
\pgfpathlineto{\pgfqpoint{1.717414in}{1.365882in}}%
\pgfpathlineto{\pgfqpoint{1.681483in}{1.347746in}}%
\pgfpathclose%
\pgfusepath{fill}%
\end{pgfscope}%
\begin{pgfscope}%
\pgfpathrectangle{\pgfqpoint{0.150000in}{0.150000in}}{\pgfqpoint{2.700000in}{1.950000in}}%
\pgfusepath{clip}%
\pgfsetbuttcap%
\pgfsetroundjoin%
\definecolor{currentfill}{rgb}{0.899326,0.817325,0.823820}%
\pgfsetfillcolor{currentfill}%
\pgfsetlinewidth{0.000000pt}%
\definecolor{currentstroke}{rgb}{0.000000,0.000000,0.000000}%
\pgfsetstrokecolor{currentstroke}%
\pgfsetdash{}{0pt}%
\pgfpathmoveto{\pgfqpoint{1.645425in}{1.329547in}}%
\pgfpathlineto{\pgfqpoint{1.681483in}{1.347746in}}%
\pgfpathlineto{\pgfqpoint{1.645043in}{1.365882in}}%
\pgfpathlineto{\pgfqpoint{1.608985in}{1.347746in}}%
\pgfpathclose%
\pgfusepath{fill}%
\end{pgfscope}%
\begin{pgfscope}%
\pgfpathrectangle{\pgfqpoint{0.150000in}{0.150000in}}{\pgfqpoint{2.700000in}{1.950000in}}%
\pgfusepath{clip}%
\pgfsetbuttcap%
\pgfsetroundjoin%
\definecolor{currentfill}{rgb}{0.899326,0.817325,0.823820}%
\pgfsetfillcolor{currentfill}%
\pgfsetlinewidth{0.000000pt}%
\definecolor{currentstroke}{rgb}{0.000000,0.000000,0.000000}%
\pgfsetstrokecolor{currentstroke}%
\pgfsetdash{}{0pt}%
\pgfpathmoveto{\pgfqpoint{1.572799in}{1.329547in}}%
\pgfpathlineto{\pgfqpoint{1.608985in}{1.347746in}}%
\pgfpathlineto{\pgfqpoint{1.572672in}{1.365882in}}%
\pgfpathlineto{\pgfqpoint{1.536486in}{1.347746in}}%
\pgfpathclose%
\pgfusepath{fill}%
\end{pgfscope}%
\begin{pgfscope}%
\pgfpathrectangle{\pgfqpoint{0.150000in}{0.150000in}}{\pgfqpoint{2.700000in}{1.950000in}}%
\pgfusepath{clip}%
\pgfsetbuttcap%
\pgfsetroundjoin%
\definecolor{currentfill}{rgb}{0.899326,0.817325,0.823820}%
\pgfsetfillcolor{currentfill}%
\pgfsetlinewidth{0.000000pt}%
\definecolor{currentstroke}{rgb}{0.000000,0.000000,0.000000}%
\pgfsetstrokecolor{currentstroke}%
\pgfsetdash{}{0pt}%
\pgfpathmoveto{\pgfqpoint{1.500174in}{1.329547in}}%
\pgfpathlineto{\pgfqpoint{1.536486in}{1.347746in}}%
\pgfpathlineto{\pgfqpoint{1.500301in}{1.365882in}}%
\pgfpathlineto{\pgfqpoint{1.463988in}{1.347746in}}%
\pgfpathclose%
\pgfusepath{fill}%
\end{pgfscope}%
\begin{pgfscope}%
\pgfpathrectangle{\pgfqpoint{0.150000in}{0.150000in}}{\pgfqpoint{2.700000in}{1.950000in}}%
\pgfusepath{clip}%
\pgfsetbuttcap%
\pgfsetroundjoin%
\definecolor{currentfill}{rgb}{0.899326,0.817325,0.823820}%
\pgfsetfillcolor{currentfill}%
\pgfsetlinewidth{0.000000pt}%
\definecolor{currentstroke}{rgb}{0.000000,0.000000,0.000000}%
\pgfsetstrokecolor{currentstroke}%
\pgfsetdash{}{0pt}%
\pgfpathmoveto{\pgfqpoint{1.427548in}{1.329547in}}%
\pgfpathlineto{\pgfqpoint{1.463988in}{1.347746in}}%
\pgfpathlineto{\pgfqpoint{1.427930in}{1.365882in}}%
\pgfpathlineto{\pgfqpoint{1.391490in}{1.347746in}}%
\pgfpathclose%
\pgfusepath{fill}%
\end{pgfscope}%
\begin{pgfscope}%
\pgfpathrectangle{\pgfqpoint{0.150000in}{0.150000in}}{\pgfqpoint{2.700000in}{1.950000in}}%
\pgfusepath{clip}%
\pgfsetbuttcap%
\pgfsetroundjoin%
\definecolor{currentfill}{rgb}{0.899326,0.817325,0.823820}%
\pgfsetfillcolor{currentfill}%
\pgfsetlinewidth{0.000000pt}%
\definecolor{currentstroke}{rgb}{0.000000,0.000000,0.000000}%
\pgfsetstrokecolor{currentstroke}%
\pgfsetdash{}{0pt}%
\pgfpathmoveto{\pgfqpoint{1.354922in}{1.329547in}}%
\pgfpathlineto{\pgfqpoint{1.391490in}{1.347746in}}%
\pgfpathlineto{\pgfqpoint{1.355559in}{1.365882in}}%
\pgfpathlineto{\pgfqpoint{1.318992in}{1.347746in}}%
\pgfpathclose%
\pgfusepath{fill}%
\end{pgfscope}%
\begin{pgfscope}%
\pgfpathrectangle{\pgfqpoint{0.150000in}{0.150000in}}{\pgfqpoint{2.700000in}{1.950000in}}%
\pgfusepath{clip}%
\pgfsetbuttcap%
\pgfsetroundjoin%
\definecolor{currentfill}{rgb}{0.899326,0.817325,0.823820}%
\pgfsetfillcolor{currentfill}%
\pgfsetlinewidth{0.000000pt}%
\definecolor{currentstroke}{rgb}{0.000000,0.000000,0.000000}%
\pgfsetstrokecolor{currentstroke}%
\pgfsetdash{}{0pt}%
\pgfpathmoveto{\pgfqpoint{1.282297in}{1.329547in}}%
\pgfpathlineto{\pgfqpoint{1.318992in}{1.347746in}}%
\pgfpathlineto{\pgfqpoint{1.283188in}{1.365882in}}%
\pgfpathlineto{\pgfqpoint{1.246494in}{1.347746in}}%
\pgfpathclose%
\pgfusepath{fill}%
\end{pgfscope}%
\begin{pgfscope}%
\pgfpathrectangle{\pgfqpoint{0.150000in}{0.150000in}}{\pgfqpoint{2.700000in}{1.950000in}}%
\pgfusepath{clip}%
\pgfsetbuttcap%
\pgfsetroundjoin%
\definecolor{currentfill}{rgb}{0.899326,0.817325,0.823820}%
\pgfsetfillcolor{currentfill}%
\pgfsetlinewidth{0.000000pt}%
\definecolor{currentstroke}{rgb}{0.000000,0.000000,0.000000}%
\pgfsetstrokecolor{currentstroke}%
\pgfsetdash{}{0pt}%
\pgfpathmoveto{\pgfqpoint{1.209671in}{1.329547in}}%
\pgfpathlineto{\pgfqpoint{1.246494in}{1.347746in}}%
\pgfpathlineto{\pgfqpoint{1.210817in}{1.365882in}}%
\pgfpathlineto{\pgfqpoint{1.173996in}{1.347746in}}%
\pgfpathclose%
\pgfusepath{fill}%
\end{pgfscope}%
\begin{pgfscope}%
\pgfpathrectangle{\pgfqpoint{0.150000in}{0.150000in}}{\pgfqpoint{2.700000in}{1.950000in}}%
\pgfusepath{clip}%
\pgfsetbuttcap%
\pgfsetroundjoin%
\definecolor{currentfill}{rgb}{0.751164,0.548483,0.564537}%
\pgfsetfillcolor{currentfill}%
\pgfsetlinewidth{0.000000pt}%
\definecolor{currentstroke}{rgb}{0.000000,0.000000,0.000000}%
\pgfsetstrokecolor{currentstroke}%
\pgfsetdash{}{0pt}%
\pgfpathmoveto{\pgfqpoint{2.080830in}{1.026172in}}%
\pgfpathlineto{\pgfqpoint{2.115251in}{1.029598in}}%
\pgfpathlineto{\pgfqpoint{2.080917in}{1.101979in}}%
\pgfpathlineto{\pgfqpoint{2.045892in}{1.091045in}}%
\pgfpathclose%
\pgfusepath{fill}%
\end{pgfscope}%
\begin{pgfscope}%
\pgfpathrectangle{\pgfqpoint{0.150000in}{0.150000in}}{\pgfqpoint{2.700000in}{1.950000in}}%
\pgfusepath{clip}%
\pgfsetbuttcap%
\pgfsetroundjoin%
\definecolor{currentfill}{rgb}{0.713174,0.479550,0.498055}%
\pgfsetfillcolor{currentfill}%
\pgfsetlinewidth{0.000000pt}%
\definecolor{currentstroke}{rgb}{0.000000,0.000000,0.000000}%
\pgfsetstrokecolor{currentstroke}%
\pgfsetdash{}{0pt}%
\pgfpathmoveto{\pgfqpoint{2.081144in}{0.957995in}}%
\pgfpathlineto{\pgfqpoint{2.115159in}{0.954058in}}%
\pgfpathlineto{\pgfqpoint{2.080830in}{1.026172in}}%
\pgfpathlineto{\pgfqpoint{2.046187in}{1.022724in}}%
\pgfpathclose%
\pgfusepath{fill}%
\end{pgfscope}%
\begin{pgfscope}%
\pgfpathrectangle{\pgfqpoint{0.150000in}{0.150000in}}{\pgfqpoint{2.700000in}{1.950000in}}%
\pgfusepath{clip}%
\pgfsetbuttcap%
\pgfsetroundjoin%
\definecolor{currentfill}{rgb}{0.899326,0.817325,0.823820}%
\pgfsetfillcolor{currentfill}%
\pgfsetlinewidth{0.000000pt}%
\definecolor{currentstroke}{rgb}{0.000000,0.000000,0.000000}%
\pgfsetstrokecolor{currentstroke}%
\pgfsetdash{}{0pt}%
\pgfpathmoveto{\pgfqpoint{1.900255in}{1.311284in}}%
\pgfpathlineto{\pgfqpoint{1.935927in}{1.329547in}}%
\pgfpathlineto{\pgfqpoint{1.898977in}{1.347746in}}%
\pgfpathlineto{\pgfqpoint{1.863302in}{1.329547in}}%
\pgfpathclose%
\pgfusepath{fill}%
\end{pgfscope}%
\begin{pgfscope}%
\pgfpathrectangle{\pgfqpoint{0.150000in}{0.150000in}}{\pgfqpoint{2.700000in}{1.950000in}}%
\pgfusepath{clip}%
\pgfsetbuttcap%
\pgfsetroundjoin%
\definecolor{currentfill}{rgb}{0.899326,0.817325,0.823820}%
\pgfsetfillcolor{currentfill}%
\pgfsetlinewidth{0.000000pt}%
\definecolor{currentstroke}{rgb}{0.000000,0.000000,0.000000}%
\pgfsetstrokecolor{currentstroke}%
\pgfsetdash{}{0pt}%
\pgfpathmoveto{\pgfqpoint{1.827501in}{1.311284in}}%
\pgfpathlineto{\pgfqpoint{1.863302in}{1.329547in}}%
\pgfpathlineto{\pgfqpoint{1.826479in}{1.347746in}}%
\pgfpathlineto{\pgfqpoint{1.790676in}{1.329547in}}%
\pgfpathclose%
\pgfusepath{fill}%
\end{pgfscope}%
\begin{pgfscope}%
\pgfpathrectangle{\pgfqpoint{0.150000in}{0.150000in}}{\pgfqpoint{2.700000in}{1.950000in}}%
\pgfusepath{clip}%
\pgfsetbuttcap%
\pgfsetroundjoin%
\definecolor{currentfill}{rgb}{0.899326,0.817325,0.823820}%
\pgfsetfillcolor{currentfill}%
\pgfsetlinewidth{0.000000pt}%
\definecolor{currentstroke}{rgb}{0.000000,0.000000,0.000000}%
\pgfsetstrokecolor{currentstroke}%
\pgfsetdash{}{0pt}%
\pgfpathmoveto{\pgfqpoint{1.754747in}{1.311284in}}%
\pgfpathlineto{\pgfqpoint{1.790676in}{1.329547in}}%
\pgfpathlineto{\pgfqpoint{1.753981in}{1.347746in}}%
\pgfpathlineto{\pgfqpoint{1.718051in}{1.329547in}}%
\pgfpathclose%
\pgfusepath{fill}%
\end{pgfscope}%
\begin{pgfscope}%
\pgfpathrectangle{\pgfqpoint{0.150000in}{0.150000in}}{\pgfqpoint{2.700000in}{1.950000in}}%
\pgfusepath{clip}%
\pgfsetbuttcap%
\pgfsetroundjoin%
\definecolor{currentfill}{rgb}{0.899326,0.817325,0.823820}%
\pgfsetfillcolor{currentfill}%
\pgfsetlinewidth{0.000000pt}%
\definecolor{currentstroke}{rgb}{0.000000,0.000000,0.000000}%
\pgfsetstrokecolor{currentstroke}%
\pgfsetdash{}{0pt}%
\pgfpathmoveto{\pgfqpoint{1.681994in}{1.311284in}}%
\pgfpathlineto{\pgfqpoint{1.718051in}{1.329547in}}%
\pgfpathlineto{\pgfqpoint{1.681483in}{1.347746in}}%
\pgfpathlineto{\pgfqpoint{1.645425in}{1.329547in}}%
\pgfpathclose%
\pgfusepath{fill}%
\end{pgfscope}%
\begin{pgfscope}%
\pgfpathrectangle{\pgfqpoint{0.150000in}{0.150000in}}{\pgfqpoint{2.700000in}{1.950000in}}%
\pgfusepath{clip}%
\pgfsetbuttcap%
\pgfsetroundjoin%
\definecolor{currentfill}{rgb}{0.899326,0.817325,0.823820}%
\pgfsetfillcolor{currentfill}%
\pgfsetlinewidth{0.000000pt}%
\definecolor{currentstroke}{rgb}{0.000000,0.000000,0.000000}%
\pgfsetstrokecolor{currentstroke}%
\pgfsetdash{}{0pt}%
\pgfpathmoveto{\pgfqpoint{1.609240in}{1.311284in}}%
\pgfpathlineto{\pgfqpoint{1.645425in}{1.329547in}}%
\pgfpathlineto{\pgfqpoint{1.608985in}{1.347746in}}%
\pgfpathlineto{\pgfqpoint{1.572799in}{1.329547in}}%
\pgfpathclose%
\pgfusepath{fill}%
\end{pgfscope}%
\begin{pgfscope}%
\pgfpathrectangle{\pgfqpoint{0.150000in}{0.150000in}}{\pgfqpoint{2.700000in}{1.950000in}}%
\pgfusepath{clip}%
\pgfsetbuttcap%
\pgfsetroundjoin%
\definecolor{currentfill}{rgb}{0.899326,0.817325,0.823820}%
\pgfsetfillcolor{currentfill}%
\pgfsetlinewidth{0.000000pt}%
\definecolor{currentstroke}{rgb}{0.000000,0.000000,0.000000}%
\pgfsetstrokecolor{currentstroke}%
\pgfsetdash{}{0pt}%
\pgfpathmoveto{\pgfqpoint{1.536486in}{1.311284in}}%
\pgfpathlineto{\pgfqpoint{1.572799in}{1.329547in}}%
\pgfpathlineto{\pgfqpoint{1.536486in}{1.347746in}}%
\pgfpathlineto{\pgfqpoint{1.500174in}{1.329547in}}%
\pgfpathclose%
\pgfusepath{fill}%
\end{pgfscope}%
\begin{pgfscope}%
\pgfpathrectangle{\pgfqpoint{0.150000in}{0.150000in}}{\pgfqpoint{2.700000in}{1.950000in}}%
\pgfusepath{clip}%
\pgfsetbuttcap%
\pgfsetroundjoin%
\definecolor{currentfill}{rgb}{0.899326,0.817325,0.823820}%
\pgfsetfillcolor{currentfill}%
\pgfsetlinewidth{0.000000pt}%
\definecolor{currentstroke}{rgb}{0.000000,0.000000,0.000000}%
\pgfsetstrokecolor{currentstroke}%
\pgfsetdash{}{0pt}%
\pgfpathmoveto{\pgfqpoint{1.463733in}{1.311284in}}%
\pgfpathlineto{\pgfqpoint{1.500174in}{1.329547in}}%
\pgfpathlineto{\pgfqpoint{1.463988in}{1.347746in}}%
\pgfpathlineto{\pgfqpoint{1.427548in}{1.329547in}}%
\pgfpathclose%
\pgfusepath{fill}%
\end{pgfscope}%
\begin{pgfscope}%
\pgfpathrectangle{\pgfqpoint{0.150000in}{0.150000in}}{\pgfqpoint{2.700000in}{1.950000in}}%
\pgfusepath{clip}%
\pgfsetbuttcap%
\pgfsetroundjoin%
\definecolor{currentfill}{rgb}{0.899326,0.817325,0.823820}%
\pgfsetfillcolor{currentfill}%
\pgfsetlinewidth{0.000000pt}%
\definecolor{currentstroke}{rgb}{0.000000,0.000000,0.000000}%
\pgfsetstrokecolor{currentstroke}%
\pgfsetdash{}{0pt}%
\pgfpathmoveto{\pgfqpoint{1.390979in}{1.311284in}}%
\pgfpathlineto{\pgfqpoint{1.427548in}{1.329547in}}%
\pgfpathlineto{\pgfqpoint{1.391490in}{1.347746in}}%
\pgfpathlineto{\pgfqpoint{1.354922in}{1.329547in}}%
\pgfpathclose%
\pgfusepath{fill}%
\end{pgfscope}%
\begin{pgfscope}%
\pgfpathrectangle{\pgfqpoint{0.150000in}{0.150000in}}{\pgfqpoint{2.700000in}{1.950000in}}%
\pgfusepath{clip}%
\pgfsetbuttcap%
\pgfsetroundjoin%
\definecolor{currentfill}{rgb}{0.899326,0.817325,0.823820}%
\pgfsetfillcolor{currentfill}%
\pgfsetlinewidth{0.000000pt}%
\definecolor{currentstroke}{rgb}{0.000000,0.000000,0.000000}%
\pgfsetstrokecolor{currentstroke}%
\pgfsetdash{}{0pt}%
\pgfpathmoveto{\pgfqpoint{1.318226in}{1.311284in}}%
\pgfpathlineto{\pgfqpoint{1.354922in}{1.329547in}}%
\pgfpathlineto{\pgfqpoint{1.318992in}{1.347746in}}%
\pgfpathlineto{\pgfqpoint{1.282297in}{1.329547in}}%
\pgfpathclose%
\pgfusepath{fill}%
\end{pgfscope}%
\begin{pgfscope}%
\pgfpathrectangle{\pgfqpoint{0.150000in}{0.150000in}}{\pgfqpoint{2.700000in}{1.950000in}}%
\pgfusepath{clip}%
\pgfsetbuttcap%
\pgfsetroundjoin%
\definecolor{currentfill}{rgb}{0.899326,0.817325,0.823820}%
\pgfsetfillcolor{currentfill}%
\pgfsetlinewidth{0.000000pt}%
\definecolor{currentstroke}{rgb}{0.000000,0.000000,0.000000}%
\pgfsetstrokecolor{currentstroke}%
\pgfsetdash{}{0pt}%
\pgfpathmoveto{\pgfqpoint{1.245472in}{1.311284in}}%
\pgfpathlineto{\pgfqpoint{1.282297in}{1.329547in}}%
\pgfpathlineto{\pgfqpoint{1.246494in}{1.347746in}}%
\pgfpathlineto{\pgfqpoint{1.209671in}{1.329547in}}%
\pgfpathclose%
\pgfusepath{fill}%
\end{pgfscope}%
\begin{pgfscope}%
\pgfpathrectangle{\pgfqpoint{0.150000in}{0.150000in}}{\pgfqpoint{2.700000in}{1.950000in}}%
\pgfusepath{clip}%
\pgfsetbuttcap%
\pgfsetroundjoin%
\definecolor{currentfill}{rgb}{0.899326,0.817325,0.823820}%
\pgfsetfillcolor{currentfill}%
\pgfsetlinewidth{0.000000pt}%
\definecolor{currentstroke}{rgb}{0.000000,0.000000,0.000000}%
\pgfsetstrokecolor{currentstroke}%
\pgfsetdash{}{0pt}%
\pgfpathmoveto{\pgfqpoint{1.172718in}{1.311284in}}%
\pgfpathlineto{\pgfqpoint{1.209671in}{1.329547in}}%
\pgfpathlineto{\pgfqpoint{1.173996in}{1.347746in}}%
\pgfpathlineto{\pgfqpoint{1.137046in}{1.329547in}}%
\pgfpathclose%
\pgfusepath{fill}%
\end{pgfscope}%
\begin{pgfscope}%
\pgfpathrectangle{\pgfqpoint{0.150000in}{0.150000in}}{\pgfqpoint{2.700000in}{1.950000in}}%
\pgfusepath{clip}%
\pgfsetbuttcap%
\pgfsetroundjoin%
\definecolor{currentfill}{rgb}{0.804350,0.644991,0.657613}%
\pgfsetfillcolor{currentfill}%
\pgfsetlinewidth{0.000000pt}%
\definecolor{currentstroke}{rgb}{0.000000,0.000000,0.000000}%
\pgfsetstrokecolor{currentstroke}%
\pgfsetdash{}{0pt}%
\pgfpathmoveto{\pgfqpoint{2.080917in}{1.101979in}}%
\pgfpathlineto{\pgfqpoint{2.115768in}{1.112860in}}%
\pgfpathlineto{\pgfqpoint{2.081807in}{1.193378in}}%
\pgfpathlineto{\pgfqpoint{2.046727in}{1.182654in}}%
\pgfpathclose%
\pgfusepath{fill}%
\end{pgfscope}%
\begin{pgfscope}%
\pgfpathrectangle{\pgfqpoint{0.150000in}{0.150000in}}{\pgfqpoint{2.700000in}{1.950000in}}%
\pgfusepath{clip}%
\pgfsetbuttcap%
\pgfsetroundjoin%
\definecolor{currentfill}{rgb}{0.686581,0.431296,0.451517}%
\pgfsetfillcolor{currentfill}%
\pgfsetlinewidth{0.000000pt}%
\definecolor{currentstroke}{rgb}{0.000000,0.000000,0.000000}%
\pgfsetstrokecolor{currentstroke}%
\pgfsetdash{}{0pt}%
\pgfpathmoveto{\pgfqpoint{2.157319in}{0.942979in}}%
\pgfpathlineto{\pgfqpoint{2.191227in}{0.939078in}}%
\pgfpathlineto{\pgfqpoint{2.151052in}{0.912522in}}%
\pgfpathlineto{\pgfqpoint{2.117194in}{0.916294in}}%
\pgfpathclose%
\pgfusepath{fill}%
\end{pgfscope}%
\begin{pgfscope}%
\pgfpathrectangle{\pgfqpoint{0.150000in}{0.150000in}}{\pgfqpoint{2.700000in}{1.950000in}}%
\pgfusepath{clip}%
\pgfsetbuttcap%
\pgfsetroundjoin%
\definecolor{currentfill}{rgb}{0.705576,0.465763,0.484758}%
\pgfsetfillcolor{currentfill}%
\pgfsetlinewidth{0.000000pt}%
\definecolor{currentstroke}{rgb}{0.000000,0.000000,0.000000}%
\pgfsetstrokecolor{currentstroke}%
\pgfsetdash{}{0pt}%
\pgfpathmoveto{\pgfqpoint{2.267207in}{0.984951in}}%
\pgfpathlineto{\pgfqpoint{2.301257in}{0.988534in}}%
\pgfpathlineto{\pgfqpoint{2.259298in}{0.946484in}}%
\pgfpathlineto{\pgfqpoint{2.225372in}{0.942793in}}%
\pgfpathclose%
\pgfusepath{fill}%
\end{pgfscope}%
\begin{pgfscope}%
\pgfpathrectangle{\pgfqpoint{0.150000in}{0.150000in}}{\pgfqpoint{2.700000in}{1.950000in}}%
\pgfusepath{clip}%
\pgfsetbuttcap%
\pgfsetroundjoin%
\definecolor{currentfill}{rgb}{0.694179,0.445083,0.464813}%
\pgfsetfillcolor{currentfill}%
\pgfsetlinewidth{0.000000pt}%
\definecolor{currentstroke}{rgb}{0.000000,0.000000,0.000000}%
\pgfsetstrokecolor{currentstroke}%
\pgfsetdash{}{0pt}%
\pgfpathmoveto{\pgfqpoint{2.082664in}{0.912489in}}%
\pgfpathlineto{\pgfqpoint{2.117194in}{0.916294in}}%
\pgfpathlineto{\pgfqpoint{2.081144in}{0.957995in}}%
\pgfpathlineto{\pgfqpoint{2.046481in}{0.954324in}}%
\pgfpathclose%
\pgfusepath{fill}%
\end{pgfscope}%
\begin{pgfscope}%
\pgfpathrectangle{\pgfqpoint{0.150000in}{0.150000in}}{\pgfqpoint{2.700000in}{1.950000in}}%
\pgfusepath{clip}%
\pgfsetbuttcap%
\pgfsetroundjoin%
\definecolor{currentfill}{rgb}{0.899326,0.817325,0.823820}%
\pgfsetfillcolor{currentfill}%
\pgfsetlinewidth{0.000000pt}%
\definecolor{currentstroke}{rgb}{0.000000,0.000000,0.000000}%
\pgfsetstrokecolor{currentstroke}%
\pgfsetdash{}{0pt}%
\pgfpathmoveto{\pgfqpoint{1.937338in}{1.292957in}}%
\pgfpathlineto{\pgfqpoint{1.973008in}{1.311284in}}%
\pgfpathlineto{\pgfqpoint{1.935927in}{1.329547in}}%
\pgfpathlineto{\pgfqpoint{1.900255in}{1.311284in}}%
\pgfpathclose%
\pgfusepath{fill}%
\end{pgfscope}%
\begin{pgfscope}%
\pgfpathrectangle{\pgfqpoint{0.150000in}{0.150000in}}{\pgfqpoint{2.700000in}{1.950000in}}%
\pgfusepath{clip}%
\pgfsetbuttcap%
\pgfsetroundjoin%
\definecolor{currentfill}{rgb}{0.899326,0.817325,0.823820}%
\pgfsetfillcolor{currentfill}%
\pgfsetlinewidth{0.000000pt}%
\definecolor{currentstroke}{rgb}{0.000000,0.000000,0.000000}%
\pgfsetstrokecolor{currentstroke}%
\pgfsetdash{}{0pt}%
\pgfpathmoveto{\pgfqpoint{1.864456in}{1.292957in}}%
\pgfpathlineto{\pgfqpoint{1.900255in}{1.311284in}}%
\pgfpathlineto{\pgfqpoint{1.863302in}{1.329547in}}%
\pgfpathlineto{\pgfqpoint{1.827501in}{1.311284in}}%
\pgfpathclose%
\pgfusepath{fill}%
\end{pgfscope}%
\begin{pgfscope}%
\pgfpathrectangle{\pgfqpoint{0.150000in}{0.150000in}}{\pgfqpoint{2.700000in}{1.950000in}}%
\pgfusepath{clip}%
\pgfsetbuttcap%
\pgfsetroundjoin%
\definecolor{currentfill}{rgb}{0.899326,0.817325,0.823820}%
\pgfsetfillcolor{currentfill}%
\pgfsetlinewidth{0.000000pt}%
\definecolor{currentstroke}{rgb}{0.000000,0.000000,0.000000}%
\pgfsetstrokecolor{currentstroke}%
\pgfsetdash{}{0pt}%
\pgfpathmoveto{\pgfqpoint{1.791574in}{1.292957in}}%
\pgfpathlineto{\pgfqpoint{1.827501in}{1.311284in}}%
\pgfpathlineto{\pgfqpoint{1.790676in}{1.329547in}}%
\pgfpathlineto{\pgfqpoint{1.754747in}{1.311284in}}%
\pgfpathclose%
\pgfusepath{fill}%
\end{pgfscope}%
\begin{pgfscope}%
\pgfpathrectangle{\pgfqpoint{0.150000in}{0.150000in}}{\pgfqpoint{2.700000in}{1.950000in}}%
\pgfusepath{clip}%
\pgfsetbuttcap%
\pgfsetroundjoin%
\definecolor{currentfill}{rgb}{0.899326,0.817325,0.823820}%
\pgfsetfillcolor{currentfill}%
\pgfsetlinewidth{0.000000pt}%
\definecolor{currentstroke}{rgb}{0.000000,0.000000,0.000000}%
\pgfsetstrokecolor{currentstroke}%
\pgfsetdash{}{0pt}%
\pgfpathmoveto{\pgfqpoint{1.718692in}{1.292957in}}%
\pgfpathlineto{\pgfqpoint{1.754747in}{1.311284in}}%
\pgfpathlineto{\pgfqpoint{1.718051in}{1.329547in}}%
\pgfpathlineto{\pgfqpoint{1.681994in}{1.311284in}}%
\pgfpathclose%
\pgfusepath{fill}%
\end{pgfscope}%
\begin{pgfscope}%
\pgfpathrectangle{\pgfqpoint{0.150000in}{0.150000in}}{\pgfqpoint{2.700000in}{1.950000in}}%
\pgfusepath{clip}%
\pgfsetbuttcap%
\pgfsetroundjoin%
\definecolor{currentfill}{rgb}{0.899326,0.817325,0.823820}%
\pgfsetfillcolor{currentfill}%
\pgfsetlinewidth{0.000000pt}%
\definecolor{currentstroke}{rgb}{0.000000,0.000000,0.000000}%
\pgfsetstrokecolor{currentstroke}%
\pgfsetdash{}{0pt}%
\pgfpathmoveto{\pgfqpoint{1.645810in}{1.292957in}}%
\pgfpathlineto{\pgfqpoint{1.681994in}{1.311284in}}%
\pgfpathlineto{\pgfqpoint{1.645425in}{1.329547in}}%
\pgfpathlineto{\pgfqpoint{1.609240in}{1.311284in}}%
\pgfpathclose%
\pgfusepath{fill}%
\end{pgfscope}%
\begin{pgfscope}%
\pgfpathrectangle{\pgfqpoint{0.150000in}{0.150000in}}{\pgfqpoint{2.700000in}{1.950000in}}%
\pgfusepath{clip}%
\pgfsetbuttcap%
\pgfsetroundjoin%
\definecolor{currentfill}{rgb}{0.899326,0.817325,0.823820}%
\pgfsetfillcolor{currentfill}%
\pgfsetlinewidth{0.000000pt}%
\definecolor{currentstroke}{rgb}{0.000000,0.000000,0.000000}%
\pgfsetstrokecolor{currentstroke}%
\pgfsetdash{}{0pt}%
\pgfpathmoveto{\pgfqpoint{1.572928in}{1.292957in}}%
\pgfpathlineto{\pgfqpoint{1.609240in}{1.311284in}}%
\pgfpathlineto{\pgfqpoint{1.572799in}{1.329547in}}%
\pgfpathlineto{\pgfqpoint{1.536486in}{1.311284in}}%
\pgfpathclose%
\pgfusepath{fill}%
\end{pgfscope}%
\begin{pgfscope}%
\pgfpathrectangle{\pgfqpoint{0.150000in}{0.150000in}}{\pgfqpoint{2.700000in}{1.950000in}}%
\pgfusepath{clip}%
\pgfsetbuttcap%
\pgfsetroundjoin%
\definecolor{currentfill}{rgb}{0.899326,0.817325,0.823820}%
\pgfsetfillcolor{currentfill}%
\pgfsetlinewidth{0.000000pt}%
\definecolor{currentstroke}{rgb}{0.000000,0.000000,0.000000}%
\pgfsetstrokecolor{currentstroke}%
\pgfsetdash{}{0pt}%
\pgfpathmoveto{\pgfqpoint{1.500045in}{1.292957in}}%
\pgfpathlineto{\pgfqpoint{1.536486in}{1.311284in}}%
\pgfpathlineto{\pgfqpoint{1.500174in}{1.329547in}}%
\pgfpathlineto{\pgfqpoint{1.463733in}{1.311284in}}%
\pgfpathclose%
\pgfusepath{fill}%
\end{pgfscope}%
\begin{pgfscope}%
\pgfpathrectangle{\pgfqpoint{0.150000in}{0.150000in}}{\pgfqpoint{2.700000in}{1.950000in}}%
\pgfusepath{clip}%
\pgfsetbuttcap%
\pgfsetroundjoin%
\definecolor{currentfill}{rgb}{0.899326,0.817325,0.823820}%
\pgfsetfillcolor{currentfill}%
\pgfsetlinewidth{0.000000pt}%
\definecolor{currentstroke}{rgb}{0.000000,0.000000,0.000000}%
\pgfsetstrokecolor{currentstroke}%
\pgfsetdash{}{0pt}%
\pgfpathmoveto{\pgfqpoint{1.427163in}{1.292957in}}%
\pgfpathlineto{\pgfqpoint{1.463733in}{1.311284in}}%
\pgfpathlineto{\pgfqpoint{1.427548in}{1.329547in}}%
\pgfpathlineto{\pgfqpoint{1.390979in}{1.311284in}}%
\pgfpathclose%
\pgfusepath{fill}%
\end{pgfscope}%
\begin{pgfscope}%
\pgfpathrectangle{\pgfqpoint{0.150000in}{0.150000in}}{\pgfqpoint{2.700000in}{1.950000in}}%
\pgfusepath{clip}%
\pgfsetbuttcap%
\pgfsetroundjoin%
\definecolor{currentfill}{rgb}{0.899326,0.817325,0.823820}%
\pgfsetfillcolor{currentfill}%
\pgfsetlinewidth{0.000000pt}%
\definecolor{currentstroke}{rgb}{0.000000,0.000000,0.000000}%
\pgfsetstrokecolor{currentstroke}%
\pgfsetdash{}{0pt}%
\pgfpathmoveto{\pgfqpoint{1.354281in}{1.292957in}}%
\pgfpathlineto{\pgfqpoint{1.390979in}{1.311284in}}%
\pgfpathlineto{\pgfqpoint{1.354922in}{1.329547in}}%
\pgfpathlineto{\pgfqpoint{1.318226in}{1.311284in}}%
\pgfpathclose%
\pgfusepath{fill}%
\end{pgfscope}%
\begin{pgfscope}%
\pgfpathrectangle{\pgfqpoint{0.150000in}{0.150000in}}{\pgfqpoint{2.700000in}{1.950000in}}%
\pgfusepath{clip}%
\pgfsetbuttcap%
\pgfsetroundjoin%
\definecolor{currentfill}{rgb}{0.899326,0.817325,0.823820}%
\pgfsetfillcolor{currentfill}%
\pgfsetlinewidth{0.000000pt}%
\definecolor{currentstroke}{rgb}{0.000000,0.000000,0.000000}%
\pgfsetstrokecolor{currentstroke}%
\pgfsetdash{}{0pt}%
\pgfpathmoveto{\pgfqpoint{1.281399in}{1.292957in}}%
\pgfpathlineto{\pgfqpoint{1.318226in}{1.311284in}}%
\pgfpathlineto{\pgfqpoint{1.282297in}{1.329547in}}%
\pgfpathlineto{\pgfqpoint{1.245472in}{1.311284in}}%
\pgfpathclose%
\pgfusepath{fill}%
\end{pgfscope}%
\begin{pgfscope}%
\pgfpathrectangle{\pgfqpoint{0.150000in}{0.150000in}}{\pgfqpoint{2.700000in}{1.950000in}}%
\pgfusepath{clip}%
\pgfsetbuttcap%
\pgfsetroundjoin%
\definecolor{currentfill}{rgb}{0.899326,0.817325,0.823820}%
\pgfsetfillcolor{currentfill}%
\pgfsetlinewidth{0.000000pt}%
\definecolor{currentstroke}{rgb}{0.000000,0.000000,0.000000}%
\pgfsetstrokecolor{currentstroke}%
\pgfsetdash{}{0pt}%
\pgfpathmoveto{\pgfqpoint{1.208517in}{1.292957in}}%
\pgfpathlineto{\pgfqpoint{1.245472in}{1.311284in}}%
\pgfpathlineto{\pgfqpoint{1.209671in}{1.329547in}}%
\pgfpathlineto{\pgfqpoint{1.172718in}{1.311284in}}%
\pgfpathclose%
\pgfusepath{fill}%
\end{pgfscope}%
\begin{pgfscope}%
\pgfpathrectangle{\pgfqpoint{0.150000in}{0.150000in}}{\pgfqpoint{2.700000in}{1.950000in}}%
\pgfusepath{clip}%
\pgfsetbuttcap%
\pgfsetroundjoin%
\definecolor{currentfill}{rgb}{0.899326,0.817325,0.823820}%
\pgfsetfillcolor{currentfill}%
\pgfsetlinewidth{0.000000pt}%
\definecolor{currentstroke}{rgb}{0.000000,0.000000,0.000000}%
\pgfsetstrokecolor{currentstroke}%
\pgfsetdash{}{0pt}%
\pgfpathmoveto{\pgfqpoint{1.135635in}{1.292957in}}%
\pgfpathlineto{\pgfqpoint{1.172718in}{1.311284in}}%
\pgfpathlineto{\pgfqpoint{1.137046in}{1.329547in}}%
\pgfpathlineto{\pgfqpoint{1.099965in}{1.311284in}}%
\pgfpathclose%
\pgfusepath{fill}%
\end{pgfscope}%
\begin{pgfscope}%
\pgfpathrectangle{\pgfqpoint{0.150000in}{0.150000in}}{\pgfqpoint{2.700000in}{1.950000in}}%
\pgfusepath{clip}%
\pgfsetbuttcap%
\pgfsetroundjoin%
\definecolor{currentfill}{rgb}{0.766360,0.576057,0.591131}%
\pgfsetfillcolor{currentfill}%
\pgfsetlinewidth{0.000000pt}%
\definecolor{currentstroke}{rgb}{0.000000,0.000000,0.000000}%
\pgfsetstrokecolor{currentstroke}%
\pgfsetdash{}{0pt}%
\pgfpathmoveto{\pgfqpoint{2.046187in}{1.022724in}}%
\pgfpathlineto{\pgfqpoint{2.080830in}{1.026172in}}%
\pgfpathlineto{\pgfqpoint{2.045892in}{1.091045in}}%
\pgfpathlineto{\pgfqpoint{2.011043in}{1.087798in}}%
\pgfpathclose%
\pgfusepath{fill}%
\end{pgfscope}%
\begin{pgfscope}%
\pgfpathrectangle{\pgfqpoint{0.150000in}{0.150000in}}{\pgfqpoint{2.700000in}{1.950000in}}%
\pgfusepath{clip}%
\pgfsetbuttcap%
\pgfsetroundjoin%
\definecolor{currentfill}{rgb}{0.808150,0.651884,0.664262}%
\pgfsetfillcolor{currentfill}%
\pgfsetlinewidth{0.000000pt}%
\definecolor{currentstroke}{rgb}{0.000000,0.000000,0.000000}%
\pgfsetstrokecolor{currentstroke}%
\pgfsetdash{}{0pt}%
\pgfpathmoveto{\pgfqpoint{2.045892in}{1.091045in}}%
\pgfpathlineto{\pgfqpoint{2.080917in}{1.101979in}}%
\pgfpathlineto{\pgfqpoint{2.046727in}{1.182654in}}%
\pgfpathlineto{\pgfqpoint{2.011119in}{1.164064in}}%
\pgfpathclose%
\pgfusepath{fill}%
\end{pgfscope}%
\begin{pgfscope}%
\pgfpathrectangle{\pgfqpoint{0.150000in}{0.150000in}}{\pgfqpoint{2.700000in}{1.950000in}}%
\pgfusepath{clip}%
\pgfsetbuttcap%
\pgfsetroundjoin%
\definecolor{currentfill}{rgb}{0.732169,0.514017,0.531296}%
\pgfsetfillcolor{currentfill}%
\pgfsetlinewidth{0.000000pt}%
\definecolor{currentstroke}{rgb}{0.000000,0.000000,0.000000}%
\pgfsetstrokecolor{currentstroke}%
\pgfsetdash{}{0pt}%
\pgfpathmoveto{\pgfqpoint{2.046481in}{0.954324in}}%
\pgfpathlineto{\pgfqpoint{2.081144in}{0.957995in}}%
\pgfpathlineto{\pgfqpoint{2.046187in}{1.022724in}}%
\pgfpathlineto{\pgfqpoint{2.011317in}{1.019254in}}%
\pgfpathclose%
\pgfusepath{fill}%
\end{pgfscope}%
\begin{pgfscope}%
\pgfpathrectangle{\pgfqpoint{0.150000in}{0.150000in}}{\pgfqpoint{2.700000in}{1.950000in}}%
\pgfusepath{clip}%
\pgfsetbuttcap%
\pgfsetroundjoin%
\definecolor{currentfill}{rgb}{0.899326,0.817325,0.823820}%
\pgfsetfillcolor{currentfill}%
\pgfsetlinewidth{0.000000pt}%
\definecolor{currentstroke}{rgb}{0.000000,0.000000,0.000000}%
\pgfsetstrokecolor{currentstroke}%
\pgfsetdash{}{0pt}%
\pgfpathmoveto{\pgfqpoint{1.974552in}{1.274564in}}%
\pgfpathlineto{\pgfqpoint{2.010220in}{1.292957in}}%
\pgfpathlineto{\pgfqpoint{1.973008in}{1.311284in}}%
\pgfpathlineto{\pgfqpoint{1.937338in}{1.292957in}}%
\pgfpathclose%
\pgfusepath{fill}%
\end{pgfscope}%
\begin{pgfscope}%
\pgfpathrectangle{\pgfqpoint{0.150000in}{0.150000in}}{\pgfqpoint{2.700000in}{1.950000in}}%
\pgfusepath{clip}%
\pgfsetbuttcap%
\pgfsetroundjoin%
\definecolor{currentfill}{rgb}{0.899326,0.817325,0.823820}%
\pgfsetfillcolor{currentfill}%
\pgfsetlinewidth{0.000000pt}%
\definecolor{currentstroke}{rgb}{0.000000,0.000000,0.000000}%
\pgfsetstrokecolor{currentstroke}%
\pgfsetdash{}{0pt}%
\pgfpathmoveto{\pgfqpoint{1.901541in}{1.274564in}}%
\pgfpathlineto{\pgfqpoint{1.937338in}{1.292957in}}%
\pgfpathlineto{\pgfqpoint{1.900255in}{1.311284in}}%
\pgfpathlineto{\pgfqpoint{1.864456in}{1.292957in}}%
\pgfpathclose%
\pgfusepath{fill}%
\end{pgfscope}%
\begin{pgfscope}%
\pgfpathrectangle{\pgfqpoint{0.150000in}{0.150000in}}{\pgfqpoint{2.700000in}{1.950000in}}%
\pgfusepath{clip}%
\pgfsetbuttcap%
\pgfsetroundjoin%
\definecolor{currentfill}{rgb}{0.899326,0.817325,0.823820}%
\pgfsetfillcolor{currentfill}%
\pgfsetlinewidth{0.000000pt}%
\definecolor{currentstroke}{rgb}{0.000000,0.000000,0.000000}%
\pgfsetstrokecolor{currentstroke}%
\pgfsetdash{}{0pt}%
\pgfpathmoveto{\pgfqpoint{1.828530in}{1.274564in}}%
\pgfpathlineto{\pgfqpoint{1.864456in}{1.292957in}}%
\pgfpathlineto{\pgfqpoint{1.827501in}{1.311284in}}%
\pgfpathlineto{\pgfqpoint{1.791574in}{1.292957in}}%
\pgfpathclose%
\pgfusepath{fill}%
\end{pgfscope}%
\begin{pgfscope}%
\pgfpathrectangle{\pgfqpoint{0.150000in}{0.150000in}}{\pgfqpoint{2.700000in}{1.950000in}}%
\pgfusepath{clip}%
\pgfsetbuttcap%
\pgfsetroundjoin%
\definecolor{currentfill}{rgb}{0.899326,0.817325,0.823820}%
\pgfsetfillcolor{currentfill}%
\pgfsetlinewidth{0.000000pt}%
\definecolor{currentstroke}{rgb}{0.000000,0.000000,0.000000}%
\pgfsetstrokecolor{currentstroke}%
\pgfsetdash{}{0pt}%
\pgfpathmoveto{\pgfqpoint{1.755519in}{1.274564in}}%
\pgfpathlineto{\pgfqpoint{1.791574in}{1.292957in}}%
\pgfpathlineto{\pgfqpoint{1.754747in}{1.311284in}}%
\pgfpathlineto{\pgfqpoint{1.718692in}{1.292957in}}%
\pgfpathclose%
\pgfusepath{fill}%
\end{pgfscope}%
\begin{pgfscope}%
\pgfpathrectangle{\pgfqpoint{0.150000in}{0.150000in}}{\pgfqpoint{2.700000in}{1.950000in}}%
\pgfusepath{clip}%
\pgfsetbuttcap%
\pgfsetroundjoin%
\definecolor{currentfill}{rgb}{0.899326,0.817325,0.823820}%
\pgfsetfillcolor{currentfill}%
\pgfsetlinewidth{0.000000pt}%
\definecolor{currentstroke}{rgb}{0.000000,0.000000,0.000000}%
\pgfsetstrokecolor{currentstroke}%
\pgfsetdash{}{0pt}%
\pgfpathmoveto{\pgfqpoint{1.682508in}{1.274564in}}%
\pgfpathlineto{\pgfqpoint{1.718692in}{1.292957in}}%
\pgfpathlineto{\pgfqpoint{1.681994in}{1.311284in}}%
\pgfpathlineto{\pgfqpoint{1.645810in}{1.292957in}}%
\pgfpathclose%
\pgfusepath{fill}%
\end{pgfscope}%
\begin{pgfscope}%
\pgfpathrectangle{\pgfqpoint{0.150000in}{0.150000in}}{\pgfqpoint{2.700000in}{1.950000in}}%
\pgfusepath{clip}%
\pgfsetbuttcap%
\pgfsetroundjoin%
\definecolor{currentfill}{rgb}{0.899326,0.817325,0.823820}%
\pgfsetfillcolor{currentfill}%
\pgfsetlinewidth{0.000000pt}%
\definecolor{currentstroke}{rgb}{0.000000,0.000000,0.000000}%
\pgfsetstrokecolor{currentstroke}%
\pgfsetdash{}{0pt}%
\pgfpathmoveto{\pgfqpoint{1.609497in}{1.274564in}}%
\pgfpathlineto{\pgfqpoint{1.645810in}{1.292957in}}%
\pgfpathlineto{\pgfqpoint{1.609240in}{1.311284in}}%
\pgfpathlineto{\pgfqpoint{1.572928in}{1.292957in}}%
\pgfpathclose%
\pgfusepath{fill}%
\end{pgfscope}%
\begin{pgfscope}%
\pgfpathrectangle{\pgfqpoint{0.150000in}{0.150000in}}{\pgfqpoint{2.700000in}{1.950000in}}%
\pgfusepath{clip}%
\pgfsetbuttcap%
\pgfsetroundjoin%
\definecolor{currentfill}{rgb}{0.899326,0.817325,0.823820}%
\pgfsetfillcolor{currentfill}%
\pgfsetlinewidth{0.000000pt}%
\definecolor{currentstroke}{rgb}{0.000000,0.000000,0.000000}%
\pgfsetstrokecolor{currentstroke}%
\pgfsetdash{}{0pt}%
\pgfpathmoveto{\pgfqpoint{1.536486in}{1.274564in}}%
\pgfpathlineto{\pgfqpoint{1.572928in}{1.292957in}}%
\pgfpathlineto{\pgfqpoint{1.536486in}{1.311284in}}%
\pgfpathlineto{\pgfqpoint{1.500045in}{1.292957in}}%
\pgfpathclose%
\pgfusepath{fill}%
\end{pgfscope}%
\begin{pgfscope}%
\pgfpathrectangle{\pgfqpoint{0.150000in}{0.150000in}}{\pgfqpoint{2.700000in}{1.950000in}}%
\pgfusepath{clip}%
\pgfsetbuttcap%
\pgfsetroundjoin%
\definecolor{currentfill}{rgb}{0.899326,0.817325,0.823820}%
\pgfsetfillcolor{currentfill}%
\pgfsetlinewidth{0.000000pt}%
\definecolor{currentstroke}{rgb}{0.000000,0.000000,0.000000}%
\pgfsetstrokecolor{currentstroke}%
\pgfsetdash{}{0pt}%
\pgfpathmoveto{\pgfqpoint{1.463476in}{1.274564in}}%
\pgfpathlineto{\pgfqpoint{1.500045in}{1.292957in}}%
\pgfpathlineto{\pgfqpoint{1.463733in}{1.311284in}}%
\pgfpathlineto{\pgfqpoint{1.427163in}{1.292957in}}%
\pgfpathclose%
\pgfusepath{fill}%
\end{pgfscope}%
\begin{pgfscope}%
\pgfpathrectangle{\pgfqpoint{0.150000in}{0.150000in}}{\pgfqpoint{2.700000in}{1.950000in}}%
\pgfusepath{clip}%
\pgfsetbuttcap%
\pgfsetroundjoin%
\definecolor{currentfill}{rgb}{0.899326,0.817325,0.823820}%
\pgfsetfillcolor{currentfill}%
\pgfsetlinewidth{0.000000pt}%
\definecolor{currentstroke}{rgb}{0.000000,0.000000,0.000000}%
\pgfsetstrokecolor{currentstroke}%
\pgfsetdash{}{0pt}%
\pgfpathmoveto{\pgfqpoint{1.390465in}{1.274564in}}%
\pgfpathlineto{\pgfqpoint{1.427163in}{1.292957in}}%
\pgfpathlineto{\pgfqpoint{1.390979in}{1.311284in}}%
\pgfpathlineto{\pgfqpoint{1.354281in}{1.292957in}}%
\pgfpathclose%
\pgfusepath{fill}%
\end{pgfscope}%
\begin{pgfscope}%
\pgfpathrectangle{\pgfqpoint{0.150000in}{0.150000in}}{\pgfqpoint{2.700000in}{1.950000in}}%
\pgfusepath{clip}%
\pgfsetbuttcap%
\pgfsetroundjoin%
\definecolor{currentfill}{rgb}{0.899326,0.817325,0.823820}%
\pgfsetfillcolor{currentfill}%
\pgfsetlinewidth{0.000000pt}%
\definecolor{currentstroke}{rgb}{0.000000,0.000000,0.000000}%
\pgfsetstrokecolor{currentstroke}%
\pgfsetdash{}{0pt}%
\pgfpathmoveto{\pgfqpoint{1.317454in}{1.274564in}}%
\pgfpathlineto{\pgfqpoint{1.354281in}{1.292957in}}%
\pgfpathlineto{\pgfqpoint{1.318226in}{1.311284in}}%
\pgfpathlineto{\pgfqpoint{1.281399in}{1.292957in}}%
\pgfpathclose%
\pgfusepath{fill}%
\end{pgfscope}%
\begin{pgfscope}%
\pgfpathrectangle{\pgfqpoint{0.150000in}{0.150000in}}{\pgfqpoint{2.700000in}{1.950000in}}%
\pgfusepath{clip}%
\pgfsetbuttcap%
\pgfsetroundjoin%
\definecolor{currentfill}{rgb}{0.899326,0.817325,0.823820}%
\pgfsetfillcolor{currentfill}%
\pgfsetlinewidth{0.000000pt}%
\definecolor{currentstroke}{rgb}{0.000000,0.000000,0.000000}%
\pgfsetstrokecolor{currentstroke}%
\pgfsetdash{}{0pt}%
\pgfpathmoveto{\pgfqpoint{1.244443in}{1.274564in}}%
\pgfpathlineto{\pgfqpoint{1.281399in}{1.292957in}}%
\pgfpathlineto{\pgfqpoint{1.245472in}{1.311284in}}%
\pgfpathlineto{\pgfqpoint{1.208517in}{1.292957in}}%
\pgfpathclose%
\pgfusepath{fill}%
\end{pgfscope}%
\begin{pgfscope}%
\pgfpathrectangle{\pgfqpoint{0.150000in}{0.150000in}}{\pgfqpoint{2.700000in}{1.950000in}}%
\pgfusepath{clip}%
\pgfsetbuttcap%
\pgfsetroundjoin%
\definecolor{currentfill}{rgb}{0.899326,0.817325,0.823820}%
\pgfsetfillcolor{currentfill}%
\pgfsetlinewidth{0.000000pt}%
\definecolor{currentstroke}{rgb}{0.000000,0.000000,0.000000}%
\pgfsetstrokecolor{currentstroke}%
\pgfsetdash{}{0pt}%
\pgfpathmoveto{\pgfqpoint{1.171432in}{1.274564in}}%
\pgfpathlineto{\pgfqpoint{1.208517in}{1.292957in}}%
\pgfpathlineto{\pgfqpoint{1.172718in}{1.311284in}}%
\pgfpathlineto{\pgfqpoint{1.135635in}{1.292957in}}%
\pgfpathclose%
\pgfusepath{fill}%
\end{pgfscope}%
\begin{pgfscope}%
\pgfpathrectangle{\pgfqpoint{0.150000in}{0.150000in}}{\pgfqpoint{2.700000in}{1.950000in}}%
\pgfusepath{clip}%
\pgfsetbuttcap%
\pgfsetroundjoin%
\definecolor{currentfill}{rgb}{0.720772,0.493336,0.511351}%
\pgfsetfillcolor{currentfill}%
\pgfsetlinewidth{0.000000pt}%
\definecolor{currentstroke}{rgb}{0.000000,0.000000,0.000000}%
\pgfsetstrokecolor{currentstroke}%
\pgfsetdash{}{0pt}%
\pgfpathmoveto{\pgfqpoint{2.232418in}{0.973649in}}%
\pgfpathlineto{\pgfqpoint{2.267207in}{0.984951in}}%
\pgfpathlineto{\pgfqpoint{2.225372in}{0.942793in}}%
\pgfpathlineto{\pgfqpoint{2.191227in}{0.939078in}}%
\pgfpathclose%
\pgfusepath{fill}%
\end{pgfscope}%
\begin{pgfscope}%
\pgfpathrectangle{\pgfqpoint{0.150000in}{0.150000in}}{\pgfqpoint{2.700000in}{1.950000in}}%
\pgfusepath{clip}%
\pgfsetbuttcap%
\pgfsetroundjoin%
\definecolor{currentfill}{rgb}{0.705576,0.465763,0.484758}%
\pgfsetfillcolor{currentfill}%
\pgfsetlinewidth{0.000000pt}%
\definecolor{currentstroke}{rgb}{0.000000,0.000000,0.000000}%
\pgfsetstrokecolor{currentstroke}%
\pgfsetdash{}{0pt}%
\pgfpathmoveto{\pgfqpoint{2.122705in}{0.939238in}}%
\pgfpathlineto{\pgfqpoint{2.157319in}{0.942979in}}%
\pgfpathlineto{\pgfqpoint{2.117194in}{0.916294in}}%
\pgfpathlineto{\pgfqpoint{2.082664in}{0.912489in}}%
\pgfpathclose%
\pgfusepath{fill}%
\end{pgfscope}%
\begin{pgfscope}%
\pgfpathrectangle{\pgfqpoint{0.150000in}{0.150000in}}{\pgfqpoint{2.700000in}{1.950000in}}%
\pgfusepath{clip}%
\pgfsetbuttcap%
\pgfsetroundjoin%
\definecolor{currentfill}{rgb}{0.868934,0.762178,0.770634}%
\pgfsetfillcolor{currentfill}%
\pgfsetlinewidth{0.000000pt}%
\definecolor{currentstroke}{rgb}{0.000000,0.000000,0.000000}%
\pgfsetstrokecolor{currentstroke}%
\pgfsetdash{}{0pt}%
\pgfpathmoveto{\pgfqpoint{2.046727in}{1.182654in}}%
\pgfpathlineto{\pgfqpoint{2.081807in}{1.193378in}}%
\pgfpathlineto{\pgfqpoint{2.047563in}{1.274564in}}%
\pgfpathlineto{\pgfqpoint{2.011899in}{1.256107in}}%
\pgfpathclose%
\pgfusepath{fill}%
\end{pgfscope}%
\begin{pgfscope}%
\pgfpathrectangle{\pgfqpoint{0.150000in}{0.150000in}}{\pgfqpoint{2.700000in}{1.950000in}}%
\pgfusepath{clip}%
\pgfsetbuttcap%
\pgfsetroundjoin%
\definecolor{currentfill}{rgb}{0.713174,0.479550,0.498055}%
\pgfsetfillcolor{currentfill}%
\pgfsetlinewidth{0.000000pt}%
\definecolor{currentstroke}{rgb}{0.000000,0.000000,0.000000}%
\pgfsetstrokecolor{currentstroke}%
\pgfsetdash{}{0pt}%
\pgfpathmoveto{\pgfqpoint{2.048288in}{0.916288in}}%
\pgfpathlineto{\pgfqpoint{2.082664in}{0.912489in}}%
\pgfpathlineto{\pgfqpoint{2.046481in}{0.954324in}}%
\pgfpathlineto{\pgfqpoint{2.011945in}{0.958291in}}%
\pgfpathclose%
\pgfusepath{fill}%
\end{pgfscope}%
\begin{pgfscope}%
\pgfpathrectangle{\pgfqpoint{0.150000in}{0.150000in}}{\pgfqpoint{2.700000in}{1.950000in}}%
\pgfusepath{clip}%
\pgfsetbuttcap%
\pgfsetroundjoin%
\definecolor{currentfill}{rgb}{0.819547,0.672564,0.684206}%
\pgfsetfillcolor{currentfill}%
\pgfsetlinewidth{0.000000pt}%
\definecolor{currentstroke}{rgb}{0.000000,0.000000,0.000000}%
\pgfsetstrokecolor{currentstroke}%
\pgfsetdash{}{0pt}%
\pgfpathmoveto{\pgfqpoint{2.011043in}{1.087798in}}%
\pgfpathlineto{\pgfqpoint{2.045892in}{1.091045in}}%
\pgfpathlineto{\pgfqpoint{2.011119in}{1.164064in}}%
\pgfpathlineto{\pgfqpoint{1.975711in}{1.153221in}}%
\pgfpathclose%
\pgfusepath{fill}%
\end{pgfscope}%
\begin{pgfscope}%
\pgfpathrectangle{\pgfqpoint{0.150000in}{0.150000in}}{\pgfqpoint{2.700000in}{1.950000in}}%
\pgfusepath{clip}%
\pgfsetbuttcap%
\pgfsetroundjoin%
\definecolor{currentfill}{rgb}{0.899326,0.817325,0.823820}%
\pgfsetfillcolor{currentfill}%
\pgfsetlinewidth{0.000000pt}%
\definecolor{currentstroke}{rgb}{0.000000,0.000000,0.000000}%
\pgfsetstrokecolor{currentstroke}%
\pgfsetdash{}{0pt}%
\pgfpathmoveto{\pgfqpoint{2.011899in}{1.256107in}}%
\pgfpathlineto{\pgfqpoint{2.047563in}{1.274564in}}%
\pgfpathlineto{\pgfqpoint{2.010220in}{1.292957in}}%
\pgfpathlineto{\pgfqpoint{1.974552in}{1.274564in}}%
\pgfpathclose%
\pgfusepath{fill}%
\end{pgfscope}%
\begin{pgfscope}%
\pgfpathrectangle{\pgfqpoint{0.150000in}{0.150000in}}{\pgfqpoint{2.700000in}{1.950000in}}%
\pgfusepath{clip}%
\pgfsetbuttcap%
\pgfsetroundjoin%
\definecolor{currentfill}{rgb}{0.899326,0.817325,0.823820}%
\pgfsetfillcolor{currentfill}%
\pgfsetlinewidth{0.000000pt}%
\definecolor{currentstroke}{rgb}{0.000000,0.000000,0.000000}%
\pgfsetstrokecolor{currentstroke}%
\pgfsetdash{}{0pt}%
\pgfpathmoveto{\pgfqpoint{1.938758in}{1.256107in}}%
\pgfpathlineto{\pgfqpoint{1.974552in}{1.274564in}}%
\pgfpathlineto{\pgfqpoint{1.937338in}{1.292957in}}%
\pgfpathlineto{\pgfqpoint{1.901541in}{1.274564in}}%
\pgfpathclose%
\pgfusepath{fill}%
\end{pgfscope}%
\begin{pgfscope}%
\pgfpathrectangle{\pgfqpoint{0.150000in}{0.150000in}}{\pgfqpoint{2.700000in}{1.950000in}}%
\pgfusepath{clip}%
\pgfsetbuttcap%
\pgfsetroundjoin%
\definecolor{currentfill}{rgb}{0.899326,0.817325,0.823820}%
\pgfsetfillcolor{currentfill}%
\pgfsetlinewidth{0.000000pt}%
\definecolor{currentstroke}{rgb}{0.000000,0.000000,0.000000}%
\pgfsetstrokecolor{currentstroke}%
\pgfsetdash{}{0pt}%
\pgfpathmoveto{\pgfqpoint{1.865618in}{1.256107in}}%
\pgfpathlineto{\pgfqpoint{1.901541in}{1.274564in}}%
\pgfpathlineto{\pgfqpoint{1.864456in}{1.292957in}}%
\pgfpathlineto{\pgfqpoint{1.828530in}{1.274564in}}%
\pgfpathclose%
\pgfusepath{fill}%
\end{pgfscope}%
\begin{pgfscope}%
\pgfpathrectangle{\pgfqpoint{0.150000in}{0.150000in}}{\pgfqpoint{2.700000in}{1.950000in}}%
\pgfusepath{clip}%
\pgfsetbuttcap%
\pgfsetroundjoin%
\definecolor{currentfill}{rgb}{0.899326,0.817325,0.823820}%
\pgfsetfillcolor{currentfill}%
\pgfsetlinewidth{0.000000pt}%
\definecolor{currentstroke}{rgb}{0.000000,0.000000,0.000000}%
\pgfsetstrokecolor{currentstroke}%
\pgfsetdash{}{0pt}%
\pgfpathmoveto{\pgfqpoint{1.792478in}{1.256107in}}%
\pgfpathlineto{\pgfqpoint{1.828530in}{1.274564in}}%
\pgfpathlineto{\pgfqpoint{1.791574in}{1.292957in}}%
\pgfpathlineto{\pgfqpoint{1.755519in}{1.274564in}}%
\pgfpathclose%
\pgfusepath{fill}%
\end{pgfscope}%
\begin{pgfscope}%
\pgfpathrectangle{\pgfqpoint{0.150000in}{0.150000in}}{\pgfqpoint{2.700000in}{1.950000in}}%
\pgfusepath{clip}%
\pgfsetbuttcap%
\pgfsetroundjoin%
\definecolor{currentfill}{rgb}{0.899326,0.817325,0.823820}%
\pgfsetfillcolor{currentfill}%
\pgfsetlinewidth{0.000000pt}%
\definecolor{currentstroke}{rgb}{0.000000,0.000000,0.000000}%
\pgfsetstrokecolor{currentstroke}%
\pgfsetdash{}{0pt}%
\pgfpathmoveto{\pgfqpoint{1.719337in}{1.256107in}}%
\pgfpathlineto{\pgfqpoint{1.755519in}{1.274564in}}%
\pgfpathlineto{\pgfqpoint{1.718692in}{1.292957in}}%
\pgfpathlineto{\pgfqpoint{1.682508in}{1.274564in}}%
\pgfpathclose%
\pgfusepath{fill}%
\end{pgfscope}%
\begin{pgfscope}%
\pgfpathrectangle{\pgfqpoint{0.150000in}{0.150000in}}{\pgfqpoint{2.700000in}{1.950000in}}%
\pgfusepath{clip}%
\pgfsetbuttcap%
\pgfsetroundjoin%
\definecolor{currentfill}{rgb}{0.899326,0.817325,0.823820}%
\pgfsetfillcolor{currentfill}%
\pgfsetlinewidth{0.000000pt}%
\definecolor{currentstroke}{rgb}{0.000000,0.000000,0.000000}%
\pgfsetstrokecolor{currentstroke}%
\pgfsetdash{}{0pt}%
\pgfpathmoveto{\pgfqpoint{1.646197in}{1.256107in}}%
\pgfpathlineto{\pgfqpoint{1.682508in}{1.274564in}}%
\pgfpathlineto{\pgfqpoint{1.645810in}{1.292957in}}%
\pgfpathlineto{\pgfqpoint{1.609497in}{1.274564in}}%
\pgfpathclose%
\pgfusepath{fill}%
\end{pgfscope}%
\begin{pgfscope}%
\pgfpathrectangle{\pgfqpoint{0.150000in}{0.150000in}}{\pgfqpoint{2.700000in}{1.950000in}}%
\pgfusepath{clip}%
\pgfsetbuttcap%
\pgfsetroundjoin%
\definecolor{currentfill}{rgb}{0.899326,0.817325,0.823820}%
\pgfsetfillcolor{currentfill}%
\pgfsetlinewidth{0.000000pt}%
\definecolor{currentstroke}{rgb}{0.000000,0.000000,0.000000}%
\pgfsetstrokecolor{currentstroke}%
\pgfsetdash{}{0pt}%
\pgfpathmoveto{\pgfqpoint{1.573057in}{1.256107in}}%
\pgfpathlineto{\pgfqpoint{1.609497in}{1.274564in}}%
\pgfpathlineto{\pgfqpoint{1.572928in}{1.292957in}}%
\pgfpathlineto{\pgfqpoint{1.536486in}{1.274564in}}%
\pgfpathclose%
\pgfusepath{fill}%
\end{pgfscope}%
\begin{pgfscope}%
\pgfpathrectangle{\pgfqpoint{0.150000in}{0.150000in}}{\pgfqpoint{2.700000in}{1.950000in}}%
\pgfusepath{clip}%
\pgfsetbuttcap%
\pgfsetroundjoin%
\definecolor{currentfill}{rgb}{0.899326,0.817325,0.823820}%
\pgfsetfillcolor{currentfill}%
\pgfsetlinewidth{0.000000pt}%
\definecolor{currentstroke}{rgb}{0.000000,0.000000,0.000000}%
\pgfsetstrokecolor{currentstroke}%
\pgfsetdash{}{0pt}%
\pgfpathmoveto{\pgfqpoint{1.499916in}{1.256107in}}%
\pgfpathlineto{\pgfqpoint{1.536486in}{1.274564in}}%
\pgfpathlineto{\pgfqpoint{1.500045in}{1.292957in}}%
\pgfpathlineto{\pgfqpoint{1.463476in}{1.274564in}}%
\pgfpathclose%
\pgfusepath{fill}%
\end{pgfscope}%
\begin{pgfscope}%
\pgfpathrectangle{\pgfqpoint{0.150000in}{0.150000in}}{\pgfqpoint{2.700000in}{1.950000in}}%
\pgfusepath{clip}%
\pgfsetbuttcap%
\pgfsetroundjoin%
\definecolor{currentfill}{rgb}{0.899326,0.817325,0.823820}%
\pgfsetfillcolor{currentfill}%
\pgfsetlinewidth{0.000000pt}%
\definecolor{currentstroke}{rgb}{0.000000,0.000000,0.000000}%
\pgfsetstrokecolor{currentstroke}%
\pgfsetdash{}{0pt}%
\pgfpathmoveto{\pgfqpoint{1.426776in}{1.256107in}}%
\pgfpathlineto{\pgfqpoint{1.463476in}{1.274564in}}%
\pgfpathlineto{\pgfqpoint{1.427163in}{1.292957in}}%
\pgfpathlineto{\pgfqpoint{1.390465in}{1.274564in}}%
\pgfpathclose%
\pgfusepath{fill}%
\end{pgfscope}%
\begin{pgfscope}%
\pgfpathrectangle{\pgfqpoint{0.150000in}{0.150000in}}{\pgfqpoint{2.700000in}{1.950000in}}%
\pgfusepath{clip}%
\pgfsetbuttcap%
\pgfsetroundjoin%
\definecolor{currentfill}{rgb}{0.899326,0.817325,0.823820}%
\pgfsetfillcolor{currentfill}%
\pgfsetlinewidth{0.000000pt}%
\definecolor{currentstroke}{rgb}{0.000000,0.000000,0.000000}%
\pgfsetstrokecolor{currentstroke}%
\pgfsetdash{}{0pt}%
\pgfpathmoveto{\pgfqpoint{1.353636in}{1.256107in}}%
\pgfpathlineto{\pgfqpoint{1.390465in}{1.274564in}}%
\pgfpathlineto{\pgfqpoint{1.354281in}{1.292957in}}%
\pgfpathlineto{\pgfqpoint{1.317454in}{1.274564in}}%
\pgfpathclose%
\pgfusepath{fill}%
\end{pgfscope}%
\begin{pgfscope}%
\pgfpathrectangle{\pgfqpoint{0.150000in}{0.150000in}}{\pgfqpoint{2.700000in}{1.950000in}}%
\pgfusepath{clip}%
\pgfsetbuttcap%
\pgfsetroundjoin%
\definecolor{currentfill}{rgb}{0.899326,0.817325,0.823820}%
\pgfsetfillcolor{currentfill}%
\pgfsetlinewidth{0.000000pt}%
\definecolor{currentstroke}{rgb}{0.000000,0.000000,0.000000}%
\pgfsetstrokecolor{currentstroke}%
\pgfsetdash{}{0pt}%
\pgfpathmoveto{\pgfqpoint{1.280495in}{1.256107in}}%
\pgfpathlineto{\pgfqpoint{1.317454in}{1.274564in}}%
\pgfpathlineto{\pgfqpoint{1.281399in}{1.292957in}}%
\pgfpathlineto{\pgfqpoint{1.244443in}{1.274564in}}%
\pgfpathclose%
\pgfusepath{fill}%
\end{pgfscope}%
\begin{pgfscope}%
\pgfpathrectangle{\pgfqpoint{0.150000in}{0.150000in}}{\pgfqpoint{2.700000in}{1.950000in}}%
\pgfusepath{clip}%
\pgfsetbuttcap%
\pgfsetroundjoin%
\definecolor{currentfill}{rgb}{0.899326,0.817325,0.823820}%
\pgfsetfillcolor{currentfill}%
\pgfsetlinewidth{0.000000pt}%
\definecolor{currentstroke}{rgb}{0.000000,0.000000,0.000000}%
\pgfsetstrokecolor{currentstroke}%
\pgfsetdash{}{0pt}%
\pgfpathmoveto{\pgfqpoint{1.207355in}{1.256107in}}%
\pgfpathlineto{\pgfqpoint{1.244443in}{1.274564in}}%
\pgfpathlineto{\pgfqpoint{1.208517in}{1.292957in}}%
\pgfpathlineto{\pgfqpoint{1.171432in}{1.274564in}}%
\pgfpathclose%
\pgfusepath{fill}%
\end{pgfscope}%
\begin{pgfscope}%
\pgfpathrectangle{\pgfqpoint{0.150000in}{0.150000in}}{\pgfqpoint{2.700000in}{1.950000in}}%
\pgfusepath{clip}%
\pgfsetbuttcap%
\pgfsetroundjoin%
\definecolor{currentfill}{rgb}{0.781556,0.603631,0.617724}%
\pgfsetfillcolor{currentfill}%
\pgfsetlinewidth{0.000000pt}%
\definecolor{currentstroke}{rgb}{0.000000,0.000000,0.000000}%
\pgfsetstrokecolor{currentstroke}%
\pgfsetdash{}{0pt}%
\pgfpathmoveto{\pgfqpoint{2.011317in}{1.019254in}}%
\pgfpathlineto{\pgfqpoint{2.046187in}{1.022724in}}%
\pgfpathlineto{\pgfqpoint{2.011043in}{1.087798in}}%
\pgfpathlineto{\pgfqpoint{1.975640in}{1.076765in}}%
\pgfpathclose%
\pgfusepath{fill}%
\end{pgfscope}%
\begin{pgfscope}%
\pgfpathrectangle{\pgfqpoint{0.150000in}{0.150000in}}{\pgfqpoint{2.700000in}{1.950000in}}%
\pgfusepath{clip}%
\pgfsetbuttcap%
\pgfsetroundjoin%
\definecolor{currentfill}{rgb}{0.929718,0.872472,0.877007}%
\pgfsetfillcolor{currentfill}%
\pgfsetlinewidth{0.000000pt}%
\definecolor{currentstroke}{rgb}{0.000000,0.000000,0.000000}%
\pgfsetstrokecolor{currentstroke}%
\pgfsetdash{}{0pt}%
\pgfpathmoveto{\pgfqpoint{1.096141in}{1.330062in}}%
\pgfpathlineto{\pgfqpoint{1.135635in}{1.292957in}}%
\pgfpathlineto{\pgfqpoint{1.099965in}{1.311284in}}%
\pgfpathlineto{\pgfqpoint{1.060292in}{1.348452in}}%
\pgfpathclose%
\pgfusepath{fill}%
\end{pgfscope}%
\begin{pgfscope}%
\pgfpathrectangle{\pgfqpoint{0.150000in}{0.150000in}}{\pgfqpoint{2.700000in}{1.950000in}}%
\pgfusepath{clip}%
\pgfsetbuttcap%
\pgfsetroundjoin%
\definecolor{currentfill}{rgb}{0.747365,0.541590,0.557889}%
\pgfsetfillcolor{currentfill}%
\pgfsetlinewidth{0.000000pt}%
\definecolor{currentstroke}{rgb}{0.000000,0.000000,0.000000}%
\pgfsetstrokecolor{currentstroke}%
\pgfsetdash{}{0pt}%
\pgfpathmoveto{\pgfqpoint{2.011945in}{0.958291in}}%
\pgfpathlineto{\pgfqpoint{2.046481in}{0.954324in}}%
\pgfpathlineto{\pgfqpoint{2.011317in}{1.019254in}}%
\pgfpathlineto{\pgfqpoint{1.975895in}{1.008041in}}%
\pgfpathclose%
\pgfusepath{fill}%
\end{pgfscope}%
\begin{pgfscope}%
\pgfpathrectangle{\pgfqpoint{0.150000in}{0.150000in}}{\pgfqpoint{2.700000in}{1.950000in}}%
\pgfusepath{clip}%
\pgfsetbuttcap%
\pgfsetroundjoin%
\definecolor{currentfill}{rgb}{0.868934,0.762178,0.770634}%
\pgfsetfillcolor{currentfill}%
\pgfsetlinewidth{0.000000pt}%
\definecolor{currentstroke}{rgb}{0.000000,0.000000,0.000000}%
\pgfsetstrokecolor{currentstroke}%
\pgfsetdash{}{0pt}%
\pgfpathmoveto{\pgfqpoint{2.011119in}{1.164064in}}%
\pgfpathlineto{\pgfqpoint{2.046727in}{1.182654in}}%
\pgfpathlineto{\pgfqpoint{2.011899in}{1.256107in}}%
\pgfpathlineto{\pgfqpoint{1.976108in}{1.237583in}}%
\pgfpathclose%
\pgfusepath{fill}%
\end{pgfscope}%
\begin{pgfscope}%
\pgfpathrectangle{\pgfqpoint{0.150000in}{0.150000in}}{\pgfqpoint{2.700000in}{1.950000in}}%
\pgfusepath{clip}%
\pgfsetbuttcap%
\pgfsetroundjoin%
\definecolor{currentfill}{rgb}{0.739767,0.527803,0.544593}%
\pgfsetfillcolor{currentfill}%
\pgfsetlinewidth{0.000000pt}%
\definecolor{currentstroke}{rgb}{0.000000,0.000000,0.000000}%
\pgfsetstrokecolor{currentstroke}%
\pgfsetdash{}{0pt}%
\pgfpathmoveto{\pgfqpoint{2.197946in}{0.969998in}}%
\pgfpathlineto{\pgfqpoint{2.232418in}{0.973649in}}%
\pgfpathlineto{\pgfqpoint{2.191227in}{0.939078in}}%
\pgfpathlineto{\pgfqpoint{2.157319in}{0.942979in}}%
\pgfpathclose%
\pgfusepath{fill}%
\end{pgfscope}%
\begin{pgfscope}%
\pgfpathrectangle{\pgfqpoint{0.150000in}{0.150000in}}{\pgfqpoint{2.700000in}{1.950000in}}%
\pgfusepath{clip}%
\pgfsetbuttcap%
\pgfsetroundjoin%
\definecolor{currentfill}{rgb}{0.899326,0.817325,0.823820}%
\pgfsetfillcolor{currentfill}%
\pgfsetlinewidth{0.000000pt}%
\definecolor{currentstroke}{rgb}{0.000000,0.000000,0.000000}%
\pgfsetstrokecolor{currentstroke}%
\pgfsetdash{}{0pt}%
\pgfpathmoveto{\pgfqpoint{1.976108in}{1.237583in}}%
\pgfpathlineto{\pgfqpoint{2.011899in}{1.256107in}}%
\pgfpathlineto{\pgfqpoint{1.974552in}{1.274564in}}%
\pgfpathlineto{\pgfqpoint{1.938758in}{1.256107in}}%
\pgfpathclose%
\pgfusepath{fill}%
\end{pgfscope}%
\begin{pgfscope}%
\pgfpathrectangle{\pgfqpoint{0.150000in}{0.150000in}}{\pgfqpoint{2.700000in}{1.950000in}}%
\pgfusepath{clip}%
\pgfsetbuttcap%
\pgfsetroundjoin%
\definecolor{currentfill}{rgb}{0.899326,0.817325,0.823820}%
\pgfsetfillcolor{currentfill}%
\pgfsetlinewidth{0.000000pt}%
\definecolor{currentstroke}{rgb}{0.000000,0.000000,0.000000}%
\pgfsetstrokecolor{currentstroke}%
\pgfsetdash{}{0pt}%
\pgfpathmoveto{\pgfqpoint{1.902837in}{1.237583in}}%
\pgfpathlineto{\pgfqpoint{1.938758in}{1.256107in}}%
\pgfpathlineto{\pgfqpoint{1.901541in}{1.274564in}}%
\pgfpathlineto{\pgfqpoint{1.865618in}{1.256107in}}%
\pgfpathclose%
\pgfusepath{fill}%
\end{pgfscope}%
\begin{pgfscope}%
\pgfpathrectangle{\pgfqpoint{0.150000in}{0.150000in}}{\pgfqpoint{2.700000in}{1.950000in}}%
\pgfusepath{clip}%
\pgfsetbuttcap%
\pgfsetroundjoin%
\definecolor{currentfill}{rgb}{0.899326,0.817325,0.823820}%
\pgfsetfillcolor{currentfill}%
\pgfsetlinewidth{0.000000pt}%
\definecolor{currentstroke}{rgb}{0.000000,0.000000,0.000000}%
\pgfsetstrokecolor{currentstroke}%
\pgfsetdash{}{0pt}%
\pgfpathmoveto{\pgfqpoint{1.829567in}{1.237583in}}%
\pgfpathlineto{\pgfqpoint{1.865618in}{1.256107in}}%
\pgfpathlineto{\pgfqpoint{1.828530in}{1.274564in}}%
\pgfpathlineto{\pgfqpoint{1.792478in}{1.256107in}}%
\pgfpathclose%
\pgfusepath{fill}%
\end{pgfscope}%
\begin{pgfscope}%
\pgfpathrectangle{\pgfqpoint{0.150000in}{0.150000in}}{\pgfqpoint{2.700000in}{1.950000in}}%
\pgfusepath{clip}%
\pgfsetbuttcap%
\pgfsetroundjoin%
\definecolor{currentfill}{rgb}{0.899326,0.817325,0.823820}%
\pgfsetfillcolor{currentfill}%
\pgfsetlinewidth{0.000000pt}%
\definecolor{currentstroke}{rgb}{0.000000,0.000000,0.000000}%
\pgfsetstrokecolor{currentstroke}%
\pgfsetdash{}{0pt}%
\pgfpathmoveto{\pgfqpoint{1.756297in}{1.237583in}}%
\pgfpathlineto{\pgfqpoint{1.792478in}{1.256107in}}%
\pgfpathlineto{\pgfqpoint{1.755519in}{1.274564in}}%
\pgfpathlineto{\pgfqpoint{1.719337in}{1.256107in}}%
\pgfpathclose%
\pgfusepath{fill}%
\end{pgfscope}%
\begin{pgfscope}%
\pgfpathrectangle{\pgfqpoint{0.150000in}{0.150000in}}{\pgfqpoint{2.700000in}{1.950000in}}%
\pgfusepath{clip}%
\pgfsetbuttcap%
\pgfsetroundjoin%
\definecolor{currentfill}{rgb}{0.899326,0.817325,0.823820}%
\pgfsetfillcolor{currentfill}%
\pgfsetlinewidth{0.000000pt}%
\definecolor{currentstroke}{rgb}{0.000000,0.000000,0.000000}%
\pgfsetstrokecolor{currentstroke}%
\pgfsetdash{}{0pt}%
\pgfpathmoveto{\pgfqpoint{1.683027in}{1.237583in}}%
\pgfpathlineto{\pgfqpoint{1.719337in}{1.256107in}}%
\pgfpathlineto{\pgfqpoint{1.682508in}{1.274564in}}%
\pgfpathlineto{\pgfqpoint{1.646197in}{1.256107in}}%
\pgfpathclose%
\pgfusepath{fill}%
\end{pgfscope}%
\begin{pgfscope}%
\pgfpathrectangle{\pgfqpoint{0.150000in}{0.150000in}}{\pgfqpoint{2.700000in}{1.950000in}}%
\pgfusepath{clip}%
\pgfsetbuttcap%
\pgfsetroundjoin%
\definecolor{currentfill}{rgb}{0.899326,0.817325,0.823820}%
\pgfsetfillcolor{currentfill}%
\pgfsetlinewidth{0.000000pt}%
\definecolor{currentstroke}{rgb}{0.000000,0.000000,0.000000}%
\pgfsetstrokecolor{currentstroke}%
\pgfsetdash{}{0pt}%
\pgfpathmoveto{\pgfqpoint{1.609757in}{1.237583in}}%
\pgfpathlineto{\pgfqpoint{1.646197in}{1.256107in}}%
\pgfpathlineto{\pgfqpoint{1.609497in}{1.274564in}}%
\pgfpathlineto{\pgfqpoint{1.573057in}{1.256107in}}%
\pgfpathclose%
\pgfusepath{fill}%
\end{pgfscope}%
\begin{pgfscope}%
\pgfpathrectangle{\pgfqpoint{0.150000in}{0.150000in}}{\pgfqpoint{2.700000in}{1.950000in}}%
\pgfusepath{clip}%
\pgfsetbuttcap%
\pgfsetroundjoin%
\definecolor{currentfill}{rgb}{0.899326,0.817325,0.823820}%
\pgfsetfillcolor{currentfill}%
\pgfsetlinewidth{0.000000pt}%
\definecolor{currentstroke}{rgb}{0.000000,0.000000,0.000000}%
\pgfsetstrokecolor{currentstroke}%
\pgfsetdash{}{0pt}%
\pgfpathmoveto{\pgfqpoint{1.536486in}{1.237583in}}%
\pgfpathlineto{\pgfqpoint{1.573057in}{1.256107in}}%
\pgfpathlineto{\pgfqpoint{1.536486in}{1.274564in}}%
\pgfpathlineto{\pgfqpoint{1.499916in}{1.256107in}}%
\pgfpathclose%
\pgfusepath{fill}%
\end{pgfscope}%
\begin{pgfscope}%
\pgfpathrectangle{\pgfqpoint{0.150000in}{0.150000in}}{\pgfqpoint{2.700000in}{1.950000in}}%
\pgfusepath{clip}%
\pgfsetbuttcap%
\pgfsetroundjoin%
\definecolor{currentfill}{rgb}{0.899326,0.817325,0.823820}%
\pgfsetfillcolor{currentfill}%
\pgfsetlinewidth{0.000000pt}%
\definecolor{currentstroke}{rgb}{0.000000,0.000000,0.000000}%
\pgfsetstrokecolor{currentstroke}%
\pgfsetdash{}{0pt}%
\pgfpathmoveto{\pgfqpoint{1.463216in}{1.237583in}}%
\pgfpathlineto{\pgfqpoint{1.499916in}{1.256107in}}%
\pgfpathlineto{\pgfqpoint{1.463476in}{1.274564in}}%
\pgfpathlineto{\pgfqpoint{1.426776in}{1.256107in}}%
\pgfpathclose%
\pgfusepath{fill}%
\end{pgfscope}%
\begin{pgfscope}%
\pgfpathrectangle{\pgfqpoint{0.150000in}{0.150000in}}{\pgfqpoint{2.700000in}{1.950000in}}%
\pgfusepath{clip}%
\pgfsetbuttcap%
\pgfsetroundjoin%
\definecolor{currentfill}{rgb}{0.899326,0.817325,0.823820}%
\pgfsetfillcolor{currentfill}%
\pgfsetlinewidth{0.000000pt}%
\definecolor{currentstroke}{rgb}{0.000000,0.000000,0.000000}%
\pgfsetstrokecolor{currentstroke}%
\pgfsetdash{}{0pt}%
\pgfpathmoveto{\pgfqpoint{1.389946in}{1.237583in}}%
\pgfpathlineto{\pgfqpoint{1.426776in}{1.256107in}}%
\pgfpathlineto{\pgfqpoint{1.390465in}{1.274564in}}%
\pgfpathlineto{\pgfqpoint{1.353636in}{1.256107in}}%
\pgfpathclose%
\pgfusepath{fill}%
\end{pgfscope}%
\begin{pgfscope}%
\pgfpathrectangle{\pgfqpoint{0.150000in}{0.150000in}}{\pgfqpoint{2.700000in}{1.950000in}}%
\pgfusepath{clip}%
\pgfsetbuttcap%
\pgfsetroundjoin%
\definecolor{currentfill}{rgb}{0.899326,0.817325,0.823820}%
\pgfsetfillcolor{currentfill}%
\pgfsetlinewidth{0.000000pt}%
\definecolor{currentstroke}{rgb}{0.000000,0.000000,0.000000}%
\pgfsetstrokecolor{currentstroke}%
\pgfsetdash{}{0pt}%
\pgfpathmoveto{\pgfqpoint{1.316676in}{1.237583in}}%
\pgfpathlineto{\pgfqpoint{1.353636in}{1.256107in}}%
\pgfpathlineto{\pgfqpoint{1.317454in}{1.274564in}}%
\pgfpathlineto{\pgfqpoint{1.280495in}{1.256107in}}%
\pgfpathclose%
\pgfusepath{fill}%
\end{pgfscope}%
\begin{pgfscope}%
\pgfpathrectangle{\pgfqpoint{0.150000in}{0.150000in}}{\pgfqpoint{2.700000in}{1.950000in}}%
\pgfusepath{clip}%
\pgfsetbuttcap%
\pgfsetroundjoin%
\definecolor{currentfill}{rgb}{0.899326,0.817325,0.823820}%
\pgfsetfillcolor{currentfill}%
\pgfsetlinewidth{0.000000pt}%
\definecolor{currentstroke}{rgb}{0.000000,0.000000,0.000000}%
\pgfsetstrokecolor{currentstroke}%
\pgfsetdash{}{0pt}%
\pgfpathmoveto{\pgfqpoint{1.243406in}{1.237583in}}%
\pgfpathlineto{\pgfqpoint{1.280495in}{1.256107in}}%
\pgfpathlineto{\pgfqpoint{1.244443in}{1.274564in}}%
\pgfpathlineto{\pgfqpoint{1.207355in}{1.256107in}}%
\pgfpathclose%
\pgfusepath{fill}%
\end{pgfscope}%
\begin{pgfscope}%
\pgfpathrectangle{\pgfqpoint{0.150000in}{0.150000in}}{\pgfqpoint{2.700000in}{1.950000in}}%
\pgfusepath{clip}%
\pgfsetbuttcap%
\pgfsetroundjoin%
\definecolor{currentfill}{rgb}{0.724571,0.500230,0.517999}%
\pgfsetfillcolor{currentfill}%
\pgfsetlinewidth{0.000000pt}%
\definecolor{currentstroke}{rgb}{0.000000,0.000000,0.000000}%
\pgfsetstrokecolor{currentstroke}%
\pgfsetdash{}{0pt}%
\pgfpathmoveto{\pgfqpoint{2.087863in}{0.935472in}}%
\pgfpathlineto{\pgfqpoint{2.122705in}{0.939238in}}%
\pgfpathlineto{\pgfqpoint{2.082664in}{0.912489in}}%
\pgfpathlineto{\pgfqpoint{2.048288in}{0.916288in}}%
\pgfpathclose%
\pgfusepath{fill}%
\end{pgfscope}%
\begin{pgfscope}%
\pgfpathrectangle{\pgfqpoint{0.150000in}{0.150000in}}{\pgfqpoint{2.700000in}{1.950000in}}%
\pgfusepath{clip}%
\pgfsetbuttcap%
\pgfsetroundjoin%
\definecolor{currentfill}{rgb}{0.827145,0.686351,0.697503}%
\pgfsetfillcolor{currentfill}%
\pgfsetlinewidth{0.000000pt}%
\definecolor{currentstroke}{rgb}{0.000000,0.000000,0.000000}%
\pgfsetstrokecolor{currentstroke}%
\pgfsetdash{}{0pt}%
\pgfpathmoveto{\pgfqpoint{1.975640in}{1.076765in}}%
\pgfpathlineto{\pgfqpoint{2.011043in}{1.087798in}}%
\pgfpathlineto{\pgfqpoint{1.975711in}{1.153221in}}%
\pgfpathlineto{\pgfqpoint{1.939824in}{1.134499in}}%
\pgfpathclose%
\pgfusepath{fill}%
\end{pgfscope}%
\begin{pgfscope}%
\pgfpathrectangle{\pgfqpoint{0.150000in}{0.150000in}}{\pgfqpoint{2.700000in}{1.950000in}}%
\pgfusepath{clip}%
\pgfsetbuttcap%
\pgfsetroundjoin%
\definecolor{currentfill}{rgb}{0.925919,0.865579,0.870358}%
\pgfsetfillcolor{currentfill}%
\pgfsetlinewidth{0.000000pt}%
\definecolor{currentstroke}{rgb}{0.000000,0.000000,0.000000}%
\pgfsetstrokecolor{currentstroke}%
\pgfsetdash{}{0pt}%
\pgfpathmoveto{\pgfqpoint{1.132418in}{1.303643in}}%
\pgfpathlineto{\pgfqpoint{1.171432in}{1.274564in}}%
\pgfpathlineto{\pgfqpoint{1.135635in}{1.292957in}}%
\pgfpathlineto{\pgfqpoint{1.096141in}{1.330062in}}%
\pgfpathclose%
\pgfusepath{fill}%
\end{pgfscope}%
\begin{pgfscope}%
\pgfpathrectangle{\pgfqpoint{0.150000in}{0.150000in}}{\pgfqpoint{2.700000in}{1.950000in}}%
\pgfusepath{clip}%
\pgfsetbuttcap%
\pgfsetroundjoin%
\definecolor{currentfill}{rgb}{0.789154,0.617417,0.631020}%
\pgfsetfillcolor{currentfill}%
\pgfsetlinewidth{0.000000pt}%
\definecolor{currentstroke}{rgb}{0.000000,0.000000,0.000000}%
\pgfsetstrokecolor{currentstroke}%
\pgfsetdash{}{0pt}%
\pgfpathmoveto{\pgfqpoint{1.975895in}{1.008041in}}%
\pgfpathlineto{\pgfqpoint{2.011317in}{1.019254in}}%
\pgfpathlineto{\pgfqpoint{1.975640in}{1.076765in}}%
\pgfpathlineto{\pgfqpoint{1.940059in}{1.065675in}}%
\pgfpathclose%
\pgfusepath{fill}%
\end{pgfscope}%
\begin{pgfscope}%
\pgfpathrectangle{\pgfqpoint{0.150000in}{0.150000in}}{\pgfqpoint{2.700000in}{1.950000in}}%
\pgfusepath{clip}%
\pgfsetbuttcap%
\pgfsetroundjoin%
\definecolor{currentfill}{rgb}{0.770159,0.582950,0.597779}%
\pgfsetfillcolor{currentfill}%
\pgfsetlinewidth{0.000000pt}%
\definecolor{currentstroke}{rgb}{0.000000,0.000000,0.000000}%
\pgfsetstrokecolor{currentstroke}%
\pgfsetdash{}{0pt}%
\pgfpathmoveto{\pgfqpoint{2.309115in}{1.019998in}}%
\pgfpathlineto{\pgfqpoint{2.343865in}{1.031234in}}%
\pgfpathlineto{\pgfqpoint{2.301257in}{0.988534in}}%
\pgfpathlineto{\pgfqpoint{2.267207in}{0.984951in}}%
\pgfpathclose%
\pgfusepath{fill}%
\end{pgfscope}%
\begin{pgfscope}%
\pgfpathrectangle{\pgfqpoint{0.150000in}{0.150000in}}{\pgfqpoint{2.700000in}{1.950000in}}%
\pgfusepath{clip}%
\pgfsetbuttcap%
\pgfsetroundjoin%
\definecolor{currentfill}{rgb}{0.872733,0.769072,0.777282}%
\pgfsetfillcolor{currentfill}%
\pgfsetlinewidth{0.000000pt}%
\definecolor{currentstroke}{rgb}{0.000000,0.000000,0.000000}%
\pgfsetstrokecolor{currentstroke}%
\pgfsetdash{}{0pt}%
\pgfpathmoveto{\pgfqpoint{1.975711in}{1.153221in}}%
\pgfpathlineto{\pgfqpoint{2.011119in}{1.164064in}}%
\pgfpathlineto{\pgfqpoint{1.976108in}{1.237583in}}%
\pgfpathlineto{\pgfqpoint{1.940189in}{1.218994in}}%
\pgfpathclose%
\pgfusepath{fill}%
\end{pgfscope}%
\begin{pgfscope}%
\pgfpathrectangle{\pgfqpoint{0.150000in}{0.150000in}}{\pgfqpoint{2.700000in}{1.950000in}}%
\pgfusepath{clip}%
\pgfsetbuttcap%
\pgfsetroundjoin%
\definecolor{currentfill}{rgb}{0.739767,0.527803,0.544593}%
\pgfsetfillcolor{currentfill}%
\pgfsetlinewidth{0.000000pt}%
\definecolor{currentstroke}{rgb}{0.000000,0.000000,0.000000}%
\pgfsetstrokecolor{currentstroke}%
\pgfsetdash{}{0pt}%
\pgfpathmoveto{\pgfqpoint{2.013636in}{0.920116in}}%
\pgfpathlineto{\pgfqpoint{2.048288in}{0.916288in}}%
\pgfpathlineto{\pgfqpoint{2.011945in}{0.958291in}}%
\pgfpathlineto{\pgfqpoint{1.976804in}{0.954595in}}%
\pgfpathclose%
\pgfusepath{fill}%
\end{pgfscope}%
\begin{pgfscope}%
\pgfpathrectangle{\pgfqpoint{0.150000in}{0.150000in}}{\pgfqpoint{2.700000in}{1.950000in}}%
\pgfusepath{clip}%
\pgfsetbuttcap%
\pgfsetroundjoin%
\definecolor{currentfill}{rgb}{0.899326,0.817325,0.823820}%
\pgfsetfillcolor{currentfill}%
\pgfsetlinewidth{0.000000pt}%
\definecolor{currentstroke}{rgb}{0.000000,0.000000,0.000000}%
\pgfsetstrokecolor{currentstroke}%
\pgfsetdash{}{0pt}%
\pgfpathmoveto{\pgfqpoint{1.940189in}{1.218994in}}%
\pgfpathlineto{\pgfqpoint{1.976108in}{1.237583in}}%
\pgfpathlineto{\pgfqpoint{1.938758in}{1.256107in}}%
\pgfpathlineto{\pgfqpoint{1.902837in}{1.237583in}}%
\pgfpathclose%
\pgfusepath{fill}%
\end{pgfscope}%
\begin{pgfscope}%
\pgfpathrectangle{\pgfqpoint{0.150000in}{0.150000in}}{\pgfqpoint{2.700000in}{1.950000in}}%
\pgfusepath{clip}%
\pgfsetbuttcap%
\pgfsetroundjoin%
\definecolor{currentfill}{rgb}{0.899326,0.817325,0.823820}%
\pgfsetfillcolor{currentfill}%
\pgfsetlinewidth{0.000000pt}%
\definecolor{currentstroke}{rgb}{0.000000,0.000000,0.000000}%
\pgfsetstrokecolor{currentstroke}%
\pgfsetdash{}{0pt}%
\pgfpathmoveto{\pgfqpoint{1.866789in}{1.218994in}}%
\pgfpathlineto{\pgfqpoint{1.902837in}{1.237583in}}%
\pgfpathlineto{\pgfqpoint{1.865618in}{1.256107in}}%
\pgfpathlineto{\pgfqpoint{1.829567in}{1.237583in}}%
\pgfpathclose%
\pgfusepath{fill}%
\end{pgfscope}%
\begin{pgfscope}%
\pgfpathrectangle{\pgfqpoint{0.150000in}{0.150000in}}{\pgfqpoint{2.700000in}{1.950000in}}%
\pgfusepath{clip}%
\pgfsetbuttcap%
\pgfsetroundjoin%
\definecolor{currentfill}{rgb}{0.899326,0.817325,0.823820}%
\pgfsetfillcolor{currentfill}%
\pgfsetlinewidth{0.000000pt}%
\definecolor{currentstroke}{rgb}{0.000000,0.000000,0.000000}%
\pgfsetstrokecolor{currentstroke}%
\pgfsetdash{}{0pt}%
\pgfpathmoveto{\pgfqpoint{1.793388in}{1.218994in}}%
\pgfpathlineto{\pgfqpoint{1.829567in}{1.237583in}}%
\pgfpathlineto{\pgfqpoint{1.792478in}{1.256107in}}%
\pgfpathlineto{\pgfqpoint{1.756297in}{1.237583in}}%
\pgfpathclose%
\pgfusepath{fill}%
\end{pgfscope}%
\begin{pgfscope}%
\pgfpathrectangle{\pgfqpoint{0.150000in}{0.150000in}}{\pgfqpoint{2.700000in}{1.950000in}}%
\pgfusepath{clip}%
\pgfsetbuttcap%
\pgfsetroundjoin%
\definecolor{currentfill}{rgb}{0.899326,0.817325,0.823820}%
\pgfsetfillcolor{currentfill}%
\pgfsetlinewidth{0.000000pt}%
\definecolor{currentstroke}{rgb}{0.000000,0.000000,0.000000}%
\pgfsetstrokecolor{currentstroke}%
\pgfsetdash{}{0pt}%
\pgfpathmoveto{\pgfqpoint{1.719988in}{1.218994in}}%
\pgfpathlineto{\pgfqpoint{1.756297in}{1.237583in}}%
\pgfpathlineto{\pgfqpoint{1.719337in}{1.256107in}}%
\pgfpathlineto{\pgfqpoint{1.683027in}{1.237583in}}%
\pgfpathclose%
\pgfusepath{fill}%
\end{pgfscope}%
\begin{pgfscope}%
\pgfpathrectangle{\pgfqpoint{0.150000in}{0.150000in}}{\pgfqpoint{2.700000in}{1.950000in}}%
\pgfusepath{clip}%
\pgfsetbuttcap%
\pgfsetroundjoin%
\definecolor{currentfill}{rgb}{0.899326,0.817325,0.823820}%
\pgfsetfillcolor{currentfill}%
\pgfsetlinewidth{0.000000pt}%
\definecolor{currentstroke}{rgb}{0.000000,0.000000,0.000000}%
\pgfsetstrokecolor{currentstroke}%
\pgfsetdash{}{0pt}%
\pgfpathmoveto{\pgfqpoint{1.646587in}{1.218994in}}%
\pgfpathlineto{\pgfqpoint{1.683027in}{1.237583in}}%
\pgfpathlineto{\pgfqpoint{1.646197in}{1.256107in}}%
\pgfpathlineto{\pgfqpoint{1.609757in}{1.237583in}}%
\pgfpathclose%
\pgfusepath{fill}%
\end{pgfscope}%
\begin{pgfscope}%
\pgfpathrectangle{\pgfqpoint{0.150000in}{0.150000in}}{\pgfqpoint{2.700000in}{1.950000in}}%
\pgfusepath{clip}%
\pgfsetbuttcap%
\pgfsetroundjoin%
\definecolor{currentfill}{rgb}{0.899326,0.817325,0.823820}%
\pgfsetfillcolor{currentfill}%
\pgfsetlinewidth{0.000000pt}%
\definecolor{currentstroke}{rgb}{0.000000,0.000000,0.000000}%
\pgfsetstrokecolor{currentstroke}%
\pgfsetdash{}{0pt}%
\pgfpathmoveto{\pgfqpoint{1.573187in}{1.218994in}}%
\pgfpathlineto{\pgfqpoint{1.609757in}{1.237583in}}%
\pgfpathlineto{\pgfqpoint{1.573057in}{1.256107in}}%
\pgfpathlineto{\pgfqpoint{1.536486in}{1.237583in}}%
\pgfpathclose%
\pgfusepath{fill}%
\end{pgfscope}%
\begin{pgfscope}%
\pgfpathrectangle{\pgfqpoint{0.150000in}{0.150000in}}{\pgfqpoint{2.700000in}{1.950000in}}%
\pgfusepath{clip}%
\pgfsetbuttcap%
\pgfsetroundjoin%
\definecolor{currentfill}{rgb}{0.899326,0.817325,0.823820}%
\pgfsetfillcolor{currentfill}%
\pgfsetlinewidth{0.000000pt}%
\definecolor{currentstroke}{rgb}{0.000000,0.000000,0.000000}%
\pgfsetstrokecolor{currentstroke}%
\pgfsetdash{}{0pt}%
\pgfpathmoveto{\pgfqpoint{1.499786in}{1.218994in}}%
\pgfpathlineto{\pgfqpoint{1.536486in}{1.237583in}}%
\pgfpathlineto{\pgfqpoint{1.499916in}{1.256107in}}%
\pgfpathlineto{\pgfqpoint{1.463216in}{1.237583in}}%
\pgfpathclose%
\pgfusepath{fill}%
\end{pgfscope}%
\begin{pgfscope}%
\pgfpathrectangle{\pgfqpoint{0.150000in}{0.150000in}}{\pgfqpoint{2.700000in}{1.950000in}}%
\pgfusepath{clip}%
\pgfsetbuttcap%
\pgfsetroundjoin%
\definecolor{currentfill}{rgb}{0.899326,0.817325,0.823820}%
\pgfsetfillcolor{currentfill}%
\pgfsetlinewidth{0.000000pt}%
\definecolor{currentstroke}{rgb}{0.000000,0.000000,0.000000}%
\pgfsetstrokecolor{currentstroke}%
\pgfsetdash{}{0pt}%
\pgfpathmoveto{\pgfqpoint{1.426386in}{1.218994in}}%
\pgfpathlineto{\pgfqpoint{1.463216in}{1.237583in}}%
\pgfpathlineto{\pgfqpoint{1.426776in}{1.256107in}}%
\pgfpathlineto{\pgfqpoint{1.389946in}{1.237583in}}%
\pgfpathclose%
\pgfusepath{fill}%
\end{pgfscope}%
\begin{pgfscope}%
\pgfpathrectangle{\pgfqpoint{0.150000in}{0.150000in}}{\pgfqpoint{2.700000in}{1.950000in}}%
\pgfusepath{clip}%
\pgfsetbuttcap%
\pgfsetroundjoin%
\definecolor{currentfill}{rgb}{0.899326,0.817325,0.823820}%
\pgfsetfillcolor{currentfill}%
\pgfsetlinewidth{0.000000pt}%
\definecolor{currentstroke}{rgb}{0.000000,0.000000,0.000000}%
\pgfsetstrokecolor{currentstroke}%
\pgfsetdash{}{0pt}%
\pgfpathmoveto{\pgfqpoint{1.352985in}{1.218994in}}%
\pgfpathlineto{\pgfqpoint{1.389946in}{1.237583in}}%
\pgfpathlineto{\pgfqpoint{1.353636in}{1.256107in}}%
\pgfpathlineto{\pgfqpoint{1.316676in}{1.237583in}}%
\pgfpathclose%
\pgfusepath{fill}%
\end{pgfscope}%
\begin{pgfscope}%
\pgfpathrectangle{\pgfqpoint{0.150000in}{0.150000in}}{\pgfqpoint{2.700000in}{1.950000in}}%
\pgfusepath{clip}%
\pgfsetbuttcap%
\pgfsetroundjoin%
\definecolor{currentfill}{rgb}{0.899326,0.817325,0.823820}%
\pgfsetfillcolor{currentfill}%
\pgfsetlinewidth{0.000000pt}%
\definecolor{currentstroke}{rgb}{0.000000,0.000000,0.000000}%
\pgfsetstrokecolor{currentstroke}%
\pgfsetdash{}{0pt}%
\pgfpathmoveto{\pgfqpoint{1.279585in}{1.218994in}}%
\pgfpathlineto{\pgfqpoint{1.316676in}{1.237583in}}%
\pgfpathlineto{\pgfqpoint{1.280495in}{1.256107in}}%
\pgfpathlineto{\pgfqpoint{1.243406in}{1.237583in}}%
\pgfpathclose%
\pgfusepath{fill}%
\end{pgfscope}%
\begin{pgfscope}%
\pgfpathrectangle{\pgfqpoint{0.150000in}{0.150000in}}{\pgfqpoint{2.700000in}{1.950000in}}%
\pgfusepath{clip}%
\pgfsetbuttcap%
\pgfsetroundjoin%
\definecolor{currentfill}{rgb}{0.925919,0.865579,0.870358}%
\pgfsetfillcolor{currentfill}%
\pgfsetlinewidth{0.000000pt}%
\definecolor{currentstroke}{rgb}{0.000000,0.000000,0.000000}%
\pgfsetstrokecolor{currentstroke}%
\pgfsetdash{}{0pt}%
\pgfpathmoveto{\pgfqpoint{1.168497in}{1.285122in}}%
\pgfpathlineto{\pgfqpoint{1.207355in}{1.256107in}}%
\pgfpathlineto{\pgfqpoint{1.171432in}{1.274564in}}%
\pgfpathlineto{\pgfqpoint{1.132418in}{1.303643in}}%
\pgfpathclose%
\pgfusepath{fill}%
\end{pgfscope}%
\begin{pgfscope}%
\pgfpathrectangle{\pgfqpoint{0.150000in}{0.150000in}}{\pgfqpoint{2.700000in}{1.950000in}}%
\pgfusepath{clip}%
\pgfsetbuttcap%
\pgfsetroundjoin%
\definecolor{currentfill}{rgb}{0.766360,0.576057,0.591131}%
\pgfsetfillcolor{currentfill}%
\pgfsetlinewidth{0.000000pt}%
\definecolor{currentstroke}{rgb}{0.000000,0.000000,0.000000}%
\pgfsetstrokecolor{currentstroke}%
\pgfsetdash{}{0pt}%
\pgfpathmoveto{\pgfqpoint{1.976804in}{0.954595in}}%
\pgfpathlineto{\pgfqpoint{2.011945in}{0.958291in}}%
\pgfpathlineto{\pgfqpoint{1.975895in}{1.008041in}}%
\pgfpathlineto{\pgfqpoint{1.940594in}{1.004503in}}%
\pgfpathclose%
\pgfusepath{fill}%
\end{pgfscope}%
\begin{pgfscope}%
\pgfpathrectangle{\pgfqpoint{0.150000in}{0.150000in}}{\pgfqpoint{2.700000in}{1.950000in}}%
\pgfusepath{clip}%
\pgfsetbuttcap%
\pgfsetroundjoin%
\definecolor{currentfill}{rgb}{0.830944,0.693244,0.704151}%
\pgfsetfillcolor{currentfill}%
\pgfsetlinewidth{0.000000pt}%
\definecolor{currentstroke}{rgb}{0.000000,0.000000,0.000000}%
\pgfsetstrokecolor{currentstroke}%
\pgfsetdash{}{0pt}%
\pgfpathmoveto{\pgfqpoint{1.940059in}{1.065675in}}%
\pgfpathlineto{\pgfqpoint{1.975640in}{1.076765in}}%
\pgfpathlineto{\pgfqpoint{1.939824in}{1.134499in}}%
\pgfpathlineto{\pgfqpoint{1.904083in}{1.123534in}}%
\pgfpathclose%
\pgfusepath{fill}%
\end{pgfscope}%
\begin{pgfscope}%
\pgfpathrectangle{\pgfqpoint{0.150000in}{0.150000in}}{\pgfqpoint{2.700000in}{1.950000in}}%
\pgfusepath{clip}%
\pgfsetbuttcap%
\pgfsetroundjoin%
\definecolor{currentfill}{rgb}{0.872733,0.769072,0.777282}%
\pgfsetfillcolor{currentfill}%
\pgfsetlinewidth{0.000000pt}%
\definecolor{currentstroke}{rgb}{0.000000,0.000000,0.000000}%
\pgfsetstrokecolor{currentstroke}%
\pgfsetdash{}{0pt}%
\pgfpathmoveto{\pgfqpoint{1.939824in}{1.134499in}}%
\pgfpathlineto{\pgfqpoint{1.975711in}{1.153221in}}%
\pgfpathlineto{\pgfqpoint{1.940189in}{1.218994in}}%
\pgfpathlineto{\pgfqpoint{1.904142in}{1.200339in}}%
\pgfpathclose%
\pgfusepath{fill}%
\end{pgfscope}%
\begin{pgfscope}%
\pgfpathrectangle{\pgfqpoint{0.150000in}{0.150000in}}{\pgfqpoint{2.700000in}{1.950000in}}%
\pgfusepath{clip}%
\pgfsetbuttcap%
\pgfsetroundjoin%
\definecolor{currentfill}{rgb}{0.754963,0.555377,0.571186}%
\pgfsetfillcolor{currentfill}%
\pgfsetlinewidth{0.000000pt}%
\definecolor{currentstroke}{rgb}{0.000000,0.000000,0.000000}%
\pgfsetstrokecolor{currentstroke}%
\pgfsetdash{}{0pt}%
\pgfpathmoveto{\pgfqpoint{2.162780in}{0.958592in}}%
\pgfpathlineto{\pgfqpoint{2.197946in}{0.969998in}}%
\pgfpathlineto{\pgfqpoint{2.157319in}{0.942979in}}%
\pgfpathlineto{\pgfqpoint{2.122705in}{0.939238in}}%
\pgfpathclose%
\pgfusepath{fill}%
\end{pgfscope}%
\begin{pgfscope}%
\pgfpathrectangle{\pgfqpoint{0.150000in}{0.150000in}}{\pgfqpoint{2.700000in}{1.950000in}}%
\pgfusepath{clip}%
\pgfsetbuttcap%
\pgfsetroundjoin%
\definecolor{currentfill}{rgb}{0.899326,0.817325,0.823820}%
\pgfsetfillcolor{currentfill}%
\pgfsetlinewidth{0.000000pt}%
\definecolor{currentstroke}{rgb}{0.000000,0.000000,0.000000}%
\pgfsetstrokecolor{currentstroke}%
\pgfsetdash{}{0pt}%
\pgfpathmoveto{\pgfqpoint{1.904142in}{1.200339in}}%
\pgfpathlineto{\pgfqpoint{1.940189in}{1.218994in}}%
\pgfpathlineto{\pgfqpoint{1.902837in}{1.237583in}}%
\pgfpathlineto{\pgfqpoint{1.866789in}{1.218994in}}%
\pgfpathclose%
\pgfusepath{fill}%
\end{pgfscope}%
\begin{pgfscope}%
\pgfpathrectangle{\pgfqpoint{0.150000in}{0.150000in}}{\pgfqpoint{2.700000in}{1.950000in}}%
\pgfusepath{clip}%
\pgfsetbuttcap%
\pgfsetroundjoin%
\definecolor{currentfill}{rgb}{0.899326,0.817325,0.823820}%
\pgfsetfillcolor{currentfill}%
\pgfsetlinewidth{0.000000pt}%
\definecolor{currentstroke}{rgb}{0.000000,0.000000,0.000000}%
\pgfsetstrokecolor{currentstroke}%
\pgfsetdash{}{0pt}%
\pgfpathmoveto{\pgfqpoint{1.830611in}{1.200339in}}%
\pgfpathlineto{\pgfqpoint{1.866789in}{1.218994in}}%
\pgfpathlineto{\pgfqpoint{1.829567in}{1.237583in}}%
\pgfpathlineto{\pgfqpoint{1.793388in}{1.218994in}}%
\pgfpathclose%
\pgfusepath{fill}%
\end{pgfscope}%
\begin{pgfscope}%
\pgfpathrectangle{\pgfqpoint{0.150000in}{0.150000in}}{\pgfqpoint{2.700000in}{1.950000in}}%
\pgfusepath{clip}%
\pgfsetbuttcap%
\pgfsetroundjoin%
\definecolor{currentfill}{rgb}{0.899326,0.817325,0.823820}%
\pgfsetfillcolor{currentfill}%
\pgfsetlinewidth{0.000000pt}%
\definecolor{currentstroke}{rgb}{0.000000,0.000000,0.000000}%
\pgfsetstrokecolor{currentstroke}%
\pgfsetdash{}{0pt}%
\pgfpathmoveto{\pgfqpoint{1.757080in}{1.200339in}}%
\pgfpathlineto{\pgfqpoint{1.793388in}{1.218994in}}%
\pgfpathlineto{\pgfqpoint{1.756297in}{1.237583in}}%
\pgfpathlineto{\pgfqpoint{1.719988in}{1.218994in}}%
\pgfpathclose%
\pgfusepath{fill}%
\end{pgfscope}%
\begin{pgfscope}%
\pgfpathrectangle{\pgfqpoint{0.150000in}{0.150000in}}{\pgfqpoint{2.700000in}{1.950000in}}%
\pgfusepath{clip}%
\pgfsetbuttcap%
\pgfsetroundjoin%
\definecolor{currentfill}{rgb}{0.899326,0.817325,0.823820}%
\pgfsetfillcolor{currentfill}%
\pgfsetlinewidth{0.000000pt}%
\definecolor{currentstroke}{rgb}{0.000000,0.000000,0.000000}%
\pgfsetstrokecolor{currentstroke}%
\pgfsetdash{}{0pt}%
\pgfpathmoveto{\pgfqpoint{1.683549in}{1.200339in}}%
\pgfpathlineto{\pgfqpoint{1.719988in}{1.218994in}}%
\pgfpathlineto{\pgfqpoint{1.683027in}{1.237583in}}%
\pgfpathlineto{\pgfqpoint{1.646587in}{1.218994in}}%
\pgfpathclose%
\pgfusepath{fill}%
\end{pgfscope}%
\begin{pgfscope}%
\pgfpathrectangle{\pgfqpoint{0.150000in}{0.150000in}}{\pgfqpoint{2.700000in}{1.950000in}}%
\pgfusepath{clip}%
\pgfsetbuttcap%
\pgfsetroundjoin%
\definecolor{currentfill}{rgb}{0.899326,0.817325,0.823820}%
\pgfsetfillcolor{currentfill}%
\pgfsetlinewidth{0.000000pt}%
\definecolor{currentstroke}{rgb}{0.000000,0.000000,0.000000}%
\pgfsetstrokecolor{currentstroke}%
\pgfsetdash{}{0pt}%
\pgfpathmoveto{\pgfqpoint{1.610018in}{1.200339in}}%
\pgfpathlineto{\pgfqpoint{1.646587in}{1.218994in}}%
\pgfpathlineto{\pgfqpoint{1.609757in}{1.237583in}}%
\pgfpathlineto{\pgfqpoint{1.573187in}{1.218994in}}%
\pgfpathclose%
\pgfusepath{fill}%
\end{pgfscope}%
\begin{pgfscope}%
\pgfpathrectangle{\pgfqpoint{0.150000in}{0.150000in}}{\pgfqpoint{2.700000in}{1.950000in}}%
\pgfusepath{clip}%
\pgfsetbuttcap%
\pgfsetroundjoin%
\definecolor{currentfill}{rgb}{0.899326,0.817325,0.823820}%
\pgfsetfillcolor{currentfill}%
\pgfsetlinewidth{0.000000pt}%
\definecolor{currentstroke}{rgb}{0.000000,0.000000,0.000000}%
\pgfsetstrokecolor{currentstroke}%
\pgfsetdash{}{0pt}%
\pgfpathmoveto{\pgfqpoint{1.536486in}{1.200339in}}%
\pgfpathlineto{\pgfqpoint{1.573187in}{1.218994in}}%
\pgfpathlineto{\pgfqpoint{1.536486in}{1.237583in}}%
\pgfpathlineto{\pgfqpoint{1.499786in}{1.218994in}}%
\pgfpathclose%
\pgfusepath{fill}%
\end{pgfscope}%
\begin{pgfscope}%
\pgfpathrectangle{\pgfqpoint{0.150000in}{0.150000in}}{\pgfqpoint{2.700000in}{1.950000in}}%
\pgfusepath{clip}%
\pgfsetbuttcap%
\pgfsetroundjoin%
\definecolor{currentfill}{rgb}{0.899326,0.817325,0.823820}%
\pgfsetfillcolor{currentfill}%
\pgfsetlinewidth{0.000000pt}%
\definecolor{currentstroke}{rgb}{0.000000,0.000000,0.000000}%
\pgfsetstrokecolor{currentstroke}%
\pgfsetdash{}{0pt}%
\pgfpathmoveto{\pgfqpoint{1.462955in}{1.200339in}}%
\pgfpathlineto{\pgfqpoint{1.499786in}{1.218994in}}%
\pgfpathlineto{\pgfqpoint{1.463216in}{1.237583in}}%
\pgfpathlineto{\pgfqpoint{1.426386in}{1.218994in}}%
\pgfpathclose%
\pgfusepath{fill}%
\end{pgfscope}%
\begin{pgfscope}%
\pgfpathrectangle{\pgfqpoint{0.150000in}{0.150000in}}{\pgfqpoint{2.700000in}{1.950000in}}%
\pgfusepath{clip}%
\pgfsetbuttcap%
\pgfsetroundjoin%
\definecolor{currentfill}{rgb}{0.899326,0.817325,0.823820}%
\pgfsetfillcolor{currentfill}%
\pgfsetlinewidth{0.000000pt}%
\definecolor{currentstroke}{rgb}{0.000000,0.000000,0.000000}%
\pgfsetstrokecolor{currentstroke}%
\pgfsetdash{}{0pt}%
\pgfpathmoveto{\pgfqpoint{1.389424in}{1.200339in}}%
\pgfpathlineto{\pgfqpoint{1.426386in}{1.218994in}}%
\pgfpathlineto{\pgfqpoint{1.389946in}{1.237583in}}%
\pgfpathlineto{\pgfqpoint{1.352985in}{1.218994in}}%
\pgfpathclose%
\pgfusepath{fill}%
\end{pgfscope}%
\begin{pgfscope}%
\pgfpathrectangle{\pgfqpoint{0.150000in}{0.150000in}}{\pgfqpoint{2.700000in}{1.950000in}}%
\pgfusepath{clip}%
\pgfsetbuttcap%
\pgfsetroundjoin%
\definecolor{currentfill}{rgb}{0.899326,0.817325,0.823820}%
\pgfsetfillcolor{currentfill}%
\pgfsetlinewidth{0.000000pt}%
\definecolor{currentstroke}{rgb}{0.000000,0.000000,0.000000}%
\pgfsetstrokecolor{currentstroke}%
\pgfsetdash{}{0pt}%
\pgfpathmoveto{\pgfqpoint{1.315893in}{1.200339in}}%
\pgfpathlineto{\pgfqpoint{1.352985in}{1.218994in}}%
\pgfpathlineto{\pgfqpoint{1.316676in}{1.237583in}}%
\pgfpathlineto{\pgfqpoint{1.279585in}{1.218994in}}%
\pgfpathclose%
\pgfusepath{fill}%
\end{pgfscope}%
\begin{pgfscope}%
\pgfpathrectangle{\pgfqpoint{0.150000in}{0.150000in}}{\pgfqpoint{2.700000in}{1.950000in}}%
\pgfusepath{clip}%
\pgfsetbuttcap%
\pgfsetroundjoin%
\definecolor{currentfill}{rgb}{0.800551,0.638097,0.650965}%
\pgfsetfillcolor{currentfill}%
\pgfsetlinewidth{0.000000pt}%
\definecolor{currentstroke}{rgb}{0.000000,0.000000,0.000000}%
\pgfsetstrokecolor{currentstroke}%
\pgfsetdash{}{0pt}%
\pgfpathmoveto{\pgfqpoint{1.940594in}{1.004503in}}%
\pgfpathlineto{\pgfqpoint{1.975895in}{1.008041in}}%
\pgfpathlineto{\pgfqpoint{1.940059in}{1.065675in}}%
\pgfpathlineto{\pgfqpoint{1.904297in}{1.054530in}}%
\pgfpathclose%
\pgfusepath{fill}%
\end{pgfscope}%
\begin{pgfscope}%
\pgfpathrectangle{\pgfqpoint{0.150000in}{0.150000in}}{\pgfqpoint{2.700000in}{1.950000in}}%
\pgfusepath{clip}%
\pgfsetbuttcap%
\pgfsetroundjoin%
\definecolor{currentfill}{rgb}{0.781556,0.603631,0.617724}%
\pgfsetfillcolor{currentfill}%
\pgfsetlinewidth{0.000000pt}%
\definecolor{currentstroke}{rgb}{0.000000,0.000000,0.000000}%
\pgfsetstrokecolor{currentstroke}%
\pgfsetdash{}{0pt}%
\pgfpathmoveto{\pgfqpoint{2.274188in}{1.008706in}}%
\pgfpathlineto{\pgfqpoint{2.309115in}{1.019998in}}%
\pgfpathlineto{\pgfqpoint{2.267207in}{0.984951in}}%
\pgfpathlineto{\pgfqpoint{2.232418in}{0.973649in}}%
\pgfpathclose%
\pgfusepath{fill}%
\end{pgfscope}%
\begin{pgfscope}%
\pgfpathrectangle{\pgfqpoint{0.150000in}{0.150000in}}{\pgfqpoint{2.700000in}{1.950000in}}%
\pgfusepath{clip}%
\pgfsetbuttcap%
\pgfsetroundjoin%
\definecolor{currentfill}{rgb}{0.925919,0.865579,0.870358}%
\pgfsetfillcolor{currentfill}%
\pgfsetlinewidth{0.000000pt}%
\definecolor{currentstroke}{rgb}{0.000000,0.000000,0.000000}%
\pgfsetstrokecolor{currentstroke}%
\pgfsetdash{}{0pt}%
\pgfpathmoveto{\pgfqpoint{1.204704in}{1.266535in}}%
\pgfpathlineto{\pgfqpoint{1.243406in}{1.237583in}}%
\pgfpathlineto{\pgfqpoint{1.207355in}{1.256107in}}%
\pgfpathlineto{\pgfqpoint{1.168497in}{1.285122in}}%
\pgfpathclose%
\pgfusepath{fill}%
\end{pgfscope}%
\begin{pgfscope}%
\pgfpathrectangle{\pgfqpoint{0.150000in}{0.150000in}}{\pgfqpoint{2.700000in}{1.950000in}}%
\pgfusepath{clip}%
\pgfsetbuttcap%
\pgfsetroundjoin%
\definecolor{currentfill}{rgb}{0.747365,0.541590,0.557889}%
\pgfsetfillcolor{currentfill}%
\pgfsetlinewidth{0.000000pt}%
\definecolor{currentstroke}{rgb}{0.000000,0.000000,0.000000}%
\pgfsetstrokecolor{currentstroke}%
\pgfsetdash{}{0pt}%
\pgfpathmoveto{\pgfqpoint{2.052793in}{0.931682in}}%
\pgfpathlineto{\pgfqpoint{2.087863in}{0.935472in}}%
\pgfpathlineto{\pgfqpoint{2.048288in}{0.916288in}}%
\pgfpathlineto{\pgfqpoint{2.013636in}{0.920116in}}%
\pgfpathclose%
\pgfusepath{fill}%
\end{pgfscope}%
\begin{pgfscope}%
\pgfpathrectangle{\pgfqpoint{0.150000in}{0.150000in}}{\pgfqpoint{2.700000in}{1.950000in}}%
\pgfusepath{clip}%
\pgfsetbuttcap%
\pgfsetroundjoin%
\definecolor{currentfill}{rgb}{0.876532,0.775965,0.783931}%
\pgfsetfillcolor{currentfill}%
\pgfsetlinewidth{0.000000pt}%
\definecolor{currentstroke}{rgb}{0.000000,0.000000,0.000000}%
\pgfsetstrokecolor{currentstroke}%
\pgfsetdash{}{0pt}%
\pgfpathmoveto{\pgfqpoint{1.904083in}{1.123534in}}%
\pgfpathlineto{\pgfqpoint{1.939824in}{1.134499in}}%
\pgfpathlineto{\pgfqpoint{1.904142in}{1.200339in}}%
\pgfpathlineto{\pgfqpoint{1.867967in}{1.181617in}}%
\pgfpathclose%
\pgfusepath{fill}%
\end{pgfscope}%
\begin{pgfscope}%
\pgfpathrectangle{\pgfqpoint{0.150000in}{0.150000in}}{\pgfqpoint{2.700000in}{1.950000in}}%
\pgfusepath{clip}%
\pgfsetbuttcap%
\pgfsetroundjoin%
\definecolor{currentfill}{rgb}{0.899326,0.817325,0.823820}%
\pgfsetfillcolor{currentfill}%
\pgfsetlinewidth{0.000000pt}%
\definecolor{currentstroke}{rgb}{0.000000,0.000000,0.000000}%
\pgfsetstrokecolor{currentstroke}%
\pgfsetdash{}{0pt}%
\pgfpathmoveto{\pgfqpoint{1.867967in}{1.181617in}}%
\pgfpathlineto{\pgfqpoint{1.904142in}{1.200339in}}%
\pgfpathlineto{\pgfqpoint{1.866789in}{1.218994in}}%
\pgfpathlineto{\pgfqpoint{1.830611in}{1.200339in}}%
\pgfpathclose%
\pgfusepath{fill}%
\end{pgfscope}%
\begin{pgfscope}%
\pgfpathrectangle{\pgfqpoint{0.150000in}{0.150000in}}{\pgfqpoint{2.700000in}{1.950000in}}%
\pgfusepath{clip}%
\pgfsetbuttcap%
\pgfsetroundjoin%
\definecolor{currentfill}{rgb}{0.899326,0.817325,0.823820}%
\pgfsetfillcolor{currentfill}%
\pgfsetlinewidth{0.000000pt}%
\definecolor{currentstroke}{rgb}{0.000000,0.000000,0.000000}%
\pgfsetstrokecolor{currentstroke}%
\pgfsetdash{}{0pt}%
\pgfpathmoveto{\pgfqpoint{1.794305in}{1.181617in}}%
\pgfpathlineto{\pgfqpoint{1.830611in}{1.200339in}}%
\pgfpathlineto{\pgfqpoint{1.793388in}{1.218994in}}%
\pgfpathlineto{\pgfqpoint{1.757080in}{1.200339in}}%
\pgfpathclose%
\pgfusepath{fill}%
\end{pgfscope}%
\begin{pgfscope}%
\pgfpathrectangle{\pgfqpoint{0.150000in}{0.150000in}}{\pgfqpoint{2.700000in}{1.950000in}}%
\pgfusepath{clip}%
\pgfsetbuttcap%
\pgfsetroundjoin%
\definecolor{currentfill}{rgb}{0.899326,0.817325,0.823820}%
\pgfsetfillcolor{currentfill}%
\pgfsetlinewidth{0.000000pt}%
\definecolor{currentstroke}{rgb}{0.000000,0.000000,0.000000}%
\pgfsetstrokecolor{currentstroke}%
\pgfsetdash{}{0pt}%
\pgfpathmoveto{\pgfqpoint{1.720643in}{1.181617in}}%
\pgfpathlineto{\pgfqpoint{1.757080in}{1.200339in}}%
\pgfpathlineto{\pgfqpoint{1.719988in}{1.218994in}}%
\pgfpathlineto{\pgfqpoint{1.683549in}{1.200339in}}%
\pgfpathclose%
\pgfusepath{fill}%
\end{pgfscope}%
\begin{pgfscope}%
\pgfpathrectangle{\pgfqpoint{0.150000in}{0.150000in}}{\pgfqpoint{2.700000in}{1.950000in}}%
\pgfusepath{clip}%
\pgfsetbuttcap%
\pgfsetroundjoin%
\definecolor{currentfill}{rgb}{0.899326,0.817325,0.823820}%
\pgfsetfillcolor{currentfill}%
\pgfsetlinewidth{0.000000pt}%
\definecolor{currentstroke}{rgb}{0.000000,0.000000,0.000000}%
\pgfsetstrokecolor{currentstroke}%
\pgfsetdash{}{0pt}%
\pgfpathmoveto{\pgfqpoint{1.646980in}{1.181617in}}%
\pgfpathlineto{\pgfqpoint{1.683549in}{1.200339in}}%
\pgfpathlineto{\pgfqpoint{1.646587in}{1.218994in}}%
\pgfpathlineto{\pgfqpoint{1.610018in}{1.200339in}}%
\pgfpathclose%
\pgfusepath{fill}%
\end{pgfscope}%
\begin{pgfscope}%
\pgfpathrectangle{\pgfqpoint{0.150000in}{0.150000in}}{\pgfqpoint{2.700000in}{1.950000in}}%
\pgfusepath{clip}%
\pgfsetbuttcap%
\pgfsetroundjoin%
\definecolor{currentfill}{rgb}{0.899326,0.817325,0.823820}%
\pgfsetfillcolor{currentfill}%
\pgfsetlinewidth{0.000000pt}%
\definecolor{currentstroke}{rgb}{0.000000,0.000000,0.000000}%
\pgfsetstrokecolor{currentstroke}%
\pgfsetdash{}{0pt}%
\pgfpathmoveto{\pgfqpoint{1.425993in}{1.181617in}}%
\pgfpathlineto{\pgfqpoint{1.462955in}{1.200339in}}%
\pgfpathlineto{\pgfqpoint{1.426386in}{1.218994in}}%
\pgfpathlineto{\pgfqpoint{1.389424in}{1.200339in}}%
\pgfpathclose%
\pgfusepath{fill}%
\end{pgfscope}%
\begin{pgfscope}%
\pgfpathrectangle{\pgfqpoint{0.150000in}{0.150000in}}{\pgfqpoint{2.700000in}{1.950000in}}%
\pgfusepath{clip}%
\pgfsetbuttcap%
\pgfsetroundjoin%
\definecolor{currentfill}{rgb}{0.899326,0.817325,0.823820}%
\pgfsetfillcolor{currentfill}%
\pgfsetlinewidth{0.000000pt}%
\definecolor{currentstroke}{rgb}{0.000000,0.000000,0.000000}%
\pgfsetstrokecolor{currentstroke}%
\pgfsetdash{}{0pt}%
\pgfpathmoveto{\pgfqpoint{1.573318in}{1.181617in}}%
\pgfpathlineto{\pgfqpoint{1.610018in}{1.200339in}}%
\pgfpathlineto{\pgfqpoint{1.573187in}{1.218994in}}%
\pgfpathlineto{\pgfqpoint{1.536486in}{1.200339in}}%
\pgfpathclose%
\pgfusepath{fill}%
\end{pgfscope}%
\begin{pgfscope}%
\pgfpathrectangle{\pgfqpoint{0.150000in}{0.150000in}}{\pgfqpoint{2.700000in}{1.950000in}}%
\pgfusepath{clip}%
\pgfsetbuttcap%
\pgfsetroundjoin%
\definecolor{currentfill}{rgb}{0.899326,0.817325,0.823820}%
\pgfsetfillcolor{currentfill}%
\pgfsetlinewidth{0.000000pt}%
\definecolor{currentstroke}{rgb}{0.000000,0.000000,0.000000}%
\pgfsetstrokecolor{currentstroke}%
\pgfsetdash{}{0pt}%
\pgfpathmoveto{\pgfqpoint{1.499655in}{1.181617in}}%
\pgfpathlineto{\pgfqpoint{1.536486in}{1.200339in}}%
\pgfpathlineto{\pgfqpoint{1.499786in}{1.218994in}}%
\pgfpathlineto{\pgfqpoint{1.462955in}{1.200339in}}%
\pgfpathclose%
\pgfusepath{fill}%
\end{pgfscope}%
\begin{pgfscope}%
\pgfpathrectangle{\pgfqpoint{0.150000in}{0.150000in}}{\pgfqpoint{2.700000in}{1.950000in}}%
\pgfusepath{clip}%
\pgfsetbuttcap%
\pgfsetroundjoin%
\definecolor{currentfill}{rgb}{0.899326,0.817325,0.823820}%
\pgfsetfillcolor{currentfill}%
\pgfsetlinewidth{0.000000pt}%
\definecolor{currentstroke}{rgb}{0.000000,0.000000,0.000000}%
\pgfsetstrokecolor{currentstroke}%
\pgfsetdash{}{0pt}%
\pgfpathmoveto{\pgfqpoint{1.352330in}{1.181617in}}%
\pgfpathlineto{\pgfqpoint{1.389424in}{1.200339in}}%
\pgfpathlineto{\pgfqpoint{1.352985in}{1.218994in}}%
\pgfpathlineto{\pgfqpoint{1.315893in}{1.200339in}}%
\pgfpathclose%
\pgfusepath{fill}%
\end{pgfscope}%
\begin{pgfscope}%
\pgfpathrectangle{\pgfqpoint{0.150000in}{0.150000in}}{\pgfqpoint{2.700000in}{1.950000in}}%
\pgfusepath{clip}%
\pgfsetbuttcap%
\pgfsetroundjoin%
\definecolor{currentfill}{rgb}{0.842341,0.713925,0.724096}%
\pgfsetfillcolor{currentfill}%
\pgfsetlinewidth{0.000000pt}%
\definecolor{currentstroke}{rgb}{0.000000,0.000000,0.000000}%
\pgfsetstrokecolor{currentstroke}%
\pgfsetdash{}{0pt}%
\pgfpathmoveto{\pgfqpoint{1.904297in}{1.054530in}}%
\pgfpathlineto{\pgfqpoint{1.940059in}{1.065675in}}%
\pgfpathlineto{\pgfqpoint{1.904083in}{1.123534in}}%
\pgfpathlineto{\pgfqpoint{1.868161in}{1.112513in}}%
\pgfpathclose%
\pgfusepath{fill}%
\end{pgfscope}%
\begin{pgfscope}%
\pgfpathrectangle{\pgfqpoint{0.150000in}{0.150000in}}{\pgfqpoint{2.700000in}{1.950000in}}%
\pgfusepath{clip}%
\pgfsetbuttcap%
\pgfsetroundjoin%
\definecolor{currentfill}{rgb}{0.762561,0.569164,0.584482}%
\pgfsetfillcolor{currentfill}%
\pgfsetlinewidth{0.000000pt}%
\definecolor{currentstroke}{rgb}{0.000000,0.000000,0.000000}%
\pgfsetstrokecolor{currentstroke}%
\pgfsetdash{}{0pt}%
\pgfpathmoveto{\pgfqpoint{1.978705in}{0.923976in}}%
\pgfpathlineto{\pgfqpoint{2.013636in}{0.920116in}}%
\pgfpathlineto{\pgfqpoint{1.976804in}{0.954595in}}%
\pgfpathlineto{\pgfqpoint{1.941735in}{0.958592in}}%
\pgfpathclose%
\pgfusepath{fill}%
\end{pgfscope}%
\begin{pgfscope}%
\pgfpathrectangle{\pgfqpoint{0.150000in}{0.150000in}}{\pgfqpoint{2.700000in}{1.950000in}}%
\pgfusepath{clip}%
\pgfsetbuttcap%
\pgfsetroundjoin%
\definecolor{currentfill}{rgb}{0.922120,0.858686,0.863710}%
\pgfsetfillcolor{currentfill}%
\pgfsetlinewidth{0.000000pt}%
\definecolor{currentstroke}{rgb}{0.000000,0.000000,0.000000}%
\pgfsetstrokecolor{currentstroke}%
\pgfsetdash{}{0pt}%
\pgfpathmoveto{\pgfqpoint{1.241262in}{1.239927in}}%
\pgfpathlineto{\pgfqpoint{1.279585in}{1.218994in}}%
\pgfpathlineto{\pgfqpoint{1.243406in}{1.237583in}}%
\pgfpathlineto{\pgfqpoint{1.204704in}{1.266535in}}%
\pgfpathclose%
\pgfusepath{fill}%
\end{pgfscope}%
\begin{pgfscope}%
\pgfpathrectangle{\pgfqpoint{0.150000in}{0.150000in}}{\pgfqpoint{2.700000in}{1.950000in}}%
\pgfusepath{clip}%
\pgfsetbuttcap%
\pgfsetroundjoin%
\definecolor{currentfill}{rgb}{0.785355,0.610524,0.624372}%
\pgfsetfillcolor{currentfill}%
\pgfsetlinewidth{0.000000pt}%
\definecolor{currentstroke}{rgb}{0.000000,0.000000,0.000000}%
\pgfsetstrokecolor{currentstroke}%
\pgfsetdash{}{0pt}%
\pgfpathmoveto{\pgfqpoint{1.941735in}{0.958592in}}%
\pgfpathlineto{\pgfqpoint{1.976804in}{0.954595in}}%
\pgfpathlineto{\pgfqpoint{1.940594in}{1.004503in}}%
\pgfpathlineto{\pgfqpoint{1.905062in}{1.000941in}}%
\pgfpathclose%
\pgfusepath{fill}%
\end{pgfscope}%
\begin{pgfscope}%
\pgfpathrectangle{\pgfqpoint{0.150000in}{0.150000in}}{\pgfqpoint{2.700000in}{1.950000in}}%
\pgfusepath{clip}%
\pgfsetbuttcap%
\pgfsetroundjoin%
\definecolor{currentfill}{rgb}{0.770159,0.582950,0.597779}%
\pgfsetfillcolor{currentfill}%
\pgfsetlinewidth{0.000000pt}%
\definecolor{currentstroke}{rgb}{0.000000,0.000000,0.000000}%
\pgfsetstrokecolor{currentstroke}%
\pgfsetdash{}{0pt}%
\pgfpathmoveto{\pgfqpoint{2.127877in}{0.954869in}}%
\pgfpathlineto{\pgfqpoint{2.162780in}{0.958592in}}%
\pgfpathlineto{\pgfqpoint{2.122705in}{0.939238in}}%
\pgfpathlineto{\pgfqpoint{2.087863in}{0.935472in}}%
\pgfpathclose%
\pgfusepath{fill}%
\end{pgfscope}%
\begin{pgfscope}%
\pgfpathrectangle{\pgfqpoint{0.150000in}{0.150000in}}{\pgfqpoint{2.700000in}{1.950000in}}%
\pgfusepath{clip}%
\pgfsetbuttcap%
\pgfsetroundjoin%
\definecolor{currentfill}{rgb}{0.792953,0.624311,0.637669}%
\pgfsetfillcolor{currentfill}%
\pgfsetlinewidth{0.000000pt}%
\definecolor{currentstroke}{rgb}{0.000000,0.000000,0.000000}%
\pgfsetstrokecolor{currentstroke}%
\pgfsetdash{}{0pt}%
\pgfpathmoveto{\pgfqpoint{2.239083in}{0.997356in}}%
\pgfpathlineto{\pgfqpoint{2.274188in}{1.008706in}}%
\pgfpathlineto{\pgfqpoint{2.232418in}{0.973649in}}%
\pgfpathlineto{\pgfqpoint{2.197946in}{0.969998in}}%
\pgfpathclose%
\pgfusepath{fill}%
\end{pgfscope}%
\begin{pgfscope}%
\pgfpathrectangle{\pgfqpoint{0.150000in}{0.150000in}}{\pgfqpoint{2.700000in}{1.950000in}}%
\pgfusepath{clip}%
\pgfsetbuttcap%
\pgfsetroundjoin%
\definecolor{currentfill}{rgb}{0.815748,0.665671,0.677558}%
\pgfsetfillcolor{currentfill}%
\pgfsetlinewidth{0.000000pt}%
\definecolor{currentstroke}{rgb}{0.000000,0.000000,0.000000}%
\pgfsetstrokecolor{currentstroke}%
\pgfsetdash{}{0pt}%
\pgfpathmoveto{\pgfqpoint{1.905062in}{1.000941in}}%
\pgfpathlineto{\pgfqpoint{1.940594in}{1.004503in}}%
\pgfpathlineto{\pgfqpoint{1.904297in}{1.054530in}}%
\pgfpathlineto{\pgfqpoint{1.868603in}{1.051129in}}%
\pgfpathclose%
\pgfusepath{fill}%
\end{pgfscope}%
\begin{pgfscope}%
\pgfpathrectangle{\pgfqpoint{0.150000in}{0.150000in}}{\pgfqpoint{2.700000in}{1.950000in}}%
\pgfusepath{clip}%
\pgfsetbuttcap%
\pgfsetroundjoin%
\definecolor{currentfill}{rgb}{0.899326,0.817325,0.823820}%
\pgfsetfillcolor{currentfill}%
\pgfsetlinewidth{0.000000pt}%
\definecolor{currentstroke}{rgb}{0.000000,0.000000,0.000000}%
\pgfsetstrokecolor{currentstroke}%
\pgfsetdash{}{0pt}%
\pgfpathmoveto{\pgfqpoint{1.831663in}{1.162828in}}%
\pgfpathlineto{\pgfqpoint{1.867967in}{1.181617in}}%
\pgfpathlineto{\pgfqpoint{1.830611in}{1.200339in}}%
\pgfpathlineto{\pgfqpoint{1.794305in}{1.181617in}}%
\pgfpathclose%
\pgfusepath{fill}%
\end{pgfscope}%
\begin{pgfscope}%
\pgfpathrectangle{\pgfqpoint{0.150000in}{0.150000in}}{\pgfqpoint{2.700000in}{1.950000in}}%
\pgfusepath{clip}%
\pgfsetbuttcap%
\pgfsetroundjoin%
\definecolor{currentfill}{rgb}{0.899326,0.817325,0.823820}%
\pgfsetfillcolor{currentfill}%
\pgfsetlinewidth{0.000000pt}%
\definecolor{currentstroke}{rgb}{0.000000,0.000000,0.000000}%
\pgfsetstrokecolor{currentstroke}%
\pgfsetdash{}{0pt}%
\pgfpathmoveto{\pgfqpoint{1.757869in}{1.162828in}}%
\pgfpathlineto{\pgfqpoint{1.794305in}{1.181617in}}%
\pgfpathlineto{\pgfqpoint{1.757080in}{1.200339in}}%
\pgfpathlineto{\pgfqpoint{1.720643in}{1.181617in}}%
\pgfpathclose%
\pgfusepath{fill}%
\end{pgfscope}%
\begin{pgfscope}%
\pgfpathrectangle{\pgfqpoint{0.150000in}{0.150000in}}{\pgfqpoint{2.700000in}{1.950000in}}%
\pgfusepath{clip}%
\pgfsetbuttcap%
\pgfsetroundjoin%
\definecolor{currentfill}{rgb}{0.899326,0.817325,0.823820}%
\pgfsetfillcolor{currentfill}%
\pgfsetlinewidth{0.000000pt}%
\definecolor{currentstroke}{rgb}{0.000000,0.000000,0.000000}%
\pgfsetstrokecolor{currentstroke}%
\pgfsetdash{}{0pt}%
\pgfpathmoveto{\pgfqpoint{1.684075in}{1.162828in}}%
\pgfpathlineto{\pgfqpoint{1.720643in}{1.181617in}}%
\pgfpathlineto{\pgfqpoint{1.683549in}{1.200339in}}%
\pgfpathlineto{\pgfqpoint{1.646980in}{1.181617in}}%
\pgfpathclose%
\pgfusepath{fill}%
\end{pgfscope}%
\begin{pgfscope}%
\pgfpathrectangle{\pgfqpoint{0.150000in}{0.150000in}}{\pgfqpoint{2.700000in}{1.950000in}}%
\pgfusepath{clip}%
\pgfsetbuttcap%
\pgfsetroundjoin%
\definecolor{currentfill}{rgb}{0.899326,0.817325,0.823820}%
\pgfsetfillcolor{currentfill}%
\pgfsetlinewidth{0.000000pt}%
\definecolor{currentstroke}{rgb}{0.000000,0.000000,0.000000}%
\pgfsetstrokecolor{currentstroke}%
\pgfsetdash{}{0pt}%
\pgfpathmoveto{\pgfqpoint{1.610281in}{1.162828in}}%
\pgfpathlineto{\pgfqpoint{1.646980in}{1.181617in}}%
\pgfpathlineto{\pgfqpoint{1.610018in}{1.200339in}}%
\pgfpathlineto{\pgfqpoint{1.573318in}{1.181617in}}%
\pgfpathclose%
\pgfusepath{fill}%
\end{pgfscope}%
\begin{pgfscope}%
\pgfpathrectangle{\pgfqpoint{0.150000in}{0.150000in}}{\pgfqpoint{2.700000in}{1.950000in}}%
\pgfusepath{clip}%
\pgfsetbuttcap%
\pgfsetroundjoin%
\definecolor{currentfill}{rgb}{0.899326,0.817325,0.823820}%
\pgfsetfillcolor{currentfill}%
\pgfsetlinewidth{0.000000pt}%
\definecolor{currentstroke}{rgb}{0.000000,0.000000,0.000000}%
\pgfsetstrokecolor{currentstroke}%
\pgfsetdash{}{0pt}%
\pgfpathmoveto{\pgfqpoint{1.462692in}{1.162828in}}%
\pgfpathlineto{\pgfqpoint{1.499655in}{1.181617in}}%
\pgfpathlineto{\pgfqpoint{1.462955in}{1.200339in}}%
\pgfpathlineto{\pgfqpoint{1.425993in}{1.181617in}}%
\pgfpathclose%
\pgfusepath{fill}%
\end{pgfscope}%
\begin{pgfscope}%
\pgfpathrectangle{\pgfqpoint{0.150000in}{0.150000in}}{\pgfqpoint{2.700000in}{1.950000in}}%
\pgfusepath{clip}%
\pgfsetbuttcap%
\pgfsetroundjoin%
\definecolor{currentfill}{rgb}{0.899326,0.817325,0.823820}%
\pgfsetfillcolor{currentfill}%
\pgfsetlinewidth{0.000000pt}%
\definecolor{currentstroke}{rgb}{0.000000,0.000000,0.000000}%
\pgfsetstrokecolor{currentstroke}%
\pgfsetdash{}{0pt}%
\pgfpathmoveto{\pgfqpoint{1.388898in}{1.162828in}}%
\pgfpathlineto{\pgfqpoint{1.425993in}{1.181617in}}%
\pgfpathlineto{\pgfqpoint{1.389424in}{1.200339in}}%
\pgfpathlineto{\pgfqpoint{1.352330in}{1.181617in}}%
\pgfpathclose%
\pgfusepath{fill}%
\end{pgfscope}%
\begin{pgfscope}%
\pgfpathrectangle{\pgfqpoint{0.150000in}{0.150000in}}{\pgfqpoint{2.700000in}{1.950000in}}%
\pgfusepath{clip}%
\pgfsetbuttcap%
\pgfsetroundjoin%
\definecolor{currentfill}{rgb}{0.899326,0.817325,0.823820}%
\pgfsetfillcolor{currentfill}%
\pgfsetlinewidth{0.000000pt}%
\definecolor{currentstroke}{rgb}{0.000000,0.000000,0.000000}%
\pgfsetstrokecolor{currentstroke}%
\pgfsetdash{}{0pt}%
\pgfpathmoveto{\pgfqpoint{1.536486in}{1.162828in}}%
\pgfpathlineto{\pgfqpoint{1.573318in}{1.181617in}}%
\pgfpathlineto{\pgfqpoint{1.536486in}{1.200339in}}%
\pgfpathlineto{\pgfqpoint{1.499655in}{1.181617in}}%
\pgfpathclose%
\pgfusepath{fill}%
\end{pgfscope}%
\begin{pgfscope}%
\pgfpathrectangle{\pgfqpoint{0.150000in}{0.150000in}}{\pgfqpoint{2.700000in}{1.950000in}}%
\pgfusepath{clip}%
\pgfsetbuttcap%
\pgfsetroundjoin%
\definecolor{currentfill}{rgb}{0.922120,0.858686,0.863710}%
\pgfsetfillcolor{currentfill}%
\pgfsetlinewidth{0.000000pt}%
\definecolor{currentstroke}{rgb}{0.000000,0.000000,0.000000}%
\pgfsetstrokecolor{currentstroke}%
\pgfsetdash{}{0pt}%
\pgfpathmoveto{\pgfqpoint{1.277702in}{1.221206in}}%
\pgfpathlineto{\pgfqpoint{1.315893in}{1.200339in}}%
\pgfpathlineto{\pgfqpoint{1.279585in}{1.218994in}}%
\pgfpathlineto{\pgfqpoint{1.241262in}{1.239927in}}%
\pgfpathclose%
\pgfusepath{fill}%
\end{pgfscope}%
\begin{pgfscope}%
\pgfpathrectangle{\pgfqpoint{0.150000in}{0.150000in}}{\pgfqpoint{2.700000in}{1.950000in}}%
\pgfusepath{clip}%
\pgfsetbuttcap%
\pgfsetroundjoin%
\definecolor{currentfill}{rgb}{0.880331,0.782858,0.790579}%
\pgfsetfillcolor{currentfill}%
\pgfsetlinewidth{0.000000pt}%
\definecolor{currentstroke}{rgb}{0.000000,0.000000,0.000000}%
\pgfsetstrokecolor{currentstroke}%
\pgfsetdash{}{0pt}%
\pgfpathmoveto{\pgfqpoint{1.868161in}{1.112513in}}%
\pgfpathlineto{\pgfqpoint{1.904083in}{1.123534in}}%
\pgfpathlineto{\pgfqpoint{1.867967in}{1.181617in}}%
\pgfpathlineto{\pgfqpoint{1.831663in}{1.162828in}}%
\pgfpathclose%
\pgfusepath{fill}%
\end{pgfscope}%
\begin{pgfscope}%
\pgfpathrectangle{\pgfqpoint{0.150000in}{0.150000in}}{\pgfqpoint{2.700000in}{1.950000in}}%
\pgfusepath{clip}%
\pgfsetbuttcap%
\pgfsetroundjoin%
\definecolor{currentfill}{rgb}{0.766360,0.576057,0.591131}%
\pgfsetfillcolor{currentfill}%
\pgfsetlinewidth{0.000000pt}%
\definecolor{currentstroke}{rgb}{0.000000,0.000000,0.000000}%
\pgfsetstrokecolor{currentstroke}%
\pgfsetdash{}{0pt}%
\pgfpathmoveto{\pgfqpoint{2.017493in}{0.927867in}}%
\pgfpathlineto{\pgfqpoint{2.052793in}{0.931682in}}%
\pgfpathlineto{\pgfqpoint{2.013636in}{0.920116in}}%
\pgfpathlineto{\pgfqpoint{1.978705in}{0.923976in}}%
\pgfpathclose%
\pgfusepath{fill}%
\end{pgfscope}%
\begin{pgfscope}%
\pgfpathrectangle{\pgfqpoint{0.150000in}{0.150000in}}{\pgfqpoint{2.700000in}{1.950000in}}%
\pgfusepath{clip}%
\pgfsetbuttcap%
\pgfsetroundjoin%
\definecolor{currentfill}{rgb}{0.849939,0.727711,0.737393}%
\pgfsetfillcolor{currentfill}%
\pgfsetlinewidth{0.000000pt}%
\definecolor{currentstroke}{rgb}{0.000000,0.000000,0.000000}%
\pgfsetstrokecolor{currentstroke}%
\pgfsetdash{}{0pt}%
\pgfpathmoveto{\pgfqpoint{1.868603in}{1.051129in}}%
\pgfpathlineto{\pgfqpoint{1.904297in}{1.054530in}}%
\pgfpathlineto{\pgfqpoint{1.868161in}{1.112513in}}%
\pgfpathlineto{\pgfqpoint{1.831836in}{1.093590in}}%
\pgfpathclose%
\pgfusepath{fill}%
\end{pgfscope}%
\begin{pgfscope}%
\pgfpathrectangle{\pgfqpoint{0.150000in}{0.150000in}}{\pgfqpoint{2.700000in}{1.950000in}}%
\pgfusepath{clip}%
\pgfsetbuttcap%
\pgfsetroundjoin%
\definecolor{currentfill}{rgb}{0.834743,0.700138,0.710800}%
\pgfsetfillcolor{currentfill}%
\pgfsetlinewidth{0.000000pt}%
\definecolor{currentstroke}{rgb}{0.000000,0.000000,0.000000}%
\pgfsetstrokecolor{currentstroke}%
\pgfsetdash{}{0pt}%
\pgfpathmoveto{\pgfqpoint{2.351613in}{1.055539in}}%
\pgfpathlineto{\pgfqpoint{2.387137in}{1.074599in}}%
\pgfpathlineto{\pgfqpoint{2.343865in}{1.031234in}}%
\pgfpathlineto{\pgfqpoint{2.309115in}{1.019998in}}%
\pgfpathclose%
\pgfusepath{fill}%
\end{pgfscope}%
\begin{pgfscope}%
\pgfpathrectangle{\pgfqpoint{0.150000in}{0.150000in}}{\pgfqpoint{2.700000in}{1.950000in}}%
\pgfusepath{clip}%
\pgfsetbuttcap%
\pgfsetroundjoin%
\definecolor{currentfill}{rgb}{0.972013,0.975460,0.980285}%
\pgfsetfillcolor{currentfill}%
\pgfsetlinewidth{0.000000pt}%
\definecolor{currentstroke}{rgb}{0.000000,0.000000,0.000000}%
\pgfsetstrokecolor{currentstroke}%
\pgfsetdash{}{0pt}%
\pgfpathmoveto{\pgfqpoint{1.053560in}{1.424363in}}%
\pgfpathlineto{\pgfqpoint{1.096141in}{1.330062in}}%
\pgfpathlineto{\pgfqpoint{1.060292in}{1.348452in}}%
\pgfpathlineto{\pgfqpoint{1.016956in}{1.450954in}}%
\pgfpathclose%
\pgfusepath{fill}%
\end{pgfscope}%
\begin{pgfscope}%
\pgfpathrectangle{\pgfqpoint{0.150000in}{0.150000in}}{\pgfqpoint{2.700000in}{1.950000in}}%
\pgfusepath{clip}%
\pgfsetbuttcap%
\pgfsetroundjoin%
\definecolor{currentfill}{rgb}{0.899326,0.817325,0.823820}%
\pgfsetfillcolor{currentfill}%
\pgfsetlinewidth{0.000000pt}%
\definecolor{currentstroke}{rgb}{0.000000,0.000000,0.000000}%
\pgfsetstrokecolor{currentstroke}%
\pgfsetdash{}{0pt}%
\pgfpathmoveto{\pgfqpoint{1.795228in}{1.143972in}}%
\pgfpathlineto{\pgfqpoint{1.831663in}{1.162828in}}%
\pgfpathlineto{\pgfqpoint{1.794305in}{1.181617in}}%
\pgfpathlineto{\pgfqpoint{1.757869in}{1.162828in}}%
\pgfpathclose%
\pgfusepath{fill}%
\end{pgfscope}%
\begin{pgfscope}%
\pgfpathrectangle{\pgfqpoint{0.150000in}{0.150000in}}{\pgfqpoint{2.700000in}{1.950000in}}%
\pgfusepath{clip}%
\pgfsetbuttcap%
\pgfsetroundjoin%
\definecolor{currentfill}{rgb}{0.899326,0.817325,0.823820}%
\pgfsetfillcolor{currentfill}%
\pgfsetlinewidth{0.000000pt}%
\definecolor{currentstroke}{rgb}{0.000000,0.000000,0.000000}%
\pgfsetstrokecolor{currentstroke}%
\pgfsetdash{}{0pt}%
\pgfpathmoveto{\pgfqpoint{1.721302in}{1.143972in}}%
\pgfpathlineto{\pgfqpoint{1.757869in}{1.162828in}}%
\pgfpathlineto{\pgfqpoint{1.720643in}{1.181617in}}%
\pgfpathlineto{\pgfqpoint{1.684075in}{1.162828in}}%
\pgfpathclose%
\pgfusepath{fill}%
\end{pgfscope}%
\begin{pgfscope}%
\pgfpathrectangle{\pgfqpoint{0.150000in}{0.150000in}}{\pgfqpoint{2.700000in}{1.950000in}}%
\pgfusepath{clip}%
\pgfsetbuttcap%
\pgfsetroundjoin%
\definecolor{currentfill}{rgb}{0.899326,0.817325,0.823820}%
\pgfsetfillcolor{currentfill}%
\pgfsetlinewidth{0.000000pt}%
\definecolor{currentstroke}{rgb}{0.000000,0.000000,0.000000}%
\pgfsetstrokecolor{currentstroke}%
\pgfsetdash{}{0pt}%
\pgfpathmoveto{\pgfqpoint{1.647376in}{1.143972in}}%
\pgfpathlineto{\pgfqpoint{1.684075in}{1.162828in}}%
\pgfpathlineto{\pgfqpoint{1.646980in}{1.181617in}}%
\pgfpathlineto{\pgfqpoint{1.610281in}{1.162828in}}%
\pgfpathclose%
\pgfusepath{fill}%
\end{pgfscope}%
\begin{pgfscope}%
\pgfpathrectangle{\pgfqpoint{0.150000in}{0.150000in}}{\pgfqpoint{2.700000in}{1.950000in}}%
\pgfusepath{clip}%
\pgfsetbuttcap%
\pgfsetroundjoin%
\definecolor{currentfill}{rgb}{0.899326,0.817325,0.823820}%
\pgfsetfillcolor{currentfill}%
\pgfsetlinewidth{0.000000pt}%
\definecolor{currentstroke}{rgb}{0.000000,0.000000,0.000000}%
\pgfsetstrokecolor{currentstroke}%
\pgfsetdash{}{0pt}%
\pgfpathmoveto{\pgfqpoint{1.425597in}{1.143972in}}%
\pgfpathlineto{\pgfqpoint{1.462692in}{1.162828in}}%
\pgfpathlineto{\pgfqpoint{1.425993in}{1.181617in}}%
\pgfpathlineto{\pgfqpoint{1.388898in}{1.162828in}}%
\pgfpathclose%
\pgfusepath{fill}%
\end{pgfscope}%
\begin{pgfscope}%
\pgfpathrectangle{\pgfqpoint{0.150000in}{0.150000in}}{\pgfqpoint{2.700000in}{1.950000in}}%
\pgfusepath{clip}%
\pgfsetbuttcap%
\pgfsetroundjoin%
\definecolor{currentfill}{rgb}{0.899326,0.817325,0.823820}%
\pgfsetfillcolor{currentfill}%
\pgfsetlinewidth{0.000000pt}%
\definecolor{currentstroke}{rgb}{0.000000,0.000000,0.000000}%
\pgfsetstrokecolor{currentstroke}%
\pgfsetdash{}{0pt}%
\pgfpathmoveto{\pgfqpoint{1.573450in}{1.143972in}}%
\pgfpathlineto{\pgfqpoint{1.610281in}{1.162828in}}%
\pgfpathlineto{\pgfqpoint{1.573318in}{1.181617in}}%
\pgfpathlineto{\pgfqpoint{1.536486in}{1.162828in}}%
\pgfpathclose%
\pgfusepath{fill}%
\end{pgfscope}%
\begin{pgfscope}%
\pgfpathrectangle{\pgfqpoint{0.150000in}{0.150000in}}{\pgfqpoint{2.700000in}{1.950000in}}%
\pgfusepath{clip}%
\pgfsetbuttcap%
\pgfsetroundjoin%
\definecolor{currentfill}{rgb}{0.899326,0.817325,0.823820}%
\pgfsetfillcolor{currentfill}%
\pgfsetlinewidth{0.000000pt}%
\definecolor{currentstroke}{rgb}{0.000000,0.000000,0.000000}%
\pgfsetstrokecolor{currentstroke}%
\pgfsetdash{}{0pt}%
\pgfpathmoveto{\pgfqpoint{1.499523in}{1.143972in}}%
\pgfpathlineto{\pgfqpoint{1.536486in}{1.162828in}}%
\pgfpathlineto{\pgfqpoint{1.499655in}{1.181617in}}%
\pgfpathlineto{\pgfqpoint{1.462692in}{1.162828in}}%
\pgfpathclose%
\pgfusepath{fill}%
\end{pgfscope}%
\begin{pgfscope}%
\pgfpathrectangle{\pgfqpoint{0.150000in}{0.150000in}}{\pgfqpoint{2.700000in}{1.950000in}}%
\pgfusepath{clip}%
\pgfsetbuttcap%
\pgfsetroundjoin%
\definecolor{currentfill}{rgb}{0.922120,0.858686,0.863710}%
\pgfsetfillcolor{currentfill}%
\pgfsetlinewidth{0.000000pt}%
\definecolor{currentstroke}{rgb}{0.000000,0.000000,0.000000}%
\pgfsetstrokecolor{currentstroke}%
\pgfsetdash{}{0pt}%
\pgfpathmoveto{\pgfqpoint{1.314274in}{1.202418in}}%
\pgfpathlineto{\pgfqpoint{1.352330in}{1.181617in}}%
\pgfpathlineto{\pgfqpoint{1.315893in}{1.200339in}}%
\pgfpathlineto{\pgfqpoint{1.277702in}{1.221206in}}%
\pgfpathclose%
\pgfusepath{fill}%
\end{pgfscope}%
\begin{pgfscope}%
\pgfpathrectangle{\pgfqpoint{0.150000in}{0.150000in}}{\pgfqpoint{2.700000in}{1.950000in}}%
\pgfusepath{clip}%
\pgfsetbuttcap%
\pgfsetroundjoin%
\definecolor{currentfill}{rgb}{0.884130,0.789752,0.797227}%
\pgfsetfillcolor{currentfill}%
\pgfsetlinewidth{0.000000pt}%
\definecolor{currentstroke}{rgb}{0.000000,0.000000,0.000000}%
\pgfsetstrokecolor{currentstroke}%
\pgfsetdash{}{0pt}%
\pgfpathmoveto{\pgfqpoint{1.831836in}{1.093590in}}%
\pgfpathlineto{\pgfqpoint{1.868161in}{1.112513in}}%
\pgfpathlineto{\pgfqpoint{1.831663in}{1.162828in}}%
\pgfpathlineto{\pgfqpoint{1.795228in}{1.143972in}}%
\pgfpathclose%
\pgfusepath{fill}%
\end{pgfscope}%
\begin{pgfscope}%
\pgfpathrectangle{\pgfqpoint{0.150000in}{0.150000in}}{\pgfqpoint{2.700000in}{1.950000in}}%
\pgfusepath{clip}%
\pgfsetbuttcap%
\pgfsetroundjoin%
\definecolor{currentfill}{rgb}{0.804350,0.644991,0.657613}%
\pgfsetfillcolor{currentfill}%
\pgfsetlinewidth{0.000000pt}%
\definecolor{currentstroke}{rgb}{0.000000,0.000000,0.000000}%
\pgfsetstrokecolor{currentstroke}%
\pgfsetdash{}{0pt}%
\pgfpathmoveto{\pgfqpoint{1.906106in}{0.954869in}}%
\pgfpathlineto{\pgfqpoint{1.941735in}{0.958592in}}%
\pgfpathlineto{\pgfqpoint{1.905062in}{1.000941in}}%
\pgfpathlineto{\pgfqpoint{1.869296in}{0.997356in}}%
\pgfpathclose%
\pgfusepath{fill}%
\end{pgfscope}%
\begin{pgfscope}%
\pgfpathrectangle{\pgfqpoint{0.150000in}{0.150000in}}{\pgfqpoint{2.700000in}{1.950000in}}%
\pgfusepath{clip}%
\pgfsetbuttcap%
\pgfsetroundjoin%
\definecolor{currentfill}{rgb}{0.804350,0.644991,0.657613}%
\pgfsetfillcolor{currentfill}%
\pgfsetlinewidth{0.000000pt}%
\definecolor{currentstroke}{rgb}{0.000000,0.000000,0.000000}%
\pgfsetstrokecolor{currentstroke}%
\pgfsetdash{}{0pt}%
\pgfpathmoveto{\pgfqpoint{2.203800in}{0.985948in}}%
\pgfpathlineto{\pgfqpoint{2.239083in}{0.997356in}}%
\pgfpathlineto{\pgfqpoint{2.197946in}{0.969998in}}%
\pgfpathlineto{\pgfqpoint{2.162780in}{0.958592in}}%
\pgfpathclose%
\pgfusepath{fill}%
\end{pgfscope}%
\begin{pgfscope}%
\pgfpathrectangle{\pgfqpoint{0.150000in}{0.150000in}}{\pgfqpoint{2.700000in}{1.950000in}}%
\pgfusepath{clip}%
\pgfsetbuttcap%
\pgfsetroundjoin%
\definecolor{currentfill}{rgb}{0.785355,0.610524,0.624372}%
\pgfsetfillcolor{currentfill}%
\pgfsetlinewidth{0.000000pt}%
\definecolor{currentstroke}{rgb}{0.000000,0.000000,0.000000}%
\pgfsetstrokecolor{currentstroke}%
\pgfsetdash{}{0pt}%
\pgfpathmoveto{\pgfqpoint{1.943492in}{0.927867in}}%
\pgfpathlineto{\pgfqpoint{1.978705in}{0.923976in}}%
\pgfpathlineto{\pgfqpoint{1.941735in}{0.958592in}}%
\pgfpathlineto{\pgfqpoint{1.906106in}{0.954869in}}%
\pgfpathclose%
\pgfusepath{fill}%
\end{pgfscope}%
\begin{pgfscope}%
\pgfpathrectangle{\pgfqpoint{0.150000in}{0.150000in}}{\pgfqpoint{2.700000in}{1.950000in}}%
\pgfusepath{clip}%
\pgfsetbuttcap%
\pgfsetroundjoin%
\definecolor{currentfill}{rgb}{0.785355,0.610524,0.624372}%
\pgfsetfillcolor{currentfill}%
\pgfsetlinewidth{0.000000pt}%
\definecolor{currentstroke}{rgb}{0.000000,0.000000,0.000000}%
\pgfsetstrokecolor{currentstroke}%
\pgfsetdash{}{0pt}%
\pgfpathmoveto{\pgfqpoint{2.092327in}{0.943359in}}%
\pgfpathlineto{\pgfqpoint{2.127877in}{0.954869in}}%
\pgfpathlineto{\pgfqpoint{2.087863in}{0.935472in}}%
\pgfpathlineto{\pgfqpoint{2.052793in}{0.931682in}}%
\pgfpathclose%
\pgfusepath{fill}%
\end{pgfscope}%
\begin{pgfscope}%
\pgfpathrectangle{\pgfqpoint{0.150000in}{0.150000in}}{\pgfqpoint{2.700000in}{1.950000in}}%
\pgfusepath{clip}%
\pgfsetbuttcap%
\pgfsetroundjoin%
\definecolor{currentfill}{rgb}{0.830944,0.693244,0.704151}%
\pgfsetfillcolor{currentfill}%
\pgfsetlinewidth{0.000000pt}%
\definecolor{currentstroke}{rgb}{0.000000,0.000000,0.000000}%
\pgfsetstrokecolor{currentstroke}%
\pgfsetdash{}{0pt}%
\pgfpathmoveto{\pgfqpoint{1.869296in}{0.997356in}}%
\pgfpathlineto{\pgfqpoint{1.905062in}{1.000941in}}%
\pgfpathlineto{\pgfqpoint{1.868603in}{1.051129in}}%
\pgfpathlineto{\pgfqpoint{1.832452in}{1.039881in}}%
\pgfpathclose%
\pgfusepath{fill}%
\end{pgfscope}%
\begin{pgfscope}%
\pgfpathrectangle{\pgfqpoint{0.150000in}{0.150000in}}{\pgfqpoint{2.700000in}{1.950000in}}%
\pgfusepath{clip}%
\pgfsetbuttcap%
\pgfsetroundjoin%
\definecolor{currentfill}{rgb}{0.978232,0.980913,0.984666}%
\pgfsetfillcolor{currentfill}%
\pgfsetlinewidth{0.000000pt}%
\definecolor{currentstroke}{rgb}{0.000000,0.000000,0.000000}%
\pgfsetstrokecolor{currentstroke}%
\pgfsetdash{}{0pt}%
\pgfpathmoveto{\pgfqpoint{1.090241in}{1.397717in}}%
\pgfpathlineto{\pgfqpoint{1.132418in}{1.303643in}}%
\pgfpathlineto{\pgfqpoint{1.096141in}{1.330062in}}%
\pgfpathlineto{\pgfqpoint{1.053560in}{1.424363in}}%
\pgfpathclose%
\pgfusepath{fill}%
\end{pgfscope}%
\begin{pgfscope}%
\pgfpathrectangle{\pgfqpoint{0.150000in}{0.150000in}}{\pgfqpoint{2.700000in}{1.950000in}}%
\pgfusepath{clip}%
\pgfsetbuttcap%
\pgfsetroundjoin%
\definecolor{currentfill}{rgb}{0.857537,0.741498,0.750689}%
\pgfsetfillcolor{currentfill}%
\pgfsetlinewidth{0.000000pt}%
\definecolor{currentstroke}{rgb}{0.000000,0.000000,0.000000}%
\pgfsetstrokecolor{currentstroke}%
\pgfsetdash{}{0pt}%
\pgfpathmoveto{\pgfqpoint{1.832452in}{1.039881in}}%
\pgfpathlineto{\pgfqpoint{1.868603in}{1.051129in}}%
\pgfpathlineto{\pgfqpoint{1.831836in}{1.093590in}}%
\pgfpathlineto{\pgfqpoint{1.795574in}{1.082446in}}%
\pgfpathclose%
\pgfusepath{fill}%
\end{pgfscope}%
\begin{pgfscope}%
\pgfpathrectangle{\pgfqpoint{0.150000in}{0.150000in}}{\pgfqpoint{2.700000in}{1.950000in}}%
\pgfusepath{clip}%
\pgfsetbuttcap%
\pgfsetroundjoin%
\definecolor{currentfill}{rgb}{0.918321,0.851792,0.857062}%
\pgfsetfillcolor{currentfill}%
\pgfsetlinewidth{0.000000pt}%
\definecolor{currentstroke}{rgb}{0.000000,0.000000,0.000000}%
\pgfsetstrokecolor{currentstroke}%
\pgfsetdash{}{0pt}%
\pgfpathmoveto{\pgfqpoint{1.351116in}{1.175621in}}%
\pgfpathlineto{\pgfqpoint{1.388898in}{1.162828in}}%
\pgfpathlineto{\pgfqpoint{1.352330in}{1.181617in}}%
\pgfpathlineto{\pgfqpoint{1.314274in}{1.202418in}}%
\pgfpathclose%
\pgfusepath{fill}%
\end{pgfscope}%
\begin{pgfscope}%
\pgfpathrectangle{\pgfqpoint{0.150000in}{0.150000in}}{\pgfqpoint{2.700000in}{1.950000in}}%
\pgfusepath{clip}%
\pgfsetbuttcap%
\pgfsetroundjoin%
\definecolor{currentfill}{rgb}{0.899326,0.817325,0.823820}%
\pgfsetfillcolor{currentfill}%
\pgfsetlinewidth{0.000000pt}%
\definecolor{currentstroke}{rgb}{0.000000,0.000000,0.000000}%
\pgfsetstrokecolor{currentstroke}%
\pgfsetdash{}{0pt}%
\pgfpathmoveto{\pgfqpoint{1.758663in}{1.125049in}}%
\pgfpathlineto{\pgfqpoint{1.795228in}{1.143972in}}%
\pgfpathlineto{\pgfqpoint{1.757869in}{1.162828in}}%
\pgfpathlineto{\pgfqpoint{1.721302in}{1.143972in}}%
\pgfpathclose%
\pgfusepath{fill}%
\end{pgfscope}%
\begin{pgfscope}%
\pgfpathrectangle{\pgfqpoint{0.150000in}{0.150000in}}{\pgfqpoint{2.700000in}{1.950000in}}%
\pgfusepath{clip}%
\pgfsetbuttcap%
\pgfsetroundjoin%
\definecolor{currentfill}{rgb}{0.899326,0.817325,0.823820}%
\pgfsetfillcolor{currentfill}%
\pgfsetlinewidth{0.000000pt}%
\definecolor{currentstroke}{rgb}{0.000000,0.000000,0.000000}%
\pgfsetstrokecolor{currentstroke}%
\pgfsetdash{}{0pt}%
\pgfpathmoveto{\pgfqpoint{1.684604in}{1.125049in}}%
\pgfpathlineto{\pgfqpoint{1.721302in}{1.143972in}}%
\pgfpathlineto{\pgfqpoint{1.684075in}{1.162828in}}%
\pgfpathlineto{\pgfqpoint{1.647376in}{1.143972in}}%
\pgfpathclose%
\pgfusepath{fill}%
\end{pgfscope}%
\begin{pgfscope}%
\pgfpathrectangle{\pgfqpoint{0.150000in}{0.150000in}}{\pgfqpoint{2.700000in}{1.950000in}}%
\pgfusepath{clip}%
\pgfsetbuttcap%
\pgfsetroundjoin%
\definecolor{currentfill}{rgb}{0.899326,0.817325,0.823820}%
\pgfsetfillcolor{currentfill}%
\pgfsetlinewidth{0.000000pt}%
\definecolor{currentstroke}{rgb}{0.000000,0.000000,0.000000}%
\pgfsetstrokecolor{currentstroke}%
\pgfsetdash{}{0pt}%
\pgfpathmoveto{\pgfqpoint{1.610545in}{1.125049in}}%
\pgfpathlineto{\pgfqpoint{1.647376in}{1.143972in}}%
\pgfpathlineto{\pgfqpoint{1.610281in}{1.162828in}}%
\pgfpathlineto{\pgfqpoint{1.573450in}{1.143972in}}%
\pgfpathclose%
\pgfusepath{fill}%
\end{pgfscope}%
\begin{pgfscope}%
\pgfpathrectangle{\pgfqpoint{0.150000in}{0.150000in}}{\pgfqpoint{2.700000in}{1.950000in}}%
\pgfusepath{clip}%
\pgfsetbuttcap%
\pgfsetroundjoin%
\definecolor{currentfill}{rgb}{0.899326,0.817325,0.823820}%
\pgfsetfillcolor{currentfill}%
\pgfsetlinewidth{0.000000pt}%
\definecolor{currentstroke}{rgb}{0.000000,0.000000,0.000000}%
\pgfsetstrokecolor{currentstroke}%
\pgfsetdash{}{0pt}%
\pgfpathmoveto{\pgfqpoint{1.536486in}{1.125049in}}%
\pgfpathlineto{\pgfqpoint{1.573450in}{1.143972in}}%
\pgfpathlineto{\pgfqpoint{1.536486in}{1.162828in}}%
\pgfpathlineto{\pgfqpoint{1.499523in}{1.143972in}}%
\pgfpathclose%
\pgfusepath{fill}%
\end{pgfscope}%
\begin{pgfscope}%
\pgfpathrectangle{\pgfqpoint{0.150000in}{0.150000in}}{\pgfqpoint{2.700000in}{1.950000in}}%
\pgfusepath{clip}%
\pgfsetbuttcap%
\pgfsetroundjoin%
\definecolor{currentfill}{rgb}{0.899326,0.817325,0.823820}%
\pgfsetfillcolor{currentfill}%
\pgfsetlinewidth{0.000000pt}%
\definecolor{currentstroke}{rgb}{0.000000,0.000000,0.000000}%
\pgfsetstrokecolor{currentstroke}%
\pgfsetdash{}{0pt}%
\pgfpathmoveto{\pgfqpoint{1.462428in}{1.125049in}}%
\pgfpathlineto{\pgfqpoint{1.499523in}{1.143972in}}%
\pgfpathlineto{\pgfqpoint{1.462692in}{1.162828in}}%
\pgfpathlineto{\pgfqpoint{1.425597in}{1.143972in}}%
\pgfpathclose%
\pgfusepath{fill}%
\end{pgfscope}%
\begin{pgfscope}%
\pgfpathrectangle{\pgfqpoint{0.150000in}{0.150000in}}{\pgfqpoint{2.700000in}{1.950000in}}%
\pgfusepath{clip}%
\pgfsetbuttcap%
\pgfsetroundjoin%
\definecolor{currentfill}{rgb}{0.842341,0.713925,0.724096}%
\pgfsetfillcolor{currentfill}%
\pgfsetlinewidth{0.000000pt}%
\definecolor{currentstroke}{rgb}{0.000000,0.000000,0.000000}%
\pgfsetstrokecolor{currentstroke}%
\pgfsetdash{}{0pt}%
\pgfpathmoveto{\pgfqpoint{2.316547in}{1.044258in}}%
\pgfpathlineto{\pgfqpoint{2.351613in}{1.055539in}}%
\pgfpathlineto{\pgfqpoint{2.309115in}{1.019998in}}%
\pgfpathlineto{\pgfqpoint{2.274188in}{1.008706in}}%
\pgfpathclose%
\pgfusepath{fill}%
\end{pgfscope}%
\begin{pgfscope}%
\pgfpathrectangle{\pgfqpoint{0.150000in}{0.150000in}}{\pgfqpoint{2.700000in}{1.950000in}}%
\pgfusepath{clip}%
\pgfsetbuttcap%
\pgfsetroundjoin%
\definecolor{currentfill}{rgb}{0.884130,0.789752,0.797227}%
\pgfsetfillcolor{currentfill}%
\pgfsetlinewidth{0.000000pt}%
\definecolor{currentstroke}{rgb}{0.000000,0.000000,0.000000}%
\pgfsetstrokecolor{currentstroke}%
\pgfsetdash{}{0pt}%
\pgfpathmoveto{\pgfqpoint{1.795574in}{1.082446in}}%
\pgfpathlineto{\pgfqpoint{1.831836in}{1.093590in}}%
\pgfpathlineto{\pgfqpoint{1.795228in}{1.143972in}}%
\pgfpathlineto{\pgfqpoint{1.758663in}{1.125049in}}%
\pgfpathclose%
\pgfusepath{fill}%
\end{pgfscope}%
\begin{pgfscope}%
\pgfpathrectangle{\pgfqpoint{0.150000in}{0.150000in}}{\pgfqpoint{2.700000in}{1.950000in}}%
\pgfusepath{clip}%
\pgfsetbuttcap%
\pgfsetroundjoin%
\definecolor{currentfill}{rgb}{0.990671,0.991820,0.993428}%
\pgfsetfillcolor{currentfill}%
\pgfsetlinewidth{0.000000pt}%
\definecolor{currentstroke}{rgb}{0.000000,0.000000,0.000000}%
\pgfsetstrokecolor{currentstroke}%
\pgfsetdash{}{0pt}%
\pgfpathmoveto{\pgfqpoint{1.126999in}{1.371014in}}%
\pgfpathlineto{\pgfqpoint{1.168497in}{1.285122in}}%
\pgfpathlineto{\pgfqpoint{1.132418in}{1.303643in}}%
\pgfpathlineto{\pgfqpoint{1.090241in}{1.397717in}}%
\pgfpathclose%
\pgfusepath{fill}%
\end{pgfscope}%
\begin{pgfscope}%
\pgfpathrectangle{\pgfqpoint{0.150000in}{0.150000in}}{\pgfqpoint{2.700000in}{1.950000in}}%
\pgfusepath{clip}%
\pgfsetbuttcap%
\pgfsetroundjoin%
\definecolor{currentfill}{rgb}{0.789154,0.617417,0.631020}%
\pgfsetfillcolor{currentfill}%
\pgfsetlinewidth{0.000000pt}%
\definecolor{currentstroke}{rgb}{0.000000,0.000000,0.000000}%
\pgfsetstrokecolor{currentstroke}%
\pgfsetdash{}{0pt}%
\pgfpathmoveto{\pgfqpoint{1.982294in}{0.931789in}}%
\pgfpathlineto{\pgfqpoint{2.017493in}{0.927867in}}%
\pgfpathlineto{\pgfqpoint{1.978705in}{0.923976in}}%
\pgfpathlineto{\pgfqpoint{1.943492in}{0.927867in}}%
\pgfpathclose%
\pgfusepath{fill}%
\end{pgfscope}%
\begin{pgfscope}%
\pgfpathrectangle{\pgfqpoint{0.150000in}{0.150000in}}{\pgfqpoint{2.700000in}{1.950000in}}%
\pgfusepath{clip}%
\pgfsetbuttcap%
\pgfsetroundjoin%
\definecolor{currentfill}{rgb}{0.811949,0.658778,0.670910}%
\pgfsetfillcolor{currentfill}%
\pgfsetlinewidth{0.000000pt}%
\definecolor{currentstroke}{rgb}{0.000000,0.000000,0.000000}%
\pgfsetstrokecolor{currentstroke}%
\pgfsetdash{}{0pt}%
\pgfpathmoveto{\pgfqpoint{2.168337in}{0.974482in}}%
\pgfpathlineto{\pgfqpoint{2.203800in}{0.985948in}}%
\pgfpathlineto{\pgfqpoint{2.162780in}{0.958592in}}%
\pgfpathlineto{\pgfqpoint{2.127877in}{0.954869in}}%
\pgfpathclose%
\pgfusepath{fill}%
\end{pgfscope}%
\begin{pgfscope}%
\pgfpathrectangle{\pgfqpoint{0.150000in}{0.150000in}}{\pgfqpoint{2.700000in}{1.950000in}}%
\pgfusepath{clip}%
\pgfsetbuttcap%
\pgfsetroundjoin%
\definecolor{currentfill}{rgb}{0.914522,0.844899,0.850414}%
\pgfsetfillcolor{currentfill}%
\pgfsetlinewidth{0.000000pt}%
\definecolor{currentstroke}{rgb}{0.000000,0.000000,0.000000}%
\pgfsetstrokecolor{currentstroke}%
\pgfsetdash{}{0pt}%
\pgfpathmoveto{\pgfqpoint{1.387923in}{1.156697in}}%
\pgfpathlineto{\pgfqpoint{1.425597in}{1.143972in}}%
\pgfpathlineto{\pgfqpoint{1.388898in}{1.162828in}}%
\pgfpathlineto{\pgfqpoint{1.351116in}{1.175621in}}%
\pgfpathclose%
\pgfusepath{fill}%
\end{pgfscope}%
\begin{pgfscope}%
\pgfpathrectangle{\pgfqpoint{0.150000in}{0.150000in}}{\pgfqpoint{2.700000in}{1.950000in}}%
\pgfusepath{clip}%
\pgfsetbuttcap%
\pgfsetroundjoin%
\definecolor{currentfill}{rgb}{0.800551,0.638097,0.650965}%
\pgfsetfillcolor{currentfill}%
\pgfsetlinewidth{0.000000pt}%
\definecolor{currentstroke}{rgb}{0.000000,0.000000,0.000000}%
\pgfsetstrokecolor{currentstroke}%
\pgfsetdash{}{0pt}%
\pgfpathmoveto{\pgfqpoint{2.056987in}{0.939564in}}%
\pgfpathlineto{\pgfqpoint{2.092327in}{0.943359in}}%
\pgfpathlineto{\pgfqpoint{2.052793in}{0.931682in}}%
\pgfpathlineto{\pgfqpoint{2.017493in}{0.927867in}}%
\pgfpathclose%
\pgfusepath{fill}%
\end{pgfscope}%
\begin{pgfscope}%
\pgfpathrectangle{\pgfqpoint{0.150000in}{0.150000in}}{\pgfqpoint{2.700000in}{1.950000in}}%
\pgfusepath{clip}%
\pgfsetbuttcap%
\pgfsetroundjoin%
\definecolor{currentfill}{rgb}{0.899326,0.817325,0.823820}%
\pgfsetfillcolor{currentfill}%
\pgfsetlinewidth{0.000000pt}%
\definecolor{currentstroke}{rgb}{0.000000,0.000000,0.000000}%
\pgfsetstrokecolor{currentstroke}%
\pgfsetdash{}{0pt}%
\pgfpathmoveto{\pgfqpoint{1.721966in}{1.106057in}}%
\pgfpathlineto{\pgfqpoint{1.758663in}{1.125049in}}%
\pgfpathlineto{\pgfqpoint{1.721302in}{1.143972in}}%
\pgfpathlineto{\pgfqpoint{1.684604in}{1.125049in}}%
\pgfpathclose%
\pgfusepath{fill}%
\end{pgfscope}%
\begin{pgfscope}%
\pgfpathrectangle{\pgfqpoint{0.150000in}{0.150000in}}{\pgfqpoint{2.700000in}{1.950000in}}%
\pgfusepath{clip}%
\pgfsetbuttcap%
\pgfsetroundjoin%
\definecolor{currentfill}{rgb}{0.899326,0.817325,0.823820}%
\pgfsetfillcolor{currentfill}%
\pgfsetlinewidth{0.000000pt}%
\definecolor{currentstroke}{rgb}{0.000000,0.000000,0.000000}%
\pgfsetstrokecolor{currentstroke}%
\pgfsetdash{}{0pt}%
\pgfpathmoveto{\pgfqpoint{1.647774in}{1.106057in}}%
\pgfpathlineto{\pgfqpoint{1.684604in}{1.125049in}}%
\pgfpathlineto{\pgfqpoint{1.647376in}{1.143972in}}%
\pgfpathlineto{\pgfqpoint{1.610545in}{1.125049in}}%
\pgfpathclose%
\pgfusepath{fill}%
\end{pgfscope}%
\begin{pgfscope}%
\pgfpathrectangle{\pgfqpoint{0.150000in}{0.150000in}}{\pgfqpoint{2.700000in}{1.950000in}}%
\pgfusepath{clip}%
\pgfsetbuttcap%
\pgfsetroundjoin%
\definecolor{currentfill}{rgb}{0.899326,0.817325,0.823820}%
\pgfsetfillcolor{currentfill}%
\pgfsetlinewidth{0.000000pt}%
\definecolor{currentstroke}{rgb}{0.000000,0.000000,0.000000}%
\pgfsetstrokecolor{currentstroke}%
\pgfsetdash{}{0pt}%
\pgfpathmoveto{\pgfqpoint{1.573582in}{1.106057in}}%
\pgfpathlineto{\pgfqpoint{1.610545in}{1.125049in}}%
\pgfpathlineto{\pgfqpoint{1.573450in}{1.143972in}}%
\pgfpathlineto{\pgfqpoint{1.536486in}{1.125049in}}%
\pgfpathclose%
\pgfusepath{fill}%
\end{pgfscope}%
\begin{pgfscope}%
\pgfpathrectangle{\pgfqpoint{0.150000in}{0.150000in}}{\pgfqpoint{2.700000in}{1.950000in}}%
\pgfusepath{clip}%
\pgfsetbuttcap%
\pgfsetroundjoin%
\definecolor{currentfill}{rgb}{0.899326,0.817325,0.823820}%
\pgfsetfillcolor{currentfill}%
\pgfsetlinewidth{0.000000pt}%
\definecolor{currentstroke}{rgb}{0.000000,0.000000,0.000000}%
\pgfsetstrokecolor{currentstroke}%
\pgfsetdash{}{0pt}%
\pgfpathmoveto{\pgfqpoint{1.499390in}{1.106057in}}%
\pgfpathlineto{\pgfqpoint{1.536486in}{1.125049in}}%
\pgfpathlineto{\pgfqpoint{1.499523in}{1.143972in}}%
\pgfpathlineto{\pgfqpoint{1.462428in}{1.125049in}}%
\pgfpathclose%
\pgfusepath{fill}%
\end{pgfscope}%
\begin{pgfscope}%
\pgfpathrectangle{\pgfqpoint{0.150000in}{0.150000in}}{\pgfqpoint{2.700000in}{1.950000in}}%
\pgfusepath{clip}%
\pgfsetbuttcap%
\pgfsetroundjoin%
\definecolor{currentfill}{rgb}{0.899326,0.817325,0.823820}%
\pgfsetfillcolor{currentfill}%
\pgfsetlinewidth{0.000000pt}%
\definecolor{currentstroke}{rgb}{0.000000,0.000000,0.000000}%
\pgfsetstrokecolor{currentstroke}%
\pgfsetdash{}{0pt}%
\pgfpathmoveto{\pgfqpoint{0.757471in}{1.106057in}}%
\pgfpathlineto{\pgfqpoint{0.795898in}{1.125049in}}%
\pgfpathlineto{\pgfqpoint{0.760261in}{1.143972in}}%
\pgfpathlineto{\pgfqpoint{0.721839in}{1.125049in}}%
\pgfpathclose%
\pgfusepath{fill}%
\end{pgfscope}%
\begin{pgfscope}%
\pgfpathrectangle{\pgfqpoint{0.150000in}{0.150000in}}{\pgfqpoint{2.700000in}{1.950000in}}%
\pgfusepath{clip}%
\pgfsetbuttcap%
\pgfsetroundjoin%
\definecolor{currentfill}{rgb}{0.823346,0.679458,0.690855}%
\pgfsetfillcolor{currentfill}%
\pgfsetlinewidth{0.000000pt}%
\definecolor{currentstroke}{rgb}{0.000000,0.000000,0.000000}%
\pgfsetstrokecolor{currentstroke}%
\pgfsetdash{}{0pt}%
\pgfpathmoveto{\pgfqpoint{1.870492in}{0.958897in}}%
\pgfpathlineto{\pgfqpoint{1.906106in}{0.954869in}}%
\pgfpathlineto{\pgfqpoint{1.869296in}{0.997356in}}%
\pgfpathlineto{\pgfqpoint{1.833293in}{0.993747in}}%
\pgfpathclose%
\pgfusepath{fill}%
\end{pgfscope}%
\begin{pgfscope}%
\pgfpathrectangle{\pgfqpoint{0.150000in}{0.150000in}}{\pgfqpoint{2.700000in}{1.950000in}}%
\pgfusepath{clip}%
\pgfsetbuttcap%
\pgfsetroundjoin%
\definecolor{currentfill}{rgb}{0.996890,0.997273,0.997809}%
\pgfsetfillcolor{currentfill}%
\pgfsetlinewidth{0.000000pt}%
\definecolor{currentstroke}{rgb}{0.000000,0.000000,0.000000}%
\pgfsetstrokecolor{currentstroke}%
\pgfsetdash{}{0pt}%
\pgfpathmoveto{\pgfqpoint{1.163834in}{1.344256in}}%
\pgfpathlineto{\pgfqpoint{1.204704in}{1.266535in}}%
\pgfpathlineto{\pgfqpoint{1.168497in}{1.285122in}}%
\pgfpathlineto{\pgfqpoint{1.126999in}{1.371014in}}%
\pgfpathclose%
\pgfusepath{fill}%
\end{pgfscope}%
\begin{pgfscope}%
\pgfpathrectangle{\pgfqpoint{0.150000in}{0.150000in}}{\pgfqpoint{2.700000in}{1.950000in}}%
\pgfusepath{clip}%
\pgfsetbuttcap%
\pgfsetroundjoin%
\definecolor{currentfill}{rgb}{0.842341,0.713925,0.724096}%
\pgfsetfillcolor{currentfill}%
\pgfsetlinewidth{0.000000pt}%
\definecolor{currentstroke}{rgb}{0.000000,0.000000,0.000000}%
\pgfsetstrokecolor{currentstroke}%
\pgfsetdash{}{0pt}%
\pgfpathmoveto{\pgfqpoint{1.833293in}{0.993747in}}%
\pgfpathlineto{\pgfqpoint{1.869296in}{0.997356in}}%
\pgfpathlineto{\pgfqpoint{1.832452in}{1.039881in}}%
\pgfpathlineto{\pgfqpoint{1.796116in}{1.028577in}}%
\pgfpathclose%
\pgfusepath{fill}%
\end{pgfscope}%
\begin{pgfscope}%
\pgfpathrectangle{\pgfqpoint{0.150000in}{0.150000in}}{\pgfqpoint{2.700000in}{1.950000in}}%
\pgfusepath{clip}%
\pgfsetbuttcap%
\pgfsetroundjoin%
\definecolor{currentfill}{rgb}{0.865135,0.755285,0.763986}%
\pgfsetfillcolor{currentfill}%
\pgfsetlinewidth{0.000000pt}%
\definecolor{currentstroke}{rgb}{0.000000,0.000000,0.000000}%
\pgfsetstrokecolor{currentstroke}%
\pgfsetdash{}{0pt}%
\pgfpathmoveto{\pgfqpoint{1.796116in}{1.028577in}}%
\pgfpathlineto{\pgfqpoint{1.832452in}{1.039881in}}%
\pgfpathlineto{\pgfqpoint{1.795574in}{1.082446in}}%
\pgfpathlineto{\pgfqpoint{1.758961in}{1.063386in}}%
\pgfpathclose%
\pgfusepath{fill}%
\end{pgfscope}%
\begin{pgfscope}%
\pgfpathrectangle{\pgfqpoint{0.150000in}{0.150000in}}{\pgfqpoint{2.700000in}{1.950000in}}%
\pgfusepath{clip}%
\pgfsetbuttcap%
\pgfsetroundjoin%
\definecolor{currentfill}{rgb}{0.887929,0.796645,0.803876}%
\pgfsetfillcolor{currentfill}%
\pgfsetlinewidth{0.000000pt}%
\definecolor{currentstroke}{rgb}{0.000000,0.000000,0.000000}%
\pgfsetstrokecolor{currentstroke}%
\pgfsetdash{}{0pt}%
\pgfpathmoveto{\pgfqpoint{1.758961in}{1.063386in}}%
\pgfpathlineto{\pgfqpoint{1.795574in}{1.082446in}}%
\pgfpathlineto{\pgfqpoint{1.758663in}{1.125049in}}%
\pgfpathlineto{\pgfqpoint{1.721966in}{1.106057in}}%
\pgfpathclose%
\pgfusepath{fill}%
\end{pgfscope}%
\begin{pgfscope}%
\pgfpathrectangle{\pgfqpoint{0.150000in}{0.150000in}}{\pgfqpoint{2.700000in}{1.950000in}}%
\pgfusepath{clip}%
\pgfsetbuttcap%
\pgfsetroundjoin%
\definecolor{currentfill}{rgb}{0.846140,0.720818,0.730744}%
\pgfsetfillcolor{currentfill}%
\pgfsetlinewidth{0.000000pt}%
\definecolor{currentstroke}{rgb}{0.000000,0.000000,0.000000}%
\pgfsetstrokecolor{currentstroke}%
\pgfsetdash{}{0pt}%
\pgfpathmoveto{\pgfqpoint{2.280741in}{1.025061in}}%
\pgfpathlineto{\pgfqpoint{2.316547in}{1.044258in}}%
\pgfpathlineto{\pgfqpoint{2.274188in}{1.008706in}}%
\pgfpathlineto{\pgfqpoint{2.239083in}{0.997356in}}%
\pgfpathclose%
\pgfusepath{fill}%
\end{pgfscope}%
\begin{pgfscope}%
\pgfpathrectangle{\pgfqpoint{0.150000in}{0.150000in}}{\pgfqpoint{2.700000in}{1.950000in}}%
\pgfusepath{clip}%
\pgfsetbuttcap%
\pgfsetroundjoin%
\definecolor{currentfill}{rgb}{0.808150,0.651884,0.664262}%
\pgfsetfillcolor{currentfill}%
\pgfsetlinewidth{0.000000pt}%
\definecolor{currentstroke}{rgb}{0.000000,0.000000,0.000000}%
\pgfsetstrokecolor{currentstroke}%
\pgfsetdash{}{0pt}%
\pgfpathmoveto{\pgfqpoint{1.907993in}{0.931789in}}%
\pgfpathlineto{\pgfqpoint{1.943492in}{0.927867in}}%
\pgfpathlineto{\pgfqpoint{1.906106in}{0.954869in}}%
\pgfpathlineto{\pgfqpoint{1.870492in}{0.958897in}}%
\pgfpathclose%
\pgfusepath{fill}%
\end{pgfscope}%
\begin{pgfscope}%
\pgfpathrectangle{\pgfqpoint{0.150000in}{0.150000in}}{\pgfqpoint{2.700000in}{1.950000in}}%
\pgfusepath{clip}%
\pgfsetbuttcap%
\pgfsetroundjoin%
\definecolor{currentfill}{rgb}{0.994301,0.989660,0.990028}%
\pgfsetfillcolor{currentfill}%
\pgfsetlinewidth{0.000000pt}%
\definecolor{currentstroke}{rgb}{0.000000,0.000000,0.000000}%
\pgfsetstrokecolor{currentstroke}%
\pgfsetdash{}{0pt}%
\pgfpathmoveto{\pgfqpoint{1.200746in}{1.317441in}}%
\pgfpathlineto{\pgfqpoint{1.241262in}{1.239927in}}%
\pgfpathlineto{\pgfqpoint{1.204704in}{1.266535in}}%
\pgfpathlineto{\pgfqpoint{1.163834in}{1.344256in}}%
\pgfpathclose%
\pgfusepath{fill}%
\end{pgfscope}%
\begin{pgfscope}%
\pgfpathrectangle{\pgfqpoint{0.150000in}{0.150000in}}{\pgfqpoint{2.700000in}{1.950000in}}%
\pgfusepath{clip}%
\pgfsetbuttcap%
\pgfsetroundjoin%
\definecolor{currentfill}{rgb}{0.914522,0.844899,0.850414}%
\pgfsetfillcolor{currentfill}%
\pgfsetlinewidth{0.000000pt}%
\definecolor{currentstroke}{rgb}{0.000000,0.000000,0.000000}%
\pgfsetstrokecolor{currentstroke}%
\pgfsetdash{}{0pt}%
\pgfpathmoveto{\pgfqpoint{1.424863in}{1.137705in}}%
\pgfpathlineto{\pgfqpoint{1.462428in}{1.125049in}}%
\pgfpathlineto{\pgfqpoint{1.425597in}{1.143972in}}%
\pgfpathlineto{\pgfqpoint{1.387923in}{1.156697in}}%
\pgfpathclose%
\pgfusepath{fill}%
\end{pgfscope}%
\begin{pgfscope}%
\pgfpathrectangle{\pgfqpoint{0.150000in}{0.150000in}}{\pgfqpoint{2.700000in}{1.950000in}}%
\pgfusepath{clip}%
\pgfsetbuttcap%
\pgfsetroundjoin%
\definecolor{currentfill}{rgb}{0.899326,0.817325,0.823820}%
\pgfsetfillcolor{currentfill}%
\pgfsetlinewidth{0.000000pt}%
\definecolor{currentstroke}{rgb}{0.000000,0.000000,0.000000}%
\pgfsetstrokecolor{currentstroke}%
\pgfsetdash{}{0pt}%
\pgfpathmoveto{\pgfqpoint{1.685138in}{1.086997in}}%
\pgfpathlineto{\pgfqpoint{1.721966in}{1.106057in}}%
\pgfpathlineto{\pgfqpoint{1.684604in}{1.125049in}}%
\pgfpathlineto{\pgfqpoint{1.647774in}{1.106057in}}%
\pgfpathclose%
\pgfusepath{fill}%
\end{pgfscope}%
\begin{pgfscope}%
\pgfpathrectangle{\pgfqpoint{0.150000in}{0.150000in}}{\pgfqpoint{2.700000in}{1.950000in}}%
\pgfusepath{clip}%
\pgfsetbuttcap%
\pgfsetroundjoin%
\definecolor{currentfill}{rgb}{0.899326,0.817325,0.823820}%
\pgfsetfillcolor{currentfill}%
\pgfsetlinewidth{0.000000pt}%
\definecolor{currentstroke}{rgb}{0.000000,0.000000,0.000000}%
\pgfsetstrokecolor{currentstroke}%
\pgfsetdash{}{0pt}%
\pgfpathmoveto{\pgfqpoint{1.610812in}{1.086997in}}%
\pgfpathlineto{\pgfqpoint{1.647774in}{1.106057in}}%
\pgfpathlineto{\pgfqpoint{1.610545in}{1.125049in}}%
\pgfpathlineto{\pgfqpoint{1.573582in}{1.106057in}}%
\pgfpathclose%
\pgfusepath{fill}%
\end{pgfscope}%
\begin{pgfscope}%
\pgfpathrectangle{\pgfqpoint{0.150000in}{0.150000in}}{\pgfqpoint{2.700000in}{1.950000in}}%
\pgfusepath{clip}%
\pgfsetbuttcap%
\pgfsetroundjoin%
\definecolor{currentfill}{rgb}{0.899326,0.817325,0.823820}%
\pgfsetfillcolor{currentfill}%
\pgfsetlinewidth{0.000000pt}%
\definecolor{currentstroke}{rgb}{0.000000,0.000000,0.000000}%
\pgfsetstrokecolor{currentstroke}%
\pgfsetdash{}{0pt}%
\pgfpathmoveto{\pgfqpoint{1.536486in}{1.086997in}}%
\pgfpathlineto{\pgfqpoint{1.573582in}{1.106057in}}%
\pgfpathlineto{\pgfqpoint{1.536486in}{1.125049in}}%
\pgfpathlineto{\pgfqpoint{1.499390in}{1.106057in}}%
\pgfpathclose%
\pgfusepath{fill}%
\end{pgfscope}%
\begin{pgfscope}%
\pgfpathrectangle{\pgfqpoint{0.150000in}{0.150000in}}{\pgfqpoint{2.700000in}{1.950000in}}%
\pgfusepath{clip}%
\pgfsetbuttcap%
\pgfsetroundjoin%
\definecolor{currentfill}{rgb}{0.899326,0.817325,0.823820}%
\pgfsetfillcolor{currentfill}%
\pgfsetlinewidth{0.000000pt}%
\definecolor{currentstroke}{rgb}{0.000000,0.000000,0.000000}%
\pgfsetstrokecolor{currentstroke}%
\pgfsetdash{}{0pt}%
\pgfpathmoveto{\pgfqpoint{0.793231in}{1.086997in}}%
\pgfpathlineto{\pgfqpoint{0.831663in}{1.106057in}}%
\pgfpathlineto{\pgfqpoint{0.795898in}{1.125049in}}%
\pgfpathlineto{\pgfqpoint{0.757471in}{1.106057in}}%
\pgfpathclose%
\pgfusepath{fill}%
\end{pgfscope}%
\begin{pgfscope}%
\pgfpathrectangle{\pgfqpoint{0.150000in}{0.150000in}}{\pgfqpoint{2.700000in}{1.950000in}}%
\pgfusepath{clip}%
\pgfsetbuttcap%
\pgfsetroundjoin%
\definecolor{currentfill}{rgb}{0.899326,0.817325,0.823820}%
\pgfsetfillcolor{currentfill}%
\pgfsetlinewidth{0.000000pt}%
\definecolor{currentstroke}{rgb}{0.000000,0.000000,0.000000}%
\pgfsetstrokecolor{currentstroke}%
\pgfsetdash{}{0pt}%
\pgfpathmoveto{\pgfqpoint{0.718905in}{1.086997in}}%
\pgfpathlineto{\pgfqpoint{0.757471in}{1.106057in}}%
\pgfpathlineto{\pgfqpoint{0.721839in}{1.125049in}}%
\pgfpathlineto{\pgfqpoint{0.683279in}{1.106057in}}%
\pgfpathclose%
\pgfusepath{fill}%
\end{pgfscope}%
\begin{pgfscope}%
\pgfpathrectangle{\pgfqpoint{0.150000in}{0.150000in}}{\pgfqpoint{2.700000in}{1.950000in}}%
\pgfusepath{clip}%
\pgfsetbuttcap%
\pgfsetroundjoin%
\definecolor{currentfill}{rgb}{0.884130,0.789752,0.797227}%
\pgfsetfillcolor{currentfill}%
\pgfsetlinewidth{0.000000pt}%
\definecolor{currentstroke}{rgb}{0.000000,0.000000,0.000000}%
\pgfsetstrokecolor{currentstroke}%
\pgfsetdash{}{0pt}%
\pgfpathmoveto{\pgfqpoint{2.392772in}{1.067868in}}%
\pgfpathlineto{\pgfqpoint{2.428393in}{1.086997in}}%
\pgfpathlineto{\pgfqpoint{2.387137in}{1.074599in}}%
\pgfpathlineto{\pgfqpoint{2.351613in}{1.055539in}}%
\pgfpathclose%
\pgfusepath{fill}%
\end{pgfscope}%
\begin{pgfscope}%
\pgfpathrectangle{\pgfqpoint{0.150000in}{0.150000in}}{\pgfqpoint{2.700000in}{1.950000in}}%
\pgfusepath{clip}%
\pgfsetbuttcap%
\pgfsetroundjoin%
\definecolor{currentfill}{rgb}{0.823346,0.679458,0.690855}%
\pgfsetfillcolor{currentfill}%
\pgfsetlinewidth{0.000000pt}%
\definecolor{currentstroke}{rgb}{0.000000,0.000000,0.000000}%
\pgfsetstrokecolor{currentstroke}%
\pgfsetdash{}{0pt}%
\pgfpathmoveto{\pgfqpoint{2.132692in}{0.962958in}}%
\pgfpathlineto{\pgfqpoint{2.168337in}{0.974482in}}%
\pgfpathlineto{\pgfqpoint{2.127877in}{0.954869in}}%
\pgfpathlineto{\pgfqpoint{2.092327in}{0.943359in}}%
\pgfpathclose%
\pgfusepath{fill}%
\end{pgfscope}%
\begin{pgfscope}%
\pgfpathrectangle{\pgfqpoint{0.150000in}{0.150000in}}{\pgfqpoint{2.700000in}{1.950000in}}%
\pgfusepath{clip}%
\pgfsetbuttcap%
\pgfsetroundjoin%
\definecolor{currentfill}{rgb}{0.948713,0.906939,0.910248}%
\pgfsetfillcolor{currentfill}%
\pgfsetlinewidth{0.000000pt}%
\definecolor{currentstroke}{rgb}{0.000000,0.000000,0.000000}%
\pgfsetstrokecolor{currentstroke}%
\pgfsetdash{}{0pt}%
\pgfpathmoveto{\pgfqpoint{0.795898in}{1.125049in}}%
\pgfpathlineto{\pgfqpoint{0.828354in}{1.231465in}}%
\pgfpathlineto{\pgfqpoint{0.791866in}{1.258345in}}%
\pgfpathlineto{\pgfqpoint{0.760261in}{1.143972in}}%
\pgfpathclose%
\pgfusepath{fill}%
\end{pgfscope}%
\begin{pgfscope}%
\pgfpathrectangle{\pgfqpoint{0.150000in}{0.150000in}}{\pgfqpoint{2.700000in}{1.950000in}}%
\pgfusepath{clip}%
\pgfsetbuttcap%
\pgfsetroundjoin%
\definecolor{currentfill}{rgb}{0.887929,0.796645,0.803876}%
\pgfsetfillcolor{currentfill}%
\pgfsetlinewidth{0.000000pt}%
\definecolor{currentstroke}{rgb}{0.000000,0.000000,0.000000}%
\pgfsetstrokecolor{currentstroke}%
\pgfsetdash{}{0pt}%
\pgfpathmoveto{\pgfqpoint{1.722355in}{1.052116in}}%
\pgfpathlineto{\pgfqpoint{1.758961in}{1.063386in}}%
\pgfpathlineto{\pgfqpoint{1.721966in}{1.106057in}}%
\pgfpathlineto{\pgfqpoint{1.685138in}{1.086997in}}%
\pgfpathclose%
\pgfusepath{fill}%
\end{pgfscope}%
\begin{pgfscope}%
\pgfpathrectangle{\pgfqpoint{0.150000in}{0.150000in}}{\pgfqpoint{2.700000in}{1.950000in}}%
\pgfusepath{clip}%
\pgfsetbuttcap%
\pgfsetroundjoin%
\definecolor{currentfill}{rgb}{0.990502,0.982767,0.983379}%
\pgfsetfillcolor{currentfill}%
\pgfsetlinewidth{0.000000pt}%
\definecolor{currentstroke}{rgb}{0.000000,0.000000,0.000000}%
\pgfsetstrokecolor{currentstroke}%
\pgfsetdash{}{0pt}%
\pgfpathmoveto{\pgfqpoint{1.237737in}{1.290570in}}%
\pgfpathlineto{\pgfqpoint{1.277702in}{1.221206in}}%
\pgfpathlineto{\pgfqpoint{1.241262in}{1.239927in}}%
\pgfpathlineto{\pgfqpoint{1.200746in}{1.317441in}}%
\pgfpathclose%
\pgfusepath{fill}%
\end{pgfscope}%
\begin{pgfscope}%
\pgfpathrectangle{\pgfqpoint{0.150000in}{0.150000in}}{\pgfqpoint{2.700000in}{1.950000in}}%
\pgfusepath{clip}%
\pgfsetbuttcap%
\pgfsetroundjoin%
\definecolor{currentfill}{rgb}{0.868934,0.762178,0.770634}%
\pgfsetfillcolor{currentfill}%
\pgfsetlinewidth{0.000000pt}%
\definecolor{currentstroke}{rgb}{0.000000,0.000000,0.000000}%
\pgfsetstrokecolor{currentstroke}%
\pgfsetdash{}{0pt}%
\pgfpathmoveto{\pgfqpoint{1.759595in}{1.017215in}}%
\pgfpathlineto{\pgfqpoint{1.796116in}{1.028577in}}%
\pgfpathlineto{\pgfqpoint{1.758961in}{1.063386in}}%
\pgfpathlineto{\pgfqpoint{1.722355in}{1.052116in}}%
\pgfpathclose%
\pgfusepath{fill}%
\end{pgfscope}%
\begin{pgfscope}%
\pgfpathrectangle{\pgfqpoint{0.150000in}{0.150000in}}{\pgfqpoint{2.700000in}{1.950000in}}%
\pgfusepath{clip}%
\pgfsetbuttcap%
\pgfsetroundjoin%
\definecolor{currentfill}{rgb}{0.849939,0.727711,0.737393}%
\pgfsetfillcolor{currentfill}%
\pgfsetlinewidth{0.000000pt}%
\definecolor{currentstroke}{rgb}{0.000000,0.000000,0.000000}%
\pgfsetstrokecolor{currentstroke}%
\pgfsetdash{}{0pt}%
\pgfpathmoveto{\pgfqpoint{2.244806in}{1.005794in}}%
\pgfpathlineto{\pgfqpoint{2.280741in}{1.025061in}}%
\pgfpathlineto{\pgfqpoint{2.239083in}{0.997356in}}%
\pgfpathlineto{\pgfqpoint{2.203800in}{0.985948in}}%
\pgfpathclose%
\pgfusepath{fill}%
\end{pgfscope}%
\begin{pgfscope}%
\pgfpathrectangle{\pgfqpoint{0.150000in}{0.150000in}}{\pgfqpoint{2.700000in}{1.950000in}}%
\pgfusepath{clip}%
\pgfsetbuttcap%
\pgfsetroundjoin%
\definecolor{currentfill}{rgb}{0.811949,0.658778,0.670910}%
\pgfsetfillcolor{currentfill}%
\pgfsetlinewidth{0.000000pt}%
\definecolor{currentstroke}{rgb}{0.000000,0.000000,0.000000}%
\pgfsetstrokecolor{currentstroke}%
\pgfsetdash{}{0pt}%
\pgfpathmoveto{\pgfqpoint{1.946499in}{0.927946in}}%
\pgfpathlineto{\pgfqpoint{1.982294in}{0.931789in}}%
\pgfpathlineto{\pgfqpoint{1.943492in}{0.927867in}}%
\pgfpathlineto{\pgfqpoint{1.907993in}{0.931789in}}%
\pgfpathclose%
\pgfusepath{fill}%
\end{pgfscope}%
\begin{pgfscope}%
\pgfpathrectangle{\pgfqpoint{0.150000in}{0.150000in}}{\pgfqpoint{2.700000in}{1.950000in}}%
\pgfusepath{clip}%
\pgfsetbuttcap%
\pgfsetroundjoin%
\definecolor{currentfill}{rgb}{0.914522,0.844899,0.850414}%
\pgfsetfillcolor{currentfill}%
\pgfsetlinewidth{0.000000pt}%
\definecolor{currentstroke}{rgb}{0.000000,0.000000,0.000000}%
\pgfsetstrokecolor{currentstroke}%
\pgfsetdash{}{0pt}%
\pgfpathmoveto{\pgfqpoint{1.461993in}{1.110714in}}%
\pgfpathlineto{\pgfqpoint{1.499390in}{1.106057in}}%
\pgfpathlineto{\pgfqpoint{1.462428in}{1.125049in}}%
\pgfpathlineto{\pgfqpoint{1.424863in}{1.137705in}}%
\pgfpathclose%
\pgfusepath{fill}%
\end{pgfscope}%
\begin{pgfscope}%
\pgfpathrectangle{\pgfqpoint{0.150000in}{0.150000in}}{\pgfqpoint{2.700000in}{1.950000in}}%
\pgfusepath{clip}%
\pgfsetbuttcap%
\pgfsetroundjoin%
\definecolor{currentfill}{rgb}{0.853738,0.734605,0.744041}%
\pgfsetfillcolor{currentfill}%
\pgfsetlinewidth{0.000000pt}%
\definecolor{currentstroke}{rgb}{0.000000,0.000000,0.000000}%
\pgfsetstrokecolor{currentstroke}%
\pgfsetdash{}{0pt}%
\pgfpathmoveto{\pgfqpoint{1.797052in}{0.990115in}}%
\pgfpathlineto{\pgfqpoint{1.833293in}{0.993747in}}%
\pgfpathlineto{\pgfqpoint{1.796116in}{1.028577in}}%
\pgfpathlineto{\pgfqpoint{1.759595in}{1.017215in}}%
\pgfpathclose%
\pgfusepath{fill}%
\end{pgfscope}%
\begin{pgfscope}%
\pgfpathrectangle{\pgfqpoint{0.150000in}{0.150000in}}{\pgfqpoint{2.700000in}{1.950000in}}%
\pgfusepath{clip}%
\pgfsetbuttcap%
\pgfsetroundjoin%
\definecolor{currentfill}{rgb}{0.815748,0.665671,0.677558}%
\pgfsetfillcolor{currentfill}%
\pgfsetlinewidth{0.000000pt}%
\definecolor{currentstroke}{rgb}{0.000000,0.000000,0.000000}%
\pgfsetstrokecolor{currentstroke}%
\pgfsetdash{}{0pt}%
\pgfpathmoveto{\pgfqpoint{2.021046in}{0.927946in}}%
\pgfpathlineto{\pgfqpoint{2.056987in}{0.939564in}}%
\pgfpathlineto{\pgfqpoint{2.017493in}{0.927867in}}%
\pgfpathlineto{\pgfqpoint{1.982294in}{0.931789in}}%
\pgfpathclose%
\pgfusepath{fill}%
\end{pgfscope}%
\begin{pgfscope}%
\pgfpathrectangle{\pgfqpoint{0.150000in}{0.150000in}}{\pgfqpoint{2.700000in}{1.950000in}}%
\pgfusepath{clip}%
\pgfsetbuttcap%
\pgfsetroundjoin%
\definecolor{currentfill}{rgb}{0.899326,0.817325,0.823820}%
\pgfsetfillcolor{currentfill}%
\pgfsetlinewidth{0.000000pt}%
\definecolor{currentstroke}{rgb}{0.000000,0.000000,0.000000}%
\pgfsetstrokecolor{currentstroke}%
\pgfsetdash{}{0pt}%
\pgfpathmoveto{\pgfqpoint{1.648176in}{1.067868in}}%
\pgfpathlineto{\pgfqpoint{1.685138in}{1.086997in}}%
\pgfpathlineto{\pgfqpoint{1.647774in}{1.106057in}}%
\pgfpathlineto{\pgfqpoint{1.610812in}{1.086997in}}%
\pgfpathclose%
\pgfusepath{fill}%
\end{pgfscope}%
\begin{pgfscope}%
\pgfpathrectangle{\pgfqpoint{0.150000in}{0.150000in}}{\pgfqpoint{2.700000in}{1.950000in}}%
\pgfusepath{clip}%
\pgfsetbuttcap%
\pgfsetroundjoin%
\definecolor{currentfill}{rgb}{0.899326,0.817325,0.823820}%
\pgfsetfillcolor{currentfill}%
\pgfsetlinewidth{0.000000pt}%
\definecolor{currentstroke}{rgb}{0.000000,0.000000,0.000000}%
\pgfsetstrokecolor{currentstroke}%
\pgfsetdash{}{0pt}%
\pgfpathmoveto{\pgfqpoint{1.573716in}{1.067868in}}%
\pgfpathlineto{\pgfqpoint{1.610812in}{1.086997in}}%
\pgfpathlineto{\pgfqpoint{1.573582in}{1.106057in}}%
\pgfpathlineto{\pgfqpoint{1.536486in}{1.086997in}}%
\pgfpathclose%
\pgfusepath{fill}%
\end{pgfscope}%
\begin{pgfscope}%
\pgfpathrectangle{\pgfqpoint{0.150000in}{0.150000in}}{\pgfqpoint{2.700000in}{1.950000in}}%
\pgfusepath{clip}%
\pgfsetbuttcap%
\pgfsetroundjoin%
\definecolor{currentfill}{rgb}{0.899326,0.817325,0.823820}%
\pgfsetfillcolor{currentfill}%
\pgfsetlinewidth{0.000000pt}%
\definecolor{currentstroke}{rgb}{0.000000,0.000000,0.000000}%
\pgfsetstrokecolor{currentstroke}%
\pgfsetdash{}{0pt}%
\pgfpathmoveto{\pgfqpoint{0.829120in}{1.067868in}}%
\pgfpathlineto{\pgfqpoint{0.867556in}{1.086997in}}%
\pgfpathlineto{\pgfqpoint{0.831663in}{1.106057in}}%
\pgfpathlineto{\pgfqpoint{0.793231in}{1.086997in}}%
\pgfpathclose%
\pgfusepath{fill}%
\end{pgfscope}%
\begin{pgfscope}%
\pgfpathrectangle{\pgfqpoint{0.150000in}{0.150000in}}{\pgfqpoint{2.700000in}{1.950000in}}%
\pgfusepath{clip}%
\pgfsetbuttcap%
\pgfsetroundjoin%
\definecolor{currentfill}{rgb}{0.899326,0.817325,0.823820}%
\pgfsetfillcolor{currentfill}%
\pgfsetlinewidth{0.000000pt}%
\definecolor{currentstroke}{rgb}{0.000000,0.000000,0.000000}%
\pgfsetstrokecolor{currentstroke}%
\pgfsetdash{}{0pt}%
\pgfpathmoveto{\pgfqpoint{0.754660in}{1.067868in}}%
\pgfpathlineto{\pgfqpoint{0.793231in}{1.086997in}}%
\pgfpathlineto{\pgfqpoint{0.757471in}{1.106057in}}%
\pgfpathlineto{\pgfqpoint{0.718905in}{1.086997in}}%
\pgfpathclose%
\pgfusepath{fill}%
\end{pgfscope}%
\begin{pgfscope}%
\pgfpathrectangle{\pgfqpoint{0.150000in}{0.150000in}}{\pgfqpoint{2.700000in}{1.950000in}}%
\pgfusepath{clip}%
\pgfsetbuttcap%
\pgfsetroundjoin%
\definecolor{currentfill}{rgb}{0.899326,0.817325,0.823820}%
\pgfsetfillcolor{currentfill}%
\pgfsetlinewidth{0.000000pt}%
\definecolor{currentstroke}{rgb}{0.000000,0.000000,0.000000}%
\pgfsetstrokecolor{currentstroke}%
\pgfsetdash{}{0pt}%
\pgfpathmoveto{\pgfqpoint{0.680201in}{1.067868in}}%
\pgfpathlineto{\pgfqpoint{0.718905in}{1.086997in}}%
\pgfpathlineto{\pgfqpoint{0.683279in}{1.106057in}}%
\pgfpathlineto{\pgfqpoint{0.644580in}{1.086997in}}%
\pgfpathclose%
\pgfusepath{fill}%
\end{pgfscope}%
\begin{pgfscope}%
\pgfpathrectangle{\pgfqpoint{0.150000in}{0.150000in}}{\pgfqpoint{2.700000in}{1.950000in}}%
\pgfusepath{clip}%
\pgfsetbuttcap%
\pgfsetroundjoin%
\definecolor{currentfill}{rgb}{0.842341,0.713925,0.724096}%
\pgfsetfillcolor{currentfill}%
\pgfsetlinewidth{0.000000pt}%
\definecolor{currentstroke}{rgb}{0.000000,0.000000,0.000000}%
\pgfsetstrokecolor{currentstroke}%
\pgfsetdash{}{0pt}%
\pgfpathmoveto{\pgfqpoint{1.834364in}{0.955148in}}%
\pgfpathlineto{\pgfqpoint{1.870492in}{0.958897in}}%
\pgfpathlineto{\pgfqpoint{1.833293in}{0.993747in}}%
\pgfpathlineto{\pgfqpoint{1.797052in}{0.990115in}}%
\pgfpathclose%
\pgfusepath{fill}%
\end{pgfscope}%
\begin{pgfscope}%
\pgfpathrectangle{\pgfqpoint{0.150000in}{0.150000in}}{\pgfqpoint{2.700000in}{1.950000in}}%
\pgfusepath{clip}%
\pgfsetbuttcap%
\pgfsetroundjoin%
\definecolor{currentfill}{rgb}{0.884130,0.789752,0.797227}%
\pgfsetfillcolor{currentfill}%
\pgfsetlinewidth{0.000000pt}%
\definecolor{currentstroke}{rgb}{0.000000,0.000000,0.000000}%
\pgfsetstrokecolor{currentstroke}%
\pgfsetdash{}{0pt}%
\pgfpathmoveto{\pgfqpoint{2.357023in}{1.048670in}}%
\pgfpathlineto{\pgfqpoint{2.392772in}{1.067868in}}%
\pgfpathlineto{\pgfqpoint{2.351613in}{1.055539in}}%
\pgfpathlineto{\pgfqpoint{2.316547in}{1.044258in}}%
\pgfpathclose%
\pgfusepath{fill}%
\end{pgfscope}%
\begin{pgfscope}%
\pgfpathrectangle{\pgfqpoint{0.150000in}{0.150000in}}{\pgfqpoint{2.700000in}{1.950000in}}%
\pgfusepath{clip}%
\pgfsetbuttcap%
\pgfsetroundjoin%
\definecolor{currentfill}{rgb}{0.986703,0.975873,0.976731}%
\pgfsetfillcolor{currentfill}%
\pgfsetlinewidth{0.000000pt}%
\definecolor{currentstroke}{rgb}{0.000000,0.000000,0.000000}%
\pgfsetstrokecolor{currentstroke}%
\pgfsetdash{}{0pt}%
\pgfpathmoveto{\pgfqpoint{1.274805in}{1.263642in}}%
\pgfpathlineto{\pgfqpoint{1.314274in}{1.202418in}}%
\pgfpathlineto{\pgfqpoint{1.277702in}{1.221206in}}%
\pgfpathlineto{\pgfqpoint{1.237737in}{1.290570in}}%
\pgfpathclose%
\pgfusepath{fill}%
\end{pgfscope}%
\begin{pgfscope}%
\pgfpathrectangle{\pgfqpoint{0.150000in}{0.150000in}}{\pgfqpoint{2.700000in}{1.950000in}}%
\pgfusepath{clip}%
\pgfsetbuttcap%
\pgfsetroundjoin%
\definecolor{currentfill}{rgb}{0.944914,0.900046,0.903600}%
\pgfsetfillcolor{currentfill}%
\pgfsetlinewidth{0.000000pt}%
\definecolor{currentstroke}{rgb}{0.000000,0.000000,0.000000}%
\pgfsetstrokecolor{currentstroke}%
\pgfsetdash{}{0pt}%
\pgfpathmoveto{\pgfqpoint{0.831663in}{1.106057in}}%
\pgfpathlineto{\pgfqpoint{0.864410in}{1.212542in}}%
\pgfpathlineto{\pgfqpoint{0.828354in}{1.231465in}}%
\pgfpathlineto{\pgfqpoint{0.795898in}{1.125049in}}%
\pgfpathclose%
\pgfusepath{fill}%
\end{pgfscope}%
\begin{pgfscope}%
\pgfpathrectangle{\pgfqpoint{0.150000in}{0.150000in}}{\pgfqpoint{2.700000in}{1.950000in}}%
\pgfusepath{clip}%
\pgfsetbuttcap%
\pgfsetroundjoin%
\definecolor{currentfill}{rgb}{0.827145,0.686351,0.697503}%
\pgfsetfillcolor{currentfill}%
\pgfsetlinewidth{0.000000pt}%
\definecolor{currentstroke}{rgb}{0.000000,0.000000,0.000000}%
\pgfsetstrokecolor{currentstroke}%
\pgfsetdash{}{0pt}%
\pgfpathmoveto{\pgfqpoint{1.871951in}{0.927946in}}%
\pgfpathlineto{\pgfqpoint{1.907993in}{0.931789in}}%
\pgfpathlineto{\pgfqpoint{1.870492in}{0.958897in}}%
\pgfpathlineto{\pgfqpoint{1.834364in}{0.955148in}}%
\pgfpathclose%
\pgfusepath{fill}%
\end{pgfscope}%
\begin{pgfscope}%
\pgfpathrectangle{\pgfqpoint{0.150000in}{0.150000in}}{\pgfqpoint{2.700000in}{1.950000in}}%
\pgfusepath{clip}%
\pgfsetbuttcap%
\pgfsetroundjoin%
\definecolor{currentfill}{rgb}{0.910723,0.838006,0.843765}%
\pgfsetfillcolor{currentfill}%
\pgfsetlinewidth{0.000000pt}%
\definecolor{currentstroke}{rgb}{0.000000,0.000000,0.000000}%
\pgfsetstrokecolor{currentstroke}%
\pgfsetdash{}{0pt}%
\pgfpathmoveto{\pgfqpoint{1.499172in}{1.091585in}}%
\pgfpathlineto{\pgfqpoint{1.536486in}{1.086997in}}%
\pgfpathlineto{\pgfqpoint{1.499390in}{1.106057in}}%
\pgfpathlineto{\pgfqpoint{1.461993in}{1.110714in}}%
\pgfpathclose%
\pgfusepath{fill}%
\end{pgfscope}%
\begin{pgfscope}%
\pgfpathrectangle{\pgfqpoint{0.150000in}{0.150000in}}{\pgfqpoint{2.700000in}{1.950000in}}%
\pgfusepath{clip}%
\pgfsetbuttcap%
\pgfsetroundjoin%
\definecolor{currentfill}{rgb}{0.830944,0.693244,0.704151}%
\pgfsetfillcolor{currentfill}%
\pgfsetlinewidth{0.000000pt}%
\definecolor{currentstroke}{rgb}{0.000000,0.000000,0.000000}%
\pgfsetstrokecolor{currentstroke}%
\pgfsetdash{}{0pt}%
\pgfpathmoveto{\pgfqpoint{2.096440in}{0.943553in}}%
\pgfpathlineto{\pgfqpoint{2.132692in}{0.962958in}}%
\pgfpathlineto{\pgfqpoint{2.092327in}{0.943359in}}%
\pgfpathlineto{\pgfqpoint{2.056987in}{0.939564in}}%
\pgfpathclose%
\pgfusepath{fill}%
\end{pgfscope}%
\begin{pgfscope}%
\pgfpathrectangle{\pgfqpoint{0.150000in}{0.150000in}}{\pgfqpoint{2.700000in}{1.950000in}}%
\pgfusepath{clip}%
\pgfsetbuttcap%
\pgfsetroundjoin%
\definecolor{currentfill}{rgb}{0.891728,0.803539,0.810524}%
\pgfsetfillcolor{currentfill}%
\pgfsetlinewidth{0.000000pt}%
\definecolor{currentstroke}{rgb}{0.000000,0.000000,0.000000}%
\pgfsetstrokecolor{currentstroke}%
\pgfsetdash{}{0pt}%
\pgfpathmoveto{\pgfqpoint{1.685562in}{1.040788in}}%
\pgfpathlineto{\pgfqpoint{1.722355in}{1.052116in}}%
\pgfpathlineto{\pgfqpoint{1.685138in}{1.086997in}}%
\pgfpathlineto{\pgfqpoint{1.648176in}{1.067868in}}%
\pgfpathclose%
\pgfusepath{fill}%
\end{pgfscope}%
\begin{pgfscope}%
\pgfpathrectangle{\pgfqpoint{0.150000in}{0.150000in}}{\pgfqpoint{2.700000in}{1.950000in}}%
\pgfusepath{clip}%
\pgfsetbuttcap%
\pgfsetroundjoin%
\definecolor{currentfill}{rgb}{0.853738,0.734605,0.744041}%
\pgfsetfillcolor{currentfill}%
\pgfsetlinewidth{0.000000pt}%
\definecolor{currentstroke}{rgb}{0.000000,0.000000,0.000000}%
\pgfsetstrokecolor{currentstroke}%
\pgfsetdash{}{0pt}%
\pgfpathmoveto{\pgfqpoint{2.208740in}{0.986458in}}%
\pgfpathlineto{\pgfqpoint{2.244806in}{1.005794in}}%
\pgfpathlineto{\pgfqpoint{2.203800in}{0.985948in}}%
\pgfpathlineto{\pgfqpoint{2.168337in}{0.974482in}}%
\pgfpathclose%
\pgfusepath{fill}%
\end{pgfscope}%
\begin{pgfscope}%
\pgfpathrectangle{\pgfqpoint{0.150000in}{0.150000in}}{\pgfqpoint{2.700000in}{1.950000in}}%
\pgfusepath{clip}%
\pgfsetbuttcap%
\pgfsetroundjoin%
\definecolor{currentfill}{rgb}{0.979105,0.962086,0.963434}%
\pgfsetfillcolor{currentfill}%
\pgfsetlinewidth{0.000000pt}%
\definecolor{currentstroke}{rgb}{0.000000,0.000000,0.000000}%
\pgfsetstrokecolor{currentstroke}%
\pgfsetdash{}{0pt}%
\pgfpathmoveto{\pgfqpoint{1.311951in}{1.236657in}}%
\pgfpathlineto{\pgfqpoint{1.351116in}{1.175621in}}%
\pgfpathlineto{\pgfqpoint{1.314274in}{1.202418in}}%
\pgfpathlineto{\pgfqpoint{1.274805in}{1.263642in}}%
\pgfpathclose%
\pgfusepath{fill}%
\end{pgfscope}%
\begin{pgfscope}%
\pgfpathrectangle{\pgfqpoint{0.150000in}{0.150000in}}{\pgfqpoint{2.700000in}{1.950000in}}%
\pgfusepath{clip}%
\pgfsetbuttcap%
\pgfsetroundjoin%
\definecolor{currentfill}{rgb}{0.899326,0.817325,0.823820}%
\pgfsetfillcolor{currentfill}%
\pgfsetlinewidth{0.000000pt}%
\definecolor{currentstroke}{rgb}{0.000000,0.000000,0.000000}%
\pgfsetstrokecolor{currentstroke}%
\pgfsetdash{}{0pt}%
\pgfpathmoveto{\pgfqpoint{2.431617in}{1.048670in}}%
\pgfpathlineto{\pgfqpoint{2.467232in}{1.067868in}}%
\pgfpathlineto{\pgfqpoint{2.428393in}{1.086997in}}%
\pgfpathlineto{\pgfqpoint{2.392772in}{1.067868in}}%
\pgfpathclose%
\pgfusepath{fill}%
\end{pgfscope}%
\begin{pgfscope}%
\pgfpathrectangle{\pgfqpoint{0.150000in}{0.150000in}}{\pgfqpoint{2.700000in}{1.950000in}}%
\pgfusepath{clip}%
\pgfsetbuttcap%
\pgfsetroundjoin%
\definecolor{currentfill}{rgb}{0.899326,0.817325,0.823820}%
\pgfsetfillcolor{currentfill}%
\pgfsetlinewidth{0.000000pt}%
\definecolor{currentstroke}{rgb}{0.000000,0.000000,0.000000}%
\pgfsetstrokecolor{currentstroke}%
\pgfsetdash{}{0pt}%
\pgfpathmoveto{\pgfqpoint{1.611081in}{1.048670in}}%
\pgfpathlineto{\pgfqpoint{1.648176in}{1.067868in}}%
\pgfpathlineto{\pgfqpoint{1.610812in}{1.086997in}}%
\pgfpathlineto{\pgfqpoint{1.573716in}{1.067868in}}%
\pgfpathclose%
\pgfusepath{fill}%
\end{pgfscope}%
\begin{pgfscope}%
\pgfpathrectangle{\pgfqpoint{0.150000in}{0.150000in}}{\pgfqpoint{2.700000in}{1.950000in}}%
\pgfusepath{clip}%
\pgfsetbuttcap%
\pgfsetroundjoin%
\definecolor{currentfill}{rgb}{0.899326,0.817325,0.823820}%
\pgfsetfillcolor{currentfill}%
\pgfsetlinewidth{0.000000pt}%
\definecolor{currentstroke}{rgb}{0.000000,0.000000,0.000000}%
\pgfsetstrokecolor{currentstroke}%
\pgfsetdash{}{0pt}%
\pgfpathmoveto{\pgfqpoint{0.865139in}{1.048670in}}%
\pgfpathlineto{\pgfqpoint{0.903580in}{1.067868in}}%
\pgfpathlineto{\pgfqpoint{0.867556in}{1.086997in}}%
\pgfpathlineto{\pgfqpoint{0.829120in}{1.067868in}}%
\pgfpathclose%
\pgfusepath{fill}%
\end{pgfscope}%
\begin{pgfscope}%
\pgfpathrectangle{\pgfqpoint{0.150000in}{0.150000in}}{\pgfqpoint{2.700000in}{1.950000in}}%
\pgfusepath{clip}%
\pgfsetbuttcap%
\pgfsetroundjoin%
\definecolor{currentfill}{rgb}{0.899326,0.817325,0.823820}%
\pgfsetfillcolor{currentfill}%
\pgfsetlinewidth{0.000000pt}%
\definecolor{currentstroke}{rgb}{0.000000,0.000000,0.000000}%
\pgfsetstrokecolor{currentstroke}%
\pgfsetdash{}{0pt}%
\pgfpathmoveto{\pgfqpoint{0.790545in}{1.048670in}}%
\pgfpathlineto{\pgfqpoint{0.829120in}{1.067868in}}%
\pgfpathlineto{\pgfqpoint{0.793231in}{1.086997in}}%
\pgfpathlineto{\pgfqpoint{0.754660in}{1.067868in}}%
\pgfpathclose%
\pgfusepath{fill}%
\end{pgfscope}%
\begin{pgfscope}%
\pgfpathrectangle{\pgfqpoint{0.150000in}{0.150000in}}{\pgfqpoint{2.700000in}{1.950000in}}%
\pgfusepath{clip}%
\pgfsetbuttcap%
\pgfsetroundjoin%
\definecolor{currentfill}{rgb}{0.899326,0.817325,0.823820}%
\pgfsetfillcolor{currentfill}%
\pgfsetlinewidth{0.000000pt}%
\definecolor{currentstroke}{rgb}{0.000000,0.000000,0.000000}%
\pgfsetstrokecolor{currentstroke}%
\pgfsetdash{}{0pt}%
\pgfpathmoveto{\pgfqpoint{0.715950in}{1.048670in}}%
\pgfpathlineto{\pgfqpoint{0.754660in}{1.067868in}}%
\pgfpathlineto{\pgfqpoint{0.718905in}{1.086997in}}%
\pgfpathlineto{\pgfqpoint{0.680201in}{1.067868in}}%
\pgfpathclose%
\pgfusepath{fill}%
\end{pgfscope}%
\begin{pgfscope}%
\pgfpathrectangle{\pgfqpoint{0.150000in}{0.150000in}}{\pgfqpoint{2.700000in}{1.950000in}}%
\pgfusepath{clip}%
\pgfsetbuttcap%
\pgfsetroundjoin%
\definecolor{currentfill}{rgb}{0.899326,0.817325,0.823820}%
\pgfsetfillcolor{currentfill}%
\pgfsetlinewidth{0.000000pt}%
\definecolor{currentstroke}{rgb}{0.000000,0.000000,0.000000}%
\pgfsetstrokecolor{currentstroke}%
\pgfsetdash{}{0pt}%
\pgfpathmoveto{\pgfqpoint{0.641356in}{1.048670in}}%
\pgfpathlineto{\pgfqpoint{0.680201in}{1.067868in}}%
\pgfpathlineto{\pgfqpoint{0.644580in}{1.086997in}}%
\pgfpathlineto{\pgfqpoint{0.605741in}{1.067868in}}%
\pgfpathclose%
\pgfusepath{fill}%
\end{pgfscope}%
\begin{pgfscope}%
\pgfpathrectangle{\pgfqpoint{0.150000in}{0.150000in}}{\pgfqpoint{2.700000in}{1.950000in}}%
\pgfusepath{clip}%
\pgfsetbuttcap%
\pgfsetroundjoin%
\definecolor{currentfill}{rgb}{0.880331,0.782858,0.790579}%
\pgfsetfillcolor{currentfill}%
\pgfsetlinewidth{0.000000pt}%
\definecolor{currentstroke}{rgb}{0.000000,0.000000,0.000000}%
\pgfsetstrokecolor{currentstroke}%
\pgfsetdash{}{0pt}%
\pgfpathmoveto{\pgfqpoint{1.723027in}{1.013652in}}%
\pgfpathlineto{\pgfqpoint{1.759595in}{1.017215in}}%
\pgfpathlineto{\pgfqpoint{1.722355in}{1.052116in}}%
\pgfpathlineto{\pgfqpoint{1.685562in}{1.040788in}}%
\pgfpathclose%
\pgfusepath{fill}%
\end{pgfscope}%
\begin{pgfscope}%
\pgfpathrectangle{\pgfqpoint{0.150000in}{0.150000in}}{\pgfqpoint{2.700000in}{1.950000in}}%
\pgfusepath{clip}%
\pgfsetbuttcap%
\pgfsetroundjoin%
\definecolor{currentfill}{rgb}{0.944914,0.900046,0.903600}%
\pgfsetfillcolor{currentfill}%
\pgfsetlinewidth{0.000000pt}%
\definecolor{currentstroke}{rgb}{0.000000,0.000000,0.000000}%
\pgfsetstrokecolor{currentstroke}%
\pgfsetdash{}{0pt}%
\pgfpathmoveto{\pgfqpoint{0.867556in}{1.086997in}}%
\pgfpathlineto{\pgfqpoint{0.901079in}{1.185536in}}%
\pgfpathlineto{\pgfqpoint{0.864410in}{1.212542in}}%
\pgfpathlineto{\pgfqpoint{0.831663in}{1.106057in}}%
\pgfpathclose%
\pgfusepath{fill}%
\end{pgfscope}%
\begin{pgfscope}%
\pgfpathrectangle{\pgfqpoint{0.150000in}{0.150000in}}{\pgfqpoint{2.700000in}{1.950000in}}%
\pgfusepath{clip}%
\pgfsetbuttcap%
\pgfsetroundjoin%
\definecolor{currentfill}{rgb}{0.887929,0.796645,0.803876}%
\pgfsetfillcolor{currentfill}%
\pgfsetlinewidth{0.000000pt}%
\definecolor{currentstroke}{rgb}{0.000000,0.000000,0.000000}%
\pgfsetstrokecolor{currentstroke}%
\pgfsetdash{}{0pt}%
\pgfpathmoveto{\pgfqpoint{2.321143in}{1.029402in}}%
\pgfpathlineto{\pgfqpoint{2.357023in}{1.048670in}}%
\pgfpathlineto{\pgfqpoint{2.316547in}{1.044258in}}%
\pgfpathlineto{\pgfqpoint{2.280741in}{1.025061in}}%
\pgfpathclose%
\pgfusepath{fill}%
\end{pgfscope}%
\begin{pgfscope}%
\pgfpathrectangle{\pgfqpoint{0.150000in}{0.150000in}}{\pgfqpoint{2.700000in}{1.950000in}}%
\pgfusepath{clip}%
\pgfsetbuttcap%
\pgfsetroundjoin%
\definecolor{currentfill}{rgb}{0.868934,0.762178,0.770634}%
\pgfsetfillcolor{currentfill}%
\pgfsetlinewidth{0.000000pt}%
\definecolor{currentstroke}{rgb}{0.000000,0.000000,0.000000}%
\pgfsetstrokecolor{currentstroke}%
\pgfsetdash{}{0pt}%
\pgfpathmoveto{\pgfqpoint{1.760571in}{0.986458in}}%
\pgfpathlineto{\pgfqpoint{1.797052in}{0.990115in}}%
\pgfpathlineto{\pgfqpoint{1.759595in}{1.017215in}}%
\pgfpathlineto{\pgfqpoint{1.723027in}{1.013652in}}%
\pgfpathclose%
\pgfusepath{fill}%
\end{pgfscope}%
\begin{pgfscope}%
\pgfpathrectangle{\pgfqpoint{0.150000in}{0.150000in}}{\pgfqpoint{2.700000in}{1.950000in}}%
\pgfusepath{clip}%
\pgfsetbuttcap%
\pgfsetroundjoin%
\definecolor{currentfill}{rgb}{0.910723,0.838006,0.843765}%
\pgfsetfillcolor{currentfill}%
\pgfsetlinewidth{0.000000pt}%
\definecolor{currentstroke}{rgb}{0.000000,0.000000,0.000000}%
\pgfsetstrokecolor{currentstroke}%
\pgfsetdash{}{0pt}%
\pgfpathmoveto{\pgfqpoint{1.536486in}{1.072386in}}%
\pgfpathlineto{\pgfqpoint{1.573716in}{1.067868in}}%
\pgfpathlineto{\pgfqpoint{1.536486in}{1.086997in}}%
\pgfpathlineto{\pgfqpoint{1.499172in}{1.091585in}}%
\pgfpathclose%
\pgfusepath{fill}%
\end{pgfscope}%
\begin{pgfscope}%
\pgfpathrectangle{\pgfqpoint{0.150000in}{0.150000in}}{\pgfqpoint{2.700000in}{1.950000in}}%
\pgfusepath{clip}%
\pgfsetbuttcap%
\pgfsetroundjoin%
\definecolor{currentfill}{rgb}{0.830944,0.693244,0.704151}%
\pgfsetfillcolor{currentfill}%
\pgfsetlinewidth{0.000000pt}%
\definecolor{currentstroke}{rgb}{0.000000,0.000000,0.000000}%
\pgfsetstrokecolor{currentstroke}%
\pgfsetdash{}{0pt}%
\pgfpathmoveto{\pgfqpoint{1.985261in}{0.924078in}}%
\pgfpathlineto{\pgfqpoint{2.021046in}{0.927946in}}%
\pgfpathlineto{\pgfqpoint{1.982294in}{0.931789in}}%
\pgfpathlineto{\pgfqpoint{1.946499in}{0.927946in}}%
\pgfpathclose%
\pgfusepath{fill}%
\end{pgfscope}%
\begin{pgfscope}%
\pgfpathrectangle{\pgfqpoint{0.150000in}{0.150000in}}{\pgfqpoint{2.700000in}{1.950000in}}%
\pgfusepath{clip}%
\pgfsetbuttcap%
\pgfsetroundjoin%
\definecolor{currentfill}{rgb}{0.830944,0.693244,0.704151}%
\pgfsetfillcolor{currentfill}%
\pgfsetlinewidth{0.000000pt}%
\definecolor{currentstroke}{rgb}{0.000000,0.000000,0.000000}%
\pgfsetstrokecolor{currentstroke}%
\pgfsetdash{}{0pt}%
\pgfpathmoveto{\pgfqpoint{1.910465in}{0.924078in}}%
\pgfpathlineto{\pgfqpoint{1.946499in}{0.927946in}}%
\pgfpathlineto{\pgfqpoint{1.907993in}{0.931789in}}%
\pgfpathlineto{\pgfqpoint{1.871951in}{0.927946in}}%
\pgfpathclose%
\pgfusepath{fill}%
\end{pgfscope}%
\begin{pgfscope}%
\pgfpathrectangle{\pgfqpoint{0.150000in}{0.150000in}}{\pgfqpoint{2.700000in}{1.950000in}}%
\pgfusepath{clip}%
\pgfsetbuttcap%
\pgfsetroundjoin%
\definecolor{currentfill}{rgb}{0.971507,0.948300,0.950138}%
\pgfsetfillcolor{currentfill}%
\pgfsetlinewidth{0.000000pt}%
\definecolor{currentstroke}{rgb}{0.000000,0.000000,0.000000}%
\pgfsetstrokecolor{currentstroke}%
\pgfsetdash{}{0pt}%
\pgfpathmoveto{\pgfqpoint{1.349176in}{1.209615in}}%
\pgfpathlineto{\pgfqpoint{1.387923in}{1.156697in}}%
\pgfpathlineto{\pgfqpoint{1.351116in}{1.175621in}}%
\pgfpathlineto{\pgfqpoint{1.311951in}{1.236657in}}%
\pgfpathclose%
\pgfusepath{fill}%
\end{pgfscope}%
\begin{pgfscope}%
\pgfpathrectangle{\pgfqpoint{0.150000in}{0.150000in}}{\pgfqpoint{2.700000in}{1.950000in}}%
\pgfusepath{clip}%
\pgfsetbuttcap%
\pgfsetroundjoin%
\definecolor{currentfill}{rgb}{0.895527,0.810432,0.817172}%
\pgfsetfillcolor{currentfill}%
\pgfsetlinewidth{0.000000pt}%
\definecolor{currentstroke}{rgb}{0.000000,0.000000,0.000000}%
\pgfsetstrokecolor{currentstroke}%
\pgfsetdash{}{0pt}%
\pgfpathmoveto{\pgfqpoint{1.648496in}{1.021521in}}%
\pgfpathlineto{\pgfqpoint{1.685562in}{1.040788in}}%
\pgfpathlineto{\pgfqpoint{1.648176in}{1.067868in}}%
\pgfpathlineto{\pgfqpoint{1.611081in}{1.048670in}}%
\pgfpathclose%
\pgfusepath{fill}%
\end{pgfscope}%
\begin{pgfscope}%
\pgfpathrectangle{\pgfqpoint{0.150000in}{0.150000in}}{\pgfqpoint{2.700000in}{1.950000in}}%
\pgfusepath{clip}%
\pgfsetbuttcap%
\pgfsetroundjoin%
\definecolor{currentfill}{rgb}{0.899326,0.817325,0.823820}%
\pgfsetfillcolor{currentfill}%
\pgfsetlinewidth{0.000000pt}%
\definecolor{currentstroke}{rgb}{0.000000,0.000000,0.000000}%
\pgfsetstrokecolor{currentstroke}%
\pgfsetdash{}{0pt}%
\pgfpathmoveto{\pgfqpoint{2.395873in}{1.029402in}}%
\pgfpathlineto{\pgfqpoint{2.431617in}{1.048670in}}%
\pgfpathlineto{\pgfqpoint{2.392772in}{1.067868in}}%
\pgfpathlineto{\pgfqpoint{2.357023in}{1.048670in}}%
\pgfpathclose%
\pgfusepath{fill}%
\end{pgfscope}%
\begin{pgfscope}%
\pgfpathrectangle{\pgfqpoint{0.150000in}{0.150000in}}{\pgfqpoint{2.700000in}{1.950000in}}%
\pgfusepath{clip}%
\pgfsetbuttcap%
\pgfsetroundjoin%
\definecolor{currentfill}{rgb}{0.899326,0.817325,0.823820}%
\pgfsetfillcolor{currentfill}%
\pgfsetlinewidth{0.000000pt}%
\definecolor{currentstroke}{rgb}{0.000000,0.000000,0.000000}%
\pgfsetstrokecolor{currentstroke}%
\pgfsetdash{}{0pt}%
\pgfpathmoveto{\pgfqpoint{0.901288in}{1.029402in}}%
\pgfpathlineto{\pgfqpoint{0.939733in}{1.048670in}}%
\pgfpathlineto{\pgfqpoint{0.903580in}{1.067868in}}%
\pgfpathlineto{\pgfqpoint{0.865139in}{1.048670in}}%
\pgfpathclose%
\pgfusepath{fill}%
\end{pgfscope}%
\begin{pgfscope}%
\pgfpathrectangle{\pgfqpoint{0.150000in}{0.150000in}}{\pgfqpoint{2.700000in}{1.950000in}}%
\pgfusepath{clip}%
\pgfsetbuttcap%
\pgfsetroundjoin%
\definecolor{currentfill}{rgb}{0.899326,0.817325,0.823820}%
\pgfsetfillcolor{currentfill}%
\pgfsetlinewidth{0.000000pt}%
\definecolor{currentstroke}{rgb}{0.000000,0.000000,0.000000}%
\pgfsetstrokecolor{currentstroke}%
\pgfsetdash{}{0pt}%
\pgfpathmoveto{\pgfqpoint{0.826559in}{1.029402in}}%
\pgfpathlineto{\pgfqpoint{0.865139in}{1.048670in}}%
\pgfpathlineto{\pgfqpoint{0.829120in}{1.067868in}}%
\pgfpathlineto{\pgfqpoint{0.790545in}{1.048670in}}%
\pgfpathclose%
\pgfusepath{fill}%
\end{pgfscope}%
\begin{pgfscope}%
\pgfpathrectangle{\pgfqpoint{0.150000in}{0.150000in}}{\pgfqpoint{2.700000in}{1.950000in}}%
\pgfusepath{clip}%
\pgfsetbuttcap%
\pgfsetroundjoin%
\definecolor{currentfill}{rgb}{0.899326,0.817325,0.823820}%
\pgfsetfillcolor{currentfill}%
\pgfsetlinewidth{0.000000pt}%
\definecolor{currentstroke}{rgb}{0.000000,0.000000,0.000000}%
\pgfsetstrokecolor{currentstroke}%
\pgfsetdash{}{0pt}%
\pgfpathmoveto{\pgfqpoint{0.751830in}{1.029402in}}%
\pgfpathlineto{\pgfqpoint{0.790545in}{1.048670in}}%
\pgfpathlineto{\pgfqpoint{0.754660in}{1.067868in}}%
\pgfpathlineto{\pgfqpoint{0.715950in}{1.048670in}}%
\pgfpathclose%
\pgfusepath{fill}%
\end{pgfscope}%
\begin{pgfscope}%
\pgfpathrectangle{\pgfqpoint{0.150000in}{0.150000in}}{\pgfqpoint{2.700000in}{1.950000in}}%
\pgfusepath{clip}%
\pgfsetbuttcap%
\pgfsetroundjoin%
\definecolor{currentfill}{rgb}{0.899326,0.817325,0.823820}%
\pgfsetfillcolor{currentfill}%
\pgfsetlinewidth{0.000000pt}%
\definecolor{currentstroke}{rgb}{0.000000,0.000000,0.000000}%
\pgfsetstrokecolor{currentstroke}%
\pgfsetdash{}{0pt}%
\pgfpathmoveto{\pgfqpoint{0.677100in}{1.029402in}}%
\pgfpathlineto{\pgfqpoint{0.715950in}{1.048670in}}%
\pgfpathlineto{\pgfqpoint{0.680201in}{1.067868in}}%
\pgfpathlineto{\pgfqpoint{0.641356in}{1.048670in}}%
\pgfpathclose%
\pgfusepath{fill}%
\end{pgfscope}%
\begin{pgfscope}%
\pgfpathrectangle{\pgfqpoint{0.150000in}{0.150000in}}{\pgfqpoint{2.700000in}{1.950000in}}%
\pgfusepath{clip}%
\pgfsetbuttcap%
\pgfsetroundjoin%
\definecolor{currentfill}{rgb}{0.861336,0.748392,0.757338}%
\pgfsetfillcolor{currentfill}%
\pgfsetlinewidth{0.000000pt}%
\definecolor{currentstroke}{rgb}{0.000000,0.000000,0.000000}%
\pgfsetstrokecolor{currentstroke}%
\pgfsetdash{}{0pt}%
\pgfpathmoveto{\pgfqpoint{1.798194in}{0.959207in}}%
\pgfpathlineto{\pgfqpoint{1.834364in}{0.955148in}}%
\pgfpathlineto{\pgfqpoint{1.797052in}{0.990115in}}%
\pgfpathlineto{\pgfqpoint{1.760571in}{0.986458in}}%
\pgfpathclose%
\pgfusepath{fill}%
\end{pgfscope}%
\begin{pgfscope}%
\pgfpathrectangle{\pgfqpoint{0.150000in}{0.150000in}}{\pgfqpoint{2.700000in}{1.950000in}}%
\pgfusepath{clip}%
\pgfsetbuttcap%
\pgfsetroundjoin%
\definecolor{currentfill}{rgb}{0.842341,0.713925,0.724096}%
\pgfsetfillcolor{currentfill}%
\pgfsetlinewidth{0.000000pt}%
\definecolor{currentstroke}{rgb}{0.000000,0.000000,0.000000}%
\pgfsetstrokecolor{currentstroke}%
\pgfsetdash{}{0pt}%
\pgfpathmoveto{\pgfqpoint{2.060453in}{0.931898in}}%
\pgfpathlineto{\pgfqpoint{2.096440in}{0.943553in}}%
\pgfpathlineto{\pgfqpoint{2.056987in}{0.939564in}}%
\pgfpathlineto{\pgfqpoint{2.021046in}{0.927946in}}%
\pgfpathclose%
\pgfusepath{fill}%
\end{pgfscope}%
\begin{pgfscope}%
\pgfpathrectangle{\pgfqpoint{0.150000in}{0.150000in}}{\pgfqpoint{2.700000in}{1.950000in}}%
\pgfusepath{clip}%
\pgfsetbuttcap%
\pgfsetroundjoin%
\definecolor{currentfill}{rgb}{0.941115,0.893153,0.896952}%
\pgfsetfillcolor{currentfill}%
\pgfsetlinewidth{0.000000pt}%
\definecolor{currentstroke}{rgb}{0.000000,0.000000,0.000000}%
\pgfsetstrokecolor{currentstroke}%
\pgfsetdash{}{0pt}%
\pgfpathmoveto{\pgfqpoint{0.903580in}{1.067868in}}%
\pgfpathlineto{\pgfqpoint{0.937371in}{1.166476in}}%
\pgfpathlineto{\pgfqpoint{0.901079in}{1.185536in}}%
\pgfpathlineto{\pgfqpoint{0.867556in}{1.086997in}}%
\pgfpathclose%
\pgfusepath{fill}%
\end{pgfscope}%
\begin{pgfscope}%
\pgfpathrectangle{\pgfqpoint{0.150000in}{0.150000in}}{\pgfqpoint{2.700000in}{1.950000in}}%
\pgfusepath{clip}%
\pgfsetbuttcap%
\pgfsetroundjoin%
\definecolor{currentfill}{rgb}{0.861336,0.748392,0.757338}%
\pgfsetfillcolor{currentfill}%
\pgfsetlinewidth{0.000000pt}%
\definecolor{currentstroke}{rgb}{0.000000,0.000000,0.000000}%
\pgfsetstrokecolor{currentstroke}%
\pgfsetdash{}{0pt}%
\pgfpathmoveto{\pgfqpoint{2.173026in}{0.974908in}}%
\pgfpathlineto{\pgfqpoint{2.208740in}{0.986458in}}%
\pgfpathlineto{\pgfqpoint{2.168337in}{0.974482in}}%
\pgfpathlineto{\pgfqpoint{2.132692in}{0.962958in}}%
\pgfpathclose%
\pgfusepath{fill}%
\end{pgfscope}%
\begin{pgfscope}%
\pgfpathrectangle{\pgfqpoint{0.150000in}{0.150000in}}{\pgfqpoint{2.700000in}{1.950000in}}%
\pgfusepath{clip}%
\pgfsetbuttcap%
\pgfsetroundjoin%
\definecolor{currentfill}{rgb}{0.887929,0.796645,0.803876}%
\pgfsetfillcolor{currentfill}%
\pgfsetlinewidth{0.000000pt}%
\definecolor{currentstroke}{rgb}{0.000000,0.000000,0.000000}%
\pgfsetstrokecolor{currentstroke}%
\pgfsetdash{}{0pt}%
\pgfpathmoveto{\pgfqpoint{2.285134in}{1.010065in}}%
\pgfpathlineto{\pgfqpoint{2.321143in}{1.029402in}}%
\pgfpathlineto{\pgfqpoint{2.280741in}{1.025061in}}%
\pgfpathlineto{\pgfqpoint{2.244806in}{1.005794in}}%
\pgfpathclose%
\pgfusepath{fill}%
\end{pgfscope}%
\begin{pgfscope}%
\pgfpathrectangle{\pgfqpoint{0.150000in}{0.150000in}}{\pgfqpoint{2.700000in}{1.950000in}}%
\pgfusepath{clip}%
\pgfsetbuttcap%
\pgfsetroundjoin%
\definecolor{currentfill}{rgb}{0.967708,0.941406,0.943490}%
\pgfsetfillcolor{currentfill}%
\pgfsetlinewidth{0.000000pt}%
\definecolor{currentstroke}{rgb}{0.000000,0.000000,0.000000}%
\pgfsetstrokecolor{currentstroke}%
\pgfsetdash{}{0pt}%
\pgfpathmoveto{\pgfqpoint{1.386480in}{1.182516in}}%
\pgfpathlineto{\pgfqpoint{1.424863in}{1.137705in}}%
\pgfpathlineto{\pgfqpoint{1.387923in}{1.156697in}}%
\pgfpathlineto{\pgfqpoint{1.349176in}{1.209615in}}%
\pgfpathclose%
\pgfusepath{fill}%
\end{pgfscope}%
\begin{pgfscope}%
\pgfpathrectangle{\pgfqpoint{0.150000in}{0.150000in}}{\pgfqpoint{2.700000in}{1.950000in}}%
\pgfusepath{clip}%
\pgfsetbuttcap%
\pgfsetroundjoin%
\definecolor{currentfill}{rgb}{0.887929,0.796645,0.803876}%
\pgfsetfillcolor{currentfill}%
\pgfsetlinewidth{0.000000pt}%
\definecolor{currentstroke}{rgb}{0.000000,0.000000,0.000000}%
\pgfsetstrokecolor{currentstroke}%
\pgfsetdash{}{0pt}%
\pgfpathmoveto{\pgfqpoint{1.686103in}{1.002184in}}%
\pgfpathlineto{\pgfqpoint{1.723027in}{1.013652in}}%
\pgfpathlineto{\pgfqpoint{1.685562in}{1.040788in}}%
\pgfpathlineto{\pgfqpoint{1.648496in}{1.021521in}}%
\pgfpathclose%
\pgfusepath{fill}%
\end{pgfscope}%
\begin{pgfscope}%
\pgfpathrectangle{\pgfqpoint{0.150000in}{0.150000in}}{\pgfqpoint{2.700000in}{1.950000in}}%
\pgfusepath{clip}%
\pgfsetbuttcap%
\pgfsetroundjoin%
\definecolor{currentfill}{rgb}{0.849939,0.727711,0.737393}%
\pgfsetfillcolor{currentfill}%
\pgfsetlinewidth{0.000000pt}%
\definecolor{currentstroke}{rgb}{0.000000,0.000000,0.000000}%
\pgfsetstrokecolor{currentstroke}%
\pgfsetdash{}{0pt}%
\pgfpathmoveto{\pgfqpoint{1.835896in}{0.931898in}}%
\pgfpathlineto{\pgfqpoint{1.871951in}{0.927946in}}%
\pgfpathlineto{\pgfqpoint{1.834364in}{0.955148in}}%
\pgfpathlineto{\pgfqpoint{1.798194in}{0.959207in}}%
\pgfpathclose%
\pgfusepath{fill}%
\end{pgfscope}%
\begin{pgfscope}%
\pgfpathrectangle{\pgfqpoint{0.150000in}{0.150000in}}{\pgfqpoint{2.700000in}{1.950000in}}%
\pgfusepath{clip}%
\pgfsetbuttcap%
\pgfsetroundjoin%
\definecolor{currentfill}{rgb}{0.910723,0.838006,0.843765}%
\pgfsetfillcolor{currentfill}%
\pgfsetlinewidth{0.000000pt}%
\definecolor{currentstroke}{rgb}{0.000000,0.000000,0.000000}%
\pgfsetstrokecolor{currentstroke}%
\pgfsetdash{}{0pt}%
\pgfpathmoveto{\pgfqpoint{1.573936in}{1.053118in}}%
\pgfpathlineto{\pgfqpoint{1.611081in}{1.048670in}}%
\pgfpathlineto{\pgfqpoint{1.573716in}{1.067868in}}%
\pgfpathlineto{\pgfqpoint{1.536486in}{1.072386in}}%
\pgfpathclose%
\pgfusepath{fill}%
\end{pgfscope}%
\begin{pgfscope}%
\pgfpathrectangle{\pgfqpoint{0.150000in}{0.150000in}}{\pgfqpoint{2.700000in}{1.950000in}}%
\pgfusepath{clip}%
\pgfsetbuttcap%
\pgfsetroundjoin%
\definecolor{currentfill}{rgb}{0.941115,0.893153,0.896952}%
\pgfsetfillcolor{currentfill}%
\pgfsetlinewidth{0.000000pt}%
\definecolor{currentstroke}{rgb}{0.000000,0.000000,0.000000}%
\pgfsetstrokecolor{currentstroke}%
\pgfsetdash{}{0pt}%
\pgfpathmoveto{\pgfqpoint{0.939733in}{1.048670in}}%
\pgfpathlineto{\pgfqpoint{0.974222in}{1.139344in}}%
\pgfpathlineto{\pgfqpoint{0.937371in}{1.166476in}}%
\pgfpathlineto{\pgfqpoint{0.903580in}{1.067868in}}%
\pgfpathclose%
\pgfusepath{fill}%
\end{pgfscope}%
\begin{pgfscope}%
\pgfpathrectangle{\pgfqpoint{0.150000in}{0.150000in}}{\pgfqpoint{2.700000in}{1.950000in}}%
\pgfusepath{clip}%
\pgfsetbuttcap%
\pgfsetroundjoin%
\definecolor{currentfill}{rgb}{0.899326,0.817325,0.823820}%
\pgfsetfillcolor{currentfill}%
\pgfsetlinewidth{0.000000pt}%
\definecolor{currentstroke}{rgb}{0.000000,0.000000,0.000000}%
\pgfsetstrokecolor{currentstroke}%
\pgfsetdash{}{0pt}%
\pgfpathmoveto{\pgfqpoint{2.359999in}{1.010065in}}%
\pgfpathlineto{\pgfqpoint{2.395873in}{1.029402in}}%
\pgfpathlineto{\pgfqpoint{2.357023in}{1.048670in}}%
\pgfpathlineto{\pgfqpoint{2.321143in}{1.029402in}}%
\pgfpathclose%
\pgfusepath{fill}%
\end{pgfscope}%
\begin{pgfscope}%
\pgfpathrectangle{\pgfqpoint{0.150000in}{0.150000in}}{\pgfqpoint{2.700000in}{1.950000in}}%
\pgfusepath{clip}%
\pgfsetbuttcap%
\pgfsetroundjoin%
\definecolor{currentfill}{rgb}{0.899326,0.817325,0.823820}%
\pgfsetfillcolor{currentfill}%
\pgfsetlinewidth{0.000000pt}%
\definecolor{currentstroke}{rgb}{0.000000,0.000000,0.000000}%
\pgfsetstrokecolor{currentstroke}%
\pgfsetdash{}{0pt}%
\pgfpathmoveto{\pgfqpoint{0.937568in}{1.010065in}}%
\pgfpathlineto{\pgfqpoint{0.976017in}{1.029402in}}%
\pgfpathlineto{\pgfqpoint{0.939733in}{1.048670in}}%
\pgfpathlineto{\pgfqpoint{0.901288in}{1.029402in}}%
\pgfpathclose%
\pgfusepath{fill}%
\end{pgfscope}%
\begin{pgfscope}%
\pgfpathrectangle{\pgfqpoint{0.150000in}{0.150000in}}{\pgfqpoint{2.700000in}{1.950000in}}%
\pgfusepath{clip}%
\pgfsetbuttcap%
\pgfsetroundjoin%
\definecolor{currentfill}{rgb}{0.899326,0.817325,0.823820}%
\pgfsetfillcolor{currentfill}%
\pgfsetlinewidth{0.000000pt}%
\definecolor{currentstroke}{rgb}{0.000000,0.000000,0.000000}%
\pgfsetstrokecolor{currentstroke}%
\pgfsetdash{}{0pt}%
\pgfpathmoveto{\pgfqpoint{0.862704in}{1.010065in}}%
\pgfpathlineto{\pgfqpoint{0.901288in}{1.029402in}}%
\pgfpathlineto{\pgfqpoint{0.865139in}{1.048670in}}%
\pgfpathlineto{\pgfqpoint{0.826559in}{1.029402in}}%
\pgfpathclose%
\pgfusepath{fill}%
\end{pgfscope}%
\begin{pgfscope}%
\pgfpathrectangle{\pgfqpoint{0.150000in}{0.150000in}}{\pgfqpoint{2.700000in}{1.950000in}}%
\pgfusepath{clip}%
\pgfsetbuttcap%
\pgfsetroundjoin%
\definecolor{currentfill}{rgb}{0.899326,0.817325,0.823820}%
\pgfsetfillcolor{currentfill}%
\pgfsetlinewidth{0.000000pt}%
\definecolor{currentstroke}{rgb}{0.000000,0.000000,0.000000}%
\pgfsetstrokecolor{currentstroke}%
\pgfsetdash{}{0pt}%
\pgfpathmoveto{\pgfqpoint{0.787839in}{1.010065in}}%
\pgfpathlineto{\pgfqpoint{0.826559in}{1.029402in}}%
\pgfpathlineto{\pgfqpoint{0.790545in}{1.048670in}}%
\pgfpathlineto{\pgfqpoint{0.751830in}{1.029402in}}%
\pgfpathclose%
\pgfusepath{fill}%
\end{pgfscope}%
\begin{pgfscope}%
\pgfpathrectangle{\pgfqpoint{0.150000in}{0.150000in}}{\pgfqpoint{2.700000in}{1.950000in}}%
\pgfusepath{clip}%
\pgfsetbuttcap%
\pgfsetroundjoin%
\definecolor{currentfill}{rgb}{0.899326,0.817325,0.823820}%
\pgfsetfillcolor{currentfill}%
\pgfsetlinewidth{0.000000pt}%
\definecolor{currentstroke}{rgb}{0.000000,0.000000,0.000000}%
\pgfsetstrokecolor{currentstroke}%
\pgfsetdash{}{0pt}%
\pgfpathmoveto{\pgfqpoint{0.712974in}{1.010065in}}%
\pgfpathlineto{\pgfqpoint{0.751830in}{1.029402in}}%
\pgfpathlineto{\pgfqpoint{0.715950in}{1.048670in}}%
\pgfpathlineto{\pgfqpoint{0.677100in}{1.029402in}}%
\pgfpathclose%
\pgfusepath{fill}%
\end{pgfscope}%
\begin{pgfscope}%
\pgfpathrectangle{\pgfqpoint{0.150000in}{0.150000in}}{\pgfqpoint{2.700000in}{1.950000in}}%
\pgfusepath{clip}%
\pgfsetbuttcap%
\pgfsetroundjoin%
\definecolor{currentfill}{rgb}{0.772993,0.800950,0.840089}%
\pgfsetfillcolor{currentfill}%
\pgfsetlinewidth{0.000000pt}%
\definecolor{currentstroke}{rgb}{0.000000,0.000000,0.000000}%
\pgfsetstrokecolor{currentstroke}%
\pgfsetdash{}{0pt}%
\pgfpathmoveto{\pgfqpoint{1.009915in}{1.521020in}}%
\pgfpathlineto{\pgfqpoint{1.053560in}{1.424363in}}%
\pgfpathlineto{\pgfqpoint{1.016956in}{1.450954in}}%
\pgfpathlineto{\pgfqpoint{0.972471in}{1.556172in}}%
\pgfpathclose%
\pgfusepath{fill}%
\end{pgfscope}%
\begin{pgfscope}%
\pgfpathrectangle{\pgfqpoint{0.150000in}{0.150000in}}{\pgfqpoint{2.700000in}{1.950000in}}%
\pgfusepath{clip}%
\pgfsetbuttcap%
\pgfsetroundjoin%
\definecolor{currentfill}{rgb}{0.963909,0.934513,0.936841}%
\pgfsetfillcolor{currentfill}%
\pgfsetlinewidth{0.000000pt}%
\definecolor{currentstroke}{rgb}{0.000000,0.000000,0.000000}%
\pgfsetstrokecolor{currentstroke}%
\pgfsetdash{}{0pt}%
\pgfpathmoveto{\pgfqpoint{1.423863in}{1.155360in}}%
\pgfpathlineto{\pgfqpoint{1.461993in}{1.110714in}}%
\pgfpathlineto{\pgfqpoint{1.424863in}{1.137705in}}%
\pgfpathlineto{\pgfqpoint{1.386480in}{1.182516in}}%
\pgfpathclose%
\pgfusepath{fill}%
\end{pgfscope}%
\begin{pgfscope}%
\pgfpathrectangle{\pgfqpoint{0.150000in}{0.150000in}}{\pgfqpoint{2.700000in}{1.950000in}}%
\pgfusepath{clip}%
\pgfsetbuttcap%
\pgfsetroundjoin%
\definecolor{currentfill}{rgb}{0.884130,0.789752,0.797227}%
\pgfsetfillcolor{currentfill}%
\pgfsetlinewidth{0.000000pt}%
\definecolor{currentstroke}{rgb}{0.000000,0.000000,0.000000}%
\pgfsetstrokecolor{currentstroke}%
\pgfsetdash{}{0pt}%
\pgfpathmoveto{\pgfqpoint{1.723846in}{0.982777in}}%
\pgfpathlineto{\pgfqpoint{1.760571in}{0.986458in}}%
\pgfpathlineto{\pgfqpoint{1.723027in}{1.013652in}}%
\pgfpathlineto{\pgfqpoint{1.686103in}{1.002184in}}%
\pgfpathclose%
\pgfusepath{fill}%
\end{pgfscope}%
\begin{pgfscope}%
\pgfpathrectangle{\pgfqpoint{0.150000in}{0.150000in}}{\pgfqpoint{2.700000in}{1.950000in}}%
\pgfusepath{clip}%
\pgfsetbuttcap%
\pgfsetroundjoin%
\definecolor{currentfill}{rgb}{0.865135,0.755285,0.763986}%
\pgfsetfillcolor{currentfill}%
\pgfsetlinewidth{0.000000pt}%
\definecolor{currentstroke}{rgb}{0.000000,0.000000,0.000000}%
\pgfsetstrokecolor{currentstroke}%
\pgfsetdash{}{0pt}%
\pgfpathmoveto{\pgfqpoint{2.136671in}{0.955431in}}%
\pgfpathlineto{\pgfqpoint{2.173026in}{0.974908in}}%
\pgfpathlineto{\pgfqpoint{2.132692in}{0.962958in}}%
\pgfpathlineto{\pgfqpoint{2.096440in}{0.943553in}}%
\pgfpathclose%
\pgfusepath{fill}%
\end{pgfscope}%
\begin{pgfscope}%
\pgfpathrectangle{\pgfqpoint{0.150000in}{0.150000in}}{\pgfqpoint{2.700000in}{1.950000in}}%
\pgfusepath{clip}%
\pgfsetbuttcap%
\pgfsetroundjoin%
\definecolor{currentfill}{rgb}{0.887929,0.796645,0.803876}%
\pgfsetfillcolor{currentfill}%
\pgfsetlinewidth{0.000000pt}%
\definecolor{currentstroke}{rgb}{0.000000,0.000000,0.000000}%
\pgfsetstrokecolor{currentstroke}%
\pgfsetdash{}{0pt}%
\pgfpathmoveto{\pgfqpoint{2.248994in}{0.990657in}}%
\pgfpathlineto{\pgfqpoint{2.285134in}{1.010065in}}%
\pgfpathlineto{\pgfqpoint{2.244806in}{1.005794in}}%
\pgfpathlineto{\pgfqpoint{2.208740in}{0.986458in}}%
\pgfpathclose%
\pgfusepath{fill}%
\end{pgfscope}%
\begin{pgfscope}%
\pgfpathrectangle{\pgfqpoint{0.150000in}{0.150000in}}{\pgfqpoint{2.700000in}{1.950000in}}%
\pgfusepath{clip}%
\pgfsetbuttcap%
\pgfsetroundjoin%
\definecolor{currentfill}{rgb}{0.906924,0.831112,0.837117}%
\pgfsetfillcolor{currentfill}%
\pgfsetlinewidth{0.000000pt}%
\definecolor{currentstroke}{rgb}{0.000000,0.000000,0.000000}%
\pgfsetstrokecolor{currentstroke}%
\pgfsetdash{}{0pt}%
\pgfpathmoveto{\pgfqpoint{1.611465in}{1.025863in}}%
\pgfpathlineto{\pgfqpoint{1.648496in}{1.021521in}}%
\pgfpathlineto{\pgfqpoint{1.611081in}{1.048670in}}%
\pgfpathlineto{\pgfqpoint{1.573936in}{1.053118in}}%
\pgfpathclose%
\pgfusepath{fill}%
\end{pgfscope}%
\begin{pgfscope}%
\pgfpathrectangle{\pgfqpoint{0.150000in}{0.150000in}}{\pgfqpoint{2.700000in}{1.950000in}}%
\pgfusepath{clip}%
\pgfsetbuttcap%
\pgfsetroundjoin%
\definecolor{currentfill}{rgb}{0.849939,0.727711,0.737393}%
\pgfsetfillcolor{currentfill}%
\pgfsetlinewidth{0.000000pt}%
\definecolor{currentstroke}{rgb}{0.000000,0.000000,0.000000}%
\pgfsetstrokecolor{currentstroke}%
\pgfsetdash{}{0pt}%
\pgfpathmoveto{\pgfqpoint{1.949236in}{0.920183in}}%
\pgfpathlineto{\pgfqpoint{1.985261in}{0.924078in}}%
\pgfpathlineto{\pgfqpoint{1.946499in}{0.927946in}}%
\pgfpathlineto{\pgfqpoint{1.910465in}{0.924078in}}%
\pgfpathclose%
\pgfusepath{fill}%
\end{pgfscope}%
\begin{pgfscope}%
\pgfpathrectangle{\pgfqpoint{0.150000in}{0.150000in}}{\pgfqpoint{2.700000in}{1.950000in}}%
\pgfusepath{clip}%
\pgfsetbuttcap%
\pgfsetroundjoin%
\definecolor{currentfill}{rgb}{0.785432,0.811857,0.848851}%
\pgfsetfillcolor{currentfill}%
\pgfsetlinewidth{0.000000pt}%
\definecolor{currentstroke}{rgb}{0.000000,0.000000,0.000000}%
\pgfsetstrokecolor{currentstroke}%
\pgfsetdash{}{0pt}%
\pgfpathmoveto{\pgfqpoint{1.047009in}{1.494143in}}%
\pgfpathlineto{\pgfqpoint{1.090241in}{1.397717in}}%
\pgfpathlineto{\pgfqpoint{1.053560in}{1.424363in}}%
\pgfpathlineto{\pgfqpoint{1.009915in}{1.521020in}}%
\pgfpathclose%
\pgfusepath{fill}%
\end{pgfscope}%
\begin{pgfscope}%
\pgfpathrectangle{\pgfqpoint{0.150000in}{0.150000in}}{\pgfqpoint{2.700000in}{1.950000in}}%
\pgfusepath{clip}%
\pgfsetbuttcap%
\pgfsetroundjoin%
\definecolor{currentfill}{rgb}{0.849939,0.727711,0.737393}%
\pgfsetfillcolor{currentfill}%
\pgfsetlinewidth{0.000000pt}%
\definecolor{currentstroke}{rgb}{0.000000,0.000000,0.000000}%
\pgfsetstrokecolor{currentstroke}%
\pgfsetdash{}{0pt}%
\pgfpathmoveto{\pgfqpoint{1.874191in}{0.920183in}}%
\pgfpathlineto{\pgfqpoint{1.910465in}{0.924078in}}%
\pgfpathlineto{\pgfqpoint{1.871951in}{0.927946in}}%
\pgfpathlineto{\pgfqpoint{1.835896in}{0.931898in}}%
\pgfpathclose%
\pgfusepath{fill}%
\end{pgfscope}%
\begin{pgfscope}%
\pgfpathrectangle{\pgfqpoint{0.150000in}{0.150000in}}{\pgfqpoint{2.700000in}{1.950000in}}%
\pgfusepath{clip}%
\pgfsetbuttcap%
\pgfsetroundjoin%
\definecolor{currentfill}{rgb}{0.849939,0.727711,0.737393}%
\pgfsetfillcolor{currentfill}%
\pgfsetlinewidth{0.000000pt}%
\definecolor{currentstroke}{rgb}{0.000000,0.000000,0.000000}%
\pgfsetstrokecolor{currentstroke}%
\pgfsetdash{}{0pt}%
\pgfpathmoveto{\pgfqpoint{2.024282in}{0.920183in}}%
\pgfpathlineto{\pgfqpoint{2.060453in}{0.931898in}}%
\pgfpathlineto{\pgfqpoint{2.021046in}{0.927946in}}%
\pgfpathlineto{\pgfqpoint{1.985261in}{0.924078in}}%
\pgfpathclose%
\pgfusepath{fill}%
\end{pgfscope}%
\begin{pgfscope}%
\pgfpathrectangle{\pgfqpoint{0.150000in}{0.150000in}}{\pgfqpoint{2.700000in}{1.950000in}}%
\pgfusepath{clip}%
\pgfsetbuttcap%
\pgfsetroundjoin%
\definecolor{currentfill}{rgb}{0.956311,0.920726,0.923545}%
\pgfsetfillcolor{currentfill}%
\pgfsetlinewidth{0.000000pt}%
\definecolor{currentstroke}{rgb}{0.000000,0.000000,0.000000}%
\pgfsetstrokecolor{currentstroke}%
\pgfsetdash{}{0pt}%
\pgfpathmoveto{\pgfqpoint{1.461324in}{1.128146in}}%
\pgfpathlineto{\pgfqpoint{1.499172in}{1.091585in}}%
\pgfpathlineto{\pgfqpoint{1.461993in}{1.110714in}}%
\pgfpathlineto{\pgfqpoint{1.423863in}{1.155360in}}%
\pgfpathclose%
\pgfusepath{fill}%
\end{pgfscope}%
\begin{pgfscope}%
\pgfpathrectangle{\pgfqpoint{0.150000in}{0.150000in}}{\pgfqpoint{2.700000in}{1.950000in}}%
\pgfusepath{clip}%
\pgfsetbuttcap%
\pgfsetroundjoin%
\definecolor{currentfill}{rgb}{0.937316,0.886259,0.890303}%
\pgfsetfillcolor{currentfill}%
\pgfsetlinewidth{0.000000pt}%
\definecolor{currentstroke}{rgb}{0.000000,0.000000,0.000000}%
\pgfsetstrokecolor{currentstroke}%
\pgfsetdash{}{0pt}%
\pgfpathmoveto{\pgfqpoint{0.976017in}{1.029402in}}%
\pgfpathlineto{\pgfqpoint{1.010752in}{1.120144in}}%
\pgfpathlineto{\pgfqpoint{0.974222in}{1.139344in}}%
\pgfpathlineto{\pgfqpoint{0.939733in}{1.048670in}}%
\pgfpathclose%
\pgfusepath{fill}%
\end{pgfscope}%
\begin{pgfscope}%
\pgfpathrectangle{\pgfqpoint{0.150000in}{0.150000in}}{\pgfqpoint{2.700000in}{1.950000in}}%
\pgfusepath{clip}%
\pgfsetbuttcap%
\pgfsetroundjoin%
\definecolor{currentfill}{rgb}{0.880331,0.782858,0.790579}%
\pgfsetfillcolor{currentfill}%
\pgfsetlinewidth{0.000000pt}%
\definecolor{currentstroke}{rgb}{0.000000,0.000000,0.000000}%
\pgfsetstrokecolor{currentstroke}%
\pgfsetdash{}{0pt}%
\pgfpathmoveto{\pgfqpoint{1.761556in}{0.955431in}}%
\pgfpathlineto{\pgfqpoint{1.798194in}{0.959207in}}%
\pgfpathlineto{\pgfqpoint{1.760571in}{0.986458in}}%
\pgfpathlineto{\pgfqpoint{1.723846in}{0.982777in}}%
\pgfpathclose%
\pgfusepath{fill}%
\end{pgfscope}%
\begin{pgfscope}%
\pgfpathrectangle{\pgfqpoint{0.150000in}{0.150000in}}{\pgfqpoint{2.700000in}{1.950000in}}%
\pgfusepath{clip}%
\pgfsetbuttcap%
\pgfsetroundjoin%
\definecolor{currentfill}{rgb}{0.903600,0.915472,0.932093}%
\pgfsetfillcolor{currentfill}%
\pgfsetlinewidth{0.000000pt}%
\definecolor{currentstroke}{rgb}{0.000000,0.000000,0.000000}%
\pgfsetstrokecolor{currentstroke}%
\pgfsetdash{}{0pt}%
\pgfpathmoveto{\pgfqpoint{0.828354in}{1.231465in}}%
\pgfpathlineto{\pgfqpoint{0.859175in}{1.371990in}}%
\pgfpathlineto{\pgfqpoint{0.822304in}{1.398980in}}%
\pgfpathlineto{\pgfqpoint{0.791866in}{1.258345in}}%
\pgfpathclose%
\pgfusepath{fill}%
\end{pgfscope}%
\begin{pgfscope}%
\pgfpathrectangle{\pgfqpoint{0.150000in}{0.150000in}}{\pgfqpoint{2.700000in}{1.950000in}}%
\pgfusepath{clip}%
\pgfsetbuttcap%
\pgfsetroundjoin%
\definecolor{currentfill}{rgb}{0.804090,0.828217,0.861994}%
\pgfsetfillcolor{currentfill}%
\pgfsetlinewidth{0.000000pt}%
\definecolor{currentstroke}{rgb}{0.000000,0.000000,0.000000}%
\pgfsetstrokecolor{currentstroke}%
\pgfsetdash{}{0pt}%
\pgfpathmoveto{\pgfqpoint{1.084526in}{1.458924in}}%
\pgfpathlineto{\pgfqpoint{1.126999in}{1.371014in}}%
\pgfpathlineto{\pgfqpoint{1.090241in}{1.397717in}}%
\pgfpathlineto{\pgfqpoint{1.047009in}{1.494143in}}%
\pgfpathclose%
\pgfusepath{fill}%
\end{pgfscope}%
\begin{pgfscope}%
\pgfpathrectangle{\pgfqpoint{0.150000in}{0.150000in}}{\pgfqpoint{2.700000in}{1.950000in}}%
\pgfusepath{clip}%
\pgfsetbuttcap%
\pgfsetroundjoin%
\definecolor{currentfill}{rgb}{0.899326,0.817325,0.823820}%
\pgfsetfillcolor{currentfill}%
\pgfsetlinewidth{0.000000pt}%
\definecolor{currentstroke}{rgb}{0.000000,0.000000,0.000000}%
\pgfsetstrokecolor{currentstroke}%
\pgfsetdash{}{0pt}%
\pgfpathmoveto{\pgfqpoint{2.323995in}{0.990657in}}%
\pgfpathlineto{\pgfqpoint{2.359999in}{1.010065in}}%
\pgfpathlineto{\pgfqpoint{2.321143in}{1.029402in}}%
\pgfpathlineto{\pgfqpoint{2.285134in}{1.010065in}}%
\pgfpathclose%
\pgfusepath{fill}%
\end{pgfscope}%
\begin{pgfscope}%
\pgfpathrectangle{\pgfqpoint{0.150000in}{0.150000in}}{\pgfqpoint{2.700000in}{1.950000in}}%
\pgfusepath{clip}%
\pgfsetbuttcap%
\pgfsetroundjoin%
\definecolor{currentfill}{rgb}{0.899326,0.817325,0.823820}%
\pgfsetfillcolor{currentfill}%
\pgfsetlinewidth{0.000000pt}%
\definecolor{currentstroke}{rgb}{0.000000,0.000000,0.000000}%
\pgfsetstrokecolor{currentstroke}%
\pgfsetdash{}{0pt}%
\pgfpathmoveto{\pgfqpoint{0.973981in}{0.990657in}}%
\pgfpathlineto{\pgfqpoint{1.012433in}{1.010065in}}%
\pgfpathlineto{\pgfqpoint{0.976017in}{1.029402in}}%
\pgfpathlineto{\pgfqpoint{0.937568in}{1.010065in}}%
\pgfpathclose%
\pgfusepath{fill}%
\end{pgfscope}%
\begin{pgfscope}%
\pgfpathrectangle{\pgfqpoint{0.150000in}{0.150000in}}{\pgfqpoint{2.700000in}{1.950000in}}%
\pgfusepath{clip}%
\pgfsetbuttcap%
\pgfsetroundjoin%
\definecolor{currentfill}{rgb}{0.899326,0.817325,0.823820}%
\pgfsetfillcolor{currentfill}%
\pgfsetlinewidth{0.000000pt}%
\definecolor{currentstroke}{rgb}{0.000000,0.000000,0.000000}%
\pgfsetstrokecolor{currentstroke}%
\pgfsetdash{}{0pt}%
\pgfpathmoveto{\pgfqpoint{0.898980in}{0.990657in}}%
\pgfpathlineto{\pgfqpoint{0.937568in}{1.010065in}}%
\pgfpathlineto{\pgfqpoint{0.901288in}{1.029402in}}%
\pgfpathlineto{\pgfqpoint{0.862704in}{1.010065in}}%
\pgfpathclose%
\pgfusepath{fill}%
\end{pgfscope}%
\begin{pgfscope}%
\pgfpathrectangle{\pgfqpoint{0.150000in}{0.150000in}}{\pgfqpoint{2.700000in}{1.950000in}}%
\pgfusepath{clip}%
\pgfsetbuttcap%
\pgfsetroundjoin%
\definecolor{currentfill}{rgb}{0.899326,0.817325,0.823820}%
\pgfsetfillcolor{currentfill}%
\pgfsetlinewidth{0.000000pt}%
\definecolor{currentstroke}{rgb}{0.000000,0.000000,0.000000}%
\pgfsetstrokecolor{currentstroke}%
\pgfsetdash{}{0pt}%
\pgfpathmoveto{\pgfqpoint{0.823979in}{0.990657in}}%
\pgfpathlineto{\pgfqpoint{0.862704in}{1.010065in}}%
\pgfpathlineto{\pgfqpoint{0.826559in}{1.029402in}}%
\pgfpathlineto{\pgfqpoint{0.787839in}{1.010065in}}%
\pgfpathclose%
\pgfusepath{fill}%
\end{pgfscope}%
\begin{pgfscope}%
\pgfpathrectangle{\pgfqpoint{0.150000in}{0.150000in}}{\pgfqpoint{2.700000in}{1.950000in}}%
\pgfusepath{clip}%
\pgfsetbuttcap%
\pgfsetroundjoin%
\definecolor{currentfill}{rgb}{0.899326,0.817325,0.823820}%
\pgfsetfillcolor{currentfill}%
\pgfsetlinewidth{0.000000pt}%
\definecolor{currentstroke}{rgb}{0.000000,0.000000,0.000000}%
\pgfsetstrokecolor{currentstroke}%
\pgfsetdash{}{0pt}%
\pgfpathmoveto{\pgfqpoint{0.748978in}{0.990657in}}%
\pgfpathlineto{\pgfqpoint{0.787839in}{1.010065in}}%
\pgfpathlineto{\pgfqpoint{0.751830in}{1.029402in}}%
\pgfpathlineto{\pgfqpoint{0.712974in}{1.010065in}}%
\pgfpathclose%
\pgfusepath{fill}%
\end{pgfscope}%
\begin{pgfscope}%
\pgfpathrectangle{\pgfqpoint{0.150000in}{0.150000in}}{\pgfqpoint{2.700000in}{1.950000in}}%
\pgfusepath{clip}%
\pgfsetbuttcap%
\pgfsetroundjoin%
\definecolor{currentfill}{rgb}{0.903125,0.824219,0.830469}%
\pgfsetfillcolor{currentfill}%
\pgfsetlinewidth{0.000000pt}%
\definecolor{currentstroke}{rgb}{0.000000,0.000000,0.000000}%
\pgfsetstrokecolor{currentstroke}%
\pgfsetdash{}{0pt}%
\pgfpathmoveto{\pgfqpoint{1.649159in}{1.006454in}}%
\pgfpathlineto{\pgfqpoint{1.686103in}{1.002184in}}%
\pgfpathlineto{\pgfqpoint{1.648496in}{1.021521in}}%
\pgfpathlineto{\pgfqpoint{1.611465in}{1.025863in}}%
\pgfpathclose%
\pgfusepath{fill}%
\end{pgfscope}%
\begin{pgfscope}%
\pgfpathrectangle{\pgfqpoint{0.150000in}{0.150000in}}{\pgfqpoint{2.700000in}{1.950000in}}%
\pgfusepath{clip}%
\pgfsetbuttcap%
\pgfsetroundjoin%
\definecolor{currentfill}{rgb}{0.868934,0.762178,0.770634}%
\pgfsetfillcolor{currentfill}%
\pgfsetlinewidth{0.000000pt}%
\definecolor{currentstroke}{rgb}{0.000000,0.000000,0.000000}%
\pgfsetstrokecolor{currentstroke}%
\pgfsetdash{}{0pt}%
\pgfpathmoveto{\pgfqpoint{2.100184in}{0.935882in}}%
\pgfpathlineto{\pgfqpoint{2.136671in}{0.955431in}}%
\pgfpathlineto{\pgfqpoint{2.096440in}{0.943553in}}%
\pgfpathlineto{\pgfqpoint{2.060453in}{0.931898in}}%
\pgfpathclose%
\pgfusepath{fill}%
\end{pgfscope}%
\begin{pgfscope}%
\pgfpathrectangle{\pgfqpoint{0.150000in}{0.150000in}}{\pgfqpoint{2.700000in}{1.950000in}}%
\pgfusepath{clip}%
\pgfsetbuttcap%
\pgfsetroundjoin%
\definecolor{currentfill}{rgb}{0.822748,0.844577,0.875138}%
\pgfsetfillcolor{currentfill}%
\pgfsetlinewidth{0.000000pt}%
\definecolor{currentstroke}{rgb}{0.000000,0.000000,0.000000}%
\pgfsetstrokecolor{currentstroke}%
\pgfsetdash{}{0pt}%
\pgfpathmoveto{\pgfqpoint{1.121749in}{1.431944in}}%
\pgfpathlineto{\pgfqpoint{1.163834in}{1.344256in}}%
\pgfpathlineto{\pgfqpoint{1.126999in}{1.371014in}}%
\pgfpathlineto{\pgfqpoint{1.084526in}{1.458924in}}%
\pgfpathclose%
\pgfusepath{fill}%
\end{pgfscope}%
\begin{pgfscope}%
\pgfpathrectangle{\pgfqpoint{0.150000in}{0.150000in}}{\pgfqpoint{2.700000in}{1.950000in}}%
\pgfusepath{clip}%
\pgfsetbuttcap%
\pgfsetroundjoin%
\definecolor{currentfill}{rgb}{0.887929,0.796645,0.803876}%
\pgfsetfillcolor{currentfill}%
\pgfsetlinewidth{0.000000pt}%
\definecolor{currentstroke}{rgb}{0.000000,0.000000,0.000000}%
\pgfsetstrokecolor{currentstroke}%
\pgfsetdash{}{0pt}%
\pgfpathmoveto{\pgfqpoint{2.212722in}{0.971179in}}%
\pgfpathlineto{\pgfqpoint{2.248994in}{0.990657in}}%
\pgfpathlineto{\pgfqpoint{2.208740in}{0.986458in}}%
\pgfpathlineto{\pgfqpoint{2.173026in}{0.974908in}}%
\pgfpathclose%
\pgfusepath{fill}%
\end{pgfscope}%
\begin{pgfscope}%
\pgfpathrectangle{\pgfqpoint{0.150000in}{0.150000in}}{\pgfqpoint{2.700000in}{1.950000in}}%
\pgfusepath{clip}%
\pgfsetbuttcap%
\pgfsetroundjoin%
\definecolor{currentfill}{rgb}{0.872733,0.769072,0.777282}%
\pgfsetfillcolor{currentfill}%
\pgfsetlinewidth{0.000000pt}%
\definecolor{currentstroke}{rgb}{0.000000,0.000000,0.000000}%
\pgfsetstrokecolor{currentstroke}%
\pgfsetdash{}{0pt}%
\pgfpathmoveto{\pgfqpoint{1.799545in}{0.935882in}}%
\pgfpathlineto{\pgfqpoint{1.835896in}{0.931898in}}%
\pgfpathlineto{\pgfqpoint{1.798194in}{0.959207in}}%
\pgfpathlineto{\pgfqpoint{1.761556in}{0.955431in}}%
\pgfpathclose%
\pgfusepath{fill}%
\end{pgfscope}%
\begin{pgfscope}%
\pgfpathrectangle{\pgfqpoint{0.150000in}{0.150000in}}{\pgfqpoint{2.700000in}{1.950000in}}%
\pgfusepath{clip}%
\pgfsetbuttcap%
\pgfsetroundjoin%
\definecolor{currentfill}{rgb}{0.916039,0.926379,0.940855}%
\pgfsetfillcolor{currentfill}%
\pgfsetlinewidth{0.000000pt}%
\definecolor{currentstroke}{rgb}{0.000000,0.000000,0.000000}%
\pgfsetstrokecolor{currentstroke}%
\pgfsetdash{}{0pt}%
\pgfpathmoveto{\pgfqpoint{0.864410in}{1.212542in}}%
\pgfpathlineto{\pgfqpoint{0.896124in}{1.344943in}}%
\pgfpathlineto{\pgfqpoint{0.859175in}{1.371990in}}%
\pgfpathlineto{\pgfqpoint{0.828354in}{1.231465in}}%
\pgfpathclose%
\pgfusepath{fill}%
\end{pgfscope}%
\begin{pgfscope}%
\pgfpathrectangle{\pgfqpoint{0.150000in}{0.150000in}}{\pgfqpoint{2.700000in}{1.950000in}}%
\pgfusepath{clip}%
\pgfsetbuttcap%
\pgfsetroundjoin%
\definecolor{currentfill}{rgb}{0.952512,0.913833,0.916896}%
\pgfsetfillcolor{currentfill}%
\pgfsetlinewidth{0.000000pt}%
\definecolor{currentstroke}{rgb}{0.000000,0.000000,0.000000}%
\pgfsetstrokecolor{currentstroke}%
\pgfsetdash{}{0pt}%
\pgfpathmoveto{\pgfqpoint{1.498837in}{1.108877in}}%
\pgfpathlineto{\pgfqpoint{1.536486in}{1.072386in}}%
\pgfpathlineto{\pgfqpoint{1.499172in}{1.091585in}}%
\pgfpathlineto{\pgfqpoint{1.461324in}{1.128146in}}%
\pgfpathclose%
\pgfusepath{fill}%
\end{pgfscope}%
\begin{pgfscope}%
\pgfpathrectangle{\pgfqpoint{0.150000in}{0.150000in}}{\pgfqpoint{2.700000in}{1.950000in}}%
\pgfusepath{clip}%
\pgfsetbuttcap%
\pgfsetroundjoin%
\definecolor{currentfill}{rgb}{0.933517,0.879366,0.883655}%
\pgfsetfillcolor{currentfill}%
\pgfsetlinewidth{0.000000pt}%
\definecolor{currentstroke}{rgb}{0.000000,0.000000,0.000000}%
\pgfsetstrokecolor{currentstroke}%
\pgfsetdash{}{0pt}%
\pgfpathmoveto{\pgfqpoint{1.012433in}{1.010065in}}%
\pgfpathlineto{\pgfqpoint{1.047787in}{1.092885in}}%
\pgfpathlineto{\pgfqpoint{1.010752in}{1.120144in}}%
\pgfpathlineto{\pgfqpoint{0.976017in}{1.029402in}}%
\pgfpathclose%
\pgfusepath{fill}%
\end{pgfscope}%
\begin{pgfscope}%
\pgfpathrectangle{\pgfqpoint{0.150000in}{0.150000in}}{\pgfqpoint{2.700000in}{1.950000in}}%
\pgfusepath{clip}%
\pgfsetbuttcap%
\pgfsetroundjoin%
\definecolor{currentfill}{rgb}{0.841406,0.860938,0.888281}%
\pgfsetfillcolor{currentfill}%
\pgfsetlinewidth{0.000000pt}%
\definecolor{currentstroke}{rgb}{0.000000,0.000000,0.000000}%
\pgfsetstrokecolor{currentstroke}%
\pgfsetdash{}{0pt}%
\pgfpathmoveto{\pgfqpoint{1.159340in}{1.396659in}}%
\pgfpathlineto{\pgfqpoint{1.200746in}{1.317441in}}%
\pgfpathlineto{\pgfqpoint{1.163834in}{1.344256in}}%
\pgfpathlineto{\pgfqpoint{1.121749in}{1.431944in}}%
\pgfpathclose%
\pgfusepath{fill}%
\end{pgfscope}%
\begin{pgfscope}%
\pgfpathrectangle{\pgfqpoint{0.150000in}{0.150000in}}{\pgfqpoint{2.700000in}{1.950000in}}%
\pgfusepath{clip}%
\pgfsetbuttcap%
\pgfsetroundjoin%
\definecolor{currentfill}{rgb}{0.899326,0.817325,0.823820}%
\pgfsetfillcolor{currentfill}%
\pgfsetlinewidth{0.000000pt}%
\definecolor{currentstroke}{rgb}{0.000000,0.000000,0.000000}%
\pgfsetstrokecolor{currentstroke}%
\pgfsetdash{}{0pt}%
\pgfpathmoveto{\pgfqpoint{2.287859in}{0.971179in}}%
\pgfpathlineto{\pgfqpoint{2.323995in}{0.990657in}}%
\pgfpathlineto{\pgfqpoint{2.285134in}{1.010065in}}%
\pgfpathlineto{\pgfqpoint{2.248994in}{0.990657in}}%
\pgfpathclose%
\pgfusepath{fill}%
\end{pgfscope}%
\begin{pgfscope}%
\pgfpathrectangle{\pgfqpoint{0.150000in}{0.150000in}}{\pgfqpoint{2.700000in}{1.950000in}}%
\pgfusepath{clip}%
\pgfsetbuttcap%
\pgfsetroundjoin%
\definecolor{currentfill}{rgb}{0.899326,0.817325,0.823820}%
\pgfsetfillcolor{currentfill}%
\pgfsetlinewidth{0.000000pt}%
\definecolor{currentstroke}{rgb}{0.000000,0.000000,0.000000}%
\pgfsetstrokecolor{currentstroke}%
\pgfsetdash{}{0pt}%
\pgfpathmoveto{\pgfqpoint{1.010525in}{0.971179in}}%
\pgfpathlineto{\pgfqpoint{1.048981in}{0.990657in}}%
\pgfpathlineto{\pgfqpoint{1.012433in}{1.010065in}}%
\pgfpathlineto{\pgfqpoint{0.973981in}{0.990657in}}%
\pgfpathclose%
\pgfusepath{fill}%
\end{pgfscope}%
\begin{pgfscope}%
\pgfpathrectangle{\pgfqpoint{0.150000in}{0.150000in}}{\pgfqpoint{2.700000in}{1.950000in}}%
\pgfusepath{clip}%
\pgfsetbuttcap%
\pgfsetroundjoin%
\definecolor{currentfill}{rgb}{0.899326,0.817325,0.823820}%
\pgfsetfillcolor{currentfill}%
\pgfsetlinewidth{0.000000pt}%
\definecolor{currentstroke}{rgb}{0.000000,0.000000,0.000000}%
\pgfsetstrokecolor{currentstroke}%
\pgfsetdash{}{0pt}%
\pgfpathmoveto{\pgfqpoint{0.935388in}{0.971179in}}%
\pgfpathlineto{\pgfqpoint{0.973981in}{0.990657in}}%
\pgfpathlineto{\pgfqpoint{0.937568in}{1.010065in}}%
\pgfpathlineto{\pgfqpoint{0.898980in}{0.990657in}}%
\pgfpathclose%
\pgfusepath{fill}%
\end{pgfscope}%
\begin{pgfscope}%
\pgfpathrectangle{\pgfqpoint{0.150000in}{0.150000in}}{\pgfqpoint{2.700000in}{1.950000in}}%
\pgfusepath{clip}%
\pgfsetbuttcap%
\pgfsetroundjoin%
\definecolor{currentfill}{rgb}{0.899326,0.817325,0.823820}%
\pgfsetfillcolor{currentfill}%
\pgfsetlinewidth{0.000000pt}%
\definecolor{currentstroke}{rgb}{0.000000,0.000000,0.000000}%
\pgfsetstrokecolor{currentstroke}%
\pgfsetdash{}{0pt}%
\pgfpathmoveto{\pgfqpoint{0.860251in}{0.971179in}}%
\pgfpathlineto{\pgfqpoint{0.898980in}{0.990657in}}%
\pgfpathlineto{\pgfqpoint{0.862704in}{1.010065in}}%
\pgfpathlineto{\pgfqpoint{0.823979in}{0.990657in}}%
\pgfpathclose%
\pgfusepath{fill}%
\end{pgfscope}%
\begin{pgfscope}%
\pgfpathrectangle{\pgfqpoint{0.150000in}{0.150000in}}{\pgfqpoint{2.700000in}{1.950000in}}%
\pgfusepath{clip}%
\pgfsetbuttcap%
\pgfsetroundjoin%
\definecolor{currentfill}{rgb}{0.899326,0.817325,0.823820}%
\pgfsetfillcolor{currentfill}%
\pgfsetlinewidth{0.000000pt}%
\definecolor{currentstroke}{rgb}{0.000000,0.000000,0.000000}%
\pgfsetstrokecolor{currentstroke}%
\pgfsetdash{}{0pt}%
\pgfpathmoveto{\pgfqpoint{0.785114in}{0.971179in}}%
\pgfpathlineto{\pgfqpoint{0.823979in}{0.990657in}}%
\pgfpathlineto{\pgfqpoint{0.787839in}{1.010065in}}%
\pgfpathlineto{\pgfqpoint{0.748978in}{0.990657in}}%
\pgfpathclose%
\pgfusepath{fill}%
\end{pgfscope}%
\begin{pgfscope}%
\pgfpathrectangle{\pgfqpoint{0.150000in}{0.150000in}}{\pgfqpoint{2.700000in}{1.950000in}}%
\pgfusepath{clip}%
\pgfsetbuttcap%
\pgfsetroundjoin%
\definecolor{currentfill}{rgb}{0.903125,0.824219,0.830469}%
\pgfsetfillcolor{currentfill}%
\pgfsetlinewidth{0.000000pt}%
\definecolor{currentstroke}{rgb}{0.000000,0.000000,0.000000}%
\pgfsetstrokecolor{currentstroke}%
\pgfsetdash{}{0pt}%
\pgfpathmoveto{\pgfqpoint{1.686990in}{0.986975in}}%
\pgfpathlineto{\pgfqpoint{1.723846in}{0.982777in}}%
\pgfpathlineto{\pgfqpoint{1.686103in}{1.002184in}}%
\pgfpathlineto{\pgfqpoint{1.649159in}{1.006454in}}%
\pgfpathclose%
\pgfusepath{fill}%
\end{pgfscope}%
\begin{pgfscope}%
\pgfpathrectangle{\pgfqpoint{0.150000in}{0.150000in}}{\pgfqpoint{2.700000in}{1.950000in}}%
\pgfusepath{clip}%
\pgfsetbuttcap%
\pgfsetroundjoin%
\definecolor{currentfill}{rgb}{0.928477,0.937286,0.949617}%
\pgfsetfillcolor{currentfill}%
\pgfsetlinewidth{0.000000pt}%
\definecolor{currentstroke}{rgb}{0.000000,0.000000,0.000000}%
\pgfsetstrokecolor{currentstroke}%
\pgfsetdash{}{0pt}%
\pgfpathmoveto{\pgfqpoint{0.901079in}{1.185536in}}%
\pgfpathlineto{\pgfqpoint{0.933612in}{1.309665in}}%
\pgfpathlineto{\pgfqpoint{0.896124in}{1.344943in}}%
\pgfpathlineto{\pgfqpoint{0.864410in}{1.212542in}}%
\pgfpathclose%
\pgfusepath{fill}%
\end{pgfscope}%
\begin{pgfscope}%
\pgfpathrectangle{\pgfqpoint{0.150000in}{0.150000in}}{\pgfqpoint{2.700000in}{1.950000in}}%
\pgfusepath{clip}%
\pgfsetbuttcap%
\pgfsetroundjoin%
\definecolor{currentfill}{rgb}{0.865135,0.755285,0.763986}%
\pgfsetfillcolor{currentfill}%
\pgfsetlinewidth{0.000000pt}%
\definecolor{currentstroke}{rgb}{0.000000,0.000000,0.000000}%
\pgfsetstrokecolor{currentstroke}%
\pgfsetdash{}{0pt}%
\pgfpathmoveto{\pgfqpoint{1.912683in}{0.908408in}}%
\pgfpathlineto{\pgfqpoint{1.949236in}{0.920183in}}%
\pgfpathlineto{\pgfqpoint{1.910465in}{0.924078in}}%
\pgfpathlineto{\pgfqpoint{1.874191in}{0.920183in}}%
\pgfpathclose%
\pgfusepath{fill}%
\end{pgfscope}%
\begin{pgfscope}%
\pgfpathrectangle{\pgfqpoint{0.150000in}{0.150000in}}{\pgfqpoint{2.700000in}{1.950000in}}%
\pgfusepath{clip}%
\pgfsetbuttcap%
\pgfsetroundjoin%
\definecolor{currentfill}{rgb}{0.865135,0.755285,0.763986}%
\pgfsetfillcolor{currentfill}%
\pgfsetlinewidth{0.000000pt}%
\definecolor{currentstroke}{rgb}{0.000000,0.000000,0.000000}%
\pgfsetstrokecolor{currentstroke}%
\pgfsetdash{}{0pt}%
\pgfpathmoveto{\pgfqpoint{1.987923in}{0.908408in}}%
\pgfpathlineto{\pgfqpoint{2.024282in}{0.920183in}}%
\pgfpathlineto{\pgfqpoint{1.985261in}{0.924078in}}%
\pgfpathlineto{\pgfqpoint{1.949236in}{0.920183in}}%
\pgfpathclose%
\pgfusepath{fill}%
\end{pgfscope}%
\begin{pgfscope}%
\pgfpathrectangle{\pgfqpoint{0.150000in}{0.150000in}}{\pgfqpoint{2.700000in}{1.950000in}}%
\pgfusepath{clip}%
\pgfsetbuttcap%
\pgfsetroundjoin%
\definecolor{currentfill}{rgb}{0.860064,0.877298,0.901425}%
\pgfsetfillcolor{currentfill}%
\pgfsetlinewidth{0.000000pt}%
\definecolor{currentstroke}{rgb}{0.000000,0.000000,0.000000}%
\pgfsetstrokecolor{currentstroke}%
\pgfsetdash{}{0pt}%
\pgfpathmoveto{\pgfqpoint{1.196953in}{1.361353in}}%
\pgfpathlineto{\pgfqpoint{1.237737in}{1.290570in}}%
\pgfpathlineto{\pgfqpoint{1.200746in}{1.317441in}}%
\pgfpathlineto{\pgfqpoint{1.159340in}{1.396659in}}%
\pgfpathclose%
\pgfusepath{fill}%
\end{pgfscope}%
\begin{pgfscope}%
\pgfpathrectangle{\pgfqpoint{0.150000in}{0.150000in}}{\pgfqpoint{2.700000in}{1.950000in}}%
\pgfusepath{clip}%
\pgfsetbuttcap%
\pgfsetroundjoin%
\definecolor{currentfill}{rgb}{0.891728,0.803539,0.810524}%
\pgfsetfillcolor{currentfill}%
\pgfsetlinewidth{0.000000pt}%
\definecolor{currentstroke}{rgb}{0.000000,0.000000,0.000000}%
\pgfsetstrokecolor{currentstroke}%
\pgfsetdash{}{0pt}%
\pgfpathmoveto{\pgfqpoint{2.176318in}{0.951629in}}%
\pgfpathlineto{\pgfqpoint{2.212722in}{0.971179in}}%
\pgfpathlineto{\pgfqpoint{2.173026in}{0.974908in}}%
\pgfpathlineto{\pgfqpoint{2.136671in}{0.955431in}}%
\pgfpathclose%
\pgfusepath{fill}%
\end{pgfscope}%
\begin{pgfscope}%
\pgfpathrectangle{\pgfqpoint{0.150000in}{0.150000in}}{\pgfqpoint{2.700000in}{1.950000in}}%
\pgfusepath{clip}%
\pgfsetbuttcap%
\pgfsetroundjoin%
\definecolor{currentfill}{rgb}{0.952512,0.913833,0.916896}%
\pgfsetfillcolor{currentfill}%
\pgfsetlinewidth{0.000000pt}%
\definecolor{currentstroke}{rgb}{0.000000,0.000000,0.000000}%
\pgfsetstrokecolor{currentstroke}%
\pgfsetdash{}{0pt}%
\pgfpathmoveto{\pgfqpoint{1.536486in}{1.081535in}}%
\pgfpathlineto{\pgfqpoint{1.573936in}{1.053118in}}%
\pgfpathlineto{\pgfqpoint{1.536486in}{1.072386in}}%
\pgfpathlineto{\pgfqpoint{1.498837in}{1.108877in}}%
\pgfpathclose%
\pgfusepath{fill}%
\end{pgfscope}%
\begin{pgfscope}%
\pgfpathrectangle{\pgfqpoint{0.150000in}{0.150000in}}{\pgfqpoint{2.700000in}{1.950000in}}%
\pgfusepath{clip}%
\pgfsetbuttcap%
\pgfsetroundjoin%
\definecolor{currentfill}{rgb}{0.872733,0.769072,0.777282}%
\pgfsetfillcolor{currentfill}%
\pgfsetlinewidth{0.000000pt}%
\definecolor{currentstroke}{rgb}{0.000000,0.000000,0.000000}%
\pgfsetstrokecolor{currentstroke}%
\pgfsetdash{}{0pt}%
\pgfpathmoveto{\pgfqpoint{1.837674in}{0.916263in}}%
\pgfpathlineto{\pgfqpoint{1.874191in}{0.920183in}}%
\pgfpathlineto{\pgfqpoint{1.835896in}{0.931898in}}%
\pgfpathlineto{\pgfqpoint{1.799545in}{0.935882in}}%
\pgfpathclose%
\pgfusepath{fill}%
\end{pgfscope}%
\begin{pgfscope}%
\pgfpathrectangle{\pgfqpoint{0.150000in}{0.150000in}}{\pgfqpoint{2.700000in}{1.950000in}}%
\pgfusepath{clip}%
\pgfsetbuttcap%
\pgfsetroundjoin%
\definecolor{currentfill}{rgb}{0.933517,0.879366,0.883655}%
\pgfsetfillcolor{currentfill}%
\pgfsetlinewidth{0.000000pt}%
\definecolor{currentstroke}{rgb}{0.000000,0.000000,0.000000}%
\pgfsetstrokecolor{currentstroke}%
\pgfsetdash{}{0pt}%
\pgfpathmoveto{\pgfqpoint{1.048981in}{0.990657in}}%
\pgfpathlineto{\pgfqpoint{1.084556in}{1.073545in}}%
\pgfpathlineto{\pgfqpoint{1.047787in}{1.092885in}}%
\pgfpathlineto{\pgfqpoint{1.012433in}{1.010065in}}%
\pgfpathclose%
\pgfusepath{fill}%
\end{pgfscope}%
\begin{pgfscope}%
\pgfpathrectangle{\pgfqpoint{0.150000in}{0.150000in}}{\pgfqpoint{2.700000in}{1.950000in}}%
\pgfusepath{clip}%
\pgfsetbuttcap%
\pgfsetroundjoin%
\definecolor{currentfill}{rgb}{0.872733,0.769072,0.777282}%
\pgfsetfillcolor{currentfill}%
\pgfsetlinewidth{0.000000pt}%
\definecolor{currentstroke}{rgb}{0.000000,0.000000,0.000000}%
\pgfsetstrokecolor{currentstroke}%
\pgfsetdash{}{0pt}%
\pgfpathmoveto{\pgfqpoint{2.063966in}{0.924130in}}%
\pgfpathlineto{\pgfqpoint{2.100184in}{0.935882in}}%
\pgfpathlineto{\pgfqpoint{2.060453in}{0.931898in}}%
\pgfpathlineto{\pgfqpoint{2.024282in}{0.920183in}}%
\pgfpathclose%
\pgfusepath{fill}%
\end{pgfscope}%
\begin{pgfscope}%
\pgfpathrectangle{\pgfqpoint{0.150000in}{0.150000in}}{\pgfqpoint{2.700000in}{1.950000in}}%
\pgfusepath{clip}%
\pgfsetbuttcap%
\pgfsetroundjoin%
\definecolor{currentfill}{rgb}{0.878722,0.893658,0.914568}%
\pgfsetfillcolor{currentfill}%
\pgfsetlinewidth{0.000000pt}%
\definecolor{currentstroke}{rgb}{0.000000,0.000000,0.000000}%
\pgfsetstrokecolor{currentstroke}%
\pgfsetdash{}{0pt}%
\pgfpathmoveto{\pgfqpoint{1.234357in}{1.334224in}}%
\pgfpathlineto{\pgfqpoint{1.274805in}{1.263642in}}%
\pgfpathlineto{\pgfqpoint{1.237737in}{1.290570in}}%
\pgfpathlineto{\pgfqpoint{1.196953in}{1.361353in}}%
\pgfpathclose%
\pgfusepath{fill}%
\end{pgfscope}%
\begin{pgfscope}%
\pgfpathrectangle{\pgfqpoint{0.150000in}{0.150000in}}{\pgfqpoint{2.700000in}{1.950000in}}%
\pgfusepath{clip}%
\pgfsetbuttcap%
\pgfsetroundjoin%
\definecolor{currentfill}{rgb}{0.940916,0.948192,0.958379}%
\pgfsetfillcolor{currentfill}%
\pgfsetlinewidth{0.000000pt}%
\definecolor{currentstroke}{rgb}{0.000000,0.000000,0.000000}%
\pgfsetstrokecolor{currentstroke}%
\pgfsetdash{}{0pt}%
\pgfpathmoveto{\pgfqpoint{0.937371in}{1.166476in}}%
\pgfpathlineto{\pgfqpoint{0.970691in}{1.282515in}}%
\pgfpathlineto{\pgfqpoint{0.933612in}{1.309665in}}%
\pgfpathlineto{\pgfqpoint{0.901079in}{1.185536in}}%
\pgfpathclose%
\pgfusepath{fill}%
\end{pgfscope}%
\begin{pgfscope}%
\pgfpathrectangle{\pgfqpoint{0.150000in}{0.150000in}}{\pgfqpoint{2.700000in}{1.950000in}}%
\pgfusepath{clip}%
\pgfsetbuttcap%
\pgfsetroundjoin%
\definecolor{currentfill}{rgb}{0.899326,0.817325,0.823820}%
\pgfsetfillcolor{currentfill}%
\pgfsetlinewidth{0.000000pt}%
\definecolor{currentstroke}{rgb}{0.000000,0.000000,0.000000}%
\pgfsetstrokecolor{currentstroke}%
\pgfsetdash{}{0pt}%
\pgfpathmoveto{\pgfqpoint{2.251592in}{0.951629in}}%
\pgfpathlineto{\pgfqpoint{2.287859in}{0.971179in}}%
\pgfpathlineto{\pgfqpoint{2.248994in}{0.990657in}}%
\pgfpathlineto{\pgfqpoint{2.212722in}{0.971179in}}%
\pgfpathclose%
\pgfusepath{fill}%
\end{pgfscope}%
\begin{pgfscope}%
\pgfpathrectangle{\pgfqpoint{0.150000in}{0.150000in}}{\pgfqpoint{2.700000in}{1.950000in}}%
\pgfusepath{clip}%
\pgfsetbuttcap%
\pgfsetroundjoin%
\definecolor{currentfill}{rgb}{0.899326,0.817325,0.823820}%
\pgfsetfillcolor{currentfill}%
\pgfsetlinewidth{0.000000pt}%
\definecolor{currentstroke}{rgb}{0.000000,0.000000,0.000000}%
\pgfsetstrokecolor{currentstroke}%
\pgfsetdash{}{0pt}%
\pgfpathmoveto{\pgfqpoint{1.047203in}{0.951629in}}%
\pgfpathlineto{\pgfqpoint{1.085663in}{0.971179in}}%
\pgfpathlineto{\pgfqpoint{1.048981in}{0.990657in}}%
\pgfpathlineto{\pgfqpoint{1.010525in}{0.971179in}}%
\pgfpathclose%
\pgfusepath{fill}%
\end{pgfscope}%
\begin{pgfscope}%
\pgfpathrectangle{\pgfqpoint{0.150000in}{0.150000in}}{\pgfqpoint{2.700000in}{1.950000in}}%
\pgfusepath{clip}%
\pgfsetbuttcap%
\pgfsetroundjoin%
\definecolor{currentfill}{rgb}{0.899326,0.817325,0.823820}%
\pgfsetfillcolor{currentfill}%
\pgfsetlinewidth{0.000000pt}%
\definecolor{currentstroke}{rgb}{0.000000,0.000000,0.000000}%
\pgfsetstrokecolor{currentstroke}%
\pgfsetdash{}{0pt}%
\pgfpathmoveto{\pgfqpoint{0.971929in}{0.951629in}}%
\pgfpathlineto{\pgfqpoint{1.010525in}{0.971179in}}%
\pgfpathlineto{\pgfqpoint{0.973981in}{0.990657in}}%
\pgfpathlineto{\pgfqpoint{0.935388in}{0.971179in}}%
\pgfpathclose%
\pgfusepath{fill}%
\end{pgfscope}%
\begin{pgfscope}%
\pgfpathrectangle{\pgfqpoint{0.150000in}{0.150000in}}{\pgfqpoint{2.700000in}{1.950000in}}%
\pgfusepath{clip}%
\pgfsetbuttcap%
\pgfsetroundjoin%
\definecolor{currentfill}{rgb}{0.899326,0.817325,0.823820}%
\pgfsetfillcolor{currentfill}%
\pgfsetlinewidth{0.000000pt}%
\definecolor{currentstroke}{rgb}{0.000000,0.000000,0.000000}%
\pgfsetstrokecolor{currentstroke}%
\pgfsetdash{}{0pt}%
\pgfpathmoveto{\pgfqpoint{0.896655in}{0.951629in}}%
\pgfpathlineto{\pgfqpoint{0.935388in}{0.971179in}}%
\pgfpathlineto{\pgfqpoint{0.898980in}{0.990657in}}%
\pgfpathlineto{\pgfqpoint{0.860251in}{0.971179in}}%
\pgfpathclose%
\pgfusepath{fill}%
\end{pgfscope}%
\begin{pgfscope}%
\pgfpathrectangle{\pgfqpoint{0.150000in}{0.150000in}}{\pgfqpoint{2.700000in}{1.950000in}}%
\pgfusepath{clip}%
\pgfsetbuttcap%
\pgfsetroundjoin%
\definecolor{currentfill}{rgb}{0.899326,0.817325,0.823820}%
\pgfsetfillcolor{currentfill}%
\pgfsetlinewidth{0.000000pt}%
\definecolor{currentstroke}{rgb}{0.000000,0.000000,0.000000}%
\pgfsetstrokecolor{currentstroke}%
\pgfsetdash{}{0pt}%
\pgfpathmoveto{\pgfqpoint{0.821381in}{0.951629in}}%
\pgfpathlineto{\pgfqpoint{0.860251in}{0.971179in}}%
\pgfpathlineto{\pgfqpoint{0.823979in}{0.990657in}}%
\pgfpathlineto{\pgfqpoint{0.785114in}{0.971179in}}%
\pgfpathclose%
\pgfusepath{fill}%
\end{pgfscope}%
\begin{pgfscope}%
\pgfpathrectangle{\pgfqpoint{0.150000in}{0.150000in}}{\pgfqpoint{2.700000in}{1.950000in}}%
\pgfusepath{clip}%
\pgfsetbuttcap%
\pgfsetroundjoin%
\definecolor{currentfill}{rgb}{0.899326,0.817325,0.823820}%
\pgfsetfillcolor{currentfill}%
\pgfsetlinewidth{0.000000pt}%
\definecolor{currentstroke}{rgb}{0.000000,0.000000,0.000000}%
\pgfsetstrokecolor{currentstroke}%
\pgfsetdash{}{0pt}%
\pgfpathmoveto{\pgfqpoint{1.724816in}{0.959521in}}%
\pgfpathlineto{\pgfqpoint{1.761556in}{0.955431in}}%
\pgfpathlineto{\pgfqpoint{1.723846in}{0.982777in}}%
\pgfpathlineto{\pgfqpoint{1.686990in}{0.986975in}}%
\pgfpathclose%
\pgfusepath{fill}%
\end{pgfscope}%
\begin{pgfscope}%
\pgfpathrectangle{\pgfqpoint{0.150000in}{0.150000in}}{\pgfqpoint{2.700000in}{1.950000in}}%
\pgfusepath{clip}%
\pgfsetbuttcap%
\pgfsetroundjoin%
\definecolor{currentfill}{rgb}{0.891161,0.904565,0.923330}%
\pgfsetfillcolor{currentfill}%
\pgfsetlinewidth{0.000000pt}%
\definecolor{currentstroke}{rgb}{0.000000,0.000000,0.000000}%
\pgfsetstrokecolor{currentstroke}%
\pgfsetdash{}{0pt}%
\pgfpathmoveto{\pgfqpoint{1.272044in}{1.298851in}}%
\pgfpathlineto{\pgfqpoint{1.311951in}{1.236657in}}%
\pgfpathlineto{\pgfqpoint{1.274805in}{1.263642in}}%
\pgfpathlineto{\pgfqpoint{1.234357in}{1.334224in}}%
\pgfpathclose%
\pgfusepath{fill}%
\end{pgfscope}%
\begin{pgfscope}%
\pgfpathrectangle{\pgfqpoint{0.150000in}{0.150000in}}{\pgfqpoint{2.700000in}{1.950000in}}%
\pgfusepath{clip}%
\pgfsetbuttcap%
\pgfsetroundjoin%
\definecolor{currentfill}{rgb}{0.944914,0.900046,0.903600}%
\pgfsetfillcolor{currentfill}%
\pgfsetlinewidth{0.000000pt}%
\definecolor{currentstroke}{rgb}{0.000000,0.000000,0.000000}%
\pgfsetstrokecolor{currentstroke}%
\pgfsetdash{}{0pt}%
\pgfpathmoveto{\pgfqpoint{1.574216in}{1.054135in}}%
\pgfpathlineto{\pgfqpoint{1.611465in}{1.025863in}}%
\pgfpathlineto{\pgfqpoint{1.573936in}{1.053118in}}%
\pgfpathlineto{\pgfqpoint{1.536486in}{1.081535in}}%
\pgfpathclose%
\pgfusepath{fill}%
\end{pgfscope}%
\begin{pgfscope}%
\pgfpathrectangle{\pgfqpoint{0.150000in}{0.150000in}}{\pgfqpoint{2.700000in}{1.950000in}}%
\pgfusepath{clip}%
\pgfsetbuttcap%
\pgfsetroundjoin%
\definecolor{currentfill}{rgb}{0.953355,0.959099,0.967142}%
\pgfsetfillcolor{currentfill}%
\pgfsetlinewidth{0.000000pt}%
\definecolor{currentstroke}{rgb}{0.000000,0.000000,0.000000}%
\pgfsetstrokecolor{currentstroke}%
\pgfsetdash{}{0pt}%
\pgfpathmoveto{\pgfqpoint{0.974222in}{1.139344in}}%
\pgfpathlineto{\pgfqpoint{1.007849in}{1.255307in}}%
\pgfpathlineto{\pgfqpoint{0.970691in}{1.282515in}}%
\pgfpathlineto{\pgfqpoint{0.937371in}{1.166476in}}%
\pgfpathclose%
\pgfusepath{fill}%
\end{pgfscope}%
\begin{pgfscope}%
\pgfpathrectangle{\pgfqpoint{0.150000in}{0.150000in}}{\pgfqpoint{2.700000in}{1.950000in}}%
\pgfusepath{clip}%
\pgfsetbuttcap%
\pgfsetroundjoin%
\definecolor{currentfill}{rgb}{0.929718,0.872472,0.877007}%
\pgfsetfillcolor{currentfill}%
\pgfsetlinewidth{0.000000pt}%
\definecolor{currentstroke}{rgb}{0.000000,0.000000,0.000000}%
\pgfsetstrokecolor{currentstroke}%
\pgfsetdash{}{0pt}%
\pgfpathmoveto{\pgfqpoint{1.085663in}{0.971179in}}%
\pgfpathlineto{\pgfqpoint{1.121777in}{1.046158in}}%
\pgfpathlineto{\pgfqpoint{1.084556in}{1.073545in}}%
\pgfpathlineto{\pgfqpoint{1.048981in}{0.990657in}}%
\pgfpathclose%
\pgfusepath{fill}%
\end{pgfscope}%
\begin{pgfscope}%
\pgfpathrectangle{\pgfqpoint{0.150000in}{0.150000in}}{\pgfqpoint{2.700000in}{1.950000in}}%
\pgfusepath{clip}%
\pgfsetbuttcap%
\pgfsetroundjoin%
\definecolor{currentfill}{rgb}{0.891728,0.803539,0.810524}%
\pgfsetfillcolor{currentfill}%
\pgfsetlinewidth{0.000000pt}%
\definecolor{currentstroke}{rgb}{0.000000,0.000000,0.000000}%
\pgfsetstrokecolor{currentstroke}%
\pgfsetdash{}{0pt}%
\pgfpathmoveto{\pgfqpoint{2.139781in}{0.932008in}}%
\pgfpathlineto{\pgfqpoint{2.176318in}{0.951629in}}%
\pgfpathlineto{\pgfqpoint{2.136671in}{0.955431in}}%
\pgfpathlineto{\pgfqpoint{2.100184in}{0.935882in}}%
\pgfpathclose%
\pgfusepath{fill}%
\end{pgfscope}%
\begin{pgfscope}%
\pgfpathrectangle{\pgfqpoint{0.150000in}{0.150000in}}{\pgfqpoint{2.700000in}{1.950000in}}%
\pgfusepath{clip}%
\pgfsetbuttcap%
\pgfsetroundjoin%
\definecolor{currentfill}{rgb}{0.872733,0.769072,0.777282}%
\pgfsetfillcolor{currentfill}%
\pgfsetlinewidth{0.000000pt}%
\definecolor{currentstroke}{rgb}{0.000000,0.000000,0.000000}%
\pgfsetstrokecolor{currentstroke}%
\pgfsetdash{}{0pt}%
\pgfpathmoveto{\pgfqpoint{1.951376in}{0.896571in}}%
\pgfpathlineto{\pgfqpoint{1.987923in}{0.908408in}}%
\pgfpathlineto{\pgfqpoint{1.949236in}{0.920183in}}%
\pgfpathlineto{\pgfqpoint{1.912683in}{0.908408in}}%
\pgfpathclose%
\pgfusepath{fill}%
\end{pgfscope}%
\begin{pgfscope}%
\pgfpathrectangle{\pgfqpoint{0.150000in}{0.150000in}}{\pgfqpoint{2.700000in}{1.950000in}}%
\pgfusepath{clip}%
\pgfsetbuttcap%
\pgfsetroundjoin%
\definecolor{currentfill}{rgb}{0.909819,0.920925,0.936474}%
\pgfsetfillcolor{currentfill}%
\pgfsetlinewidth{0.000000pt}%
\definecolor{currentstroke}{rgb}{0.000000,0.000000,0.000000}%
\pgfsetstrokecolor{currentstroke}%
\pgfsetdash{}{0pt}%
\pgfpathmoveto{\pgfqpoint{1.309580in}{1.271619in}}%
\pgfpathlineto{\pgfqpoint{1.349176in}{1.209615in}}%
\pgfpathlineto{\pgfqpoint{1.311951in}{1.236657in}}%
\pgfpathlineto{\pgfqpoint{1.272044in}{1.298851in}}%
\pgfpathclose%
\pgfusepath{fill}%
\end{pgfscope}%
\begin{pgfscope}%
\pgfpathrectangle{\pgfqpoint{0.150000in}{0.150000in}}{\pgfqpoint{2.700000in}{1.950000in}}%
\pgfusepath{clip}%
\pgfsetbuttcap%
\pgfsetroundjoin%
\definecolor{currentfill}{rgb}{0.895527,0.810432,0.817172}%
\pgfsetfillcolor{currentfill}%
\pgfsetlinewidth{0.000000pt}%
\definecolor{currentstroke}{rgb}{0.000000,0.000000,0.000000}%
\pgfsetstrokecolor{currentstroke}%
\pgfsetdash{}{0pt}%
\pgfpathmoveto{\pgfqpoint{1.762895in}{0.939899in}}%
\pgfpathlineto{\pgfqpoint{1.799545in}{0.935882in}}%
\pgfpathlineto{\pgfqpoint{1.761556in}{0.955431in}}%
\pgfpathlineto{\pgfqpoint{1.724816in}{0.959521in}}%
\pgfpathclose%
\pgfusepath{fill}%
\end{pgfscope}%
\begin{pgfscope}%
\pgfpathrectangle{\pgfqpoint{0.150000in}{0.150000in}}{\pgfqpoint{2.700000in}{1.950000in}}%
\pgfusepath{clip}%
\pgfsetbuttcap%
\pgfsetroundjoin%
\definecolor{currentfill}{rgb}{0.880331,0.782858,0.790579}%
\pgfsetfillcolor{currentfill}%
\pgfsetlinewidth{0.000000pt}%
\definecolor{currentstroke}{rgb}{0.000000,0.000000,0.000000}%
\pgfsetstrokecolor{currentstroke}%
\pgfsetdash{}{0pt}%
\pgfpathmoveto{\pgfqpoint{1.876201in}{0.904438in}}%
\pgfpathlineto{\pgfqpoint{1.912683in}{0.908408in}}%
\pgfpathlineto{\pgfqpoint{1.874191in}{0.920183in}}%
\pgfpathlineto{\pgfqpoint{1.837674in}{0.916263in}}%
\pgfpathclose%
\pgfusepath{fill}%
\end{pgfscope}%
\begin{pgfscope}%
\pgfpathrectangle{\pgfqpoint{0.150000in}{0.150000in}}{\pgfqpoint{2.700000in}{1.950000in}}%
\pgfusepath{clip}%
\pgfsetbuttcap%
\pgfsetroundjoin%
\definecolor{currentfill}{rgb}{0.941115,0.893153,0.896952}%
\pgfsetfillcolor{currentfill}%
\pgfsetlinewidth{0.000000pt}%
\definecolor{currentstroke}{rgb}{0.000000,0.000000,0.000000}%
\pgfsetstrokecolor{currentstroke}%
\pgfsetdash{}{0pt}%
\pgfpathmoveto{\pgfqpoint{1.612026in}{1.026677in}}%
\pgfpathlineto{\pgfqpoint{1.649159in}{1.006454in}}%
\pgfpathlineto{\pgfqpoint{1.611465in}{1.025863in}}%
\pgfpathlineto{\pgfqpoint{1.574216in}{1.054135in}}%
\pgfpathclose%
\pgfusepath{fill}%
\end{pgfscope}%
\begin{pgfscope}%
\pgfpathrectangle{\pgfqpoint{0.150000in}{0.150000in}}{\pgfqpoint{2.700000in}{1.950000in}}%
\pgfusepath{clip}%
\pgfsetbuttcap%
\pgfsetroundjoin%
\definecolor{currentfill}{rgb}{0.899326,0.817325,0.823820}%
\pgfsetfillcolor{currentfill}%
\pgfsetlinewidth{0.000000pt}%
\definecolor{currentstroke}{rgb}{0.000000,0.000000,0.000000}%
\pgfsetstrokecolor{currentstroke}%
\pgfsetdash{}{0pt}%
\pgfpathmoveto{\pgfqpoint{2.215193in}{0.932008in}}%
\pgfpathlineto{\pgfqpoint{2.251592in}{0.951629in}}%
\pgfpathlineto{\pgfqpoint{2.212722in}{0.971179in}}%
\pgfpathlineto{\pgfqpoint{2.176318in}{0.951629in}}%
\pgfpathclose%
\pgfusepath{fill}%
\end{pgfscope}%
\begin{pgfscope}%
\pgfpathrectangle{\pgfqpoint{0.150000in}{0.150000in}}{\pgfqpoint{2.700000in}{1.950000in}}%
\pgfusepath{clip}%
\pgfsetbuttcap%
\pgfsetroundjoin%
\definecolor{currentfill}{rgb}{0.899326,0.817325,0.823820}%
\pgfsetfillcolor{currentfill}%
\pgfsetlinewidth{0.000000pt}%
\definecolor{currentstroke}{rgb}{0.000000,0.000000,0.000000}%
\pgfsetstrokecolor{currentstroke}%
\pgfsetdash{}{0pt}%
\pgfpathmoveto{\pgfqpoint{1.084016in}{0.932008in}}%
\pgfpathlineto{\pgfqpoint{1.122478in}{0.951629in}}%
\pgfpathlineto{\pgfqpoint{1.085663in}{0.971179in}}%
\pgfpathlineto{\pgfqpoint{1.047203in}{0.951629in}}%
\pgfpathclose%
\pgfusepath{fill}%
\end{pgfscope}%
\begin{pgfscope}%
\pgfpathrectangle{\pgfqpoint{0.150000in}{0.150000in}}{\pgfqpoint{2.700000in}{1.950000in}}%
\pgfusepath{clip}%
\pgfsetbuttcap%
\pgfsetroundjoin%
\definecolor{currentfill}{rgb}{0.899326,0.817325,0.823820}%
\pgfsetfillcolor{currentfill}%
\pgfsetlinewidth{0.000000pt}%
\definecolor{currentstroke}{rgb}{0.000000,0.000000,0.000000}%
\pgfsetstrokecolor{currentstroke}%
\pgfsetdash{}{0pt}%
\pgfpathmoveto{\pgfqpoint{1.008604in}{0.932008in}}%
\pgfpathlineto{\pgfqpoint{1.047203in}{0.951629in}}%
\pgfpathlineto{\pgfqpoint{1.010525in}{0.971179in}}%
\pgfpathlineto{\pgfqpoint{0.971929in}{0.951629in}}%
\pgfpathclose%
\pgfusepath{fill}%
\end{pgfscope}%
\begin{pgfscope}%
\pgfpathrectangle{\pgfqpoint{0.150000in}{0.150000in}}{\pgfqpoint{2.700000in}{1.950000in}}%
\pgfusepath{clip}%
\pgfsetbuttcap%
\pgfsetroundjoin%
\definecolor{currentfill}{rgb}{0.899326,0.817325,0.823820}%
\pgfsetfillcolor{currentfill}%
\pgfsetlinewidth{0.000000pt}%
\definecolor{currentstroke}{rgb}{0.000000,0.000000,0.000000}%
\pgfsetstrokecolor{currentstroke}%
\pgfsetdash{}{0pt}%
\pgfpathmoveto{\pgfqpoint{0.933192in}{0.932008in}}%
\pgfpathlineto{\pgfqpoint{0.971929in}{0.951629in}}%
\pgfpathlineto{\pgfqpoint{0.935388in}{0.971179in}}%
\pgfpathlineto{\pgfqpoint{0.896655in}{0.951629in}}%
\pgfpathclose%
\pgfusepath{fill}%
\end{pgfscope}%
\begin{pgfscope}%
\pgfpathrectangle{\pgfqpoint{0.150000in}{0.150000in}}{\pgfqpoint{2.700000in}{1.950000in}}%
\pgfusepath{clip}%
\pgfsetbuttcap%
\pgfsetroundjoin%
\definecolor{currentfill}{rgb}{0.899326,0.817325,0.823820}%
\pgfsetfillcolor{currentfill}%
\pgfsetlinewidth{0.000000pt}%
\definecolor{currentstroke}{rgb}{0.000000,0.000000,0.000000}%
\pgfsetstrokecolor{currentstroke}%
\pgfsetdash{}{0pt}%
\pgfpathmoveto{\pgfqpoint{0.857780in}{0.932008in}}%
\pgfpathlineto{\pgfqpoint{0.896655in}{0.951629in}}%
\pgfpathlineto{\pgfqpoint{0.860251in}{0.971179in}}%
\pgfpathlineto{\pgfqpoint{0.821381in}{0.951629in}}%
\pgfpathclose%
\pgfusepath{fill}%
\end{pgfscope}%
\begin{pgfscope}%
\pgfpathrectangle{\pgfqpoint{0.150000in}{0.150000in}}{\pgfqpoint{2.700000in}{1.950000in}}%
\pgfusepath{clip}%
\pgfsetbuttcap%
\pgfsetroundjoin%
\definecolor{currentfill}{rgb}{0.880331,0.782858,0.790579}%
\pgfsetfillcolor{currentfill}%
\pgfsetlinewidth{0.000000pt}%
\definecolor{currentstroke}{rgb}{0.000000,0.000000,0.000000}%
\pgfsetstrokecolor{currentstroke}%
\pgfsetdash{}{0pt}%
\pgfpathmoveto{\pgfqpoint{2.027185in}{0.904438in}}%
\pgfpathlineto{\pgfqpoint{2.063966in}{0.924130in}}%
\pgfpathlineto{\pgfqpoint{2.024282in}{0.920183in}}%
\pgfpathlineto{\pgfqpoint{1.987923in}{0.908408in}}%
\pgfpathclose%
\pgfusepath{fill}%
\end{pgfscope}%
\begin{pgfscope}%
\pgfpathrectangle{\pgfqpoint{0.150000in}{0.150000in}}{\pgfqpoint{2.700000in}{1.950000in}}%
\pgfusepath{clip}%
\pgfsetbuttcap%
\pgfsetroundjoin%
\definecolor{currentfill}{rgb}{0.959574,0.964553,0.971523}%
\pgfsetfillcolor{currentfill}%
\pgfsetlinewidth{0.000000pt}%
\definecolor{currentstroke}{rgb}{0.000000,0.000000,0.000000}%
\pgfsetstrokecolor{currentstroke}%
\pgfsetdash{}{0pt}%
\pgfpathmoveto{\pgfqpoint{1.010752in}{1.120144in}}%
\pgfpathlineto{\pgfqpoint{1.045085in}{1.228041in}}%
\pgfpathlineto{\pgfqpoint{1.007849in}{1.255307in}}%
\pgfpathlineto{\pgfqpoint{0.974222in}{1.139344in}}%
\pgfpathclose%
\pgfusepath{fill}%
\end{pgfscope}%
\begin{pgfscope}%
\pgfpathrectangle{\pgfqpoint{0.150000in}{0.150000in}}{\pgfqpoint{2.700000in}{1.950000in}}%
\pgfusepath{clip}%
\pgfsetbuttcap%
\pgfsetroundjoin%
\definecolor{currentfill}{rgb}{0.928477,0.937286,0.949617}%
\pgfsetfillcolor{currentfill}%
\pgfsetlinewidth{0.000000pt}%
\definecolor{currentstroke}{rgb}{0.000000,0.000000,0.000000}%
\pgfsetstrokecolor{currentstroke}%
\pgfsetdash{}{0pt}%
\pgfpathmoveto{\pgfqpoint{1.347341in}{1.236179in}}%
\pgfpathlineto{\pgfqpoint{1.386480in}{1.182516in}}%
\pgfpathlineto{\pgfqpoint{1.349176in}{1.209615in}}%
\pgfpathlineto{\pgfqpoint{1.309580in}{1.271619in}}%
\pgfpathclose%
\pgfusepath{fill}%
\end{pgfscope}%
\begin{pgfscope}%
\pgfpathrectangle{\pgfqpoint{0.150000in}{0.150000in}}{\pgfqpoint{2.700000in}{1.950000in}}%
\pgfusepath{clip}%
\pgfsetbuttcap%
\pgfsetroundjoin%
\definecolor{currentfill}{rgb}{0.929718,0.872472,0.877007}%
\pgfsetfillcolor{currentfill}%
\pgfsetlinewidth{0.000000pt}%
\definecolor{currentstroke}{rgb}{0.000000,0.000000,0.000000}%
\pgfsetstrokecolor{currentstroke}%
\pgfsetdash{}{0pt}%
\pgfpathmoveto{\pgfqpoint{1.122478in}{0.951629in}}%
\pgfpathlineto{\pgfqpoint{1.158788in}{1.026677in}}%
\pgfpathlineto{\pgfqpoint{1.121777in}{1.046158in}}%
\pgfpathlineto{\pgfqpoint{1.085663in}{0.971179in}}%
\pgfpathclose%
\pgfusepath{fill}%
\end{pgfscope}%
\begin{pgfscope}%
\pgfpathrectangle{\pgfqpoint{0.150000in}{0.150000in}}{\pgfqpoint{2.700000in}{1.950000in}}%
\pgfusepath{clip}%
\pgfsetbuttcap%
\pgfsetroundjoin%
\definecolor{currentfill}{rgb}{0.947135,0.953646,0.962760}%
\pgfsetfillcolor{currentfill}%
\pgfsetlinewidth{0.000000pt}%
\definecolor{currentstroke}{rgb}{0.000000,0.000000,0.000000}%
\pgfsetstrokecolor{currentstroke}%
\pgfsetdash{}{0pt}%
\pgfpathmoveto{\pgfqpoint{1.385125in}{1.200717in}}%
\pgfpathlineto{\pgfqpoint{1.423863in}{1.155360in}}%
\pgfpathlineto{\pgfqpoint{1.386480in}{1.182516in}}%
\pgfpathlineto{\pgfqpoint{1.347341in}{1.236179in}}%
\pgfpathclose%
\pgfusepath{fill}%
\end{pgfscope}%
\begin{pgfscope}%
\pgfpathrectangle{\pgfqpoint{0.150000in}{0.150000in}}{\pgfqpoint{2.700000in}{1.950000in}}%
\pgfusepath{clip}%
\pgfsetbuttcap%
\pgfsetroundjoin%
\definecolor{currentfill}{rgb}{0.891728,0.803539,0.810524}%
\pgfsetfillcolor{currentfill}%
\pgfsetlinewidth{0.000000pt}%
\definecolor{currentstroke}{rgb}{0.000000,0.000000,0.000000}%
\pgfsetstrokecolor{currentstroke}%
\pgfsetdash{}{0pt}%
\pgfpathmoveto{\pgfqpoint{2.103110in}{0.912316in}}%
\pgfpathlineto{\pgfqpoint{2.139781in}{0.932008in}}%
\pgfpathlineto{\pgfqpoint{2.100184in}{0.935882in}}%
\pgfpathlineto{\pgfqpoint{2.063966in}{0.924130in}}%
\pgfpathclose%
\pgfusepath{fill}%
\end{pgfscope}%
\begin{pgfscope}%
\pgfpathrectangle{\pgfqpoint{0.150000in}{0.150000in}}{\pgfqpoint{2.700000in}{1.950000in}}%
\pgfusepath{clip}%
\pgfsetbuttcap%
\pgfsetroundjoin%
\definecolor{currentfill}{rgb}{0.933517,0.879366,0.883655}%
\pgfsetfillcolor{currentfill}%
\pgfsetlinewidth{0.000000pt}%
\definecolor{currentstroke}{rgb}{0.000000,0.000000,0.000000}%
\pgfsetstrokecolor{currentstroke}%
\pgfsetdash{}{0pt}%
\pgfpathmoveto{\pgfqpoint{1.649917in}{0.999160in}}%
\pgfpathlineto{\pgfqpoint{1.686990in}{0.986975in}}%
\pgfpathlineto{\pgfqpoint{1.649159in}{1.006454in}}%
\pgfpathlineto{\pgfqpoint{1.612026in}{1.026677in}}%
\pgfpathclose%
\pgfusepath{fill}%
\end{pgfscope}%
\begin{pgfscope}%
\pgfpathrectangle{\pgfqpoint{0.150000in}{0.150000in}}{\pgfqpoint{2.700000in}{1.950000in}}%
\pgfusepath{clip}%
\pgfsetbuttcap%
\pgfsetroundjoin%
\definecolor{currentfill}{rgb}{0.895527,0.810432,0.817172}%
\pgfsetfillcolor{currentfill}%
\pgfsetlinewidth{0.000000pt}%
\definecolor{currentstroke}{rgb}{0.000000,0.000000,0.000000}%
\pgfsetstrokecolor{currentstroke}%
\pgfsetdash{}{0pt}%
\pgfpathmoveto{\pgfqpoint{1.801113in}{0.920206in}}%
\pgfpathlineto{\pgfqpoint{1.837674in}{0.916263in}}%
\pgfpathlineto{\pgfqpoint{1.799545in}{0.935882in}}%
\pgfpathlineto{\pgfqpoint{1.762895in}{0.939899in}}%
\pgfpathclose%
\pgfusepath{fill}%
\end{pgfscope}%
\begin{pgfscope}%
\pgfpathrectangle{\pgfqpoint{0.150000in}{0.150000in}}{\pgfqpoint{2.700000in}{1.950000in}}%
\pgfusepath{clip}%
\pgfsetbuttcap%
\pgfsetroundjoin%
\definecolor{currentfill}{rgb}{0.972013,0.975460,0.980285}%
\pgfsetfillcolor{currentfill}%
\pgfsetlinewidth{0.000000pt}%
\definecolor{currentstroke}{rgb}{0.000000,0.000000,0.000000}%
\pgfsetstrokecolor{currentstroke}%
\pgfsetdash{}{0pt}%
\pgfpathmoveto{\pgfqpoint{1.047787in}{1.092885in}}%
\pgfpathlineto{\pgfqpoint{1.082402in}{1.200717in}}%
\pgfpathlineto{\pgfqpoint{1.045085in}{1.228041in}}%
\pgfpathlineto{\pgfqpoint{1.010752in}{1.120144in}}%
\pgfpathclose%
\pgfusepath{fill}%
\end{pgfscope}%
\begin{pgfscope}%
\pgfpathrectangle{\pgfqpoint{0.150000in}{0.150000in}}{\pgfqpoint{2.700000in}{1.950000in}}%
\pgfusepath{clip}%
\pgfsetbuttcap%
\pgfsetroundjoin%
\definecolor{currentfill}{rgb}{0.899326,0.817325,0.823820}%
\pgfsetfillcolor{currentfill}%
\pgfsetlinewidth{0.000000pt}%
\definecolor{currentstroke}{rgb}{0.000000,0.000000,0.000000}%
\pgfsetstrokecolor{currentstroke}%
\pgfsetdash{}{0pt}%
\pgfpathmoveto{\pgfqpoint{2.178660in}{0.912316in}}%
\pgfpathlineto{\pgfqpoint{2.215193in}{0.932008in}}%
\pgfpathlineto{\pgfqpoint{2.176318in}{0.951629in}}%
\pgfpathlineto{\pgfqpoint{2.139781in}{0.932008in}}%
\pgfpathclose%
\pgfusepath{fill}%
\end{pgfscope}%
\begin{pgfscope}%
\pgfpathrectangle{\pgfqpoint{0.150000in}{0.150000in}}{\pgfqpoint{2.700000in}{1.950000in}}%
\pgfusepath{clip}%
\pgfsetbuttcap%
\pgfsetroundjoin%
\definecolor{currentfill}{rgb}{0.899326,0.817325,0.823820}%
\pgfsetfillcolor{currentfill}%
\pgfsetlinewidth{0.000000pt}%
\definecolor{currentstroke}{rgb}{0.000000,0.000000,0.000000}%
\pgfsetstrokecolor{currentstroke}%
\pgfsetdash{}{0pt}%
\pgfpathmoveto{\pgfqpoint{1.120962in}{0.912316in}}%
\pgfpathlineto{\pgfqpoint{1.159427in}{0.932008in}}%
\pgfpathlineto{\pgfqpoint{1.122478in}{0.951629in}}%
\pgfpathlineto{\pgfqpoint{1.084016in}{0.932008in}}%
\pgfpathclose%
\pgfusepath{fill}%
\end{pgfscope}%
\begin{pgfscope}%
\pgfpathrectangle{\pgfqpoint{0.150000in}{0.150000in}}{\pgfqpoint{2.700000in}{1.950000in}}%
\pgfusepath{clip}%
\pgfsetbuttcap%
\pgfsetroundjoin%
\definecolor{currentfill}{rgb}{0.899326,0.817325,0.823820}%
\pgfsetfillcolor{currentfill}%
\pgfsetlinewidth{0.000000pt}%
\definecolor{currentstroke}{rgb}{0.000000,0.000000,0.000000}%
\pgfsetstrokecolor{currentstroke}%
\pgfsetdash{}{0pt}%
\pgfpathmoveto{\pgfqpoint{0.969863in}{0.912316in}}%
\pgfpathlineto{\pgfqpoint{1.008604in}{0.932008in}}%
\pgfpathlineto{\pgfqpoint{0.971929in}{0.951629in}}%
\pgfpathlineto{\pgfqpoint{0.933192in}{0.932008in}}%
\pgfpathclose%
\pgfusepath{fill}%
\end{pgfscope}%
\begin{pgfscope}%
\pgfpathrectangle{\pgfqpoint{0.150000in}{0.150000in}}{\pgfqpoint{2.700000in}{1.950000in}}%
\pgfusepath{clip}%
\pgfsetbuttcap%
\pgfsetroundjoin%
\definecolor{currentfill}{rgb}{0.899326,0.817325,0.823820}%
\pgfsetfillcolor{currentfill}%
\pgfsetlinewidth{0.000000pt}%
\definecolor{currentstroke}{rgb}{0.000000,0.000000,0.000000}%
\pgfsetstrokecolor{currentstroke}%
\pgfsetdash{}{0pt}%
\pgfpathmoveto{\pgfqpoint{0.894313in}{0.912316in}}%
\pgfpathlineto{\pgfqpoint{0.933192in}{0.932008in}}%
\pgfpathlineto{\pgfqpoint{0.896655in}{0.951629in}}%
\pgfpathlineto{\pgfqpoint{0.857780in}{0.932008in}}%
\pgfpathclose%
\pgfusepath{fill}%
\end{pgfscope}%
\begin{pgfscope}%
\pgfpathrectangle{\pgfqpoint{0.150000in}{0.150000in}}{\pgfqpoint{2.700000in}{1.950000in}}%
\pgfusepath{clip}%
\pgfsetbuttcap%
\pgfsetroundjoin%
\definecolor{currentfill}{rgb}{0.899326,0.817325,0.823820}%
\pgfsetfillcolor{currentfill}%
\pgfsetlinewidth{0.000000pt}%
\definecolor{currentstroke}{rgb}{0.000000,0.000000,0.000000}%
\pgfsetstrokecolor{currentstroke}%
\pgfsetdash{}{0pt}%
\pgfpathmoveto{\pgfqpoint{1.045413in}{0.912316in}}%
\pgfpathlineto{\pgfqpoint{1.084016in}{0.932008in}}%
\pgfpathlineto{\pgfqpoint{1.047203in}{0.951629in}}%
\pgfpathlineto{\pgfqpoint{1.008604in}{0.932008in}}%
\pgfpathclose%
\pgfusepath{fill}%
\end{pgfscope}%
\begin{pgfscope}%
\pgfpathrectangle{\pgfqpoint{0.150000in}{0.150000in}}{\pgfqpoint{2.700000in}{1.950000in}}%
\pgfusepath{clip}%
\pgfsetbuttcap%
\pgfsetroundjoin%
\definecolor{currentfill}{rgb}{0.965794,0.970006,0.975904}%
\pgfsetfillcolor{currentfill}%
\pgfsetlinewidth{0.000000pt}%
\definecolor{currentstroke}{rgb}{0.000000,0.000000,0.000000}%
\pgfsetstrokecolor{currentstroke}%
\pgfsetdash{}{0pt}%
\pgfpathmoveto{\pgfqpoint{1.422844in}{1.173335in}}%
\pgfpathlineto{\pgfqpoint{1.461324in}{1.128146in}}%
\pgfpathlineto{\pgfqpoint{1.423863in}{1.155360in}}%
\pgfpathlineto{\pgfqpoint{1.385125in}{1.200717in}}%
\pgfpathclose%
\pgfusepath{fill}%
\end{pgfscope}%
\begin{pgfscope}%
\pgfpathrectangle{\pgfqpoint{0.150000in}{0.150000in}}{\pgfqpoint{2.700000in}{1.950000in}}%
\pgfusepath{clip}%
\pgfsetbuttcap%
\pgfsetroundjoin%
\definecolor{currentfill}{rgb}{0.884130,0.789752,0.797227}%
\pgfsetfillcolor{currentfill}%
\pgfsetlinewidth{0.000000pt}%
\definecolor{currentstroke}{rgb}{0.000000,0.000000,0.000000}%
\pgfsetstrokecolor{currentstroke}%
\pgfsetdash{}{0pt}%
\pgfpathmoveto{\pgfqpoint{1.914638in}{0.884673in}}%
\pgfpathlineto{\pgfqpoint{1.951376in}{0.896571in}}%
\pgfpathlineto{\pgfqpoint{1.912683in}{0.908408in}}%
\pgfpathlineto{\pgfqpoint{1.876201in}{0.904438in}}%
\pgfpathclose%
\pgfusepath{fill}%
\end{pgfscope}%
\begin{pgfscope}%
\pgfpathrectangle{\pgfqpoint{0.150000in}{0.150000in}}{\pgfqpoint{2.700000in}{1.950000in}}%
\pgfusepath{clip}%
\pgfsetbuttcap%
\pgfsetroundjoin%
\definecolor{currentfill}{rgb}{0.884130,0.789752,0.797227}%
\pgfsetfillcolor{currentfill}%
\pgfsetlinewidth{0.000000pt}%
\definecolor{currentstroke}{rgb}{0.000000,0.000000,0.000000}%
\pgfsetstrokecolor{currentstroke}%
\pgfsetdash{}{0pt}%
\pgfpathmoveto{\pgfqpoint{1.990269in}{0.884673in}}%
\pgfpathlineto{\pgfqpoint{2.027185in}{0.904438in}}%
\pgfpathlineto{\pgfqpoint{1.987923in}{0.908408in}}%
\pgfpathlineto{\pgfqpoint{1.951376in}{0.896571in}}%
\pgfpathclose%
\pgfusepath{fill}%
\end{pgfscope}%
\begin{pgfscope}%
\pgfpathrectangle{\pgfqpoint{0.150000in}{0.150000in}}{\pgfqpoint{2.700000in}{1.950000in}}%
\pgfusepath{clip}%
\pgfsetbuttcap%
\pgfsetroundjoin%
\definecolor{currentfill}{rgb}{0.925919,0.865579,0.870358}%
\pgfsetfillcolor{currentfill}%
\pgfsetlinewidth{0.000000pt}%
\definecolor{currentstroke}{rgb}{0.000000,0.000000,0.000000}%
\pgfsetstrokecolor{currentstroke}%
\pgfsetdash{}{0pt}%
\pgfpathmoveto{\pgfqpoint{1.159427in}{0.932008in}}%
\pgfpathlineto{\pgfqpoint{1.196195in}{0.999160in}}%
\pgfpathlineto{\pgfqpoint{1.158788in}{1.026677in}}%
\pgfpathlineto{\pgfqpoint{1.122478in}{0.951629in}}%
\pgfpathclose%
\pgfusepath{fill}%
\end{pgfscope}%
\begin{pgfscope}%
\pgfpathrectangle{\pgfqpoint{0.150000in}{0.150000in}}{\pgfqpoint{2.700000in}{1.950000in}}%
\pgfusepath{clip}%
\pgfsetbuttcap%
\pgfsetroundjoin%
\definecolor{currentfill}{rgb}{0.929718,0.872472,0.877007}%
\pgfsetfillcolor{currentfill}%
\pgfsetlinewidth{0.000000pt}%
\definecolor{currentstroke}{rgb}{0.000000,0.000000,0.000000}%
\pgfsetstrokecolor{currentstroke}%
\pgfsetdash{}{0pt}%
\pgfpathmoveto{\pgfqpoint{1.687888in}{0.971584in}}%
\pgfpathlineto{\pgfqpoint{1.724816in}{0.959521in}}%
\pgfpathlineto{\pgfqpoint{1.686990in}{0.986975in}}%
\pgfpathlineto{\pgfqpoint{1.649917in}{0.999160in}}%
\pgfpathclose%
\pgfusepath{fill}%
\end{pgfscope}%
\begin{pgfscope}%
\pgfpathrectangle{\pgfqpoint{0.150000in}{0.150000in}}{\pgfqpoint{2.700000in}{1.950000in}}%
\pgfusepath{clip}%
\pgfsetbuttcap%
\pgfsetroundjoin%
\definecolor{currentfill}{rgb}{0.984452,0.986366,0.989047}%
\pgfsetfillcolor{currentfill}%
\pgfsetlinewidth{0.000000pt}%
\definecolor{currentstroke}{rgb}{0.000000,0.000000,0.000000}%
\pgfsetstrokecolor{currentstroke}%
\pgfsetdash{}{0pt}%
\pgfpathmoveto{\pgfqpoint{1.084556in}{1.073545in}}%
\pgfpathlineto{\pgfqpoint{1.119798in}{1.173335in}}%
\pgfpathlineto{\pgfqpoint{1.082402in}{1.200717in}}%
\pgfpathlineto{\pgfqpoint{1.047787in}{1.092885in}}%
\pgfpathclose%
\pgfusepath{fill}%
\end{pgfscope}%
\begin{pgfscope}%
\pgfpathrectangle{\pgfqpoint{0.150000in}{0.150000in}}{\pgfqpoint{2.700000in}{1.950000in}}%
\pgfusepath{clip}%
\pgfsetbuttcap%
\pgfsetroundjoin%
\definecolor{currentfill}{rgb}{0.978232,0.980913,0.984666}%
\pgfsetfillcolor{currentfill}%
\pgfsetlinewidth{0.000000pt}%
\definecolor{currentstroke}{rgb}{0.000000,0.000000,0.000000}%
\pgfsetstrokecolor{currentstroke}%
\pgfsetdash{}{0pt}%
\pgfpathmoveto{\pgfqpoint{1.460702in}{1.137806in}}%
\pgfpathlineto{\pgfqpoint{1.498837in}{1.108877in}}%
\pgfpathlineto{\pgfqpoint{1.461324in}{1.128146in}}%
\pgfpathlineto{\pgfqpoint{1.422844in}{1.173335in}}%
\pgfpathclose%
\pgfusepath{fill}%
\end{pgfscope}%
\begin{pgfscope}%
\pgfpathrectangle{\pgfqpoint{0.150000in}{0.150000in}}{\pgfqpoint{2.700000in}{1.950000in}}%
\pgfusepath{clip}%
\pgfsetbuttcap%
\pgfsetroundjoin%
\definecolor{currentfill}{rgb}{0.895527,0.810432,0.817172}%
\pgfsetfillcolor{currentfill}%
\pgfsetlinewidth{0.000000pt}%
\definecolor{currentstroke}{rgb}{0.000000,0.000000,0.000000}%
\pgfsetstrokecolor{currentstroke}%
\pgfsetdash{}{0pt}%
\pgfpathmoveto{\pgfqpoint{2.066305in}{0.892551in}}%
\pgfpathlineto{\pgfqpoint{2.103110in}{0.912316in}}%
\pgfpathlineto{\pgfqpoint{2.063966in}{0.924130in}}%
\pgfpathlineto{\pgfqpoint{2.027185in}{0.904438in}}%
\pgfpathclose%
\pgfusepath{fill}%
\end{pgfscope}%
\begin{pgfscope}%
\pgfpathrectangle{\pgfqpoint{0.150000in}{0.150000in}}{\pgfqpoint{2.700000in}{1.950000in}}%
\pgfusepath{clip}%
\pgfsetbuttcap%
\pgfsetroundjoin%
\definecolor{currentfill}{rgb}{0.899326,0.817325,0.823820}%
\pgfsetfillcolor{currentfill}%
\pgfsetlinewidth{0.000000pt}%
\definecolor{currentstroke}{rgb}{0.000000,0.000000,0.000000}%
\pgfsetstrokecolor{currentstroke}%
\pgfsetdash{}{0pt}%
\pgfpathmoveto{\pgfqpoint{2.141993in}{0.892551in}}%
\pgfpathlineto{\pgfqpoint{2.178660in}{0.912316in}}%
\pgfpathlineto{\pgfqpoint{2.139781in}{0.932008in}}%
\pgfpathlineto{\pgfqpoint{2.103110in}{0.912316in}}%
\pgfpathclose%
\pgfusepath{fill}%
\end{pgfscope}%
\begin{pgfscope}%
\pgfpathrectangle{\pgfqpoint{0.150000in}{0.150000in}}{\pgfqpoint{2.700000in}{1.950000in}}%
\pgfusepath{clip}%
\pgfsetbuttcap%
\pgfsetroundjoin%
\definecolor{currentfill}{rgb}{0.899326,0.817325,0.823820}%
\pgfsetfillcolor{currentfill}%
\pgfsetlinewidth{0.000000pt}%
\definecolor{currentstroke}{rgb}{0.000000,0.000000,0.000000}%
\pgfsetstrokecolor{currentstroke}%
\pgfsetdash{}{0pt}%
\pgfpathmoveto{\pgfqpoint{1.158045in}{0.892551in}}%
\pgfpathlineto{\pgfqpoint{1.196512in}{0.912316in}}%
\pgfpathlineto{\pgfqpoint{1.159427in}{0.932008in}}%
\pgfpathlineto{\pgfqpoint{1.120962in}{0.912316in}}%
\pgfpathclose%
\pgfusepath{fill}%
\end{pgfscope}%
\begin{pgfscope}%
\pgfpathrectangle{\pgfqpoint{0.150000in}{0.150000in}}{\pgfqpoint{2.700000in}{1.950000in}}%
\pgfusepath{clip}%
\pgfsetbuttcap%
\pgfsetroundjoin%
\definecolor{currentfill}{rgb}{0.899326,0.817325,0.823820}%
\pgfsetfillcolor{currentfill}%
\pgfsetlinewidth{0.000000pt}%
\definecolor{currentstroke}{rgb}{0.000000,0.000000,0.000000}%
\pgfsetstrokecolor{currentstroke}%
\pgfsetdash{}{0pt}%
\pgfpathmoveto{\pgfqpoint{1.006668in}{0.892551in}}%
\pgfpathlineto{\pgfqpoint{1.045413in}{0.912316in}}%
\pgfpathlineto{\pgfqpoint{1.008604in}{0.932008in}}%
\pgfpathlineto{\pgfqpoint{0.969863in}{0.912316in}}%
\pgfpathclose%
\pgfusepath{fill}%
\end{pgfscope}%
\begin{pgfscope}%
\pgfpathrectangle{\pgfqpoint{0.150000in}{0.150000in}}{\pgfqpoint{2.700000in}{1.950000in}}%
\pgfusepath{clip}%
\pgfsetbuttcap%
\pgfsetroundjoin%
\definecolor{currentfill}{rgb}{0.899326,0.817325,0.823820}%
\pgfsetfillcolor{currentfill}%
\pgfsetlinewidth{0.000000pt}%
\definecolor{currentstroke}{rgb}{0.000000,0.000000,0.000000}%
\pgfsetstrokecolor{currentstroke}%
\pgfsetdash{}{0pt}%
\pgfpathmoveto{\pgfqpoint{0.930980in}{0.892551in}}%
\pgfpathlineto{\pgfqpoint{0.969863in}{0.912316in}}%
\pgfpathlineto{\pgfqpoint{0.933192in}{0.932008in}}%
\pgfpathlineto{\pgfqpoint{0.894313in}{0.912316in}}%
\pgfpathclose%
\pgfusepath{fill}%
\end{pgfscope}%
\begin{pgfscope}%
\pgfpathrectangle{\pgfqpoint{0.150000in}{0.150000in}}{\pgfqpoint{2.700000in}{1.950000in}}%
\pgfusepath{clip}%
\pgfsetbuttcap%
\pgfsetroundjoin%
\definecolor{currentfill}{rgb}{0.899326,0.817325,0.823820}%
\pgfsetfillcolor{currentfill}%
\pgfsetlinewidth{0.000000pt}%
\definecolor{currentstroke}{rgb}{0.000000,0.000000,0.000000}%
\pgfsetstrokecolor{currentstroke}%
\pgfsetdash{}{0pt}%
\pgfpathmoveto{\pgfqpoint{1.082356in}{0.892551in}}%
\pgfpathlineto{\pgfqpoint{1.120962in}{0.912316in}}%
\pgfpathlineto{\pgfqpoint{1.084016in}{0.932008in}}%
\pgfpathlineto{\pgfqpoint{1.045413in}{0.912316in}}%
\pgfpathclose%
\pgfusepath{fill}%
\end{pgfscope}%
\begin{pgfscope}%
\pgfpathrectangle{\pgfqpoint{0.150000in}{0.150000in}}{\pgfqpoint{2.700000in}{1.950000in}}%
\pgfusepath{clip}%
\pgfsetbuttcap%
\pgfsetroundjoin%
\definecolor{currentfill}{rgb}{0.899326,0.817325,0.823820}%
\pgfsetfillcolor{currentfill}%
\pgfsetlinewidth{0.000000pt}%
\definecolor{currentstroke}{rgb}{0.000000,0.000000,0.000000}%
\pgfsetstrokecolor{currentstroke}%
\pgfsetdash{}{0pt}%
\pgfpathmoveto{\pgfqpoint{1.839705in}{0.908342in}}%
\pgfpathlineto{\pgfqpoint{1.876201in}{0.904438in}}%
\pgfpathlineto{\pgfqpoint{1.837674in}{0.916263in}}%
\pgfpathlineto{\pgfqpoint{1.801113in}{0.920206in}}%
\pgfpathclose%
\pgfusepath{fill}%
\end{pgfscope}%
\begin{pgfscope}%
\pgfpathrectangle{\pgfqpoint{0.150000in}{0.150000in}}{\pgfqpoint{2.700000in}{1.950000in}}%
\pgfusepath{clip}%
\pgfsetbuttcap%
\pgfsetroundjoin%
\definecolor{currentfill}{rgb}{0.698361,0.735509,0.787515}%
\pgfsetfillcolor{currentfill}%
\pgfsetlinewidth{0.000000pt}%
\definecolor{currentstroke}{rgb}{0.000000,0.000000,0.000000}%
\pgfsetstrokecolor{currentstroke}%
\pgfsetdash{}{0pt}%
\pgfpathmoveto{\pgfqpoint{0.859175in}{1.371990in}}%
\pgfpathlineto{\pgfqpoint{0.891091in}{1.506856in}}%
\pgfpathlineto{\pgfqpoint{0.853333in}{1.542342in}}%
\pgfpathlineto{\pgfqpoint{0.822304in}{1.398980in}}%
\pgfpathclose%
\pgfusepath{fill}%
\end{pgfscope}%
\begin{pgfscope}%
\pgfpathrectangle{\pgfqpoint{0.150000in}{0.150000in}}{\pgfqpoint{2.700000in}{1.950000in}}%
\pgfusepath{clip}%
\pgfsetbuttcap%
\pgfsetroundjoin%
\definecolor{currentfill}{rgb}{0.996890,0.997273,0.997809}%
\pgfsetfillcolor{currentfill}%
\pgfsetlinewidth{0.000000pt}%
\definecolor{currentstroke}{rgb}{0.000000,0.000000,0.000000}%
\pgfsetstrokecolor{currentstroke}%
\pgfsetdash{}{0pt}%
\pgfpathmoveto{\pgfqpoint{1.121777in}{1.046158in}}%
\pgfpathlineto{\pgfqpoint{1.157565in}{1.137806in}}%
\pgfpathlineto{\pgfqpoint{1.119798in}{1.173335in}}%
\pgfpathlineto{\pgfqpoint{1.084556in}{1.073545in}}%
\pgfpathclose%
\pgfusepath{fill}%
\end{pgfscope}%
\begin{pgfscope}%
\pgfpathrectangle{\pgfqpoint{0.150000in}{0.150000in}}{\pgfqpoint{2.700000in}{1.950000in}}%
\pgfusepath{clip}%
\pgfsetbuttcap%
\pgfsetroundjoin%
\definecolor{currentfill}{rgb}{0.922120,0.858686,0.863710}%
\pgfsetfillcolor{currentfill}%
\pgfsetlinewidth{0.000000pt}%
\definecolor{currentstroke}{rgb}{0.000000,0.000000,0.000000}%
\pgfsetstrokecolor{currentstroke}%
\pgfsetdash{}{0pt}%
\pgfpathmoveto{\pgfqpoint{1.725941in}{0.943950in}}%
\pgfpathlineto{\pgfqpoint{1.762895in}{0.939899in}}%
\pgfpathlineto{\pgfqpoint{1.724816in}{0.959521in}}%
\pgfpathlineto{\pgfqpoint{1.687888in}{0.971584in}}%
\pgfpathclose%
\pgfusepath{fill}%
\end{pgfscope}%
\begin{pgfscope}%
\pgfpathrectangle{\pgfqpoint{0.150000in}{0.150000in}}{\pgfqpoint{2.700000in}{1.950000in}}%
\pgfusepath{clip}%
\pgfsetbuttcap%
\pgfsetroundjoin%
\definecolor{currentfill}{rgb}{0.990671,0.991820,0.993428}%
\pgfsetfillcolor{currentfill}%
\pgfsetlinewidth{0.000000pt}%
\definecolor{currentstroke}{rgb}{0.000000,0.000000,0.000000}%
\pgfsetstrokecolor{currentstroke}%
\pgfsetdash{}{0pt}%
\pgfpathmoveto{\pgfqpoint{1.498554in}{1.110318in}}%
\pgfpathlineto{\pgfqpoint{1.536486in}{1.081535in}}%
\pgfpathlineto{\pgfqpoint{1.498837in}{1.108877in}}%
\pgfpathlineto{\pgfqpoint{1.460702in}{1.137806in}}%
\pgfpathclose%
\pgfusepath{fill}%
\end{pgfscope}%
\begin{pgfscope}%
\pgfpathrectangle{\pgfqpoint{0.150000in}{0.150000in}}{\pgfqpoint{2.700000in}{1.950000in}}%
\pgfusepath{clip}%
\pgfsetbuttcap%
\pgfsetroundjoin%
\definecolor{currentfill}{rgb}{0.925919,0.865579,0.870358}%
\pgfsetfillcolor{currentfill}%
\pgfsetlinewidth{0.000000pt}%
\definecolor{currentstroke}{rgb}{0.000000,0.000000,0.000000}%
\pgfsetstrokecolor{currentstroke}%
\pgfsetdash{}{0pt}%
\pgfpathmoveto{\pgfqpoint{1.196512in}{0.912316in}}%
\pgfpathlineto{\pgfqpoint{1.233450in}{0.979536in}}%
\pgfpathlineto{\pgfqpoint{1.196195in}{0.999160in}}%
\pgfpathlineto{\pgfqpoint{1.159427in}{0.932008in}}%
\pgfpathclose%
\pgfusepath{fill}%
\end{pgfscope}%
\begin{pgfscope}%
\pgfpathrectangle{\pgfqpoint{0.150000in}{0.150000in}}{\pgfqpoint{2.700000in}{1.950000in}}%
\pgfusepath{clip}%
\pgfsetbuttcap%
\pgfsetroundjoin%
\definecolor{currentfill}{rgb}{0.887929,0.796645,0.803876}%
\pgfsetfillcolor{currentfill}%
\pgfsetlinewidth{0.000000pt}%
\definecolor{currentstroke}{rgb}{0.000000,0.000000,0.000000}%
\pgfsetstrokecolor{currentstroke}%
\pgfsetdash{}{0pt}%
\pgfpathmoveto{\pgfqpoint{1.953217in}{0.864837in}}%
\pgfpathlineto{\pgfqpoint{1.990269in}{0.884673in}}%
\pgfpathlineto{\pgfqpoint{1.951376in}{0.896571in}}%
\pgfpathlineto{\pgfqpoint{1.914638in}{0.884673in}}%
\pgfpathclose%
\pgfusepath{fill}%
\end{pgfscope}%
\begin{pgfscope}%
\pgfpathrectangle{\pgfqpoint{0.150000in}{0.150000in}}{\pgfqpoint{2.700000in}{1.950000in}}%
\pgfusepath{clip}%
\pgfsetbuttcap%
\pgfsetroundjoin%
\definecolor{currentfill}{rgb}{0.723238,0.757322,0.805040}%
\pgfsetfillcolor{currentfill}%
\pgfsetlinewidth{0.000000pt}%
\definecolor{currentstroke}{rgb}{0.000000,0.000000,0.000000}%
\pgfsetstrokecolor{currentstroke}%
\pgfsetdash{}{0pt}%
\pgfpathmoveto{\pgfqpoint{0.896124in}{1.344943in}}%
\pgfpathlineto{\pgfqpoint{0.928873in}{1.471349in}}%
\pgfpathlineto{\pgfqpoint{0.891091in}{1.506856in}}%
\pgfpathlineto{\pgfqpoint{0.859175in}{1.371990in}}%
\pgfpathclose%
\pgfusepath{fill}%
\end{pgfscope}%
\begin{pgfscope}%
\pgfpathrectangle{\pgfqpoint{0.150000in}{0.150000in}}{\pgfqpoint{2.700000in}{1.950000in}}%
\pgfusepath{clip}%
\pgfsetbuttcap%
\pgfsetroundjoin%
\definecolor{currentfill}{rgb}{0.994301,0.989660,0.990028}%
\pgfsetfillcolor{currentfill}%
\pgfsetlinewidth{0.000000pt}%
\definecolor{currentstroke}{rgb}{0.000000,0.000000,0.000000}%
\pgfsetstrokecolor{currentstroke}%
\pgfsetdash{}{0pt}%
\pgfpathmoveto{\pgfqpoint{1.158788in}{1.026677in}}%
\pgfpathlineto{\pgfqpoint{1.195092in}{1.110318in}}%
\pgfpathlineto{\pgfqpoint{1.157565in}{1.137806in}}%
\pgfpathlineto{\pgfqpoint{1.121777in}{1.046158in}}%
\pgfpathclose%
\pgfusepath{fill}%
\end{pgfscope}%
\begin{pgfscope}%
\pgfpathrectangle{\pgfqpoint{0.150000in}{0.150000in}}{\pgfqpoint{2.700000in}{1.950000in}}%
\pgfusepath{clip}%
\pgfsetbuttcap%
\pgfsetroundjoin%
\definecolor{currentfill}{rgb}{0.994301,0.989660,0.990028}%
\pgfsetfillcolor{currentfill}%
\pgfsetlinewidth{0.000000pt}%
\definecolor{currentstroke}{rgb}{0.000000,0.000000,0.000000}%
\pgfsetstrokecolor{currentstroke}%
\pgfsetdash{}{0pt}%
\pgfpathmoveto{\pgfqpoint{1.536486in}{1.074722in}}%
\pgfpathlineto{\pgfqpoint{1.574216in}{1.054135in}}%
\pgfpathlineto{\pgfqpoint{1.536486in}{1.081535in}}%
\pgfpathlineto{\pgfqpoint{1.498554in}{1.110318in}}%
\pgfpathclose%
\pgfusepath{fill}%
\end{pgfscope}%
\begin{pgfscope}%
\pgfpathrectangle{\pgfqpoint{0.150000in}{0.150000in}}{\pgfqpoint{2.700000in}{1.950000in}}%
\pgfusepath{clip}%
\pgfsetbuttcap%
\pgfsetroundjoin%
\definecolor{currentfill}{rgb}{0.895527,0.810432,0.817172}%
\pgfsetfillcolor{currentfill}%
\pgfsetlinewidth{0.000000pt}%
\definecolor{currentstroke}{rgb}{0.000000,0.000000,0.000000}%
\pgfsetstrokecolor{currentstroke}%
\pgfsetdash{}{0pt}%
\pgfpathmoveto{\pgfqpoint{2.029365in}{0.872713in}}%
\pgfpathlineto{\pgfqpoint{2.066305in}{0.892551in}}%
\pgfpathlineto{\pgfqpoint{2.027185in}{0.904438in}}%
\pgfpathlineto{\pgfqpoint{1.990269in}{0.884673in}}%
\pgfpathclose%
\pgfusepath{fill}%
\end{pgfscope}%
\begin{pgfscope}%
\pgfpathrectangle{\pgfqpoint{0.150000in}{0.150000in}}{\pgfqpoint{2.700000in}{1.950000in}}%
\pgfusepath{clip}%
\pgfsetbuttcap%
\pgfsetroundjoin%
\definecolor{currentfill}{rgb}{0.741896,0.773683,0.818183}%
\pgfsetfillcolor{currentfill}%
\pgfsetlinewidth{0.000000pt}%
\definecolor{currentstroke}{rgb}{0.000000,0.000000,0.000000}%
\pgfsetstrokecolor{currentstroke}%
\pgfsetdash{}{0pt}%
\pgfpathmoveto{\pgfqpoint{0.933612in}{1.309665in}}%
\pgfpathlineto{\pgfqpoint{0.966238in}{1.444159in}}%
\pgfpathlineto{\pgfqpoint{0.928873in}{1.471349in}}%
\pgfpathlineto{\pgfqpoint{0.896124in}{1.344943in}}%
\pgfpathclose%
\pgfusepath{fill}%
\end{pgfscope}%
\begin{pgfscope}%
\pgfpathrectangle{\pgfqpoint{0.150000in}{0.150000in}}{\pgfqpoint{2.700000in}{1.950000in}}%
\pgfusepath{clip}%
\pgfsetbuttcap%
\pgfsetroundjoin%
\definecolor{currentfill}{rgb}{0.899326,0.817325,0.823820}%
\pgfsetfillcolor{currentfill}%
\pgfsetlinewidth{0.000000pt}%
\definecolor{currentstroke}{rgb}{0.000000,0.000000,0.000000}%
\pgfsetstrokecolor{currentstroke}%
\pgfsetdash{}{0pt}%
\pgfpathmoveto{\pgfqpoint{2.105192in}{0.872713in}}%
\pgfpathlineto{\pgfqpoint{2.141993in}{0.892551in}}%
\pgfpathlineto{\pgfqpoint{2.103110in}{0.912316in}}%
\pgfpathlineto{\pgfqpoint{2.066305in}{0.892551in}}%
\pgfpathclose%
\pgfusepath{fill}%
\end{pgfscope}%
\begin{pgfscope}%
\pgfpathrectangle{\pgfqpoint{0.150000in}{0.150000in}}{\pgfqpoint{2.700000in}{1.950000in}}%
\pgfusepath{clip}%
\pgfsetbuttcap%
\pgfsetroundjoin%
\definecolor{currentfill}{rgb}{0.899326,0.817325,0.823820}%
\pgfsetfillcolor{currentfill}%
\pgfsetlinewidth{0.000000pt}%
\definecolor{currentstroke}{rgb}{0.000000,0.000000,0.000000}%
\pgfsetstrokecolor{currentstroke}%
\pgfsetdash{}{0pt}%
\pgfpathmoveto{\pgfqpoint{1.195263in}{0.872713in}}%
\pgfpathlineto{\pgfqpoint{1.233733in}{0.892551in}}%
\pgfpathlineto{\pgfqpoint{1.196512in}{0.912316in}}%
\pgfpathlineto{\pgfqpoint{1.158045in}{0.892551in}}%
\pgfpathclose%
\pgfusepath{fill}%
\end{pgfscope}%
\begin{pgfscope}%
\pgfpathrectangle{\pgfqpoint{0.150000in}{0.150000in}}{\pgfqpoint{2.700000in}{1.950000in}}%
\pgfusepath{clip}%
\pgfsetbuttcap%
\pgfsetroundjoin%
\definecolor{currentfill}{rgb}{0.899326,0.817325,0.823820}%
\pgfsetfillcolor{currentfill}%
\pgfsetlinewidth{0.000000pt}%
\definecolor{currentstroke}{rgb}{0.000000,0.000000,0.000000}%
\pgfsetstrokecolor{currentstroke}%
\pgfsetdash{}{0pt}%
\pgfpathmoveto{\pgfqpoint{1.119436in}{0.872713in}}%
\pgfpathlineto{\pgfqpoint{1.158045in}{0.892551in}}%
\pgfpathlineto{\pgfqpoint{1.120962in}{0.912316in}}%
\pgfpathlineto{\pgfqpoint{1.082356in}{0.892551in}}%
\pgfpathclose%
\pgfusepath{fill}%
\end{pgfscope}%
\begin{pgfscope}%
\pgfpathrectangle{\pgfqpoint{0.150000in}{0.150000in}}{\pgfqpoint{2.700000in}{1.950000in}}%
\pgfusepath{clip}%
\pgfsetbuttcap%
\pgfsetroundjoin%
\definecolor{currentfill}{rgb}{0.899326,0.817325,0.823820}%
\pgfsetfillcolor{currentfill}%
\pgfsetlinewidth{0.000000pt}%
\definecolor{currentstroke}{rgb}{0.000000,0.000000,0.000000}%
\pgfsetstrokecolor{currentstroke}%
\pgfsetdash{}{0pt}%
\pgfpathmoveto{\pgfqpoint{1.043608in}{0.872713in}}%
\pgfpathlineto{\pgfqpoint{1.082356in}{0.892551in}}%
\pgfpathlineto{\pgfqpoint{1.045413in}{0.912316in}}%
\pgfpathlineto{\pgfqpoint{1.006668in}{0.892551in}}%
\pgfpathclose%
\pgfusepath{fill}%
\end{pgfscope}%
\begin{pgfscope}%
\pgfpathrectangle{\pgfqpoint{0.150000in}{0.150000in}}{\pgfqpoint{2.700000in}{1.950000in}}%
\pgfusepath{clip}%
\pgfsetbuttcap%
\pgfsetroundjoin%
\definecolor{currentfill}{rgb}{0.899326,0.817325,0.823820}%
\pgfsetfillcolor{currentfill}%
\pgfsetlinewidth{0.000000pt}%
\definecolor{currentstroke}{rgb}{0.000000,0.000000,0.000000}%
\pgfsetstrokecolor{currentstroke}%
\pgfsetdash{}{0pt}%
\pgfpathmoveto{\pgfqpoint{0.967781in}{0.872713in}}%
\pgfpathlineto{\pgfqpoint{1.006668in}{0.892551in}}%
\pgfpathlineto{\pgfqpoint{0.969863in}{0.912316in}}%
\pgfpathlineto{\pgfqpoint{0.930980in}{0.892551in}}%
\pgfpathclose%
\pgfusepath{fill}%
\end{pgfscope}%
\begin{pgfscope}%
\pgfpathrectangle{\pgfqpoint{0.150000in}{0.150000in}}{\pgfqpoint{2.700000in}{1.950000in}}%
\pgfusepath{clip}%
\pgfsetbuttcap%
\pgfsetroundjoin%
\definecolor{currentfill}{rgb}{0.982904,0.968980,0.970083}%
\pgfsetfillcolor{currentfill}%
\pgfsetlinewidth{0.000000pt}%
\definecolor{currentstroke}{rgb}{0.000000,0.000000,0.000000}%
\pgfsetstrokecolor{currentstroke}%
\pgfsetdash{}{0pt}%
\pgfpathmoveto{\pgfqpoint{1.574442in}{1.039104in}}%
\pgfpathlineto{\pgfqpoint{1.612026in}{1.026677in}}%
\pgfpathlineto{\pgfqpoint{1.574216in}{1.054135in}}%
\pgfpathlineto{\pgfqpoint{1.536486in}{1.074722in}}%
\pgfpathclose%
\pgfusepath{fill}%
\end{pgfscope}%
\begin{pgfscope}%
\pgfpathrectangle{\pgfqpoint{0.150000in}{0.150000in}}{\pgfqpoint{2.700000in}{1.950000in}}%
\pgfusepath{clip}%
\pgfsetbuttcap%
\pgfsetroundjoin%
\definecolor{currentfill}{rgb}{0.922120,0.858686,0.863710}%
\pgfsetfillcolor{currentfill}%
\pgfsetlinewidth{0.000000pt}%
\definecolor{currentstroke}{rgb}{0.000000,0.000000,0.000000}%
\pgfsetstrokecolor{currentstroke}%
\pgfsetdash{}{0pt}%
\pgfpathmoveto{\pgfqpoint{1.233733in}{0.892551in}}%
\pgfpathlineto{\pgfqpoint{1.271046in}{0.951889in}}%
\pgfpathlineto{\pgfqpoint{1.233450in}{0.979536in}}%
\pgfpathlineto{\pgfqpoint{1.196512in}{0.912316in}}%
\pgfpathclose%
\pgfusepath{fill}%
\end{pgfscope}%
\begin{pgfscope}%
\pgfpathrectangle{\pgfqpoint{0.150000in}{0.150000in}}{\pgfqpoint{2.700000in}{1.950000in}}%
\pgfusepath{clip}%
\pgfsetbuttcap%
\pgfsetroundjoin%
\definecolor{currentfill}{rgb}{0.922120,0.858686,0.863710}%
\pgfsetfillcolor{currentfill}%
\pgfsetlinewidth{0.000000pt}%
\definecolor{currentstroke}{rgb}{0.000000,0.000000,0.000000}%
\pgfsetstrokecolor{currentstroke}%
\pgfsetdash{}{0pt}%
\pgfpathmoveto{\pgfqpoint{1.764425in}{0.932121in}}%
\pgfpathlineto{\pgfqpoint{1.801113in}{0.920206in}}%
\pgfpathlineto{\pgfqpoint{1.762895in}{0.939899in}}%
\pgfpathlineto{\pgfqpoint{1.725941in}{0.943950in}}%
\pgfpathclose%
\pgfusepath{fill}%
\end{pgfscope}%
\begin{pgfscope}%
\pgfpathrectangle{\pgfqpoint{0.150000in}{0.150000in}}{\pgfqpoint{2.700000in}{1.950000in}}%
\pgfusepath{clip}%
\pgfsetbuttcap%
\pgfsetroundjoin%
\definecolor{currentfill}{rgb}{0.903125,0.824219,0.830469}%
\pgfsetfillcolor{currentfill}%
\pgfsetlinewidth{0.000000pt}%
\definecolor{currentstroke}{rgb}{0.000000,0.000000,0.000000}%
\pgfsetstrokecolor{currentstroke}%
\pgfsetdash{}{0pt}%
\pgfpathmoveto{\pgfqpoint{1.878235in}{0.888503in}}%
\pgfpathlineto{\pgfqpoint{1.914638in}{0.884673in}}%
\pgfpathlineto{\pgfqpoint{1.876201in}{0.904438in}}%
\pgfpathlineto{\pgfqpoint{1.839705in}{0.908342in}}%
\pgfpathclose%
\pgfusepath{fill}%
\end{pgfscope}%
\begin{pgfscope}%
\pgfpathrectangle{\pgfqpoint{0.150000in}{0.150000in}}{\pgfqpoint{2.700000in}{1.950000in}}%
\pgfusepath{clip}%
\pgfsetbuttcap%
\pgfsetroundjoin%
\definecolor{currentfill}{rgb}{0.567754,0.620987,0.695512}%
\pgfsetfillcolor{currentfill}%
\pgfsetlinewidth{0.000000pt}%
\definecolor{currentstroke}{rgb}{0.000000,0.000000,0.000000}%
\pgfsetstrokecolor{currentstroke}%
\pgfsetdash{}{0pt}%
\pgfpathmoveto{\pgfqpoint{0.964726in}{1.628624in}}%
\pgfpathlineto{\pgfqpoint{1.009915in}{1.521020in}}%
\pgfpathlineto{\pgfqpoint{0.972471in}{1.556172in}}%
\pgfpathlineto{\pgfqpoint{0.926793in}{1.664216in}}%
\pgfpathclose%
\pgfusepath{fill}%
\end{pgfscope}%
\begin{pgfscope}%
\pgfpathrectangle{\pgfqpoint{0.150000in}{0.150000in}}{\pgfqpoint{2.700000in}{1.950000in}}%
\pgfusepath{clip}%
\pgfsetbuttcap%
\pgfsetroundjoin%
\definecolor{currentfill}{rgb}{0.760555,0.790043,0.831327}%
\pgfsetfillcolor{currentfill}%
\pgfsetlinewidth{0.000000pt}%
\definecolor{currentstroke}{rgb}{0.000000,0.000000,0.000000}%
\pgfsetstrokecolor{currentstroke}%
\pgfsetdash{}{0pt}%
\pgfpathmoveto{\pgfqpoint{0.970691in}{1.282515in}}%
\pgfpathlineto{\pgfqpoint{1.004094in}{1.408584in}}%
\pgfpathlineto{\pgfqpoint{0.966238in}{1.444159in}}%
\pgfpathlineto{\pgfqpoint{0.933612in}{1.309665in}}%
\pgfpathclose%
\pgfusepath{fill}%
\end{pgfscope}%
\begin{pgfscope}%
\pgfpathrectangle{\pgfqpoint{0.150000in}{0.150000in}}{\pgfqpoint{2.700000in}{1.950000in}}%
\pgfusepath{clip}%
\pgfsetbuttcap%
\pgfsetroundjoin%
\definecolor{currentfill}{rgb}{0.986703,0.975873,0.976731}%
\pgfsetfillcolor{currentfill}%
\pgfsetlinewidth{0.000000pt}%
\definecolor{currentstroke}{rgb}{0.000000,0.000000,0.000000}%
\pgfsetstrokecolor{currentstroke}%
\pgfsetdash{}{0pt}%
\pgfpathmoveto{\pgfqpoint{1.196195in}{0.999160in}}%
\pgfpathlineto{\pgfqpoint{1.232700in}{1.082772in}}%
\pgfpathlineto{\pgfqpoint{1.195092in}{1.110318in}}%
\pgfpathlineto{\pgfqpoint{1.158788in}{1.026677in}}%
\pgfpathclose%
\pgfusepath{fill}%
\end{pgfscope}%
\begin{pgfscope}%
\pgfpathrectangle{\pgfqpoint{0.150000in}{0.150000in}}{\pgfqpoint{2.700000in}{1.950000in}}%
\pgfusepath{clip}%
\pgfsetbuttcap%
\pgfsetroundjoin%
\definecolor{currentfill}{rgb}{0.592632,0.642800,0.713036}%
\pgfsetfillcolor{currentfill}%
\pgfsetlinewidth{0.000000pt}%
\definecolor{currentstroke}{rgb}{0.000000,0.000000,0.000000}%
\pgfsetstrokecolor{currentstroke}%
\pgfsetdash{}{0pt}%
\pgfpathmoveto{\pgfqpoint{1.002682in}{1.593011in}}%
\pgfpathlineto{\pgfqpoint{1.047009in}{1.494143in}}%
\pgfpathlineto{\pgfqpoint{1.009915in}{1.521020in}}%
\pgfpathlineto{\pgfqpoint{0.964726in}{1.628624in}}%
\pgfpathclose%
\pgfusepath{fill}%
\end{pgfscope}%
\begin{pgfscope}%
\pgfpathrectangle{\pgfqpoint{0.150000in}{0.150000in}}{\pgfqpoint{2.700000in}{1.950000in}}%
\pgfusepath{clip}%
\pgfsetbuttcap%
\pgfsetroundjoin%
\definecolor{currentfill}{rgb}{0.971507,0.948300,0.950138}%
\pgfsetfillcolor{currentfill}%
\pgfsetlinewidth{0.000000pt}%
\definecolor{currentstroke}{rgb}{0.000000,0.000000,0.000000}%
\pgfsetstrokecolor{currentstroke}%
\pgfsetdash{}{0pt}%
\pgfpathmoveto{\pgfqpoint{1.612479in}{1.011465in}}%
\pgfpathlineto{\pgfqpoint{1.649917in}{0.999160in}}%
\pgfpathlineto{\pgfqpoint{1.612026in}{1.026677in}}%
\pgfpathlineto{\pgfqpoint{1.574442in}{1.039104in}}%
\pgfpathclose%
\pgfusepath{fill}%
\end{pgfscope}%
\begin{pgfscope}%
\pgfpathrectangle{\pgfqpoint{0.150000in}{0.150000in}}{\pgfqpoint{2.700000in}{1.950000in}}%
\pgfusepath{clip}%
\pgfsetbuttcap%
\pgfsetroundjoin%
\definecolor{currentfill}{rgb}{0.785432,0.811857,0.848851}%
\pgfsetfillcolor{currentfill}%
\pgfsetlinewidth{0.000000pt}%
\definecolor{currentstroke}{rgb}{0.000000,0.000000,0.000000}%
\pgfsetstrokecolor{currentstroke}%
\pgfsetdash{}{0pt}%
\pgfpathmoveto{\pgfqpoint{1.007849in}{1.255307in}}%
\pgfpathlineto{\pgfqpoint{1.041973in}{1.372988in}}%
\pgfpathlineto{\pgfqpoint{1.004094in}{1.408584in}}%
\pgfpathlineto{\pgfqpoint{0.970691in}{1.282515in}}%
\pgfpathclose%
\pgfusepath{fill}%
\end{pgfscope}%
\begin{pgfscope}%
\pgfpathrectangle{\pgfqpoint{0.150000in}{0.150000in}}{\pgfqpoint{2.700000in}{1.950000in}}%
\pgfusepath{clip}%
\pgfsetbuttcap%
\pgfsetroundjoin%
\definecolor{currentfill}{rgb}{0.617509,0.664614,0.730561}%
\pgfsetfillcolor{currentfill}%
\pgfsetlinewidth{0.000000pt}%
\definecolor{currentstroke}{rgb}{0.000000,0.000000,0.000000}%
\pgfsetstrokecolor{currentstroke}%
\pgfsetdash{}{0pt}%
\pgfpathmoveto{\pgfqpoint{1.041044in}{1.548924in}}%
\pgfpathlineto{\pgfqpoint{1.084526in}{1.458924in}}%
\pgfpathlineto{\pgfqpoint{1.047009in}{1.494143in}}%
\pgfpathlineto{\pgfqpoint{1.002682in}{1.593011in}}%
\pgfpathclose%
\pgfusepath{fill}%
\end{pgfscope}%
\begin{pgfscope}%
\pgfpathrectangle{\pgfqpoint{0.150000in}{0.150000in}}{\pgfqpoint{2.700000in}{1.950000in}}%
\pgfusepath{clip}%
\pgfsetbuttcap%
\pgfsetroundjoin%
\definecolor{currentfill}{rgb}{0.895527,0.810432,0.817172}%
\pgfsetfillcolor{currentfill}%
\pgfsetlinewidth{0.000000pt}%
\definecolor{currentstroke}{rgb}{0.000000,0.000000,0.000000}%
\pgfsetstrokecolor{currentstroke}%
\pgfsetdash{}{0pt}%
\pgfpathmoveto{\pgfqpoint{1.992288in}{0.852803in}}%
\pgfpathlineto{\pgfqpoint{2.029365in}{0.872713in}}%
\pgfpathlineto{\pgfqpoint{1.990269in}{0.884673in}}%
\pgfpathlineto{\pgfqpoint{1.953217in}{0.864837in}}%
\pgfpathclose%
\pgfusepath{fill}%
\end{pgfscope}%
\begin{pgfscope}%
\pgfpathrectangle{\pgfqpoint{0.150000in}{0.150000in}}{\pgfqpoint{2.700000in}{1.950000in}}%
\pgfusepath{clip}%
\pgfsetbuttcap%
\pgfsetroundjoin%
\definecolor{currentfill}{rgb}{0.648606,0.691881,0.752466}%
\pgfsetfillcolor{currentfill}%
\pgfsetlinewidth{0.000000pt}%
\definecolor{currentstroke}{rgb}{0.000000,0.000000,0.000000}%
\pgfsetstrokecolor{currentstroke}%
\pgfsetdash{}{0pt}%
\pgfpathmoveto{\pgfqpoint{1.079016in}{1.513292in}}%
\pgfpathlineto{\pgfqpoint{1.121749in}{1.431944in}}%
\pgfpathlineto{\pgfqpoint{1.084526in}{1.458924in}}%
\pgfpathlineto{\pgfqpoint{1.041044in}{1.548924in}}%
\pgfpathclose%
\pgfusepath{fill}%
\end{pgfscope}%
\begin{pgfscope}%
\pgfpathrectangle{\pgfqpoint{0.150000in}{0.150000in}}{\pgfqpoint{2.700000in}{1.950000in}}%
\pgfusepath{clip}%
\pgfsetbuttcap%
\pgfsetroundjoin%
\definecolor{currentfill}{rgb}{0.804090,0.828217,0.861994}%
\pgfsetfillcolor{currentfill}%
\pgfsetlinewidth{0.000000pt}%
\definecolor{currentstroke}{rgb}{0.000000,0.000000,0.000000}%
\pgfsetstrokecolor{currentstroke}%
\pgfsetdash{}{0pt}%
\pgfpathmoveto{\pgfqpoint{1.045085in}{1.228041in}}%
\pgfpathlineto{\pgfqpoint{1.079875in}{1.337370in}}%
\pgfpathlineto{\pgfqpoint{1.041973in}{1.372988in}}%
\pgfpathlineto{\pgfqpoint{1.007849in}{1.255307in}}%
\pgfpathclose%
\pgfusepath{fill}%
\end{pgfscope}%
\begin{pgfscope}%
\pgfpathrectangle{\pgfqpoint{0.150000in}{0.150000in}}{\pgfqpoint{2.700000in}{1.950000in}}%
\pgfusepath{clip}%
\pgfsetbuttcap%
\pgfsetroundjoin%
\definecolor{currentfill}{rgb}{0.899326,0.817325,0.823820}%
\pgfsetfillcolor{currentfill}%
\pgfsetlinewidth{0.000000pt}%
\definecolor{currentstroke}{rgb}{0.000000,0.000000,0.000000}%
\pgfsetstrokecolor{currentstroke}%
\pgfsetdash{}{0pt}%
\pgfpathmoveto{\pgfqpoint{2.068255in}{0.852803in}}%
\pgfpathlineto{\pgfqpoint{2.105192in}{0.872713in}}%
\pgfpathlineto{\pgfqpoint{2.066305in}{0.892551in}}%
\pgfpathlineto{\pgfqpoint{2.029365in}{0.872713in}}%
\pgfpathclose%
\pgfusepath{fill}%
\end{pgfscope}%
\begin{pgfscope}%
\pgfpathrectangle{\pgfqpoint{0.150000in}{0.150000in}}{\pgfqpoint{2.700000in}{1.950000in}}%
\pgfusepath{clip}%
\pgfsetbuttcap%
\pgfsetroundjoin%
\definecolor{currentfill}{rgb}{0.899326,0.817325,0.823820}%
\pgfsetfillcolor{currentfill}%
\pgfsetlinewidth{0.000000pt}%
\definecolor{currentstroke}{rgb}{0.000000,0.000000,0.000000}%
\pgfsetstrokecolor{currentstroke}%
\pgfsetdash{}{0pt}%
\pgfpathmoveto{\pgfqpoint{1.232619in}{0.852803in}}%
\pgfpathlineto{\pgfqpoint{1.271091in}{0.872713in}}%
\pgfpathlineto{\pgfqpoint{1.233733in}{0.892551in}}%
\pgfpathlineto{\pgfqpoint{1.195263in}{0.872713in}}%
\pgfpathclose%
\pgfusepath{fill}%
\end{pgfscope}%
\begin{pgfscope}%
\pgfpathrectangle{\pgfqpoint{0.150000in}{0.150000in}}{\pgfqpoint{2.700000in}{1.950000in}}%
\pgfusepath{clip}%
\pgfsetbuttcap%
\pgfsetroundjoin%
\definecolor{currentfill}{rgb}{0.899326,0.817325,0.823820}%
\pgfsetfillcolor{currentfill}%
\pgfsetlinewidth{0.000000pt}%
\definecolor{currentstroke}{rgb}{0.000000,0.000000,0.000000}%
\pgfsetstrokecolor{currentstroke}%
\pgfsetdash{}{0pt}%
\pgfpathmoveto{\pgfqpoint{1.156652in}{0.852803in}}%
\pgfpathlineto{\pgfqpoint{1.195263in}{0.872713in}}%
\pgfpathlineto{\pgfqpoint{1.158045in}{0.892551in}}%
\pgfpathlineto{\pgfqpoint{1.119436in}{0.872713in}}%
\pgfpathclose%
\pgfusepath{fill}%
\end{pgfscope}%
\begin{pgfscope}%
\pgfpathrectangle{\pgfqpoint{0.150000in}{0.150000in}}{\pgfqpoint{2.700000in}{1.950000in}}%
\pgfusepath{clip}%
\pgfsetbuttcap%
\pgfsetroundjoin%
\definecolor{currentfill}{rgb}{0.899326,0.817325,0.823820}%
\pgfsetfillcolor{currentfill}%
\pgfsetlinewidth{0.000000pt}%
\definecolor{currentstroke}{rgb}{0.000000,0.000000,0.000000}%
\pgfsetstrokecolor{currentstroke}%
\pgfsetdash{}{0pt}%
\pgfpathmoveto{\pgfqpoint{1.080685in}{0.852803in}}%
\pgfpathlineto{\pgfqpoint{1.119436in}{0.872713in}}%
\pgfpathlineto{\pgfqpoint{1.082356in}{0.892551in}}%
\pgfpathlineto{\pgfqpoint{1.043608in}{0.872713in}}%
\pgfpathclose%
\pgfusepath{fill}%
\end{pgfscope}%
\begin{pgfscope}%
\pgfpathrectangle{\pgfqpoint{0.150000in}{0.150000in}}{\pgfqpoint{2.700000in}{1.950000in}}%
\pgfusepath{clip}%
\pgfsetbuttcap%
\pgfsetroundjoin%
\definecolor{currentfill}{rgb}{0.899326,0.817325,0.823820}%
\pgfsetfillcolor{currentfill}%
\pgfsetlinewidth{0.000000pt}%
\definecolor{currentstroke}{rgb}{0.000000,0.000000,0.000000}%
\pgfsetstrokecolor{currentstroke}%
\pgfsetdash{}{0pt}%
\pgfpathmoveto{\pgfqpoint{1.004718in}{0.852803in}}%
\pgfpathlineto{\pgfqpoint{1.043608in}{0.872713in}}%
\pgfpathlineto{\pgfqpoint{1.006668in}{0.892551in}}%
\pgfpathlineto{\pgfqpoint{0.967781in}{0.872713in}}%
\pgfpathclose%
\pgfusepath{fill}%
\end{pgfscope}%
\begin{pgfscope}%
\pgfpathrectangle{\pgfqpoint{0.150000in}{0.150000in}}{\pgfqpoint{2.700000in}{1.950000in}}%
\pgfusepath{clip}%
\pgfsetbuttcap%
\pgfsetroundjoin%
\definecolor{currentfill}{rgb}{0.960110,0.927619,0.930193}%
\pgfsetfillcolor{currentfill}%
\pgfsetlinewidth{0.000000pt}%
\definecolor{currentstroke}{rgb}{0.000000,0.000000,0.000000}%
\pgfsetstrokecolor{currentstroke}%
\pgfsetdash{}{0pt}%
\pgfpathmoveto{\pgfqpoint{1.650509in}{0.975779in}}%
\pgfpathlineto{\pgfqpoint{1.687888in}{0.971584in}}%
\pgfpathlineto{\pgfqpoint{1.649917in}{0.999160in}}%
\pgfpathlineto{\pgfqpoint{1.612479in}{1.011465in}}%
\pgfpathclose%
\pgfusepath{fill}%
\end{pgfscope}%
\begin{pgfscope}%
\pgfpathrectangle{\pgfqpoint{0.150000in}{0.150000in}}{\pgfqpoint{2.700000in}{1.950000in}}%
\pgfusepath{clip}%
\pgfsetbuttcap%
\pgfsetroundjoin%
\definecolor{currentfill}{rgb}{0.922120,0.858686,0.863710}%
\pgfsetfillcolor{currentfill}%
\pgfsetlinewidth{0.000000pt}%
\definecolor{currentstroke}{rgb}{0.000000,0.000000,0.000000}%
\pgfsetstrokecolor{currentstroke}%
\pgfsetdash{}{0pt}%
\pgfpathmoveto{\pgfqpoint{1.271091in}{0.872713in}}%
\pgfpathlineto{\pgfqpoint{1.308548in}{0.932121in}}%
\pgfpathlineto{\pgfqpoint{1.271046in}{0.951889in}}%
\pgfpathlineto{\pgfqpoint{1.233733in}{0.892551in}}%
\pgfpathclose%
\pgfusepath{fill}%
\end{pgfscope}%
\begin{pgfscope}%
\pgfpathrectangle{\pgfqpoint{0.150000in}{0.150000in}}{\pgfqpoint{2.700000in}{1.950000in}}%
\pgfusepath{clip}%
\pgfsetbuttcap%
\pgfsetroundjoin%
\definecolor{currentfill}{rgb}{0.673483,0.713695,0.769991}%
\pgfsetfillcolor{currentfill}%
\pgfsetlinewidth{0.000000pt}%
\definecolor{currentstroke}{rgb}{0.000000,0.000000,0.000000}%
\pgfsetstrokecolor{currentstroke}%
\pgfsetdash{}{0pt}%
\pgfpathmoveto{\pgfqpoint{1.117012in}{1.477639in}}%
\pgfpathlineto{\pgfqpoint{1.159340in}{1.396659in}}%
\pgfpathlineto{\pgfqpoint{1.121749in}{1.431944in}}%
\pgfpathlineto{\pgfqpoint{1.079016in}{1.513292in}}%
\pgfpathclose%
\pgfusepath{fill}%
\end{pgfscope}%
\begin{pgfscope}%
\pgfpathrectangle{\pgfqpoint{0.150000in}{0.150000in}}{\pgfqpoint{2.700000in}{1.950000in}}%
\pgfusepath{clip}%
\pgfsetbuttcap%
\pgfsetroundjoin%
\definecolor{currentfill}{rgb}{0.982904,0.968980,0.970083}%
\pgfsetfillcolor{currentfill}%
\pgfsetlinewidth{0.000000pt}%
\definecolor{currentstroke}{rgb}{0.000000,0.000000,0.000000}%
\pgfsetstrokecolor{currentstroke}%
\pgfsetdash{}{0pt}%
\pgfpathmoveto{\pgfqpoint{1.233450in}{0.979536in}}%
\pgfpathlineto{\pgfqpoint{1.270388in}{1.055167in}}%
\pgfpathlineto{\pgfqpoint{1.232700in}{1.082772in}}%
\pgfpathlineto{\pgfqpoint{1.196195in}{0.999160in}}%
\pgfpathclose%
\pgfusepath{fill}%
\end{pgfscope}%
\begin{pgfscope}%
\pgfpathrectangle{\pgfqpoint{0.150000in}{0.150000in}}{\pgfqpoint{2.700000in}{1.950000in}}%
\pgfusepath{clip}%
\pgfsetbuttcap%
\pgfsetroundjoin%
\definecolor{currentfill}{rgb}{0.903125,0.824219,0.830469}%
\pgfsetfillcolor{currentfill}%
\pgfsetlinewidth{0.000000pt}%
\definecolor{currentstroke}{rgb}{0.000000,0.000000,0.000000}%
\pgfsetstrokecolor{currentstroke}%
\pgfsetdash{}{0pt}%
\pgfpathmoveto{\pgfqpoint{1.916907in}{0.868591in}}%
\pgfpathlineto{\pgfqpoint{1.953217in}{0.864837in}}%
\pgfpathlineto{\pgfqpoint{1.914638in}{0.884673in}}%
\pgfpathlineto{\pgfqpoint{1.878235in}{0.888503in}}%
\pgfpathclose%
\pgfusepath{fill}%
\end{pgfscope}%
\begin{pgfscope}%
\pgfpathrectangle{\pgfqpoint{0.150000in}{0.150000in}}{\pgfqpoint{2.700000in}{1.950000in}}%
\pgfusepath{clip}%
\pgfsetbuttcap%
\pgfsetroundjoin%
\definecolor{currentfill}{rgb}{0.698361,0.735509,0.787515}%
\pgfsetfillcolor{currentfill}%
\pgfsetlinewidth{0.000000pt}%
\definecolor{currentstroke}{rgb}{0.000000,0.000000,0.000000}%
\pgfsetstrokecolor{currentstroke}%
\pgfsetdash{}{0pt}%
\pgfpathmoveto{\pgfqpoint{1.155031in}{1.441965in}}%
\pgfpathlineto{\pgfqpoint{1.196953in}{1.361353in}}%
\pgfpathlineto{\pgfqpoint{1.159340in}{1.396659in}}%
\pgfpathlineto{\pgfqpoint{1.117012in}{1.477639in}}%
\pgfpathclose%
\pgfusepath{fill}%
\end{pgfscope}%
\begin{pgfscope}%
\pgfpathrectangle{\pgfqpoint{0.150000in}{0.150000in}}{\pgfqpoint{2.700000in}{1.950000in}}%
\pgfusepath{clip}%
\pgfsetbuttcap%
\pgfsetroundjoin%
\definecolor{currentfill}{rgb}{0.822748,0.844577,0.875138}%
\pgfsetfillcolor{currentfill}%
\pgfsetlinewidth{0.000000pt}%
\definecolor{currentstroke}{rgb}{0.000000,0.000000,0.000000}%
\pgfsetstrokecolor{currentstroke}%
\pgfsetdash{}{0pt}%
\pgfpathmoveto{\pgfqpoint{1.082402in}{1.200717in}}%
\pgfpathlineto{\pgfqpoint{1.117800in}{1.301731in}}%
\pgfpathlineto{\pgfqpoint{1.079875in}{1.337370in}}%
\pgfpathlineto{\pgfqpoint{1.045085in}{1.228041in}}%
\pgfpathclose%
\pgfusepath{fill}%
\end{pgfscope}%
\begin{pgfscope}%
\pgfpathrectangle{\pgfqpoint{0.150000in}{0.150000in}}{\pgfqpoint{2.700000in}{1.950000in}}%
\pgfusepath{clip}%
\pgfsetbuttcap%
\pgfsetroundjoin%
\definecolor{currentfill}{rgb}{0.723238,0.757322,0.805040}%
\pgfsetfillcolor{currentfill}%
\pgfsetlinewidth{0.000000pt}%
\definecolor{currentstroke}{rgb}{0.000000,0.000000,0.000000}%
\pgfsetstrokecolor{currentstroke}%
\pgfsetdash{}{0pt}%
\pgfpathmoveto{\pgfqpoint{1.193338in}{1.397929in}}%
\pgfpathlineto{\pgfqpoint{1.234357in}{1.334224in}}%
\pgfpathlineto{\pgfqpoint{1.196953in}{1.361353in}}%
\pgfpathlineto{\pgfqpoint{1.155031in}{1.441965in}}%
\pgfpathclose%
\pgfusepath{fill}%
\end{pgfscope}%
\begin{pgfscope}%
\pgfpathrectangle{\pgfqpoint{0.150000in}{0.150000in}}{\pgfqpoint{2.700000in}{1.950000in}}%
\pgfusepath{clip}%
\pgfsetbuttcap%
\pgfsetroundjoin%
\definecolor{currentfill}{rgb}{0.929718,0.872472,0.877007}%
\pgfsetfillcolor{currentfill}%
\pgfsetlinewidth{0.000000pt}%
\definecolor{currentstroke}{rgb}{0.000000,0.000000,0.000000}%
\pgfsetstrokecolor{currentstroke}%
\pgfsetdash{}{0pt}%
\pgfpathmoveto{\pgfqpoint{1.803111in}{0.920229in}}%
\pgfpathlineto{\pgfqpoint{1.839705in}{0.908342in}}%
\pgfpathlineto{\pgfqpoint{1.801113in}{0.920206in}}%
\pgfpathlineto{\pgfqpoint{1.764425in}{0.932121in}}%
\pgfpathclose%
\pgfusepath{fill}%
\end{pgfscope}%
\begin{pgfscope}%
\pgfpathrectangle{\pgfqpoint{0.150000in}{0.150000in}}{\pgfqpoint{2.700000in}{1.950000in}}%
\pgfusepath{clip}%
\pgfsetbuttcap%
\pgfsetroundjoin%
\definecolor{currentfill}{rgb}{0.948713,0.906939,0.910248}%
\pgfsetfillcolor{currentfill}%
\pgfsetlinewidth{0.000000pt}%
\definecolor{currentstroke}{rgb}{0.000000,0.000000,0.000000}%
\pgfsetstrokecolor{currentstroke}%
\pgfsetdash{}{0pt}%
\pgfpathmoveto{\pgfqpoint{1.688680in}{0.948034in}}%
\pgfpathlineto{\pgfqpoint{1.725941in}{0.943950in}}%
\pgfpathlineto{\pgfqpoint{1.687888in}{0.971584in}}%
\pgfpathlineto{\pgfqpoint{1.650509in}{0.975779in}}%
\pgfpathclose%
\pgfusepath{fill}%
\end{pgfscope}%
\begin{pgfscope}%
\pgfpathrectangle{\pgfqpoint{0.150000in}{0.150000in}}{\pgfqpoint{2.700000in}{1.950000in}}%
\pgfusepath{clip}%
\pgfsetbuttcap%
\pgfsetroundjoin%
\definecolor{currentfill}{rgb}{0.754335,0.784589,0.826945}%
\pgfsetfillcolor{currentfill}%
\pgfsetlinewidth{0.000000pt}%
\definecolor{currentstroke}{rgb}{0.000000,0.000000,0.000000}%
\pgfsetstrokecolor{currentstroke}%
\pgfsetdash{}{0pt}%
\pgfpathmoveto{\pgfqpoint{1.231373in}{1.362236in}}%
\pgfpathlineto{\pgfqpoint{1.272044in}{1.298851in}}%
\pgfpathlineto{\pgfqpoint{1.234357in}{1.334224in}}%
\pgfpathlineto{\pgfqpoint{1.193338in}{1.397929in}}%
\pgfpathclose%
\pgfusepath{fill}%
\end{pgfscope}%
\begin{pgfscope}%
\pgfpathrectangle{\pgfqpoint{0.150000in}{0.150000in}}{\pgfqpoint{2.700000in}{1.950000in}}%
\pgfusepath{clip}%
\pgfsetbuttcap%
\pgfsetroundjoin%
\definecolor{currentfill}{rgb}{0.847626,0.866391,0.892662}%
\pgfsetfillcolor{currentfill}%
\pgfsetlinewidth{0.000000pt}%
\definecolor{currentstroke}{rgb}{0.000000,0.000000,0.000000}%
\pgfsetstrokecolor{currentstroke}%
\pgfsetdash{}{0pt}%
\pgfpathmoveto{\pgfqpoint{1.119798in}{1.173335in}}%
\pgfpathlineto{\pgfqpoint{1.155747in}{1.266070in}}%
\pgfpathlineto{\pgfqpoint{1.117800in}{1.301731in}}%
\pgfpathlineto{\pgfqpoint{1.082402in}{1.200717in}}%
\pgfpathclose%
\pgfusepath{fill}%
\end{pgfscope}%
\begin{pgfscope}%
\pgfpathrectangle{\pgfqpoint{0.150000in}{0.150000in}}{\pgfqpoint{2.700000in}{1.950000in}}%
\pgfusepath{clip}%
\pgfsetbuttcap%
\pgfsetroundjoin%
\definecolor{currentfill}{rgb}{0.975306,0.955193,0.956786}%
\pgfsetfillcolor{currentfill}%
\pgfsetlinewidth{0.000000pt}%
\definecolor{currentstroke}{rgb}{0.000000,0.000000,0.000000}%
\pgfsetstrokecolor{currentstroke}%
\pgfsetdash{}{0pt}%
\pgfpathmoveto{\pgfqpoint{1.271046in}{0.951889in}}%
\pgfpathlineto{\pgfqpoint{1.308158in}{1.027503in}}%
\pgfpathlineto{\pgfqpoint{1.270388in}{1.055167in}}%
\pgfpathlineto{\pgfqpoint{1.233450in}{0.979536in}}%
\pgfpathclose%
\pgfusepath{fill}%
\end{pgfscope}%
\begin{pgfscope}%
\pgfpathrectangle{\pgfqpoint{0.150000in}{0.150000in}}{\pgfqpoint{2.700000in}{1.950000in}}%
\pgfusepath{clip}%
\pgfsetbuttcap%
\pgfsetroundjoin%
\definecolor{currentfill}{rgb}{0.895527,0.810432,0.817172}%
\pgfsetfillcolor{currentfill}%
\pgfsetlinewidth{0.000000pt}%
\definecolor{currentstroke}{rgb}{0.000000,0.000000,0.000000}%
\pgfsetstrokecolor{currentstroke}%
\pgfsetdash{}{0pt}%
\pgfpathmoveto{\pgfqpoint{1.270317in}{0.824944in}}%
\pgfpathlineto{\pgfqpoint{1.308761in}{0.844927in}}%
\pgfpathlineto{\pgfqpoint{1.271091in}{0.872713in}}%
\pgfpathlineto{\pgfqpoint{1.232619in}{0.852803in}}%
\pgfpathclose%
\pgfusepath{fill}%
\end{pgfscope}%
\begin{pgfscope}%
\pgfpathrectangle{\pgfqpoint{0.150000in}{0.150000in}}{\pgfqpoint{2.700000in}{1.950000in}}%
\pgfusepath{clip}%
\pgfsetbuttcap%
\pgfsetroundjoin%
\definecolor{currentfill}{rgb}{0.779213,0.806403,0.844470}%
\pgfsetfillcolor{currentfill}%
\pgfsetlinewidth{0.000000pt}%
\definecolor{currentstroke}{rgb}{0.000000,0.000000,0.000000}%
\pgfsetstrokecolor{currentstroke}%
\pgfsetdash{}{0pt}%
\pgfpathmoveto{\pgfqpoint{1.269432in}{1.326522in}}%
\pgfpathlineto{\pgfqpoint{1.309580in}{1.271619in}}%
\pgfpathlineto{\pgfqpoint{1.272044in}{1.298851in}}%
\pgfpathlineto{\pgfqpoint{1.231373in}{1.362236in}}%
\pgfpathclose%
\pgfusepath{fill}%
\end{pgfscope}%
\begin{pgfscope}%
\pgfpathrectangle{\pgfqpoint{0.150000in}{0.150000in}}{\pgfqpoint{2.700000in}{1.950000in}}%
\pgfusepath{clip}%
\pgfsetbuttcap%
\pgfsetroundjoin%
\definecolor{currentfill}{rgb}{0.914522,0.844899,0.850414}%
\pgfsetfillcolor{currentfill}%
\pgfsetlinewidth{0.000000pt}%
\definecolor{currentstroke}{rgb}{0.000000,0.000000,0.000000}%
\pgfsetstrokecolor{currentstroke}%
\pgfsetdash{}{0pt}%
\pgfpathmoveto{\pgfqpoint{1.308761in}{0.844927in}}%
\pgfpathlineto{\pgfqpoint{1.346334in}{0.904342in}}%
\pgfpathlineto{\pgfqpoint{1.308548in}{0.932121in}}%
\pgfpathlineto{\pgfqpoint{1.271091in}{0.872713in}}%
\pgfpathclose%
\pgfusepath{fill}%
\end{pgfscope}%
\begin{pgfscope}%
\pgfpathrectangle{\pgfqpoint{0.150000in}{0.150000in}}{\pgfqpoint{2.700000in}{1.950000in}}%
\pgfusepath{clip}%
\pgfsetbuttcap%
\pgfsetroundjoin%
\definecolor{currentfill}{rgb}{0.899326,0.817325,0.823820}%
\pgfsetfillcolor{currentfill}%
\pgfsetlinewidth{0.000000pt}%
\definecolor{currentstroke}{rgb}{0.000000,0.000000,0.000000}%
\pgfsetstrokecolor{currentstroke}%
\pgfsetdash{}{0pt}%
\pgfpathmoveto{\pgfqpoint{2.031182in}{0.832819in}}%
\pgfpathlineto{\pgfqpoint{2.068255in}{0.852803in}}%
\pgfpathlineto{\pgfqpoint{2.029365in}{0.872713in}}%
\pgfpathlineto{\pgfqpoint{1.992288in}{0.852803in}}%
\pgfpathclose%
\pgfusepath{fill}%
\end{pgfscope}%
\begin{pgfscope}%
\pgfpathrectangle{\pgfqpoint{0.150000in}{0.150000in}}{\pgfqpoint{2.700000in}{1.950000in}}%
\pgfusepath{clip}%
\pgfsetbuttcap%
\pgfsetroundjoin%
\definecolor{currentfill}{rgb}{0.899326,0.817325,0.823820}%
\pgfsetfillcolor{currentfill}%
\pgfsetlinewidth{0.000000pt}%
\definecolor{currentstroke}{rgb}{0.000000,0.000000,0.000000}%
\pgfsetstrokecolor{currentstroke}%
\pgfsetdash{}{0pt}%
\pgfpathmoveto{\pgfqpoint{1.194005in}{0.832819in}}%
\pgfpathlineto{\pgfqpoint{1.232619in}{0.852803in}}%
\pgfpathlineto{\pgfqpoint{1.195263in}{0.872713in}}%
\pgfpathlineto{\pgfqpoint{1.156652in}{0.852803in}}%
\pgfpathclose%
\pgfusepath{fill}%
\end{pgfscope}%
\begin{pgfscope}%
\pgfpathrectangle{\pgfqpoint{0.150000in}{0.150000in}}{\pgfqpoint{2.700000in}{1.950000in}}%
\pgfusepath{clip}%
\pgfsetbuttcap%
\pgfsetroundjoin%
\definecolor{currentfill}{rgb}{0.899326,0.817325,0.823820}%
\pgfsetfillcolor{currentfill}%
\pgfsetlinewidth{0.000000pt}%
\definecolor{currentstroke}{rgb}{0.000000,0.000000,0.000000}%
\pgfsetstrokecolor{currentstroke}%
\pgfsetdash{}{0pt}%
\pgfpathmoveto{\pgfqpoint{1.117898in}{0.832819in}}%
\pgfpathlineto{\pgfqpoint{1.156652in}{0.852803in}}%
\pgfpathlineto{\pgfqpoint{1.119436in}{0.872713in}}%
\pgfpathlineto{\pgfqpoint{1.080685in}{0.852803in}}%
\pgfpathclose%
\pgfusepath{fill}%
\end{pgfscope}%
\begin{pgfscope}%
\pgfpathrectangle{\pgfqpoint{0.150000in}{0.150000in}}{\pgfqpoint{2.700000in}{1.950000in}}%
\pgfusepath{clip}%
\pgfsetbuttcap%
\pgfsetroundjoin%
\definecolor{currentfill}{rgb}{0.899326,0.817325,0.823820}%
\pgfsetfillcolor{currentfill}%
\pgfsetlinewidth{0.000000pt}%
\definecolor{currentstroke}{rgb}{0.000000,0.000000,0.000000}%
\pgfsetstrokecolor{currentstroke}%
\pgfsetdash{}{0pt}%
\pgfpathmoveto{\pgfqpoint{1.041791in}{0.832819in}}%
\pgfpathlineto{\pgfqpoint{1.080685in}{0.852803in}}%
\pgfpathlineto{\pgfqpoint{1.043608in}{0.872713in}}%
\pgfpathlineto{\pgfqpoint{1.004718in}{0.852803in}}%
\pgfpathclose%
\pgfusepath{fill}%
\end{pgfscope}%
\begin{pgfscope}%
\pgfpathrectangle{\pgfqpoint{0.150000in}{0.150000in}}{\pgfqpoint{2.700000in}{1.950000in}}%
\pgfusepath{clip}%
\pgfsetbuttcap%
\pgfsetroundjoin%
\definecolor{currentfill}{rgb}{0.797871,0.822763,0.857613}%
\pgfsetfillcolor{currentfill}%
\pgfsetlinewidth{0.000000pt}%
\definecolor{currentstroke}{rgb}{0.000000,0.000000,0.000000}%
\pgfsetstrokecolor{currentstroke}%
\pgfsetdash{}{0pt}%
\pgfpathmoveto{\pgfqpoint{1.307513in}{1.290786in}}%
\pgfpathlineto{\pgfqpoint{1.347341in}{1.236179in}}%
\pgfpathlineto{\pgfqpoint{1.309580in}{1.271619in}}%
\pgfpathlineto{\pgfqpoint{1.269432in}{1.326522in}}%
\pgfpathclose%
\pgfusepath{fill}%
\end{pgfscope}%
\begin{pgfscope}%
\pgfpathrectangle{\pgfqpoint{0.150000in}{0.150000in}}{\pgfqpoint{2.700000in}{1.950000in}}%
\pgfusepath{clip}%
\pgfsetbuttcap%
\pgfsetroundjoin%
\definecolor{currentfill}{rgb}{0.866284,0.882751,0.905806}%
\pgfsetfillcolor{currentfill}%
\pgfsetlinewidth{0.000000pt}%
\definecolor{currentstroke}{rgb}{0.000000,0.000000,0.000000}%
\pgfsetstrokecolor{currentstroke}%
\pgfsetdash{}{0pt}%
\pgfpathmoveto{\pgfqpoint{1.157565in}{1.137806in}}%
\pgfpathlineto{\pgfqpoint{1.193453in}{1.238589in}}%
\pgfpathlineto{\pgfqpoint{1.155747in}{1.266070in}}%
\pgfpathlineto{\pgfqpoint{1.119798in}{1.173335in}}%
\pgfpathclose%
\pgfusepath{fill}%
\end{pgfscope}%
\begin{pgfscope}%
\pgfpathrectangle{\pgfqpoint{0.150000in}{0.150000in}}{\pgfqpoint{2.700000in}{1.950000in}}%
\pgfusepath{clip}%
\pgfsetbuttcap%
\pgfsetroundjoin%
\definecolor{currentfill}{rgb}{0.828968,0.850031,0.879519}%
\pgfsetfillcolor{currentfill}%
\pgfsetlinewidth{0.000000pt}%
\definecolor{currentstroke}{rgb}{0.000000,0.000000,0.000000}%
\pgfsetstrokecolor{currentstroke}%
\pgfsetdash{}{0pt}%
\pgfpathmoveto{\pgfqpoint{1.345765in}{1.246802in}}%
\pgfpathlineto{\pgfqpoint{1.385125in}{1.200717in}}%
\pgfpathlineto{\pgfqpoint{1.347341in}{1.236179in}}%
\pgfpathlineto{\pgfqpoint{1.307513in}{1.290786in}}%
\pgfpathclose%
\pgfusepath{fill}%
\end{pgfscope}%
\begin{pgfscope}%
\pgfpathrectangle{\pgfqpoint{0.150000in}{0.150000in}}{\pgfqpoint{2.700000in}{1.950000in}}%
\pgfusepath{clip}%
\pgfsetbuttcap%
\pgfsetroundjoin%
\definecolor{currentfill}{rgb}{0.860064,0.877298,0.901425}%
\pgfsetfillcolor{currentfill}%
\pgfsetlinewidth{0.000000pt}%
\definecolor{currentstroke}{rgb}{0.000000,0.000000,0.000000}%
\pgfsetstrokecolor{currentstroke}%
\pgfsetdash{}{0pt}%
\pgfpathmoveto{\pgfqpoint{1.383863in}{1.211048in}}%
\pgfpathlineto{\pgfqpoint{1.422844in}{1.173335in}}%
\pgfpathlineto{\pgfqpoint{1.385125in}{1.200717in}}%
\pgfpathlineto{\pgfqpoint{1.345765in}{1.246802in}}%
\pgfpathclose%
\pgfusepath{fill}%
\end{pgfscope}%
\begin{pgfscope}%
\pgfpathrectangle{\pgfqpoint{0.150000in}{0.150000in}}{\pgfqpoint{2.700000in}{1.950000in}}%
\pgfusepath{clip}%
\pgfsetbuttcap%
\pgfsetroundjoin%
\definecolor{currentfill}{rgb}{0.906924,0.831112,0.837117}%
\pgfsetfillcolor{currentfill}%
\pgfsetlinewidth{0.000000pt}%
\definecolor{currentstroke}{rgb}{0.000000,0.000000,0.000000}%
\pgfsetstrokecolor{currentstroke}%
\pgfsetdash{}{0pt}%
\pgfpathmoveto{\pgfqpoint{1.956045in}{0.856517in}}%
\pgfpathlineto{\pgfqpoint{1.992288in}{0.852803in}}%
\pgfpathlineto{\pgfqpoint{1.953217in}{0.864837in}}%
\pgfpathlineto{\pgfqpoint{1.916907in}{0.868591in}}%
\pgfpathclose%
\pgfusepath{fill}%
\end{pgfscope}%
\begin{pgfscope}%
\pgfpathrectangle{\pgfqpoint{0.150000in}{0.150000in}}{\pgfqpoint{2.700000in}{1.950000in}}%
\pgfusepath{clip}%
\pgfsetbuttcap%
\pgfsetroundjoin%
\definecolor{currentfill}{rgb}{0.967708,0.941406,0.943490}%
\pgfsetfillcolor{currentfill}%
\pgfsetlinewidth{0.000000pt}%
\definecolor{currentstroke}{rgb}{0.000000,0.000000,0.000000}%
\pgfsetstrokecolor{currentstroke}%
\pgfsetdash{}{0pt}%
\pgfpathmoveto{\pgfqpoint{1.308548in}{0.932121in}}%
\pgfpathlineto{\pgfqpoint{1.346008in}{0.999779in}}%
\pgfpathlineto{\pgfqpoint{1.308158in}{1.027503in}}%
\pgfpathlineto{\pgfqpoint{1.271046in}{0.951889in}}%
\pgfpathclose%
\pgfusepath{fill}%
\end{pgfscope}%
\begin{pgfscope}%
\pgfpathrectangle{\pgfqpoint{0.150000in}{0.150000in}}{\pgfqpoint{2.700000in}{1.950000in}}%
\pgfusepath{clip}%
\pgfsetbuttcap%
\pgfsetroundjoin%
\definecolor{currentfill}{rgb}{0.884942,0.899112,0.918949}%
\pgfsetfillcolor{currentfill}%
\pgfsetlinewidth{0.000000pt}%
\definecolor{currentstroke}{rgb}{0.000000,0.000000,0.000000}%
\pgfsetstrokecolor{currentstroke}%
\pgfsetdash{}{0pt}%
\pgfpathmoveto{\pgfqpoint{1.195092in}{1.110318in}}%
\pgfpathlineto{\pgfqpoint{1.231476in}{1.202860in}}%
\pgfpathlineto{\pgfqpoint{1.193453in}{1.238589in}}%
\pgfpathlineto{\pgfqpoint{1.157565in}{1.137806in}}%
\pgfpathclose%
\pgfusepath{fill}%
\end{pgfscope}%
\begin{pgfscope}%
\pgfpathrectangle{\pgfqpoint{0.150000in}{0.150000in}}{\pgfqpoint{2.700000in}{1.950000in}}%
\pgfusepath{clip}%
\pgfsetbuttcap%
\pgfsetroundjoin%
\definecolor{currentfill}{rgb}{0.887929,0.796645,0.803876}%
\pgfsetfillcolor{currentfill}%
\pgfsetlinewidth{0.000000pt}%
\definecolor{currentstroke}{rgb}{0.000000,0.000000,0.000000}%
\pgfsetstrokecolor{currentstroke}%
\pgfsetdash{}{0pt}%
\pgfpathmoveto{\pgfqpoint{1.308097in}{0.797025in}}%
\pgfpathlineto{\pgfqpoint{1.346366in}{0.824944in}}%
\pgfpathlineto{\pgfqpoint{1.308761in}{0.844927in}}%
\pgfpathlineto{\pgfqpoint{1.270317in}{0.824944in}}%
\pgfpathclose%
\pgfusepath{fill}%
\end{pgfscope}%
\begin{pgfscope}%
\pgfpathrectangle{\pgfqpoint{0.150000in}{0.150000in}}{\pgfqpoint{2.700000in}{1.950000in}}%
\pgfusepath{clip}%
\pgfsetbuttcap%
\pgfsetroundjoin%
\definecolor{currentfill}{rgb}{0.910723,0.838006,0.843765}%
\pgfsetfillcolor{currentfill}%
\pgfsetlinewidth{0.000000pt}%
\definecolor{currentstroke}{rgb}{0.000000,0.000000,0.000000}%
\pgfsetstrokecolor{currentstroke}%
\pgfsetdash{}{0pt}%
\pgfpathmoveto{\pgfqpoint{1.346366in}{0.824944in}}%
\pgfpathlineto{\pgfqpoint{1.384201in}{0.876504in}}%
\pgfpathlineto{\pgfqpoint{1.346334in}{0.904342in}}%
\pgfpathlineto{\pgfqpoint{1.308761in}{0.844927in}}%
\pgfpathclose%
\pgfusepath{fill}%
\end{pgfscope}%
\begin{pgfscope}%
\pgfpathrectangle{\pgfqpoint{0.150000in}{0.150000in}}{\pgfqpoint{2.700000in}{1.950000in}}%
\pgfusepath{clip}%
\pgfsetbuttcap%
\pgfsetroundjoin%
\definecolor{currentfill}{rgb}{0.884942,0.899112,0.918949}%
\pgfsetfillcolor{currentfill}%
\pgfsetlinewidth{0.000000pt}%
\definecolor{currentstroke}{rgb}{0.000000,0.000000,0.000000}%
\pgfsetstrokecolor{currentstroke}%
\pgfsetdash{}{0pt}%
\pgfpathmoveto{\pgfqpoint{1.421984in}{1.175273in}}%
\pgfpathlineto{\pgfqpoint{1.460702in}{1.137806in}}%
\pgfpathlineto{\pgfqpoint{1.422844in}{1.173335in}}%
\pgfpathlineto{\pgfqpoint{1.383863in}{1.211048in}}%
\pgfpathclose%
\pgfusepath{fill}%
\end{pgfscope}%
\begin{pgfscope}%
\pgfpathrectangle{\pgfqpoint{0.150000in}{0.150000in}}{\pgfqpoint{2.700000in}{1.950000in}}%
\pgfusepath{clip}%
\pgfsetbuttcap%
\pgfsetroundjoin%
\definecolor{currentfill}{rgb}{0.952512,0.913833,0.916896}%
\pgfsetfillcolor{currentfill}%
\pgfsetlinewidth{0.000000pt}%
\definecolor{currentstroke}{rgb}{0.000000,0.000000,0.000000}%
\pgfsetstrokecolor{currentstroke}%
\pgfsetdash{}{0pt}%
\pgfpathmoveto{\pgfqpoint{1.727375in}{0.944153in}}%
\pgfpathlineto{\pgfqpoint{1.764425in}{0.932121in}}%
\pgfpathlineto{\pgfqpoint{1.725941in}{0.943950in}}%
\pgfpathlineto{\pgfqpoint{1.688680in}{0.948034in}}%
\pgfpathclose%
\pgfusepath{fill}%
\end{pgfscope}%
\begin{pgfscope}%
\pgfpathrectangle{\pgfqpoint{0.150000in}{0.150000in}}{\pgfqpoint{2.700000in}{1.950000in}}%
\pgfusepath{clip}%
\pgfsetbuttcap%
\pgfsetroundjoin%
\definecolor{currentfill}{rgb}{0.891728,0.803539,0.810524}%
\pgfsetfillcolor{currentfill}%
\pgfsetlinewidth{0.000000pt}%
\definecolor{currentstroke}{rgb}{0.000000,0.000000,0.000000}%
\pgfsetstrokecolor{currentstroke}%
\pgfsetdash{}{0pt}%
\pgfpathmoveto{\pgfqpoint{1.231967in}{0.797025in}}%
\pgfpathlineto{\pgfqpoint{1.270317in}{0.824944in}}%
\pgfpathlineto{\pgfqpoint{1.232619in}{0.852803in}}%
\pgfpathlineto{\pgfqpoint{1.194005in}{0.832819in}}%
\pgfpathclose%
\pgfusepath{fill}%
\end{pgfscope}%
\begin{pgfscope}%
\pgfpathrectangle{\pgfqpoint{0.150000in}{0.150000in}}{\pgfqpoint{2.700000in}{1.950000in}}%
\pgfusepath{clip}%
\pgfsetbuttcap%
\pgfsetroundjoin%
\definecolor{currentfill}{rgb}{0.933517,0.879366,0.883655}%
\pgfsetfillcolor{currentfill}%
\pgfsetlinewidth{0.000000pt}%
\definecolor{currentstroke}{rgb}{0.000000,0.000000,0.000000}%
\pgfsetstrokecolor{currentstroke}%
\pgfsetdash{}{0pt}%
\pgfpathmoveto{\pgfqpoint{1.842000in}{0.908276in}}%
\pgfpathlineto{\pgfqpoint{1.878235in}{0.888503in}}%
\pgfpathlineto{\pgfqpoint{1.839705in}{0.908342in}}%
\pgfpathlineto{\pgfqpoint{1.803111in}{0.920229in}}%
\pgfpathclose%
\pgfusepath{fill}%
\end{pgfscope}%
\begin{pgfscope}%
\pgfpathrectangle{\pgfqpoint{0.150000in}{0.150000in}}{\pgfqpoint{2.700000in}{1.950000in}}%
\pgfusepath{clip}%
\pgfsetbuttcap%
\pgfsetroundjoin%
\definecolor{currentfill}{rgb}{0.909819,0.920925,0.936474}%
\pgfsetfillcolor{currentfill}%
\pgfsetlinewidth{0.000000pt}%
\definecolor{currentstroke}{rgb}{0.000000,0.000000,0.000000}%
\pgfsetstrokecolor{currentstroke}%
\pgfsetdash{}{0pt}%
\pgfpathmoveto{\pgfqpoint{1.232700in}{1.082772in}}%
\pgfpathlineto{\pgfqpoint{1.269521in}{1.167110in}}%
\pgfpathlineto{\pgfqpoint{1.231476in}{1.202860in}}%
\pgfpathlineto{\pgfqpoint{1.195092in}{1.110318in}}%
\pgfpathclose%
\pgfusepath{fill}%
\end{pgfscope}%
\begin{pgfscope}%
\pgfpathrectangle{\pgfqpoint{0.150000in}{0.150000in}}{\pgfqpoint{2.700000in}{1.950000in}}%
\pgfusepath{clip}%
\pgfsetbuttcap%
\pgfsetroundjoin%
\definecolor{currentfill}{rgb}{0.909819,0.920925,0.936474}%
\pgfsetfillcolor{currentfill}%
\pgfsetlinewidth{0.000000pt}%
\definecolor{currentstroke}{rgb}{0.000000,0.000000,0.000000}%
\pgfsetstrokecolor{currentstroke}%
\pgfsetdash{}{0pt}%
\pgfpathmoveto{\pgfqpoint{1.460129in}{1.139476in}}%
\pgfpathlineto{\pgfqpoint{1.498554in}{1.110318in}}%
\pgfpathlineto{\pgfqpoint{1.460702in}{1.137806in}}%
\pgfpathlineto{\pgfqpoint{1.421984in}{1.175273in}}%
\pgfpathclose%
\pgfusepath{fill}%
\end{pgfscope}%
\begin{pgfscope}%
\pgfpathrectangle{\pgfqpoint{0.150000in}{0.150000in}}{\pgfqpoint{2.700000in}{1.950000in}}%
\pgfusepath{clip}%
\pgfsetbuttcap%
\pgfsetroundjoin%
\definecolor{currentfill}{rgb}{0.899326,0.817325,0.823820}%
\pgfsetfillcolor{currentfill}%
\pgfsetlinewidth{0.000000pt}%
\definecolor{currentstroke}{rgb}{0.000000,0.000000,0.000000}%
\pgfsetstrokecolor{currentstroke}%
\pgfsetdash{}{0pt}%
\pgfpathmoveto{\pgfqpoint{1.155249in}{0.812761in}}%
\pgfpathlineto{\pgfqpoint{1.194005in}{0.832819in}}%
\pgfpathlineto{\pgfqpoint{1.156652in}{0.852803in}}%
\pgfpathlineto{\pgfqpoint{1.117898in}{0.832819in}}%
\pgfpathclose%
\pgfusepath{fill}%
\end{pgfscope}%
\begin{pgfscope}%
\pgfpathrectangle{\pgfqpoint{0.150000in}{0.150000in}}{\pgfqpoint{2.700000in}{1.950000in}}%
\pgfusepath{clip}%
\pgfsetbuttcap%
\pgfsetroundjoin%
\definecolor{currentfill}{rgb}{0.899326,0.817325,0.823820}%
\pgfsetfillcolor{currentfill}%
\pgfsetlinewidth{0.000000pt}%
\definecolor{currentstroke}{rgb}{0.000000,0.000000,0.000000}%
\pgfsetstrokecolor{currentstroke}%
\pgfsetdash{}{0pt}%
\pgfpathmoveto{\pgfqpoint{1.079001in}{0.812761in}}%
\pgfpathlineto{\pgfqpoint{1.117898in}{0.832819in}}%
\pgfpathlineto{\pgfqpoint{1.080685in}{0.852803in}}%
\pgfpathlineto{\pgfqpoint{1.041791in}{0.832819in}}%
\pgfpathclose%
\pgfusepath{fill}%
\end{pgfscope}%
\begin{pgfscope}%
\pgfpathrectangle{\pgfqpoint{0.150000in}{0.150000in}}{\pgfqpoint{2.700000in}{1.950000in}}%
\pgfusepath{clip}%
\pgfsetbuttcap%
\pgfsetroundjoin%
\definecolor{currentfill}{rgb}{0.934697,0.942739,0.953998}%
\pgfsetfillcolor{currentfill}%
\pgfsetlinewidth{0.000000pt}%
\definecolor{currentstroke}{rgb}{0.000000,0.000000,0.000000}%
\pgfsetstrokecolor{currentstroke}%
\pgfsetdash{}{0pt}%
\pgfpathmoveto{\pgfqpoint{1.498326in}{1.095544in}}%
\pgfpathlineto{\pgfqpoint{1.536486in}{1.074722in}}%
\pgfpathlineto{\pgfqpoint{1.498554in}{1.110318in}}%
\pgfpathlineto{\pgfqpoint{1.460129in}{1.139476in}}%
\pgfpathclose%
\pgfusepath{fill}%
\end{pgfscope}%
\begin{pgfscope}%
\pgfpathrectangle{\pgfqpoint{0.150000in}{0.150000in}}{\pgfqpoint{2.700000in}{1.950000in}}%
\pgfusepath{clip}%
\pgfsetbuttcap%
\pgfsetroundjoin%
\definecolor{currentfill}{rgb}{0.960110,0.927619,0.930193}%
\pgfsetfillcolor{currentfill}%
\pgfsetlinewidth{0.000000pt}%
\definecolor{currentstroke}{rgb}{0.000000,0.000000,0.000000}%
\pgfsetstrokecolor{currentstroke}%
\pgfsetdash{}{0pt}%
\pgfpathmoveto{\pgfqpoint{1.346334in}{0.904342in}}%
\pgfpathlineto{\pgfqpoint{1.384058in}{0.963996in}}%
\pgfpathlineto{\pgfqpoint{1.346008in}{0.999779in}}%
\pgfpathlineto{\pgfqpoint{1.308548in}{0.932121in}}%
\pgfpathclose%
\pgfusepath{fill}%
\end{pgfscope}%
\begin{pgfscope}%
\pgfpathrectangle{\pgfqpoint{0.150000in}{0.150000in}}{\pgfqpoint{2.700000in}{1.950000in}}%
\pgfusepath{clip}%
\pgfsetbuttcap%
\pgfsetroundjoin%
\definecolor{currentfill}{rgb}{0.965794,0.970006,0.975904}%
\pgfsetfillcolor{currentfill}%
\pgfsetlinewidth{0.000000pt}%
\definecolor{currentstroke}{rgb}{0.000000,0.000000,0.000000}%
\pgfsetstrokecolor{currentstroke}%
\pgfsetdash{}{0pt}%
\pgfpathmoveto{\pgfqpoint{1.536486in}{1.059728in}}%
\pgfpathlineto{\pgfqpoint{1.574442in}{1.039104in}}%
\pgfpathlineto{\pgfqpoint{1.536486in}{1.074722in}}%
\pgfpathlineto{\pgfqpoint{1.498326in}{1.095544in}}%
\pgfpathclose%
\pgfusepath{fill}%
\end{pgfscope}%
\begin{pgfscope}%
\pgfpathrectangle{\pgfqpoint{0.150000in}{0.150000in}}{\pgfqpoint{2.700000in}{1.950000in}}%
\pgfusepath{clip}%
\pgfsetbuttcap%
\pgfsetroundjoin%
\definecolor{currentfill}{rgb}{0.884130,0.789752,0.797227}%
\pgfsetfillcolor{currentfill}%
\pgfsetlinewidth{0.000000pt}%
\definecolor{currentstroke}{rgb}{0.000000,0.000000,0.000000}%
\pgfsetstrokecolor{currentstroke}%
\pgfsetdash{}{0pt}%
\pgfpathmoveto{\pgfqpoint{1.269745in}{0.769047in}}%
\pgfpathlineto{\pgfqpoint{1.308097in}{0.797025in}}%
\pgfpathlineto{\pgfqpoint{1.270317in}{0.824944in}}%
\pgfpathlineto{\pgfqpoint{1.231967in}{0.797025in}}%
\pgfpathclose%
\pgfusepath{fill}%
\end{pgfscope}%
\begin{pgfscope}%
\pgfpathrectangle{\pgfqpoint{0.150000in}{0.150000in}}{\pgfqpoint{2.700000in}{1.950000in}}%
\pgfusepath{clip}%
\pgfsetbuttcap%
\pgfsetroundjoin%
\definecolor{currentfill}{rgb}{0.928477,0.937286,0.949617}%
\pgfsetfillcolor{currentfill}%
\pgfsetlinewidth{0.000000pt}%
\definecolor{currentstroke}{rgb}{0.000000,0.000000,0.000000}%
\pgfsetstrokecolor{currentstroke}%
\pgfsetdash{}{0pt}%
\pgfpathmoveto{\pgfqpoint{1.270388in}{1.055167in}}%
\pgfpathlineto{\pgfqpoint{1.307590in}{1.131337in}}%
\pgfpathlineto{\pgfqpoint{1.269521in}{1.167110in}}%
\pgfpathlineto{\pgfqpoint{1.232700in}{1.082772in}}%
\pgfpathclose%
\pgfusepath{fill}%
\end{pgfscope}%
\begin{pgfscope}%
\pgfpathrectangle{\pgfqpoint{0.150000in}{0.150000in}}{\pgfqpoint{2.700000in}{1.950000in}}%
\pgfusepath{clip}%
\pgfsetbuttcap%
\pgfsetroundjoin%
\definecolor{currentfill}{rgb}{0.884130,0.789752,0.797227}%
\pgfsetfillcolor{currentfill}%
\pgfsetlinewidth{0.000000pt}%
\definecolor{currentstroke}{rgb}{0.000000,0.000000,0.000000}%
\pgfsetstrokecolor{currentstroke}%
\pgfsetdash{}{0pt}%
\pgfpathmoveto{\pgfqpoint{1.345957in}{0.769047in}}%
\pgfpathlineto{\pgfqpoint{1.384109in}{0.804887in}}%
\pgfpathlineto{\pgfqpoint{1.346366in}{0.824944in}}%
\pgfpathlineto{\pgfqpoint{1.308097in}{0.797025in}}%
\pgfpathclose%
\pgfusepath{fill}%
\end{pgfscope}%
\begin{pgfscope}%
\pgfpathrectangle{\pgfqpoint{0.150000in}{0.150000in}}{\pgfqpoint{2.700000in}{1.950000in}}%
\pgfusepath{clip}%
\pgfsetbuttcap%
\pgfsetroundjoin%
\definecolor{currentfill}{rgb}{0.990671,0.991820,0.993428}%
\pgfsetfillcolor{currentfill}%
\pgfsetlinewidth{0.000000pt}%
\definecolor{currentstroke}{rgb}{0.000000,0.000000,0.000000}%
\pgfsetstrokecolor{currentstroke}%
\pgfsetdash{}{0pt}%
\pgfpathmoveto{\pgfqpoint{1.574671in}{1.023891in}}%
\pgfpathlineto{\pgfqpoint{1.612479in}{1.011465in}}%
\pgfpathlineto{\pgfqpoint{1.574442in}{1.039104in}}%
\pgfpathlineto{\pgfqpoint{1.536486in}{1.059728in}}%
\pgfpathclose%
\pgfusepath{fill}%
\end{pgfscope}%
\begin{pgfscope}%
\pgfpathrectangle{\pgfqpoint{0.150000in}{0.150000in}}{\pgfqpoint{2.700000in}{1.950000in}}%
\pgfusepath{clip}%
\pgfsetbuttcap%
\pgfsetroundjoin%
\definecolor{currentfill}{rgb}{0.906924,0.831112,0.837117}%
\pgfsetfillcolor{currentfill}%
\pgfsetlinewidth{0.000000pt}%
\definecolor{currentstroke}{rgb}{0.000000,0.000000,0.000000}%
\pgfsetstrokecolor{currentstroke}%
\pgfsetdash{}{0pt}%
\pgfpathmoveto{\pgfqpoint{1.384109in}{0.804887in}}%
\pgfpathlineto{\pgfqpoint{1.422061in}{0.856517in}}%
\pgfpathlineto{\pgfqpoint{1.384201in}{0.876504in}}%
\pgfpathlineto{\pgfqpoint{1.346366in}{0.824944in}}%
\pgfpathclose%
\pgfusepath{fill}%
\end{pgfscope}%
\begin{pgfscope}%
\pgfpathrectangle{\pgfqpoint{0.150000in}{0.150000in}}{\pgfqpoint{2.700000in}{1.950000in}}%
\pgfusepath{clip}%
\pgfsetbuttcap%
\pgfsetroundjoin%
\definecolor{currentfill}{rgb}{0.948713,0.906939,0.910248}%
\pgfsetfillcolor{currentfill}%
\pgfsetlinewidth{0.000000pt}%
\definecolor{currentstroke}{rgb}{0.000000,0.000000,0.000000}%
\pgfsetstrokecolor{currentstroke}%
\pgfsetdash{}{0pt}%
\pgfpathmoveto{\pgfqpoint{1.384201in}{0.876504in}}%
\pgfpathlineto{\pgfqpoint{1.422042in}{0.936166in}}%
\pgfpathlineto{\pgfqpoint{1.384058in}{0.963996in}}%
\pgfpathlineto{\pgfqpoint{1.346334in}{0.904342in}}%
\pgfpathclose%
\pgfusepath{fill}%
\end{pgfscope}%
\begin{pgfscope}%
\pgfpathrectangle{\pgfqpoint{0.150000in}{0.150000in}}{\pgfqpoint{2.700000in}{1.950000in}}%
\pgfusepath{clip}%
\pgfsetbuttcap%
\pgfsetroundjoin%
\definecolor{currentfill}{rgb}{0.891728,0.803539,0.810524}%
\pgfsetfillcolor{currentfill}%
\pgfsetlinewidth{0.000000pt}%
\definecolor{currentstroke}{rgb}{0.000000,0.000000,0.000000}%
\pgfsetstrokecolor{currentstroke}%
\pgfsetdash{}{0pt}%
\pgfpathmoveto{\pgfqpoint{1.193269in}{0.776895in}}%
\pgfpathlineto{\pgfqpoint{1.231967in}{0.797025in}}%
\pgfpathlineto{\pgfqpoint{1.194005in}{0.832819in}}%
\pgfpathlineto{\pgfqpoint{1.155249in}{0.812761in}}%
\pgfpathclose%
\pgfusepath{fill}%
\end{pgfscope}%
\begin{pgfscope}%
\pgfpathrectangle{\pgfqpoint{0.150000in}{0.150000in}}{\pgfqpoint{2.700000in}{1.950000in}}%
\pgfusepath{clip}%
\pgfsetbuttcap%
\pgfsetroundjoin%
\definecolor{currentfill}{rgb}{0.947135,0.953646,0.962760}%
\pgfsetfillcolor{currentfill}%
\pgfsetlinewidth{0.000000pt}%
\definecolor{currentstroke}{rgb}{0.000000,0.000000,0.000000}%
\pgfsetstrokecolor{currentstroke}%
\pgfsetdash{}{0pt}%
\pgfpathmoveto{\pgfqpoint{1.308158in}{1.027503in}}%
\pgfpathlineto{\pgfqpoint{1.345682in}{1.095544in}}%
\pgfpathlineto{\pgfqpoint{1.307590in}{1.131337in}}%
\pgfpathlineto{\pgfqpoint{1.270388in}{1.055167in}}%
\pgfpathclose%
\pgfusepath{fill}%
\end{pgfscope}%
\begin{pgfscope}%
\pgfpathrectangle{\pgfqpoint{0.150000in}{0.150000in}}{\pgfqpoint{2.700000in}{1.950000in}}%
\pgfusepath{clip}%
\pgfsetbuttcap%
\pgfsetroundjoin%
\definecolor{currentfill}{rgb}{0.990502,0.982767,0.983379}%
\pgfsetfillcolor{currentfill}%
\pgfsetlinewidth{0.000000pt}%
\definecolor{currentstroke}{rgb}{0.000000,0.000000,0.000000}%
\pgfsetstrokecolor{currentstroke}%
\pgfsetdash{}{0pt}%
\pgfpathmoveto{\pgfqpoint{1.612878in}{0.988032in}}%
\pgfpathlineto{\pgfqpoint{1.650509in}{0.975779in}}%
\pgfpathlineto{\pgfqpoint{1.612479in}{1.011465in}}%
\pgfpathlineto{\pgfqpoint{1.574671in}{1.023891in}}%
\pgfpathclose%
\pgfusepath{fill}%
\end{pgfscope}%
\begin{pgfscope}%
\pgfpathrectangle{\pgfqpoint{0.150000in}{0.150000in}}{\pgfqpoint{2.700000in}{1.950000in}}%
\pgfusepath{clip}%
\pgfsetbuttcap%
\pgfsetroundjoin%
\definecolor{currentfill}{rgb}{0.910723,0.838006,0.843765}%
\pgfsetfillcolor{currentfill}%
\pgfsetlinewidth{0.000000pt}%
\definecolor{currentstroke}{rgb}{0.000000,0.000000,0.000000}%
\pgfsetstrokecolor{currentstroke}%
\pgfsetdash{}{0pt}%
\pgfpathmoveto{\pgfqpoint{1.995035in}{0.836457in}}%
\pgfpathlineto{\pgfqpoint{2.031182in}{0.832819in}}%
\pgfpathlineto{\pgfqpoint{1.992288in}{0.852803in}}%
\pgfpathlineto{\pgfqpoint{1.956045in}{0.856517in}}%
\pgfpathclose%
\pgfusepath{fill}%
\end{pgfscope}%
\begin{pgfscope}%
\pgfpathrectangle{\pgfqpoint{0.150000in}{0.150000in}}{\pgfqpoint{2.700000in}{1.950000in}}%
\pgfusepath{clip}%
\pgfsetbuttcap%
\pgfsetroundjoin%
\definecolor{currentfill}{rgb}{0.872733,0.769072,0.777282}%
\pgfsetfillcolor{currentfill}%
\pgfsetlinewidth{0.000000pt}%
\definecolor{currentstroke}{rgb}{0.000000,0.000000,0.000000}%
\pgfsetstrokecolor{currentstroke}%
\pgfsetdash{}{0pt}%
\pgfpathmoveto{\pgfqpoint{1.307782in}{0.733184in}}%
\pgfpathlineto{\pgfqpoint{1.345957in}{0.769047in}}%
\pgfpathlineto{\pgfqpoint{1.308097in}{0.797025in}}%
\pgfpathlineto{\pgfqpoint{1.269745in}{0.769047in}}%
\pgfpathclose%
\pgfusepath{fill}%
\end{pgfscope}%
\begin{pgfscope}%
\pgfpathrectangle{\pgfqpoint{0.150000in}{0.150000in}}{\pgfqpoint{2.700000in}{1.950000in}}%
\pgfusepath{clip}%
\pgfsetbuttcap%
\pgfsetroundjoin%
\definecolor{currentfill}{rgb}{0.876532,0.775965,0.783931}%
\pgfsetfillcolor{currentfill}%
\pgfsetlinewidth{0.000000pt}%
\definecolor{currentstroke}{rgb}{0.000000,0.000000,0.000000}%
\pgfsetstrokecolor{currentstroke}%
\pgfsetdash{}{0pt}%
\pgfpathmoveto{\pgfqpoint{1.231312in}{0.741008in}}%
\pgfpathlineto{\pgfqpoint{1.269745in}{0.769047in}}%
\pgfpathlineto{\pgfqpoint{1.231967in}{0.797025in}}%
\pgfpathlineto{\pgfqpoint{1.193269in}{0.776895in}}%
\pgfpathclose%
\pgfusepath{fill}%
\end{pgfscope}%
\begin{pgfscope}%
\pgfpathrectangle{\pgfqpoint{0.150000in}{0.150000in}}{\pgfqpoint{2.700000in}{1.950000in}}%
\pgfusepath{clip}%
\pgfsetbuttcap%
\pgfsetroundjoin%
\definecolor{currentfill}{rgb}{0.979105,0.962086,0.963434}%
\pgfsetfillcolor{currentfill}%
\pgfsetlinewidth{0.000000pt}%
\definecolor{currentstroke}{rgb}{0.000000,0.000000,0.000000}%
\pgfsetstrokecolor{currentstroke}%
\pgfsetdash{}{0pt}%
\pgfpathmoveto{\pgfqpoint{1.651197in}{0.960163in}}%
\pgfpathlineto{\pgfqpoint{1.688680in}{0.948034in}}%
\pgfpathlineto{\pgfqpoint{1.650509in}{0.975779in}}%
\pgfpathlineto{\pgfqpoint{1.612878in}{0.988032in}}%
\pgfpathclose%
\pgfusepath{fill}%
\end{pgfscope}%
\begin{pgfscope}%
\pgfpathrectangle{\pgfqpoint{0.150000in}{0.150000in}}{\pgfqpoint{2.700000in}{1.950000in}}%
\pgfusepath{clip}%
\pgfsetbuttcap%
\pgfsetroundjoin%
\definecolor{currentfill}{rgb}{0.941115,0.893153,0.896952}%
\pgfsetfillcolor{currentfill}%
\pgfsetlinewidth{0.000000pt}%
\definecolor{currentstroke}{rgb}{0.000000,0.000000,0.000000}%
\pgfsetstrokecolor{currentstroke}%
\pgfsetdash{}{0pt}%
\pgfpathmoveto{\pgfqpoint{1.881094in}{0.896259in}}%
\pgfpathlineto{\pgfqpoint{1.916907in}{0.868591in}}%
\pgfpathlineto{\pgfqpoint{1.878235in}{0.888503in}}%
\pgfpathlineto{\pgfqpoint{1.842000in}{0.908276in}}%
\pgfpathclose%
\pgfusepath{fill}%
\end{pgfscope}%
\begin{pgfscope}%
\pgfpathrectangle{\pgfqpoint{0.150000in}{0.150000in}}{\pgfqpoint{2.700000in}{1.950000in}}%
\pgfusepath{clip}%
\pgfsetbuttcap%
\pgfsetroundjoin%
\definecolor{currentfill}{rgb}{0.899326,0.817325,0.823820}%
\pgfsetfillcolor{currentfill}%
\pgfsetlinewidth{0.000000pt}%
\definecolor{currentstroke}{rgb}{0.000000,0.000000,0.000000}%
\pgfsetstrokecolor{currentstroke}%
\pgfsetdash{}{0pt}%
\pgfpathmoveto{\pgfqpoint{1.116349in}{0.792629in}}%
\pgfpathlineto{\pgfqpoint{1.155249in}{0.812761in}}%
\pgfpathlineto{\pgfqpoint{1.117898in}{0.832819in}}%
\pgfpathlineto{\pgfqpoint{1.079001in}{0.812761in}}%
\pgfpathclose%
\pgfusepath{fill}%
\end{pgfscope}%
\begin{pgfscope}%
\pgfpathrectangle{\pgfqpoint{0.150000in}{0.150000in}}{\pgfqpoint{2.700000in}{1.950000in}}%
\pgfusepath{clip}%
\pgfsetbuttcap%
\pgfsetroundjoin%
\definecolor{currentfill}{rgb}{0.965794,0.970006,0.975904}%
\pgfsetfillcolor{currentfill}%
\pgfsetlinewidth{0.000000pt}%
\definecolor{currentstroke}{rgb}{0.000000,0.000000,0.000000}%
\pgfsetstrokecolor{currentstroke}%
\pgfsetdash{}{0pt}%
\pgfpathmoveto{\pgfqpoint{1.346008in}{0.999779in}}%
\pgfpathlineto{\pgfqpoint{1.383678in}{1.067816in}}%
\pgfpathlineto{\pgfqpoint{1.345682in}{1.095544in}}%
\pgfpathlineto{\pgfqpoint{1.308158in}{1.027503in}}%
\pgfpathclose%
\pgfusepath{fill}%
\end{pgfscope}%
\begin{pgfscope}%
\pgfpathrectangle{\pgfqpoint{0.150000in}{0.150000in}}{\pgfqpoint{2.700000in}{1.950000in}}%
\pgfusepath{clip}%
\pgfsetbuttcap%
\pgfsetroundjoin%
\definecolor{currentfill}{rgb}{0.861336,0.748392,0.757338}%
\pgfsetfillcolor{currentfill}%
\pgfsetlinewidth{0.000000pt}%
\definecolor{currentstroke}{rgb}{0.000000,0.000000,0.000000}%
\pgfsetstrokecolor{currentstroke}%
\pgfsetdash{}{0pt}%
\pgfpathmoveto{\pgfqpoint{1.269378in}{0.705098in}}%
\pgfpathlineto{\pgfqpoint{1.307782in}{0.733184in}}%
\pgfpathlineto{\pgfqpoint{1.269745in}{0.769047in}}%
\pgfpathlineto{\pgfqpoint{1.231312in}{0.741008in}}%
\pgfpathclose%
\pgfusepath{fill}%
\end{pgfscope}%
\begin{pgfscope}%
\pgfpathrectangle{\pgfqpoint{0.150000in}{0.150000in}}{\pgfqpoint{2.700000in}{1.950000in}}%
\pgfusepath{clip}%
\pgfsetbuttcap%
\pgfsetroundjoin%
\definecolor{currentfill}{rgb}{0.884130,0.789752,0.797227}%
\pgfsetfillcolor{currentfill}%
\pgfsetlinewidth{0.000000pt}%
\definecolor{currentstroke}{rgb}{0.000000,0.000000,0.000000}%
\pgfsetstrokecolor{currentstroke}%
\pgfsetdash{}{0pt}%
\pgfpathmoveto{\pgfqpoint{1.383781in}{0.748843in}}%
\pgfpathlineto{\pgfqpoint{1.421992in}{0.784756in}}%
\pgfpathlineto{\pgfqpoint{1.384109in}{0.804887in}}%
\pgfpathlineto{\pgfqpoint{1.345957in}{0.769047in}}%
\pgfpathclose%
\pgfusepath{fill}%
\end{pgfscope}%
\begin{pgfscope}%
\pgfpathrectangle{\pgfqpoint{0.150000in}{0.150000in}}{\pgfqpoint{2.700000in}{1.950000in}}%
\pgfusepath{clip}%
\pgfsetbuttcap%
\pgfsetroundjoin%
\definecolor{currentfill}{rgb}{0.944914,0.900046,0.903600}%
\pgfsetfillcolor{currentfill}%
\pgfsetlinewidth{0.000000pt}%
\definecolor{currentstroke}{rgb}{0.000000,0.000000,0.000000}%
\pgfsetstrokecolor{currentstroke}%
\pgfsetdash{}{0pt}%
\pgfpathmoveto{\pgfqpoint{1.422061in}{0.856517in}}%
\pgfpathlineto{\pgfqpoint{1.460108in}{0.908276in}}%
\pgfpathlineto{\pgfqpoint{1.422042in}{0.936166in}}%
\pgfpathlineto{\pgfqpoint{1.384201in}{0.876504in}}%
\pgfpathclose%
\pgfusepath{fill}%
\end{pgfscope}%
\begin{pgfscope}%
\pgfpathrectangle{\pgfqpoint{0.150000in}{0.150000in}}{\pgfqpoint{2.700000in}{1.950000in}}%
\pgfusepath{clip}%
\pgfsetbuttcap%
\pgfsetroundjoin%
\definecolor{currentfill}{rgb}{0.865135,0.755285,0.763986}%
\pgfsetfillcolor{currentfill}%
\pgfsetlinewidth{0.000000pt}%
\definecolor{currentstroke}{rgb}{0.000000,0.000000,0.000000}%
\pgfsetstrokecolor{currentstroke}%
\pgfsetdash{}{0pt}%
\pgfpathmoveto{\pgfqpoint{1.345695in}{0.705098in}}%
\pgfpathlineto{\pgfqpoint{1.383781in}{0.748843in}}%
\pgfpathlineto{\pgfqpoint{1.345957in}{0.769047in}}%
\pgfpathlineto{\pgfqpoint{1.307782in}{0.733184in}}%
\pgfpathclose%
\pgfusepath{fill}%
\end{pgfscope}%
\begin{pgfscope}%
\pgfpathrectangle{\pgfqpoint{0.150000in}{0.150000in}}{\pgfqpoint{2.700000in}{1.950000in}}%
\pgfusepath{clip}%
\pgfsetbuttcap%
\pgfsetroundjoin%
\definecolor{currentfill}{rgb}{0.963909,0.934513,0.936841}%
\pgfsetfillcolor{currentfill}%
\pgfsetlinewidth{0.000000pt}%
\definecolor{currentstroke}{rgb}{0.000000,0.000000,0.000000}%
\pgfsetstrokecolor{currentstroke}%
\pgfsetdash{}{0pt}%
\pgfpathmoveto{\pgfqpoint{1.766333in}{0.940245in}}%
\pgfpathlineto{\pgfqpoint{1.803111in}{0.920229in}}%
\pgfpathlineto{\pgfqpoint{1.764425in}{0.932121in}}%
\pgfpathlineto{\pgfqpoint{1.727375in}{0.944153in}}%
\pgfpathclose%
\pgfusepath{fill}%
\end{pgfscope}%
\begin{pgfscope}%
\pgfpathrectangle{\pgfqpoint{0.150000in}{0.150000in}}{\pgfqpoint{2.700000in}{1.950000in}}%
\pgfusepath{clip}%
\pgfsetbuttcap%
\pgfsetroundjoin%
\definecolor{currentfill}{rgb}{0.499341,0.560999,0.647319}%
\pgfsetfillcolor{currentfill}%
\pgfsetlinewidth{0.000000pt}%
\definecolor{currentstroke}{rgb}{0.000000,0.000000,0.000000}%
\pgfsetstrokecolor{currentstroke}%
\pgfsetdash{}{0pt}%
\pgfpathmoveto{\pgfqpoint{0.891091in}{1.506856in}}%
\pgfpathlineto{\pgfqpoint{0.923107in}{1.652722in}}%
\pgfpathlineto{\pgfqpoint{0.884969in}{1.688511in}}%
\pgfpathlineto{\pgfqpoint{0.853333in}{1.542342in}}%
\pgfpathclose%
\pgfusepath{fill}%
\end{pgfscope}%
\begin{pgfscope}%
\pgfpathrectangle{\pgfqpoint{0.150000in}{0.150000in}}{\pgfqpoint{2.700000in}{1.950000in}}%
\pgfusepath{clip}%
\pgfsetbuttcap%
\pgfsetroundjoin%
\definecolor{currentfill}{rgb}{0.906924,0.831112,0.837117}%
\pgfsetfillcolor{currentfill}%
\pgfsetlinewidth{0.000000pt}%
\definecolor{currentstroke}{rgb}{0.000000,0.000000,0.000000}%
\pgfsetstrokecolor{currentstroke}%
\pgfsetdash{}{0pt}%
\pgfpathmoveto{\pgfqpoint{1.421992in}{0.784756in}}%
\pgfpathlineto{\pgfqpoint{1.460121in}{0.828546in}}%
\pgfpathlineto{\pgfqpoint{1.422061in}{0.856517in}}%
\pgfpathlineto{\pgfqpoint{1.384109in}{0.804887in}}%
\pgfpathclose%
\pgfusepath{fill}%
\end{pgfscope}%
\begin{pgfscope}%
\pgfpathrectangle{\pgfqpoint{0.150000in}{0.150000in}}{\pgfqpoint{2.700000in}{1.950000in}}%
\pgfusepath{clip}%
\pgfsetbuttcap%
\pgfsetroundjoin%
\definecolor{currentfill}{rgb}{0.530438,0.588266,0.669225}%
\pgfsetfillcolor{currentfill}%
\pgfsetlinewidth{0.000000pt}%
\definecolor{currentstroke}{rgb}{0.000000,0.000000,0.000000}%
\pgfsetstrokecolor{currentstroke}%
\pgfsetdash{}{0pt}%
\pgfpathmoveto{\pgfqpoint{0.928873in}{1.471349in}}%
\pgfpathlineto{\pgfqpoint{0.961715in}{1.608367in}}%
\pgfpathlineto{\pgfqpoint{0.923107in}{1.652722in}}%
\pgfpathlineto{\pgfqpoint{0.891091in}{1.506856in}}%
\pgfpathclose%
\pgfusepath{fill}%
\end{pgfscope}%
\begin{pgfscope}%
\pgfpathrectangle{\pgfqpoint{0.150000in}{0.150000in}}{\pgfqpoint{2.700000in}{1.950000in}}%
\pgfusepath{clip}%
\pgfsetbuttcap%
\pgfsetroundjoin%
\definecolor{currentfill}{rgb}{0.846140,0.720818,0.730744}%
\pgfsetfillcolor{currentfill}%
\pgfsetlinewidth{0.000000pt}%
\definecolor{currentstroke}{rgb}{0.000000,0.000000,0.000000}%
\pgfsetstrokecolor{currentstroke}%
\pgfsetdash{}{0pt}%
\pgfpathmoveto{\pgfqpoint{1.307467in}{0.669167in}}%
\pgfpathlineto{\pgfqpoint{1.345695in}{0.705098in}}%
\pgfpathlineto{\pgfqpoint{1.307782in}{0.733184in}}%
\pgfpathlineto{\pgfqpoint{1.269378in}{0.705098in}}%
\pgfpathclose%
\pgfusepath{fill}%
\end{pgfscope}%
\begin{pgfscope}%
\pgfpathrectangle{\pgfqpoint{0.150000in}{0.150000in}}{\pgfqpoint{2.700000in}{1.950000in}}%
\pgfusepath{clip}%
\pgfsetbuttcap%
\pgfsetroundjoin%
\definecolor{currentfill}{rgb}{0.868934,0.762178,0.770634}%
\pgfsetfillcolor{currentfill}%
\pgfsetlinewidth{0.000000pt}%
\definecolor{currentstroke}{rgb}{0.000000,0.000000,0.000000}%
\pgfsetstrokecolor{currentstroke}%
\pgfsetdash{}{0pt}%
\pgfpathmoveto{\pgfqpoint{1.193061in}{0.705098in}}%
\pgfpathlineto{\pgfqpoint{1.231312in}{0.741008in}}%
\pgfpathlineto{\pgfqpoint{1.193269in}{0.776895in}}%
\pgfpathlineto{\pgfqpoint{1.154723in}{0.748843in}}%
\pgfpathclose%
\pgfusepath{fill}%
\end{pgfscope}%
\begin{pgfscope}%
\pgfpathrectangle{\pgfqpoint{0.150000in}{0.150000in}}{\pgfqpoint{2.700000in}{1.950000in}}%
\pgfusepath{clip}%
\pgfsetbuttcap%
\pgfsetroundjoin%
\definecolor{currentfill}{rgb}{0.984452,0.986366,0.989047}%
\pgfsetfillcolor{currentfill}%
\pgfsetlinewidth{0.000000pt}%
\definecolor{currentstroke}{rgb}{0.000000,0.000000,0.000000}%
\pgfsetstrokecolor{currentstroke}%
\pgfsetdash{}{0pt}%
\pgfpathmoveto{\pgfqpoint{1.384058in}{0.963996in}}%
\pgfpathlineto{\pgfqpoint{1.421846in}{1.031953in}}%
\pgfpathlineto{\pgfqpoint{1.383678in}{1.067816in}}%
\pgfpathlineto{\pgfqpoint{1.346008in}{0.999779in}}%
\pgfpathclose%
\pgfusepath{fill}%
\end{pgfscope}%
\begin{pgfscope}%
\pgfpathrectangle{\pgfqpoint{0.150000in}{0.150000in}}{\pgfqpoint{2.700000in}{1.950000in}}%
\pgfusepath{clip}%
\pgfsetbuttcap%
\pgfsetroundjoin%
\definecolor{currentfill}{rgb}{0.887929,0.796645,0.803876}%
\pgfsetfillcolor{currentfill}%
\pgfsetlinewidth{0.000000pt}%
\definecolor{currentstroke}{rgb}{0.000000,0.000000,0.000000}%
\pgfsetstrokecolor{currentstroke}%
\pgfsetdash{}{0pt}%
\pgfpathmoveto{\pgfqpoint{1.154723in}{0.748843in}}%
\pgfpathlineto{\pgfqpoint{1.193269in}{0.776895in}}%
\pgfpathlineto{\pgfqpoint{1.155249in}{0.812761in}}%
\pgfpathlineto{\pgfqpoint{1.116349in}{0.792629in}}%
\pgfpathclose%
\pgfusepath{fill}%
\end{pgfscope}%
\begin{pgfscope}%
\pgfpathrectangle{\pgfqpoint{0.150000in}{0.150000in}}{\pgfqpoint{2.700000in}{1.950000in}}%
\pgfusepath{clip}%
\pgfsetbuttcap%
\pgfsetroundjoin%
\definecolor{currentfill}{rgb}{0.849939,0.727711,0.737393}%
\pgfsetfillcolor{currentfill}%
\pgfsetlinewidth{0.000000pt}%
\definecolor{currentstroke}{rgb}{0.000000,0.000000,0.000000}%
\pgfsetstrokecolor{currentstroke}%
\pgfsetdash{}{0pt}%
\pgfpathmoveto{\pgfqpoint{1.231127in}{0.669167in}}%
\pgfpathlineto{\pgfqpoint{1.269378in}{0.705098in}}%
\pgfpathlineto{\pgfqpoint{1.231312in}{0.741008in}}%
\pgfpathlineto{\pgfqpoint{1.193061in}{0.705098in}}%
\pgfpathclose%
\pgfusepath{fill}%
\end{pgfscope}%
\begin{pgfscope}%
\pgfpathrectangle{\pgfqpoint{0.150000in}{0.150000in}}{\pgfqpoint{2.700000in}{1.950000in}}%
\pgfusepath{clip}%
\pgfsetbuttcap%
\pgfsetroundjoin%
\definecolor{currentfill}{rgb}{0.555316,0.610080,0.686749}%
\pgfsetfillcolor{currentfill}%
\pgfsetlinewidth{0.000000pt}%
\definecolor{currentstroke}{rgb}{0.000000,0.000000,0.000000}%
\pgfsetstrokecolor{currentstroke}%
\pgfsetdash{}{0pt}%
\pgfpathmoveto{\pgfqpoint{0.966238in}{1.444159in}}%
\pgfpathlineto{\pgfqpoint{0.999870in}{1.572560in}}%
\pgfpathlineto{\pgfqpoint{0.961715in}{1.608367in}}%
\pgfpathlineto{\pgfqpoint{0.928873in}{1.471349in}}%
\pgfpathclose%
\pgfusepath{fill}%
\end{pgfscope}%
\begin{pgfscope}%
\pgfpathrectangle{\pgfqpoint{0.150000in}{0.150000in}}{\pgfqpoint{2.700000in}{1.950000in}}%
\pgfusepath{clip}%
\pgfsetbuttcap%
\pgfsetroundjoin%
\definecolor{currentfill}{rgb}{0.830944,0.693244,0.704151}%
\pgfsetfillcolor{currentfill}%
\pgfsetlinewidth{0.000000pt}%
\definecolor{currentstroke}{rgb}{0.000000,0.000000,0.000000}%
\pgfsetstrokecolor{currentstroke}%
\pgfsetdash{}{0pt}%
\pgfpathmoveto{\pgfqpoint{1.269423in}{0.625467in}}%
\pgfpathlineto{\pgfqpoint{1.307467in}{0.669167in}}%
\pgfpathlineto{\pgfqpoint{1.269378in}{0.705098in}}%
\pgfpathlineto{\pgfqpoint{1.231127in}{0.669167in}}%
\pgfpathclose%
\pgfusepath{fill}%
\end{pgfscope}%
\begin{pgfscope}%
\pgfpathrectangle{\pgfqpoint{0.150000in}{0.150000in}}{\pgfqpoint{2.700000in}{1.950000in}}%
\pgfusepath{clip}%
\pgfsetbuttcap%
\pgfsetroundjoin%
\definecolor{currentfill}{rgb}{0.586412,0.637347,0.708655}%
\pgfsetfillcolor{currentfill}%
\pgfsetlinewidth{0.000000pt}%
\definecolor{currentstroke}{rgb}{0.000000,0.000000,0.000000}%
\pgfsetstrokecolor{currentstroke}%
\pgfsetdash{}{0pt}%
\pgfpathmoveto{\pgfqpoint{1.004094in}{1.408584in}}%
\pgfpathlineto{\pgfqpoint{1.038435in}{1.528250in}}%
\pgfpathlineto{\pgfqpoint{0.999870in}{1.572560in}}%
\pgfpathlineto{\pgfqpoint{0.966238in}{1.444159in}}%
\pgfpathclose%
\pgfusepath{fill}%
\end{pgfscope}%
\begin{pgfscope}%
\pgfpathrectangle{\pgfqpoint{0.150000in}{0.150000in}}{\pgfqpoint{2.700000in}{1.950000in}}%
\pgfusepath{clip}%
\pgfsetbuttcap%
\pgfsetroundjoin%
\definecolor{currentfill}{rgb}{0.994301,0.989660,0.990028}%
\pgfsetfillcolor{currentfill}%
\pgfsetlinewidth{0.000000pt}%
\definecolor{currentstroke}{rgb}{0.000000,0.000000,0.000000}%
\pgfsetstrokecolor{currentstroke}%
\pgfsetdash{}{0pt}%
\pgfpathmoveto{\pgfqpoint{1.422042in}{0.936166in}}%
\pgfpathlineto{\pgfqpoint{1.460036in}{0.996069in}}%
\pgfpathlineto{\pgfqpoint{1.421846in}{1.031953in}}%
\pgfpathlineto{\pgfqpoint{1.384058in}{0.963996in}}%
\pgfpathclose%
\pgfusepath{fill}%
\end{pgfscope}%
\begin{pgfscope}%
\pgfpathrectangle{\pgfqpoint{0.150000in}{0.150000in}}{\pgfqpoint{2.700000in}{1.950000in}}%
\pgfusepath{clip}%
\pgfsetbuttcap%
\pgfsetroundjoin%
\definecolor{currentfill}{rgb}{0.617509,0.664614,0.730561}%
\pgfsetfillcolor{currentfill}%
\pgfsetlinewidth{0.000000pt}%
\definecolor{currentstroke}{rgb}{0.000000,0.000000,0.000000}%
\pgfsetstrokecolor{currentstroke}%
\pgfsetdash{}{0pt}%
\pgfpathmoveto{\pgfqpoint{1.041973in}{1.372988in}}%
\pgfpathlineto{\pgfqpoint{1.076607in}{1.492425in}}%
\pgfpathlineto{\pgfqpoint{1.038435in}{1.528250in}}%
\pgfpathlineto{\pgfqpoint{1.004094in}{1.408584in}}%
\pgfpathclose%
\pgfusepath{fill}%
\end{pgfscope}%
\begin{pgfscope}%
\pgfpathrectangle{\pgfqpoint{0.150000in}{0.150000in}}{\pgfqpoint{2.700000in}{1.950000in}}%
\pgfusepath{clip}%
\pgfsetbuttcap%
\pgfsetroundjoin%
\definecolor{currentfill}{rgb}{0.648606,0.691881,0.752466}%
\pgfsetfillcolor{currentfill}%
\pgfsetlinewidth{0.000000pt}%
\definecolor{currentstroke}{rgb}{0.000000,0.000000,0.000000}%
\pgfsetstrokecolor{currentstroke}%
\pgfsetdash{}{0pt}%
\pgfpathmoveto{\pgfqpoint{1.079875in}{1.337370in}}%
\pgfpathlineto{\pgfqpoint{1.115130in}{1.448160in}}%
\pgfpathlineto{\pgfqpoint{1.076607in}{1.492425in}}%
\pgfpathlineto{\pgfqpoint{1.041973in}{1.372988in}}%
\pgfpathclose%
\pgfusepath{fill}%
\end{pgfscope}%
\begin{pgfscope}%
\pgfpathrectangle{\pgfqpoint{0.150000in}{0.150000in}}{\pgfqpoint{2.700000in}{1.950000in}}%
\pgfusepath{clip}%
\pgfsetbuttcap%
\pgfsetroundjoin%
\definecolor{currentfill}{rgb}{0.815748,0.665671,0.677558}%
\pgfsetfillcolor{currentfill}%
\pgfsetlinewidth{0.000000pt}%
\definecolor{currentstroke}{rgb}{0.000000,0.000000,0.000000}%
\pgfsetstrokecolor{currentstroke}%
\pgfsetdash{}{0pt}%
\pgfpathmoveto{\pgfqpoint{1.307328in}{0.597239in}}%
\pgfpathlineto{\pgfqpoint{1.345431in}{0.640973in}}%
\pgfpathlineto{\pgfqpoint{1.307467in}{0.669167in}}%
\pgfpathlineto{\pgfqpoint{1.269423in}{0.625467in}}%
\pgfpathclose%
\pgfusepath{fill}%
\end{pgfscope}%
\begin{pgfscope}%
\pgfpathrectangle{\pgfqpoint{0.150000in}{0.150000in}}{\pgfqpoint{2.700000in}{1.950000in}}%
\pgfusepath{clip}%
\pgfsetbuttcap%
\pgfsetroundjoin%
\definecolor{currentfill}{rgb}{0.679703,0.719148,0.774372}%
\pgfsetfillcolor{currentfill}%
\pgfsetlinewidth{0.000000pt}%
\definecolor{currentstroke}{rgb}{0.000000,0.000000,0.000000}%
\pgfsetstrokecolor{currentstroke}%
\pgfsetdash{}{0pt}%
\pgfpathmoveto{\pgfqpoint{1.117800in}{1.301731in}}%
\pgfpathlineto{\pgfqpoint{1.153318in}{1.412316in}}%
\pgfpathlineto{\pgfqpoint{1.115130in}{1.448160in}}%
\pgfpathlineto{\pgfqpoint{1.079875in}{1.337370in}}%
\pgfpathclose%
\pgfusepath{fill}%
\end{pgfscope}%
\begin{pgfscope}%
\pgfpathrectangle{\pgfqpoint{0.150000in}{0.150000in}}{\pgfqpoint{2.700000in}{1.950000in}}%
\pgfusepath{clip}%
\pgfsetbuttcap%
\pgfsetroundjoin%
\definecolor{currentfill}{rgb}{0.834743,0.700138,0.710800}%
\pgfsetfillcolor{currentfill}%
\pgfsetlinewidth{0.000000pt}%
\definecolor{currentstroke}{rgb}{0.000000,0.000000,0.000000}%
\pgfsetstrokecolor{currentstroke}%
\pgfsetdash{}{0pt}%
\pgfpathmoveto{\pgfqpoint{1.345431in}{0.640973in}}%
\pgfpathlineto{\pgfqpoint{1.383689in}{0.676952in}}%
\pgfpathlineto{\pgfqpoint{1.345695in}{0.705098in}}%
\pgfpathlineto{\pgfqpoint{1.307467in}{0.669167in}}%
\pgfpathclose%
\pgfusepath{fill}%
\end{pgfscope}%
\begin{pgfscope}%
\pgfpathrectangle{\pgfqpoint{0.150000in}{0.150000in}}{\pgfqpoint{2.700000in}{1.950000in}}%
\pgfusepath{clip}%
\pgfsetbuttcap%
\pgfsetroundjoin%
\definecolor{currentfill}{rgb}{0.937316,0.886259,0.890303}%
\pgfsetfillcolor{currentfill}%
\pgfsetlinewidth{0.000000pt}%
\definecolor{currentstroke}{rgb}{0.000000,0.000000,0.000000}%
\pgfsetstrokecolor{currentstroke}%
\pgfsetdash{}{0pt}%
\pgfpathmoveto{\pgfqpoint{1.460121in}{0.828546in}}%
\pgfpathlineto{\pgfqpoint{1.498256in}{0.880326in}}%
\pgfpathlineto{\pgfqpoint{1.460108in}{0.908276in}}%
\pgfpathlineto{\pgfqpoint{1.422061in}{0.856517in}}%
\pgfpathclose%
\pgfusepath{fill}%
\end{pgfscope}%
\begin{pgfscope}%
\pgfpathrectangle{\pgfqpoint{0.150000in}{0.150000in}}{\pgfqpoint{2.700000in}{1.950000in}}%
\pgfusepath{clip}%
\pgfsetbuttcap%
\pgfsetroundjoin%
\definecolor{currentfill}{rgb}{0.857537,0.741498,0.750689}%
\pgfsetfillcolor{currentfill}%
\pgfsetlinewidth{0.000000pt}%
\definecolor{currentstroke}{rgb}{0.000000,0.000000,0.000000}%
\pgfsetstrokecolor{currentstroke}%
\pgfsetdash{}{0pt}%
\pgfpathmoveto{\pgfqpoint{1.383689in}{0.676952in}}%
\pgfpathlineto{\pgfqpoint{1.421834in}{0.720731in}}%
\pgfpathlineto{\pgfqpoint{1.383781in}{0.748843in}}%
\pgfpathlineto{\pgfqpoint{1.345695in}{0.705098in}}%
\pgfpathclose%
\pgfusepath{fill}%
\end{pgfscope}%
\begin{pgfscope}%
\pgfpathrectangle{\pgfqpoint{0.150000in}{0.150000in}}{\pgfqpoint{2.700000in}{1.950000in}}%
\pgfusepath{clip}%
\pgfsetbuttcap%
\pgfsetroundjoin%
\definecolor{currentfill}{rgb}{0.710800,0.746415,0.796278}%
\pgfsetfillcolor{currentfill}%
\pgfsetlinewidth{0.000000pt}%
\definecolor{currentstroke}{rgb}{0.000000,0.000000,0.000000}%
\pgfsetstrokecolor{currentstroke}%
\pgfsetdash{}{0pt}%
\pgfpathmoveto{\pgfqpoint{1.155747in}{1.266070in}}%
\pgfpathlineto{\pgfqpoint{1.191798in}{1.368097in}}%
\pgfpathlineto{\pgfqpoint{1.153318in}{1.412316in}}%
\pgfpathlineto{\pgfqpoint{1.117800in}{1.301731in}}%
\pgfpathclose%
\pgfusepath{fill}%
\end{pgfscope}%
\begin{pgfscope}%
\pgfpathrectangle{\pgfqpoint{0.150000in}{0.150000in}}{\pgfqpoint{2.700000in}{1.950000in}}%
\pgfusepath{clip}%
\pgfsetbuttcap%
\pgfsetroundjoin%
\definecolor{currentfill}{rgb}{0.880331,0.782858,0.790579}%
\pgfsetfillcolor{currentfill}%
\pgfsetlinewidth{0.000000pt}%
\definecolor{currentstroke}{rgb}{0.000000,0.000000,0.000000}%
\pgfsetstrokecolor{currentstroke}%
\pgfsetdash{}{0pt}%
\pgfpathmoveto{\pgfqpoint{1.421834in}{0.720731in}}%
\pgfpathlineto{\pgfqpoint{1.460015in}{0.764551in}}%
\pgfpathlineto{\pgfqpoint{1.421992in}{0.784756in}}%
\pgfpathlineto{\pgfqpoint{1.383781in}{0.748843in}}%
\pgfpathclose%
\pgfusepath{fill}%
\end{pgfscope}%
\begin{pgfscope}%
\pgfpathrectangle{\pgfqpoint{0.150000in}{0.150000in}}{\pgfqpoint{2.700000in}{1.950000in}}%
\pgfusepath{clip}%
\pgfsetbuttcap%
\pgfsetroundjoin%
\definecolor{currentfill}{rgb}{0.735677,0.768229,0.813802}%
\pgfsetfillcolor{currentfill}%
\pgfsetlinewidth{0.000000pt}%
\definecolor{currentstroke}{rgb}{0.000000,0.000000,0.000000}%
\pgfsetstrokecolor{currentstroke}%
\pgfsetdash{}{0pt}%
\pgfpathmoveto{\pgfqpoint{1.193453in}{1.238589in}}%
\pgfpathlineto{\pgfqpoint{1.230004in}{1.332235in}}%
\pgfpathlineto{\pgfqpoint{1.191798in}{1.368097in}}%
\pgfpathlineto{\pgfqpoint{1.155747in}{1.266070in}}%
\pgfpathclose%
\pgfusepath{fill}%
\end{pgfscope}%
\begin{pgfscope}%
\pgfpathrectangle{\pgfqpoint{0.150000in}{0.150000in}}{\pgfqpoint{2.700000in}{1.950000in}}%
\pgfusepath{clip}%
\pgfsetbuttcap%
\pgfsetroundjoin%
\definecolor{currentfill}{rgb}{0.982904,0.968980,0.970083}%
\pgfsetfillcolor{currentfill}%
\pgfsetlinewidth{0.000000pt}%
\definecolor{currentstroke}{rgb}{0.000000,0.000000,0.000000}%
\pgfsetstrokecolor{currentstroke}%
\pgfsetdash{}{0pt}%
\pgfpathmoveto{\pgfqpoint{1.460108in}{0.908276in}}%
\pgfpathlineto{\pgfqpoint{1.498250in}{0.960163in}}%
\pgfpathlineto{\pgfqpoint{1.460036in}{0.996069in}}%
\pgfpathlineto{\pgfqpoint{1.422042in}{0.936166in}}%
\pgfpathclose%
\pgfusepath{fill}%
\end{pgfscope}%
\begin{pgfscope}%
\pgfpathrectangle{\pgfqpoint{0.150000in}{0.150000in}}{\pgfqpoint{2.700000in}{1.950000in}}%
\pgfusepath{clip}%
\pgfsetbuttcap%
\pgfsetroundjoin%
\definecolor{currentfill}{rgb}{0.982904,0.968980,0.970083}%
\pgfsetfillcolor{currentfill}%
\pgfsetlinewidth{0.000000pt}%
\definecolor{currentstroke}{rgb}{0.000000,0.000000,0.000000}%
\pgfsetstrokecolor{currentstroke}%
\pgfsetdash{}{0pt}%
\pgfpathmoveto{\pgfqpoint{1.689956in}{0.956305in}}%
\pgfpathlineto{\pgfqpoint{1.727375in}{0.944153in}}%
\pgfpathlineto{\pgfqpoint{1.688680in}{0.948034in}}%
\pgfpathlineto{\pgfqpoint{1.651197in}{0.960163in}}%
\pgfpathclose%
\pgfusepath{fill}%
\end{pgfscope}%
\begin{pgfscope}%
\pgfpathrectangle{\pgfqpoint{0.150000in}{0.150000in}}{\pgfqpoint{2.700000in}{1.950000in}}%
\pgfusepath{clip}%
\pgfsetbuttcap%
\pgfsetroundjoin%
\definecolor{currentfill}{rgb}{0.903125,0.824219,0.830469}%
\pgfsetfillcolor{currentfill}%
\pgfsetlinewidth{0.000000pt}%
\definecolor{currentstroke}{rgb}{0.000000,0.000000,0.000000}%
\pgfsetstrokecolor{currentstroke}%
\pgfsetdash{}{0pt}%
\pgfpathmoveto{\pgfqpoint{1.460015in}{0.764551in}}%
\pgfpathlineto{\pgfqpoint{1.498233in}{0.808412in}}%
\pgfpathlineto{\pgfqpoint{1.460121in}{0.828546in}}%
\pgfpathlineto{\pgfqpoint{1.421992in}{0.784756in}}%
\pgfpathclose%
\pgfusepath{fill}%
\end{pgfscope}%
\begin{pgfscope}%
\pgfpathrectangle{\pgfqpoint{0.150000in}{0.150000in}}{\pgfqpoint{2.700000in}{1.950000in}}%
\pgfusepath{clip}%
\pgfsetbuttcap%
\pgfsetroundjoin%
\definecolor{currentfill}{rgb}{0.760555,0.790043,0.831327}%
\pgfsetfillcolor{currentfill}%
\pgfsetlinewidth{0.000000pt}%
\definecolor{currentstroke}{rgb}{0.000000,0.000000,0.000000}%
\pgfsetstrokecolor{currentstroke}%
\pgfsetdash{}{0pt}%
\pgfpathmoveto{\pgfqpoint{1.231476in}{1.202860in}}%
\pgfpathlineto{\pgfqpoint{1.268441in}{1.288061in}}%
\pgfpathlineto{\pgfqpoint{1.230004in}{1.332235in}}%
\pgfpathlineto{\pgfqpoint{1.193453in}{1.238589in}}%
\pgfpathclose%
\pgfusepath{fill}%
\end{pgfscope}%
\begin{pgfscope}%
\pgfpathrectangle{\pgfqpoint{0.150000in}{0.150000in}}{\pgfqpoint{2.700000in}{1.950000in}}%
\pgfusepath{clip}%
\pgfsetbuttcap%
\pgfsetroundjoin%
\definecolor{currentfill}{rgb}{0.791651,0.817310,0.853232}%
\pgfsetfillcolor{currentfill}%
\pgfsetlinewidth{0.000000pt}%
\definecolor{currentstroke}{rgb}{0.000000,0.000000,0.000000}%
\pgfsetstrokecolor{currentstroke}%
\pgfsetdash{}{0pt}%
\pgfpathmoveto{\pgfqpoint{1.269521in}{1.167110in}}%
\pgfpathlineto{\pgfqpoint{1.306663in}{1.252181in}}%
\pgfpathlineto{\pgfqpoint{1.268441in}{1.288061in}}%
\pgfpathlineto{\pgfqpoint{1.231476in}{1.202860in}}%
\pgfpathclose%
\pgfusepath{fill}%
\end{pgfscope}%
\begin{pgfscope}%
\pgfpathrectangle{\pgfqpoint{0.150000in}{0.150000in}}{\pgfqpoint{2.700000in}{1.950000in}}%
\pgfusepath{clip}%
\pgfsetbuttcap%
\pgfsetroundjoin%
\definecolor{currentfill}{rgb}{0.944914,0.900046,0.903600}%
\pgfsetfillcolor{currentfill}%
\pgfsetlinewidth{0.000000pt}%
\definecolor{currentstroke}{rgb}{0.000000,0.000000,0.000000}%
\pgfsetstrokecolor{currentstroke}%
\pgfsetdash{}{0pt}%
\pgfpathmoveto{\pgfqpoint{1.920393in}{0.884180in}}%
\pgfpathlineto{\pgfqpoint{1.956045in}{0.856517in}}%
\pgfpathlineto{\pgfqpoint{1.916907in}{0.868591in}}%
\pgfpathlineto{\pgfqpoint{1.881094in}{0.896259in}}%
\pgfpathclose%
\pgfusepath{fill}%
\end{pgfscope}%
\begin{pgfscope}%
\pgfpathrectangle{\pgfqpoint{0.150000in}{0.150000in}}{\pgfqpoint{2.700000in}{1.950000in}}%
\pgfusepath{clip}%
\pgfsetbuttcap%
\pgfsetroundjoin%
\definecolor{currentfill}{rgb}{0.530438,0.588266,0.669225}%
\pgfsetfillcolor{currentfill}%
\pgfsetlinewidth{0.000000pt}%
\definecolor{currentstroke}{rgb}{0.000000,0.000000,0.000000}%
\pgfsetstrokecolor{currentstroke}%
\pgfsetdash{}{0pt}%
\pgfpathmoveto{\pgfqpoint{1.038435in}{1.528250in}}%
\pgfpathlineto{\pgfqpoint{1.079016in}{1.513292in}}%
\pgfpathlineto{\pgfqpoint{1.041044in}{1.548924in}}%
\pgfpathlineto{\pgfqpoint{0.999870in}{1.572560in}}%
\pgfpathclose%
\pgfusepath{fill}%
\end{pgfscope}%
\begin{pgfscope}%
\pgfpathrectangle{\pgfqpoint{0.150000in}{0.150000in}}{\pgfqpoint{2.700000in}{1.950000in}}%
\pgfusepath{clip}%
\pgfsetbuttcap%
\pgfsetroundjoin%
\definecolor{currentfill}{rgb}{0.661045,0.702788,0.761229}%
\pgfsetfillcolor{currentfill}%
\pgfsetlinewidth{0.000000pt}%
\definecolor{currentstroke}{rgb}{0.000000,0.000000,0.000000}%
\pgfsetstrokecolor{currentstroke}%
\pgfsetdash{}{0pt}%
\pgfpathmoveto{\pgfqpoint{1.191798in}{1.368097in}}%
\pgfpathlineto{\pgfqpoint{1.231373in}{1.362236in}}%
\pgfpathlineto{\pgfqpoint{1.193338in}{1.397929in}}%
\pgfpathlineto{\pgfqpoint{1.153318in}{1.412316in}}%
\pgfpathclose%
\pgfusepath{fill}%
\end{pgfscope}%
\begin{pgfscope}%
\pgfpathrectangle{\pgfqpoint{0.150000in}{0.150000in}}{\pgfqpoint{2.700000in}{1.950000in}}%
\pgfusepath{clip}%
\pgfsetbuttcap%
\pgfsetroundjoin%
\definecolor{currentfill}{rgb}{0.430928,0.501011,0.599127}%
\pgfsetfillcolor{currentfill}%
\pgfsetlinewidth{0.000000pt}%
\definecolor{currentstroke}{rgb}{0.000000,0.000000,0.000000}%
\pgfsetstrokecolor{currentstroke}%
\pgfsetdash{}{0pt}%
\pgfpathmoveto{\pgfqpoint{0.923107in}{1.652722in}}%
\pgfpathlineto{\pgfqpoint{0.964726in}{1.628624in}}%
\pgfpathlineto{\pgfqpoint{0.926793in}{1.664216in}}%
\pgfpathlineto{\pgfqpoint{0.884969in}{1.688511in}}%
\pgfpathclose%
\pgfusepath{fill}%
\end{pgfscope}%
\begin{pgfscope}%
\pgfpathrectangle{\pgfqpoint{0.150000in}{0.150000in}}{\pgfqpoint{2.700000in}{1.950000in}}%
\pgfusepath{clip}%
\pgfsetbuttcap%
\pgfsetroundjoin%
\definecolor{currentfill}{rgb}{0.493122,0.555545,0.642938}%
\pgfsetfillcolor{currentfill}%
\pgfsetlinewidth{0.000000pt}%
\definecolor{currentstroke}{rgb}{0.000000,0.000000,0.000000}%
\pgfsetstrokecolor{currentstroke}%
\pgfsetdash{}{0pt}%
\pgfpathmoveto{\pgfqpoint{0.999870in}{1.572560in}}%
\pgfpathlineto{\pgfqpoint{1.041044in}{1.548924in}}%
\pgfpathlineto{\pgfqpoint{1.002682in}{1.593011in}}%
\pgfpathlineto{\pgfqpoint{0.961715in}{1.608367in}}%
\pgfpathclose%
\pgfusepath{fill}%
\end{pgfscope}%
\begin{pgfscope}%
\pgfpathrectangle{\pgfqpoint{0.150000in}{0.150000in}}{\pgfqpoint{2.700000in}{1.950000in}}%
\pgfusepath{clip}%
\pgfsetbuttcap%
\pgfsetroundjoin%
\definecolor{currentfill}{rgb}{0.561535,0.615533,0.691131}%
\pgfsetfillcolor{currentfill}%
\pgfsetlinewidth{0.000000pt}%
\definecolor{currentstroke}{rgb}{0.000000,0.000000,0.000000}%
\pgfsetstrokecolor{currentstroke}%
\pgfsetdash{}{0pt}%
\pgfpathmoveto{\pgfqpoint{1.076607in}{1.492425in}}%
\pgfpathlineto{\pgfqpoint{1.117012in}{1.477639in}}%
\pgfpathlineto{\pgfqpoint{1.079016in}{1.513292in}}%
\pgfpathlineto{\pgfqpoint{1.038435in}{1.528250in}}%
\pgfpathclose%
\pgfusepath{fill}%
\end{pgfscope}%
\begin{pgfscope}%
\pgfpathrectangle{\pgfqpoint{0.150000in}{0.150000in}}{\pgfqpoint{2.700000in}{1.950000in}}%
\pgfusepath{clip}%
\pgfsetbuttcap%
\pgfsetroundjoin%
\definecolor{currentfill}{rgb}{0.623729,0.670067,0.734942}%
\pgfsetfillcolor{currentfill}%
\pgfsetlinewidth{0.000000pt}%
\definecolor{currentstroke}{rgb}{0.000000,0.000000,0.000000}%
\pgfsetstrokecolor{currentstroke}%
\pgfsetdash{}{0pt}%
\pgfpathmoveto{\pgfqpoint{1.153318in}{1.412316in}}%
\pgfpathlineto{\pgfqpoint{1.193338in}{1.397929in}}%
\pgfpathlineto{\pgfqpoint{1.155031in}{1.441965in}}%
\pgfpathlineto{\pgfqpoint{1.115130in}{1.448160in}}%
\pgfpathclose%
\pgfusepath{fill}%
\end{pgfscope}%
\begin{pgfscope}%
\pgfpathrectangle{\pgfqpoint{0.150000in}{0.150000in}}{\pgfqpoint{2.700000in}{1.950000in}}%
\pgfusepath{clip}%
\pgfsetbuttcap%
\pgfsetroundjoin%
\definecolor{currentfill}{rgb}{0.692142,0.730055,0.783134}%
\pgfsetfillcolor{currentfill}%
\pgfsetlinewidth{0.000000pt}%
\definecolor{currentstroke}{rgb}{0.000000,0.000000,0.000000}%
\pgfsetstrokecolor{currentstroke}%
\pgfsetdash{}{0pt}%
\pgfpathmoveto{\pgfqpoint{1.230004in}{1.332235in}}%
\pgfpathlineto{\pgfqpoint{1.269432in}{1.326522in}}%
\pgfpathlineto{\pgfqpoint{1.231373in}{1.362236in}}%
\pgfpathlineto{\pgfqpoint{1.191798in}{1.368097in}}%
\pgfpathclose%
\pgfusepath{fill}%
\end{pgfscope}%
\begin{pgfscope}%
\pgfpathrectangle{\pgfqpoint{0.150000in}{0.150000in}}{\pgfqpoint{2.700000in}{1.950000in}}%
\pgfusepath{clip}%
\pgfsetbuttcap%
\pgfsetroundjoin%
\definecolor{currentfill}{rgb}{0.462025,0.528278,0.621032}%
\pgfsetfillcolor{currentfill}%
\pgfsetlinewidth{0.000000pt}%
\definecolor{currentstroke}{rgb}{0.000000,0.000000,0.000000}%
\pgfsetstrokecolor{currentstroke}%
\pgfsetdash{}{0pt}%
\pgfpathmoveto{\pgfqpoint{0.961715in}{1.608367in}}%
\pgfpathlineto{\pgfqpoint{1.002682in}{1.593011in}}%
\pgfpathlineto{\pgfqpoint{0.964726in}{1.628624in}}%
\pgfpathlineto{\pgfqpoint{0.923107in}{1.652722in}}%
\pgfpathclose%
\pgfusepath{fill}%
\end{pgfscope}%
\begin{pgfscope}%
\pgfpathrectangle{\pgfqpoint{0.150000in}{0.150000in}}{\pgfqpoint{2.700000in}{1.950000in}}%
\pgfusepath{clip}%
\pgfsetbuttcap%
\pgfsetroundjoin%
\definecolor{currentfill}{rgb}{0.754335,0.784589,0.826945}%
\pgfsetfillcolor{currentfill}%
\pgfsetlinewidth{0.000000pt}%
\definecolor{currentstroke}{rgb}{0.000000,0.000000,0.000000}%
\pgfsetstrokecolor{currentstroke}%
\pgfsetdash{}{0pt}%
\pgfpathmoveto{\pgfqpoint{1.306663in}{1.252181in}}%
\pgfpathlineto{\pgfqpoint{1.345765in}{1.246802in}}%
\pgfpathlineto{\pgfqpoint{1.307513in}{1.290786in}}%
\pgfpathlineto{\pgfqpoint{1.268441in}{1.288061in}}%
\pgfpathclose%
\pgfusepath{fill}%
\end{pgfscope}%
\begin{pgfscope}%
\pgfpathrectangle{\pgfqpoint{0.150000in}{0.150000in}}{\pgfqpoint{2.700000in}{1.950000in}}%
\pgfusepath{clip}%
\pgfsetbuttcap%
\pgfsetroundjoin%
\definecolor{currentfill}{rgb}{0.967708,0.941406,0.943490}%
\pgfsetfillcolor{currentfill}%
\pgfsetlinewidth{0.000000pt}%
\definecolor{currentstroke}{rgb}{0.000000,0.000000,0.000000}%
\pgfsetstrokecolor{currentstroke}%
\pgfsetdash{}{0pt}%
\pgfpathmoveto{\pgfqpoint{1.498256in}{0.880326in}}%
\pgfpathlineto{\pgfqpoint{1.536486in}{0.924236in}}%
\pgfpathlineto{\pgfqpoint{1.498250in}{0.960163in}}%
\pgfpathlineto{\pgfqpoint{1.460108in}{0.908276in}}%
\pgfpathclose%
\pgfusepath{fill}%
\end{pgfscope}%
\begin{pgfscope}%
\pgfpathrectangle{\pgfqpoint{0.150000in}{0.150000in}}{\pgfqpoint{2.700000in}{1.950000in}}%
\pgfusepath{clip}%
\pgfsetbuttcap%
\pgfsetroundjoin%
\definecolor{currentfill}{rgb}{0.822748,0.844577,0.875138}%
\pgfsetfillcolor{currentfill}%
\pgfsetlinewidth{0.000000pt}%
\definecolor{currentstroke}{rgb}{0.000000,0.000000,0.000000}%
\pgfsetstrokecolor{currentstroke}%
\pgfsetdash{}{0pt}%
\pgfpathmoveto{\pgfqpoint{1.307590in}{1.131337in}}%
\pgfpathlineto{\pgfqpoint{1.344909in}{1.216279in}}%
\pgfpathlineto{\pgfqpoint{1.306663in}{1.252181in}}%
\pgfpathlineto{\pgfqpoint{1.269521in}{1.167110in}}%
\pgfpathclose%
\pgfusepath{fill}%
\end{pgfscope}%
\begin{pgfscope}%
\pgfpathrectangle{\pgfqpoint{0.150000in}{0.150000in}}{\pgfqpoint{2.700000in}{1.950000in}}%
\pgfusepath{clip}%
\pgfsetbuttcap%
\pgfsetroundjoin%
\definecolor{currentfill}{rgb}{0.592632,0.642800,0.713036}%
\pgfsetfillcolor{currentfill}%
\pgfsetlinewidth{0.000000pt}%
\definecolor{currentstroke}{rgb}{0.000000,0.000000,0.000000}%
\pgfsetstrokecolor{currentstroke}%
\pgfsetdash{}{0pt}%
\pgfpathmoveto{\pgfqpoint{1.115130in}{1.448160in}}%
\pgfpathlineto{\pgfqpoint{1.155031in}{1.441965in}}%
\pgfpathlineto{\pgfqpoint{1.117012in}{1.477639in}}%
\pgfpathlineto{\pgfqpoint{1.076607in}{1.492425in}}%
\pgfpathclose%
\pgfusepath{fill}%
\end{pgfscope}%
\begin{pgfscope}%
\pgfpathrectangle{\pgfqpoint{0.150000in}{0.150000in}}{\pgfqpoint{2.700000in}{1.950000in}}%
\pgfusepath{clip}%
\pgfsetbuttcap%
\pgfsetroundjoin%
\definecolor{currentfill}{rgb}{0.723238,0.757322,0.805040}%
\pgfsetfillcolor{currentfill}%
\pgfsetlinewidth{0.000000pt}%
\definecolor{currentstroke}{rgb}{0.000000,0.000000,0.000000}%
\pgfsetstrokecolor{currentstroke}%
\pgfsetdash{}{0pt}%
\pgfpathmoveto{\pgfqpoint{1.268441in}{1.288061in}}%
\pgfpathlineto{\pgfqpoint{1.307513in}{1.290786in}}%
\pgfpathlineto{\pgfqpoint{1.269432in}{1.326522in}}%
\pgfpathlineto{\pgfqpoint{1.230004in}{1.332235in}}%
\pgfpathclose%
\pgfusepath{fill}%
\end{pgfscope}%
\begin{pgfscope}%
\pgfpathrectangle{\pgfqpoint{0.150000in}{0.150000in}}{\pgfqpoint{2.700000in}{1.950000in}}%
\pgfusepath{clip}%
\pgfsetbuttcap%
\pgfsetroundjoin%
\definecolor{currentfill}{rgb}{0.785432,0.811857,0.848851}%
\pgfsetfillcolor{currentfill}%
\pgfsetlinewidth{0.000000pt}%
\definecolor{currentstroke}{rgb}{0.000000,0.000000,0.000000}%
\pgfsetstrokecolor{currentstroke}%
\pgfsetdash{}{0pt}%
\pgfpathmoveto{\pgfqpoint{1.344909in}{1.216279in}}%
\pgfpathlineto{\pgfqpoint{1.383863in}{1.211048in}}%
\pgfpathlineto{\pgfqpoint{1.345765in}{1.246802in}}%
\pgfpathlineto{\pgfqpoint{1.306663in}{1.252181in}}%
\pgfpathclose%
\pgfusepath{fill}%
\end{pgfscope}%
\begin{pgfscope}%
\pgfpathrectangle{\pgfqpoint{0.150000in}{0.150000in}}{\pgfqpoint{2.700000in}{1.950000in}}%
\pgfusepath{clip}%
\pgfsetbuttcap%
\pgfsetroundjoin%
\definecolor{currentfill}{rgb}{0.853845,0.871844,0.897044}%
\pgfsetfillcolor{currentfill}%
\pgfsetlinewidth{0.000000pt}%
\definecolor{currentstroke}{rgb}{0.000000,0.000000,0.000000}%
\pgfsetstrokecolor{currentstroke}%
\pgfsetdash{}{0pt}%
\pgfpathmoveto{\pgfqpoint{1.345682in}{1.095544in}}%
\pgfpathlineto{\pgfqpoint{1.383297in}{1.172154in}}%
\pgfpathlineto{\pgfqpoint{1.344909in}{1.216279in}}%
\pgfpathlineto{\pgfqpoint{1.307590in}{1.131337in}}%
\pgfpathclose%
\pgfusepath{fill}%
\end{pgfscope}%
\begin{pgfscope}%
\pgfpathrectangle{\pgfqpoint{0.150000in}{0.150000in}}{\pgfqpoint{2.700000in}{1.950000in}}%
\pgfusepath{clip}%
\pgfsetbuttcap%
\pgfsetroundjoin%
\definecolor{currentfill}{rgb}{0.916039,0.926379,0.940855}%
\pgfsetfillcolor{currentfill}%
\pgfsetlinewidth{0.000000pt}%
\definecolor{currentstroke}{rgb}{0.000000,0.000000,0.000000}%
\pgfsetstrokecolor{currentstroke}%
\pgfsetdash{}{0pt}%
\pgfpathmoveto{\pgfqpoint{1.498184in}{1.056215in}}%
\pgfpathlineto{\pgfqpoint{1.536486in}{1.059728in}}%
\pgfpathlineto{\pgfqpoint{1.498326in}{1.095544in}}%
\pgfpathlineto{\pgfqpoint{1.459905in}{1.092154in}}%
\pgfpathclose%
\pgfusepath{fill}%
\end{pgfscope}%
\begin{pgfscope}%
\pgfpathrectangle{\pgfqpoint{0.150000in}{0.150000in}}{\pgfqpoint{2.700000in}{1.950000in}}%
\pgfusepath{clip}%
\pgfsetbuttcap%
\pgfsetroundjoin%
\definecolor{currentfill}{rgb}{0.816529,0.839124,0.870757}%
\pgfsetfillcolor{currentfill}%
\pgfsetlinewidth{0.000000pt}%
\definecolor{currentstroke}{rgb}{0.000000,0.000000,0.000000}%
\pgfsetstrokecolor{currentstroke}%
\pgfsetdash{}{0pt}%
\pgfpathmoveto{\pgfqpoint{1.383297in}{1.172154in}}%
\pgfpathlineto{\pgfqpoint{1.421984in}{1.175273in}}%
\pgfpathlineto{\pgfqpoint{1.383863in}{1.211048in}}%
\pgfpathlineto{\pgfqpoint{1.344909in}{1.216279in}}%
\pgfpathclose%
\pgfusepath{fill}%
\end{pgfscope}%
\begin{pgfscope}%
\pgfpathrectangle{\pgfqpoint{0.150000in}{0.150000in}}{\pgfqpoint{2.700000in}{1.950000in}}%
\pgfusepath{clip}%
\pgfsetbuttcap%
\pgfsetroundjoin%
\definecolor{currentfill}{rgb}{0.929718,0.872472,0.877007}%
\pgfsetfillcolor{currentfill}%
\pgfsetlinewidth{0.000000pt}%
\definecolor{currentstroke}{rgb}{0.000000,0.000000,0.000000}%
\pgfsetstrokecolor{currentstroke}%
\pgfsetdash{}{0pt}%
\pgfpathmoveto{\pgfqpoint{1.498233in}{0.808412in}}%
\pgfpathlineto{\pgfqpoint{1.536486in}{0.852315in}}%
\pgfpathlineto{\pgfqpoint{1.498256in}{0.880326in}}%
\pgfpathlineto{\pgfqpoint{1.460121in}{0.828546in}}%
\pgfpathclose%
\pgfusepath{fill}%
\end{pgfscope}%
\begin{pgfscope}%
\pgfpathrectangle{\pgfqpoint{0.150000in}{0.150000in}}{\pgfqpoint{2.700000in}{1.950000in}}%
\pgfusepath{clip}%
\pgfsetbuttcap%
\pgfsetroundjoin%
\definecolor{currentfill}{rgb}{0.884942,0.899112,0.918949}%
\pgfsetfillcolor{currentfill}%
\pgfsetlinewidth{0.000000pt}%
\definecolor{currentstroke}{rgb}{0.000000,0.000000,0.000000}%
\pgfsetstrokecolor{currentstroke}%
\pgfsetdash{}{0pt}%
\pgfpathmoveto{\pgfqpoint{1.459905in}{1.092154in}}%
\pgfpathlineto{\pgfqpoint{1.498326in}{1.095544in}}%
\pgfpathlineto{\pgfqpoint{1.460129in}{1.139476in}}%
\pgfpathlineto{\pgfqpoint{1.421559in}{1.136234in}}%
\pgfpathclose%
\pgfusepath{fill}%
\end{pgfscope}%
\begin{pgfscope}%
\pgfpathrectangle{\pgfqpoint{0.150000in}{0.150000in}}{\pgfqpoint{2.700000in}{1.950000in}}%
\pgfusepath{clip}%
\pgfsetbuttcap%
\pgfsetroundjoin%
\definecolor{currentfill}{rgb}{0.947135,0.953646,0.962760}%
\pgfsetfillcolor{currentfill}%
\pgfsetlinewidth{0.000000pt}%
\definecolor{currentstroke}{rgb}{0.000000,0.000000,0.000000}%
\pgfsetstrokecolor{currentstroke}%
\pgfsetdash{}{0pt}%
\pgfpathmoveto{\pgfqpoint{1.536486in}{1.012181in}}%
\pgfpathlineto{\pgfqpoint{1.574671in}{1.023891in}}%
\pgfpathlineto{\pgfqpoint{1.536486in}{1.059728in}}%
\pgfpathlineto{\pgfqpoint{1.498184in}{1.056215in}}%
\pgfpathclose%
\pgfusepath{fill}%
\end{pgfscope}%
\begin{pgfscope}%
\pgfpathrectangle{\pgfqpoint{0.150000in}{0.150000in}}{\pgfqpoint{2.700000in}{1.950000in}}%
\pgfusepath{clip}%
\pgfsetbuttcap%
\pgfsetroundjoin%
\definecolor{currentfill}{rgb}{0.847626,0.866391,0.892662}%
\pgfsetfillcolor{currentfill}%
\pgfsetlinewidth{0.000000pt}%
\definecolor{currentstroke}{rgb}{0.000000,0.000000,0.000000}%
\pgfsetstrokecolor{currentstroke}%
\pgfsetdash{}{0pt}%
\pgfpathmoveto{\pgfqpoint{1.421559in}{1.136234in}}%
\pgfpathlineto{\pgfqpoint{1.460129in}{1.139476in}}%
\pgfpathlineto{\pgfqpoint{1.421984in}{1.175273in}}%
\pgfpathlineto{\pgfqpoint{1.383297in}{1.172154in}}%
\pgfpathclose%
\pgfusepath{fill}%
\end{pgfscope}%
\begin{pgfscope}%
\pgfpathrectangle{\pgfqpoint{0.150000in}{0.150000in}}{\pgfqpoint{2.700000in}{1.950000in}}%
\pgfusepath{clip}%
\pgfsetbuttcap%
\pgfsetroundjoin%
\definecolor{currentfill}{rgb}{0.978232,0.980913,0.984666}%
\pgfsetfillcolor{currentfill}%
\pgfsetlinewidth{0.000000pt}%
\definecolor{currentstroke}{rgb}{0.000000,0.000000,0.000000}%
\pgfsetstrokecolor{currentstroke}%
\pgfsetdash{}{0pt}%
\pgfpathmoveto{\pgfqpoint{1.574783in}{0.976224in}}%
\pgfpathlineto{\pgfqpoint{1.612878in}{0.988032in}}%
\pgfpathlineto{\pgfqpoint{1.574671in}{1.023891in}}%
\pgfpathlineto{\pgfqpoint{1.536486in}{1.012181in}}%
\pgfpathclose%
\pgfusepath{fill}%
\end{pgfscope}%
\begin{pgfscope}%
\pgfpathrectangle{\pgfqpoint{0.150000in}{0.150000in}}{\pgfqpoint{2.700000in}{1.950000in}}%
\pgfusepath{clip}%
\pgfsetbuttcap%
\pgfsetroundjoin%
\definecolor{currentfill}{rgb}{0.878722,0.893658,0.914568}%
\pgfsetfillcolor{currentfill}%
\pgfsetlinewidth{0.000000pt}%
\definecolor{currentstroke}{rgb}{0.000000,0.000000,0.000000}%
\pgfsetstrokecolor{currentstroke}%
\pgfsetdash{}{0pt}%
\pgfpathmoveto{\pgfqpoint{1.383678in}{1.067816in}}%
\pgfpathlineto{\pgfqpoint{1.421559in}{1.136234in}}%
\pgfpathlineto{\pgfqpoint{1.383297in}{1.172154in}}%
\pgfpathlineto{\pgfqpoint{1.345682in}{1.095544in}}%
\pgfpathclose%
\pgfusepath{fill}%
\end{pgfscope}%
\begin{pgfscope}%
\pgfpathrectangle{\pgfqpoint{0.150000in}{0.150000in}}{\pgfqpoint{2.700000in}{1.950000in}}%
\pgfusepath{clip}%
\pgfsetbuttcap%
\pgfsetroundjoin%
\definecolor{currentfill}{rgb}{0.975306,0.955193,0.956786}%
\pgfsetfillcolor{currentfill}%
\pgfsetlinewidth{0.000000pt}%
\definecolor{currentstroke}{rgb}{0.000000,0.000000,0.000000}%
\pgfsetstrokecolor{currentstroke}%
\pgfsetdash{}{0pt}%
\pgfpathmoveto{\pgfqpoint{1.805349in}{0.928276in}}%
\pgfpathlineto{\pgfqpoint{1.842000in}{0.908276in}}%
\pgfpathlineto{\pgfqpoint{1.803111in}{0.920229in}}%
\pgfpathlineto{\pgfqpoint{1.766333in}{0.940245in}}%
\pgfpathclose%
\pgfusepath{fill}%
\end{pgfscope}%
\begin{pgfscope}%
\pgfpathrectangle{\pgfqpoint{0.150000in}{0.150000in}}{\pgfqpoint{2.700000in}{1.950000in}}%
\pgfusepath{clip}%
\pgfsetbuttcap%
\pgfsetroundjoin%
\definecolor{currentfill}{rgb}{0.909819,0.920925,0.936474}%
\pgfsetfillcolor{currentfill}%
\pgfsetlinewidth{0.000000pt}%
\definecolor{currentstroke}{rgb}{0.000000,0.000000,0.000000}%
\pgfsetstrokecolor{currentstroke}%
\pgfsetdash{}{0pt}%
\pgfpathmoveto{\pgfqpoint{1.421846in}{1.031953in}}%
\pgfpathlineto{\pgfqpoint{1.459905in}{1.092154in}}%
\pgfpathlineto{\pgfqpoint{1.421559in}{1.136234in}}%
\pgfpathlineto{\pgfqpoint{1.383678in}{1.067816in}}%
\pgfpathclose%
\pgfusepath{fill}%
\end{pgfscope}%
\begin{pgfscope}%
\pgfpathrectangle{\pgfqpoint{0.150000in}{0.150000in}}{\pgfqpoint{2.700000in}{1.950000in}}%
\pgfusepath{clip}%
\pgfsetbuttcap%
\pgfsetroundjoin%
\definecolor{currentfill}{rgb}{0.940916,0.948192,0.958379}%
\pgfsetfillcolor{currentfill}%
\pgfsetlinewidth{0.000000pt}%
\definecolor{currentstroke}{rgb}{0.000000,0.000000,0.000000}%
\pgfsetstrokecolor{currentstroke}%
\pgfsetdash{}{0pt}%
\pgfpathmoveto{\pgfqpoint{1.460036in}{0.996069in}}%
\pgfpathlineto{\pgfqpoint{1.498184in}{1.056215in}}%
\pgfpathlineto{\pgfqpoint{1.459905in}{1.092154in}}%
\pgfpathlineto{\pgfqpoint{1.421846in}{1.031953in}}%
\pgfpathclose%
\pgfusepath{fill}%
\end{pgfscope}%
\begin{pgfscope}%
\pgfpathrectangle{\pgfqpoint{0.150000in}{0.150000in}}{\pgfqpoint{2.700000in}{1.950000in}}%
\pgfusepath{clip}%
\pgfsetbuttcap%
\pgfsetroundjoin%
\definecolor{currentfill}{rgb}{0.956311,0.920726,0.923545}%
\pgfsetfillcolor{currentfill}%
\pgfsetlinewidth{0.000000pt}%
\definecolor{currentstroke}{rgb}{0.000000,0.000000,0.000000}%
\pgfsetstrokecolor{currentstroke}%
\pgfsetdash{}{0pt}%
\pgfpathmoveto{\pgfqpoint{1.536486in}{0.852315in}}%
\pgfpathlineto{\pgfqpoint{1.574776in}{0.896259in}}%
\pgfpathlineto{\pgfqpoint{1.536486in}{0.924236in}}%
\pgfpathlineto{\pgfqpoint{1.498256in}{0.880326in}}%
\pgfpathclose%
\pgfusepath{fill}%
\end{pgfscope}%
\begin{pgfscope}%
\pgfpathrectangle{\pgfqpoint{0.150000in}{0.150000in}}{\pgfqpoint{2.700000in}{1.950000in}}%
\pgfusepath{clip}%
\pgfsetbuttcap%
\pgfsetroundjoin%
\definecolor{currentfill}{rgb}{0.972013,0.975460,0.980285}%
\pgfsetfillcolor{currentfill}%
\pgfsetlinewidth{0.000000pt}%
\definecolor{currentstroke}{rgb}{0.000000,0.000000,0.000000}%
\pgfsetstrokecolor{currentstroke}%
\pgfsetdash{}{0pt}%
\pgfpathmoveto{\pgfqpoint{1.498250in}{0.960163in}}%
\pgfpathlineto{\pgfqpoint{1.536486in}{1.012181in}}%
\pgfpathlineto{\pgfqpoint{1.498184in}{1.056215in}}%
\pgfpathlineto{\pgfqpoint{1.460036in}{0.996069in}}%
\pgfpathclose%
\pgfusepath{fill}%
\end{pgfscope}%
\begin{pgfscope}%
\pgfpathrectangle{\pgfqpoint{0.150000in}{0.150000in}}{\pgfqpoint{2.700000in}{1.950000in}}%
\pgfusepath{clip}%
\pgfsetbuttcap%
\pgfsetroundjoin%
\definecolor{currentfill}{rgb}{0.861336,0.748392,0.757338}%
\pgfsetfillcolor{currentfill}%
\pgfsetlinewidth{0.000000pt}%
\definecolor{currentstroke}{rgb}{0.000000,0.000000,0.000000}%
\pgfsetstrokecolor{currentstroke}%
\pgfsetdash{}{0pt}%
\pgfpathmoveto{\pgfqpoint{1.421497in}{0.672131in}}%
\pgfpathlineto{\pgfqpoint{1.459850in}{0.708211in}}%
\pgfpathlineto{\pgfqpoint{1.421834in}{0.720731in}}%
\pgfpathlineto{\pgfqpoint{1.383689in}{0.676952in}}%
\pgfpathclose%
\pgfusepath{fill}%
\end{pgfscope}%
\begin{pgfscope}%
\pgfpathrectangle{\pgfqpoint{0.150000in}{0.150000in}}{\pgfqpoint{2.700000in}{1.950000in}}%
\pgfusepath{clip}%
\pgfsetbuttcap%
\pgfsetroundjoin%
\definecolor{currentfill}{rgb}{0.996890,0.997273,0.997809}%
\pgfsetfillcolor{currentfill}%
\pgfsetlinewidth{0.000000pt}%
\definecolor{currentstroke}{rgb}{0.000000,0.000000,0.000000}%
\pgfsetstrokecolor{currentstroke}%
\pgfsetdash{}{0pt}%
\pgfpathmoveto{\pgfqpoint{1.536486in}{0.924236in}}%
\pgfpathlineto{\pgfqpoint{1.574783in}{0.976224in}}%
\pgfpathlineto{\pgfqpoint{1.536486in}{1.012181in}}%
\pgfpathlineto{\pgfqpoint{1.498250in}{0.960163in}}%
\pgfpathclose%
\pgfusepath{fill}%
\end{pgfscope}%
\begin{pgfscope}%
\pgfpathrectangle{\pgfqpoint{0.150000in}{0.150000in}}{\pgfqpoint{2.700000in}{1.950000in}}%
\pgfusepath{clip}%
\pgfsetbuttcap%
\pgfsetroundjoin%
\definecolor{currentfill}{rgb}{0.996890,0.997273,0.997809}%
\pgfsetfillcolor{currentfill}%
\pgfsetlinewidth{0.000000pt}%
\definecolor{currentstroke}{rgb}{0.000000,0.000000,0.000000}%
\pgfsetstrokecolor{currentstroke}%
\pgfsetdash{}{0pt}%
\pgfpathmoveto{\pgfqpoint{1.613162in}{0.948269in}}%
\pgfpathlineto{\pgfqpoint{1.651197in}{0.960163in}}%
\pgfpathlineto{\pgfqpoint{1.612878in}{0.988032in}}%
\pgfpathlineto{\pgfqpoint{1.574783in}{0.976224in}}%
\pgfpathclose%
\pgfusepath{fill}%
\end{pgfscope}%
\begin{pgfscope}%
\pgfpathrectangle{\pgfqpoint{0.150000in}{0.150000in}}{\pgfqpoint{2.700000in}{1.950000in}}%
\pgfusepath{clip}%
\pgfsetbuttcap%
\pgfsetroundjoin%
\definecolor{currentfill}{rgb}{0.819547,0.672564,0.684206}%
\pgfsetfillcolor{currentfill}%
\pgfsetlinewidth{0.000000pt}%
\definecolor{currentstroke}{rgb}{0.000000,0.000000,0.000000}%
\pgfsetstrokecolor{currentstroke}%
\pgfsetdash{}{0pt}%
\pgfpathmoveto{\pgfqpoint{1.344870in}{0.592148in}}%
\pgfpathlineto{\pgfqpoint{1.383121in}{0.636028in}}%
\pgfpathlineto{\pgfqpoint{1.345431in}{0.640973in}}%
\pgfpathlineto{\pgfqpoint{1.307328in}{0.597239in}}%
\pgfpathclose%
\pgfusepath{fill}%
\end{pgfscope}%
\begin{pgfscope}%
\pgfpathrectangle{\pgfqpoint{0.150000in}{0.150000in}}{\pgfqpoint{2.700000in}{1.950000in}}%
\pgfusepath{clip}%
\pgfsetbuttcap%
\pgfsetroundjoin%
\definecolor{currentfill}{rgb}{0.880331,0.782858,0.790579}%
\pgfsetfillcolor{currentfill}%
\pgfsetlinewidth{0.000000pt}%
\definecolor{currentstroke}{rgb}{0.000000,0.000000,0.000000}%
\pgfsetstrokecolor{currentstroke}%
\pgfsetdash{}{0pt}%
\pgfpathmoveto{\pgfqpoint{1.459850in}{0.708211in}}%
\pgfpathlineto{\pgfqpoint{1.498180in}{0.744270in}}%
\pgfpathlineto{\pgfqpoint{1.460015in}{0.764551in}}%
\pgfpathlineto{\pgfqpoint{1.421834in}{0.720731in}}%
\pgfpathclose%
\pgfusepath{fill}%
\end{pgfscope}%
\begin{pgfscope}%
\pgfpathrectangle{\pgfqpoint{0.150000in}{0.150000in}}{\pgfqpoint{2.700000in}{1.950000in}}%
\pgfusepath{clip}%
\pgfsetbuttcap%
\pgfsetroundjoin%
\definecolor{currentfill}{rgb}{0.842341,0.713925,0.724096}%
\pgfsetfillcolor{currentfill}%
\pgfsetlinewidth{0.000000pt}%
\definecolor{currentstroke}{rgb}{0.000000,0.000000,0.000000}%
\pgfsetstrokecolor{currentstroke}%
\pgfsetdash{}{0pt}%
\pgfpathmoveto{\pgfqpoint{1.383121in}{0.636028in}}%
\pgfpathlineto{\pgfqpoint{1.421497in}{0.672131in}}%
\pgfpathlineto{\pgfqpoint{1.383689in}{0.676952in}}%
\pgfpathlineto{\pgfqpoint{1.345431in}{0.640973in}}%
\pgfpathclose%
\pgfusepath{fill}%
\end{pgfscope}%
\begin{pgfscope}%
\pgfpathrectangle{\pgfqpoint{0.150000in}{0.150000in}}{\pgfqpoint{2.700000in}{1.950000in}}%
\pgfusepath{clip}%
\pgfsetbuttcap%
\pgfsetroundjoin%
\definecolor{currentfill}{rgb}{0.906924,0.831112,0.837117}%
\pgfsetfillcolor{currentfill}%
\pgfsetlinewidth{0.000000pt}%
\definecolor{currentstroke}{rgb}{0.000000,0.000000,0.000000}%
\pgfsetstrokecolor{currentstroke}%
\pgfsetdash{}{0pt}%
\pgfpathmoveto{\pgfqpoint{1.498180in}{0.744270in}}%
\pgfpathlineto{\pgfqpoint{1.536486in}{0.796112in}}%
\pgfpathlineto{\pgfqpoint{1.498233in}{0.808412in}}%
\pgfpathlineto{\pgfqpoint{1.460015in}{0.764551in}}%
\pgfpathclose%
\pgfusepath{fill}%
\end{pgfscope}%
\begin{pgfscope}%
\pgfpathrectangle{\pgfqpoint{0.150000in}{0.150000in}}{\pgfqpoint{2.700000in}{1.950000in}}%
\pgfusepath{clip}%
\pgfsetbuttcap%
\pgfsetroundjoin%
\definecolor{currentfill}{rgb}{0.986703,0.975873,0.976731}%
\pgfsetfillcolor{currentfill}%
\pgfsetlinewidth{0.000000pt}%
\definecolor{currentstroke}{rgb}{0.000000,0.000000,0.000000}%
\pgfsetstrokecolor{currentstroke}%
\pgfsetdash{}{0pt}%
\pgfpathmoveto{\pgfqpoint{1.574776in}{0.896259in}}%
\pgfpathlineto{\pgfqpoint{1.613162in}{0.948269in}}%
\pgfpathlineto{\pgfqpoint{1.574783in}{0.976224in}}%
\pgfpathlineto{\pgfqpoint{1.536486in}{0.924236in}}%
\pgfpathclose%
\pgfusepath{fill}%
\end{pgfscope}%
\begin{pgfscope}%
\pgfpathrectangle{\pgfqpoint{0.150000in}{0.150000in}}{\pgfqpoint{2.700000in}{1.950000in}}%
\pgfusepath{clip}%
\pgfsetbuttcap%
\pgfsetroundjoin%
\definecolor{currentfill}{rgb}{0.952512,0.913833,0.916896}%
\pgfsetfillcolor{currentfill}%
\pgfsetlinewidth{0.000000pt}%
\definecolor{currentstroke}{rgb}{0.000000,0.000000,0.000000}%
\pgfsetstrokecolor{currentstroke}%
\pgfsetdash{}{0pt}%
\pgfpathmoveto{\pgfqpoint{1.959901in}{0.872036in}}%
\pgfpathlineto{\pgfqpoint{1.995035in}{0.836457in}}%
\pgfpathlineto{\pgfqpoint{1.956045in}{0.856517in}}%
\pgfpathlineto{\pgfqpoint{1.920393in}{0.884180in}}%
\pgfpathclose%
\pgfusepath{fill}%
\end{pgfscope}%
\begin{pgfscope}%
\pgfpathrectangle{\pgfqpoint{0.150000in}{0.150000in}}{\pgfqpoint{2.700000in}{1.950000in}}%
\pgfusepath{clip}%
\pgfsetbuttcap%
\pgfsetroundjoin%
\definecolor{currentfill}{rgb}{0.933517,0.879366,0.883655}%
\pgfsetfillcolor{currentfill}%
\pgfsetlinewidth{0.000000pt}%
\definecolor{currentstroke}{rgb}{0.000000,0.000000,0.000000}%
\pgfsetstrokecolor{currentstroke}%
\pgfsetdash{}{0pt}%
\pgfpathmoveto{\pgfqpoint{1.536486in}{0.796112in}}%
\pgfpathlineto{\pgfqpoint{1.574859in}{0.840125in}}%
\pgfpathlineto{\pgfqpoint{1.536486in}{0.852315in}}%
\pgfpathlineto{\pgfqpoint{1.498233in}{0.808412in}}%
\pgfpathclose%
\pgfusepath{fill}%
\end{pgfscope}%
\begin{pgfscope}%
\pgfpathrectangle{\pgfqpoint{0.150000in}{0.150000in}}{\pgfqpoint{2.700000in}{1.950000in}}%
\pgfusepath{clip}%
\pgfsetbuttcap%
\pgfsetroundjoin%
\definecolor{currentfill}{rgb}{0.996890,0.997273,0.997809}%
\pgfsetfillcolor{currentfill}%
\pgfsetlinewidth{0.000000pt}%
\definecolor{currentstroke}{rgb}{0.000000,0.000000,0.000000}%
\pgfsetstrokecolor{currentstroke}%
\pgfsetdash{}{0pt}%
\pgfpathmoveto{\pgfqpoint{1.728980in}{0.952419in}}%
\pgfpathlineto{\pgfqpoint{1.766333in}{0.940245in}}%
\pgfpathlineto{\pgfqpoint{1.727375in}{0.944153in}}%
\pgfpathlineto{\pgfqpoint{1.689956in}{0.956305in}}%
\pgfpathclose%
\pgfusepath{fill}%
\end{pgfscope}%
\begin{pgfscope}%
\pgfpathrectangle{\pgfqpoint{0.150000in}{0.150000in}}{\pgfqpoint{2.700000in}{1.950000in}}%
\pgfusepath{clip}%
\pgfsetbuttcap%
\pgfsetroundjoin%
\definecolor{currentfill}{rgb}{0.960110,0.927619,0.930193}%
\pgfsetfillcolor{currentfill}%
\pgfsetlinewidth{0.000000pt}%
\definecolor{currentstroke}{rgb}{0.000000,0.000000,0.000000}%
\pgfsetstrokecolor{currentstroke}%
\pgfsetdash{}{0pt}%
\pgfpathmoveto{\pgfqpoint{1.574859in}{0.840125in}}%
\pgfpathlineto{\pgfqpoint{1.613328in}{0.892177in}}%
\pgfpathlineto{\pgfqpoint{1.574776in}{0.896259in}}%
\pgfpathlineto{\pgfqpoint{1.536486in}{0.852315in}}%
\pgfpathclose%
\pgfusepath{fill}%
\end{pgfscope}%
\begin{pgfscope}%
\pgfpathrectangle{\pgfqpoint{0.150000in}{0.150000in}}{\pgfqpoint{2.700000in}{1.950000in}}%
\pgfusepath{clip}%
\pgfsetbuttcap%
\pgfsetroundjoin%
\definecolor{currentfill}{rgb}{0.990671,0.991820,0.993428}%
\pgfsetfillcolor{currentfill}%
\pgfsetlinewidth{0.000000pt}%
\definecolor{currentstroke}{rgb}{0.000000,0.000000,0.000000}%
\pgfsetstrokecolor{currentstroke}%
\pgfsetdash{}{0pt}%
\pgfpathmoveto{\pgfqpoint{1.651893in}{0.944359in}}%
\pgfpathlineto{\pgfqpoint{1.689956in}{0.956305in}}%
\pgfpathlineto{\pgfqpoint{1.651197in}{0.960163in}}%
\pgfpathlineto{\pgfqpoint{1.613162in}{0.948269in}}%
\pgfpathclose%
\pgfusepath{fill}%
\end{pgfscope}%
\begin{pgfscope}%
\pgfpathrectangle{\pgfqpoint{0.150000in}{0.150000in}}{\pgfqpoint{2.700000in}{1.950000in}}%
\pgfusepath{clip}%
\pgfsetbuttcap%
\pgfsetroundjoin%
\definecolor{currentfill}{rgb}{0.884130,0.789752,0.797227}%
\pgfsetfillcolor{currentfill}%
\pgfsetlinewidth{0.000000pt}%
\definecolor{currentstroke}{rgb}{0.000000,0.000000,0.000000}%
\pgfsetstrokecolor{currentstroke}%
\pgfsetdash{}{0pt}%
\pgfpathmoveto{\pgfqpoint{1.498067in}{0.695626in}}%
\pgfpathlineto{\pgfqpoint{1.536486in}{0.723914in}}%
\pgfpathlineto{\pgfqpoint{1.498180in}{0.744270in}}%
\pgfpathlineto{\pgfqpoint{1.459850in}{0.708211in}}%
\pgfpathclose%
\pgfusepath{fill}%
\end{pgfscope}%
\begin{pgfscope}%
\pgfpathrectangle{\pgfqpoint{0.150000in}{0.150000in}}{\pgfqpoint{2.700000in}{1.950000in}}%
\pgfusepath{clip}%
\pgfsetbuttcap%
\pgfsetroundjoin%
\definecolor{currentfill}{rgb}{0.986703,0.975873,0.976731}%
\pgfsetfillcolor{currentfill}%
\pgfsetlinewidth{0.000000pt}%
\definecolor{currentstroke}{rgb}{0.000000,0.000000,0.000000}%
\pgfsetstrokecolor{currentstroke}%
\pgfsetdash{}{0pt}%
\pgfpathmoveto{\pgfqpoint{1.844811in}{0.924290in}}%
\pgfpathlineto{\pgfqpoint{1.881094in}{0.896259in}}%
\pgfpathlineto{\pgfqpoint{1.842000in}{0.908276in}}%
\pgfpathlineto{\pgfqpoint{1.805349in}{0.928276in}}%
\pgfpathclose%
\pgfusepath{fill}%
\end{pgfscope}%
\begin{pgfscope}%
\pgfpathrectangle{\pgfqpoint{0.150000in}{0.150000in}}{\pgfqpoint{2.700000in}{1.950000in}}%
\pgfusepath{clip}%
\pgfsetbuttcap%
\pgfsetroundjoin%
\definecolor{currentfill}{rgb}{0.906924,0.831112,0.837117}%
\pgfsetfillcolor{currentfill}%
\pgfsetlinewidth{0.000000pt}%
\definecolor{currentstroke}{rgb}{0.000000,0.000000,0.000000}%
\pgfsetstrokecolor{currentstroke}%
\pgfsetdash{}{0pt}%
\pgfpathmoveto{\pgfqpoint{1.536486in}{0.723914in}}%
\pgfpathlineto{\pgfqpoint{1.574912in}{0.775827in}}%
\pgfpathlineto{\pgfqpoint{1.536486in}{0.796112in}}%
\pgfpathlineto{\pgfqpoint{1.498180in}{0.744270in}}%
\pgfpathclose%
\pgfusepath{fill}%
\end{pgfscope}%
\begin{pgfscope}%
\pgfpathrectangle{\pgfqpoint{0.150000in}{0.150000in}}{\pgfqpoint{2.700000in}{1.950000in}}%
\pgfusepath{clip}%
\pgfsetbuttcap%
\pgfsetroundjoin%
\definecolor{currentfill}{rgb}{0.868934,0.762178,0.770634}%
\pgfsetfillcolor{currentfill}%
\pgfsetlinewidth{0.000000pt}%
\definecolor{currentstroke}{rgb}{0.000000,0.000000,0.000000}%
\pgfsetstrokecolor{currentstroke}%
\pgfsetdash{}{0pt}%
\pgfpathmoveto{\pgfqpoint{1.459625in}{0.659446in}}%
\pgfpathlineto{\pgfqpoint{1.498067in}{0.695626in}}%
\pgfpathlineto{\pgfqpoint{1.459850in}{0.708211in}}%
\pgfpathlineto{\pgfqpoint{1.421497in}{0.672131in}}%
\pgfpathclose%
\pgfusepath{fill}%
\end{pgfscope}%
\begin{pgfscope}%
\pgfpathrectangle{\pgfqpoint{0.150000in}{0.150000in}}{\pgfqpoint{2.700000in}{1.950000in}}%
\pgfusepath{clip}%
\pgfsetbuttcap%
\pgfsetroundjoin%
\definecolor{currentfill}{rgb}{0.990502,0.982767,0.983379}%
\pgfsetfillcolor{currentfill}%
\pgfsetlinewidth{0.000000pt}%
\definecolor{currentstroke}{rgb}{0.000000,0.000000,0.000000}%
\pgfsetstrokecolor{currentstroke}%
\pgfsetdash{}{0pt}%
\pgfpathmoveto{\pgfqpoint{1.613328in}{0.892177in}}%
\pgfpathlineto{\pgfqpoint{1.651893in}{0.944359in}}%
\pgfpathlineto{\pgfqpoint{1.613162in}{0.948269in}}%
\pgfpathlineto{\pgfqpoint{1.574776in}{0.896259in}}%
\pgfpathclose%
\pgfusepath{fill}%
\end{pgfscope}%
\begin{pgfscope}%
\pgfpathrectangle{\pgfqpoint{0.150000in}{0.150000in}}{\pgfqpoint{2.700000in}{1.950000in}}%
\pgfusepath{clip}%
\pgfsetbuttcap%
\pgfsetroundjoin%
\definecolor{currentfill}{rgb}{0.853738,0.734605,0.744041}%
\pgfsetfillcolor{currentfill}%
\pgfsetlinewidth{0.000000pt}%
\definecolor{currentstroke}{rgb}{0.000000,0.000000,0.000000}%
\pgfsetstrokecolor{currentstroke}%
\pgfsetdash{}{0pt}%
\pgfpathmoveto{\pgfqpoint{1.421069in}{0.631049in}}%
\pgfpathlineto{\pgfqpoint{1.459625in}{0.659446in}}%
\pgfpathlineto{\pgfqpoint{1.421497in}{0.672131in}}%
\pgfpathlineto{\pgfqpoint{1.383121in}{0.636028in}}%
\pgfpathclose%
\pgfusepath{fill}%
\end{pgfscope}%
\begin{pgfscope}%
\pgfpathrectangle{\pgfqpoint{0.150000in}{0.150000in}}{\pgfqpoint{2.700000in}{1.950000in}}%
\pgfusepath{clip}%
\pgfsetbuttcap%
\pgfsetroundjoin%
\definecolor{currentfill}{rgb}{0.838542,0.707031,0.717448}%
\pgfsetfillcolor{currentfill}%
\pgfsetlinewidth{0.000000pt}%
\definecolor{currentstroke}{rgb}{0.000000,0.000000,0.000000}%
\pgfsetstrokecolor{currentstroke}%
\pgfsetdash{}{0pt}%
\pgfpathmoveto{\pgfqpoint{1.382549in}{0.594800in}}%
\pgfpathlineto{\pgfqpoint{1.421069in}{0.631049in}}%
\pgfpathlineto{\pgfqpoint{1.383121in}{0.636028in}}%
\pgfpathlineto{\pgfqpoint{1.344870in}{0.592148in}}%
\pgfpathclose%
\pgfusepath{fill}%
\end{pgfscope}%
\begin{pgfscope}%
\pgfpathrectangle{\pgfqpoint{0.150000in}{0.150000in}}{\pgfqpoint{2.700000in}{1.950000in}}%
\pgfusepath{clip}%
\pgfsetbuttcap%
\pgfsetroundjoin%
\definecolor{currentfill}{rgb}{0.941115,0.893153,0.896952}%
\pgfsetfillcolor{currentfill}%
\pgfsetlinewidth{0.000000pt}%
\definecolor{currentstroke}{rgb}{0.000000,0.000000,0.000000}%
\pgfsetstrokecolor{currentstroke}%
\pgfsetdash{}{0pt}%
\pgfpathmoveto{\pgfqpoint{1.574912in}{0.775827in}}%
\pgfpathlineto{\pgfqpoint{1.613494in}{0.835841in}}%
\pgfpathlineto{\pgfqpoint{1.574859in}{0.840125in}}%
\pgfpathlineto{\pgfqpoint{1.536486in}{0.796112in}}%
\pgfpathclose%
\pgfusepath{fill}%
\end{pgfscope}%
\begin{pgfscope}%
\pgfpathrectangle{\pgfqpoint{0.150000in}{0.150000in}}{\pgfqpoint{2.700000in}{1.950000in}}%
\pgfusepath{clip}%
\pgfsetbuttcap%
\pgfsetroundjoin%
\definecolor{currentfill}{rgb}{0.887929,0.796645,0.803876}%
\pgfsetfillcolor{currentfill}%
\pgfsetlinewidth{0.000000pt}%
\definecolor{currentstroke}{rgb}{0.000000,0.000000,0.000000}%
\pgfsetstrokecolor{currentstroke}%
\pgfsetdash{}{0pt}%
\pgfpathmoveto{\pgfqpoint{1.536486in}{0.675120in}}%
\pgfpathlineto{\pgfqpoint{1.574936in}{0.703482in}}%
\pgfpathlineto{\pgfqpoint{1.536486in}{0.723914in}}%
\pgfpathlineto{\pgfqpoint{1.498067in}{0.695626in}}%
\pgfpathclose%
\pgfusepath{fill}%
\end{pgfscope}%
\begin{pgfscope}%
\pgfpathrectangle{\pgfqpoint{0.150000in}{0.150000in}}{\pgfqpoint{2.700000in}{1.950000in}}%
\pgfusepath{clip}%
\pgfsetbuttcap%
\pgfsetroundjoin%
\definecolor{currentfill}{rgb}{0.910723,0.838006,0.843765}%
\pgfsetfillcolor{currentfill}%
\pgfsetlinewidth{0.000000pt}%
\definecolor{currentstroke}{rgb}{0.000000,0.000000,0.000000}%
\pgfsetstrokecolor{currentstroke}%
\pgfsetdash{}{0pt}%
\pgfpathmoveto{\pgfqpoint{1.574936in}{0.703482in}}%
\pgfpathlineto{\pgfqpoint{1.613541in}{0.763386in}}%
\pgfpathlineto{\pgfqpoint{1.574912in}{0.775827in}}%
\pgfpathlineto{\pgfqpoint{1.536486in}{0.723914in}}%
\pgfpathclose%
\pgfusepath{fill}%
\end{pgfscope}%
\begin{pgfscope}%
\pgfpathrectangle{\pgfqpoint{0.150000in}{0.150000in}}{\pgfqpoint{2.700000in}{1.950000in}}%
\pgfusepath{clip}%
\pgfsetbuttcap%
\pgfsetroundjoin%
\definecolor{currentfill}{rgb}{0.978232,0.980913,0.984666}%
\pgfsetfillcolor{currentfill}%
\pgfsetlinewidth{0.000000pt}%
\definecolor{currentstroke}{rgb}{0.000000,0.000000,0.000000}%
\pgfsetstrokecolor{currentstroke}%
\pgfsetdash{}{0pt}%
\pgfpathmoveto{\pgfqpoint{1.768273in}{0.948507in}}%
\pgfpathlineto{\pgfqpoint{1.805349in}{0.928276in}}%
\pgfpathlineto{\pgfqpoint{1.766333in}{0.940245in}}%
\pgfpathlineto{\pgfqpoint{1.728980in}{0.952419in}}%
\pgfpathclose%
\pgfusepath{fill}%
\end{pgfscope}%
\begin{pgfscope}%
\pgfpathrectangle{\pgfqpoint{0.150000in}{0.150000in}}{\pgfqpoint{2.700000in}{1.950000in}}%
\pgfusepath{clip}%
\pgfsetbuttcap%
\pgfsetroundjoin%
\definecolor{currentfill}{rgb}{0.975306,0.955193,0.956786}%
\pgfsetfillcolor{currentfill}%
\pgfsetlinewidth{0.000000pt}%
\definecolor{currentstroke}{rgb}{0.000000,0.000000,0.000000}%
\pgfsetstrokecolor{currentstroke}%
\pgfsetdash{}{0pt}%
\pgfpathmoveto{\pgfqpoint{1.613494in}{0.835841in}}%
\pgfpathlineto{\pgfqpoint{1.652144in}{0.888066in}}%
\pgfpathlineto{\pgfqpoint{1.613328in}{0.892177in}}%
\pgfpathlineto{\pgfqpoint{1.574859in}{0.840125in}}%
\pgfpathclose%
\pgfusepath{fill}%
\end{pgfscope}%
\begin{pgfscope}%
\pgfpathrectangle{\pgfqpoint{0.150000in}{0.150000in}}{\pgfqpoint{2.700000in}{1.950000in}}%
\pgfusepath{clip}%
\pgfsetbuttcap%
\pgfsetroundjoin%
\definecolor{currentfill}{rgb}{0.994301,0.989660,0.990028}%
\pgfsetfillcolor{currentfill}%
\pgfsetlinewidth{0.000000pt}%
\definecolor{currentstroke}{rgb}{0.000000,0.000000,0.000000}%
\pgfsetstrokecolor{currentstroke}%
\pgfsetdash{}{0pt}%
\pgfpathmoveto{\pgfqpoint{1.884273in}{0.912205in}}%
\pgfpathlineto{\pgfqpoint{1.920393in}{0.884180in}}%
\pgfpathlineto{\pgfqpoint{1.881094in}{0.896259in}}%
\pgfpathlineto{\pgfqpoint{1.844811in}{0.924290in}}%
\pgfpathclose%
\pgfusepath{fill}%
\end{pgfscope}%
\begin{pgfscope}%
\pgfpathrectangle{\pgfqpoint{0.150000in}{0.150000in}}{\pgfqpoint{2.700000in}{1.950000in}}%
\pgfusepath{clip}%
\pgfsetbuttcap%
\pgfsetroundjoin%
\definecolor{currentfill}{rgb}{0.880331,0.782858,0.790579}%
\pgfsetfillcolor{currentfill}%
\pgfsetlinewidth{0.000000pt}%
\definecolor{currentstroke}{rgb}{0.000000,0.000000,0.000000}%
\pgfsetstrokecolor{currentstroke}%
\pgfsetdash{}{0pt}%
\pgfpathmoveto{\pgfqpoint{1.497924in}{0.654536in}}%
\pgfpathlineto{\pgfqpoint{1.536486in}{0.675120in}}%
\pgfpathlineto{\pgfqpoint{1.498067in}{0.695626in}}%
\pgfpathlineto{\pgfqpoint{1.459625in}{0.659446in}}%
\pgfpathclose%
\pgfusepath{fill}%
\end{pgfscope}%
\begin{pgfscope}%
\pgfpathrectangle{\pgfqpoint{0.150000in}{0.150000in}}{\pgfqpoint{2.700000in}{1.950000in}}%
\pgfusepath{clip}%
\pgfsetbuttcap%
\pgfsetroundjoin%
\definecolor{currentfill}{rgb}{0.959574,0.964553,0.971523}%
\pgfsetfillcolor{currentfill}%
\pgfsetlinewidth{0.000000pt}%
\definecolor{currentstroke}{rgb}{0.000000,0.000000,0.000000}%
\pgfsetstrokecolor{currentstroke}%
\pgfsetdash{}{0pt}%
\pgfpathmoveto{\pgfqpoint{1.691011in}{0.948507in}}%
\pgfpathlineto{\pgfqpoint{1.728980in}{0.952419in}}%
\pgfpathlineto{\pgfqpoint{1.689956in}{0.956305in}}%
\pgfpathlineto{\pgfqpoint{1.651893in}{0.944359in}}%
\pgfpathclose%
\pgfusepath{fill}%
\end{pgfscope}%
\begin{pgfscope}%
\pgfpathrectangle{\pgfqpoint{0.150000in}{0.150000in}}{\pgfqpoint{2.700000in}{1.950000in}}%
\pgfusepath{clip}%
\pgfsetbuttcap%
\pgfsetroundjoin%
\definecolor{currentfill}{rgb}{0.887929,0.796645,0.803876}%
\pgfsetfillcolor{currentfill}%
\pgfsetlinewidth{0.000000pt}%
\definecolor{currentstroke}{rgb}{0.000000,0.000000,0.000000}%
\pgfsetstrokecolor{currentstroke}%
\pgfsetdash{}{0pt}%
\pgfpathmoveto{\pgfqpoint{1.575079in}{0.662390in}}%
\pgfpathlineto{\pgfqpoint{1.613528in}{0.682974in}}%
\pgfpathlineto{\pgfqpoint{1.574936in}{0.703482in}}%
\pgfpathlineto{\pgfqpoint{1.536486in}{0.675120in}}%
\pgfpathclose%
\pgfusepath{fill}%
\end{pgfscope}%
\begin{pgfscope}%
\pgfpathrectangle{\pgfqpoint{0.150000in}{0.150000in}}{\pgfqpoint{2.700000in}{1.950000in}}%
\pgfusepath{clip}%
\pgfsetbuttcap%
\pgfsetroundjoin%
\definecolor{currentfill}{rgb}{0.948713,0.906939,0.910248}%
\pgfsetfillcolor{currentfill}%
\pgfsetlinewidth{0.000000pt}%
\definecolor{currentstroke}{rgb}{0.000000,0.000000,0.000000}%
\pgfsetstrokecolor{currentstroke}%
\pgfsetdash{}{0pt}%
\pgfpathmoveto{\pgfqpoint{1.613541in}{0.763386in}}%
\pgfpathlineto{\pgfqpoint{1.652305in}{0.823533in}}%
\pgfpathlineto{\pgfqpoint{1.613494in}{0.835841in}}%
\pgfpathlineto{\pgfqpoint{1.574912in}{0.775827in}}%
\pgfpathclose%
\pgfusepath{fill}%
\end{pgfscope}%
\begin{pgfscope}%
\pgfpathrectangle{\pgfqpoint{0.150000in}{0.150000in}}{\pgfqpoint{2.700000in}{1.950000in}}%
\pgfusepath{clip}%
\pgfsetbuttcap%
\pgfsetroundjoin%
\definecolor{currentfill}{rgb}{0.868934,0.762178,0.770634}%
\pgfsetfillcolor{currentfill}%
\pgfsetlinewidth{0.000000pt}%
\definecolor{currentstroke}{rgb}{0.000000,0.000000,0.000000}%
\pgfsetstrokecolor{currentstroke}%
\pgfsetdash{}{0pt}%
\pgfpathmoveto{\pgfqpoint{1.459277in}{0.626036in}}%
\pgfpathlineto{\pgfqpoint{1.497924in}{0.654536in}}%
\pgfpathlineto{\pgfqpoint{1.459625in}{0.659446in}}%
\pgfpathlineto{\pgfqpoint{1.421069in}{0.631049in}}%
\pgfpathclose%
\pgfusepath{fill}%
\end{pgfscope}%
\begin{pgfscope}%
\pgfpathrectangle{\pgfqpoint{0.150000in}{0.150000in}}{\pgfqpoint{2.700000in}{1.950000in}}%
\pgfusepath{clip}%
\pgfsetbuttcap%
\pgfsetroundjoin%
\definecolor{currentfill}{rgb}{0.984452,0.986366,0.989047}%
\pgfsetfillcolor{currentfill}%
\pgfsetlinewidth{0.000000pt}%
\definecolor{currentstroke}{rgb}{0.000000,0.000000,0.000000}%
\pgfsetstrokecolor{currentstroke}%
\pgfsetdash{}{0pt}%
\pgfpathmoveto{\pgfqpoint{1.652144in}{0.888066in}}%
\pgfpathlineto{\pgfqpoint{1.691011in}{0.948507in}}%
\pgfpathlineto{\pgfqpoint{1.651893in}{0.944359in}}%
\pgfpathlineto{\pgfqpoint{1.613328in}{0.892177in}}%
\pgfpathclose%
\pgfusepath{fill}%
\end{pgfscope}%
\begin{pgfscope}%
\pgfpathrectangle{\pgfqpoint{0.150000in}{0.150000in}}{\pgfqpoint{2.700000in}{1.950000in}}%
\pgfusepath{clip}%
\pgfsetbuttcap%
\pgfsetroundjoin%
\definecolor{currentfill}{rgb}{0.910723,0.838006,0.843765}%
\pgfsetfillcolor{currentfill}%
\pgfsetlinewidth{0.000000pt}%
\definecolor{currentstroke}{rgb}{0.000000,0.000000,0.000000}%
\pgfsetstrokecolor{currentstroke}%
\pgfsetdash{}{0pt}%
\pgfpathmoveto{\pgfqpoint{1.613528in}{0.682974in}}%
\pgfpathlineto{\pgfqpoint{1.652285in}{0.742948in}}%
\pgfpathlineto{\pgfqpoint{1.613541in}{0.763386in}}%
\pgfpathlineto{\pgfqpoint{1.574936in}{0.703482in}}%
\pgfpathclose%
\pgfusepath{fill}%
\end{pgfscope}%
\begin{pgfscope}%
\pgfpathrectangle{\pgfqpoint{0.150000in}{0.150000in}}{\pgfqpoint{2.700000in}{1.950000in}}%
\pgfusepath{clip}%
\pgfsetbuttcap%
\pgfsetroundjoin%
\definecolor{currentfill}{rgb}{0.861336,0.748392,0.757338}%
\pgfsetfillcolor{currentfill}%
\pgfsetlinewidth{0.000000pt}%
\definecolor{currentstroke}{rgb}{0.000000,0.000000,0.000000}%
\pgfsetstrokecolor{currentstroke}%
\pgfsetdash{}{0pt}%
\pgfpathmoveto{\pgfqpoint{1.420546in}{0.597474in}}%
\pgfpathlineto{\pgfqpoint{1.459277in}{0.626036in}}%
\pgfpathlineto{\pgfqpoint{1.421069in}{0.631049in}}%
\pgfpathlineto{\pgfqpoint{1.382549in}{0.594800in}}%
\pgfpathclose%
\pgfusepath{fill}%
\end{pgfscope}%
\begin{pgfscope}%
\pgfpathrectangle{\pgfqpoint{0.150000in}{0.150000in}}{\pgfqpoint{2.700000in}{1.950000in}}%
\pgfusepath{clip}%
\pgfsetbuttcap%
\pgfsetroundjoin%
\definecolor{currentfill}{rgb}{0.990671,0.991820,0.993428}%
\pgfsetfillcolor{currentfill}%
\pgfsetlinewidth{0.000000pt}%
\definecolor{currentstroke}{rgb}{0.000000,0.000000,0.000000}%
\pgfsetstrokecolor{currentstroke}%
\pgfsetdash{}{0pt}%
\pgfpathmoveto{\pgfqpoint{1.924249in}{0.908140in}}%
\pgfpathlineto{\pgfqpoint{1.959901in}{0.872036in}}%
\pgfpathlineto{\pgfqpoint{1.920393in}{0.884180in}}%
\pgfpathlineto{\pgfqpoint{1.884273in}{0.912205in}}%
\pgfpathclose%
\pgfusepath{fill}%
\end{pgfscope}%
\begin{pgfscope}%
\pgfpathrectangle{\pgfqpoint{0.150000in}{0.150000in}}{\pgfqpoint{2.700000in}{1.950000in}}%
\pgfusepath{clip}%
\pgfsetbuttcap%
\pgfsetroundjoin%
\definecolor{currentfill}{rgb}{0.887929,0.796645,0.803876}%
\pgfsetfillcolor{currentfill}%
\pgfsetlinewidth{0.000000pt}%
\definecolor{currentstroke}{rgb}{0.000000,0.000000,0.000000}%
\pgfsetstrokecolor{currentstroke}%
\pgfsetdash{}{0pt}%
\pgfpathmoveto{\pgfqpoint{1.536486in}{0.641728in}}%
\pgfpathlineto{\pgfqpoint{1.575079in}{0.662390in}}%
\pgfpathlineto{\pgfqpoint{1.536486in}{0.675120in}}%
\pgfpathlineto{\pgfqpoint{1.497924in}{0.654536in}}%
\pgfpathclose%
\pgfusepath{fill}%
\end{pgfscope}%
\begin{pgfscope}%
\pgfpathrectangle{\pgfqpoint{0.150000in}{0.150000in}}{\pgfqpoint{2.700000in}{1.950000in}}%
\pgfusepath{clip}%
\pgfsetbuttcap%
\pgfsetroundjoin%
\definecolor{currentfill}{rgb}{0.990502,0.982767,0.983379}%
\pgfsetfillcolor{currentfill}%
\pgfsetlinewidth{0.000000pt}%
\definecolor{currentstroke}{rgb}{0.000000,0.000000,0.000000}%
\pgfsetstrokecolor{currentstroke}%
\pgfsetdash{}{0pt}%
\pgfpathmoveto{\pgfqpoint{1.652305in}{0.823533in}}%
\pgfpathlineto{\pgfqpoint{1.691227in}{0.883927in}}%
\pgfpathlineto{\pgfqpoint{1.652144in}{0.888066in}}%
\pgfpathlineto{\pgfqpoint{1.613494in}{0.835841in}}%
\pgfpathclose%
\pgfusepath{fill}%
\end{pgfscope}%
\begin{pgfscope}%
\pgfpathrectangle{\pgfqpoint{0.150000in}{0.150000in}}{\pgfqpoint{2.700000in}{1.950000in}}%
\pgfusepath{clip}%
\pgfsetbuttcap%
\pgfsetroundjoin%
\definecolor{currentfill}{rgb}{0.947135,0.953646,0.962760}%
\pgfsetfillcolor{currentfill}%
\pgfsetlinewidth{0.000000pt}%
\definecolor{currentstroke}{rgb}{0.000000,0.000000,0.000000}%
\pgfsetstrokecolor{currentstroke}%
\pgfsetdash{}{0pt}%
\pgfpathmoveto{\pgfqpoint{1.808050in}{0.952691in}}%
\pgfpathlineto{\pgfqpoint{1.844811in}{0.924290in}}%
\pgfpathlineto{\pgfqpoint{1.805349in}{0.928276in}}%
\pgfpathlineto{\pgfqpoint{1.768273in}{0.948507in}}%
\pgfpathclose%
\pgfusepath{fill}%
\end{pgfscope}%
\begin{pgfscope}%
\pgfpathrectangle{\pgfqpoint{0.150000in}{0.150000in}}{\pgfqpoint{2.700000in}{1.950000in}}%
\pgfusepath{clip}%
\pgfsetbuttcap%
\pgfsetroundjoin%
\definecolor{currentfill}{rgb}{0.895527,0.810432,0.817172}%
\pgfsetfillcolor{currentfill}%
\pgfsetlinewidth{0.000000pt}%
\definecolor{currentstroke}{rgb}{0.000000,0.000000,0.000000}%
\pgfsetstrokecolor{currentstroke}%
\pgfsetdash{}{0pt}%
\pgfpathmoveto{\pgfqpoint{1.613878in}{0.649592in}}%
\pgfpathlineto{\pgfqpoint{1.652356in}{0.670255in}}%
\pgfpathlineto{\pgfqpoint{1.613528in}{0.682974in}}%
\pgfpathlineto{\pgfqpoint{1.575079in}{0.662390in}}%
\pgfpathclose%
\pgfusepath{fill}%
\end{pgfscope}%
\begin{pgfscope}%
\pgfpathrectangle{\pgfqpoint{0.150000in}{0.150000in}}{\pgfqpoint{2.700000in}{1.950000in}}%
\pgfusepath{clip}%
\pgfsetbuttcap%
\pgfsetroundjoin%
\definecolor{currentfill}{rgb}{0.956311,0.920726,0.923545}%
\pgfsetfillcolor{currentfill}%
\pgfsetlinewidth{0.000000pt}%
\definecolor{currentstroke}{rgb}{0.000000,0.000000,0.000000}%
\pgfsetstrokecolor{currentstroke}%
\pgfsetdash{}{0pt}%
\pgfpathmoveto{\pgfqpoint{1.652285in}{0.742948in}}%
\pgfpathlineto{\pgfqpoint{1.691322in}{0.811160in}}%
\pgfpathlineto{\pgfqpoint{1.652305in}{0.823533in}}%
\pgfpathlineto{\pgfqpoint{1.613541in}{0.763386in}}%
\pgfpathclose%
\pgfusepath{fill}%
\end{pgfscope}%
\begin{pgfscope}%
\pgfpathrectangle{\pgfqpoint{0.150000in}{0.150000in}}{\pgfqpoint{2.700000in}{1.950000in}}%
\pgfusepath{clip}%
\pgfsetbuttcap%
\pgfsetroundjoin%
\definecolor{currentfill}{rgb}{0.914522,0.844899,0.850414}%
\pgfsetfillcolor{currentfill}%
\pgfsetlinewidth{0.000000pt}%
\definecolor{currentstroke}{rgb}{0.000000,0.000000,0.000000}%
\pgfsetstrokecolor{currentstroke}%
\pgfsetdash{}{0pt}%
\pgfpathmoveto{\pgfqpoint{1.652356in}{0.670255in}}%
\pgfpathlineto{\pgfqpoint{1.691295in}{0.730362in}}%
\pgfpathlineto{\pgfqpoint{1.652285in}{0.742948in}}%
\pgfpathlineto{\pgfqpoint{1.613528in}{0.682974in}}%
\pgfpathclose%
\pgfusepath{fill}%
\end{pgfscope}%
\begin{pgfscope}%
\pgfpathrectangle{\pgfqpoint{0.150000in}{0.150000in}}{\pgfqpoint{2.700000in}{1.950000in}}%
\pgfusepath{clip}%
\pgfsetbuttcap%
\pgfsetroundjoin%
\definecolor{currentfill}{rgb}{0.928477,0.937286,0.949617}%
\pgfsetfillcolor{currentfill}%
\pgfsetlinewidth{0.000000pt}%
\definecolor{currentstroke}{rgb}{0.000000,0.000000,0.000000}%
\pgfsetstrokecolor{currentstroke}%
\pgfsetdash{}{0pt}%
\pgfpathmoveto{\pgfqpoint{1.730308in}{0.944568in}}%
\pgfpathlineto{\pgfqpoint{1.768273in}{0.948507in}}%
\pgfpathlineto{\pgfqpoint{1.728980in}{0.952419in}}%
\pgfpathlineto{\pgfqpoint{1.691011in}{0.948507in}}%
\pgfpathclose%
\pgfusepath{fill}%
\end{pgfscope}%
\begin{pgfscope}%
\pgfpathrectangle{\pgfqpoint{0.150000in}{0.150000in}}{\pgfqpoint{2.700000in}{1.950000in}}%
\pgfusepath{clip}%
\pgfsetbuttcap%
\pgfsetroundjoin%
\definecolor{currentfill}{rgb}{0.884130,0.789752,0.797227}%
\pgfsetfillcolor{currentfill}%
\pgfsetlinewidth{0.000000pt}%
\definecolor{currentstroke}{rgb}{0.000000,0.000000,0.000000}%
\pgfsetstrokecolor{currentstroke}%
\pgfsetdash{}{0pt}%
\pgfpathmoveto{\pgfqpoint{1.497749in}{0.620988in}}%
\pgfpathlineto{\pgfqpoint{1.536486in}{0.641728in}}%
\pgfpathlineto{\pgfqpoint{1.497924in}{0.654536in}}%
\pgfpathlineto{\pgfqpoint{1.459277in}{0.626036in}}%
\pgfpathclose%
\pgfusepath{fill}%
\end{pgfscope}%
\begin{pgfscope}%
\pgfpathrectangle{\pgfqpoint{0.150000in}{0.150000in}}{\pgfqpoint{2.700000in}{1.950000in}}%
\pgfusepath{clip}%
\pgfsetbuttcap%
\pgfsetroundjoin%
\definecolor{currentfill}{rgb}{0.953355,0.959099,0.967142}%
\pgfsetfillcolor{currentfill}%
\pgfsetlinewidth{0.000000pt}%
\definecolor{currentstroke}{rgb}{0.000000,0.000000,0.000000}%
\pgfsetstrokecolor{currentstroke}%
\pgfsetdash{}{0pt}%
\pgfpathmoveto{\pgfqpoint{1.691227in}{0.883927in}}%
\pgfpathlineto{\pgfqpoint{1.730308in}{0.944568in}}%
\pgfpathlineto{\pgfqpoint{1.691011in}{0.948507in}}%
\pgfpathlineto{\pgfqpoint{1.652144in}{0.888066in}}%
\pgfpathclose%
\pgfusepath{fill}%
\end{pgfscope}%
\begin{pgfscope}%
\pgfpathrectangle{\pgfqpoint{0.150000in}{0.150000in}}{\pgfqpoint{2.700000in}{1.950000in}}%
\pgfusepath{clip}%
\pgfsetbuttcap%
\pgfsetroundjoin%
\definecolor{currentfill}{rgb}{0.895527,0.810432,0.817172}%
\pgfsetfillcolor{currentfill}%
\pgfsetlinewidth{0.000000pt}%
\definecolor{currentstroke}{rgb}{0.000000,0.000000,0.000000}%
\pgfsetstrokecolor{currentstroke}%
\pgfsetdash{}{0pt}%
\pgfpathmoveto{\pgfqpoint{1.575255in}{0.628851in}}%
\pgfpathlineto{\pgfqpoint{1.613878in}{0.649592in}}%
\pgfpathlineto{\pgfqpoint{1.575079in}{0.662390in}}%
\pgfpathlineto{\pgfqpoint{1.536486in}{0.641728in}}%
\pgfpathclose%
\pgfusepath{fill}%
\end{pgfscope}%
\begin{pgfscope}%
\pgfpathrectangle{\pgfqpoint{0.150000in}{0.150000in}}{\pgfqpoint{2.700000in}{1.950000in}}%
\pgfusepath{clip}%
\pgfsetbuttcap%
\pgfsetroundjoin%
\definecolor{currentfill}{rgb}{0.880331,0.782858,0.790579}%
\pgfsetfillcolor{currentfill}%
\pgfsetlinewidth{0.000000pt}%
\definecolor{currentstroke}{rgb}{0.000000,0.000000,0.000000}%
\pgfsetstrokecolor{currentstroke}%
\pgfsetdash{}{0pt}%
\pgfpathmoveto{\pgfqpoint{1.458926in}{0.592322in}}%
\pgfpathlineto{\pgfqpoint{1.497749in}{0.620988in}}%
\pgfpathlineto{\pgfqpoint{1.459277in}{0.626036in}}%
\pgfpathlineto{\pgfqpoint{1.420546in}{0.597474in}}%
\pgfpathclose%
\pgfusepath{fill}%
\end{pgfscope}%
\begin{pgfscope}%
\pgfpathrectangle{\pgfqpoint{0.150000in}{0.150000in}}{\pgfqpoint{2.700000in}{1.950000in}}%
\pgfusepath{clip}%
\pgfsetbuttcap%
\pgfsetroundjoin%
\definecolor{currentfill}{rgb}{0.899326,0.817325,0.823820}%
\pgfsetfillcolor{currentfill}%
\pgfsetlinewidth{0.000000pt}%
\definecolor{currentstroke}{rgb}{0.000000,0.000000,0.000000}%
\pgfsetstrokecolor{currentstroke}%
\pgfsetdash{}{0pt}%
\pgfpathmoveto{\pgfqpoint{1.652791in}{0.628851in}}%
\pgfpathlineto{\pgfqpoint{1.691269in}{0.649592in}}%
\pgfpathlineto{\pgfqpoint{1.652356in}{0.670255in}}%
\pgfpathlineto{\pgfqpoint{1.613878in}{0.649592in}}%
\pgfpathclose%
\pgfusepath{fill}%
\end{pgfscope}%
\begin{pgfscope}%
\pgfpathrectangle{\pgfqpoint{0.150000in}{0.150000in}}{\pgfqpoint{2.700000in}{1.950000in}}%
\pgfusepath{clip}%
\pgfsetbuttcap%
\pgfsetroundjoin%
\definecolor{currentfill}{rgb}{0.922120,0.858686,0.863710}%
\pgfsetfillcolor{currentfill}%
\pgfsetlinewidth{0.000000pt}%
\definecolor{currentstroke}{rgb}{0.000000,0.000000,0.000000}%
\pgfsetstrokecolor{currentstroke}%
\pgfsetdash{}{0pt}%
\pgfpathmoveto{\pgfqpoint{1.691269in}{0.649592in}}%
\pgfpathlineto{\pgfqpoint{1.730361in}{0.709769in}}%
\pgfpathlineto{\pgfqpoint{1.691295in}{0.730362in}}%
\pgfpathlineto{\pgfqpoint{1.652356in}{0.670255in}}%
\pgfpathclose%
\pgfusepath{fill}%
\end{pgfscope}%
\begin{pgfscope}%
\pgfpathrectangle{\pgfqpoint{0.150000in}{0.150000in}}{\pgfqpoint{2.700000in}{1.950000in}}%
\pgfusepath{clip}%
\pgfsetbuttcap%
\pgfsetroundjoin%
\definecolor{currentfill}{rgb}{0.996890,0.997273,0.997809}%
\pgfsetfillcolor{currentfill}%
\pgfsetlinewidth{0.000000pt}%
\definecolor{currentstroke}{rgb}{0.000000,0.000000,0.000000}%
\pgfsetstrokecolor{currentstroke}%
\pgfsetdash{}{0pt}%
\pgfpathmoveto{\pgfqpoint{1.691322in}{0.811160in}}%
\pgfpathlineto{\pgfqpoint{1.730580in}{0.879759in}}%
\pgfpathlineto{\pgfqpoint{1.691227in}{0.883927in}}%
\pgfpathlineto{\pgfqpoint{1.652305in}{0.823533in}}%
\pgfpathclose%
\pgfusepath{fill}%
\end{pgfscope}%
\begin{pgfscope}%
\pgfpathrectangle{\pgfqpoint{0.150000in}{0.150000in}}{\pgfqpoint{2.700000in}{1.950000in}}%
\pgfusepath{clip}%
\pgfsetbuttcap%
\pgfsetroundjoin%
\definecolor{currentfill}{rgb}{0.963909,0.934513,0.936841}%
\pgfsetfillcolor{currentfill}%
\pgfsetlinewidth{0.000000pt}%
\definecolor{currentstroke}{rgb}{0.000000,0.000000,0.000000}%
\pgfsetstrokecolor{currentstroke}%
\pgfsetdash{}{0pt}%
\pgfpathmoveto{\pgfqpoint{1.691295in}{0.730362in}}%
\pgfpathlineto{\pgfqpoint{1.730547in}{0.798721in}}%
\pgfpathlineto{\pgfqpoint{1.691322in}{0.811160in}}%
\pgfpathlineto{\pgfqpoint{1.652285in}{0.742948in}}%
\pgfpathclose%
\pgfusepath{fill}%
\end{pgfscope}%
\begin{pgfscope}%
\pgfpathrectangle{\pgfqpoint{0.150000in}{0.150000in}}{\pgfqpoint{2.700000in}{1.950000in}}%
\pgfusepath{clip}%
\pgfsetbuttcap%
\pgfsetroundjoin%
\definecolor{currentfill}{rgb}{0.922258,0.931832,0.945236}%
\pgfsetfillcolor{currentfill}%
\pgfsetlinewidth{0.000000pt}%
\definecolor{currentstroke}{rgb}{0.000000,0.000000,0.000000}%
\pgfsetstrokecolor{currentstroke}%
\pgfsetdash{}{0pt}%
\pgfpathmoveto{\pgfqpoint{1.847920in}{0.948749in}}%
\pgfpathlineto{\pgfqpoint{1.884273in}{0.912205in}}%
\pgfpathlineto{\pgfqpoint{1.844811in}{0.924290in}}%
\pgfpathlineto{\pgfqpoint{1.808050in}{0.952691in}}%
\pgfpathclose%
\pgfusepath{fill}%
\end{pgfscope}%
\begin{pgfscope}%
\pgfpathrectangle{\pgfqpoint{0.150000in}{0.150000in}}{\pgfqpoint{2.700000in}{1.950000in}}%
\pgfusepath{clip}%
\pgfsetbuttcap%
\pgfsetroundjoin%
\definecolor{currentfill}{rgb}{0.895527,0.810432,0.817172}%
\pgfsetfillcolor{currentfill}%
\pgfsetlinewidth{0.000000pt}%
\definecolor{currentstroke}{rgb}{0.000000,0.000000,0.000000}%
\pgfsetstrokecolor{currentstroke}%
\pgfsetdash{}{0pt}%
\pgfpathmoveto{\pgfqpoint{1.536486in}{0.608032in}}%
\pgfpathlineto{\pgfqpoint{1.575255in}{0.628851in}}%
\pgfpathlineto{\pgfqpoint{1.536486in}{0.641728in}}%
\pgfpathlineto{\pgfqpoint{1.497749in}{0.620988in}}%
\pgfpathclose%
\pgfusepath{fill}%
\end{pgfscope}%
\begin{pgfscope}%
\pgfpathrectangle{\pgfqpoint{0.150000in}{0.150000in}}{\pgfqpoint{2.700000in}{1.950000in}}%
\pgfusepath{clip}%
\pgfsetbuttcap%
\pgfsetroundjoin%
\definecolor{currentfill}{rgb}{0.899326,0.817325,0.823820}%
\pgfsetfillcolor{currentfill}%
\pgfsetlinewidth{0.000000pt}%
\definecolor{currentstroke}{rgb}{0.000000,0.000000,0.000000}%
\pgfsetstrokecolor{currentstroke}%
\pgfsetdash{}{0pt}%
\pgfpathmoveto{\pgfqpoint{1.614169in}{0.608032in}}%
\pgfpathlineto{\pgfqpoint{1.652791in}{0.628851in}}%
\pgfpathlineto{\pgfqpoint{1.613878in}{0.649592in}}%
\pgfpathlineto{\pgfqpoint{1.575255in}{0.628851in}}%
\pgfpathclose%
\pgfusepath{fill}%
\end{pgfscope}%
\begin{pgfscope}%
\pgfpathrectangle{\pgfqpoint{0.150000in}{0.150000in}}{\pgfqpoint{2.700000in}{1.950000in}}%
\pgfusepath{clip}%
\pgfsetbuttcap%
\pgfsetroundjoin%
\definecolor{currentfill}{rgb}{0.897381,0.910018,0.927711}%
\pgfsetfillcolor{currentfill}%
\pgfsetlinewidth{0.000000pt}%
\definecolor{currentstroke}{rgb}{0.000000,0.000000,0.000000}%
\pgfsetstrokecolor{currentstroke}%
\pgfsetdash{}{0pt}%
\pgfpathmoveto{\pgfqpoint{1.769877in}{0.940602in}}%
\pgfpathlineto{\pgfqpoint{1.808050in}{0.952691in}}%
\pgfpathlineto{\pgfqpoint{1.768273in}{0.948507in}}%
\pgfpathlineto{\pgfqpoint{1.730308in}{0.944568in}}%
\pgfpathclose%
\pgfusepath{fill}%
\end{pgfscope}%
\begin{pgfscope}%
\pgfpathrectangle{\pgfqpoint{0.150000in}{0.150000in}}{\pgfqpoint{2.700000in}{1.950000in}}%
\pgfusepath{clip}%
\pgfsetbuttcap%
\pgfsetroundjoin%
\definecolor{currentfill}{rgb}{0.922258,0.931832,0.945236}%
\pgfsetfillcolor{currentfill}%
\pgfsetlinewidth{0.000000pt}%
\definecolor{currentstroke}{rgb}{0.000000,0.000000,0.000000}%
\pgfsetstrokecolor{currentstroke}%
\pgfsetdash{}{0pt}%
\pgfpathmoveto{\pgfqpoint{1.730580in}{0.879759in}}%
\pgfpathlineto{\pgfqpoint{1.769877in}{0.940602in}}%
\pgfpathlineto{\pgfqpoint{1.730308in}{0.944568in}}%
\pgfpathlineto{\pgfqpoint{1.691227in}{0.883927in}}%
\pgfpathclose%
\pgfusepath{fill}%
\end{pgfscope}%
\begin{pgfscope}%
\pgfpathrectangle{\pgfqpoint{0.150000in}{0.150000in}}{\pgfqpoint{2.700000in}{1.950000in}}%
\pgfusepath{clip}%
\pgfsetbuttcap%
\pgfsetroundjoin%
\definecolor{currentfill}{rgb}{0.967708,0.941406,0.943490}%
\pgfsetfillcolor{currentfill}%
\pgfsetlinewidth{0.000000pt}%
\definecolor{currentstroke}{rgb}{0.000000,0.000000,0.000000}%
\pgfsetstrokecolor{currentstroke}%
\pgfsetdash{}{0pt}%
\pgfpathmoveto{\pgfqpoint{1.730361in}{0.709769in}}%
\pgfpathlineto{\pgfqpoint{1.769981in}{0.786215in}}%
\pgfpathlineto{\pgfqpoint{1.730547in}{0.798721in}}%
\pgfpathlineto{\pgfqpoint{1.691295in}{0.730362in}}%
\pgfpathclose%
\pgfusepath{fill}%
\end{pgfscope}%
\begin{pgfscope}%
\pgfpathrectangle{\pgfqpoint{0.150000in}{0.150000in}}{\pgfqpoint{2.700000in}{1.950000in}}%
\pgfusepath{clip}%
\pgfsetbuttcap%
\pgfsetroundjoin%
\definecolor{currentfill}{rgb}{0.891728,0.803539,0.810524}%
\pgfsetfillcolor{currentfill}%
\pgfsetlinewidth{0.000000pt}%
\definecolor{currentstroke}{rgb}{0.000000,0.000000,0.000000}%
\pgfsetstrokecolor{currentstroke}%
\pgfsetdash{}{0pt}%
\pgfpathmoveto{\pgfqpoint{1.497572in}{0.587135in}}%
\pgfpathlineto{\pgfqpoint{1.536486in}{0.608032in}}%
\pgfpathlineto{\pgfqpoint{1.497749in}{0.620988in}}%
\pgfpathlineto{\pgfqpoint{1.458926in}{0.592322in}}%
\pgfpathclose%
\pgfusepath{fill}%
\end{pgfscope}%
\begin{pgfscope}%
\pgfpathrectangle{\pgfqpoint{0.150000in}{0.150000in}}{\pgfqpoint{2.700000in}{1.950000in}}%
\pgfusepath{clip}%
\pgfsetbuttcap%
\pgfsetroundjoin%
\definecolor{currentfill}{rgb}{0.978232,0.980913,0.984666}%
\pgfsetfillcolor{currentfill}%
\pgfsetlinewidth{0.000000pt}%
\definecolor{currentstroke}{rgb}{0.000000,0.000000,0.000000}%
\pgfsetstrokecolor{currentstroke}%
\pgfsetdash{}{0pt}%
\pgfpathmoveto{\pgfqpoint{1.730547in}{0.798721in}}%
\pgfpathlineto{\pgfqpoint{1.770206in}{0.875563in}}%
\pgfpathlineto{\pgfqpoint{1.730580in}{0.879759in}}%
\pgfpathlineto{\pgfqpoint{1.691322in}{0.811160in}}%
\pgfpathclose%
\pgfusepath{fill}%
\end{pgfscope}%
\begin{pgfscope}%
\pgfpathrectangle{\pgfqpoint{0.150000in}{0.150000in}}{\pgfqpoint{2.700000in}{1.950000in}}%
\pgfusepath{clip}%
\pgfsetbuttcap%
\pgfsetroundjoin%
\definecolor{currentfill}{rgb}{0.899326,0.817325,0.823820}%
\pgfsetfillcolor{currentfill}%
\pgfsetlinewidth{0.000000pt}%
\definecolor{currentstroke}{rgb}{0.000000,0.000000,0.000000}%
\pgfsetstrokecolor{currentstroke}%
\pgfsetdash{}{0pt}%
\pgfpathmoveto{\pgfqpoint{1.575401in}{0.587135in}}%
\pgfpathlineto{\pgfqpoint{1.614169in}{0.608032in}}%
\pgfpathlineto{\pgfqpoint{1.575255in}{0.628851in}}%
\pgfpathlineto{\pgfqpoint{1.536486in}{0.608032in}}%
\pgfpathclose%
\pgfusepath{fill}%
\end{pgfscope}%
\begin{pgfscope}%
\pgfpathrectangle{\pgfqpoint{0.150000in}{0.150000in}}{\pgfqpoint{2.700000in}{1.950000in}}%
\pgfusepath{clip}%
\pgfsetbuttcap%
\pgfsetroundjoin%
\definecolor{currentfill}{rgb}{0.897381,0.910018,0.927711}%
\pgfsetfillcolor{currentfill}%
\pgfsetlinewidth{0.000000pt}%
\definecolor{currentstroke}{rgb}{0.000000,0.000000,0.000000}%
\pgfsetstrokecolor{currentstroke}%
\pgfsetdash{}{0pt}%
\pgfpathmoveto{\pgfqpoint{1.888066in}{0.944781in}}%
\pgfpathlineto{\pgfqpoint{1.924249in}{0.908140in}}%
\pgfpathlineto{\pgfqpoint{1.884273in}{0.912205in}}%
\pgfpathlineto{\pgfqpoint{1.847920in}{0.948749in}}%
\pgfpathclose%
\pgfusepath{fill}%
\end{pgfscope}%
\begin{pgfscope}%
\pgfpathrectangle{\pgfqpoint{0.150000in}{0.150000in}}{\pgfqpoint{2.700000in}{1.950000in}}%
\pgfusepath{clip}%
\pgfsetbuttcap%
\pgfsetroundjoin%
\definecolor{currentfill}{rgb}{0.899326,0.817325,0.823820}%
\pgfsetfillcolor{currentfill}%
\pgfsetlinewidth{0.000000pt}%
\definecolor{currentstroke}{rgb}{0.000000,0.000000,0.000000}%
\pgfsetstrokecolor{currentstroke}%
\pgfsetdash{}{0pt}%
\pgfpathmoveto{\pgfqpoint{1.536486in}{0.566158in}}%
\pgfpathlineto{\pgfqpoint{1.575401in}{0.587135in}}%
\pgfpathlineto{\pgfqpoint{1.536486in}{0.608032in}}%
\pgfpathlineto{\pgfqpoint{1.497572in}{0.587135in}}%
\pgfpathclose%
\pgfusepath{fill}%
\end{pgfscope}%
\begin{pgfscope}%
\pgfpathrectangle{\pgfqpoint{0.150000in}{0.150000in}}{\pgfqpoint{2.700000in}{1.950000in}}%
\pgfusepath{clip}%
\pgfsetbuttcap%
\pgfsetroundjoin%
\definecolor{currentfill}{rgb}{0.866284,0.882751,0.905806}%
\pgfsetfillcolor{currentfill}%
\pgfsetlinewidth{0.000000pt}%
\definecolor{currentstroke}{rgb}{0.000000,0.000000,0.000000}%
\pgfsetstrokecolor{currentstroke}%
\pgfsetdash{}{0pt}%
\pgfpathmoveto{\pgfqpoint{1.809938in}{0.944781in}}%
\pgfpathlineto{\pgfqpoint{1.847920in}{0.948749in}}%
\pgfpathlineto{\pgfqpoint{1.808050in}{0.952691in}}%
\pgfpathlineto{\pgfqpoint{1.769877in}{0.940602in}}%
\pgfpathclose%
\pgfusepath{fill}%
\end{pgfscope}%
\begin{pgfscope}%
\pgfpathrectangle{\pgfqpoint{0.150000in}{0.150000in}}{\pgfqpoint{2.700000in}{1.950000in}}%
\pgfusepath{clip}%
\pgfsetbuttcap%
\pgfsetroundjoin%
\definecolor{currentfill}{rgb}{0.891161,0.904565,0.923330}%
\pgfsetfillcolor{currentfill}%
\pgfsetlinewidth{0.000000pt}%
\definecolor{currentstroke}{rgb}{0.000000,0.000000,0.000000}%
\pgfsetstrokecolor{currentstroke}%
\pgfsetdash{}{0pt}%
\pgfpathmoveto{\pgfqpoint{1.770206in}{0.875563in}}%
\pgfpathlineto{\pgfqpoint{1.809938in}{0.944781in}}%
\pgfpathlineto{\pgfqpoint{1.769877in}{0.940602in}}%
\pgfpathlineto{\pgfqpoint{1.730580in}{0.879759in}}%
\pgfpathclose%
\pgfusepath{fill}%
\end{pgfscope}%
\begin{pgfscope}%
\pgfpathrectangle{\pgfqpoint{0.150000in}{0.150000in}}{\pgfqpoint{2.700000in}{1.950000in}}%
\pgfusepath{clip}%
\pgfsetbuttcap%
\pgfsetroundjoin%
\definecolor{currentfill}{rgb}{0.959574,0.964553,0.971523}%
\pgfsetfillcolor{currentfill}%
\pgfsetlinewidth{0.000000pt}%
\definecolor{currentstroke}{rgb}{0.000000,0.000000,0.000000}%
\pgfsetstrokecolor{currentstroke}%
\pgfsetdash{}{0pt}%
\pgfpathmoveto{\pgfqpoint{1.769981in}{0.786215in}}%
\pgfpathlineto{\pgfqpoint{1.809891in}{0.863218in}}%
\pgfpathlineto{\pgfqpoint{1.770206in}{0.875563in}}%
\pgfpathlineto{\pgfqpoint{1.730547in}{0.798721in}}%
\pgfpathclose%
\pgfusepath{fill}%
\end{pgfscope}%
\begin{pgfscope}%
\pgfpathrectangle{\pgfqpoint{0.150000in}{0.150000in}}{\pgfqpoint{2.700000in}{1.950000in}}%
\pgfusepath{clip}%
\pgfsetbuttcap%
\pgfsetroundjoin%
\definecolor{currentfill}{rgb}{0.835187,0.855484,0.883900}%
\pgfsetfillcolor{currentfill}%
\pgfsetlinewidth{0.000000pt}%
\definecolor{currentstroke}{rgb}{0.000000,0.000000,0.000000}%
\pgfsetstrokecolor{currentstroke}%
\pgfsetdash{}{0pt}%
\pgfpathmoveto{\pgfqpoint{1.850092in}{0.940784in}}%
\pgfpathlineto{\pgfqpoint{1.888066in}{0.944781in}}%
\pgfpathlineto{\pgfqpoint{1.847920in}{0.948749in}}%
\pgfpathlineto{\pgfqpoint{1.809938in}{0.944781in}}%
\pgfpathclose%
\pgfusepath{fill}%
\end{pgfscope}%
\begin{pgfscope}%
\pgfpathrectangle{\pgfqpoint{0.150000in}{0.150000in}}{\pgfqpoint{2.700000in}{1.950000in}}%
\pgfusepath{clip}%
\pgfsetbuttcap%
\pgfsetroundjoin%
\definecolor{currentfill}{rgb}{0.866284,0.882751,0.905806}%
\pgfsetfillcolor{currentfill}%
\pgfsetlinewidth{0.000000pt}%
\definecolor{currentstroke}{rgb}{0.000000,0.000000,0.000000}%
\pgfsetstrokecolor{currentstroke}%
\pgfsetdash{}{0pt}%
\pgfpathmoveto{\pgfqpoint{1.809891in}{0.863218in}}%
\pgfpathlineto{\pgfqpoint{1.850092in}{0.940784in}}%
\pgfpathlineto{\pgfqpoint{1.809938in}{0.944781in}}%
\pgfpathlineto{\pgfqpoint{1.770206in}{0.875563in}}%
\pgfpathclose%
\pgfusepath{fill}%
\end{pgfscope}%
\end{pgfpicture}%
\makeatother%
\endgroup%
}
            \hfill
            \subbottom[\label{fig:parameterised-incompetent-games-d}]%
                {%% Creator: Matplotlib, PGF backend
%%
%% To include the figure in your LaTeX document, write
%%   \input{<filename>.pgf}
%%
%% Make sure the required packages are loaded in your preamble
%%   \usepackage{pgf}
%%
%% Figures using additional raster images can only be included by \input if
%% they are in the same directory as the main LaTeX file. For loading figures
%% from other directories you can use the `import` package
%%   \usepackage{import}
%% and then include the figures with
%%   \import{<path to file>}{<filename>.pgf}
%%
%% Matplotlib used the following preamble
%%   \usepackage{fontspec}
%%   \setmainfont{DejaVuSerif.ttf}[Path=C:/Users/Thomas/anaconda3/lib/site-packages/matplotlib/mpl-data/fonts/ttf/]
%%   \setsansfont{DejaVuSans.ttf}[Path=C:/Users/Thomas/anaconda3/lib/site-packages/matplotlib/mpl-data/fonts/ttf/]
%%   \setmonofont{DejaVuSansMono.ttf}[Path=C:/Users/Thomas/anaconda3/lib/site-packages/matplotlib/mpl-data/fonts/ttf/]
%%
\begingroup%
\makeatletter%
\begin{pgfpicture}%
\pgfpathrectangle{\pgfpointorigin}{\pgfqpoint{3.000000in}{2.250000in}}%
\pgfusepath{use as bounding box, clip}%
\begin{pgfscope}%
\pgfsetbuttcap%
\pgfsetmiterjoin%
\definecolor{currentfill}{rgb}{1.000000,1.000000,1.000000}%
\pgfsetfillcolor{currentfill}%
\pgfsetlinewidth{0.000000pt}%
\definecolor{currentstroke}{rgb}{1.000000,1.000000,1.000000}%
\pgfsetstrokecolor{currentstroke}%
\pgfsetdash{}{0pt}%
\pgfpathmoveto{\pgfqpoint{0.000000in}{0.000000in}}%
\pgfpathlineto{\pgfqpoint{3.000000in}{0.000000in}}%
\pgfpathlineto{\pgfqpoint{3.000000in}{2.250000in}}%
\pgfpathlineto{\pgfqpoint{0.000000in}{2.250000in}}%
\pgfpathclose%
\pgfusepath{fill}%
\end{pgfscope}%
\begin{pgfscope}%
\pgfsetbuttcap%
\pgfsetmiterjoin%
\definecolor{currentfill}{rgb}{1.000000,1.000000,1.000000}%
\pgfsetfillcolor{currentfill}%
\pgfsetlinewidth{0.000000pt}%
\definecolor{currentstroke}{rgb}{0.000000,0.000000,0.000000}%
\pgfsetstrokecolor{currentstroke}%
\pgfsetstrokeopacity{0.000000}%
\pgfsetdash{}{0pt}%
\pgfpathmoveto{\pgfqpoint{0.150000in}{0.150000in}}%
\pgfpathlineto{\pgfqpoint{2.850000in}{0.150000in}}%
\pgfpathlineto{\pgfqpoint{2.850000in}{2.100000in}}%
\pgfpathlineto{\pgfqpoint{0.150000in}{2.100000in}}%
\pgfpathclose%
\pgfusepath{fill}%
\end{pgfscope}%
\begin{pgfscope}%
\pgfsetbuttcap%
\pgfsetmiterjoin%
\definecolor{currentfill}{rgb}{0.950000,0.950000,0.950000}%
\pgfsetfillcolor{currentfill}%
\pgfsetfillopacity{0.500000}%
\pgfsetlinewidth{1.003750pt}%
\definecolor{currentstroke}{rgb}{0.950000,0.950000,0.950000}%
\pgfsetstrokecolor{currentstroke}%
\pgfsetstrokeopacity{0.500000}%
\pgfsetdash{}{0pt}%
\pgfpathmoveto{\pgfqpoint{2.573296in}{0.776948in}}%
\pgfpathlineto{\pgfqpoint{1.536486in}{1.299017in}}%
\pgfpathlineto{\pgfqpoint{1.536486in}{2.074448in}}%
\pgfpathlineto{\pgfqpoint{2.652584in}{1.554387in}}%
\pgfusepath{stroke,fill}%
\end{pgfscope}%
\begin{pgfscope}%
\pgfsetbuttcap%
\pgfsetmiterjoin%
\definecolor{currentfill}{rgb}{0.900000,0.900000,0.900000}%
\pgfsetfillcolor{currentfill}%
\pgfsetfillopacity{0.500000}%
\pgfsetlinewidth{1.003750pt}%
\definecolor{currentstroke}{rgb}{0.900000,0.900000,0.900000}%
\pgfsetstrokecolor{currentstroke}%
\pgfsetstrokeopacity{0.500000}%
\pgfsetdash{}{0pt}%
\pgfpathmoveto{\pgfqpoint{0.499677in}{0.776948in}}%
\pgfpathlineto{\pgfqpoint{1.536486in}{1.299017in}}%
\pgfpathlineto{\pgfqpoint{1.536486in}{2.074448in}}%
\pgfpathlineto{\pgfqpoint{0.420389in}{1.554387in}}%
\pgfusepath{stroke,fill}%
\end{pgfscope}%
\begin{pgfscope}%
\pgfsetbuttcap%
\pgfsetmiterjoin%
\definecolor{currentfill}{rgb}{0.925000,0.925000,0.925000}%
\pgfsetfillcolor{currentfill}%
\pgfsetfillopacity{0.500000}%
\pgfsetlinewidth{1.003750pt}%
\definecolor{currentstroke}{rgb}{0.925000,0.925000,0.925000}%
\pgfsetstrokecolor{currentstroke}%
\pgfsetstrokeopacity{0.500000}%
\pgfsetdash{}{0pt}%
\pgfpathmoveto{\pgfqpoint{1.536486in}{0.199655in}}%
\pgfpathlineto{\pgfqpoint{2.573296in}{0.776948in}}%
\pgfpathlineto{\pgfqpoint{1.536486in}{1.299017in}}%
\pgfpathlineto{\pgfqpoint{0.499677in}{0.776948in}}%
\pgfusepath{stroke,fill}%
\end{pgfscope}%
\begin{pgfscope}%
\pgfsetrectcap%
\pgfsetroundjoin%
\pgfsetlinewidth{0.803000pt}%
\definecolor{currentstroke}{rgb}{0.000000,0.000000,0.000000}%
\pgfsetstrokecolor{currentstroke}%
\pgfsetdash{}{0pt}%
\pgfpathmoveto{\pgfqpoint{2.573296in}{0.776948in}}%
\pgfpathlineto{\pgfqpoint{1.536486in}{0.199655in}}%
\pgfusepath{stroke}%
\end{pgfscope}%
\begin{pgfscope}%
\definecolor{textcolor}{rgb}{0.000000,0.000000,0.000000}%
\pgfsetstrokecolor{textcolor}%
\pgfsetfillcolor{textcolor}%
\pgftext[x=2.017747in,y=0.045475in,left,base,rotate=29.108966]{\color{textcolor}\sffamily\fontsize{8.000000}{9.600000}\selectfont Player 2 (\(\displaystyle \mu\))}%
\end{pgfscope}%
\begin{pgfscope}%
\pgfsetbuttcap%
\pgfsetroundjoin%
\pgfsetlinewidth{0.803000pt}%
\definecolor{currentstroke}{rgb}{0.690196,0.690196,0.690196}%
\pgfsetstrokecolor{currentstroke}%
\pgfsetdash{}{0pt}%
\pgfpathmoveto{\pgfqpoint{1.605722in}{0.238205in}}%
\pgfpathlineto{\pgfqpoint{0.568749in}{0.811728in}}%
\pgfpathlineto{\pgfqpoint{0.494997in}{1.589151in}}%
\pgfusepath{stroke}%
\end{pgfscope}%
\begin{pgfscope}%
\pgfsetbuttcap%
\pgfsetroundjoin%
\pgfsetlinewidth{0.803000pt}%
\definecolor{currentstroke}{rgb}{0.690196,0.690196,0.690196}%
\pgfsetstrokecolor{currentstroke}%
\pgfsetdash{}{0pt}%
\pgfpathmoveto{\pgfqpoint{1.793262in}{0.342627in}}%
\pgfpathlineto{\pgfqpoint{0.755965in}{0.905998in}}%
\pgfpathlineto{\pgfqpoint{0.697035in}{1.683294in}}%
\pgfusepath{stroke}%
\end{pgfscope}%
\begin{pgfscope}%
\pgfsetbuttcap%
\pgfsetroundjoin%
\pgfsetlinewidth{0.803000pt}%
\definecolor{currentstroke}{rgb}{0.690196,0.690196,0.690196}%
\pgfsetstrokecolor{currentstroke}%
\pgfsetdash{}{0pt}%
\pgfpathmoveto{\pgfqpoint{1.977414in}{0.445162in}}%
\pgfpathlineto{\pgfqpoint{0.939964in}{0.998647in}}%
\pgfpathlineto{\pgfqpoint{0.895342in}{1.775698in}}%
\pgfusepath{stroke}%
\end{pgfscope}%
\begin{pgfscope}%
\pgfsetbuttcap%
\pgfsetroundjoin%
\pgfsetlinewidth{0.803000pt}%
\definecolor{currentstroke}{rgb}{0.690196,0.690196,0.690196}%
\pgfsetstrokecolor{currentstroke}%
\pgfsetdash{}{0pt}%
\pgfpathmoveto{\pgfqpoint{2.158267in}{0.545861in}}%
\pgfpathlineto{\pgfqpoint{1.120829in}{1.089719in}}%
\pgfpathlineto{\pgfqpoint{1.090021in}{1.866411in}}%
\pgfusepath{stroke}%
\end{pgfscope}%
\begin{pgfscope}%
\pgfsetbuttcap%
\pgfsetroundjoin%
\pgfsetlinewidth{0.803000pt}%
\definecolor{currentstroke}{rgb}{0.690196,0.690196,0.690196}%
\pgfsetstrokecolor{currentstroke}%
\pgfsetdash{}{0pt}%
\pgfpathmoveto{\pgfqpoint{2.335912in}{0.644773in}}%
\pgfpathlineto{\pgfqpoint{1.298639in}{1.179253in}}%
\pgfpathlineto{\pgfqpoint{1.281170in}{1.955480in}}%
\pgfusepath{stroke}%
\end{pgfscope}%
\begin{pgfscope}%
\pgfsetbuttcap%
\pgfsetroundjoin%
\pgfsetlinewidth{0.803000pt}%
\definecolor{currentstroke}{rgb}{0.690196,0.690196,0.690196}%
\pgfsetstrokecolor{currentstroke}%
\pgfsetdash{}{0pt}%
\pgfpathmoveto{\pgfqpoint{2.510430in}{0.741945in}}%
\pgfpathlineto{\pgfqpoint{1.473472in}{1.267287in}}%
\pgfpathlineto{\pgfqpoint{1.468885in}{2.042948in}}%
\pgfusepath{stroke}%
\end{pgfscope}%
\begin{pgfscope}%
\pgfsetrectcap%
\pgfsetroundjoin%
\pgfsetlinewidth{0.803000pt}%
\definecolor{currentstroke}{rgb}{0.000000,0.000000,0.000000}%
\pgfsetstrokecolor{currentstroke}%
\pgfsetdash{}{0pt}%
\pgfpathmoveto{\pgfqpoint{1.596992in}{0.243033in}}%
\pgfpathlineto{\pgfqpoint{1.623203in}{0.228537in}}%
\pgfusepath{stroke}%
\end{pgfscope}%
\begin{pgfscope}%
\definecolor{textcolor}{rgb}{0.000000,0.000000,0.000000}%
\pgfsetstrokecolor{textcolor}%
\pgfsetfillcolor{textcolor}%
\pgftext[x=1.680378in,y=0.147403in,,top]{\color{textcolor}\sffamily\fontsize{6.000000}{7.200000}\selectfont \(\displaystyle 0.0\)}%
\end{pgfscope}%
\begin{pgfscope}%
\pgfsetrectcap%
\pgfsetroundjoin%
\pgfsetlinewidth{0.803000pt}%
\definecolor{currentstroke}{rgb}{0.000000,0.000000,0.000000}%
\pgfsetstrokecolor{currentstroke}%
\pgfsetdash{}{0pt}%
\pgfpathmoveto{\pgfqpoint{1.784534in}{0.347367in}}%
\pgfpathlineto{\pgfqpoint{1.810740in}{0.333134in}}%
\pgfusepath{stroke}%
\end{pgfscope}%
\begin{pgfscope}%
\definecolor{textcolor}{rgb}{0.000000,0.000000,0.000000}%
\pgfsetstrokecolor{textcolor}%
\pgfsetfillcolor{textcolor}%
\pgftext[x=1.866959in,y=0.252496in,,top]{\color{textcolor}\sffamily\fontsize{6.000000}{7.200000}\selectfont \(\displaystyle 0.2\)}%
\end{pgfscope}%
\begin{pgfscope}%
\pgfsetrectcap%
\pgfsetroundjoin%
\pgfsetlinewidth{0.803000pt}%
\definecolor{currentstroke}{rgb}{0.000000,0.000000,0.000000}%
\pgfsetstrokecolor{currentstroke}%
\pgfsetdash{}{0pt}%
\pgfpathmoveto{\pgfqpoint{1.968688in}{0.449817in}}%
\pgfpathlineto{\pgfqpoint{1.994886in}{0.435840in}}%
\pgfusepath{stroke}%
\end{pgfscope}%
\begin{pgfscope}%
\definecolor{textcolor}{rgb}{0.000000,0.000000,0.000000}%
\pgfsetstrokecolor{textcolor}%
\pgfsetfillcolor{textcolor}%
\pgftext[x=2.050175in,y=0.355693in,,top]{\color{textcolor}\sffamily\fontsize{6.000000}{7.200000}\selectfont \(\displaystyle 0.4\)}%
\end{pgfscope}%
\begin{pgfscope}%
\pgfsetrectcap%
\pgfsetroundjoin%
\pgfsetlinewidth{0.803000pt}%
\definecolor{currentstroke}{rgb}{0.000000,0.000000,0.000000}%
\pgfsetstrokecolor{currentstroke}%
\pgfsetdash{}{0pt}%
\pgfpathmoveto{\pgfqpoint{2.149546in}{0.550433in}}%
\pgfpathlineto{\pgfqpoint{2.175732in}{0.536706in}}%
\pgfusepath{stroke}%
\end{pgfscope}%
\begin{pgfscope}%
\definecolor{textcolor}{rgb}{0.000000,0.000000,0.000000}%
\pgfsetstrokecolor{textcolor}%
\pgfsetfillcolor{textcolor}%
\pgftext[x=2.230114in,y=0.457045in,,top]{\color{textcolor}\sffamily\fontsize{6.000000}{7.200000}\selectfont \(\displaystyle 0.6\)}%
\end{pgfscope}%
\begin{pgfscope}%
\pgfsetrectcap%
\pgfsetroundjoin%
\pgfsetlinewidth{0.803000pt}%
\definecolor{currentstroke}{rgb}{0.000000,0.000000,0.000000}%
\pgfsetstrokecolor{currentstroke}%
\pgfsetdash{}{0pt}%
\pgfpathmoveto{\pgfqpoint{2.327195in}{0.649264in}}%
\pgfpathlineto{\pgfqpoint{2.353366in}{0.635779in}}%
\pgfusepath{stroke}%
\end{pgfscope}%
\begin{pgfscope}%
\definecolor{textcolor}{rgb}{0.000000,0.000000,0.000000}%
\pgfsetstrokecolor{textcolor}%
\pgfsetfillcolor{textcolor}%
\pgftext[x=2.406864in,y=0.556601in,,top]{\color{textcolor}\sffamily\fontsize{6.000000}{7.200000}\selectfont \(\displaystyle 0.8\)}%
\end{pgfscope}%
\begin{pgfscope}%
\pgfsetrectcap%
\pgfsetroundjoin%
\pgfsetlinewidth{0.803000pt}%
\definecolor{currentstroke}{rgb}{0.000000,0.000000,0.000000}%
\pgfsetstrokecolor{currentstroke}%
\pgfsetdash{}{0pt}%
\pgfpathmoveto{\pgfqpoint{2.501720in}{0.746357in}}%
\pgfpathlineto{\pgfqpoint{2.527872in}{0.733108in}}%
\pgfusepath{stroke}%
\end{pgfscope}%
\begin{pgfscope}%
\definecolor{textcolor}{rgb}{0.000000,0.000000,0.000000}%
\pgfsetstrokecolor{textcolor}%
\pgfsetfillcolor{textcolor}%
\pgftext[x=2.580510in,y=0.654408in,,top]{\color{textcolor}\sffamily\fontsize{6.000000}{7.200000}\selectfont \(\displaystyle 1.0\)}%
\end{pgfscope}%
\begin{pgfscope}%
\pgfsetrectcap%
\pgfsetroundjoin%
\pgfsetlinewidth{0.803000pt}%
\definecolor{currentstroke}{rgb}{0.000000,0.000000,0.000000}%
\pgfsetstrokecolor{currentstroke}%
\pgfsetdash{}{0pt}%
\pgfpathmoveto{\pgfqpoint{0.499677in}{0.776948in}}%
\pgfpathlineto{\pgfqpoint{1.536486in}{0.199655in}}%
\pgfusepath{stroke}%
\end{pgfscope}%
\begin{pgfscope}%
\definecolor{textcolor}{rgb}{0.000000,0.000000,0.000000}%
\pgfsetstrokecolor{textcolor}%
\pgfsetfillcolor{textcolor}%
\pgftext[x=0.492803in,y=0.358631in,left,base,rotate=330.891034]{\color{textcolor}\sffamily\fontsize{8.000000}{9.600000}\selectfont Player 1 (\(\displaystyle \lambda\))}%
\end{pgfscope}%
\begin{pgfscope}%
\pgfsetbuttcap%
\pgfsetroundjoin%
\pgfsetlinewidth{0.803000pt}%
\definecolor{currentstroke}{rgb}{0.690196,0.690196,0.690196}%
\pgfsetstrokecolor{currentstroke}%
\pgfsetdash{}{0pt}%
\pgfpathmoveto{\pgfqpoint{2.577976in}{1.589151in}}%
\pgfpathlineto{\pgfqpoint{2.504223in}{0.811728in}}%
\pgfpathlineto{\pgfqpoint{1.467251in}{0.238205in}}%
\pgfusepath{stroke}%
\end{pgfscope}%
\begin{pgfscope}%
\pgfsetbuttcap%
\pgfsetroundjoin%
\pgfsetlinewidth{0.803000pt}%
\definecolor{currentstroke}{rgb}{0.690196,0.690196,0.690196}%
\pgfsetstrokecolor{currentstroke}%
\pgfsetdash{}{0pt}%
\pgfpathmoveto{\pgfqpoint{2.375938in}{1.683294in}}%
\pgfpathlineto{\pgfqpoint{2.317008in}{0.905998in}}%
\pgfpathlineto{\pgfqpoint{1.279711in}{0.342627in}}%
\pgfusepath{stroke}%
\end{pgfscope}%
\begin{pgfscope}%
\pgfsetbuttcap%
\pgfsetroundjoin%
\pgfsetlinewidth{0.803000pt}%
\definecolor{currentstroke}{rgb}{0.690196,0.690196,0.690196}%
\pgfsetstrokecolor{currentstroke}%
\pgfsetdash{}{0pt}%
\pgfpathmoveto{\pgfqpoint{2.177631in}{1.775698in}}%
\pgfpathlineto{\pgfqpoint{2.133009in}{0.998647in}}%
\pgfpathlineto{\pgfqpoint{1.095559in}{0.445162in}}%
\pgfusepath{stroke}%
\end{pgfscope}%
\begin{pgfscope}%
\pgfsetbuttcap%
\pgfsetroundjoin%
\pgfsetlinewidth{0.803000pt}%
\definecolor{currentstroke}{rgb}{0.690196,0.690196,0.690196}%
\pgfsetstrokecolor{currentstroke}%
\pgfsetdash{}{0pt}%
\pgfpathmoveto{\pgfqpoint{1.982952in}{1.866411in}}%
\pgfpathlineto{\pgfqpoint{1.952144in}{1.089719in}}%
\pgfpathlineto{\pgfqpoint{0.914705in}{0.545861in}}%
\pgfusepath{stroke}%
\end{pgfscope}%
\begin{pgfscope}%
\pgfsetbuttcap%
\pgfsetroundjoin%
\pgfsetlinewidth{0.803000pt}%
\definecolor{currentstroke}{rgb}{0.690196,0.690196,0.690196}%
\pgfsetstrokecolor{currentstroke}%
\pgfsetdash{}{0pt}%
\pgfpathmoveto{\pgfqpoint{1.791803in}{1.955480in}}%
\pgfpathlineto{\pgfqpoint{1.774334in}{1.179253in}}%
\pgfpathlineto{\pgfqpoint{0.737061in}{0.644773in}}%
\pgfusepath{stroke}%
\end{pgfscope}%
\begin{pgfscope}%
\pgfsetbuttcap%
\pgfsetroundjoin%
\pgfsetlinewidth{0.803000pt}%
\definecolor{currentstroke}{rgb}{0.690196,0.690196,0.690196}%
\pgfsetstrokecolor{currentstroke}%
\pgfsetdash{}{0pt}%
\pgfpathmoveto{\pgfqpoint{1.604088in}{2.042948in}}%
\pgfpathlineto{\pgfqpoint{1.599501in}{1.267287in}}%
\pgfpathlineto{\pgfqpoint{0.562543in}{0.741945in}}%
\pgfusepath{stroke}%
\end{pgfscope}%
\begin{pgfscope}%
\pgfsetrectcap%
\pgfsetroundjoin%
\pgfsetlinewidth{0.803000pt}%
\definecolor{currentstroke}{rgb}{0.000000,0.000000,0.000000}%
\pgfsetstrokecolor{currentstroke}%
\pgfsetdash{}{0pt}%
\pgfpathmoveto{\pgfqpoint{1.475981in}{0.243033in}}%
\pgfpathlineto{\pgfqpoint{1.449770in}{0.228537in}}%
\pgfusepath{stroke}%
\end{pgfscope}%
\begin{pgfscope}%
\definecolor{textcolor}{rgb}{0.000000,0.000000,0.000000}%
\pgfsetstrokecolor{textcolor}%
\pgfsetfillcolor{textcolor}%
\pgftext[x=1.392595in,y=0.147403in,,top]{\color{textcolor}\sffamily\fontsize{6.000000}{7.200000}\selectfont \(\displaystyle 0.0\)}%
\end{pgfscope}%
\begin{pgfscope}%
\pgfsetrectcap%
\pgfsetroundjoin%
\pgfsetlinewidth{0.803000pt}%
\definecolor{currentstroke}{rgb}{0.000000,0.000000,0.000000}%
\pgfsetstrokecolor{currentstroke}%
\pgfsetdash{}{0pt}%
\pgfpathmoveto{\pgfqpoint{1.288439in}{0.347367in}}%
\pgfpathlineto{\pgfqpoint{1.262233in}{0.333134in}}%
\pgfusepath{stroke}%
\end{pgfscope}%
\begin{pgfscope}%
\definecolor{textcolor}{rgb}{0.000000,0.000000,0.000000}%
\pgfsetstrokecolor{textcolor}%
\pgfsetfillcolor{textcolor}%
\pgftext[x=1.206013in,y=0.252496in,,top]{\color{textcolor}\sffamily\fontsize{6.000000}{7.200000}\selectfont \(\displaystyle 0.2\)}%
\end{pgfscope}%
\begin{pgfscope}%
\pgfsetrectcap%
\pgfsetroundjoin%
\pgfsetlinewidth{0.803000pt}%
\definecolor{currentstroke}{rgb}{0.000000,0.000000,0.000000}%
\pgfsetstrokecolor{currentstroke}%
\pgfsetdash{}{0pt}%
\pgfpathmoveto{\pgfqpoint{1.104285in}{0.449817in}}%
\pgfpathlineto{\pgfqpoint{1.078087in}{0.435840in}}%
\pgfusepath{stroke}%
\end{pgfscope}%
\begin{pgfscope}%
\definecolor{textcolor}{rgb}{0.000000,0.000000,0.000000}%
\pgfsetstrokecolor{textcolor}%
\pgfsetfillcolor{textcolor}%
\pgftext[x=1.022798in,y=0.355693in,,top]{\color{textcolor}\sffamily\fontsize{6.000000}{7.200000}\selectfont \(\displaystyle 0.4\)}%
\end{pgfscope}%
\begin{pgfscope}%
\pgfsetrectcap%
\pgfsetroundjoin%
\pgfsetlinewidth{0.803000pt}%
\definecolor{currentstroke}{rgb}{0.000000,0.000000,0.000000}%
\pgfsetstrokecolor{currentstroke}%
\pgfsetdash{}{0pt}%
\pgfpathmoveto{\pgfqpoint{0.923427in}{0.550433in}}%
\pgfpathlineto{\pgfqpoint{0.897241in}{0.536706in}}%
\pgfusepath{stroke}%
\end{pgfscope}%
\begin{pgfscope}%
\definecolor{textcolor}{rgb}{0.000000,0.000000,0.000000}%
\pgfsetstrokecolor{textcolor}%
\pgfsetfillcolor{textcolor}%
\pgftext[x=0.842859in,y=0.457045in,,top]{\color{textcolor}\sffamily\fontsize{6.000000}{7.200000}\selectfont \(\displaystyle 0.6\)}%
\end{pgfscope}%
\begin{pgfscope}%
\pgfsetrectcap%
\pgfsetroundjoin%
\pgfsetlinewidth{0.803000pt}%
\definecolor{currentstroke}{rgb}{0.000000,0.000000,0.000000}%
\pgfsetstrokecolor{currentstroke}%
\pgfsetdash{}{0pt}%
\pgfpathmoveto{\pgfqpoint{0.745778in}{0.649264in}}%
\pgfpathlineto{\pgfqpoint{0.719607in}{0.635779in}}%
\pgfusepath{stroke}%
\end{pgfscope}%
\begin{pgfscope}%
\definecolor{textcolor}{rgb}{0.000000,0.000000,0.000000}%
\pgfsetstrokecolor{textcolor}%
\pgfsetfillcolor{textcolor}%
\pgftext[x=0.666109in,y=0.556601in,,top]{\color{textcolor}\sffamily\fontsize{6.000000}{7.200000}\selectfont \(\displaystyle 0.8\)}%
\end{pgfscope}%
\begin{pgfscope}%
\pgfsetrectcap%
\pgfsetroundjoin%
\pgfsetlinewidth{0.803000pt}%
\definecolor{currentstroke}{rgb}{0.000000,0.000000,0.000000}%
\pgfsetstrokecolor{currentstroke}%
\pgfsetdash{}{0pt}%
\pgfpathmoveto{\pgfqpoint{0.571253in}{0.746357in}}%
\pgfpathlineto{\pgfqpoint{0.545101in}{0.733108in}}%
\pgfusepath{stroke}%
\end{pgfscope}%
\begin{pgfscope}%
\definecolor{textcolor}{rgb}{0.000000,0.000000,0.000000}%
\pgfsetstrokecolor{textcolor}%
\pgfsetfillcolor{textcolor}%
\pgftext[x=0.492463in,y=0.654408in,,top]{\color{textcolor}\sffamily\fontsize{6.000000}{7.200000}\selectfont \(\displaystyle 1.0\)}%
\end{pgfscope}%
\begin{pgfscope}%
\pgfsetrectcap%
\pgfsetroundjoin%
\pgfsetlinewidth{0.803000pt}%
\definecolor{currentstroke}{rgb}{0.000000,0.000000,0.000000}%
\pgfsetstrokecolor{currentstroke}%
\pgfsetdash{}{0pt}%
\pgfpathmoveto{\pgfqpoint{0.499677in}{0.776948in}}%
\pgfpathlineto{\pgfqpoint{0.420389in}{1.554387in}}%
\pgfusepath{stroke}%
\end{pgfscope}%
\begin{pgfscope}%
\definecolor{textcolor}{rgb}{0.000000,0.000000,0.000000}%
\pgfsetstrokecolor{textcolor}%
\pgfsetfillcolor{textcolor}%
\pgftext[x=0.041630in,y=1.401767in,left,base,rotate=275.823265]{\color{textcolor}\sffamily\fontsize{8.000000}{9.600000}\selectfont \(\displaystyle \mathsf{val}(G_{\lambda, \mu}\))}%
\end{pgfscope}%
\begin{pgfscope}%
\pgfsetbuttcap%
\pgfsetroundjoin%
\pgfsetlinewidth{0.803000pt}%
\definecolor{currentstroke}{rgb}{0.690196,0.690196,0.690196}%
\pgfsetstrokecolor{currentstroke}%
\pgfsetdash{}{0pt}%
\pgfpathmoveto{\pgfqpoint{0.498202in}{0.791413in}}%
\pgfpathlineto{\pgfqpoint{1.536486in}{1.313496in}}%
\pgfpathlineto{\pgfqpoint{2.574771in}{0.791413in}}%
\pgfusepath{stroke}%
\end{pgfscope}%
\begin{pgfscope}%
\pgfsetbuttcap%
\pgfsetroundjoin%
\pgfsetlinewidth{0.803000pt}%
\definecolor{currentstroke}{rgb}{0.690196,0.690196,0.690196}%
\pgfsetstrokecolor{currentstroke}%
\pgfsetdash{}{0pt}%
\pgfpathmoveto{\pgfqpoint{0.487177in}{0.899517in}}%
\pgfpathlineto{\pgfqpoint{1.536486in}{1.421646in}}%
\pgfpathlineto{\pgfqpoint{2.585796in}{0.899517in}}%
\pgfusepath{stroke}%
\end{pgfscope}%
\begin{pgfscope}%
\pgfsetbuttcap%
\pgfsetroundjoin%
\pgfsetlinewidth{0.803000pt}%
\definecolor{currentstroke}{rgb}{0.690196,0.690196,0.690196}%
\pgfsetstrokecolor{currentstroke}%
\pgfsetdash{}{0pt}%
\pgfpathmoveto{\pgfqpoint{0.475915in}{1.009942in}}%
\pgfpathlineto{\pgfqpoint{1.536486in}{1.532004in}}%
\pgfpathlineto{\pgfqpoint{2.597058in}{1.009942in}}%
\pgfusepath{stroke}%
\end{pgfscope}%
\begin{pgfscope}%
\pgfsetbuttcap%
\pgfsetroundjoin%
\pgfsetlinewidth{0.803000pt}%
\definecolor{currentstroke}{rgb}{0.690196,0.690196,0.690196}%
\pgfsetstrokecolor{currentstroke}%
\pgfsetdash{}{0pt}%
\pgfpathmoveto{\pgfqpoint{0.464409in}{1.122763in}}%
\pgfpathlineto{\pgfqpoint{1.536486in}{1.644638in}}%
\pgfpathlineto{\pgfqpoint{2.608564in}{1.122763in}}%
\pgfusepath{stroke}%
\end{pgfscope}%
\begin{pgfscope}%
\pgfsetbuttcap%
\pgfsetroundjoin%
\pgfsetlinewidth{0.803000pt}%
\definecolor{currentstroke}{rgb}{0.690196,0.690196,0.690196}%
\pgfsetstrokecolor{currentstroke}%
\pgfsetdash{}{0pt}%
\pgfpathmoveto{\pgfqpoint{0.452650in}{1.238058in}}%
\pgfpathlineto{\pgfqpoint{1.536486in}{1.759619in}}%
\pgfpathlineto{\pgfqpoint{2.620323in}{1.238058in}}%
\pgfusepath{stroke}%
\end{pgfscope}%
\begin{pgfscope}%
\pgfsetbuttcap%
\pgfsetroundjoin%
\pgfsetlinewidth{0.803000pt}%
\definecolor{currentstroke}{rgb}{0.690196,0.690196,0.690196}%
\pgfsetstrokecolor{currentstroke}%
\pgfsetdash{}{0pt}%
\pgfpathmoveto{\pgfqpoint{0.440631in}{1.355911in}}%
\pgfpathlineto{\pgfqpoint{1.536486in}{1.877023in}}%
\pgfpathlineto{\pgfqpoint{2.632342in}{1.355911in}}%
\pgfusepath{stroke}%
\end{pgfscope}%
\begin{pgfscope}%
\pgfsetbuttcap%
\pgfsetroundjoin%
\pgfsetlinewidth{0.803000pt}%
\definecolor{currentstroke}{rgb}{0.690196,0.690196,0.690196}%
\pgfsetstrokecolor{currentstroke}%
\pgfsetdash{}{0pt}%
\pgfpathmoveto{\pgfqpoint{0.428342in}{1.476407in}}%
\pgfpathlineto{\pgfqpoint{1.536486in}{1.996925in}}%
\pgfpathlineto{\pgfqpoint{2.644631in}{1.476407in}}%
\pgfusepath{stroke}%
\end{pgfscope}%
\begin{pgfscope}%
\pgfsetrectcap%
\pgfsetroundjoin%
\pgfsetlinewidth{0.803000pt}%
\definecolor{currentstroke}{rgb}{0.000000,0.000000,0.000000}%
\pgfsetstrokecolor{currentstroke}%
\pgfsetdash{}{0pt}%
\pgfpathmoveto{\pgfqpoint{0.506923in}{0.795798in}}%
\pgfpathlineto{\pgfqpoint{0.480740in}{0.782632in}}%
\pgfusepath{stroke}%
\end{pgfscope}%
\begin{pgfscope}%
\definecolor{textcolor}{rgb}{0.000000,0.000000,0.000000}%
\pgfsetstrokecolor{textcolor}%
\pgfsetfillcolor{textcolor}%
\pgftext[x=0.355298in,y=0.791413in,,top]{\color{textcolor}\sffamily\fontsize{6.000000}{7.200000}\selectfont \(\displaystyle -0.2\)}%
\end{pgfscope}%
\begin{pgfscope}%
\pgfsetrectcap%
\pgfsetroundjoin%
\pgfsetlinewidth{0.803000pt}%
\definecolor{currentstroke}{rgb}{0.000000,0.000000,0.000000}%
\pgfsetstrokecolor{currentstroke}%
\pgfsetdash{}{0pt}%
\pgfpathmoveto{\pgfqpoint{0.495994in}{0.903905in}}%
\pgfpathlineto{\pgfqpoint{0.469520in}{0.890731in}}%
\pgfusepath{stroke}%
\end{pgfscope}%
\begin{pgfscope}%
\definecolor{textcolor}{rgb}{0.000000,0.000000,0.000000}%
\pgfsetstrokecolor{textcolor}%
\pgfsetfillcolor{textcolor}%
\pgftext[x=0.342756in,y=0.899517in,,top]{\color{textcolor}\sffamily\fontsize{6.000000}{7.200000}\selectfont \(\displaystyle -0.1\)}%
\end{pgfscope}%
\begin{pgfscope}%
\pgfsetrectcap%
\pgfsetroundjoin%
\pgfsetlinewidth{0.803000pt}%
\definecolor{currentstroke}{rgb}{0.000000,0.000000,0.000000}%
\pgfsetstrokecolor{currentstroke}%
\pgfsetdash{}{0pt}%
\pgfpathmoveto{\pgfqpoint{0.484832in}{1.014331in}}%
\pgfpathlineto{\pgfqpoint{0.458059in}{1.001152in}}%
\pgfusepath{stroke}%
\end{pgfscope}%
\begin{pgfscope}%
\definecolor{textcolor}{rgb}{0.000000,0.000000,0.000000}%
\pgfsetstrokecolor{textcolor}%
\pgfsetfillcolor{textcolor}%
\pgftext[x=0.329944in,y=1.009942in,,top]{\color{textcolor}\sffamily\fontsize{6.000000}{7.200000}\selectfont \(\displaystyle 0.0\)}%
\end{pgfscope}%
\begin{pgfscope}%
\pgfsetrectcap%
\pgfsetroundjoin%
\pgfsetlinewidth{0.803000pt}%
\definecolor{currentstroke}{rgb}{0.000000,0.000000,0.000000}%
\pgfsetstrokecolor{currentstroke}%
\pgfsetdash{}{0pt}%
\pgfpathmoveto{\pgfqpoint{0.473427in}{1.127153in}}%
\pgfpathlineto{\pgfqpoint{0.446350in}{1.113972in}}%
\pgfusepath{stroke}%
\end{pgfscope}%
\begin{pgfscope}%
\definecolor{textcolor}{rgb}{0.000000,0.000000,0.000000}%
\pgfsetstrokecolor{textcolor}%
\pgfsetfillcolor{textcolor}%
\pgftext[x=0.316854in,y=1.122763in,,top]{\color{textcolor}\sffamily\fontsize{6.000000}{7.200000}\selectfont \(\displaystyle 0.1\)}%
\end{pgfscope}%
\begin{pgfscope}%
\pgfsetrectcap%
\pgfsetroundjoin%
\pgfsetlinewidth{0.803000pt}%
\definecolor{currentstroke}{rgb}{0.000000,0.000000,0.000000}%
\pgfsetstrokecolor{currentstroke}%
\pgfsetdash{}{0pt}%
\pgfpathmoveto{\pgfqpoint{0.461772in}{1.242448in}}%
\pgfpathlineto{\pgfqpoint{0.434383in}{1.229268in}}%
\pgfusepath{stroke}%
\end{pgfscope}%
\begin{pgfscope}%
\definecolor{textcolor}{rgb}{0.000000,0.000000,0.000000}%
\pgfsetstrokecolor{textcolor}%
\pgfsetfillcolor{textcolor}%
\pgftext[x=0.303477in,y=1.238058in,,top]{\color{textcolor}\sffamily\fontsize{6.000000}{7.200000}\selectfont \(\displaystyle 0.2\)}%
\end{pgfscope}%
\begin{pgfscope}%
\pgfsetrectcap%
\pgfsetroundjoin%
\pgfsetlinewidth{0.803000pt}%
\definecolor{currentstroke}{rgb}{0.000000,0.000000,0.000000}%
\pgfsetstrokecolor{currentstroke}%
\pgfsetdash{}{0pt}%
\pgfpathmoveto{\pgfqpoint{0.449859in}{1.360299in}}%
\pgfpathlineto{\pgfqpoint{0.422150in}{1.347123in}}%
\pgfusepath{stroke}%
\end{pgfscope}%
\begin{pgfscope}%
\definecolor{textcolor}{rgb}{0.000000,0.000000,0.000000}%
\pgfsetstrokecolor{textcolor}%
\pgfsetfillcolor{textcolor}%
\pgftext[x=0.289803in,y=1.355911in,,top]{\color{textcolor}\sffamily\fontsize{6.000000}{7.200000}\selectfont \(\displaystyle 0.3\)}%
\end{pgfscope}%
\begin{pgfscope}%
\pgfsetrectcap%
\pgfsetroundjoin%
\pgfsetlinewidth{0.803000pt}%
\definecolor{currentstroke}{rgb}{0.000000,0.000000,0.000000}%
\pgfsetstrokecolor{currentstroke}%
\pgfsetdash{}{0pt}%
\pgfpathmoveto{\pgfqpoint{0.437679in}{1.480793in}}%
\pgfpathlineto{\pgfqpoint{0.409643in}{1.467624in}}%
\pgfusepath{stroke}%
\end{pgfscope}%
\begin{pgfscope}%
\definecolor{textcolor}{rgb}{0.000000,0.000000,0.000000}%
\pgfsetstrokecolor{textcolor}%
\pgfsetfillcolor{textcolor}%
\pgftext[x=0.275823in,y=1.476407in,,top]{\color{textcolor}\sffamily\fontsize{6.000000}{7.200000}\selectfont \(\displaystyle 0.4\)}%
\end{pgfscope}%
\begin{pgfscope}%
\pgfpathrectangle{\pgfqpoint{0.150000in}{0.150000in}}{\pgfqpoint{2.700000in}{1.950000in}}%
\pgfusepath{clip}%
\pgfsetbuttcap%
\pgfsetroundjoin%
\definecolor{currentfill}{rgb}{0.846140,0.720818,0.730744}%
\pgfsetfillcolor{currentfill}%
\pgfsetlinewidth{0.000000pt}%
\definecolor{currentstroke}{rgb}{0.000000,0.000000,0.000000}%
\pgfsetstrokecolor{currentstroke}%
\pgfsetdash{}{0pt}%
\pgfpathmoveto{\pgfqpoint{2.407910in}{0.772237in}}%
\pgfpathlineto{\pgfqpoint{2.442626in}{0.791413in}}%
\pgfpathlineto{\pgfqpoint{2.409433in}{0.864294in}}%
\pgfpathlineto{\pgfqpoint{2.374537in}{0.845179in}}%
\pgfpathclose%
\pgfusepath{fill}%
\end{pgfscope}%
\begin{pgfscope}%
\pgfpathrectangle{\pgfqpoint{0.150000in}{0.150000in}}{\pgfqpoint{2.700000in}{1.950000in}}%
\pgfusepath{clip}%
\pgfsetbuttcap%
\pgfsetroundjoin%
\definecolor{currentfill}{rgb}{0.846140,0.720818,0.730744}%
\pgfsetfillcolor{currentfill}%
\pgfsetlinewidth{0.000000pt}%
\definecolor{currentstroke}{rgb}{0.000000,0.000000,0.000000}%
\pgfsetstrokecolor{currentstroke}%
\pgfsetdash{}{0pt}%
\pgfpathmoveto{\pgfqpoint{2.373073in}{0.752993in}}%
\pgfpathlineto{\pgfqpoint{2.407910in}{0.772237in}}%
\pgfpathlineto{\pgfqpoint{2.374537in}{0.845179in}}%
\pgfpathlineto{\pgfqpoint{2.339518in}{0.825997in}}%
\pgfpathclose%
\pgfusepath{fill}%
\end{pgfscope}%
\begin{pgfscope}%
\pgfpathrectangle{\pgfqpoint{0.150000in}{0.150000in}}{\pgfqpoint{2.700000in}{1.950000in}}%
\pgfusepath{clip}%
\pgfsetbuttcap%
\pgfsetroundjoin%
\definecolor{currentfill}{rgb}{0.887929,0.796645,0.803876}%
\pgfsetfillcolor{currentfill}%
\pgfsetlinewidth{0.000000pt}%
\definecolor{currentstroke}{rgb}{0.000000,0.000000,0.000000}%
\pgfsetstrokecolor{currentstroke}%
\pgfsetdash{}{0pt}%
\pgfpathmoveto{\pgfqpoint{2.374537in}{0.845179in}}%
\pgfpathlineto{\pgfqpoint{2.409433in}{0.864294in}}%
\pgfpathlineto{\pgfqpoint{2.376007in}{0.937689in}}%
\pgfpathlineto{\pgfqpoint{2.340929in}{0.918637in}}%
\pgfpathclose%
\pgfusepath{fill}%
\end{pgfscope}%
\begin{pgfscope}%
\pgfpathrectangle{\pgfqpoint{0.150000in}{0.150000in}}{\pgfqpoint{2.700000in}{1.950000in}}%
\pgfusepath{clip}%
\pgfsetbuttcap%
\pgfsetroundjoin%
\definecolor{currentfill}{rgb}{0.846140,0.720818,0.730744}%
\pgfsetfillcolor{currentfill}%
\pgfsetlinewidth{0.000000pt}%
\definecolor{currentstroke}{rgb}{0.000000,0.000000,0.000000}%
\pgfsetstrokecolor{currentstroke}%
\pgfsetdash{}{0pt}%
\pgfpathmoveto{\pgfqpoint{2.338112in}{0.733682in}}%
\pgfpathlineto{\pgfqpoint{2.373073in}{0.752993in}}%
\pgfpathlineto{\pgfqpoint{2.339518in}{0.825997in}}%
\pgfpathlineto{\pgfqpoint{2.304375in}{0.806747in}}%
\pgfpathclose%
\pgfusepath{fill}%
\end{pgfscope}%
\begin{pgfscope}%
\pgfpathrectangle{\pgfqpoint{0.150000in}{0.150000in}}{\pgfqpoint{2.700000in}{1.950000in}}%
\pgfusepath{clip}%
\pgfsetbuttcap%
\pgfsetroundjoin%
\definecolor{currentfill}{rgb}{0.887929,0.796645,0.803876}%
\pgfsetfillcolor{currentfill}%
\pgfsetlinewidth{0.000000pt}%
\definecolor{currentstroke}{rgb}{0.000000,0.000000,0.000000}%
\pgfsetstrokecolor{currentstroke}%
\pgfsetdash{}{0pt}%
\pgfpathmoveto{\pgfqpoint{2.339518in}{0.825997in}}%
\pgfpathlineto{\pgfqpoint{2.374537in}{0.845179in}}%
\pgfpathlineto{\pgfqpoint{2.340929in}{0.918637in}}%
\pgfpathlineto{\pgfqpoint{2.305726in}{0.899517in}}%
\pgfpathclose%
\pgfusepath{fill}%
\end{pgfscope}%
\begin{pgfscope}%
\pgfpathrectangle{\pgfqpoint{0.150000in}{0.150000in}}{\pgfqpoint{2.700000in}{1.950000in}}%
\pgfusepath{clip}%
\pgfsetbuttcap%
\pgfsetroundjoin%
\definecolor{currentfill}{rgb}{0.846140,0.720818,0.730744}%
\pgfsetfillcolor{currentfill}%
\pgfsetlinewidth{0.000000pt}%
\definecolor{currentstroke}{rgb}{0.000000,0.000000,0.000000}%
\pgfsetstrokecolor{currentstroke}%
\pgfsetdash{}{0pt}%
\pgfpathmoveto{\pgfqpoint{2.303028in}{0.714302in}}%
\pgfpathlineto{\pgfqpoint{2.338112in}{0.733682in}}%
\pgfpathlineto{\pgfqpoint{2.304375in}{0.806747in}}%
\pgfpathlineto{\pgfqpoint{2.269106in}{0.787428in}}%
\pgfpathclose%
\pgfusepath{fill}%
\end{pgfscope}%
\begin{pgfscope}%
\pgfpathrectangle{\pgfqpoint{0.150000in}{0.150000in}}{\pgfqpoint{2.700000in}{1.950000in}}%
\pgfusepath{clip}%
\pgfsetbuttcap%
\pgfsetroundjoin%
\definecolor{currentfill}{rgb}{0.933517,0.879366,0.883655}%
\pgfsetfillcolor{currentfill}%
\pgfsetlinewidth{0.000000pt}%
\definecolor{currentstroke}{rgb}{0.000000,0.000000,0.000000}%
\pgfsetstrokecolor{currentstroke}%
\pgfsetdash{}{0pt}%
\pgfpathmoveto{\pgfqpoint{2.340929in}{0.918637in}}%
\pgfpathlineto{\pgfqpoint{2.376007in}{0.937689in}}%
\pgfpathlineto{\pgfqpoint{2.342344in}{1.011603in}}%
\pgfpathlineto{\pgfqpoint{2.307082in}{0.992614in}}%
\pgfpathclose%
\pgfusepath{fill}%
\end{pgfscope}%
\begin{pgfscope}%
\pgfpathrectangle{\pgfqpoint{0.150000in}{0.150000in}}{\pgfqpoint{2.700000in}{1.950000in}}%
\pgfusepath{clip}%
\pgfsetbuttcap%
\pgfsetroundjoin%
\definecolor{currentfill}{rgb}{0.887929,0.796645,0.803876}%
\pgfsetfillcolor{currentfill}%
\pgfsetlinewidth{0.000000pt}%
\definecolor{currentstroke}{rgb}{0.000000,0.000000,0.000000}%
\pgfsetstrokecolor{currentstroke}%
\pgfsetdash{}{0pt}%
\pgfpathmoveto{\pgfqpoint{2.304375in}{0.806747in}}%
\pgfpathlineto{\pgfqpoint{2.339518in}{0.825997in}}%
\pgfpathlineto{\pgfqpoint{2.305726in}{0.899517in}}%
\pgfpathlineto{\pgfqpoint{2.270398in}{0.880329in}}%
\pgfpathclose%
\pgfusepath{fill}%
\end{pgfscope}%
\begin{pgfscope}%
\pgfpathrectangle{\pgfqpoint{0.150000in}{0.150000in}}{\pgfqpoint{2.700000in}{1.950000in}}%
\pgfusepath{clip}%
\pgfsetbuttcap%
\pgfsetroundjoin%
\definecolor{currentfill}{rgb}{0.846140,0.720818,0.730744}%
\pgfsetfillcolor{currentfill}%
\pgfsetlinewidth{0.000000pt}%
\definecolor{currentstroke}{rgb}{0.000000,0.000000,0.000000}%
\pgfsetstrokecolor{currentstroke}%
\pgfsetdash{}{0pt}%
\pgfpathmoveto{\pgfqpoint{2.267819in}{0.694854in}}%
\pgfpathlineto{\pgfqpoint{2.303028in}{0.714302in}}%
\pgfpathlineto{\pgfqpoint{2.269106in}{0.787428in}}%
\pgfpathlineto{\pgfqpoint{2.233713in}{0.768041in}}%
\pgfpathclose%
\pgfusepath{fill}%
\end{pgfscope}%
\begin{pgfscope}%
\pgfpathrectangle{\pgfqpoint{0.150000in}{0.150000in}}{\pgfqpoint{2.700000in}{1.950000in}}%
\pgfusepath{clip}%
\pgfsetbuttcap%
\pgfsetroundjoin%
\definecolor{currentfill}{rgb}{0.933517,0.879366,0.883655}%
\pgfsetfillcolor{currentfill}%
\pgfsetlinewidth{0.000000pt}%
\definecolor{currentstroke}{rgb}{0.000000,0.000000,0.000000}%
\pgfsetstrokecolor{currentstroke}%
\pgfsetdash{}{0pt}%
\pgfpathmoveto{\pgfqpoint{2.305726in}{0.899517in}}%
\pgfpathlineto{\pgfqpoint{2.340929in}{0.918637in}}%
\pgfpathlineto{\pgfqpoint{2.307082in}{0.992614in}}%
\pgfpathlineto{\pgfqpoint{2.271694in}{0.973559in}}%
\pgfpathclose%
\pgfusepath{fill}%
\end{pgfscope}%
\begin{pgfscope}%
\pgfpathrectangle{\pgfqpoint{0.150000in}{0.150000in}}{\pgfqpoint{2.700000in}{1.950000in}}%
\pgfusepath{clip}%
\pgfsetbuttcap%
\pgfsetroundjoin%
\definecolor{currentfill}{rgb}{0.887929,0.796645,0.803876}%
\pgfsetfillcolor{currentfill}%
\pgfsetlinewidth{0.000000pt}%
\definecolor{currentstroke}{rgb}{0.000000,0.000000,0.000000}%
\pgfsetstrokecolor{currentstroke}%
\pgfsetdash{}{0pt}%
\pgfpathmoveto{\pgfqpoint{2.269106in}{0.787428in}}%
\pgfpathlineto{\pgfqpoint{2.304375in}{0.806747in}}%
\pgfpathlineto{\pgfqpoint{2.270398in}{0.880329in}}%
\pgfpathlineto{\pgfqpoint{2.234944in}{0.861073in}}%
\pgfpathclose%
\pgfusepath{fill}%
\end{pgfscope}%
\begin{pgfscope}%
\pgfpathrectangle{\pgfqpoint{0.150000in}{0.150000in}}{\pgfqpoint{2.700000in}{1.950000in}}%
\pgfusepath{clip}%
\pgfsetbuttcap%
\pgfsetroundjoin%
\definecolor{currentfill}{rgb}{0.846140,0.720818,0.730744}%
\pgfsetfillcolor{currentfill}%
\pgfsetlinewidth{0.000000pt}%
\definecolor{currentstroke}{rgb}{0.000000,0.000000,0.000000}%
\pgfsetstrokecolor{currentstroke}%
\pgfsetdash{}{0pt}%
\pgfpathmoveto{\pgfqpoint{2.232486in}{0.675336in}}%
\pgfpathlineto{\pgfqpoint{2.267819in}{0.694854in}}%
\pgfpathlineto{\pgfqpoint{2.233713in}{0.768041in}}%
\pgfpathlineto{\pgfqpoint{2.198193in}{0.748585in}}%
\pgfpathclose%
\pgfusepath{fill}%
\end{pgfscope}%
\begin{pgfscope}%
\pgfpathrectangle{\pgfqpoint{0.150000in}{0.150000in}}{\pgfqpoint{2.700000in}{1.950000in}}%
\pgfusepath{clip}%
\pgfsetbuttcap%
\pgfsetroundjoin%
\definecolor{currentfill}{rgb}{0.979105,0.962086,0.963434}%
\pgfsetfillcolor{currentfill}%
\pgfsetlinewidth{0.000000pt}%
\definecolor{currentstroke}{rgb}{0.000000,0.000000,0.000000}%
\pgfsetstrokecolor{currentstroke}%
\pgfsetdash{}{0pt}%
\pgfpathmoveto{\pgfqpoint{2.307082in}{0.992614in}}%
\pgfpathlineto{\pgfqpoint{2.342344in}{1.011603in}}%
\pgfpathlineto{\pgfqpoint{2.308443in}{1.086040in}}%
\pgfpathlineto{\pgfqpoint{2.272995in}{1.067118in}}%
\pgfpathclose%
\pgfusepath{fill}%
\end{pgfscope}%
\begin{pgfscope}%
\pgfpathrectangle{\pgfqpoint{0.150000in}{0.150000in}}{\pgfqpoint{2.700000in}{1.950000in}}%
\pgfusepath{clip}%
\pgfsetbuttcap%
\pgfsetroundjoin%
\definecolor{currentfill}{rgb}{0.933517,0.879366,0.883655}%
\pgfsetfillcolor{currentfill}%
\pgfsetlinewidth{0.000000pt}%
\definecolor{currentstroke}{rgb}{0.000000,0.000000,0.000000}%
\pgfsetstrokecolor{currentstroke}%
\pgfsetdash{}{0pt}%
\pgfpathmoveto{\pgfqpoint{2.270398in}{0.880329in}}%
\pgfpathlineto{\pgfqpoint{2.305726in}{0.899517in}}%
\pgfpathlineto{\pgfqpoint{2.271694in}{0.973559in}}%
\pgfpathlineto{\pgfqpoint{2.236180in}{0.954435in}}%
\pgfpathclose%
\pgfusepath{fill}%
\end{pgfscope}%
\begin{pgfscope}%
\pgfpathrectangle{\pgfqpoint{0.150000in}{0.150000in}}{\pgfqpoint{2.700000in}{1.950000in}}%
\pgfusepath{clip}%
\pgfsetbuttcap%
\pgfsetroundjoin%
\definecolor{currentfill}{rgb}{0.887929,0.796645,0.803876}%
\pgfsetfillcolor{currentfill}%
\pgfsetlinewidth{0.000000pt}%
\definecolor{currentstroke}{rgb}{0.000000,0.000000,0.000000}%
\pgfsetstrokecolor{currentstroke}%
\pgfsetdash{}{0pt}%
\pgfpathmoveto{\pgfqpoint{2.233713in}{0.768041in}}%
\pgfpathlineto{\pgfqpoint{2.269106in}{0.787428in}}%
\pgfpathlineto{\pgfqpoint{2.234944in}{0.861073in}}%
\pgfpathlineto{\pgfqpoint{2.199364in}{0.841748in}}%
\pgfpathclose%
\pgfusepath{fill}%
\end{pgfscope}%
\begin{pgfscope}%
\pgfpathrectangle{\pgfqpoint{0.150000in}{0.150000in}}{\pgfqpoint{2.700000in}{1.950000in}}%
\pgfusepath{clip}%
\pgfsetbuttcap%
\pgfsetroundjoin%
\definecolor{currentfill}{rgb}{0.846140,0.720818,0.730744}%
\pgfsetfillcolor{currentfill}%
\pgfsetlinewidth{0.000000pt}%
\definecolor{currentstroke}{rgb}{0.000000,0.000000,0.000000}%
\pgfsetstrokecolor{currentstroke}%
\pgfsetdash{}{0pt}%
\pgfpathmoveto{\pgfqpoint{2.197027in}{0.655749in}}%
\pgfpathlineto{\pgfqpoint{2.232486in}{0.675336in}}%
\pgfpathlineto{\pgfqpoint{2.198193in}{0.748585in}}%
\pgfpathlineto{\pgfqpoint{2.162547in}{0.729059in}}%
\pgfpathclose%
\pgfusepath{fill}%
\end{pgfscope}%
\begin{pgfscope}%
\pgfpathrectangle{\pgfqpoint{0.150000in}{0.150000in}}{\pgfqpoint{2.700000in}{1.950000in}}%
\pgfusepath{clip}%
\pgfsetbuttcap%
\pgfsetroundjoin%
\definecolor{currentfill}{rgb}{0.524219,0.582812,0.664844}%
\pgfsetfillcolor{currentfill}%
\pgfsetlinewidth{0.000000pt}%
\definecolor{currentstroke}{rgb}{0.000000,0.000000,0.000000}%
\pgfsetstrokecolor{currentstroke}%
\pgfsetdash{}{0pt}%
\pgfpathmoveto{\pgfqpoint{1.536486in}{1.814265in}}%
\pgfpathlineto{\pgfqpoint{1.573230in}{1.831849in}}%
\pgfpathlineto{\pgfqpoint{1.536486in}{1.849370in}}%
\pgfpathlineto{\pgfqpoint{1.499743in}{1.831849in}}%
\pgfpathclose%
\pgfusepath{fill}%
\end{pgfscope}%
\begin{pgfscope}%
\pgfpathrectangle{\pgfqpoint{0.150000in}{0.150000in}}{\pgfqpoint{2.700000in}{1.950000in}}%
\pgfusepath{clip}%
\pgfsetbuttcap%
\pgfsetroundjoin%
\definecolor{currentfill}{rgb}{0.979105,0.962086,0.963434}%
\pgfsetfillcolor{currentfill}%
\pgfsetlinewidth{0.000000pt}%
\definecolor{currentstroke}{rgb}{0.000000,0.000000,0.000000}%
\pgfsetstrokecolor{currentstroke}%
\pgfsetdash{}{0pt}%
\pgfpathmoveto{\pgfqpoint{2.271694in}{0.973559in}}%
\pgfpathlineto{\pgfqpoint{2.307082in}{0.992614in}}%
\pgfpathlineto{\pgfqpoint{2.272995in}{1.067118in}}%
\pgfpathlineto{\pgfqpoint{2.237420in}{1.048127in}}%
\pgfpathclose%
\pgfusepath{fill}%
\end{pgfscope}%
\begin{pgfscope}%
\pgfpathrectangle{\pgfqpoint{0.150000in}{0.150000in}}{\pgfqpoint{2.700000in}{1.950000in}}%
\pgfusepath{clip}%
\pgfsetbuttcap%
\pgfsetroundjoin%
\definecolor{currentfill}{rgb}{0.933517,0.879366,0.883655}%
\pgfsetfillcolor{currentfill}%
\pgfsetlinewidth{0.000000pt}%
\definecolor{currentstroke}{rgb}{0.000000,0.000000,0.000000}%
\pgfsetstrokecolor{currentstroke}%
\pgfsetdash{}{0pt}%
\pgfpathmoveto{\pgfqpoint{2.234944in}{0.861073in}}%
\pgfpathlineto{\pgfqpoint{2.270398in}{0.880329in}}%
\pgfpathlineto{\pgfqpoint{2.236180in}{0.954435in}}%
\pgfpathlineto{\pgfqpoint{2.200539in}{0.935243in}}%
\pgfpathclose%
\pgfusepath{fill}%
\end{pgfscope}%
\begin{pgfscope}%
\pgfpathrectangle{\pgfqpoint{0.150000in}{0.150000in}}{\pgfqpoint{2.700000in}{1.950000in}}%
\pgfusepath{clip}%
\pgfsetbuttcap%
\pgfsetroundjoin%
\definecolor{currentfill}{rgb}{0.887929,0.796645,0.803876}%
\pgfsetfillcolor{currentfill}%
\pgfsetlinewidth{0.000000pt}%
\definecolor{currentstroke}{rgb}{0.000000,0.000000,0.000000}%
\pgfsetstrokecolor{currentstroke}%
\pgfsetdash{}{0pt}%
\pgfpathmoveto{\pgfqpoint{2.198193in}{0.748585in}}%
\pgfpathlineto{\pgfqpoint{2.233713in}{0.768041in}}%
\pgfpathlineto{\pgfqpoint{2.199364in}{0.841748in}}%
\pgfpathlineto{\pgfqpoint{2.163656in}{0.822355in}}%
\pgfpathclose%
\pgfusepath{fill}%
\end{pgfscope}%
\begin{pgfscope}%
\pgfpathrectangle{\pgfqpoint{0.150000in}{0.150000in}}{\pgfqpoint{2.700000in}{1.950000in}}%
\pgfusepath{clip}%
\pgfsetbuttcap%
\pgfsetroundjoin%
\definecolor{currentfill}{rgb}{0.846140,0.720818,0.730744}%
\pgfsetfillcolor{currentfill}%
\pgfsetlinewidth{0.000000pt}%
\definecolor{currentstroke}{rgb}{0.000000,0.000000,0.000000}%
\pgfsetstrokecolor{currentstroke}%
\pgfsetdash{}{0pt}%
\pgfpathmoveto{\pgfqpoint{2.161441in}{0.636092in}}%
\pgfpathlineto{\pgfqpoint{2.197027in}{0.655749in}}%
\pgfpathlineto{\pgfqpoint{2.162547in}{0.729059in}}%
\pgfpathlineto{\pgfqpoint{2.126773in}{0.709463in}}%
\pgfpathclose%
\pgfusepath{fill}%
\end{pgfscope}%
\begin{pgfscope}%
\pgfpathrectangle{\pgfqpoint{0.150000in}{0.150000in}}{\pgfqpoint{2.700000in}{1.950000in}}%
\pgfusepath{clip}%
\pgfsetbuttcap%
\pgfsetroundjoin%
\definecolor{currentfill}{rgb}{0.524219,0.582812,0.664844}%
\pgfsetfillcolor{currentfill}%
\pgfsetlinewidth{0.000000pt}%
\definecolor{currentstroke}{rgb}{0.000000,0.000000,0.000000}%
\pgfsetstrokecolor{currentstroke}%
\pgfsetdash{}{0pt}%
\pgfpathmoveto{\pgfqpoint{1.573361in}{1.796618in}}%
\pgfpathlineto{\pgfqpoint{1.610104in}{1.814265in}}%
\pgfpathlineto{\pgfqpoint{1.573230in}{1.831849in}}%
\pgfpathlineto{\pgfqpoint{1.536486in}{1.814265in}}%
\pgfpathclose%
\pgfusepath{fill}%
\end{pgfscope}%
\begin{pgfscope}%
\pgfpathrectangle{\pgfqpoint{0.150000in}{0.150000in}}{\pgfqpoint{2.700000in}{1.950000in}}%
\pgfusepath{clip}%
\pgfsetbuttcap%
\pgfsetroundjoin%
\definecolor{currentfill}{rgb}{0.524219,0.582812,0.664844}%
\pgfsetfillcolor{currentfill}%
\pgfsetlinewidth{0.000000pt}%
\definecolor{currentstroke}{rgb}{0.000000,0.000000,0.000000}%
\pgfsetstrokecolor{currentstroke}%
\pgfsetdash{}{0pt}%
\pgfpathmoveto{\pgfqpoint{1.499612in}{1.796618in}}%
\pgfpathlineto{\pgfqpoint{1.536486in}{1.814265in}}%
\pgfpathlineto{\pgfqpoint{1.499743in}{1.831849in}}%
\pgfpathlineto{\pgfqpoint{1.462869in}{1.814265in}}%
\pgfpathclose%
\pgfusepath{fill}%
\end{pgfscope}%
\begin{pgfscope}%
\pgfpathrectangle{\pgfqpoint{0.150000in}{0.150000in}}{\pgfqpoint{2.700000in}{1.950000in}}%
\pgfusepath{clip}%
\pgfsetbuttcap%
\pgfsetroundjoin%
\definecolor{currentfill}{rgb}{0.979105,0.962086,0.963434}%
\pgfsetfillcolor{currentfill}%
\pgfsetlinewidth{0.000000pt}%
\definecolor{currentstroke}{rgb}{0.000000,0.000000,0.000000}%
\pgfsetstrokecolor{currentstroke}%
\pgfsetdash{}{0pt}%
\pgfpathmoveto{\pgfqpoint{2.236180in}{0.954435in}}%
\pgfpathlineto{\pgfqpoint{2.271694in}{0.973559in}}%
\pgfpathlineto{\pgfqpoint{2.237420in}{1.048127in}}%
\pgfpathlineto{\pgfqpoint{2.201718in}{1.029069in}}%
\pgfpathclose%
\pgfusepath{fill}%
\end{pgfscope}%
\begin{pgfscope}%
\pgfpathrectangle{\pgfqpoint{0.150000in}{0.150000in}}{\pgfqpoint{2.700000in}{1.950000in}}%
\pgfusepath{clip}%
\pgfsetbuttcap%
\pgfsetroundjoin%
\definecolor{currentfill}{rgb}{0.933517,0.879366,0.883655}%
\pgfsetfillcolor{currentfill}%
\pgfsetlinewidth{0.000000pt}%
\definecolor{currentstroke}{rgb}{0.000000,0.000000,0.000000}%
\pgfsetstrokecolor{currentstroke}%
\pgfsetdash{}{0pt}%
\pgfpathmoveto{\pgfqpoint{2.199364in}{0.841748in}}%
\pgfpathlineto{\pgfqpoint{2.234944in}{0.861073in}}%
\pgfpathlineto{\pgfqpoint{2.200539in}{0.935243in}}%
\pgfpathlineto{\pgfqpoint{2.164770in}{0.915982in}}%
\pgfpathclose%
\pgfusepath{fill}%
\end{pgfscope}%
\begin{pgfscope}%
\pgfpathrectangle{\pgfqpoint{0.150000in}{0.150000in}}{\pgfqpoint{2.700000in}{1.950000in}}%
\pgfusepath{clip}%
\pgfsetbuttcap%
\pgfsetroundjoin%
\definecolor{currentfill}{rgb}{0.887929,0.796645,0.803876}%
\pgfsetfillcolor{currentfill}%
\pgfsetlinewidth{0.000000pt}%
\definecolor{currentstroke}{rgb}{0.000000,0.000000,0.000000}%
\pgfsetstrokecolor{currentstroke}%
\pgfsetdash{}{0pt}%
\pgfpathmoveto{\pgfqpoint{2.162547in}{0.729059in}}%
\pgfpathlineto{\pgfqpoint{2.198193in}{0.748585in}}%
\pgfpathlineto{\pgfqpoint{2.163656in}{0.822355in}}%
\pgfpathlineto{\pgfqpoint{2.127821in}{0.802891in}}%
\pgfpathclose%
\pgfusepath{fill}%
\end{pgfscope}%
\begin{pgfscope}%
\pgfpathrectangle{\pgfqpoint{0.150000in}{0.150000in}}{\pgfqpoint{2.700000in}{1.950000in}}%
\pgfusepath{clip}%
\pgfsetbuttcap%
\pgfsetroundjoin%
\definecolor{currentfill}{rgb}{0.846140,0.720818,0.730744}%
\pgfsetfillcolor{currentfill}%
\pgfsetlinewidth{0.000000pt}%
\definecolor{currentstroke}{rgb}{0.000000,0.000000,0.000000}%
\pgfsetstrokecolor{currentstroke}%
\pgfsetdash{}{0pt}%
\pgfpathmoveto{\pgfqpoint{2.125728in}{0.616366in}}%
\pgfpathlineto{\pgfqpoint{2.161441in}{0.636092in}}%
\pgfpathlineto{\pgfqpoint{2.126773in}{0.709463in}}%
\pgfpathlineto{\pgfqpoint{2.090871in}{0.689797in}}%
\pgfpathclose%
\pgfusepath{fill}%
\end{pgfscope}%
\begin{pgfscope}%
\pgfpathrectangle{\pgfqpoint{0.150000in}{0.150000in}}{\pgfqpoint{2.700000in}{1.950000in}}%
\pgfusepath{clip}%
\pgfsetbuttcap%
\pgfsetroundjoin%
\definecolor{currentfill}{rgb}{0.524219,0.582812,0.664844}%
\pgfsetfillcolor{currentfill}%
\pgfsetlinewidth{0.000000pt}%
\definecolor{currentstroke}{rgb}{0.000000,0.000000,0.000000}%
\pgfsetstrokecolor{currentstroke}%
\pgfsetdash{}{0pt}%
\pgfpathmoveto{\pgfqpoint{1.610367in}{1.778908in}}%
\pgfpathlineto{\pgfqpoint{1.647110in}{1.796618in}}%
\pgfpathlineto{\pgfqpoint{1.610104in}{1.814265in}}%
\pgfpathlineto{\pgfqpoint{1.573361in}{1.796618in}}%
\pgfpathclose%
\pgfusepath{fill}%
\end{pgfscope}%
\begin{pgfscope}%
\pgfpathrectangle{\pgfqpoint{0.150000in}{0.150000in}}{\pgfqpoint{2.700000in}{1.950000in}}%
\pgfusepath{clip}%
\pgfsetbuttcap%
\pgfsetroundjoin%
\definecolor{currentfill}{rgb}{0.524219,0.582812,0.664844}%
\pgfsetfillcolor{currentfill}%
\pgfsetlinewidth{0.000000pt}%
\definecolor{currentstroke}{rgb}{0.000000,0.000000,0.000000}%
\pgfsetstrokecolor{currentstroke}%
\pgfsetdash{}{0pt}%
\pgfpathmoveto{\pgfqpoint{1.536486in}{1.778908in}}%
\pgfpathlineto{\pgfqpoint{1.573361in}{1.796618in}}%
\pgfpathlineto{\pgfqpoint{1.536486in}{1.814265in}}%
\pgfpathlineto{\pgfqpoint{1.499612in}{1.796618in}}%
\pgfpathclose%
\pgfusepath{fill}%
\end{pgfscope}%
\begin{pgfscope}%
\pgfpathrectangle{\pgfqpoint{0.150000in}{0.150000in}}{\pgfqpoint{2.700000in}{1.950000in}}%
\pgfusepath{clip}%
\pgfsetbuttcap%
\pgfsetroundjoin%
\definecolor{currentfill}{rgb}{0.524219,0.582812,0.664844}%
\pgfsetfillcolor{currentfill}%
\pgfsetlinewidth{0.000000pt}%
\definecolor{currentstroke}{rgb}{0.000000,0.000000,0.000000}%
\pgfsetstrokecolor{currentstroke}%
\pgfsetdash{}{0pt}%
\pgfpathmoveto{\pgfqpoint{1.462606in}{1.778908in}}%
\pgfpathlineto{\pgfqpoint{1.499612in}{1.796618in}}%
\pgfpathlineto{\pgfqpoint{1.462869in}{1.814265in}}%
\pgfpathlineto{\pgfqpoint{1.425863in}{1.796618in}}%
\pgfpathclose%
\pgfusepath{fill}%
\end{pgfscope}%
\begin{pgfscope}%
\pgfpathrectangle{\pgfqpoint{0.150000in}{0.150000in}}{\pgfqpoint{2.700000in}{1.950000in}}%
\pgfusepath{clip}%
\pgfsetbuttcap%
\pgfsetroundjoin%
\definecolor{currentfill}{rgb}{0.959574,0.964553,0.971523}%
\pgfsetfillcolor{currentfill}%
\pgfsetlinewidth{0.000000pt}%
\definecolor{currentstroke}{rgb}{0.000000,0.000000,0.000000}%
\pgfsetstrokecolor{currentstroke}%
\pgfsetdash{}{0pt}%
\pgfpathmoveto{\pgfqpoint{2.272995in}{1.067118in}}%
\pgfpathlineto{\pgfqpoint{2.308443in}{1.086040in}}%
\pgfpathlineto{\pgfqpoint{2.275090in}{1.172303in}}%
\pgfpathlineto{\pgfqpoint{2.239418in}{1.153447in}}%
\pgfpathclose%
\pgfusepath{fill}%
\end{pgfscope}%
\begin{pgfscope}%
\pgfpathrectangle{\pgfqpoint{0.150000in}{0.150000in}}{\pgfqpoint{2.700000in}{1.950000in}}%
\pgfusepath{clip}%
\pgfsetbuttcap%
\pgfsetroundjoin%
\definecolor{currentfill}{rgb}{0.979105,0.962086,0.963434}%
\pgfsetfillcolor{currentfill}%
\pgfsetlinewidth{0.000000pt}%
\definecolor{currentstroke}{rgb}{0.000000,0.000000,0.000000}%
\pgfsetstrokecolor{currentstroke}%
\pgfsetdash{}{0pt}%
\pgfpathmoveto{\pgfqpoint{2.200539in}{0.935243in}}%
\pgfpathlineto{\pgfqpoint{2.236180in}{0.954435in}}%
\pgfpathlineto{\pgfqpoint{2.201718in}{1.029069in}}%
\pgfpathlineto{\pgfqpoint{2.165887in}{1.009942in}}%
\pgfpathclose%
\pgfusepath{fill}%
\end{pgfscope}%
\begin{pgfscope}%
\pgfpathrectangle{\pgfqpoint{0.150000in}{0.150000in}}{\pgfqpoint{2.700000in}{1.950000in}}%
\pgfusepath{clip}%
\pgfsetbuttcap%
\pgfsetroundjoin%
\definecolor{currentfill}{rgb}{0.933517,0.879366,0.883655}%
\pgfsetfillcolor{currentfill}%
\pgfsetlinewidth{0.000000pt}%
\definecolor{currentstroke}{rgb}{0.000000,0.000000,0.000000}%
\pgfsetstrokecolor{currentstroke}%
\pgfsetdash{}{0pt}%
\pgfpathmoveto{\pgfqpoint{2.163656in}{0.822355in}}%
\pgfpathlineto{\pgfqpoint{2.199364in}{0.841748in}}%
\pgfpathlineto{\pgfqpoint{2.164770in}{0.915982in}}%
\pgfpathlineto{\pgfqpoint{2.128873in}{0.896652in}}%
\pgfpathclose%
\pgfusepath{fill}%
\end{pgfscope}%
\begin{pgfscope}%
\pgfpathrectangle{\pgfqpoint{0.150000in}{0.150000in}}{\pgfqpoint{2.700000in}{1.950000in}}%
\pgfusepath{clip}%
\pgfsetbuttcap%
\pgfsetroundjoin%
\definecolor{currentfill}{rgb}{0.887929,0.796645,0.803876}%
\pgfsetfillcolor{currentfill}%
\pgfsetlinewidth{0.000000pt}%
\definecolor{currentstroke}{rgb}{0.000000,0.000000,0.000000}%
\pgfsetstrokecolor{currentstroke}%
\pgfsetdash{}{0pt}%
\pgfpathmoveto{\pgfqpoint{2.126773in}{0.709463in}}%
\pgfpathlineto{\pgfqpoint{2.162547in}{0.729059in}}%
\pgfpathlineto{\pgfqpoint{2.127821in}{0.802891in}}%
\pgfpathlineto{\pgfqpoint{2.091857in}{0.783358in}}%
\pgfpathclose%
\pgfusepath{fill}%
\end{pgfscope}%
\begin{pgfscope}%
\pgfpathrectangle{\pgfqpoint{0.150000in}{0.150000in}}{\pgfqpoint{2.700000in}{1.950000in}}%
\pgfusepath{clip}%
\pgfsetbuttcap%
\pgfsetroundjoin%
\definecolor{currentfill}{rgb}{0.846140,0.720818,0.730744}%
\pgfsetfillcolor{currentfill}%
\pgfsetlinewidth{0.000000pt}%
\definecolor{currentstroke}{rgb}{0.000000,0.000000,0.000000}%
\pgfsetstrokecolor{currentstroke}%
\pgfsetdash{}{0pt}%
\pgfpathmoveto{\pgfqpoint{2.089888in}{0.596568in}}%
\pgfpathlineto{\pgfqpoint{2.125728in}{0.616366in}}%
\pgfpathlineto{\pgfqpoint{2.090871in}{0.689797in}}%
\pgfpathlineto{\pgfqpoint{2.054840in}{0.670061in}}%
\pgfpathclose%
\pgfusepath{fill}%
\end{pgfscope}%
\begin{pgfscope}%
\pgfpathrectangle{\pgfqpoint{0.150000in}{0.150000in}}{\pgfqpoint{2.700000in}{1.950000in}}%
\pgfusepath{clip}%
\pgfsetbuttcap%
\pgfsetroundjoin%
\definecolor{currentfill}{rgb}{0.524219,0.582812,0.664844}%
\pgfsetfillcolor{currentfill}%
\pgfsetlinewidth{0.000000pt}%
\definecolor{currentstroke}{rgb}{0.000000,0.000000,0.000000}%
\pgfsetstrokecolor{currentstroke}%
\pgfsetdash{}{0pt}%
\pgfpathmoveto{\pgfqpoint{1.647506in}{1.761134in}}%
\pgfpathlineto{\pgfqpoint{1.684248in}{1.778908in}}%
\pgfpathlineto{\pgfqpoint{1.647110in}{1.796618in}}%
\pgfpathlineto{\pgfqpoint{1.610367in}{1.778908in}}%
\pgfpathclose%
\pgfusepath{fill}%
\end{pgfscope}%
\begin{pgfscope}%
\pgfpathrectangle{\pgfqpoint{0.150000in}{0.150000in}}{\pgfqpoint{2.700000in}{1.950000in}}%
\pgfusepath{clip}%
\pgfsetbuttcap%
\pgfsetroundjoin%
\definecolor{currentfill}{rgb}{0.524219,0.582812,0.664844}%
\pgfsetfillcolor{currentfill}%
\pgfsetlinewidth{0.000000pt}%
\definecolor{currentstroke}{rgb}{0.000000,0.000000,0.000000}%
\pgfsetstrokecolor{currentstroke}%
\pgfsetdash{}{0pt}%
\pgfpathmoveto{\pgfqpoint{1.573493in}{1.761134in}}%
\pgfpathlineto{\pgfqpoint{1.610367in}{1.778908in}}%
\pgfpathlineto{\pgfqpoint{1.573361in}{1.796618in}}%
\pgfpathlineto{\pgfqpoint{1.536486in}{1.778908in}}%
\pgfpathclose%
\pgfusepath{fill}%
\end{pgfscope}%
\begin{pgfscope}%
\pgfpathrectangle{\pgfqpoint{0.150000in}{0.150000in}}{\pgfqpoint{2.700000in}{1.950000in}}%
\pgfusepath{clip}%
\pgfsetbuttcap%
\pgfsetroundjoin%
\definecolor{currentfill}{rgb}{0.524219,0.582812,0.664844}%
\pgfsetfillcolor{currentfill}%
\pgfsetlinewidth{0.000000pt}%
\definecolor{currentstroke}{rgb}{0.000000,0.000000,0.000000}%
\pgfsetstrokecolor{currentstroke}%
\pgfsetdash{}{0pt}%
\pgfpathmoveto{\pgfqpoint{1.499480in}{1.761134in}}%
\pgfpathlineto{\pgfqpoint{1.536486in}{1.778908in}}%
\pgfpathlineto{\pgfqpoint{1.499612in}{1.796618in}}%
\pgfpathlineto{\pgfqpoint{1.462606in}{1.778908in}}%
\pgfpathclose%
\pgfusepath{fill}%
\end{pgfscope}%
\begin{pgfscope}%
\pgfpathrectangle{\pgfqpoint{0.150000in}{0.150000in}}{\pgfqpoint{2.700000in}{1.950000in}}%
\pgfusepath{clip}%
\pgfsetbuttcap%
\pgfsetroundjoin%
\definecolor{currentfill}{rgb}{0.524219,0.582812,0.664844}%
\pgfsetfillcolor{currentfill}%
\pgfsetlinewidth{0.000000pt}%
\definecolor{currentstroke}{rgb}{0.000000,0.000000,0.000000}%
\pgfsetstrokecolor{currentstroke}%
\pgfsetdash{}{0pt}%
\pgfpathmoveto{\pgfqpoint{1.425467in}{1.761134in}}%
\pgfpathlineto{\pgfqpoint{1.462606in}{1.778908in}}%
\pgfpathlineto{\pgfqpoint{1.425863in}{1.796618in}}%
\pgfpathlineto{\pgfqpoint{1.388725in}{1.778908in}}%
\pgfpathclose%
\pgfusepath{fill}%
\end{pgfscope}%
\begin{pgfscope}%
\pgfpathrectangle{\pgfqpoint{0.150000in}{0.150000in}}{\pgfqpoint{2.700000in}{1.950000in}}%
\pgfusepath{clip}%
\pgfsetbuttcap%
\pgfsetroundjoin%
\definecolor{currentfill}{rgb}{0.959574,0.964553,0.971523}%
\pgfsetfillcolor{currentfill}%
\pgfsetlinewidth{0.000000pt}%
\definecolor{currentstroke}{rgb}{0.000000,0.000000,0.000000}%
\pgfsetstrokecolor{currentstroke}%
\pgfsetdash{}{0pt}%
\pgfpathmoveto{\pgfqpoint{2.237420in}{1.048127in}}%
\pgfpathlineto{\pgfqpoint{2.272995in}{1.067118in}}%
\pgfpathlineto{\pgfqpoint{2.239418in}{1.153447in}}%
\pgfpathlineto{\pgfqpoint{2.203617in}{1.134523in}}%
\pgfpathclose%
\pgfusepath{fill}%
\end{pgfscope}%
\begin{pgfscope}%
\pgfpathrectangle{\pgfqpoint{0.150000in}{0.150000in}}{\pgfqpoint{2.700000in}{1.950000in}}%
\pgfusepath{clip}%
\pgfsetbuttcap%
\pgfsetroundjoin%
\definecolor{currentfill}{rgb}{0.979105,0.962086,0.963434}%
\pgfsetfillcolor{currentfill}%
\pgfsetlinewidth{0.000000pt}%
\definecolor{currentstroke}{rgb}{0.000000,0.000000,0.000000}%
\pgfsetstrokecolor{currentstroke}%
\pgfsetdash{}{0pt}%
\pgfpathmoveto{\pgfqpoint{2.164770in}{0.915982in}}%
\pgfpathlineto{\pgfqpoint{2.200539in}{0.935243in}}%
\pgfpathlineto{\pgfqpoint{2.165887in}{1.009942in}}%
\pgfpathlineto{\pgfqpoint{2.129928in}{0.990746in}}%
\pgfpathclose%
\pgfusepath{fill}%
\end{pgfscope}%
\begin{pgfscope}%
\pgfpathrectangle{\pgfqpoint{0.150000in}{0.150000in}}{\pgfqpoint{2.700000in}{1.950000in}}%
\pgfusepath{clip}%
\pgfsetbuttcap%
\pgfsetroundjoin%
\definecolor{currentfill}{rgb}{0.933517,0.879366,0.883655}%
\pgfsetfillcolor{currentfill}%
\pgfsetlinewidth{0.000000pt}%
\definecolor{currentstroke}{rgb}{0.000000,0.000000,0.000000}%
\pgfsetstrokecolor{currentstroke}%
\pgfsetdash{}{0pt}%
\pgfpathmoveto{\pgfqpoint{2.127821in}{0.802891in}}%
\pgfpathlineto{\pgfqpoint{2.163656in}{0.822355in}}%
\pgfpathlineto{\pgfqpoint{2.128873in}{0.896652in}}%
\pgfpathlineto{\pgfqpoint{2.092847in}{0.877252in}}%
\pgfpathclose%
\pgfusepath{fill}%
\end{pgfscope}%
\begin{pgfscope}%
\pgfpathrectangle{\pgfqpoint{0.150000in}{0.150000in}}{\pgfqpoint{2.700000in}{1.950000in}}%
\pgfusepath{clip}%
\pgfsetbuttcap%
\pgfsetroundjoin%
\definecolor{currentfill}{rgb}{0.887929,0.796645,0.803876}%
\pgfsetfillcolor{currentfill}%
\pgfsetlinewidth{0.000000pt}%
\definecolor{currentstroke}{rgb}{0.000000,0.000000,0.000000}%
\pgfsetstrokecolor{currentstroke}%
\pgfsetdash{}{0pt}%
\pgfpathmoveto{\pgfqpoint{2.090871in}{0.689797in}}%
\pgfpathlineto{\pgfqpoint{2.126773in}{0.709463in}}%
\pgfpathlineto{\pgfqpoint{2.091857in}{0.783358in}}%
\pgfpathlineto{\pgfqpoint{2.055764in}{0.763755in}}%
\pgfpathclose%
\pgfusepath{fill}%
\end{pgfscope}%
\begin{pgfscope}%
\pgfpathrectangle{\pgfqpoint{0.150000in}{0.150000in}}{\pgfqpoint{2.700000in}{1.950000in}}%
\pgfusepath{clip}%
\pgfsetbuttcap%
\pgfsetroundjoin%
\definecolor{currentfill}{rgb}{0.846140,0.720818,0.730744}%
\pgfsetfillcolor{currentfill}%
\pgfsetlinewidth{0.000000pt}%
\definecolor{currentstroke}{rgb}{0.000000,0.000000,0.000000}%
\pgfsetstrokecolor{currentstroke}%
\pgfsetdash{}{0pt}%
\pgfpathmoveto{\pgfqpoint{2.053920in}{0.576700in}}%
\pgfpathlineto{\pgfqpoint{2.089888in}{0.596568in}}%
\pgfpathlineto{\pgfqpoint{2.054840in}{0.670061in}}%
\pgfpathlineto{\pgfqpoint{2.018680in}{0.650254in}}%
\pgfpathclose%
\pgfusepath{fill}%
\end{pgfscope}%
\begin{pgfscope}%
\pgfpathrectangle{\pgfqpoint{0.150000in}{0.150000in}}{\pgfqpoint{2.700000in}{1.950000in}}%
\pgfusepath{clip}%
\pgfsetbuttcap%
\pgfsetroundjoin%
\definecolor{currentfill}{rgb}{0.524219,0.582812,0.664844}%
\pgfsetfillcolor{currentfill}%
\pgfsetlinewidth{0.000000pt}%
\definecolor{currentstroke}{rgb}{0.000000,0.000000,0.000000}%
\pgfsetstrokecolor{currentstroke}%
\pgfsetdash{}{0pt}%
\pgfpathmoveto{\pgfqpoint{1.684779in}{1.743296in}}%
\pgfpathlineto{\pgfqpoint{1.721520in}{1.761134in}}%
\pgfpathlineto{\pgfqpoint{1.684248in}{1.778908in}}%
\pgfpathlineto{\pgfqpoint{1.647506in}{1.761134in}}%
\pgfpathclose%
\pgfusepath{fill}%
\end{pgfscope}%
\begin{pgfscope}%
\pgfpathrectangle{\pgfqpoint{0.150000in}{0.150000in}}{\pgfqpoint{2.700000in}{1.950000in}}%
\pgfusepath{clip}%
\pgfsetbuttcap%
\pgfsetroundjoin%
\definecolor{currentfill}{rgb}{0.524219,0.582812,0.664844}%
\pgfsetfillcolor{currentfill}%
\pgfsetlinewidth{0.000000pt}%
\definecolor{currentstroke}{rgb}{0.000000,0.000000,0.000000}%
\pgfsetstrokecolor{currentstroke}%
\pgfsetdash{}{0pt}%
\pgfpathmoveto{\pgfqpoint{1.610633in}{1.743296in}}%
\pgfpathlineto{\pgfqpoint{1.647506in}{1.761134in}}%
\pgfpathlineto{\pgfqpoint{1.610367in}{1.778908in}}%
\pgfpathlineto{\pgfqpoint{1.573493in}{1.761134in}}%
\pgfpathclose%
\pgfusepath{fill}%
\end{pgfscope}%
\begin{pgfscope}%
\pgfpathrectangle{\pgfqpoint{0.150000in}{0.150000in}}{\pgfqpoint{2.700000in}{1.950000in}}%
\pgfusepath{clip}%
\pgfsetbuttcap%
\pgfsetroundjoin%
\definecolor{currentfill}{rgb}{0.524219,0.582812,0.664844}%
\pgfsetfillcolor{currentfill}%
\pgfsetlinewidth{0.000000pt}%
\definecolor{currentstroke}{rgb}{0.000000,0.000000,0.000000}%
\pgfsetstrokecolor{currentstroke}%
\pgfsetdash{}{0pt}%
\pgfpathmoveto{\pgfqpoint{1.536486in}{1.743296in}}%
\pgfpathlineto{\pgfqpoint{1.573493in}{1.761134in}}%
\pgfpathlineto{\pgfqpoint{1.536486in}{1.778908in}}%
\pgfpathlineto{\pgfqpoint{1.499480in}{1.761134in}}%
\pgfpathclose%
\pgfusepath{fill}%
\end{pgfscope}%
\begin{pgfscope}%
\pgfpathrectangle{\pgfqpoint{0.150000in}{0.150000in}}{\pgfqpoint{2.700000in}{1.950000in}}%
\pgfusepath{clip}%
\pgfsetbuttcap%
\pgfsetroundjoin%
\definecolor{currentfill}{rgb}{0.524219,0.582812,0.664844}%
\pgfsetfillcolor{currentfill}%
\pgfsetlinewidth{0.000000pt}%
\definecolor{currentstroke}{rgb}{0.000000,0.000000,0.000000}%
\pgfsetstrokecolor{currentstroke}%
\pgfsetdash{}{0pt}%
\pgfpathmoveto{\pgfqpoint{1.462340in}{1.743296in}}%
\pgfpathlineto{\pgfqpoint{1.499480in}{1.761134in}}%
\pgfpathlineto{\pgfqpoint{1.462606in}{1.778908in}}%
\pgfpathlineto{\pgfqpoint{1.425467in}{1.761134in}}%
\pgfpathclose%
\pgfusepath{fill}%
\end{pgfscope}%
\begin{pgfscope}%
\pgfpathrectangle{\pgfqpoint{0.150000in}{0.150000in}}{\pgfqpoint{2.700000in}{1.950000in}}%
\pgfusepath{clip}%
\pgfsetbuttcap%
\pgfsetroundjoin%
\definecolor{currentfill}{rgb}{0.524219,0.582812,0.664844}%
\pgfsetfillcolor{currentfill}%
\pgfsetlinewidth{0.000000pt}%
\definecolor{currentstroke}{rgb}{0.000000,0.000000,0.000000}%
\pgfsetstrokecolor{currentstroke}%
\pgfsetdash{}{0pt}%
\pgfpathmoveto{\pgfqpoint{1.388194in}{1.743296in}}%
\pgfpathlineto{\pgfqpoint{1.425467in}{1.761134in}}%
\pgfpathlineto{\pgfqpoint{1.388725in}{1.778908in}}%
\pgfpathlineto{\pgfqpoint{1.351453in}{1.761134in}}%
\pgfpathclose%
\pgfusepath{fill}%
\end{pgfscope}%
\begin{pgfscope}%
\pgfpathrectangle{\pgfqpoint{0.150000in}{0.150000in}}{\pgfqpoint{2.700000in}{1.950000in}}%
\pgfusepath{clip}%
\pgfsetbuttcap%
\pgfsetroundjoin%
\definecolor{currentfill}{rgb}{0.959574,0.964553,0.971523}%
\pgfsetfillcolor{currentfill}%
\pgfsetlinewidth{0.000000pt}%
\definecolor{currentstroke}{rgb}{0.000000,0.000000,0.000000}%
\pgfsetstrokecolor{currentstroke}%
\pgfsetdash{}{0pt}%
\pgfpathmoveto{\pgfqpoint{2.201718in}{1.029069in}}%
\pgfpathlineto{\pgfqpoint{2.237420in}{1.048127in}}%
\pgfpathlineto{\pgfqpoint{2.203617in}{1.134523in}}%
\pgfpathlineto{\pgfqpoint{2.167688in}{1.115531in}}%
\pgfpathclose%
\pgfusepath{fill}%
\end{pgfscope}%
\begin{pgfscope}%
\pgfpathrectangle{\pgfqpoint{0.150000in}{0.150000in}}{\pgfqpoint{2.700000in}{1.950000in}}%
\pgfusepath{clip}%
\pgfsetbuttcap%
\pgfsetroundjoin%
\definecolor{currentfill}{rgb}{0.878722,0.893658,0.914568}%
\pgfsetfillcolor{currentfill}%
\pgfsetlinewidth{0.000000pt}%
\definecolor{currentstroke}{rgb}{0.000000,0.000000,0.000000}%
\pgfsetstrokecolor{currentstroke}%
\pgfsetdash{}{0pt}%
\pgfpathmoveto{\pgfqpoint{2.239418in}{1.153447in}}%
\pgfpathlineto{\pgfqpoint{2.275090in}{1.172303in}}%
\pgfpathlineto{\pgfqpoint{2.240669in}{1.247928in}}%
\pgfpathlineto{\pgfqpoint{2.204807in}{1.229140in}}%
\pgfpathclose%
\pgfusepath{fill}%
\end{pgfscope}%
\begin{pgfscope}%
\pgfpathrectangle{\pgfqpoint{0.150000in}{0.150000in}}{\pgfqpoint{2.700000in}{1.950000in}}%
\pgfusepath{clip}%
\pgfsetbuttcap%
\pgfsetroundjoin%
\definecolor{currentfill}{rgb}{0.979105,0.962086,0.963434}%
\pgfsetfillcolor{currentfill}%
\pgfsetlinewidth{0.000000pt}%
\definecolor{currentstroke}{rgb}{0.000000,0.000000,0.000000}%
\pgfsetstrokecolor{currentstroke}%
\pgfsetdash{}{0pt}%
\pgfpathmoveto{\pgfqpoint{2.128873in}{0.896652in}}%
\pgfpathlineto{\pgfqpoint{2.164770in}{0.915982in}}%
\pgfpathlineto{\pgfqpoint{2.129928in}{0.990746in}}%
\pgfpathlineto{\pgfqpoint{2.093840in}{0.971481in}}%
\pgfpathclose%
\pgfusepath{fill}%
\end{pgfscope}%
\begin{pgfscope}%
\pgfpathrectangle{\pgfqpoint{0.150000in}{0.150000in}}{\pgfqpoint{2.700000in}{1.950000in}}%
\pgfusepath{clip}%
\pgfsetbuttcap%
\pgfsetroundjoin%
\definecolor{currentfill}{rgb}{0.933517,0.879366,0.883655}%
\pgfsetfillcolor{currentfill}%
\pgfsetlinewidth{0.000000pt}%
\definecolor{currentstroke}{rgb}{0.000000,0.000000,0.000000}%
\pgfsetstrokecolor{currentstroke}%
\pgfsetdash{}{0pt}%
\pgfpathmoveto{\pgfqpoint{2.091857in}{0.783358in}}%
\pgfpathlineto{\pgfqpoint{2.127821in}{0.802891in}}%
\pgfpathlineto{\pgfqpoint{2.092847in}{0.877252in}}%
\pgfpathlineto{\pgfqpoint{2.056691in}{0.857783in}}%
\pgfpathclose%
\pgfusepath{fill}%
\end{pgfscope}%
\begin{pgfscope}%
\pgfpathrectangle{\pgfqpoint{0.150000in}{0.150000in}}{\pgfqpoint{2.700000in}{1.950000in}}%
\pgfusepath{clip}%
\pgfsetbuttcap%
\pgfsetroundjoin%
\definecolor{currentfill}{rgb}{0.887929,0.796645,0.803876}%
\pgfsetfillcolor{currentfill}%
\pgfsetlinewidth{0.000000pt}%
\definecolor{currentstroke}{rgb}{0.000000,0.000000,0.000000}%
\pgfsetstrokecolor{currentstroke}%
\pgfsetdash{}{0pt}%
\pgfpathmoveto{\pgfqpoint{2.054840in}{0.670061in}}%
\pgfpathlineto{\pgfqpoint{2.090871in}{0.689797in}}%
\pgfpathlineto{\pgfqpoint{2.055764in}{0.763755in}}%
\pgfpathlineto{\pgfqpoint{2.019541in}{0.744081in}}%
\pgfpathclose%
\pgfusepath{fill}%
\end{pgfscope}%
\begin{pgfscope}%
\pgfpathrectangle{\pgfqpoint{0.150000in}{0.150000in}}{\pgfqpoint{2.700000in}{1.950000in}}%
\pgfusepath{clip}%
\pgfsetbuttcap%
\pgfsetroundjoin%
\definecolor{currentfill}{rgb}{0.846140,0.720818,0.730744}%
\pgfsetfillcolor{currentfill}%
\pgfsetlinewidth{0.000000pt}%
\definecolor{currentstroke}{rgb}{0.000000,0.000000,0.000000}%
\pgfsetstrokecolor{currentstroke}%
\pgfsetdash{}{0pt}%
\pgfpathmoveto{\pgfqpoint{2.017822in}{0.556761in}}%
\pgfpathlineto{\pgfqpoint{2.053920in}{0.576700in}}%
\pgfpathlineto{\pgfqpoint{2.018680in}{0.650254in}}%
\pgfpathlineto{\pgfqpoint{1.982389in}{0.630375in}}%
\pgfpathclose%
\pgfusepath{fill}%
\end{pgfscope}%
\begin{pgfscope}%
\pgfpathrectangle{\pgfqpoint{0.150000in}{0.150000in}}{\pgfqpoint{2.700000in}{1.950000in}}%
\pgfusepath{clip}%
\pgfsetbuttcap%
\pgfsetroundjoin%
\definecolor{currentfill}{rgb}{0.524219,0.582812,0.664844}%
\pgfsetfillcolor{currentfill}%
\pgfsetlinewidth{0.000000pt}%
\definecolor{currentstroke}{rgb}{0.000000,0.000000,0.000000}%
\pgfsetstrokecolor{currentstroke}%
\pgfsetdash{}{0pt}%
\pgfpathmoveto{\pgfqpoint{1.722186in}{1.725395in}}%
\pgfpathlineto{\pgfqpoint{1.758925in}{1.743296in}}%
\pgfpathlineto{\pgfqpoint{1.721520in}{1.761134in}}%
\pgfpathlineto{\pgfqpoint{1.684779in}{1.743296in}}%
\pgfpathclose%
\pgfusepath{fill}%
\end{pgfscope}%
\begin{pgfscope}%
\pgfpathrectangle{\pgfqpoint{0.150000in}{0.150000in}}{\pgfqpoint{2.700000in}{1.950000in}}%
\pgfusepath{clip}%
\pgfsetbuttcap%
\pgfsetroundjoin%
\definecolor{currentfill}{rgb}{0.524219,0.582812,0.664844}%
\pgfsetfillcolor{currentfill}%
\pgfsetlinewidth{0.000000pt}%
\definecolor{currentstroke}{rgb}{0.000000,0.000000,0.000000}%
\pgfsetstrokecolor{currentstroke}%
\pgfsetdash{}{0pt}%
\pgfpathmoveto{\pgfqpoint{1.647906in}{1.725395in}}%
\pgfpathlineto{\pgfqpoint{1.684779in}{1.743296in}}%
\pgfpathlineto{\pgfqpoint{1.647506in}{1.761134in}}%
\pgfpathlineto{\pgfqpoint{1.610633in}{1.743296in}}%
\pgfpathclose%
\pgfusepath{fill}%
\end{pgfscope}%
\begin{pgfscope}%
\pgfpathrectangle{\pgfqpoint{0.150000in}{0.150000in}}{\pgfqpoint{2.700000in}{1.950000in}}%
\pgfusepath{clip}%
\pgfsetbuttcap%
\pgfsetroundjoin%
\definecolor{currentfill}{rgb}{0.524219,0.582812,0.664844}%
\pgfsetfillcolor{currentfill}%
\pgfsetlinewidth{0.000000pt}%
\definecolor{currentstroke}{rgb}{0.000000,0.000000,0.000000}%
\pgfsetstrokecolor{currentstroke}%
\pgfsetdash{}{0pt}%
\pgfpathmoveto{\pgfqpoint{1.573626in}{1.725395in}}%
\pgfpathlineto{\pgfqpoint{1.610633in}{1.743296in}}%
\pgfpathlineto{\pgfqpoint{1.573493in}{1.761134in}}%
\pgfpathlineto{\pgfqpoint{1.536486in}{1.743296in}}%
\pgfpathclose%
\pgfusepath{fill}%
\end{pgfscope}%
\begin{pgfscope}%
\pgfpathrectangle{\pgfqpoint{0.150000in}{0.150000in}}{\pgfqpoint{2.700000in}{1.950000in}}%
\pgfusepath{clip}%
\pgfsetbuttcap%
\pgfsetroundjoin%
\definecolor{currentfill}{rgb}{0.524219,0.582812,0.664844}%
\pgfsetfillcolor{currentfill}%
\pgfsetlinewidth{0.000000pt}%
\definecolor{currentstroke}{rgb}{0.000000,0.000000,0.000000}%
\pgfsetstrokecolor{currentstroke}%
\pgfsetdash{}{0pt}%
\pgfpathmoveto{\pgfqpoint{1.499347in}{1.725395in}}%
\pgfpathlineto{\pgfqpoint{1.536486in}{1.743296in}}%
\pgfpathlineto{\pgfqpoint{1.499480in}{1.761134in}}%
\pgfpathlineto{\pgfqpoint{1.462340in}{1.743296in}}%
\pgfpathclose%
\pgfusepath{fill}%
\end{pgfscope}%
\begin{pgfscope}%
\pgfpathrectangle{\pgfqpoint{0.150000in}{0.150000in}}{\pgfqpoint{2.700000in}{1.950000in}}%
\pgfusepath{clip}%
\pgfsetbuttcap%
\pgfsetroundjoin%
\definecolor{currentfill}{rgb}{0.524219,0.582812,0.664844}%
\pgfsetfillcolor{currentfill}%
\pgfsetlinewidth{0.000000pt}%
\definecolor{currentstroke}{rgb}{0.000000,0.000000,0.000000}%
\pgfsetstrokecolor{currentstroke}%
\pgfsetdash{}{0pt}%
\pgfpathmoveto{\pgfqpoint{1.425067in}{1.725395in}}%
\pgfpathlineto{\pgfqpoint{1.462340in}{1.743296in}}%
\pgfpathlineto{\pgfqpoint{1.425467in}{1.761134in}}%
\pgfpathlineto{\pgfqpoint{1.388194in}{1.743296in}}%
\pgfpathclose%
\pgfusepath{fill}%
\end{pgfscope}%
\begin{pgfscope}%
\pgfpathrectangle{\pgfqpoint{0.150000in}{0.150000in}}{\pgfqpoint{2.700000in}{1.950000in}}%
\pgfusepath{clip}%
\pgfsetbuttcap%
\pgfsetroundjoin%
\definecolor{currentfill}{rgb}{0.524219,0.582812,0.664844}%
\pgfsetfillcolor{currentfill}%
\pgfsetlinewidth{0.000000pt}%
\definecolor{currentstroke}{rgb}{0.000000,0.000000,0.000000}%
\pgfsetstrokecolor{currentstroke}%
\pgfsetdash{}{0pt}%
\pgfpathmoveto{\pgfqpoint{1.350787in}{1.725395in}}%
\pgfpathlineto{\pgfqpoint{1.388194in}{1.743296in}}%
\pgfpathlineto{\pgfqpoint{1.351453in}{1.761134in}}%
\pgfpathlineto{\pgfqpoint{1.314048in}{1.743296in}}%
\pgfpathclose%
\pgfusepath{fill}%
\end{pgfscope}%
\begin{pgfscope}%
\pgfpathrectangle{\pgfqpoint{0.150000in}{0.150000in}}{\pgfqpoint{2.700000in}{1.950000in}}%
\pgfusepath{clip}%
\pgfsetbuttcap%
\pgfsetroundjoin%
\definecolor{currentfill}{rgb}{0.959574,0.964553,0.971523}%
\pgfsetfillcolor{currentfill}%
\pgfsetlinewidth{0.000000pt}%
\definecolor{currentstroke}{rgb}{0.000000,0.000000,0.000000}%
\pgfsetstrokecolor{currentstroke}%
\pgfsetdash{}{0pt}%
\pgfpathmoveto{\pgfqpoint{2.165887in}{1.009942in}}%
\pgfpathlineto{\pgfqpoint{2.201718in}{1.029069in}}%
\pgfpathlineto{\pgfqpoint{2.167688in}{1.115531in}}%
\pgfpathlineto{\pgfqpoint{2.131629in}{1.096471in}}%
\pgfpathclose%
\pgfusepath{fill}%
\end{pgfscope}%
\begin{pgfscope}%
\pgfpathrectangle{\pgfqpoint{0.150000in}{0.150000in}}{\pgfqpoint{2.700000in}{1.950000in}}%
\pgfusepath{clip}%
\pgfsetbuttcap%
\pgfsetroundjoin%
\definecolor{currentfill}{rgb}{0.878722,0.893658,0.914568}%
\pgfsetfillcolor{currentfill}%
\pgfsetlinewidth{0.000000pt}%
\definecolor{currentstroke}{rgb}{0.000000,0.000000,0.000000}%
\pgfsetstrokecolor{currentstroke}%
\pgfsetdash{}{0pt}%
\pgfpathmoveto{\pgfqpoint{2.203617in}{1.134523in}}%
\pgfpathlineto{\pgfqpoint{2.239418in}{1.153447in}}%
\pgfpathlineto{\pgfqpoint{2.204807in}{1.229140in}}%
\pgfpathlineto{\pgfqpoint{2.168816in}{1.210285in}}%
\pgfpathclose%
\pgfusepath{fill}%
\end{pgfscope}%
\begin{pgfscope}%
\pgfpathrectangle{\pgfqpoint{0.150000in}{0.150000in}}{\pgfqpoint{2.700000in}{1.950000in}}%
\pgfusepath{clip}%
\pgfsetbuttcap%
\pgfsetroundjoin%
\definecolor{currentfill}{rgb}{0.979105,0.962086,0.963434}%
\pgfsetfillcolor{currentfill}%
\pgfsetlinewidth{0.000000pt}%
\definecolor{currentstroke}{rgb}{0.000000,0.000000,0.000000}%
\pgfsetstrokecolor{currentstroke}%
\pgfsetdash{}{0pt}%
\pgfpathmoveto{\pgfqpoint{2.092847in}{0.877252in}}%
\pgfpathlineto{\pgfqpoint{2.128873in}{0.896652in}}%
\pgfpathlineto{\pgfqpoint{2.093840in}{0.971481in}}%
\pgfpathlineto{\pgfqpoint{2.057621in}{0.952147in}}%
\pgfpathclose%
\pgfusepath{fill}%
\end{pgfscope}%
\begin{pgfscope}%
\pgfpathrectangle{\pgfqpoint{0.150000in}{0.150000in}}{\pgfqpoint{2.700000in}{1.950000in}}%
\pgfusepath{clip}%
\pgfsetbuttcap%
\pgfsetroundjoin%
\definecolor{currentfill}{rgb}{0.933517,0.879366,0.883655}%
\pgfsetfillcolor{currentfill}%
\pgfsetlinewidth{0.000000pt}%
\definecolor{currentstroke}{rgb}{0.000000,0.000000,0.000000}%
\pgfsetstrokecolor{currentstroke}%
\pgfsetdash{}{0pt}%
\pgfpathmoveto{\pgfqpoint{2.055764in}{0.763755in}}%
\pgfpathlineto{\pgfqpoint{2.091857in}{0.783358in}}%
\pgfpathlineto{\pgfqpoint{2.056691in}{0.857783in}}%
\pgfpathlineto{\pgfqpoint{2.020405in}{0.838243in}}%
\pgfpathclose%
\pgfusepath{fill}%
\end{pgfscope}%
\begin{pgfscope}%
\pgfpathrectangle{\pgfqpoint{0.150000in}{0.150000in}}{\pgfqpoint{2.700000in}{1.950000in}}%
\pgfusepath{clip}%
\pgfsetbuttcap%
\pgfsetroundjoin%
\definecolor{currentfill}{rgb}{0.887929,0.796645,0.803876}%
\pgfsetfillcolor{currentfill}%
\pgfsetlinewidth{0.000000pt}%
\definecolor{currentstroke}{rgb}{0.000000,0.000000,0.000000}%
\pgfsetstrokecolor{currentstroke}%
\pgfsetdash{}{0pt}%
\pgfpathmoveto{\pgfqpoint{2.018680in}{0.650254in}}%
\pgfpathlineto{\pgfqpoint{2.054840in}{0.670061in}}%
\pgfpathlineto{\pgfqpoint{2.019541in}{0.744081in}}%
\pgfpathlineto{\pgfqpoint{1.983187in}{0.724336in}}%
\pgfpathclose%
\pgfusepath{fill}%
\end{pgfscope}%
\begin{pgfscope}%
\pgfpathrectangle{\pgfqpoint{0.150000in}{0.150000in}}{\pgfqpoint{2.700000in}{1.950000in}}%
\pgfusepath{clip}%
\pgfsetbuttcap%
\pgfsetroundjoin%
\definecolor{currentfill}{rgb}{0.846140,0.720818,0.730744}%
\pgfsetfillcolor{currentfill}%
\pgfsetlinewidth{0.000000pt}%
\definecolor{currentstroke}{rgb}{0.000000,0.000000,0.000000}%
\pgfsetstrokecolor{currentstroke}%
\pgfsetdash{}{0pt}%
\pgfpathmoveto{\pgfqpoint{1.981595in}{0.536749in}}%
\pgfpathlineto{\pgfqpoint{2.017822in}{0.556761in}}%
\pgfpathlineto{\pgfqpoint{1.982389in}{0.630375in}}%
\pgfpathlineto{\pgfqpoint{1.945968in}{0.610425in}}%
\pgfpathclose%
\pgfusepath{fill}%
\end{pgfscope}%
\begin{pgfscope}%
\pgfpathrectangle{\pgfqpoint{0.150000in}{0.150000in}}{\pgfqpoint{2.700000in}{1.950000in}}%
\pgfusepath{clip}%
\pgfsetbuttcap%
\pgfsetroundjoin%
\definecolor{currentfill}{rgb}{0.524219,0.582812,0.664844}%
\pgfsetfillcolor{currentfill}%
\pgfsetlinewidth{0.000000pt}%
\definecolor{currentstroke}{rgb}{0.000000,0.000000,0.000000}%
\pgfsetstrokecolor{currentstroke}%
\pgfsetdash{}{0pt}%
\pgfpathmoveto{\pgfqpoint{1.759727in}{1.707428in}}%
\pgfpathlineto{\pgfqpoint{1.796465in}{1.725395in}}%
\pgfpathlineto{\pgfqpoint{1.758925in}{1.743296in}}%
\pgfpathlineto{\pgfqpoint{1.722186in}{1.725395in}}%
\pgfpathclose%
\pgfusepath{fill}%
\end{pgfscope}%
\begin{pgfscope}%
\pgfpathrectangle{\pgfqpoint{0.150000in}{0.150000in}}{\pgfqpoint{2.700000in}{1.950000in}}%
\pgfusepath{clip}%
\pgfsetbuttcap%
\pgfsetroundjoin%
\definecolor{currentfill}{rgb}{0.524219,0.582812,0.664844}%
\pgfsetfillcolor{currentfill}%
\pgfsetlinewidth{0.000000pt}%
\definecolor{currentstroke}{rgb}{0.000000,0.000000,0.000000}%
\pgfsetstrokecolor{currentstroke}%
\pgfsetdash{}{0pt}%
\pgfpathmoveto{\pgfqpoint{1.685314in}{1.707428in}}%
\pgfpathlineto{\pgfqpoint{1.722186in}{1.725395in}}%
\pgfpathlineto{\pgfqpoint{1.684779in}{1.743296in}}%
\pgfpathlineto{\pgfqpoint{1.647906in}{1.725395in}}%
\pgfpathclose%
\pgfusepath{fill}%
\end{pgfscope}%
\begin{pgfscope}%
\pgfpathrectangle{\pgfqpoint{0.150000in}{0.150000in}}{\pgfqpoint{2.700000in}{1.950000in}}%
\pgfusepath{clip}%
\pgfsetbuttcap%
\pgfsetroundjoin%
\definecolor{currentfill}{rgb}{0.524219,0.582812,0.664844}%
\pgfsetfillcolor{currentfill}%
\pgfsetlinewidth{0.000000pt}%
\definecolor{currentstroke}{rgb}{0.000000,0.000000,0.000000}%
\pgfsetstrokecolor{currentstroke}%
\pgfsetdash{}{0pt}%
\pgfpathmoveto{\pgfqpoint{1.610900in}{1.707428in}}%
\pgfpathlineto{\pgfqpoint{1.647906in}{1.725395in}}%
\pgfpathlineto{\pgfqpoint{1.610633in}{1.743296in}}%
\pgfpathlineto{\pgfqpoint{1.573626in}{1.725395in}}%
\pgfpathclose%
\pgfusepath{fill}%
\end{pgfscope}%
\begin{pgfscope}%
\pgfpathrectangle{\pgfqpoint{0.150000in}{0.150000in}}{\pgfqpoint{2.700000in}{1.950000in}}%
\pgfusepath{clip}%
\pgfsetbuttcap%
\pgfsetroundjoin%
\definecolor{currentfill}{rgb}{0.524219,0.582812,0.664844}%
\pgfsetfillcolor{currentfill}%
\pgfsetlinewidth{0.000000pt}%
\definecolor{currentstroke}{rgb}{0.000000,0.000000,0.000000}%
\pgfsetstrokecolor{currentstroke}%
\pgfsetdash{}{0pt}%
\pgfpathmoveto{\pgfqpoint{1.536486in}{1.707428in}}%
\pgfpathlineto{\pgfqpoint{1.573626in}{1.725395in}}%
\pgfpathlineto{\pgfqpoint{1.536486in}{1.743296in}}%
\pgfpathlineto{\pgfqpoint{1.499347in}{1.725395in}}%
\pgfpathclose%
\pgfusepath{fill}%
\end{pgfscope}%
\begin{pgfscope}%
\pgfpathrectangle{\pgfqpoint{0.150000in}{0.150000in}}{\pgfqpoint{2.700000in}{1.950000in}}%
\pgfusepath{clip}%
\pgfsetbuttcap%
\pgfsetroundjoin%
\definecolor{currentfill}{rgb}{0.524219,0.582812,0.664844}%
\pgfsetfillcolor{currentfill}%
\pgfsetlinewidth{0.000000pt}%
\definecolor{currentstroke}{rgb}{0.000000,0.000000,0.000000}%
\pgfsetstrokecolor{currentstroke}%
\pgfsetdash{}{0pt}%
\pgfpathmoveto{\pgfqpoint{1.462073in}{1.707428in}}%
\pgfpathlineto{\pgfqpoint{1.499347in}{1.725395in}}%
\pgfpathlineto{\pgfqpoint{1.462340in}{1.743296in}}%
\pgfpathlineto{\pgfqpoint{1.425067in}{1.725395in}}%
\pgfpathclose%
\pgfusepath{fill}%
\end{pgfscope}%
\begin{pgfscope}%
\pgfpathrectangle{\pgfqpoint{0.150000in}{0.150000in}}{\pgfqpoint{2.700000in}{1.950000in}}%
\pgfusepath{clip}%
\pgfsetbuttcap%
\pgfsetroundjoin%
\definecolor{currentfill}{rgb}{0.524219,0.582812,0.664844}%
\pgfsetfillcolor{currentfill}%
\pgfsetlinewidth{0.000000pt}%
\definecolor{currentstroke}{rgb}{0.000000,0.000000,0.000000}%
\pgfsetstrokecolor{currentstroke}%
\pgfsetdash{}{0pt}%
\pgfpathmoveto{\pgfqpoint{1.387659in}{1.707428in}}%
\pgfpathlineto{\pgfqpoint{1.425067in}{1.725395in}}%
\pgfpathlineto{\pgfqpoint{1.388194in}{1.743296in}}%
\pgfpathlineto{\pgfqpoint{1.350787in}{1.725395in}}%
\pgfpathclose%
\pgfusepath{fill}%
\end{pgfscope}%
\begin{pgfscope}%
\pgfpathrectangle{\pgfqpoint{0.150000in}{0.150000in}}{\pgfqpoint{2.700000in}{1.950000in}}%
\pgfusepath{clip}%
\pgfsetbuttcap%
\pgfsetroundjoin%
\definecolor{currentfill}{rgb}{0.524219,0.582812,0.664844}%
\pgfsetfillcolor{currentfill}%
\pgfsetlinewidth{0.000000pt}%
\definecolor{currentstroke}{rgb}{0.000000,0.000000,0.000000}%
\pgfsetstrokecolor{currentstroke}%
\pgfsetdash{}{0pt}%
\pgfpathmoveto{\pgfqpoint{1.313246in}{1.707428in}}%
\pgfpathlineto{\pgfqpoint{1.350787in}{1.725395in}}%
\pgfpathlineto{\pgfqpoint{1.314048in}{1.743296in}}%
\pgfpathlineto{\pgfqpoint{1.276508in}{1.725395in}}%
\pgfpathclose%
\pgfusepath{fill}%
\end{pgfscope}%
\begin{pgfscope}%
\pgfpathrectangle{\pgfqpoint{0.150000in}{0.150000in}}{\pgfqpoint{2.700000in}{1.950000in}}%
\pgfusepath{clip}%
\pgfsetbuttcap%
\pgfsetroundjoin%
\definecolor{currentfill}{rgb}{0.959574,0.964553,0.971523}%
\pgfsetfillcolor{currentfill}%
\pgfsetlinewidth{0.000000pt}%
\definecolor{currentstroke}{rgb}{0.000000,0.000000,0.000000}%
\pgfsetstrokecolor{currentstroke}%
\pgfsetdash{}{0pt}%
\pgfpathmoveto{\pgfqpoint{2.129928in}{0.990746in}}%
\pgfpathlineto{\pgfqpoint{2.165887in}{1.009942in}}%
\pgfpathlineto{\pgfqpoint{2.131629in}{1.096471in}}%
\pgfpathlineto{\pgfqpoint{2.095440in}{1.077342in}}%
\pgfpathclose%
\pgfusepath{fill}%
\end{pgfscope}%
\begin{pgfscope}%
\pgfpathrectangle{\pgfqpoint{0.150000in}{0.150000in}}{\pgfqpoint{2.700000in}{1.950000in}}%
\pgfusepath{clip}%
\pgfsetbuttcap%
\pgfsetroundjoin%
\definecolor{currentfill}{rgb}{0.804090,0.828217,0.861994}%
\pgfsetfillcolor{currentfill}%
\pgfsetlinewidth{0.000000pt}%
\definecolor{currentstroke}{rgb}{0.000000,0.000000,0.000000}%
\pgfsetstrokecolor{currentstroke}%
\pgfsetdash{}{0pt}%
\pgfpathmoveto{\pgfqpoint{2.204807in}{1.229140in}}%
\pgfpathlineto{\pgfqpoint{2.240669in}{1.247928in}}%
\pgfpathlineto{\pgfqpoint{2.206001in}{1.324096in}}%
\pgfpathlineto{\pgfqpoint{2.169948in}{1.305378in}}%
\pgfpathclose%
\pgfusepath{fill}%
\end{pgfscope}%
\begin{pgfscope}%
\pgfpathrectangle{\pgfqpoint{0.150000in}{0.150000in}}{\pgfqpoint{2.700000in}{1.950000in}}%
\pgfusepath{clip}%
\pgfsetbuttcap%
\pgfsetroundjoin%
\definecolor{currentfill}{rgb}{0.878722,0.893658,0.914568}%
\pgfsetfillcolor{currentfill}%
\pgfsetlinewidth{0.000000pt}%
\definecolor{currentstroke}{rgb}{0.000000,0.000000,0.000000}%
\pgfsetstrokecolor{currentstroke}%
\pgfsetdash{}{0pt}%
\pgfpathmoveto{\pgfqpoint{2.167688in}{1.115531in}}%
\pgfpathlineto{\pgfqpoint{2.203617in}{1.134523in}}%
\pgfpathlineto{\pgfqpoint{2.168816in}{1.210285in}}%
\pgfpathlineto{\pgfqpoint{2.132694in}{1.191362in}}%
\pgfpathclose%
\pgfusepath{fill}%
\end{pgfscope}%
\begin{pgfscope}%
\pgfpathrectangle{\pgfqpoint{0.150000in}{0.150000in}}{\pgfqpoint{2.700000in}{1.950000in}}%
\pgfusepath{clip}%
\pgfsetbuttcap%
\pgfsetroundjoin%
\definecolor{currentfill}{rgb}{0.979105,0.962086,0.963434}%
\pgfsetfillcolor{currentfill}%
\pgfsetlinewidth{0.000000pt}%
\definecolor{currentstroke}{rgb}{0.000000,0.000000,0.000000}%
\pgfsetstrokecolor{currentstroke}%
\pgfsetdash{}{0pt}%
\pgfpathmoveto{\pgfqpoint{2.056691in}{0.857783in}}%
\pgfpathlineto{\pgfqpoint{2.092847in}{0.877252in}}%
\pgfpathlineto{\pgfqpoint{2.057621in}{0.952147in}}%
\pgfpathlineto{\pgfqpoint{2.021272in}{0.932743in}}%
\pgfpathclose%
\pgfusepath{fill}%
\end{pgfscope}%
\begin{pgfscope}%
\pgfpathrectangle{\pgfqpoint{0.150000in}{0.150000in}}{\pgfqpoint{2.700000in}{1.950000in}}%
\pgfusepath{clip}%
\pgfsetbuttcap%
\pgfsetroundjoin%
\definecolor{currentfill}{rgb}{0.933517,0.879366,0.883655}%
\pgfsetfillcolor{currentfill}%
\pgfsetlinewidth{0.000000pt}%
\definecolor{currentstroke}{rgb}{0.000000,0.000000,0.000000}%
\pgfsetstrokecolor{currentstroke}%
\pgfsetdash{}{0pt}%
\pgfpathmoveto{\pgfqpoint{2.019541in}{0.744081in}}%
\pgfpathlineto{\pgfqpoint{2.055764in}{0.763755in}}%
\pgfpathlineto{\pgfqpoint{2.020405in}{0.838243in}}%
\pgfpathlineto{\pgfqpoint{1.983987in}{0.818633in}}%
\pgfpathclose%
\pgfusepath{fill}%
\end{pgfscope}%
\begin{pgfscope}%
\pgfpathrectangle{\pgfqpoint{0.150000in}{0.150000in}}{\pgfqpoint{2.700000in}{1.950000in}}%
\pgfusepath{clip}%
\pgfsetbuttcap%
\pgfsetroundjoin%
\definecolor{currentfill}{rgb}{0.887929,0.796645,0.803876}%
\pgfsetfillcolor{currentfill}%
\pgfsetlinewidth{0.000000pt}%
\definecolor{currentstroke}{rgb}{0.000000,0.000000,0.000000}%
\pgfsetstrokecolor{currentstroke}%
\pgfsetdash{}{0pt}%
\pgfpathmoveto{\pgfqpoint{1.982389in}{0.630375in}}%
\pgfpathlineto{\pgfqpoint{2.018680in}{0.650254in}}%
\pgfpathlineto{\pgfqpoint{1.983187in}{0.724336in}}%
\pgfpathlineto{\pgfqpoint{1.946702in}{0.704520in}}%
\pgfpathclose%
\pgfusepath{fill}%
\end{pgfscope}%
\begin{pgfscope}%
\pgfpathrectangle{\pgfqpoint{0.150000in}{0.150000in}}{\pgfqpoint{2.700000in}{1.950000in}}%
\pgfusepath{clip}%
\pgfsetbuttcap%
\pgfsetroundjoin%
\definecolor{currentfill}{rgb}{0.846140,0.720818,0.730744}%
\pgfsetfillcolor{currentfill}%
\pgfsetlinewidth{0.000000pt}%
\definecolor{currentstroke}{rgb}{0.000000,0.000000,0.000000}%
\pgfsetstrokecolor{currentstroke}%
\pgfsetdash{}{0pt}%
\pgfpathmoveto{\pgfqpoint{1.945237in}{0.516666in}}%
\pgfpathlineto{\pgfqpoint{1.981595in}{0.536749in}}%
\pgfpathlineto{\pgfqpoint{1.945968in}{0.610425in}}%
\pgfpathlineto{\pgfqpoint{1.909415in}{0.590402in}}%
\pgfpathclose%
\pgfusepath{fill}%
\end{pgfscope}%
\begin{pgfscope}%
\pgfpathrectangle{\pgfqpoint{0.150000in}{0.150000in}}{\pgfqpoint{2.700000in}{1.950000in}}%
\pgfusepath{clip}%
\pgfsetbuttcap%
\pgfsetroundjoin%
\definecolor{currentfill}{rgb}{0.524219,0.582812,0.664844}%
\pgfsetfillcolor{currentfill}%
\pgfsetlinewidth{0.000000pt}%
\definecolor{currentstroke}{rgb}{0.000000,0.000000,0.000000}%
\pgfsetstrokecolor{currentstroke}%
\pgfsetdash{}{0pt}%
\pgfpathmoveto{\pgfqpoint{1.797404in}{1.689397in}}%
\pgfpathlineto{\pgfqpoint{1.834141in}{1.707428in}}%
\pgfpathlineto{\pgfqpoint{1.796465in}{1.725395in}}%
\pgfpathlineto{\pgfqpoint{1.759727in}{1.707428in}}%
\pgfpathclose%
\pgfusepath{fill}%
\end{pgfscope}%
\begin{pgfscope}%
\pgfpathrectangle{\pgfqpoint{0.150000in}{0.150000in}}{\pgfqpoint{2.700000in}{1.950000in}}%
\pgfusepath{clip}%
\pgfsetbuttcap%
\pgfsetroundjoin%
\definecolor{currentfill}{rgb}{0.524219,0.582812,0.664844}%
\pgfsetfillcolor{currentfill}%
\pgfsetlinewidth{0.000000pt}%
\definecolor{currentstroke}{rgb}{0.000000,0.000000,0.000000}%
\pgfsetstrokecolor{currentstroke}%
\pgfsetdash{}{0pt}%
\pgfpathmoveto{\pgfqpoint{1.722856in}{1.689397in}}%
\pgfpathlineto{\pgfqpoint{1.759727in}{1.707428in}}%
\pgfpathlineto{\pgfqpoint{1.722186in}{1.725395in}}%
\pgfpathlineto{\pgfqpoint{1.685314in}{1.707428in}}%
\pgfpathclose%
\pgfusepath{fill}%
\end{pgfscope}%
\begin{pgfscope}%
\pgfpathrectangle{\pgfqpoint{0.150000in}{0.150000in}}{\pgfqpoint{2.700000in}{1.950000in}}%
\pgfusepath{clip}%
\pgfsetbuttcap%
\pgfsetroundjoin%
\definecolor{currentfill}{rgb}{0.524219,0.582812,0.664844}%
\pgfsetfillcolor{currentfill}%
\pgfsetlinewidth{0.000000pt}%
\definecolor{currentstroke}{rgb}{0.000000,0.000000,0.000000}%
\pgfsetstrokecolor{currentstroke}%
\pgfsetdash{}{0pt}%
\pgfpathmoveto{\pgfqpoint{1.648308in}{1.689397in}}%
\pgfpathlineto{\pgfqpoint{1.685314in}{1.707428in}}%
\pgfpathlineto{\pgfqpoint{1.647906in}{1.725395in}}%
\pgfpathlineto{\pgfqpoint{1.610900in}{1.707428in}}%
\pgfpathclose%
\pgfusepath{fill}%
\end{pgfscope}%
\begin{pgfscope}%
\pgfpathrectangle{\pgfqpoint{0.150000in}{0.150000in}}{\pgfqpoint{2.700000in}{1.950000in}}%
\pgfusepath{clip}%
\pgfsetbuttcap%
\pgfsetroundjoin%
\definecolor{currentfill}{rgb}{0.524219,0.582812,0.664844}%
\pgfsetfillcolor{currentfill}%
\pgfsetlinewidth{0.000000pt}%
\definecolor{currentstroke}{rgb}{0.000000,0.000000,0.000000}%
\pgfsetstrokecolor{currentstroke}%
\pgfsetdash{}{0pt}%
\pgfpathmoveto{\pgfqpoint{1.573760in}{1.689397in}}%
\pgfpathlineto{\pgfqpoint{1.610900in}{1.707428in}}%
\pgfpathlineto{\pgfqpoint{1.573626in}{1.725395in}}%
\pgfpathlineto{\pgfqpoint{1.536486in}{1.707428in}}%
\pgfpathclose%
\pgfusepath{fill}%
\end{pgfscope}%
\begin{pgfscope}%
\pgfpathrectangle{\pgfqpoint{0.150000in}{0.150000in}}{\pgfqpoint{2.700000in}{1.950000in}}%
\pgfusepath{clip}%
\pgfsetbuttcap%
\pgfsetroundjoin%
\definecolor{currentfill}{rgb}{0.524219,0.582812,0.664844}%
\pgfsetfillcolor{currentfill}%
\pgfsetlinewidth{0.000000pt}%
\definecolor{currentstroke}{rgb}{0.000000,0.000000,0.000000}%
\pgfsetstrokecolor{currentstroke}%
\pgfsetdash{}{0pt}%
\pgfpathmoveto{\pgfqpoint{1.499213in}{1.689397in}}%
\pgfpathlineto{\pgfqpoint{1.536486in}{1.707428in}}%
\pgfpathlineto{\pgfqpoint{1.499347in}{1.725395in}}%
\pgfpathlineto{\pgfqpoint{1.462073in}{1.707428in}}%
\pgfpathclose%
\pgfusepath{fill}%
\end{pgfscope}%
\begin{pgfscope}%
\pgfpathrectangle{\pgfqpoint{0.150000in}{0.150000in}}{\pgfqpoint{2.700000in}{1.950000in}}%
\pgfusepath{clip}%
\pgfsetbuttcap%
\pgfsetroundjoin%
\definecolor{currentfill}{rgb}{0.524219,0.582812,0.664844}%
\pgfsetfillcolor{currentfill}%
\pgfsetlinewidth{0.000000pt}%
\definecolor{currentstroke}{rgb}{0.000000,0.000000,0.000000}%
\pgfsetstrokecolor{currentstroke}%
\pgfsetdash{}{0pt}%
\pgfpathmoveto{\pgfqpoint{1.424665in}{1.689397in}}%
\pgfpathlineto{\pgfqpoint{1.462073in}{1.707428in}}%
\pgfpathlineto{\pgfqpoint{1.425067in}{1.725395in}}%
\pgfpathlineto{\pgfqpoint{1.387659in}{1.707428in}}%
\pgfpathclose%
\pgfusepath{fill}%
\end{pgfscope}%
\begin{pgfscope}%
\pgfpathrectangle{\pgfqpoint{0.150000in}{0.150000in}}{\pgfqpoint{2.700000in}{1.950000in}}%
\pgfusepath{clip}%
\pgfsetbuttcap%
\pgfsetroundjoin%
\definecolor{currentfill}{rgb}{0.524219,0.582812,0.664844}%
\pgfsetfillcolor{currentfill}%
\pgfsetlinewidth{0.000000pt}%
\definecolor{currentstroke}{rgb}{0.000000,0.000000,0.000000}%
\pgfsetstrokecolor{currentstroke}%
\pgfsetdash{}{0pt}%
\pgfpathmoveto{\pgfqpoint{1.350117in}{1.689397in}}%
\pgfpathlineto{\pgfqpoint{1.387659in}{1.707428in}}%
\pgfpathlineto{\pgfqpoint{1.350787in}{1.725395in}}%
\pgfpathlineto{\pgfqpoint{1.313246in}{1.707428in}}%
\pgfpathclose%
\pgfusepath{fill}%
\end{pgfscope}%
\begin{pgfscope}%
\pgfpathrectangle{\pgfqpoint{0.150000in}{0.150000in}}{\pgfqpoint{2.700000in}{1.950000in}}%
\pgfusepath{clip}%
\pgfsetbuttcap%
\pgfsetroundjoin%
\definecolor{currentfill}{rgb}{0.524219,0.582812,0.664844}%
\pgfsetfillcolor{currentfill}%
\pgfsetlinewidth{0.000000pt}%
\definecolor{currentstroke}{rgb}{0.000000,0.000000,0.000000}%
\pgfsetstrokecolor{currentstroke}%
\pgfsetdash{}{0pt}%
\pgfpathmoveto{\pgfqpoint{1.275569in}{1.689397in}}%
\pgfpathlineto{\pgfqpoint{1.313246in}{1.707428in}}%
\pgfpathlineto{\pgfqpoint{1.276508in}{1.725395in}}%
\pgfpathlineto{\pgfqpoint{1.238832in}{1.707428in}}%
\pgfpathclose%
\pgfusepath{fill}%
\end{pgfscope}%
\begin{pgfscope}%
\pgfpathrectangle{\pgfqpoint{0.150000in}{0.150000in}}{\pgfqpoint{2.700000in}{1.950000in}}%
\pgfusepath{clip}%
\pgfsetbuttcap%
\pgfsetroundjoin%
\definecolor{currentfill}{rgb}{0.959574,0.964553,0.971523}%
\pgfsetfillcolor{currentfill}%
\pgfsetlinewidth{0.000000pt}%
\definecolor{currentstroke}{rgb}{0.000000,0.000000,0.000000}%
\pgfsetstrokecolor{currentstroke}%
\pgfsetdash{}{0pt}%
\pgfpathmoveto{\pgfqpoint{2.093840in}{0.971481in}}%
\pgfpathlineto{\pgfqpoint{2.129928in}{0.990746in}}%
\pgfpathlineto{\pgfqpoint{2.095440in}{1.077342in}}%
\pgfpathlineto{\pgfqpoint{2.059120in}{1.058143in}}%
\pgfpathclose%
\pgfusepath{fill}%
\end{pgfscope}%
\begin{pgfscope}%
\pgfpathrectangle{\pgfqpoint{0.150000in}{0.150000in}}{\pgfqpoint{2.700000in}{1.950000in}}%
\pgfusepath{clip}%
\pgfsetbuttcap%
\pgfsetroundjoin%
\definecolor{currentfill}{rgb}{0.804090,0.828217,0.861994}%
\pgfsetfillcolor{currentfill}%
\pgfsetlinewidth{0.000000pt}%
\definecolor{currentstroke}{rgb}{0.000000,0.000000,0.000000}%
\pgfsetstrokecolor{currentstroke}%
\pgfsetdash{}{0pt}%
\pgfpathmoveto{\pgfqpoint{2.168816in}{1.210285in}}%
\pgfpathlineto{\pgfqpoint{2.204807in}{1.229140in}}%
\pgfpathlineto{\pgfqpoint{2.169948in}{1.305378in}}%
\pgfpathlineto{\pgfqpoint{2.133764in}{1.286593in}}%
\pgfpathclose%
\pgfusepath{fill}%
\end{pgfscope}%
\begin{pgfscope}%
\pgfpathrectangle{\pgfqpoint{0.150000in}{0.150000in}}{\pgfqpoint{2.700000in}{1.950000in}}%
\pgfusepath{clip}%
\pgfsetbuttcap%
\pgfsetroundjoin%
\definecolor{currentfill}{rgb}{0.878722,0.893658,0.914568}%
\pgfsetfillcolor{currentfill}%
\pgfsetlinewidth{0.000000pt}%
\definecolor{currentstroke}{rgb}{0.000000,0.000000,0.000000}%
\pgfsetstrokecolor{currentstroke}%
\pgfsetdash{}{0pt}%
\pgfpathmoveto{\pgfqpoint{2.131629in}{1.096471in}}%
\pgfpathlineto{\pgfqpoint{2.167688in}{1.115531in}}%
\pgfpathlineto{\pgfqpoint{2.132694in}{1.191362in}}%
\pgfpathlineto{\pgfqpoint{2.096442in}{1.172370in}}%
\pgfpathclose%
\pgfusepath{fill}%
\end{pgfscope}%
\begin{pgfscope}%
\pgfpathrectangle{\pgfqpoint{0.150000in}{0.150000in}}{\pgfqpoint{2.700000in}{1.950000in}}%
\pgfusepath{clip}%
\pgfsetbuttcap%
\pgfsetroundjoin%
\definecolor{currentfill}{rgb}{0.979105,0.962086,0.963434}%
\pgfsetfillcolor{currentfill}%
\pgfsetlinewidth{0.000000pt}%
\definecolor{currentstroke}{rgb}{0.000000,0.000000,0.000000}%
\pgfsetstrokecolor{currentstroke}%
\pgfsetdash{}{0pt}%
\pgfpathmoveto{\pgfqpoint{2.020405in}{0.838243in}}%
\pgfpathlineto{\pgfqpoint{2.056691in}{0.857783in}}%
\pgfpathlineto{\pgfqpoint{2.021272in}{0.932743in}}%
\pgfpathlineto{\pgfqpoint{1.984790in}{0.913269in}}%
\pgfpathclose%
\pgfusepath{fill}%
\end{pgfscope}%
\begin{pgfscope}%
\pgfpathrectangle{\pgfqpoint{0.150000in}{0.150000in}}{\pgfqpoint{2.700000in}{1.950000in}}%
\pgfusepath{clip}%
\pgfsetbuttcap%
\pgfsetroundjoin%
\definecolor{currentfill}{rgb}{0.933517,0.879366,0.883655}%
\pgfsetfillcolor{currentfill}%
\pgfsetlinewidth{0.000000pt}%
\definecolor{currentstroke}{rgb}{0.000000,0.000000,0.000000}%
\pgfsetstrokecolor{currentstroke}%
\pgfsetdash{}{0pt}%
\pgfpathmoveto{\pgfqpoint{1.983187in}{0.724336in}}%
\pgfpathlineto{\pgfqpoint{2.019541in}{0.744081in}}%
\pgfpathlineto{\pgfqpoint{1.983987in}{0.818633in}}%
\pgfpathlineto{\pgfqpoint{1.947438in}{0.798952in}}%
\pgfpathclose%
\pgfusepath{fill}%
\end{pgfscope}%
\begin{pgfscope}%
\pgfpathrectangle{\pgfqpoint{0.150000in}{0.150000in}}{\pgfqpoint{2.700000in}{1.950000in}}%
\pgfusepath{clip}%
\pgfsetbuttcap%
\pgfsetroundjoin%
\definecolor{currentfill}{rgb}{0.887929,0.796645,0.803876}%
\pgfsetfillcolor{currentfill}%
\pgfsetlinewidth{0.000000pt}%
\definecolor{currentstroke}{rgb}{0.000000,0.000000,0.000000}%
\pgfsetstrokecolor{currentstroke}%
\pgfsetdash{}{0pt}%
\pgfpathmoveto{\pgfqpoint{1.945968in}{0.610425in}}%
\pgfpathlineto{\pgfqpoint{1.982389in}{0.630375in}}%
\pgfpathlineto{\pgfqpoint{1.946702in}{0.704520in}}%
\pgfpathlineto{\pgfqpoint{1.910084in}{0.684632in}}%
\pgfpathclose%
\pgfusepath{fill}%
\end{pgfscope}%
\begin{pgfscope}%
\pgfpathrectangle{\pgfqpoint{0.150000in}{0.150000in}}{\pgfqpoint{2.700000in}{1.950000in}}%
\pgfusepath{clip}%
\pgfsetbuttcap%
\pgfsetroundjoin%
\definecolor{currentfill}{rgb}{0.846140,0.720818,0.730744}%
\pgfsetfillcolor{currentfill}%
\pgfsetlinewidth{0.000000pt}%
\definecolor{currentstroke}{rgb}{0.000000,0.000000,0.000000}%
\pgfsetstrokecolor{currentstroke}%
\pgfsetdash{}{0pt}%
\pgfpathmoveto{\pgfqpoint{1.908748in}{0.496511in}}%
\pgfpathlineto{\pgfqpoint{1.945237in}{0.516666in}}%
\pgfpathlineto{\pgfqpoint{1.909415in}{0.590402in}}%
\pgfpathlineto{\pgfqpoint{1.872730in}{0.570308in}}%
\pgfpathclose%
\pgfusepath{fill}%
\end{pgfscope}%
\begin{pgfscope}%
\pgfpathrectangle{\pgfqpoint{0.150000in}{0.150000in}}{\pgfqpoint{2.700000in}{1.950000in}}%
\pgfusepath{clip}%
\pgfsetbuttcap%
\pgfsetroundjoin%
\definecolor{currentfill}{rgb}{0.524219,0.582812,0.664844}%
\pgfsetfillcolor{currentfill}%
\pgfsetlinewidth{0.000000pt}%
\definecolor{currentstroke}{rgb}{0.000000,0.000000,0.000000}%
\pgfsetstrokecolor{currentstroke}%
\pgfsetdash{}{0pt}%
\pgfpathmoveto{\pgfqpoint{1.835218in}{1.671301in}}%
\pgfpathlineto{\pgfqpoint{1.871952in}{1.689397in}}%
\pgfpathlineto{\pgfqpoint{1.834141in}{1.707428in}}%
\pgfpathlineto{\pgfqpoint{1.797404in}{1.689397in}}%
\pgfpathclose%
\pgfusepath{fill}%
\end{pgfscope}%
\begin{pgfscope}%
\pgfpathrectangle{\pgfqpoint{0.150000in}{0.150000in}}{\pgfqpoint{2.700000in}{1.950000in}}%
\pgfusepath{clip}%
\pgfsetbuttcap%
\pgfsetroundjoin%
\definecolor{currentfill}{rgb}{0.524219,0.582812,0.664844}%
\pgfsetfillcolor{currentfill}%
\pgfsetlinewidth{0.000000pt}%
\definecolor{currentstroke}{rgb}{0.000000,0.000000,0.000000}%
\pgfsetstrokecolor{currentstroke}%
\pgfsetdash{}{0pt}%
\pgfpathmoveto{\pgfqpoint{1.760535in}{1.671301in}}%
\pgfpathlineto{\pgfqpoint{1.797404in}{1.689397in}}%
\pgfpathlineto{\pgfqpoint{1.759727in}{1.707428in}}%
\pgfpathlineto{\pgfqpoint{1.722856in}{1.689397in}}%
\pgfpathclose%
\pgfusepath{fill}%
\end{pgfscope}%
\begin{pgfscope}%
\pgfpathrectangle{\pgfqpoint{0.150000in}{0.150000in}}{\pgfqpoint{2.700000in}{1.950000in}}%
\pgfusepath{clip}%
\pgfsetbuttcap%
\pgfsetroundjoin%
\definecolor{currentfill}{rgb}{0.524219,0.582812,0.664844}%
\pgfsetfillcolor{currentfill}%
\pgfsetlinewidth{0.000000pt}%
\definecolor{currentstroke}{rgb}{0.000000,0.000000,0.000000}%
\pgfsetstrokecolor{currentstroke}%
\pgfsetdash{}{0pt}%
\pgfpathmoveto{\pgfqpoint{1.685852in}{1.671301in}}%
\pgfpathlineto{\pgfqpoint{1.722856in}{1.689397in}}%
\pgfpathlineto{\pgfqpoint{1.685314in}{1.707428in}}%
\pgfpathlineto{\pgfqpoint{1.648308in}{1.689397in}}%
\pgfpathclose%
\pgfusepath{fill}%
\end{pgfscope}%
\begin{pgfscope}%
\pgfpathrectangle{\pgfqpoint{0.150000in}{0.150000in}}{\pgfqpoint{2.700000in}{1.950000in}}%
\pgfusepath{clip}%
\pgfsetbuttcap%
\pgfsetroundjoin%
\definecolor{currentfill}{rgb}{0.524219,0.582812,0.664844}%
\pgfsetfillcolor{currentfill}%
\pgfsetlinewidth{0.000000pt}%
\definecolor{currentstroke}{rgb}{0.000000,0.000000,0.000000}%
\pgfsetstrokecolor{currentstroke}%
\pgfsetdash{}{0pt}%
\pgfpathmoveto{\pgfqpoint{1.611169in}{1.671301in}}%
\pgfpathlineto{\pgfqpoint{1.648308in}{1.689397in}}%
\pgfpathlineto{\pgfqpoint{1.610900in}{1.707428in}}%
\pgfpathlineto{\pgfqpoint{1.573760in}{1.689397in}}%
\pgfpathclose%
\pgfusepath{fill}%
\end{pgfscope}%
\begin{pgfscope}%
\pgfpathrectangle{\pgfqpoint{0.150000in}{0.150000in}}{\pgfqpoint{2.700000in}{1.950000in}}%
\pgfusepath{clip}%
\pgfsetbuttcap%
\pgfsetroundjoin%
\definecolor{currentfill}{rgb}{0.524219,0.582812,0.664844}%
\pgfsetfillcolor{currentfill}%
\pgfsetlinewidth{0.000000pt}%
\definecolor{currentstroke}{rgb}{0.000000,0.000000,0.000000}%
\pgfsetstrokecolor{currentstroke}%
\pgfsetdash{}{0pt}%
\pgfpathmoveto{\pgfqpoint{1.536486in}{1.671301in}}%
\pgfpathlineto{\pgfqpoint{1.573760in}{1.689397in}}%
\pgfpathlineto{\pgfqpoint{1.536486in}{1.707428in}}%
\pgfpathlineto{\pgfqpoint{1.499213in}{1.689397in}}%
\pgfpathclose%
\pgfusepath{fill}%
\end{pgfscope}%
\begin{pgfscope}%
\pgfpathrectangle{\pgfqpoint{0.150000in}{0.150000in}}{\pgfqpoint{2.700000in}{1.950000in}}%
\pgfusepath{clip}%
\pgfsetbuttcap%
\pgfsetroundjoin%
\definecolor{currentfill}{rgb}{0.524219,0.582812,0.664844}%
\pgfsetfillcolor{currentfill}%
\pgfsetlinewidth{0.000000pt}%
\definecolor{currentstroke}{rgb}{0.000000,0.000000,0.000000}%
\pgfsetstrokecolor{currentstroke}%
\pgfsetdash{}{0pt}%
\pgfpathmoveto{\pgfqpoint{1.461804in}{1.671301in}}%
\pgfpathlineto{\pgfqpoint{1.499213in}{1.689397in}}%
\pgfpathlineto{\pgfqpoint{1.462073in}{1.707428in}}%
\pgfpathlineto{\pgfqpoint{1.424665in}{1.689397in}}%
\pgfpathclose%
\pgfusepath{fill}%
\end{pgfscope}%
\begin{pgfscope}%
\pgfpathrectangle{\pgfqpoint{0.150000in}{0.150000in}}{\pgfqpoint{2.700000in}{1.950000in}}%
\pgfusepath{clip}%
\pgfsetbuttcap%
\pgfsetroundjoin%
\definecolor{currentfill}{rgb}{0.524219,0.582812,0.664844}%
\pgfsetfillcolor{currentfill}%
\pgfsetlinewidth{0.000000pt}%
\definecolor{currentstroke}{rgb}{0.000000,0.000000,0.000000}%
\pgfsetstrokecolor{currentstroke}%
\pgfsetdash{}{0pt}%
\pgfpathmoveto{\pgfqpoint{1.387121in}{1.671301in}}%
\pgfpathlineto{\pgfqpoint{1.424665in}{1.689397in}}%
\pgfpathlineto{\pgfqpoint{1.387659in}{1.707428in}}%
\pgfpathlineto{\pgfqpoint{1.350117in}{1.689397in}}%
\pgfpathclose%
\pgfusepath{fill}%
\end{pgfscope}%
\begin{pgfscope}%
\pgfpathrectangle{\pgfqpoint{0.150000in}{0.150000in}}{\pgfqpoint{2.700000in}{1.950000in}}%
\pgfusepath{clip}%
\pgfsetbuttcap%
\pgfsetroundjoin%
\definecolor{currentfill}{rgb}{0.524219,0.582812,0.664844}%
\pgfsetfillcolor{currentfill}%
\pgfsetlinewidth{0.000000pt}%
\definecolor{currentstroke}{rgb}{0.000000,0.000000,0.000000}%
\pgfsetstrokecolor{currentstroke}%
\pgfsetdash{}{0pt}%
\pgfpathmoveto{\pgfqpoint{1.312438in}{1.671301in}}%
\pgfpathlineto{\pgfqpoint{1.350117in}{1.689397in}}%
\pgfpathlineto{\pgfqpoint{1.313246in}{1.707428in}}%
\pgfpathlineto{\pgfqpoint{1.275569in}{1.689397in}}%
\pgfpathclose%
\pgfusepath{fill}%
\end{pgfscope}%
\begin{pgfscope}%
\pgfpathrectangle{\pgfqpoint{0.150000in}{0.150000in}}{\pgfqpoint{2.700000in}{1.950000in}}%
\pgfusepath{clip}%
\pgfsetbuttcap%
\pgfsetroundjoin%
\definecolor{currentfill}{rgb}{0.524219,0.582812,0.664844}%
\pgfsetfillcolor{currentfill}%
\pgfsetlinewidth{0.000000pt}%
\definecolor{currentstroke}{rgb}{0.000000,0.000000,0.000000}%
\pgfsetstrokecolor{currentstroke}%
\pgfsetdash{}{0pt}%
\pgfpathmoveto{\pgfqpoint{1.237755in}{1.671301in}}%
\pgfpathlineto{\pgfqpoint{1.275569in}{1.689397in}}%
\pgfpathlineto{\pgfqpoint{1.238832in}{1.707428in}}%
\pgfpathlineto{\pgfqpoint{1.201021in}{1.689397in}}%
\pgfpathclose%
\pgfusepath{fill}%
\end{pgfscope}%
\begin{pgfscope}%
\pgfpathrectangle{\pgfqpoint{0.150000in}{0.150000in}}{\pgfqpoint{2.700000in}{1.950000in}}%
\pgfusepath{clip}%
\pgfsetbuttcap%
\pgfsetroundjoin%
\definecolor{currentfill}{rgb}{0.959574,0.964553,0.971523}%
\pgfsetfillcolor{currentfill}%
\pgfsetlinewidth{0.000000pt}%
\definecolor{currentstroke}{rgb}{0.000000,0.000000,0.000000}%
\pgfsetstrokecolor{currentstroke}%
\pgfsetdash{}{0pt}%
\pgfpathmoveto{\pgfqpoint{2.057621in}{0.952147in}}%
\pgfpathlineto{\pgfqpoint{2.093840in}{0.971481in}}%
\pgfpathlineto{\pgfqpoint{2.059120in}{1.058143in}}%
\pgfpathlineto{\pgfqpoint{2.022668in}{1.038876in}}%
\pgfpathclose%
\pgfusepath{fill}%
\end{pgfscope}%
\begin{pgfscope}%
\pgfpathrectangle{\pgfqpoint{0.150000in}{0.150000in}}{\pgfqpoint{2.700000in}{1.950000in}}%
\pgfusepath{clip}%
\pgfsetbuttcap%
\pgfsetroundjoin%
\definecolor{currentfill}{rgb}{0.729458,0.762776,0.809421}%
\pgfsetfillcolor{currentfill}%
\pgfsetlinewidth{0.000000pt}%
\definecolor{currentstroke}{rgb}{0.000000,0.000000,0.000000}%
\pgfsetstrokecolor{currentstroke}%
\pgfsetdash{}{0pt}%
\pgfpathmoveto{\pgfqpoint{2.169948in}{1.305378in}}%
\pgfpathlineto{\pgfqpoint{2.206001in}{1.324096in}}%
\pgfpathlineto{\pgfqpoint{2.171084in}{1.400813in}}%
\pgfpathlineto{\pgfqpoint{2.134837in}{1.382166in}}%
\pgfpathclose%
\pgfusepath{fill}%
\end{pgfscope}%
\begin{pgfscope}%
\pgfpathrectangle{\pgfqpoint{0.150000in}{0.150000in}}{\pgfqpoint{2.700000in}{1.950000in}}%
\pgfusepath{clip}%
\pgfsetbuttcap%
\pgfsetroundjoin%
\definecolor{currentfill}{rgb}{0.804090,0.828217,0.861994}%
\pgfsetfillcolor{currentfill}%
\pgfsetlinewidth{0.000000pt}%
\definecolor{currentstroke}{rgb}{0.000000,0.000000,0.000000}%
\pgfsetstrokecolor{currentstroke}%
\pgfsetdash{}{0pt}%
\pgfpathmoveto{\pgfqpoint{2.132694in}{1.191362in}}%
\pgfpathlineto{\pgfqpoint{2.168816in}{1.210285in}}%
\pgfpathlineto{\pgfqpoint{2.133764in}{1.286593in}}%
\pgfpathlineto{\pgfqpoint{2.097448in}{1.267740in}}%
\pgfpathclose%
\pgfusepath{fill}%
\end{pgfscope}%
\begin{pgfscope}%
\pgfpathrectangle{\pgfqpoint{0.150000in}{0.150000in}}{\pgfqpoint{2.700000in}{1.950000in}}%
\pgfusepath{clip}%
\pgfsetbuttcap%
\pgfsetroundjoin%
\definecolor{currentfill}{rgb}{0.878722,0.893658,0.914568}%
\pgfsetfillcolor{currentfill}%
\pgfsetlinewidth{0.000000pt}%
\definecolor{currentstroke}{rgb}{0.000000,0.000000,0.000000}%
\pgfsetstrokecolor{currentstroke}%
\pgfsetdash{}{0pt}%
\pgfpathmoveto{\pgfqpoint{2.095440in}{1.077342in}}%
\pgfpathlineto{\pgfqpoint{2.131629in}{1.096471in}}%
\pgfpathlineto{\pgfqpoint{2.096442in}{1.172370in}}%
\pgfpathlineto{\pgfqpoint{2.060059in}{1.153309in}}%
\pgfpathclose%
\pgfusepath{fill}%
\end{pgfscope}%
\begin{pgfscope}%
\pgfpathrectangle{\pgfqpoint{0.150000in}{0.150000in}}{\pgfqpoint{2.700000in}{1.950000in}}%
\pgfusepath{clip}%
\pgfsetbuttcap%
\pgfsetroundjoin%
\definecolor{currentfill}{rgb}{0.979105,0.962086,0.963434}%
\pgfsetfillcolor{currentfill}%
\pgfsetlinewidth{0.000000pt}%
\definecolor{currentstroke}{rgb}{0.000000,0.000000,0.000000}%
\pgfsetstrokecolor{currentstroke}%
\pgfsetdash{}{0pt}%
\pgfpathmoveto{\pgfqpoint{1.983987in}{0.818633in}}%
\pgfpathlineto{\pgfqpoint{2.020405in}{0.838243in}}%
\pgfpathlineto{\pgfqpoint{1.984790in}{0.913269in}}%
\pgfpathlineto{\pgfqpoint{1.948177in}{0.893724in}}%
\pgfpathclose%
\pgfusepath{fill}%
\end{pgfscope}%
\begin{pgfscope}%
\pgfpathrectangle{\pgfqpoint{0.150000in}{0.150000in}}{\pgfqpoint{2.700000in}{1.950000in}}%
\pgfusepath{clip}%
\pgfsetbuttcap%
\pgfsetroundjoin%
\definecolor{currentfill}{rgb}{0.933517,0.879366,0.883655}%
\pgfsetfillcolor{currentfill}%
\pgfsetlinewidth{0.000000pt}%
\definecolor{currentstroke}{rgb}{0.000000,0.000000,0.000000}%
\pgfsetstrokecolor{currentstroke}%
\pgfsetdash{}{0pt}%
\pgfpathmoveto{\pgfqpoint{1.946702in}{0.704520in}}%
\pgfpathlineto{\pgfqpoint{1.983187in}{0.724336in}}%
\pgfpathlineto{\pgfqpoint{1.947438in}{0.798952in}}%
\pgfpathlineto{\pgfqpoint{1.910756in}{0.779199in}}%
\pgfpathclose%
\pgfusepath{fill}%
\end{pgfscope}%
\begin{pgfscope}%
\pgfpathrectangle{\pgfqpoint{0.150000in}{0.150000in}}{\pgfqpoint{2.700000in}{1.950000in}}%
\pgfusepath{clip}%
\pgfsetbuttcap%
\pgfsetroundjoin%
\definecolor{currentfill}{rgb}{0.887929,0.796645,0.803876}%
\pgfsetfillcolor{currentfill}%
\pgfsetlinewidth{0.000000pt}%
\definecolor{currentstroke}{rgb}{0.000000,0.000000,0.000000}%
\pgfsetstrokecolor{currentstroke}%
\pgfsetdash{}{0pt}%
\pgfpathmoveto{\pgfqpoint{1.909415in}{0.590402in}}%
\pgfpathlineto{\pgfqpoint{1.945968in}{0.610425in}}%
\pgfpathlineto{\pgfqpoint{1.910084in}{0.684632in}}%
\pgfpathlineto{\pgfqpoint{1.873334in}{0.664671in}}%
\pgfpathclose%
\pgfusepath{fill}%
\end{pgfscope}%
\begin{pgfscope}%
\pgfpathrectangle{\pgfqpoint{0.150000in}{0.150000in}}{\pgfqpoint{2.700000in}{1.950000in}}%
\pgfusepath{clip}%
\pgfsetbuttcap%
\pgfsetroundjoin%
\definecolor{currentfill}{rgb}{0.846140,0.720818,0.730744}%
\pgfsetfillcolor{currentfill}%
\pgfsetlinewidth{0.000000pt}%
\definecolor{currentstroke}{rgb}{0.000000,0.000000,0.000000}%
\pgfsetstrokecolor{currentstroke}%
\pgfsetdash{}{0pt}%
\pgfpathmoveto{\pgfqpoint{1.872127in}{0.476282in}}%
\pgfpathlineto{\pgfqpoint{1.908748in}{0.496511in}}%
\pgfpathlineto{\pgfqpoint{1.872730in}{0.570308in}}%
\pgfpathlineto{\pgfqpoint{1.835911in}{0.550140in}}%
\pgfpathclose%
\pgfusepath{fill}%
\end{pgfscope}%
\begin{pgfscope}%
\pgfpathrectangle{\pgfqpoint{0.150000in}{0.150000in}}{\pgfqpoint{2.700000in}{1.950000in}}%
\pgfusepath{clip}%
\pgfsetbuttcap%
\pgfsetroundjoin%
\definecolor{currentfill}{rgb}{0.524219,0.582812,0.664844}%
\pgfsetfillcolor{currentfill}%
\pgfsetlinewidth{0.000000pt}%
\definecolor{currentstroke}{rgb}{0.000000,0.000000,0.000000}%
\pgfsetstrokecolor{currentstroke}%
\pgfsetdash{}{0pt}%
\pgfpathmoveto{\pgfqpoint{1.873168in}{1.653139in}}%
\pgfpathlineto{\pgfqpoint{1.909900in}{1.671301in}}%
\pgfpathlineto{\pgfqpoint{1.871952in}{1.689397in}}%
\pgfpathlineto{\pgfqpoint{1.835218in}{1.671301in}}%
\pgfpathclose%
\pgfusepath{fill}%
\end{pgfscope}%
\begin{pgfscope}%
\pgfpathrectangle{\pgfqpoint{0.150000in}{0.150000in}}{\pgfqpoint{2.700000in}{1.950000in}}%
\pgfusepath{clip}%
\pgfsetbuttcap%
\pgfsetroundjoin%
\definecolor{currentfill}{rgb}{0.524219,0.582812,0.664844}%
\pgfsetfillcolor{currentfill}%
\pgfsetlinewidth{0.000000pt}%
\definecolor{currentstroke}{rgb}{0.000000,0.000000,0.000000}%
\pgfsetstrokecolor{currentstroke}%
\pgfsetdash{}{0pt}%
\pgfpathmoveto{\pgfqpoint{1.798350in}{1.653139in}}%
\pgfpathlineto{\pgfqpoint{1.835218in}{1.671301in}}%
\pgfpathlineto{\pgfqpoint{1.797404in}{1.689397in}}%
\pgfpathlineto{\pgfqpoint{1.760535in}{1.671301in}}%
\pgfpathclose%
\pgfusepath{fill}%
\end{pgfscope}%
\begin{pgfscope}%
\pgfpathrectangle{\pgfqpoint{0.150000in}{0.150000in}}{\pgfqpoint{2.700000in}{1.950000in}}%
\pgfusepath{clip}%
\pgfsetbuttcap%
\pgfsetroundjoin%
\definecolor{currentfill}{rgb}{0.524219,0.582812,0.664844}%
\pgfsetfillcolor{currentfill}%
\pgfsetlinewidth{0.000000pt}%
\definecolor{currentstroke}{rgb}{0.000000,0.000000,0.000000}%
\pgfsetstrokecolor{currentstroke}%
\pgfsetdash{}{0pt}%
\pgfpathmoveto{\pgfqpoint{1.723532in}{1.653139in}}%
\pgfpathlineto{\pgfqpoint{1.760535in}{1.671301in}}%
\pgfpathlineto{\pgfqpoint{1.722856in}{1.689397in}}%
\pgfpathlineto{\pgfqpoint{1.685852in}{1.671301in}}%
\pgfpathclose%
\pgfusepath{fill}%
\end{pgfscope}%
\begin{pgfscope}%
\pgfpathrectangle{\pgfqpoint{0.150000in}{0.150000in}}{\pgfqpoint{2.700000in}{1.950000in}}%
\pgfusepath{clip}%
\pgfsetbuttcap%
\pgfsetroundjoin%
\definecolor{currentfill}{rgb}{0.524219,0.582812,0.664844}%
\pgfsetfillcolor{currentfill}%
\pgfsetlinewidth{0.000000pt}%
\definecolor{currentstroke}{rgb}{0.000000,0.000000,0.000000}%
\pgfsetstrokecolor{currentstroke}%
\pgfsetdash{}{0pt}%
\pgfpathmoveto{\pgfqpoint{1.648714in}{1.653139in}}%
\pgfpathlineto{\pgfqpoint{1.685852in}{1.671301in}}%
\pgfpathlineto{\pgfqpoint{1.648308in}{1.689397in}}%
\pgfpathlineto{\pgfqpoint{1.611169in}{1.671301in}}%
\pgfpathclose%
\pgfusepath{fill}%
\end{pgfscope}%
\begin{pgfscope}%
\pgfpathrectangle{\pgfqpoint{0.150000in}{0.150000in}}{\pgfqpoint{2.700000in}{1.950000in}}%
\pgfusepath{clip}%
\pgfsetbuttcap%
\pgfsetroundjoin%
\definecolor{currentfill}{rgb}{0.524219,0.582812,0.664844}%
\pgfsetfillcolor{currentfill}%
\pgfsetlinewidth{0.000000pt}%
\definecolor{currentstroke}{rgb}{0.000000,0.000000,0.000000}%
\pgfsetstrokecolor{currentstroke}%
\pgfsetdash{}{0pt}%
\pgfpathmoveto{\pgfqpoint{1.573896in}{1.653139in}}%
\pgfpathlineto{\pgfqpoint{1.611169in}{1.671301in}}%
\pgfpathlineto{\pgfqpoint{1.573760in}{1.689397in}}%
\pgfpathlineto{\pgfqpoint{1.536486in}{1.671301in}}%
\pgfpathclose%
\pgfusepath{fill}%
\end{pgfscope}%
\begin{pgfscope}%
\pgfpathrectangle{\pgfqpoint{0.150000in}{0.150000in}}{\pgfqpoint{2.700000in}{1.950000in}}%
\pgfusepath{clip}%
\pgfsetbuttcap%
\pgfsetroundjoin%
\definecolor{currentfill}{rgb}{0.524219,0.582812,0.664844}%
\pgfsetfillcolor{currentfill}%
\pgfsetlinewidth{0.000000pt}%
\definecolor{currentstroke}{rgb}{0.000000,0.000000,0.000000}%
\pgfsetstrokecolor{currentstroke}%
\pgfsetdash{}{0pt}%
\pgfpathmoveto{\pgfqpoint{1.499077in}{1.653139in}}%
\pgfpathlineto{\pgfqpoint{1.536486in}{1.671301in}}%
\pgfpathlineto{\pgfqpoint{1.499213in}{1.689397in}}%
\pgfpathlineto{\pgfqpoint{1.461804in}{1.671301in}}%
\pgfpathclose%
\pgfusepath{fill}%
\end{pgfscope}%
\begin{pgfscope}%
\pgfpathrectangle{\pgfqpoint{0.150000in}{0.150000in}}{\pgfqpoint{2.700000in}{1.950000in}}%
\pgfusepath{clip}%
\pgfsetbuttcap%
\pgfsetroundjoin%
\definecolor{currentfill}{rgb}{0.524219,0.582812,0.664844}%
\pgfsetfillcolor{currentfill}%
\pgfsetlinewidth{0.000000pt}%
\definecolor{currentstroke}{rgb}{0.000000,0.000000,0.000000}%
\pgfsetstrokecolor{currentstroke}%
\pgfsetdash{}{0pt}%
\pgfpathmoveto{\pgfqpoint{1.424259in}{1.653139in}}%
\pgfpathlineto{\pgfqpoint{1.461804in}{1.671301in}}%
\pgfpathlineto{\pgfqpoint{1.424665in}{1.689397in}}%
\pgfpathlineto{\pgfqpoint{1.387121in}{1.671301in}}%
\pgfpathclose%
\pgfusepath{fill}%
\end{pgfscope}%
\begin{pgfscope}%
\pgfpathrectangle{\pgfqpoint{0.150000in}{0.150000in}}{\pgfqpoint{2.700000in}{1.950000in}}%
\pgfusepath{clip}%
\pgfsetbuttcap%
\pgfsetroundjoin%
\definecolor{currentfill}{rgb}{0.524219,0.582812,0.664844}%
\pgfsetfillcolor{currentfill}%
\pgfsetlinewidth{0.000000pt}%
\definecolor{currentstroke}{rgb}{0.000000,0.000000,0.000000}%
\pgfsetstrokecolor{currentstroke}%
\pgfsetdash{}{0pt}%
\pgfpathmoveto{\pgfqpoint{1.349441in}{1.653139in}}%
\pgfpathlineto{\pgfqpoint{1.387121in}{1.671301in}}%
\pgfpathlineto{\pgfqpoint{1.350117in}{1.689397in}}%
\pgfpathlineto{\pgfqpoint{1.312438in}{1.671301in}}%
\pgfpathclose%
\pgfusepath{fill}%
\end{pgfscope}%
\begin{pgfscope}%
\pgfpathrectangle{\pgfqpoint{0.150000in}{0.150000in}}{\pgfqpoint{2.700000in}{1.950000in}}%
\pgfusepath{clip}%
\pgfsetbuttcap%
\pgfsetroundjoin%
\definecolor{currentfill}{rgb}{0.524219,0.582812,0.664844}%
\pgfsetfillcolor{currentfill}%
\pgfsetlinewidth{0.000000pt}%
\definecolor{currentstroke}{rgb}{0.000000,0.000000,0.000000}%
\pgfsetstrokecolor{currentstroke}%
\pgfsetdash{}{0pt}%
\pgfpathmoveto{\pgfqpoint{1.274623in}{1.653139in}}%
\pgfpathlineto{\pgfqpoint{1.312438in}{1.671301in}}%
\pgfpathlineto{\pgfqpoint{1.275569in}{1.689397in}}%
\pgfpathlineto{\pgfqpoint{1.237755in}{1.671301in}}%
\pgfpathclose%
\pgfusepath{fill}%
\end{pgfscope}%
\begin{pgfscope}%
\pgfpathrectangle{\pgfqpoint{0.150000in}{0.150000in}}{\pgfqpoint{2.700000in}{1.950000in}}%
\pgfusepath{clip}%
\pgfsetbuttcap%
\pgfsetroundjoin%
\definecolor{currentfill}{rgb}{0.524219,0.582812,0.664844}%
\pgfsetfillcolor{currentfill}%
\pgfsetlinewidth{0.000000pt}%
\definecolor{currentstroke}{rgb}{0.000000,0.000000,0.000000}%
\pgfsetstrokecolor{currentstroke}%
\pgfsetdash{}{0pt}%
\pgfpathmoveto{\pgfqpoint{1.199805in}{1.653139in}}%
\pgfpathlineto{\pgfqpoint{1.237755in}{1.671301in}}%
\pgfpathlineto{\pgfqpoint{1.201021in}{1.689397in}}%
\pgfpathlineto{\pgfqpoint{1.163073in}{1.671301in}}%
\pgfpathclose%
\pgfusepath{fill}%
\end{pgfscope}%
\begin{pgfscope}%
\pgfpathrectangle{\pgfqpoint{0.150000in}{0.150000in}}{\pgfqpoint{2.700000in}{1.950000in}}%
\pgfusepath{clip}%
\pgfsetbuttcap%
\pgfsetroundjoin%
\definecolor{currentfill}{rgb}{0.959574,0.964553,0.971523}%
\pgfsetfillcolor{currentfill}%
\pgfsetlinewidth{0.000000pt}%
\definecolor{currentstroke}{rgb}{0.000000,0.000000,0.000000}%
\pgfsetstrokecolor{currentstroke}%
\pgfsetdash{}{0pt}%
\pgfpathmoveto{\pgfqpoint{2.021272in}{0.932743in}}%
\pgfpathlineto{\pgfqpoint{2.057621in}{0.952147in}}%
\pgfpathlineto{\pgfqpoint{2.022668in}{1.038876in}}%
\pgfpathlineto{\pgfqpoint{1.986084in}{1.019538in}}%
\pgfpathclose%
\pgfusepath{fill}%
\end{pgfscope}%
\begin{pgfscope}%
\pgfpathrectangle{\pgfqpoint{0.150000in}{0.150000in}}{\pgfqpoint{2.700000in}{1.950000in}}%
\pgfusepath{clip}%
\pgfsetbuttcap%
\pgfsetroundjoin%
\definecolor{currentfill}{rgb}{0.729458,0.762776,0.809421}%
\pgfsetfillcolor{currentfill}%
\pgfsetlinewidth{0.000000pt}%
\definecolor{currentstroke}{rgb}{0.000000,0.000000,0.000000}%
\pgfsetstrokecolor{currentstroke}%
\pgfsetdash{}{0pt}%
\pgfpathmoveto{\pgfqpoint{2.133764in}{1.286593in}}%
\pgfpathlineto{\pgfqpoint{2.169948in}{1.305378in}}%
\pgfpathlineto{\pgfqpoint{2.134837in}{1.382166in}}%
\pgfpathlineto{\pgfqpoint{2.098458in}{1.363453in}}%
\pgfpathclose%
\pgfusepath{fill}%
\end{pgfscope}%
\begin{pgfscope}%
\pgfpathrectangle{\pgfqpoint{0.150000in}{0.150000in}}{\pgfqpoint{2.700000in}{1.950000in}}%
\pgfusepath{clip}%
\pgfsetbuttcap%
\pgfsetroundjoin%
\definecolor{currentfill}{rgb}{0.804090,0.828217,0.861994}%
\pgfsetfillcolor{currentfill}%
\pgfsetlinewidth{0.000000pt}%
\definecolor{currentstroke}{rgb}{0.000000,0.000000,0.000000}%
\pgfsetstrokecolor{currentstroke}%
\pgfsetdash{}{0pt}%
\pgfpathmoveto{\pgfqpoint{2.096442in}{1.172370in}}%
\pgfpathlineto{\pgfqpoint{2.132694in}{1.191362in}}%
\pgfpathlineto{\pgfqpoint{2.097448in}{1.267740in}}%
\pgfpathlineto{\pgfqpoint{2.061001in}{1.248818in}}%
\pgfpathclose%
\pgfusepath{fill}%
\end{pgfscope}%
\begin{pgfscope}%
\pgfpathrectangle{\pgfqpoint{0.150000in}{0.150000in}}{\pgfqpoint{2.700000in}{1.950000in}}%
\pgfusepath{clip}%
\pgfsetbuttcap%
\pgfsetroundjoin%
\definecolor{currentfill}{rgb}{0.878722,0.893658,0.914568}%
\pgfsetfillcolor{currentfill}%
\pgfsetlinewidth{0.000000pt}%
\definecolor{currentstroke}{rgb}{0.000000,0.000000,0.000000}%
\pgfsetstrokecolor{currentstroke}%
\pgfsetdash{}{0pt}%
\pgfpathmoveto{\pgfqpoint{2.059120in}{1.058143in}}%
\pgfpathlineto{\pgfqpoint{2.095440in}{1.077342in}}%
\pgfpathlineto{\pgfqpoint{2.060059in}{1.153309in}}%
\pgfpathlineto{\pgfqpoint{2.023543in}{1.134179in}}%
\pgfpathclose%
\pgfusepath{fill}%
\end{pgfscope}%
\begin{pgfscope}%
\pgfpathrectangle{\pgfqpoint{0.150000in}{0.150000in}}{\pgfqpoint{2.700000in}{1.950000in}}%
\pgfusepath{clip}%
\pgfsetbuttcap%
\pgfsetroundjoin%
\definecolor{currentfill}{rgb}{0.979105,0.962086,0.963434}%
\pgfsetfillcolor{currentfill}%
\pgfsetlinewidth{0.000000pt}%
\definecolor{currentstroke}{rgb}{0.000000,0.000000,0.000000}%
\pgfsetstrokecolor{currentstroke}%
\pgfsetdash{}{0pt}%
\pgfpathmoveto{\pgfqpoint{1.947438in}{0.798952in}}%
\pgfpathlineto{\pgfqpoint{1.983987in}{0.818633in}}%
\pgfpathlineto{\pgfqpoint{1.948177in}{0.893724in}}%
\pgfpathlineto{\pgfqpoint{1.911430in}{0.874108in}}%
\pgfpathclose%
\pgfusepath{fill}%
\end{pgfscope}%
\begin{pgfscope}%
\pgfpathrectangle{\pgfqpoint{0.150000in}{0.150000in}}{\pgfqpoint{2.700000in}{1.950000in}}%
\pgfusepath{clip}%
\pgfsetbuttcap%
\pgfsetroundjoin%
\definecolor{currentfill}{rgb}{0.933517,0.879366,0.883655}%
\pgfsetfillcolor{currentfill}%
\pgfsetlinewidth{0.000000pt}%
\definecolor{currentstroke}{rgb}{0.000000,0.000000,0.000000}%
\pgfsetstrokecolor{currentstroke}%
\pgfsetdash{}{0pt}%
\pgfpathmoveto{\pgfqpoint{1.910084in}{0.684632in}}%
\pgfpathlineto{\pgfqpoint{1.946702in}{0.704520in}}%
\pgfpathlineto{\pgfqpoint{1.910756in}{0.779199in}}%
\pgfpathlineto{\pgfqpoint{1.873941in}{0.759375in}}%
\pgfpathclose%
\pgfusepath{fill}%
\end{pgfscope}%
\begin{pgfscope}%
\pgfpathrectangle{\pgfqpoint{0.150000in}{0.150000in}}{\pgfqpoint{2.700000in}{1.950000in}}%
\pgfusepath{clip}%
\pgfsetbuttcap%
\pgfsetroundjoin%
\definecolor{currentfill}{rgb}{0.887929,0.796645,0.803876}%
\pgfsetfillcolor{currentfill}%
\pgfsetlinewidth{0.000000pt}%
\definecolor{currentstroke}{rgb}{0.000000,0.000000,0.000000}%
\pgfsetstrokecolor{currentstroke}%
\pgfsetdash{}{0pt}%
\pgfpathmoveto{\pgfqpoint{1.872730in}{0.570308in}}%
\pgfpathlineto{\pgfqpoint{1.909415in}{0.590402in}}%
\pgfpathlineto{\pgfqpoint{1.873334in}{0.664671in}}%
\pgfpathlineto{\pgfqpoint{1.836451in}{0.644639in}}%
\pgfpathclose%
\pgfusepath{fill}%
\end{pgfscope}%
\begin{pgfscope}%
\pgfpathrectangle{\pgfqpoint{0.150000in}{0.150000in}}{\pgfqpoint{2.700000in}{1.950000in}}%
\pgfusepath{clip}%
\pgfsetbuttcap%
\pgfsetroundjoin%
\definecolor{currentfill}{rgb}{0.846140,0.720818,0.730744}%
\pgfsetfillcolor{currentfill}%
\pgfsetlinewidth{0.000000pt}%
\definecolor{currentstroke}{rgb}{0.000000,0.000000,0.000000}%
\pgfsetstrokecolor{currentstroke}%
\pgfsetdash{}{0pt}%
\pgfpathmoveto{\pgfqpoint{1.835374in}{0.455980in}}%
\pgfpathlineto{\pgfqpoint{1.872127in}{0.476282in}}%
\pgfpathlineto{\pgfqpoint{1.835911in}{0.550140in}}%
\pgfpathlineto{\pgfqpoint{1.798959in}{0.529899in}}%
\pgfpathclose%
\pgfusepath{fill}%
\end{pgfscope}%
\begin{pgfscope}%
\pgfpathrectangle{\pgfqpoint{0.150000in}{0.150000in}}{\pgfqpoint{2.700000in}{1.950000in}}%
\pgfusepath{clip}%
\pgfsetbuttcap%
\pgfsetroundjoin%
\definecolor{currentfill}{rgb}{0.524219,0.582812,0.664844}%
\pgfsetfillcolor{currentfill}%
\pgfsetlinewidth{0.000000pt}%
\definecolor{currentstroke}{rgb}{0.000000,0.000000,0.000000}%
\pgfsetstrokecolor{currentstroke}%
\pgfsetdash{}{0pt}%
\pgfpathmoveto{\pgfqpoint{1.911256in}{1.634911in}}%
\pgfpathlineto{\pgfqpoint{1.947986in}{1.653139in}}%
\pgfpathlineto{\pgfqpoint{1.909900in}{1.671301in}}%
\pgfpathlineto{\pgfqpoint{1.873168in}{1.653139in}}%
\pgfpathclose%
\pgfusepath{fill}%
\end{pgfscope}%
\begin{pgfscope}%
\pgfpathrectangle{\pgfqpoint{0.150000in}{0.150000in}}{\pgfqpoint{2.700000in}{1.950000in}}%
\pgfusepath{clip}%
\pgfsetbuttcap%
\pgfsetroundjoin%
\definecolor{currentfill}{rgb}{0.524219,0.582812,0.664844}%
\pgfsetfillcolor{currentfill}%
\pgfsetlinewidth{0.000000pt}%
\definecolor{currentstroke}{rgb}{0.000000,0.000000,0.000000}%
\pgfsetstrokecolor{currentstroke}%
\pgfsetdash{}{0pt}%
\pgfpathmoveto{\pgfqpoint{1.836302in}{1.634911in}}%
\pgfpathlineto{\pgfqpoint{1.873168in}{1.653139in}}%
\pgfpathlineto{\pgfqpoint{1.835218in}{1.671301in}}%
\pgfpathlineto{\pgfqpoint{1.798350in}{1.653139in}}%
\pgfpathclose%
\pgfusepath{fill}%
\end{pgfscope}%
\begin{pgfscope}%
\pgfpathrectangle{\pgfqpoint{0.150000in}{0.150000in}}{\pgfqpoint{2.700000in}{1.950000in}}%
\pgfusepath{clip}%
\pgfsetbuttcap%
\pgfsetroundjoin%
\definecolor{currentfill}{rgb}{0.524219,0.582812,0.664844}%
\pgfsetfillcolor{currentfill}%
\pgfsetlinewidth{0.000000pt}%
\definecolor{currentstroke}{rgb}{0.000000,0.000000,0.000000}%
\pgfsetstrokecolor{currentstroke}%
\pgfsetdash{}{0pt}%
\pgfpathmoveto{\pgfqpoint{1.761348in}{1.634911in}}%
\pgfpathlineto{\pgfqpoint{1.798350in}{1.653139in}}%
\pgfpathlineto{\pgfqpoint{1.760535in}{1.671301in}}%
\pgfpathlineto{\pgfqpoint{1.723532in}{1.653139in}}%
\pgfpathclose%
\pgfusepath{fill}%
\end{pgfscope}%
\begin{pgfscope}%
\pgfpathrectangle{\pgfqpoint{0.150000in}{0.150000in}}{\pgfqpoint{2.700000in}{1.950000in}}%
\pgfusepath{clip}%
\pgfsetbuttcap%
\pgfsetroundjoin%
\definecolor{currentfill}{rgb}{0.524219,0.582812,0.664844}%
\pgfsetfillcolor{currentfill}%
\pgfsetlinewidth{0.000000pt}%
\definecolor{currentstroke}{rgb}{0.000000,0.000000,0.000000}%
\pgfsetstrokecolor{currentstroke}%
\pgfsetdash{}{0pt}%
\pgfpathmoveto{\pgfqpoint{1.686394in}{1.634911in}}%
\pgfpathlineto{\pgfqpoint{1.723532in}{1.653139in}}%
\pgfpathlineto{\pgfqpoint{1.685852in}{1.671301in}}%
\pgfpathlineto{\pgfqpoint{1.648714in}{1.653139in}}%
\pgfpathclose%
\pgfusepath{fill}%
\end{pgfscope}%
\begin{pgfscope}%
\pgfpathrectangle{\pgfqpoint{0.150000in}{0.150000in}}{\pgfqpoint{2.700000in}{1.950000in}}%
\pgfusepath{clip}%
\pgfsetbuttcap%
\pgfsetroundjoin%
\definecolor{currentfill}{rgb}{0.524219,0.582812,0.664844}%
\pgfsetfillcolor{currentfill}%
\pgfsetlinewidth{0.000000pt}%
\definecolor{currentstroke}{rgb}{0.000000,0.000000,0.000000}%
\pgfsetstrokecolor{currentstroke}%
\pgfsetdash{}{0pt}%
\pgfpathmoveto{\pgfqpoint{1.611440in}{1.634911in}}%
\pgfpathlineto{\pgfqpoint{1.648714in}{1.653139in}}%
\pgfpathlineto{\pgfqpoint{1.611169in}{1.671301in}}%
\pgfpathlineto{\pgfqpoint{1.573896in}{1.653139in}}%
\pgfpathclose%
\pgfusepath{fill}%
\end{pgfscope}%
\begin{pgfscope}%
\pgfpathrectangle{\pgfqpoint{0.150000in}{0.150000in}}{\pgfqpoint{2.700000in}{1.950000in}}%
\pgfusepath{clip}%
\pgfsetbuttcap%
\pgfsetroundjoin%
\definecolor{currentfill}{rgb}{0.524219,0.582812,0.664844}%
\pgfsetfillcolor{currentfill}%
\pgfsetlinewidth{0.000000pt}%
\definecolor{currentstroke}{rgb}{0.000000,0.000000,0.000000}%
\pgfsetstrokecolor{currentstroke}%
\pgfsetdash{}{0pt}%
\pgfpathmoveto{\pgfqpoint{1.536486in}{1.634911in}}%
\pgfpathlineto{\pgfqpoint{1.573896in}{1.653139in}}%
\pgfpathlineto{\pgfqpoint{1.536486in}{1.671301in}}%
\pgfpathlineto{\pgfqpoint{1.499077in}{1.653139in}}%
\pgfpathclose%
\pgfusepath{fill}%
\end{pgfscope}%
\begin{pgfscope}%
\pgfpathrectangle{\pgfqpoint{0.150000in}{0.150000in}}{\pgfqpoint{2.700000in}{1.950000in}}%
\pgfusepath{clip}%
\pgfsetbuttcap%
\pgfsetroundjoin%
\definecolor{currentfill}{rgb}{0.524219,0.582812,0.664844}%
\pgfsetfillcolor{currentfill}%
\pgfsetlinewidth{0.000000pt}%
\definecolor{currentstroke}{rgb}{0.000000,0.000000,0.000000}%
\pgfsetstrokecolor{currentstroke}%
\pgfsetdash{}{0pt}%
\pgfpathmoveto{\pgfqpoint{1.461532in}{1.634911in}}%
\pgfpathlineto{\pgfqpoint{1.499077in}{1.653139in}}%
\pgfpathlineto{\pgfqpoint{1.461804in}{1.671301in}}%
\pgfpathlineto{\pgfqpoint{1.424259in}{1.653139in}}%
\pgfpathclose%
\pgfusepath{fill}%
\end{pgfscope}%
\begin{pgfscope}%
\pgfpathrectangle{\pgfqpoint{0.150000in}{0.150000in}}{\pgfqpoint{2.700000in}{1.950000in}}%
\pgfusepath{clip}%
\pgfsetbuttcap%
\pgfsetroundjoin%
\definecolor{currentfill}{rgb}{0.524219,0.582812,0.664844}%
\pgfsetfillcolor{currentfill}%
\pgfsetlinewidth{0.000000pt}%
\definecolor{currentstroke}{rgb}{0.000000,0.000000,0.000000}%
\pgfsetstrokecolor{currentstroke}%
\pgfsetdash{}{0pt}%
\pgfpathmoveto{\pgfqpoint{1.386578in}{1.634911in}}%
\pgfpathlineto{\pgfqpoint{1.424259in}{1.653139in}}%
\pgfpathlineto{\pgfqpoint{1.387121in}{1.671301in}}%
\pgfpathlineto{\pgfqpoint{1.349441in}{1.653139in}}%
\pgfpathclose%
\pgfusepath{fill}%
\end{pgfscope}%
\begin{pgfscope}%
\pgfpathrectangle{\pgfqpoint{0.150000in}{0.150000in}}{\pgfqpoint{2.700000in}{1.950000in}}%
\pgfusepath{clip}%
\pgfsetbuttcap%
\pgfsetroundjoin%
\definecolor{currentfill}{rgb}{0.524219,0.582812,0.664844}%
\pgfsetfillcolor{currentfill}%
\pgfsetlinewidth{0.000000pt}%
\definecolor{currentstroke}{rgb}{0.000000,0.000000,0.000000}%
\pgfsetstrokecolor{currentstroke}%
\pgfsetdash{}{0pt}%
\pgfpathmoveto{\pgfqpoint{1.311624in}{1.634911in}}%
\pgfpathlineto{\pgfqpoint{1.349441in}{1.653139in}}%
\pgfpathlineto{\pgfqpoint{1.312438in}{1.671301in}}%
\pgfpathlineto{\pgfqpoint{1.274623in}{1.653139in}}%
\pgfpathclose%
\pgfusepath{fill}%
\end{pgfscope}%
\begin{pgfscope}%
\pgfpathrectangle{\pgfqpoint{0.150000in}{0.150000in}}{\pgfqpoint{2.700000in}{1.950000in}}%
\pgfusepath{clip}%
\pgfsetbuttcap%
\pgfsetroundjoin%
\definecolor{currentfill}{rgb}{0.524219,0.582812,0.664844}%
\pgfsetfillcolor{currentfill}%
\pgfsetlinewidth{0.000000pt}%
\definecolor{currentstroke}{rgb}{0.000000,0.000000,0.000000}%
\pgfsetstrokecolor{currentstroke}%
\pgfsetdash{}{0pt}%
\pgfpathmoveto{\pgfqpoint{1.236670in}{1.634911in}}%
\pgfpathlineto{\pgfqpoint{1.274623in}{1.653139in}}%
\pgfpathlineto{\pgfqpoint{1.237755in}{1.671301in}}%
\pgfpathlineto{\pgfqpoint{1.199805in}{1.653139in}}%
\pgfpathclose%
\pgfusepath{fill}%
\end{pgfscope}%
\begin{pgfscope}%
\pgfpathrectangle{\pgfqpoint{0.150000in}{0.150000in}}{\pgfqpoint{2.700000in}{1.950000in}}%
\pgfusepath{clip}%
\pgfsetbuttcap%
\pgfsetroundjoin%
\definecolor{currentfill}{rgb}{0.524219,0.582812,0.664844}%
\pgfsetfillcolor{currentfill}%
\pgfsetlinewidth{0.000000pt}%
\definecolor{currentstroke}{rgb}{0.000000,0.000000,0.000000}%
\pgfsetstrokecolor{currentstroke}%
\pgfsetdash{}{0pt}%
\pgfpathmoveto{\pgfqpoint{1.161716in}{1.634911in}}%
\pgfpathlineto{\pgfqpoint{1.199805in}{1.653139in}}%
\pgfpathlineto{\pgfqpoint{1.163073in}{1.671301in}}%
\pgfpathlineto{\pgfqpoint{1.124987in}{1.653139in}}%
\pgfpathclose%
\pgfusepath{fill}%
\end{pgfscope}%
\begin{pgfscope}%
\pgfpathrectangle{\pgfqpoint{0.150000in}{0.150000in}}{\pgfqpoint{2.700000in}{1.950000in}}%
\pgfusepath{clip}%
\pgfsetbuttcap%
\pgfsetroundjoin%
\definecolor{currentfill}{rgb}{0.959574,0.964553,0.971523}%
\pgfsetfillcolor{currentfill}%
\pgfsetlinewidth{0.000000pt}%
\definecolor{currentstroke}{rgb}{0.000000,0.000000,0.000000}%
\pgfsetstrokecolor{currentstroke}%
\pgfsetdash{}{0pt}%
\pgfpathmoveto{\pgfqpoint{1.984790in}{0.913269in}}%
\pgfpathlineto{\pgfqpoint{2.021272in}{0.932743in}}%
\pgfpathlineto{\pgfqpoint{1.986084in}{1.019538in}}%
\pgfpathlineto{\pgfqpoint{1.949367in}{1.000130in}}%
\pgfpathclose%
\pgfusepath{fill}%
\end{pgfscope}%
\begin{pgfscope}%
\pgfpathrectangle{\pgfqpoint{0.150000in}{0.150000in}}{\pgfqpoint{2.700000in}{1.950000in}}%
\pgfusepath{clip}%
\pgfsetbuttcap%
\pgfsetroundjoin%
\definecolor{currentfill}{rgb}{0.661045,0.702788,0.761229}%
\pgfsetfillcolor{currentfill}%
\pgfsetlinewidth{0.000000pt}%
\definecolor{currentstroke}{rgb}{0.000000,0.000000,0.000000}%
\pgfsetstrokecolor{currentstroke}%
\pgfsetdash{}{0pt}%
\pgfpathmoveto{\pgfqpoint{2.134837in}{1.382166in}}%
\pgfpathlineto{\pgfqpoint{2.171084in}{1.400813in}}%
\pgfpathlineto{\pgfqpoint{2.135914in}{1.478084in}}%
\pgfpathlineto{\pgfqpoint{2.099471in}{1.459511in}}%
\pgfpathclose%
\pgfusepath{fill}%
\end{pgfscope}%
\begin{pgfscope}%
\pgfpathrectangle{\pgfqpoint{0.150000in}{0.150000in}}{\pgfqpoint{2.700000in}{1.950000in}}%
\pgfusepath{clip}%
\pgfsetbuttcap%
\pgfsetroundjoin%
\definecolor{currentfill}{rgb}{0.729458,0.762776,0.809421}%
\pgfsetfillcolor{currentfill}%
\pgfsetlinewidth{0.000000pt}%
\definecolor{currentstroke}{rgb}{0.000000,0.000000,0.000000}%
\pgfsetstrokecolor{currentstroke}%
\pgfsetdash{}{0pt}%
\pgfpathmoveto{\pgfqpoint{2.097448in}{1.267740in}}%
\pgfpathlineto{\pgfqpoint{2.133764in}{1.286593in}}%
\pgfpathlineto{\pgfqpoint{2.098458in}{1.363453in}}%
\pgfpathlineto{\pgfqpoint{2.061947in}{1.344671in}}%
\pgfpathclose%
\pgfusepath{fill}%
\end{pgfscope}%
\begin{pgfscope}%
\pgfpathrectangle{\pgfqpoint{0.150000in}{0.150000in}}{\pgfqpoint{2.700000in}{1.950000in}}%
\pgfusepath{clip}%
\pgfsetbuttcap%
\pgfsetroundjoin%
\definecolor{currentfill}{rgb}{0.804090,0.828217,0.861994}%
\pgfsetfillcolor{currentfill}%
\pgfsetlinewidth{0.000000pt}%
\definecolor{currentstroke}{rgb}{0.000000,0.000000,0.000000}%
\pgfsetstrokecolor{currentstroke}%
\pgfsetdash{}{0pt}%
\pgfpathmoveto{\pgfqpoint{2.060059in}{1.153309in}}%
\pgfpathlineto{\pgfqpoint{2.096442in}{1.172370in}}%
\pgfpathlineto{\pgfqpoint{2.061001in}{1.248818in}}%
\pgfpathlineto{\pgfqpoint{2.024422in}{1.229827in}}%
\pgfpathclose%
\pgfusepath{fill}%
\end{pgfscope}%
\begin{pgfscope}%
\pgfpathrectangle{\pgfqpoint{0.150000in}{0.150000in}}{\pgfqpoint{2.700000in}{1.950000in}}%
\pgfusepath{clip}%
\pgfsetbuttcap%
\pgfsetroundjoin%
\definecolor{currentfill}{rgb}{0.878722,0.893658,0.914568}%
\pgfsetfillcolor{currentfill}%
\pgfsetlinewidth{0.000000pt}%
\definecolor{currentstroke}{rgb}{0.000000,0.000000,0.000000}%
\pgfsetstrokecolor{currentstroke}%
\pgfsetdash{}{0pt}%
\pgfpathmoveto{\pgfqpoint{2.022668in}{1.038876in}}%
\pgfpathlineto{\pgfqpoint{2.059120in}{1.058143in}}%
\pgfpathlineto{\pgfqpoint{2.023543in}{1.134179in}}%
\pgfpathlineto{\pgfqpoint{1.986895in}{1.114980in}}%
\pgfpathclose%
\pgfusepath{fill}%
\end{pgfscope}%
\begin{pgfscope}%
\pgfpathrectangle{\pgfqpoint{0.150000in}{0.150000in}}{\pgfqpoint{2.700000in}{1.950000in}}%
\pgfusepath{clip}%
\pgfsetbuttcap%
\pgfsetroundjoin%
\definecolor{currentfill}{rgb}{0.979105,0.962086,0.963434}%
\pgfsetfillcolor{currentfill}%
\pgfsetlinewidth{0.000000pt}%
\definecolor{currentstroke}{rgb}{0.000000,0.000000,0.000000}%
\pgfsetstrokecolor{currentstroke}%
\pgfsetdash{}{0pt}%
\pgfpathmoveto{\pgfqpoint{1.910756in}{0.779199in}}%
\pgfpathlineto{\pgfqpoint{1.947438in}{0.798952in}}%
\pgfpathlineto{\pgfqpoint{1.911430in}{0.874108in}}%
\pgfpathlineto{\pgfqpoint{1.874550in}{0.854421in}}%
\pgfpathclose%
\pgfusepath{fill}%
\end{pgfscope}%
\begin{pgfscope}%
\pgfpathrectangle{\pgfqpoint{0.150000in}{0.150000in}}{\pgfqpoint{2.700000in}{1.950000in}}%
\pgfusepath{clip}%
\pgfsetbuttcap%
\pgfsetroundjoin%
\definecolor{currentfill}{rgb}{0.933517,0.879366,0.883655}%
\pgfsetfillcolor{currentfill}%
\pgfsetlinewidth{0.000000pt}%
\definecolor{currentstroke}{rgb}{0.000000,0.000000,0.000000}%
\pgfsetstrokecolor{currentstroke}%
\pgfsetdash{}{0pt}%
\pgfpathmoveto{\pgfqpoint{1.873334in}{0.664671in}}%
\pgfpathlineto{\pgfqpoint{1.910084in}{0.684632in}}%
\pgfpathlineto{\pgfqpoint{1.873941in}{0.759375in}}%
\pgfpathlineto{\pgfqpoint{1.836992in}{0.739479in}}%
\pgfpathclose%
\pgfusepath{fill}%
\end{pgfscope}%
\begin{pgfscope}%
\pgfpathrectangle{\pgfqpoint{0.150000in}{0.150000in}}{\pgfqpoint{2.700000in}{1.950000in}}%
\pgfusepath{clip}%
\pgfsetbuttcap%
\pgfsetroundjoin%
\definecolor{currentfill}{rgb}{0.887929,0.796645,0.803876}%
\pgfsetfillcolor{currentfill}%
\pgfsetlinewidth{0.000000pt}%
\definecolor{currentstroke}{rgb}{0.000000,0.000000,0.000000}%
\pgfsetstrokecolor{currentstroke}%
\pgfsetdash{}{0pt}%
\pgfpathmoveto{\pgfqpoint{1.835911in}{0.550140in}}%
\pgfpathlineto{\pgfqpoint{1.872730in}{0.570308in}}%
\pgfpathlineto{\pgfqpoint{1.836451in}{0.644639in}}%
\pgfpathlineto{\pgfqpoint{1.799433in}{0.624533in}}%
\pgfpathclose%
\pgfusepath{fill}%
\end{pgfscope}%
\begin{pgfscope}%
\pgfpathrectangle{\pgfqpoint{0.150000in}{0.150000in}}{\pgfqpoint{2.700000in}{1.950000in}}%
\pgfusepath{clip}%
\pgfsetbuttcap%
\pgfsetroundjoin%
\definecolor{currentfill}{rgb}{0.846140,0.720818,0.730744}%
\pgfsetfillcolor{currentfill}%
\pgfsetlinewidth{0.000000pt}%
\definecolor{currentstroke}{rgb}{0.000000,0.000000,0.000000}%
\pgfsetstrokecolor{currentstroke}%
\pgfsetdash{}{0pt}%
\pgfpathmoveto{\pgfqpoint{1.798487in}{0.435605in}}%
\pgfpathlineto{\pgfqpoint{1.835374in}{0.455980in}}%
\pgfpathlineto{\pgfqpoint{1.798959in}{0.529899in}}%
\pgfpathlineto{\pgfqpoint{1.761873in}{0.509584in}}%
\pgfpathclose%
\pgfusepath{fill}%
\end{pgfscope}%
\begin{pgfscope}%
\pgfpathrectangle{\pgfqpoint{0.150000in}{0.150000in}}{\pgfqpoint{2.700000in}{1.950000in}}%
\pgfusepath{clip}%
\pgfsetbuttcap%
\pgfsetroundjoin%
\definecolor{currentfill}{rgb}{0.524219,0.582812,0.664844}%
\pgfsetfillcolor{currentfill}%
\pgfsetlinewidth{0.000000pt}%
\definecolor{currentstroke}{rgb}{0.000000,0.000000,0.000000}%
\pgfsetstrokecolor{currentstroke}%
\pgfsetdash{}{0pt}%
\pgfpathmoveto{\pgfqpoint{1.949483in}{1.616617in}}%
\pgfpathlineto{\pgfqpoint{1.986210in}{1.634911in}}%
\pgfpathlineto{\pgfqpoint{1.947986in}{1.653139in}}%
\pgfpathlineto{\pgfqpoint{1.911256in}{1.634911in}}%
\pgfpathclose%
\pgfusepath{fill}%
\end{pgfscope}%
\begin{pgfscope}%
\pgfpathrectangle{\pgfqpoint{0.150000in}{0.150000in}}{\pgfqpoint{2.700000in}{1.950000in}}%
\pgfusepath{clip}%
\pgfsetbuttcap%
\pgfsetroundjoin%
\definecolor{currentfill}{rgb}{0.524219,0.582812,0.664844}%
\pgfsetfillcolor{currentfill}%
\pgfsetlinewidth{0.000000pt}%
\definecolor{currentstroke}{rgb}{0.000000,0.000000,0.000000}%
\pgfsetstrokecolor{currentstroke}%
\pgfsetdash{}{0pt}%
\pgfpathmoveto{\pgfqpoint{1.874393in}{1.616617in}}%
\pgfpathlineto{\pgfqpoint{1.911256in}{1.634911in}}%
\pgfpathlineto{\pgfqpoint{1.873168in}{1.653139in}}%
\pgfpathlineto{\pgfqpoint{1.836302in}{1.634911in}}%
\pgfpathclose%
\pgfusepath{fill}%
\end{pgfscope}%
\begin{pgfscope}%
\pgfpathrectangle{\pgfqpoint{0.150000in}{0.150000in}}{\pgfqpoint{2.700000in}{1.950000in}}%
\pgfusepath{clip}%
\pgfsetbuttcap%
\pgfsetroundjoin%
\definecolor{currentfill}{rgb}{0.524219,0.582812,0.664844}%
\pgfsetfillcolor{currentfill}%
\pgfsetlinewidth{0.000000pt}%
\definecolor{currentstroke}{rgb}{0.000000,0.000000,0.000000}%
\pgfsetstrokecolor{currentstroke}%
\pgfsetdash{}{0pt}%
\pgfpathmoveto{\pgfqpoint{1.799303in}{1.616617in}}%
\pgfpathlineto{\pgfqpoint{1.836302in}{1.634911in}}%
\pgfpathlineto{\pgfqpoint{1.798350in}{1.653139in}}%
\pgfpathlineto{\pgfqpoint{1.761348in}{1.634911in}}%
\pgfpathclose%
\pgfusepath{fill}%
\end{pgfscope}%
\begin{pgfscope}%
\pgfpathrectangle{\pgfqpoint{0.150000in}{0.150000in}}{\pgfqpoint{2.700000in}{1.950000in}}%
\pgfusepath{clip}%
\pgfsetbuttcap%
\pgfsetroundjoin%
\definecolor{currentfill}{rgb}{0.524219,0.582812,0.664844}%
\pgfsetfillcolor{currentfill}%
\pgfsetlinewidth{0.000000pt}%
\definecolor{currentstroke}{rgb}{0.000000,0.000000,0.000000}%
\pgfsetstrokecolor{currentstroke}%
\pgfsetdash{}{0pt}%
\pgfpathmoveto{\pgfqpoint{1.724212in}{1.616617in}}%
\pgfpathlineto{\pgfqpoint{1.761348in}{1.634911in}}%
\pgfpathlineto{\pgfqpoint{1.723532in}{1.653139in}}%
\pgfpathlineto{\pgfqpoint{1.686394in}{1.634911in}}%
\pgfpathclose%
\pgfusepath{fill}%
\end{pgfscope}%
\begin{pgfscope}%
\pgfpathrectangle{\pgfqpoint{0.150000in}{0.150000in}}{\pgfqpoint{2.700000in}{1.950000in}}%
\pgfusepath{clip}%
\pgfsetbuttcap%
\pgfsetroundjoin%
\definecolor{currentfill}{rgb}{0.524219,0.582812,0.664844}%
\pgfsetfillcolor{currentfill}%
\pgfsetlinewidth{0.000000pt}%
\definecolor{currentstroke}{rgb}{0.000000,0.000000,0.000000}%
\pgfsetstrokecolor{currentstroke}%
\pgfsetdash{}{0pt}%
\pgfpathmoveto{\pgfqpoint{1.649122in}{1.616617in}}%
\pgfpathlineto{\pgfqpoint{1.686394in}{1.634911in}}%
\pgfpathlineto{\pgfqpoint{1.648714in}{1.653139in}}%
\pgfpathlineto{\pgfqpoint{1.611440in}{1.634911in}}%
\pgfpathclose%
\pgfusepath{fill}%
\end{pgfscope}%
\begin{pgfscope}%
\pgfpathrectangle{\pgfqpoint{0.150000in}{0.150000in}}{\pgfqpoint{2.700000in}{1.950000in}}%
\pgfusepath{clip}%
\pgfsetbuttcap%
\pgfsetroundjoin%
\definecolor{currentfill}{rgb}{0.524219,0.582812,0.664844}%
\pgfsetfillcolor{currentfill}%
\pgfsetlinewidth{0.000000pt}%
\definecolor{currentstroke}{rgb}{0.000000,0.000000,0.000000}%
\pgfsetstrokecolor{currentstroke}%
\pgfsetdash{}{0pt}%
\pgfpathmoveto{\pgfqpoint{1.574032in}{1.616617in}}%
\pgfpathlineto{\pgfqpoint{1.611440in}{1.634911in}}%
\pgfpathlineto{\pgfqpoint{1.573896in}{1.653139in}}%
\pgfpathlineto{\pgfqpoint{1.536486in}{1.634911in}}%
\pgfpathclose%
\pgfusepath{fill}%
\end{pgfscope}%
\begin{pgfscope}%
\pgfpathrectangle{\pgfqpoint{0.150000in}{0.150000in}}{\pgfqpoint{2.700000in}{1.950000in}}%
\pgfusepath{clip}%
\pgfsetbuttcap%
\pgfsetroundjoin%
\definecolor{currentfill}{rgb}{0.524219,0.582812,0.664844}%
\pgfsetfillcolor{currentfill}%
\pgfsetlinewidth{0.000000pt}%
\definecolor{currentstroke}{rgb}{0.000000,0.000000,0.000000}%
\pgfsetstrokecolor{currentstroke}%
\pgfsetdash{}{0pt}%
\pgfpathmoveto{\pgfqpoint{1.498941in}{1.616617in}}%
\pgfpathlineto{\pgfqpoint{1.536486in}{1.634911in}}%
\pgfpathlineto{\pgfqpoint{1.499077in}{1.653139in}}%
\pgfpathlineto{\pgfqpoint{1.461532in}{1.634911in}}%
\pgfpathclose%
\pgfusepath{fill}%
\end{pgfscope}%
\begin{pgfscope}%
\pgfpathrectangle{\pgfqpoint{0.150000in}{0.150000in}}{\pgfqpoint{2.700000in}{1.950000in}}%
\pgfusepath{clip}%
\pgfsetbuttcap%
\pgfsetroundjoin%
\definecolor{currentfill}{rgb}{0.524219,0.582812,0.664844}%
\pgfsetfillcolor{currentfill}%
\pgfsetlinewidth{0.000000pt}%
\definecolor{currentstroke}{rgb}{0.000000,0.000000,0.000000}%
\pgfsetstrokecolor{currentstroke}%
\pgfsetdash{}{0pt}%
\pgfpathmoveto{\pgfqpoint{1.423851in}{1.616617in}}%
\pgfpathlineto{\pgfqpoint{1.461532in}{1.634911in}}%
\pgfpathlineto{\pgfqpoint{1.424259in}{1.653139in}}%
\pgfpathlineto{\pgfqpoint{1.386578in}{1.634911in}}%
\pgfpathclose%
\pgfusepath{fill}%
\end{pgfscope}%
\begin{pgfscope}%
\pgfpathrectangle{\pgfqpoint{0.150000in}{0.150000in}}{\pgfqpoint{2.700000in}{1.950000in}}%
\pgfusepath{clip}%
\pgfsetbuttcap%
\pgfsetroundjoin%
\definecolor{currentfill}{rgb}{0.524219,0.582812,0.664844}%
\pgfsetfillcolor{currentfill}%
\pgfsetlinewidth{0.000000pt}%
\definecolor{currentstroke}{rgb}{0.000000,0.000000,0.000000}%
\pgfsetstrokecolor{currentstroke}%
\pgfsetdash{}{0pt}%
\pgfpathmoveto{\pgfqpoint{1.348761in}{1.616617in}}%
\pgfpathlineto{\pgfqpoint{1.386578in}{1.634911in}}%
\pgfpathlineto{\pgfqpoint{1.349441in}{1.653139in}}%
\pgfpathlineto{\pgfqpoint{1.311624in}{1.634911in}}%
\pgfpathclose%
\pgfusepath{fill}%
\end{pgfscope}%
\begin{pgfscope}%
\pgfpathrectangle{\pgfqpoint{0.150000in}{0.150000in}}{\pgfqpoint{2.700000in}{1.950000in}}%
\pgfusepath{clip}%
\pgfsetbuttcap%
\pgfsetroundjoin%
\definecolor{currentfill}{rgb}{0.524219,0.582812,0.664844}%
\pgfsetfillcolor{currentfill}%
\pgfsetlinewidth{0.000000pt}%
\definecolor{currentstroke}{rgb}{0.000000,0.000000,0.000000}%
\pgfsetstrokecolor{currentstroke}%
\pgfsetdash{}{0pt}%
\pgfpathmoveto{\pgfqpoint{1.273670in}{1.616617in}}%
\pgfpathlineto{\pgfqpoint{1.311624in}{1.634911in}}%
\pgfpathlineto{\pgfqpoint{1.274623in}{1.653139in}}%
\pgfpathlineto{\pgfqpoint{1.236670in}{1.634911in}}%
\pgfpathclose%
\pgfusepath{fill}%
\end{pgfscope}%
\begin{pgfscope}%
\pgfpathrectangle{\pgfqpoint{0.150000in}{0.150000in}}{\pgfqpoint{2.700000in}{1.950000in}}%
\pgfusepath{clip}%
\pgfsetbuttcap%
\pgfsetroundjoin%
\definecolor{currentfill}{rgb}{0.524219,0.582812,0.664844}%
\pgfsetfillcolor{currentfill}%
\pgfsetlinewidth{0.000000pt}%
\definecolor{currentstroke}{rgb}{0.000000,0.000000,0.000000}%
\pgfsetstrokecolor{currentstroke}%
\pgfsetdash{}{0pt}%
\pgfpathmoveto{\pgfqpoint{1.198580in}{1.616617in}}%
\pgfpathlineto{\pgfqpoint{1.236670in}{1.634911in}}%
\pgfpathlineto{\pgfqpoint{1.199805in}{1.653139in}}%
\pgfpathlineto{\pgfqpoint{1.161716in}{1.634911in}}%
\pgfpathclose%
\pgfusepath{fill}%
\end{pgfscope}%
\begin{pgfscope}%
\pgfpathrectangle{\pgfqpoint{0.150000in}{0.150000in}}{\pgfqpoint{2.700000in}{1.950000in}}%
\pgfusepath{clip}%
\pgfsetbuttcap%
\pgfsetroundjoin%
\definecolor{currentfill}{rgb}{0.524219,0.582812,0.664844}%
\pgfsetfillcolor{currentfill}%
\pgfsetlinewidth{0.000000pt}%
\definecolor{currentstroke}{rgb}{0.000000,0.000000,0.000000}%
\pgfsetstrokecolor{currentstroke}%
\pgfsetdash{}{0pt}%
\pgfpathmoveto{\pgfqpoint{1.123490in}{1.616617in}}%
\pgfpathlineto{\pgfqpoint{1.161716in}{1.634911in}}%
\pgfpathlineto{\pgfqpoint{1.124987in}{1.653139in}}%
\pgfpathlineto{\pgfqpoint{1.086762in}{1.634911in}}%
\pgfpathclose%
\pgfusepath{fill}%
\end{pgfscope}%
\begin{pgfscope}%
\pgfpathrectangle{\pgfqpoint{0.150000in}{0.150000in}}{\pgfqpoint{2.700000in}{1.950000in}}%
\pgfusepath{clip}%
\pgfsetbuttcap%
\pgfsetroundjoin%
\definecolor{currentfill}{rgb}{0.959574,0.964553,0.971523}%
\pgfsetfillcolor{currentfill}%
\pgfsetlinewidth{0.000000pt}%
\definecolor{currentstroke}{rgb}{0.000000,0.000000,0.000000}%
\pgfsetstrokecolor{currentstroke}%
\pgfsetdash{}{0pt}%
\pgfpathmoveto{\pgfqpoint{1.948177in}{0.893724in}}%
\pgfpathlineto{\pgfqpoint{1.984790in}{0.913269in}}%
\pgfpathlineto{\pgfqpoint{1.949367in}{1.000130in}}%
\pgfpathlineto{\pgfqpoint{1.912516in}{0.980651in}}%
\pgfpathclose%
\pgfusepath{fill}%
\end{pgfscope}%
\begin{pgfscope}%
\pgfpathrectangle{\pgfqpoint{0.150000in}{0.150000in}}{\pgfqpoint{2.700000in}{1.950000in}}%
\pgfusepath{clip}%
\pgfsetbuttcap%
\pgfsetroundjoin%
\definecolor{currentfill}{rgb}{0.661045,0.702788,0.761229}%
\pgfsetfillcolor{currentfill}%
\pgfsetlinewidth{0.000000pt}%
\definecolor{currentstroke}{rgb}{0.000000,0.000000,0.000000}%
\pgfsetstrokecolor{currentstroke}%
\pgfsetdash{}{0pt}%
\pgfpathmoveto{\pgfqpoint{2.098458in}{1.363453in}}%
\pgfpathlineto{\pgfqpoint{2.134837in}{1.382166in}}%
\pgfpathlineto{\pgfqpoint{2.099471in}{1.459511in}}%
\pgfpathlineto{\pgfqpoint{2.062896in}{1.440870in}}%
\pgfpathclose%
\pgfusepath{fill}%
\end{pgfscope}%
\begin{pgfscope}%
\pgfpathrectangle{\pgfqpoint{0.150000in}{0.150000in}}{\pgfqpoint{2.700000in}{1.950000in}}%
\pgfusepath{clip}%
\pgfsetbuttcap%
\pgfsetroundjoin%
\definecolor{currentfill}{rgb}{0.729458,0.762776,0.809421}%
\pgfsetfillcolor{currentfill}%
\pgfsetlinewidth{0.000000pt}%
\definecolor{currentstroke}{rgb}{0.000000,0.000000,0.000000}%
\pgfsetstrokecolor{currentstroke}%
\pgfsetdash{}{0pt}%
\pgfpathmoveto{\pgfqpoint{2.061001in}{1.248818in}}%
\pgfpathlineto{\pgfqpoint{2.097448in}{1.267740in}}%
\pgfpathlineto{\pgfqpoint{2.061947in}{1.344671in}}%
\pgfpathlineto{\pgfqpoint{2.025303in}{1.325820in}}%
\pgfpathclose%
\pgfusepath{fill}%
\end{pgfscope}%
\begin{pgfscope}%
\pgfpathrectangle{\pgfqpoint{0.150000in}{0.150000in}}{\pgfqpoint{2.700000in}{1.950000in}}%
\pgfusepath{clip}%
\pgfsetbuttcap%
\pgfsetroundjoin%
\definecolor{currentfill}{rgb}{0.804090,0.828217,0.861994}%
\pgfsetfillcolor{currentfill}%
\pgfsetlinewidth{0.000000pt}%
\definecolor{currentstroke}{rgb}{0.000000,0.000000,0.000000}%
\pgfsetstrokecolor{currentstroke}%
\pgfsetdash{}{0pt}%
\pgfpathmoveto{\pgfqpoint{2.023543in}{1.134179in}}%
\pgfpathlineto{\pgfqpoint{2.060059in}{1.153309in}}%
\pgfpathlineto{\pgfqpoint{2.024422in}{1.229827in}}%
\pgfpathlineto{\pgfqpoint{1.987709in}{1.210767in}}%
\pgfpathclose%
\pgfusepath{fill}%
\end{pgfscope}%
\begin{pgfscope}%
\pgfpathrectangle{\pgfqpoint{0.150000in}{0.150000in}}{\pgfqpoint{2.700000in}{1.950000in}}%
\pgfusepath{clip}%
\pgfsetbuttcap%
\pgfsetroundjoin%
\definecolor{currentfill}{rgb}{0.878722,0.893658,0.914568}%
\pgfsetfillcolor{currentfill}%
\pgfsetlinewidth{0.000000pt}%
\definecolor{currentstroke}{rgb}{0.000000,0.000000,0.000000}%
\pgfsetstrokecolor{currentstroke}%
\pgfsetdash{}{0pt}%
\pgfpathmoveto{\pgfqpoint{1.986084in}{1.019538in}}%
\pgfpathlineto{\pgfqpoint{2.022668in}{1.038876in}}%
\pgfpathlineto{\pgfqpoint{1.986895in}{1.114980in}}%
\pgfpathlineto{\pgfqpoint{1.950113in}{1.095711in}}%
\pgfpathclose%
\pgfusepath{fill}%
\end{pgfscope}%
\begin{pgfscope}%
\pgfpathrectangle{\pgfqpoint{0.150000in}{0.150000in}}{\pgfqpoint{2.700000in}{1.950000in}}%
\pgfusepath{clip}%
\pgfsetbuttcap%
\pgfsetroundjoin%
\definecolor{currentfill}{rgb}{0.979105,0.962086,0.963434}%
\pgfsetfillcolor{currentfill}%
\pgfsetlinewidth{0.000000pt}%
\definecolor{currentstroke}{rgb}{0.000000,0.000000,0.000000}%
\pgfsetstrokecolor{currentstroke}%
\pgfsetdash{}{0pt}%
\pgfpathmoveto{\pgfqpoint{1.873941in}{0.759375in}}%
\pgfpathlineto{\pgfqpoint{1.910756in}{0.779199in}}%
\pgfpathlineto{\pgfqpoint{1.874550in}{0.854421in}}%
\pgfpathlineto{\pgfqpoint{1.837535in}{0.834662in}}%
\pgfpathclose%
\pgfusepath{fill}%
\end{pgfscope}%
\begin{pgfscope}%
\pgfpathrectangle{\pgfqpoint{0.150000in}{0.150000in}}{\pgfqpoint{2.700000in}{1.950000in}}%
\pgfusepath{clip}%
\pgfsetbuttcap%
\pgfsetroundjoin%
\definecolor{currentfill}{rgb}{0.933517,0.879366,0.883655}%
\pgfsetfillcolor{currentfill}%
\pgfsetlinewidth{0.000000pt}%
\definecolor{currentstroke}{rgb}{0.000000,0.000000,0.000000}%
\pgfsetstrokecolor{currentstroke}%
\pgfsetdash{}{0pt}%
\pgfpathmoveto{\pgfqpoint{1.836451in}{0.644639in}}%
\pgfpathlineto{\pgfqpoint{1.873334in}{0.664671in}}%
\pgfpathlineto{\pgfqpoint{1.836992in}{0.739479in}}%
\pgfpathlineto{\pgfqpoint{1.799908in}{0.719510in}}%
\pgfpathclose%
\pgfusepath{fill}%
\end{pgfscope}%
\begin{pgfscope}%
\pgfpathrectangle{\pgfqpoint{0.150000in}{0.150000in}}{\pgfqpoint{2.700000in}{1.950000in}}%
\pgfusepath{clip}%
\pgfsetbuttcap%
\pgfsetroundjoin%
\definecolor{currentfill}{rgb}{0.887929,0.796645,0.803876}%
\pgfsetfillcolor{currentfill}%
\pgfsetlinewidth{0.000000pt}%
\definecolor{currentstroke}{rgb}{0.000000,0.000000,0.000000}%
\pgfsetstrokecolor{currentstroke}%
\pgfsetdash{}{0pt}%
\pgfpathmoveto{\pgfqpoint{1.798959in}{0.529899in}}%
\pgfpathlineto{\pgfqpoint{1.835911in}{0.550140in}}%
\pgfpathlineto{\pgfqpoint{1.799433in}{0.624533in}}%
\pgfpathlineto{\pgfqpoint{1.762280in}{0.604354in}}%
\pgfpathclose%
\pgfusepath{fill}%
\end{pgfscope}%
\begin{pgfscope}%
\pgfpathrectangle{\pgfqpoint{0.150000in}{0.150000in}}{\pgfqpoint{2.700000in}{1.950000in}}%
\pgfusepath{clip}%
\pgfsetbuttcap%
\pgfsetroundjoin%
\definecolor{currentfill}{rgb}{0.846140,0.720818,0.730744}%
\pgfsetfillcolor{currentfill}%
\pgfsetlinewidth{0.000000pt}%
\definecolor{currentstroke}{rgb}{0.000000,0.000000,0.000000}%
\pgfsetstrokecolor{currentstroke}%
\pgfsetdash{}{0pt}%
\pgfpathmoveto{\pgfqpoint{1.761467in}{0.415155in}}%
\pgfpathlineto{\pgfqpoint{1.798487in}{0.435605in}}%
\pgfpathlineto{\pgfqpoint{1.761873in}{0.509584in}}%
\pgfpathlineto{\pgfqpoint{1.724651in}{0.489195in}}%
\pgfpathclose%
\pgfusepath{fill}%
\end{pgfscope}%
\begin{pgfscope}%
\pgfpathrectangle{\pgfqpoint{0.150000in}{0.150000in}}{\pgfqpoint{2.700000in}{1.950000in}}%
\pgfusepath{clip}%
\pgfsetbuttcap%
\pgfsetroundjoin%
\definecolor{currentfill}{rgb}{0.524219,0.582812,0.664844}%
\pgfsetfillcolor{currentfill}%
\pgfsetlinewidth{0.000000pt}%
\definecolor{currentstroke}{rgb}{0.000000,0.000000,0.000000}%
\pgfsetstrokecolor{currentstroke}%
\pgfsetdash{}{0pt}%
\pgfpathmoveto{\pgfqpoint{1.987850in}{1.598256in}}%
\pgfpathlineto{\pgfqpoint{2.024574in}{1.616617in}}%
\pgfpathlineto{\pgfqpoint{1.986210in}{1.634911in}}%
\pgfpathlineto{\pgfqpoint{1.949483in}{1.616617in}}%
\pgfpathclose%
\pgfusepath{fill}%
\end{pgfscope}%
\begin{pgfscope}%
\pgfpathrectangle{\pgfqpoint{0.150000in}{0.150000in}}{\pgfqpoint{2.700000in}{1.950000in}}%
\pgfusepath{clip}%
\pgfsetbuttcap%
\pgfsetroundjoin%
\definecolor{currentfill}{rgb}{0.524219,0.582812,0.664844}%
\pgfsetfillcolor{currentfill}%
\pgfsetlinewidth{0.000000pt}%
\definecolor{currentstroke}{rgb}{0.000000,0.000000,0.000000}%
\pgfsetstrokecolor{currentstroke}%
\pgfsetdash{}{0pt}%
\pgfpathmoveto{\pgfqpoint{1.912622in}{1.598256in}}%
\pgfpathlineto{\pgfqpoint{1.949483in}{1.616617in}}%
\pgfpathlineto{\pgfqpoint{1.911256in}{1.634911in}}%
\pgfpathlineto{\pgfqpoint{1.874393in}{1.616617in}}%
\pgfpathclose%
\pgfusepath{fill}%
\end{pgfscope}%
\begin{pgfscope}%
\pgfpathrectangle{\pgfqpoint{0.150000in}{0.150000in}}{\pgfqpoint{2.700000in}{1.950000in}}%
\pgfusepath{clip}%
\pgfsetbuttcap%
\pgfsetroundjoin%
\definecolor{currentfill}{rgb}{0.524219,0.582812,0.664844}%
\pgfsetfillcolor{currentfill}%
\pgfsetlinewidth{0.000000pt}%
\definecolor{currentstroke}{rgb}{0.000000,0.000000,0.000000}%
\pgfsetstrokecolor{currentstroke}%
\pgfsetdash{}{0pt}%
\pgfpathmoveto{\pgfqpoint{1.837395in}{1.598256in}}%
\pgfpathlineto{\pgfqpoint{1.874393in}{1.616617in}}%
\pgfpathlineto{\pgfqpoint{1.836302in}{1.634911in}}%
\pgfpathlineto{\pgfqpoint{1.799303in}{1.616617in}}%
\pgfpathclose%
\pgfusepath{fill}%
\end{pgfscope}%
\begin{pgfscope}%
\pgfpathrectangle{\pgfqpoint{0.150000in}{0.150000in}}{\pgfqpoint{2.700000in}{1.950000in}}%
\pgfusepath{clip}%
\pgfsetbuttcap%
\pgfsetroundjoin%
\definecolor{currentfill}{rgb}{0.524219,0.582812,0.664844}%
\pgfsetfillcolor{currentfill}%
\pgfsetlinewidth{0.000000pt}%
\definecolor{currentstroke}{rgb}{0.000000,0.000000,0.000000}%
\pgfsetstrokecolor{currentstroke}%
\pgfsetdash{}{0pt}%
\pgfpathmoveto{\pgfqpoint{1.762168in}{1.598256in}}%
\pgfpathlineto{\pgfqpoint{1.799303in}{1.616617in}}%
\pgfpathlineto{\pgfqpoint{1.761348in}{1.634911in}}%
\pgfpathlineto{\pgfqpoint{1.724212in}{1.616617in}}%
\pgfpathclose%
\pgfusepath{fill}%
\end{pgfscope}%
\begin{pgfscope}%
\pgfpathrectangle{\pgfqpoint{0.150000in}{0.150000in}}{\pgfqpoint{2.700000in}{1.950000in}}%
\pgfusepath{clip}%
\pgfsetbuttcap%
\pgfsetroundjoin%
\definecolor{currentfill}{rgb}{0.524219,0.582812,0.664844}%
\pgfsetfillcolor{currentfill}%
\pgfsetlinewidth{0.000000pt}%
\definecolor{currentstroke}{rgb}{0.000000,0.000000,0.000000}%
\pgfsetstrokecolor{currentstroke}%
\pgfsetdash{}{0pt}%
\pgfpathmoveto{\pgfqpoint{1.686941in}{1.598256in}}%
\pgfpathlineto{\pgfqpoint{1.724212in}{1.616617in}}%
\pgfpathlineto{\pgfqpoint{1.686394in}{1.634911in}}%
\pgfpathlineto{\pgfqpoint{1.649122in}{1.616617in}}%
\pgfpathclose%
\pgfusepath{fill}%
\end{pgfscope}%
\begin{pgfscope}%
\pgfpathrectangle{\pgfqpoint{0.150000in}{0.150000in}}{\pgfqpoint{2.700000in}{1.950000in}}%
\pgfusepath{clip}%
\pgfsetbuttcap%
\pgfsetroundjoin%
\definecolor{currentfill}{rgb}{0.524219,0.582812,0.664844}%
\pgfsetfillcolor{currentfill}%
\pgfsetlinewidth{0.000000pt}%
\definecolor{currentstroke}{rgb}{0.000000,0.000000,0.000000}%
\pgfsetstrokecolor{currentstroke}%
\pgfsetdash{}{0pt}%
\pgfpathmoveto{\pgfqpoint{1.611714in}{1.598256in}}%
\pgfpathlineto{\pgfqpoint{1.649122in}{1.616617in}}%
\pgfpathlineto{\pgfqpoint{1.611440in}{1.634911in}}%
\pgfpathlineto{\pgfqpoint{1.574032in}{1.616617in}}%
\pgfpathclose%
\pgfusepath{fill}%
\end{pgfscope}%
\begin{pgfscope}%
\pgfpathrectangle{\pgfqpoint{0.150000in}{0.150000in}}{\pgfqpoint{2.700000in}{1.950000in}}%
\pgfusepath{clip}%
\pgfsetbuttcap%
\pgfsetroundjoin%
\definecolor{currentfill}{rgb}{0.524219,0.582812,0.664844}%
\pgfsetfillcolor{currentfill}%
\pgfsetlinewidth{0.000000pt}%
\definecolor{currentstroke}{rgb}{0.000000,0.000000,0.000000}%
\pgfsetstrokecolor{currentstroke}%
\pgfsetdash{}{0pt}%
\pgfpathmoveto{\pgfqpoint{1.536486in}{1.598256in}}%
\pgfpathlineto{\pgfqpoint{1.574032in}{1.616617in}}%
\pgfpathlineto{\pgfqpoint{1.536486in}{1.634911in}}%
\pgfpathlineto{\pgfqpoint{1.498941in}{1.616617in}}%
\pgfpathclose%
\pgfusepath{fill}%
\end{pgfscope}%
\begin{pgfscope}%
\pgfpathrectangle{\pgfqpoint{0.150000in}{0.150000in}}{\pgfqpoint{2.700000in}{1.950000in}}%
\pgfusepath{clip}%
\pgfsetbuttcap%
\pgfsetroundjoin%
\definecolor{currentfill}{rgb}{0.524219,0.582812,0.664844}%
\pgfsetfillcolor{currentfill}%
\pgfsetlinewidth{0.000000pt}%
\definecolor{currentstroke}{rgb}{0.000000,0.000000,0.000000}%
\pgfsetstrokecolor{currentstroke}%
\pgfsetdash{}{0pt}%
\pgfpathmoveto{\pgfqpoint{1.461259in}{1.598256in}}%
\pgfpathlineto{\pgfqpoint{1.498941in}{1.616617in}}%
\pgfpathlineto{\pgfqpoint{1.461532in}{1.634911in}}%
\pgfpathlineto{\pgfqpoint{1.423851in}{1.616617in}}%
\pgfpathclose%
\pgfusepath{fill}%
\end{pgfscope}%
\begin{pgfscope}%
\pgfpathrectangle{\pgfqpoint{0.150000in}{0.150000in}}{\pgfqpoint{2.700000in}{1.950000in}}%
\pgfusepath{clip}%
\pgfsetbuttcap%
\pgfsetroundjoin%
\definecolor{currentfill}{rgb}{0.524219,0.582812,0.664844}%
\pgfsetfillcolor{currentfill}%
\pgfsetlinewidth{0.000000pt}%
\definecolor{currentstroke}{rgb}{0.000000,0.000000,0.000000}%
\pgfsetstrokecolor{currentstroke}%
\pgfsetdash{}{0pt}%
\pgfpathmoveto{\pgfqpoint{1.386032in}{1.598256in}}%
\pgfpathlineto{\pgfqpoint{1.423851in}{1.616617in}}%
\pgfpathlineto{\pgfqpoint{1.386578in}{1.634911in}}%
\pgfpathlineto{\pgfqpoint{1.348761in}{1.616617in}}%
\pgfpathclose%
\pgfusepath{fill}%
\end{pgfscope}%
\begin{pgfscope}%
\pgfpathrectangle{\pgfqpoint{0.150000in}{0.150000in}}{\pgfqpoint{2.700000in}{1.950000in}}%
\pgfusepath{clip}%
\pgfsetbuttcap%
\pgfsetroundjoin%
\definecolor{currentfill}{rgb}{0.524219,0.582812,0.664844}%
\pgfsetfillcolor{currentfill}%
\pgfsetlinewidth{0.000000pt}%
\definecolor{currentstroke}{rgb}{0.000000,0.000000,0.000000}%
\pgfsetstrokecolor{currentstroke}%
\pgfsetdash{}{0pt}%
\pgfpathmoveto{\pgfqpoint{1.310805in}{1.598256in}}%
\pgfpathlineto{\pgfqpoint{1.348761in}{1.616617in}}%
\pgfpathlineto{\pgfqpoint{1.311624in}{1.634911in}}%
\pgfpathlineto{\pgfqpoint{1.273670in}{1.616617in}}%
\pgfpathclose%
\pgfusepath{fill}%
\end{pgfscope}%
\begin{pgfscope}%
\pgfpathrectangle{\pgfqpoint{0.150000in}{0.150000in}}{\pgfqpoint{2.700000in}{1.950000in}}%
\pgfusepath{clip}%
\pgfsetbuttcap%
\pgfsetroundjoin%
\definecolor{currentfill}{rgb}{0.524219,0.582812,0.664844}%
\pgfsetfillcolor{currentfill}%
\pgfsetlinewidth{0.000000pt}%
\definecolor{currentstroke}{rgb}{0.000000,0.000000,0.000000}%
\pgfsetstrokecolor{currentstroke}%
\pgfsetdash{}{0pt}%
\pgfpathmoveto{\pgfqpoint{1.235578in}{1.598256in}}%
\pgfpathlineto{\pgfqpoint{1.273670in}{1.616617in}}%
\pgfpathlineto{\pgfqpoint{1.236670in}{1.634911in}}%
\pgfpathlineto{\pgfqpoint{1.198580in}{1.616617in}}%
\pgfpathclose%
\pgfusepath{fill}%
\end{pgfscope}%
\begin{pgfscope}%
\pgfpathrectangle{\pgfqpoint{0.150000in}{0.150000in}}{\pgfqpoint{2.700000in}{1.950000in}}%
\pgfusepath{clip}%
\pgfsetbuttcap%
\pgfsetroundjoin%
\definecolor{currentfill}{rgb}{0.524219,0.582812,0.664844}%
\pgfsetfillcolor{currentfill}%
\pgfsetlinewidth{0.000000pt}%
\definecolor{currentstroke}{rgb}{0.000000,0.000000,0.000000}%
\pgfsetstrokecolor{currentstroke}%
\pgfsetdash{}{0pt}%
\pgfpathmoveto{\pgfqpoint{1.160351in}{1.598256in}}%
\pgfpathlineto{\pgfqpoint{1.198580in}{1.616617in}}%
\pgfpathlineto{\pgfqpoint{1.161716in}{1.634911in}}%
\pgfpathlineto{\pgfqpoint{1.123490in}{1.616617in}}%
\pgfpathclose%
\pgfusepath{fill}%
\end{pgfscope}%
\begin{pgfscope}%
\pgfpathrectangle{\pgfqpoint{0.150000in}{0.150000in}}{\pgfqpoint{2.700000in}{1.950000in}}%
\pgfusepath{clip}%
\pgfsetbuttcap%
\pgfsetroundjoin%
\definecolor{currentfill}{rgb}{0.524219,0.582812,0.664844}%
\pgfsetfillcolor{currentfill}%
\pgfsetlinewidth{0.000000pt}%
\definecolor{currentstroke}{rgb}{0.000000,0.000000,0.000000}%
\pgfsetstrokecolor{currentstroke}%
\pgfsetdash{}{0pt}%
\pgfpathmoveto{\pgfqpoint{1.085123in}{1.598256in}}%
\pgfpathlineto{\pgfqpoint{1.123490in}{1.616617in}}%
\pgfpathlineto{\pgfqpoint{1.086762in}{1.634911in}}%
\pgfpathlineto{\pgfqpoint{1.048399in}{1.616617in}}%
\pgfpathclose%
\pgfusepath{fill}%
\end{pgfscope}%
\begin{pgfscope}%
\pgfpathrectangle{\pgfqpoint{0.150000in}{0.150000in}}{\pgfqpoint{2.700000in}{1.950000in}}%
\pgfusepath{clip}%
\pgfsetbuttcap%
\pgfsetroundjoin%
\definecolor{currentfill}{rgb}{0.959574,0.964553,0.971523}%
\pgfsetfillcolor{currentfill}%
\pgfsetlinewidth{0.000000pt}%
\definecolor{currentstroke}{rgb}{0.000000,0.000000,0.000000}%
\pgfsetstrokecolor{currentstroke}%
\pgfsetdash{}{0pt}%
\pgfpathmoveto{\pgfqpoint{1.911430in}{0.874108in}}%
\pgfpathlineto{\pgfqpoint{1.948177in}{0.893724in}}%
\pgfpathlineto{\pgfqpoint{1.912516in}{0.980651in}}%
\pgfpathlineto{\pgfqpoint{1.875531in}{0.961101in}}%
\pgfpathclose%
\pgfusepath{fill}%
\end{pgfscope}%
\begin{pgfscope}%
\pgfpathrectangle{\pgfqpoint{0.150000in}{0.150000in}}{\pgfqpoint{2.700000in}{1.950000in}}%
\pgfusepath{clip}%
\pgfsetbuttcap%
\pgfsetroundjoin%
\definecolor{currentfill}{rgb}{0.586412,0.637347,0.708655}%
\pgfsetfillcolor{currentfill}%
\pgfsetlinewidth{0.000000pt}%
\definecolor{currentstroke}{rgb}{0.000000,0.000000,0.000000}%
\pgfsetstrokecolor{currentstroke}%
\pgfsetdash{}{0pt}%
\pgfpathmoveto{\pgfqpoint{2.099471in}{1.459511in}}%
\pgfpathlineto{\pgfqpoint{2.135914in}{1.478084in}}%
\pgfpathlineto{\pgfqpoint{2.100488in}{1.555916in}}%
\pgfpathlineto{\pgfqpoint{2.063849in}{1.537417in}}%
\pgfpathclose%
\pgfusepath{fill}%
\end{pgfscope}%
\begin{pgfscope}%
\pgfpathrectangle{\pgfqpoint{0.150000in}{0.150000in}}{\pgfqpoint{2.700000in}{1.950000in}}%
\pgfusepath{clip}%
\pgfsetbuttcap%
\pgfsetroundjoin%
\definecolor{currentfill}{rgb}{0.661045,0.702788,0.761229}%
\pgfsetfillcolor{currentfill}%
\pgfsetlinewidth{0.000000pt}%
\definecolor{currentstroke}{rgb}{0.000000,0.000000,0.000000}%
\pgfsetstrokecolor{currentstroke}%
\pgfsetdash{}{0pt}%
\pgfpathmoveto{\pgfqpoint{2.061947in}{1.344671in}}%
\pgfpathlineto{\pgfqpoint{2.098458in}{1.363453in}}%
\pgfpathlineto{\pgfqpoint{2.062896in}{1.440870in}}%
\pgfpathlineto{\pgfqpoint{2.026188in}{1.422161in}}%
\pgfpathclose%
\pgfusepath{fill}%
\end{pgfscope}%
\begin{pgfscope}%
\pgfpathrectangle{\pgfqpoint{0.150000in}{0.150000in}}{\pgfqpoint{2.700000in}{1.950000in}}%
\pgfusepath{clip}%
\pgfsetbuttcap%
\pgfsetroundjoin%
\definecolor{currentfill}{rgb}{0.729458,0.762776,0.809421}%
\pgfsetfillcolor{currentfill}%
\pgfsetlinewidth{0.000000pt}%
\definecolor{currentstroke}{rgb}{0.000000,0.000000,0.000000}%
\pgfsetstrokecolor{currentstroke}%
\pgfsetdash{}{0pt}%
\pgfpathmoveto{\pgfqpoint{2.024422in}{1.229827in}}%
\pgfpathlineto{\pgfqpoint{2.061001in}{1.248818in}}%
\pgfpathlineto{\pgfqpoint{2.025303in}{1.325820in}}%
\pgfpathlineto{\pgfqpoint{1.988525in}{1.306901in}}%
\pgfpathclose%
\pgfusepath{fill}%
\end{pgfscope}%
\begin{pgfscope}%
\pgfpathrectangle{\pgfqpoint{0.150000in}{0.150000in}}{\pgfqpoint{2.700000in}{1.950000in}}%
\pgfusepath{clip}%
\pgfsetbuttcap%
\pgfsetroundjoin%
\definecolor{currentfill}{rgb}{0.804090,0.828217,0.861994}%
\pgfsetfillcolor{currentfill}%
\pgfsetlinewidth{0.000000pt}%
\definecolor{currentstroke}{rgb}{0.000000,0.000000,0.000000}%
\pgfsetstrokecolor{currentstroke}%
\pgfsetdash{}{0pt}%
\pgfpathmoveto{\pgfqpoint{1.986895in}{1.114980in}}%
\pgfpathlineto{\pgfqpoint{2.023543in}{1.134179in}}%
\pgfpathlineto{\pgfqpoint{1.987709in}{1.210767in}}%
\pgfpathlineto{\pgfqpoint{1.950862in}{1.191638in}}%
\pgfpathclose%
\pgfusepath{fill}%
\end{pgfscope}%
\begin{pgfscope}%
\pgfpathrectangle{\pgfqpoint{0.150000in}{0.150000in}}{\pgfqpoint{2.700000in}{1.950000in}}%
\pgfusepath{clip}%
\pgfsetbuttcap%
\pgfsetroundjoin%
\definecolor{currentfill}{rgb}{0.878722,0.893658,0.914568}%
\pgfsetfillcolor{currentfill}%
\pgfsetlinewidth{0.000000pt}%
\definecolor{currentstroke}{rgb}{0.000000,0.000000,0.000000}%
\pgfsetstrokecolor{currentstroke}%
\pgfsetdash{}{0pt}%
\pgfpathmoveto{\pgfqpoint{1.949367in}{1.000130in}}%
\pgfpathlineto{\pgfqpoint{1.986084in}{1.019538in}}%
\pgfpathlineto{\pgfqpoint{1.950113in}{1.095711in}}%
\pgfpathlineto{\pgfqpoint{1.913197in}{1.076371in}}%
\pgfpathclose%
\pgfusepath{fill}%
\end{pgfscope}%
\begin{pgfscope}%
\pgfpathrectangle{\pgfqpoint{0.150000in}{0.150000in}}{\pgfqpoint{2.700000in}{1.950000in}}%
\pgfusepath{clip}%
\pgfsetbuttcap%
\pgfsetroundjoin%
\definecolor{currentfill}{rgb}{0.979105,0.962086,0.963434}%
\pgfsetfillcolor{currentfill}%
\pgfsetlinewidth{0.000000pt}%
\definecolor{currentstroke}{rgb}{0.000000,0.000000,0.000000}%
\pgfsetstrokecolor{currentstroke}%
\pgfsetdash{}{0pt}%
\pgfpathmoveto{\pgfqpoint{1.836992in}{0.739479in}}%
\pgfpathlineto{\pgfqpoint{1.873941in}{0.759375in}}%
\pgfpathlineto{\pgfqpoint{1.837535in}{0.834662in}}%
\pgfpathlineto{\pgfqpoint{1.800385in}{0.814830in}}%
\pgfpathclose%
\pgfusepath{fill}%
\end{pgfscope}%
\begin{pgfscope}%
\pgfpathrectangle{\pgfqpoint{0.150000in}{0.150000in}}{\pgfqpoint{2.700000in}{1.950000in}}%
\pgfusepath{clip}%
\pgfsetbuttcap%
\pgfsetroundjoin%
\definecolor{currentfill}{rgb}{0.933517,0.879366,0.883655}%
\pgfsetfillcolor{currentfill}%
\pgfsetlinewidth{0.000000pt}%
\definecolor{currentstroke}{rgb}{0.000000,0.000000,0.000000}%
\pgfsetstrokecolor{currentstroke}%
\pgfsetdash{}{0pt}%
\pgfpathmoveto{\pgfqpoint{1.799433in}{0.624533in}}%
\pgfpathlineto{\pgfqpoint{1.836451in}{0.644639in}}%
\pgfpathlineto{\pgfqpoint{1.799908in}{0.719510in}}%
\pgfpathlineto{\pgfqpoint{1.762689in}{0.699468in}}%
\pgfpathclose%
\pgfusepath{fill}%
\end{pgfscope}%
\begin{pgfscope}%
\pgfpathrectangle{\pgfqpoint{0.150000in}{0.150000in}}{\pgfqpoint{2.700000in}{1.950000in}}%
\pgfusepath{clip}%
\pgfsetbuttcap%
\pgfsetroundjoin%
\definecolor{currentfill}{rgb}{0.887929,0.796645,0.803876}%
\pgfsetfillcolor{currentfill}%
\pgfsetlinewidth{0.000000pt}%
\definecolor{currentstroke}{rgb}{0.000000,0.000000,0.000000}%
\pgfsetstrokecolor{currentstroke}%
\pgfsetdash{}{0pt}%
\pgfpathmoveto{\pgfqpoint{1.761873in}{0.509584in}}%
\pgfpathlineto{\pgfqpoint{1.798959in}{0.529899in}}%
\pgfpathlineto{\pgfqpoint{1.762280in}{0.604354in}}%
\pgfpathlineto{\pgfqpoint{1.724991in}{0.584102in}}%
\pgfpathclose%
\pgfusepath{fill}%
\end{pgfscope}%
\begin{pgfscope}%
\pgfpathrectangle{\pgfqpoint{0.150000in}{0.150000in}}{\pgfqpoint{2.700000in}{1.950000in}}%
\pgfusepath{clip}%
\pgfsetbuttcap%
\pgfsetroundjoin%
\definecolor{currentfill}{rgb}{0.846140,0.720818,0.730744}%
\pgfsetfillcolor{currentfill}%
\pgfsetlinewidth{0.000000pt}%
\definecolor{currentstroke}{rgb}{0.000000,0.000000,0.000000}%
\pgfsetstrokecolor{currentstroke}%
\pgfsetdash{}{0pt}%
\pgfpathmoveto{\pgfqpoint{1.724311in}{0.394632in}}%
\pgfpathlineto{\pgfqpoint{1.761467in}{0.415155in}}%
\pgfpathlineto{\pgfqpoint{1.724651in}{0.489195in}}%
\pgfpathlineto{\pgfqpoint{1.687293in}{0.468732in}}%
\pgfpathclose%
\pgfusepath{fill}%
\end{pgfscope}%
\begin{pgfscope}%
\pgfpathrectangle{\pgfqpoint{0.150000in}{0.150000in}}{\pgfqpoint{2.700000in}{1.950000in}}%
\pgfusepath{clip}%
\pgfsetbuttcap%
\pgfsetroundjoin%
\definecolor{currentfill}{rgb}{0.524219,0.582812,0.664844}%
\pgfsetfillcolor{currentfill}%
\pgfsetlinewidth{0.000000pt}%
\definecolor{currentstroke}{rgb}{0.000000,0.000000,0.000000}%
\pgfsetstrokecolor{currentstroke}%
\pgfsetdash{}{0pt}%
\pgfpathmoveto{\pgfqpoint{2.026356in}{1.579828in}}%
\pgfpathlineto{\pgfqpoint{2.063077in}{1.598256in}}%
\pgfpathlineto{\pgfqpoint{2.024574in}{1.616617in}}%
\pgfpathlineto{\pgfqpoint{1.987850in}{1.598256in}}%
\pgfpathclose%
\pgfusepath{fill}%
\end{pgfscope}%
\begin{pgfscope}%
\pgfpathrectangle{\pgfqpoint{0.150000in}{0.150000in}}{\pgfqpoint{2.700000in}{1.950000in}}%
\pgfusepath{clip}%
\pgfsetbuttcap%
\pgfsetroundjoin%
\definecolor{currentfill}{rgb}{0.524219,0.582812,0.664844}%
\pgfsetfillcolor{currentfill}%
\pgfsetlinewidth{0.000000pt}%
\definecolor{currentstroke}{rgb}{0.000000,0.000000,0.000000}%
\pgfsetstrokecolor{currentstroke}%
\pgfsetdash{}{0pt}%
\pgfpathmoveto{\pgfqpoint{1.950991in}{1.579828in}}%
\pgfpathlineto{\pgfqpoint{1.987850in}{1.598256in}}%
\pgfpathlineto{\pgfqpoint{1.949483in}{1.616617in}}%
\pgfpathlineto{\pgfqpoint{1.912622in}{1.598256in}}%
\pgfpathclose%
\pgfusepath{fill}%
\end{pgfscope}%
\begin{pgfscope}%
\pgfpathrectangle{\pgfqpoint{0.150000in}{0.150000in}}{\pgfqpoint{2.700000in}{1.950000in}}%
\pgfusepath{clip}%
\pgfsetbuttcap%
\pgfsetroundjoin%
\definecolor{currentfill}{rgb}{0.524219,0.582812,0.664844}%
\pgfsetfillcolor{currentfill}%
\pgfsetlinewidth{0.000000pt}%
\definecolor{currentstroke}{rgb}{0.000000,0.000000,0.000000}%
\pgfsetstrokecolor{currentstroke}%
\pgfsetdash{}{0pt}%
\pgfpathmoveto{\pgfqpoint{1.875627in}{1.579828in}}%
\pgfpathlineto{\pgfqpoint{1.912622in}{1.598256in}}%
\pgfpathlineto{\pgfqpoint{1.874393in}{1.616617in}}%
\pgfpathlineto{\pgfqpoint{1.837395in}{1.598256in}}%
\pgfpathclose%
\pgfusepath{fill}%
\end{pgfscope}%
\begin{pgfscope}%
\pgfpathrectangle{\pgfqpoint{0.150000in}{0.150000in}}{\pgfqpoint{2.700000in}{1.950000in}}%
\pgfusepath{clip}%
\pgfsetbuttcap%
\pgfsetroundjoin%
\definecolor{currentfill}{rgb}{0.524219,0.582812,0.664844}%
\pgfsetfillcolor{currentfill}%
\pgfsetlinewidth{0.000000pt}%
\definecolor{currentstroke}{rgb}{0.000000,0.000000,0.000000}%
\pgfsetstrokecolor{currentstroke}%
\pgfsetdash{}{0pt}%
\pgfpathmoveto{\pgfqpoint{1.800262in}{1.579828in}}%
\pgfpathlineto{\pgfqpoint{1.837395in}{1.598256in}}%
\pgfpathlineto{\pgfqpoint{1.799303in}{1.616617in}}%
\pgfpathlineto{\pgfqpoint{1.762168in}{1.598256in}}%
\pgfpathclose%
\pgfusepath{fill}%
\end{pgfscope}%
\begin{pgfscope}%
\pgfpathrectangle{\pgfqpoint{0.150000in}{0.150000in}}{\pgfqpoint{2.700000in}{1.950000in}}%
\pgfusepath{clip}%
\pgfsetbuttcap%
\pgfsetroundjoin%
\definecolor{currentfill}{rgb}{0.524219,0.582812,0.664844}%
\pgfsetfillcolor{currentfill}%
\pgfsetlinewidth{0.000000pt}%
\definecolor{currentstroke}{rgb}{0.000000,0.000000,0.000000}%
\pgfsetstrokecolor{currentstroke}%
\pgfsetdash{}{0pt}%
\pgfpathmoveto{\pgfqpoint{1.724898in}{1.579828in}}%
\pgfpathlineto{\pgfqpoint{1.762168in}{1.598256in}}%
\pgfpathlineto{\pgfqpoint{1.724212in}{1.616617in}}%
\pgfpathlineto{\pgfqpoint{1.686941in}{1.598256in}}%
\pgfpathclose%
\pgfusepath{fill}%
\end{pgfscope}%
\begin{pgfscope}%
\pgfpathrectangle{\pgfqpoint{0.150000in}{0.150000in}}{\pgfqpoint{2.700000in}{1.950000in}}%
\pgfusepath{clip}%
\pgfsetbuttcap%
\pgfsetroundjoin%
\definecolor{currentfill}{rgb}{0.524219,0.582812,0.664844}%
\pgfsetfillcolor{currentfill}%
\pgfsetlinewidth{0.000000pt}%
\definecolor{currentstroke}{rgb}{0.000000,0.000000,0.000000}%
\pgfsetstrokecolor{currentstroke}%
\pgfsetdash{}{0pt}%
\pgfpathmoveto{\pgfqpoint{1.649533in}{1.579828in}}%
\pgfpathlineto{\pgfqpoint{1.686941in}{1.598256in}}%
\pgfpathlineto{\pgfqpoint{1.649122in}{1.616617in}}%
\pgfpathlineto{\pgfqpoint{1.611714in}{1.598256in}}%
\pgfpathclose%
\pgfusepath{fill}%
\end{pgfscope}%
\begin{pgfscope}%
\pgfpathrectangle{\pgfqpoint{0.150000in}{0.150000in}}{\pgfqpoint{2.700000in}{1.950000in}}%
\pgfusepath{clip}%
\pgfsetbuttcap%
\pgfsetroundjoin%
\definecolor{currentfill}{rgb}{0.524219,0.582812,0.664844}%
\pgfsetfillcolor{currentfill}%
\pgfsetlinewidth{0.000000pt}%
\definecolor{currentstroke}{rgb}{0.000000,0.000000,0.000000}%
\pgfsetstrokecolor{currentstroke}%
\pgfsetdash{}{0pt}%
\pgfpathmoveto{\pgfqpoint{1.574169in}{1.579828in}}%
\pgfpathlineto{\pgfqpoint{1.611714in}{1.598256in}}%
\pgfpathlineto{\pgfqpoint{1.574032in}{1.616617in}}%
\pgfpathlineto{\pgfqpoint{1.536486in}{1.598256in}}%
\pgfpathclose%
\pgfusepath{fill}%
\end{pgfscope}%
\begin{pgfscope}%
\pgfpathrectangle{\pgfqpoint{0.150000in}{0.150000in}}{\pgfqpoint{2.700000in}{1.950000in}}%
\pgfusepath{clip}%
\pgfsetbuttcap%
\pgfsetroundjoin%
\definecolor{currentfill}{rgb}{0.524219,0.582812,0.664844}%
\pgfsetfillcolor{currentfill}%
\pgfsetlinewidth{0.000000pt}%
\definecolor{currentstroke}{rgb}{0.000000,0.000000,0.000000}%
\pgfsetstrokecolor{currentstroke}%
\pgfsetdash{}{0pt}%
\pgfpathmoveto{\pgfqpoint{1.498804in}{1.579828in}}%
\pgfpathlineto{\pgfqpoint{1.536486in}{1.598256in}}%
\pgfpathlineto{\pgfqpoint{1.498941in}{1.616617in}}%
\pgfpathlineto{\pgfqpoint{1.461259in}{1.598256in}}%
\pgfpathclose%
\pgfusepath{fill}%
\end{pgfscope}%
\begin{pgfscope}%
\pgfpathrectangle{\pgfqpoint{0.150000in}{0.150000in}}{\pgfqpoint{2.700000in}{1.950000in}}%
\pgfusepath{clip}%
\pgfsetbuttcap%
\pgfsetroundjoin%
\definecolor{currentfill}{rgb}{0.524219,0.582812,0.664844}%
\pgfsetfillcolor{currentfill}%
\pgfsetlinewidth{0.000000pt}%
\definecolor{currentstroke}{rgb}{0.000000,0.000000,0.000000}%
\pgfsetstrokecolor{currentstroke}%
\pgfsetdash{}{0pt}%
\pgfpathmoveto{\pgfqpoint{1.423440in}{1.579828in}}%
\pgfpathlineto{\pgfqpoint{1.461259in}{1.598256in}}%
\pgfpathlineto{\pgfqpoint{1.423851in}{1.616617in}}%
\pgfpathlineto{\pgfqpoint{1.386032in}{1.598256in}}%
\pgfpathclose%
\pgfusepath{fill}%
\end{pgfscope}%
\begin{pgfscope}%
\pgfpathrectangle{\pgfqpoint{0.150000in}{0.150000in}}{\pgfqpoint{2.700000in}{1.950000in}}%
\pgfusepath{clip}%
\pgfsetbuttcap%
\pgfsetroundjoin%
\definecolor{currentfill}{rgb}{0.524219,0.582812,0.664844}%
\pgfsetfillcolor{currentfill}%
\pgfsetlinewidth{0.000000pt}%
\definecolor{currentstroke}{rgb}{0.000000,0.000000,0.000000}%
\pgfsetstrokecolor{currentstroke}%
\pgfsetdash{}{0pt}%
\pgfpathmoveto{\pgfqpoint{1.348075in}{1.579828in}}%
\pgfpathlineto{\pgfqpoint{1.386032in}{1.598256in}}%
\pgfpathlineto{\pgfqpoint{1.348761in}{1.616617in}}%
\pgfpathlineto{\pgfqpoint{1.310805in}{1.598256in}}%
\pgfpathclose%
\pgfusepath{fill}%
\end{pgfscope}%
\begin{pgfscope}%
\pgfpathrectangle{\pgfqpoint{0.150000in}{0.150000in}}{\pgfqpoint{2.700000in}{1.950000in}}%
\pgfusepath{clip}%
\pgfsetbuttcap%
\pgfsetroundjoin%
\definecolor{currentfill}{rgb}{0.524219,0.582812,0.664844}%
\pgfsetfillcolor{currentfill}%
\pgfsetlinewidth{0.000000pt}%
\definecolor{currentstroke}{rgb}{0.000000,0.000000,0.000000}%
\pgfsetstrokecolor{currentstroke}%
\pgfsetdash{}{0pt}%
\pgfpathmoveto{\pgfqpoint{1.272711in}{1.579828in}}%
\pgfpathlineto{\pgfqpoint{1.310805in}{1.598256in}}%
\pgfpathlineto{\pgfqpoint{1.273670in}{1.616617in}}%
\pgfpathlineto{\pgfqpoint{1.235578in}{1.598256in}}%
\pgfpathclose%
\pgfusepath{fill}%
\end{pgfscope}%
\begin{pgfscope}%
\pgfpathrectangle{\pgfqpoint{0.150000in}{0.150000in}}{\pgfqpoint{2.700000in}{1.950000in}}%
\pgfusepath{clip}%
\pgfsetbuttcap%
\pgfsetroundjoin%
\definecolor{currentfill}{rgb}{0.524219,0.582812,0.664844}%
\pgfsetfillcolor{currentfill}%
\pgfsetlinewidth{0.000000pt}%
\definecolor{currentstroke}{rgb}{0.000000,0.000000,0.000000}%
\pgfsetstrokecolor{currentstroke}%
\pgfsetdash{}{0pt}%
\pgfpathmoveto{\pgfqpoint{1.197346in}{1.579828in}}%
\pgfpathlineto{\pgfqpoint{1.235578in}{1.598256in}}%
\pgfpathlineto{\pgfqpoint{1.198580in}{1.616617in}}%
\pgfpathlineto{\pgfqpoint{1.160351in}{1.598256in}}%
\pgfpathclose%
\pgfusepath{fill}%
\end{pgfscope}%
\begin{pgfscope}%
\pgfpathrectangle{\pgfqpoint{0.150000in}{0.150000in}}{\pgfqpoint{2.700000in}{1.950000in}}%
\pgfusepath{clip}%
\pgfsetbuttcap%
\pgfsetroundjoin%
\definecolor{currentfill}{rgb}{0.524219,0.582812,0.664844}%
\pgfsetfillcolor{currentfill}%
\pgfsetlinewidth{0.000000pt}%
\definecolor{currentstroke}{rgb}{0.000000,0.000000,0.000000}%
\pgfsetstrokecolor{currentstroke}%
\pgfsetdash{}{0pt}%
\pgfpathmoveto{\pgfqpoint{1.121982in}{1.579828in}}%
\pgfpathlineto{\pgfqpoint{1.160351in}{1.598256in}}%
\pgfpathlineto{\pgfqpoint{1.123490in}{1.616617in}}%
\pgfpathlineto{\pgfqpoint{1.085123in}{1.598256in}}%
\pgfpathclose%
\pgfusepath{fill}%
\end{pgfscope}%
\begin{pgfscope}%
\pgfpathrectangle{\pgfqpoint{0.150000in}{0.150000in}}{\pgfqpoint{2.700000in}{1.950000in}}%
\pgfusepath{clip}%
\pgfsetbuttcap%
\pgfsetroundjoin%
\definecolor{currentfill}{rgb}{0.524219,0.582812,0.664844}%
\pgfsetfillcolor{currentfill}%
\pgfsetlinewidth{0.000000pt}%
\definecolor{currentstroke}{rgb}{0.000000,0.000000,0.000000}%
\pgfsetstrokecolor{currentstroke}%
\pgfsetdash{}{0pt}%
\pgfpathmoveto{\pgfqpoint{1.046617in}{1.579828in}}%
\pgfpathlineto{\pgfqpoint{1.085123in}{1.598256in}}%
\pgfpathlineto{\pgfqpoint{1.048399in}{1.616617in}}%
\pgfpathlineto{\pgfqpoint{1.009896in}{1.598256in}}%
\pgfpathclose%
\pgfusepath{fill}%
\end{pgfscope}%
\begin{pgfscope}%
\pgfpathrectangle{\pgfqpoint{0.150000in}{0.150000in}}{\pgfqpoint{2.700000in}{1.950000in}}%
\pgfusepath{clip}%
\pgfsetbuttcap%
\pgfsetroundjoin%
\definecolor{currentfill}{rgb}{0.959574,0.964553,0.971523}%
\pgfsetfillcolor{currentfill}%
\pgfsetlinewidth{0.000000pt}%
\definecolor{currentstroke}{rgb}{0.000000,0.000000,0.000000}%
\pgfsetstrokecolor{currentstroke}%
\pgfsetdash{}{0pt}%
\pgfpathmoveto{\pgfqpoint{1.874550in}{0.854421in}}%
\pgfpathlineto{\pgfqpoint{1.911430in}{0.874108in}}%
\pgfpathlineto{\pgfqpoint{1.875531in}{0.961101in}}%
\pgfpathlineto{\pgfqpoint{1.838411in}{0.941479in}}%
\pgfpathclose%
\pgfusepath{fill}%
\end{pgfscope}%
\begin{pgfscope}%
\pgfpathrectangle{\pgfqpoint{0.150000in}{0.150000in}}{\pgfqpoint{2.700000in}{1.950000in}}%
\pgfusepath{clip}%
\pgfsetbuttcap%
\pgfsetroundjoin%
\definecolor{currentfill}{rgb}{0.586412,0.637347,0.708655}%
\pgfsetfillcolor{currentfill}%
\pgfsetlinewidth{0.000000pt}%
\definecolor{currentstroke}{rgb}{0.000000,0.000000,0.000000}%
\pgfsetstrokecolor{currentstroke}%
\pgfsetdash{}{0pt}%
\pgfpathmoveto{\pgfqpoint{2.062896in}{1.440870in}}%
\pgfpathlineto{\pgfqpoint{2.099471in}{1.459511in}}%
\pgfpathlineto{\pgfqpoint{2.063849in}{1.537417in}}%
\pgfpathlineto{\pgfqpoint{2.027075in}{1.518851in}}%
\pgfpathclose%
\pgfusepath{fill}%
\end{pgfscope}%
\begin{pgfscope}%
\pgfpathrectangle{\pgfqpoint{0.150000in}{0.150000in}}{\pgfqpoint{2.700000in}{1.950000in}}%
\pgfusepath{clip}%
\pgfsetbuttcap%
\pgfsetroundjoin%
\definecolor{currentfill}{rgb}{0.661045,0.702788,0.761229}%
\pgfsetfillcolor{currentfill}%
\pgfsetlinewidth{0.000000pt}%
\definecolor{currentstroke}{rgb}{0.000000,0.000000,0.000000}%
\pgfsetstrokecolor{currentstroke}%
\pgfsetdash{}{0pt}%
\pgfpathmoveto{\pgfqpoint{2.025303in}{1.325820in}}%
\pgfpathlineto{\pgfqpoint{2.061947in}{1.344671in}}%
\pgfpathlineto{\pgfqpoint{2.026188in}{1.422161in}}%
\pgfpathlineto{\pgfqpoint{1.989345in}{1.403384in}}%
\pgfpathclose%
\pgfusepath{fill}%
\end{pgfscope}%
\begin{pgfscope}%
\pgfpathrectangle{\pgfqpoint{0.150000in}{0.150000in}}{\pgfqpoint{2.700000in}{1.950000in}}%
\pgfusepath{clip}%
\pgfsetbuttcap%
\pgfsetroundjoin%
\definecolor{currentfill}{rgb}{0.729458,0.762776,0.809421}%
\pgfsetfillcolor{currentfill}%
\pgfsetlinewidth{0.000000pt}%
\definecolor{currentstroke}{rgb}{0.000000,0.000000,0.000000}%
\pgfsetstrokecolor{currentstroke}%
\pgfsetdash{}{0pt}%
\pgfpathmoveto{\pgfqpoint{1.987709in}{1.210767in}}%
\pgfpathlineto{\pgfqpoint{2.024422in}{1.229827in}}%
\pgfpathlineto{\pgfqpoint{1.988525in}{1.306901in}}%
\pgfpathlineto{\pgfqpoint{1.951613in}{1.287913in}}%
\pgfpathclose%
\pgfusepath{fill}%
\end{pgfscope}%
\begin{pgfscope}%
\pgfpathrectangle{\pgfqpoint{0.150000in}{0.150000in}}{\pgfqpoint{2.700000in}{1.950000in}}%
\pgfusepath{clip}%
\pgfsetbuttcap%
\pgfsetroundjoin%
\definecolor{currentfill}{rgb}{0.804090,0.828217,0.861994}%
\pgfsetfillcolor{currentfill}%
\pgfsetlinewidth{0.000000pt}%
\definecolor{currentstroke}{rgb}{0.000000,0.000000,0.000000}%
\pgfsetstrokecolor{currentstroke}%
\pgfsetdash{}{0pt}%
\pgfpathmoveto{\pgfqpoint{1.950113in}{1.095711in}}%
\pgfpathlineto{\pgfqpoint{1.986895in}{1.114980in}}%
\pgfpathlineto{\pgfqpoint{1.950862in}{1.191638in}}%
\pgfpathlineto{\pgfqpoint{1.913880in}{1.172438in}}%
\pgfpathclose%
\pgfusepath{fill}%
\end{pgfscope}%
\begin{pgfscope}%
\pgfpathrectangle{\pgfqpoint{0.150000in}{0.150000in}}{\pgfqpoint{2.700000in}{1.950000in}}%
\pgfusepath{clip}%
\pgfsetbuttcap%
\pgfsetroundjoin%
\definecolor{currentfill}{rgb}{0.878722,0.893658,0.914568}%
\pgfsetfillcolor{currentfill}%
\pgfsetlinewidth{0.000000pt}%
\definecolor{currentstroke}{rgb}{0.000000,0.000000,0.000000}%
\pgfsetstrokecolor{currentstroke}%
\pgfsetdash{}{0pt}%
\pgfpathmoveto{\pgfqpoint{1.912516in}{0.980651in}}%
\pgfpathlineto{\pgfqpoint{1.949367in}{1.000130in}}%
\pgfpathlineto{\pgfqpoint{1.913197in}{1.076371in}}%
\pgfpathlineto{\pgfqpoint{1.876146in}{1.056961in}}%
\pgfpathclose%
\pgfusepath{fill}%
\end{pgfscope}%
\begin{pgfscope}%
\pgfpathrectangle{\pgfqpoint{0.150000in}{0.150000in}}{\pgfqpoint{2.700000in}{1.950000in}}%
\pgfusepath{clip}%
\pgfsetbuttcap%
\pgfsetroundjoin%
\definecolor{currentfill}{rgb}{0.536657,0.593719,0.673606}%
\pgfsetfillcolor{currentfill}%
\pgfsetlinewidth{0.000000pt}%
\definecolor{currentstroke}{rgb}{0.000000,0.000000,0.000000}%
\pgfsetstrokecolor{currentstroke}%
\pgfsetdash{}{0pt}%
\pgfpathmoveto{\pgfqpoint{2.063849in}{1.537417in}}%
\pgfpathlineto{\pgfqpoint{2.100488in}{1.555916in}}%
\pgfpathlineto{\pgfqpoint{2.063077in}{1.598256in}}%
\pgfpathlineto{\pgfqpoint{2.026356in}{1.579828in}}%
\pgfpathclose%
\pgfusepath{fill}%
\end{pgfscope}%
\begin{pgfscope}%
\pgfpathrectangle{\pgfqpoint{0.150000in}{0.150000in}}{\pgfqpoint{2.700000in}{1.950000in}}%
\pgfusepath{clip}%
\pgfsetbuttcap%
\pgfsetroundjoin%
\definecolor{currentfill}{rgb}{0.979105,0.962086,0.963434}%
\pgfsetfillcolor{currentfill}%
\pgfsetlinewidth{0.000000pt}%
\definecolor{currentstroke}{rgb}{0.000000,0.000000,0.000000}%
\pgfsetstrokecolor{currentstroke}%
\pgfsetdash{}{0pt}%
\pgfpathmoveto{\pgfqpoint{1.799908in}{0.719510in}}%
\pgfpathlineto{\pgfqpoint{1.836992in}{0.739479in}}%
\pgfpathlineto{\pgfqpoint{1.800385in}{0.814830in}}%
\pgfpathlineto{\pgfqpoint{1.763099in}{0.794926in}}%
\pgfpathclose%
\pgfusepath{fill}%
\end{pgfscope}%
\begin{pgfscope}%
\pgfpathrectangle{\pgfqpoint{0.150000in}{0.150000in}}{\pgfqpoint{2.700000in}{1.950000in}}%
\pgfusepath{clip}%
\pgfsetbuttcap%
\pgfsetroundjoin%
\definecolor{currentfill}{rgb}{0.933517,0.879366,0.883655}%
\pgfsetfillcolor{currentfill}%
\pgfsetlinewidth{0.000000pt}%
\definecolor{currentstroke}{rgb}{0.000000,0.000000,0.000000}%
\pgfsetstrokecolor{currentstroke}%
\pgfsetdash{}{0pt}%
\pgfpathmoveto{\pgfqpoint{1.762280in}{0.604354in}}%
\pgfpathlineto{\pgfqpoint{1.799433in}{0.624533in}}%
\pgfpathlineto{\pgfqpoint{1.762689in}{0.699468in}}%
\pgfpathlineto{\pgfqpoint{1.725334in}{0.679352in}}%
\pgfpathclose%
\pgfusepath{fill}%
\end{pgfscope}%
\begin{pgfscope}%
\pgfpathrectangle{\pgfqpoint{0.150000in}{0.150000in}}{\pgfqpoint{2.700000in}{1.950000in}}%
\pgfusepath{clip}%
\pgfsetbuttcap%
\pgfsetroundjoin%
\definecolor{currentfill}{rgb}{0.887929,0.796645,0.803876}%
\pgfsetfillcolor{currentfill}%
\pgfsetlinewidth{0.000000pt}%
\definecolor{currentstroke}{rgb}{0.000000,0.000000,0.000000}%
\pgfsetstrokecolor{currentstroke}%
\pgfsetdash{}{0pt}%
\pgfpathmoveto{\pgfqpoint{1.724651in}{0.489195in}}%
\pgfpathlineto{\pgfqpoint{1.761873in}{0.509584in}}%
\pgfpathlineto{\pgfqpoint{1.724991in}{0.584102in}}%
\pgfpathlineto{\pgfqpoint{1.687566in}{0.563775in}}%
\pgfpathclose%
\pgfusepath{fill}%
\end{pgfscope}%
\begin{pgfscope}%
\pgfpathrectangle{\pgfqpoint{0.150000in}{0.150000in}}{\pgfqpoint{2.700000in}{1.950000in}}%
\pgfusepath{clip}%
\pgfsetbuttcap%
\pgfsetroundjoin%
\definecolor{currentfill}{rgb}{0.846140,0.720818,0.730744}%
\pgfsetfillcolor{currentfill}%
\pgfsetlinewidth{0.000000pt}%
\definecolor{currentstroke}{rgb}{0.000000,0.000000,0.000000}%
\pgfsetstrokecolor{currentstroke}%
\pgfsetdash{}{0pt}%
\pgfpathmoveto{\pgfqpoint{1.687020in}{0.374033in}}%
\pgfpathlineto{\pgfqpoint{1.724311in}{0.394632in}}%
\pgfpathlineto{\pgfqpoint{1.687293in}{0.468732in}}%
\pgfpathlineto{\pgfqpoint{1.649798in}{0.448194in}}%
\pgfpathclose%
\pgfusepath{fill}%
\end{pgfscope}%
\begin{pgfscope}%
\pgfpathrectangle{\pgfqpoint{0.150000in}{0.150000in}}{\pgfqpoint{2.700000in}{1.950000in}}%
\pgfusepath{clip}%
\pgfsetbuttcap%
\pgfsetroundjoin%
\definecolor{currentfill}{rgb}{0.524219,0.582812,0.664844}%
\pgfsetfillcolor{currentfill}%
\pgfsetlinewidth{0.000000pt}%
\definecolor{currentstroke}{rgb}{0.000000,0.000000,0.000000}%
\pgfsetstrokecolor{currentstroke}%
\pgfsetdash{}{0pt}%
\pgfpathmoveto{\pgfqpoint{1.989501in}{1.561332in}}%
\pgfpathlineto{\pgfqpoint{2.026356in}{1.579828in}}%
\pgfpathlineto{\pgfqpoint{1.987850in}{1.598256in}}%
\pgfpathlineto{\pgfqpoint{1.950991in}{1.579828in}}%
\pgfpathclose%
\pgfusepath{fill}%
\end{pgfscope}%
\begin{pgfscope}%
\pgfpathrectangle{\pgfqpoint{0.150000in}{0.150000in}}{\pgfqpoint{2.700000in}{1.950000in}}%
\pgfusepath{clip}%
\pgfsetbuttcap%
\pgfsetroundjoin%
\definecolor{currentfill}{rgb}{0.524219,0.582812,0.664844}%
\pgfsetfillcolor{currentfill}%
\pgfsetlinewidth{0.000000pt}%
\definecolor{currentstroke}{rgb}{0.000000,0.000000,0.000000}%
\pgfsetstrokecolor{currentstroke}%
\pgfsetdash{}{0pt}%
\pgfpathmoveto{\pgfqpoint{1.913998in}{1.561332in}}%
\pgfpathlineto{\pgfqpoint{1.950991in}{1.579828in}}%
\pgfpathlineto{\pgfqpoint{1.912622in}{1.598256in}}%
\pgfpathlineto{\pgfqpoint{1.875627in}{1.579828in}}%
\pgfpathclose%
\pgfusepath{fill}%
\end{pgfscope}%
\begin{pgfscope}%
\pgfpathrectangle{\pgfqpoint{0.150000in}{0.150000in}}{\pgfqpoint{2.700000in}{1.950000in}}%
\pgfusepath{clip}%
\pgfsetbuttcap%
\pgfsetroundjoin%
\definecolor{currentfill}{rgb}{0.524219,0.582812,0.664844}%
\pgfsetfillcolor{currentfill}%
\pgfsetlinewidth{0.000000pt}%
\definecolor{currentstroke}{rgb}{0.000000,0.000000,0.000000}%
\pgfsetstrokecolor{currentstroke}%
\pgfsetdash{}{0pt}%
\pgfpathmoveto{\pgfqpoint{1.838496in}{1.561332in}}%
\pgfpathlineto{\pgfqpoint{1.875627in}{1.579828in}}%
\pgfpathlineto{\pgfqpoint{1.837395in}{1.598256in}}%
\pgfpathlineto{\pgfqpoint{1.800262in}{1.579828in}}%
\pgfpathclose%
\pgfusepath{fill}%
\end{pgfscope}%
\begin{pgfscope}%
\pgfpathrectangle{\pgfqpoint{0.150000in}{0.150000in}}{\pgfqpoint{2.700000in}{1.950000in}}%
\pgfusepath{clip}%
\pgfsetbuttcap%
\pgfsetroundjoin%
\definecolor{currentfill}{rgb}{0.524219,0.582812,0.664844}%
\pgfsetfillcolor{currentfill}%
\pgfsetlinewidth{0.000000pt}%
\definecolor{currentstroke}{rgb}{0.000000,0.000000,0.000000}%
\pgfsetstrokecolor{currentstroke}%
\pgfsetdash{}{0pt}%
\pgfpathmoveto{\pgfqpoint{1.762994in}{1.561332in}}%
\pgfpathlineto{\pgfqpoint{1.800262in}{1.579828in}}%
\pgfpathlineto{\pgfqpoint{1.762168in}{1.598256in}}%
\pgfpathlineto{\pgfqpoint{1.724898in}{1.579828in}}%
\pgfpathclose%
\pgfusepath{fill}%
\end{pgfscope}%
\begin{pgfscope}%
\pgfpathrectangle{\pgfqpoint{0.150000in}{0.150000in}}{\pgfqpoint{2.700000in}{1.950000in}}%
\pgfusepath{clip}%
\pgfsetbuttcap%
\pgfsetroundjoin%
\definecolor{currentfill}{rgb}{0.524219,0.582812,0.664844}%
\pgfsetfillcolor{currentfill}%
\pgfsetlinewidth{0.000000pt}%
\definecolor{currentstroke}{rgb}{0.000000,0.000000,0.000000}%
\pgfsetstrokecolor{currentstroke}%
\pgfsetdash{}{0pt}%
\pgfpathmoveto{\pgfqpoint{1.687491in}{1.561332in}}%
\pgfpathlineto{\pgfqpoint{1.724898in}{1.579828in}}%
\pgfpathlineto{\pgfqpoint{1.686941in}{1.598256in}}%
\pgfpathlineto{\pgfqpoint{1.649533in}{1.579828in}}%
\pgfpathclose%
\pgfusepath{fill}%
\end{pgfscope}%
\begin{pgfscope}%
\pgfpathrectangle{\pgfqpoint{0.150000in}{0.150000in}}{\pgfqpoint{2.700000in}{1.950000in}}%
\pgfusepath{clip}%
\pgfsetbuttcap%
\pgfsetroundjoin%
\definecolor{currentfill}{rgb}{0.524219,0.582812,0.664844}%
\pgfsetfillcolor{currentfill}%
\pgfsetlinewidth{0.000000pt}%
\definecolor{currentstroke}{rgb}{0.000000,0.000000,0.000000}%
\pgfsetstrokecolor{currentstroke}%
\pgfsetdash{}{0pt}%
\pgfpathmoveto{\pgfqpoint{1.611989in}{1.561332in}}%
\pgfpathlineto{\pgfqpoint{1.649533in}{1.579828in}}%
\pgfpathlineto{\pgfqpoint{1.611714in}{1.598256in}}%
\pgfpathlineto{\pgfqpoint{1.574169in}{1.579828in}}%
\pgfpathclose%
\pgfusepath{fill}%
\end{pgfscope}%
\begin{pgfscope}%
\pgfpathrectangle{\pgfqpoint{0.150000in}{0.150000in}}{\pgfqpoint{2.700000in}{1.950000in}}%
\pgfusepath{clip}%
\pgfsetbuttcap%
\pgfsetroundjoin%
\definecolor{currentfill}{rgb}{0.524219,0.582812,0.664844}%
\pgfsetfillcolor{currentfill}%
\pgfsetlinewidth{0.000000pt}%
\definecolor{currentstroke}{rgb}{0.000000,0.000000,0.000000}%
\pgfsetstrokecolor{currentstroke}%
\pgfsetdash{}{0pt}%
\pgfpathmoveto{\pgfqpoint{1.536486in}{1.561332in}}%
\pgfpathlineto{\pgfqpoint{1.574169in}{1.579828in}}%
\pgfpathlineto{\pgfqpoint{1.536486in}{1.598256in}}%
\pgfpathlineto{\pgfqpoint{1.498804in}{1.579828in}}%
\pgfpathclose%
\pgfusepath{fill}%
\end{pgfscope}%
\begin{pgfscope}%
\pgfpathrectangle{\pgfqpoint{0.150000in}{0.150000in}}{\pgfqpoint{2.700000in}{1.950000in}}%
\pgfusepath{clip}%
\pgfsetbuttcap%
\pgfsetroundjoin%
\definecolor{currentfill}{rgb}{0.524219,0.582812,0.664844}%
\pgfsetfillcolor{currentfill}%
\pgfsetlinewidth{0.000000pt}%
\definecolor{currentstroke}{rgb}{0.000000,0.000000,0.000000}%
\pgfsetstrokecolor{currentstroke}%
\pgfsetdash{}{0pt}%
\pgfpathmoveto{\pgfqpoint{1.460984in}{1.561332in}}%
\pgfpathlineto{\pgfqpoint{1.498804in}{1.579828in}}%
\pgfpathlineto{\pgfqpoint{1.461259in}{1.598256in}}%
\pgfpathlineto{\pgfqpoint{1.423440in}{1.579828in}}%
\pgfpathclose%
\pgfusepath{fill}%
\end{pgfscope}%
\begin{pgfscope}%
\pgfpathrectangle{\pgfqpoint{0.150000in}{0.150000in}}{\pgfqpoint{2.700000in}{1.950000in}}%
\pgfusepath{clip}%
\pgfsetbuttcap%
\pgfsetroundjoin%
\definecolor{currentfill}{rgb}{0.524219,0.582812,0.664844}%
\pgfsetfillcolor{currentfill}%
\pgfsetlinewidth{0.000000pt}%
\definecolor{currentstroke}{rgb}{0.000000,0.000000,0.000000}%
\pgfsetstrokecolor{currentstroke}%
\pgfsetdash{}{0pt}%
\pgfpathmoveto{\pgfqpoint{1.385482in}{1.561332in}}%
\pgfpathlineto{\pgfqpoint{1.423440in}{1.579828in}}%
\pgfpathlineto{\pgfqpoint{1.386032in}{1.598256in}}%
\pgfpathlineto{\pgfqpoint{1.348075in}{1.579828in}}%
\pgfpathclose%
\pgfusepath{fill}%
\end{pgfscope}%
\begin{pgfscope}%
\pgfpathrectangle{\pgfqpoint{0.150000in}{0.150000in}}{\pgfqpoint{2.700000in}{1.950000in}}%
\pgfusepath{clip}%
\pgfsetbuttcap%
\pgfsetroundjoin%
\definecolor{currentfill}{rgb}{0.524219,0.582812,0.664844}%
\pgfsetfillcolor{currentfill}%
\pgfsetlinewidth{0.000000pt}%
\definecolor{currentstroke}{rgb}{0.000000,0.000000,0.000000}%
\pgfsetstrokecolor{currentstroke}%
\pgfsetdash{}{0pt}%
\pgfpathmoveto{\pgfqpoint{1.309979in}{1.561332in}}%
\pgfpathlineto{\pgfqpoint{1.348075in}{1.579828in}}%
\pgfpathlineto{\pgfqpoint{1.310805in}{1.598256in}}%
\pgfpathlineto{\pgfqpoint{1.272711in}{1.579828in}}%
\pgfpathclose%
\pgfusepath{fill}%
\end{pgfscope}%
\begin{pgfscope}%
\pgfpathrectangle{\pgfqpoint{0.150000in}{0.150000in}}{\pgfqpoint{2.700000in}{1.950000in}}%
\pgfusepath{clip}%
\pgfsetbuttcap%
\pgfsetroundjoin%
\definecolor{currentfill}{rgb}{0.524219,0.582812,0.664844}%
\pgfsetfillcolor{currentfill}%
\pgfsetlinewidth{0.000000pt}%
\definecolor{currentstroke}{rgb}{0.000000,0.000000,0.000000}%
\pgfsetstrokecolor{currentstroke}%
\pgfsetdash{}{0pt}%
\pgfpathmoveto{\pgfqpoint{1.234477in}{1.561332in}}%
\pgfpathlineto{\pgfqpoint{1.272711in}{1.579828in}}%
\pgfpathlineto{\pgfqpoint{1.235578in}{1.598256in}}%
\pgfpathlineto{\pgfqpoint{1.197346in}{1.579828in}}%
\pgfpathclose%
\pgfusepath{fill}%
\end{pgfscope}%
\begin{pgfscope}%
\pgfpathrectangle{\pgfqpoint{0.150000in}{0.150000in}}{\pgfqpoint{2.700000in}{1.950000in}}%
\pgfusepath{clip}%
\pgfsetbuttcap%
\pgfsetroundjoin%
\definecolor{currentfill}{rgb}{0.524219,0.582812,0.664844}%
\pgfsetfillcolor{currentfill}%
\pgfsetlinewidth{0.000000pt}%
\definecolor{currentstroke}{rgb}{0.000000,0.000000,0.000000}%
\pgfsetstrokecolor{currentstroke}%
\pgfsetdash{}{0pt}%
\pgfpathmoveto{\pgfqpoint{1.158975in}{1.561332in}}%
\pgfpathlineto{\pgfqpoint{1.197346in}{1.579828in}}%
\pgfpathlineto{\pgfqpoint{1.160351in}{1.598256in}}%
\pgfpathlineto{\pgfqpoint{1.121982in}{1.579828in}}%
\pgfpathclose%
\pgfusepath{fill}%
\end{pgfscope}%
\begin{pgfscope}%
\pgfpathrectangle{\pgfqpoint{0.150000in}{0.150000in}}{\pgfqpoint{2.700000in}{1.950000in}}%
\pgfusepath{clip}%
\pgfsetbuttcap%
\pgfsetroundjoin%
\definecolor{currentfill}{rgb}{0.524219,0.582812,0.664844}%
\pgfsetfillcolor{currentfill}%
\pgfsetlinewidth{0.000000pt}%
\definecolor{currentstroke}{rgb}{0.000000,0.000000,0.000000}%
\pgfsetstrokecolor{currentstroke}%
\pgfsetdash{}{0pt}%
\pgfpathmoveto{\pgfqpoint{1.083472in}{1.561332in}}%
\pgfpathlineto{\pgfqpoint{1.121982in}{1.579828in}}%
\pgfpathlineto{\pgfqpoint{1.085123in}{1.598256in}}%
\pgfpathlineto{\pgfqpoint{1.046617in}{1.579828in}}%
\pgfpathclose%
\pgfusepath{fill}%
\end{pgfscope}%
\begin{pgfscope}%
\pgfpathrectangle{\pgfqpoint{0.150000in}{0.150000in}}{\pgfqpoint{2.700000in}{1.950000in}}%
\pgfusepath{clip}%
\pgfsetbuttcap%
\pgfsetroundjoin%
\definecolor{currentfill}{rgb}{0.524219,0.582812,0.664844}%
\pgfsetfillcolor{currentfill}%
\pgfsetlinewidth{0.000000pt}%
\definecolor{currentstroke}{rgb}{0.000000,0.000000,0.000000}%
\pgfsetstrokecolor{currentstroke}%
\pgfsetdash{}{0pt}%
\pgfpathmoveto{\pgfqpoint{1.007970in}{1.561332in}}%
\pgfpathlineto{\pgfqpoint{1.046617in}{1.579828in}}%
\pgfpathlineto{\pgfqpoint{1.009896in}{1.598256in}}%
\pgfpathlineto{\pgfqpoint{0.971253in}{1.579828in}}%
\pgfpathclose%
\pgfusepath{fill}%
\end{pgfscope}%
\begin{pgfscope}%
\pgfpathrectangle{\pgfqpoint{0.150000in}{0.150000in}}{\pgfqpoint{2.700000in}{1.950000in}}%
\pgfusepath{clip}%
\pgfsetbuttcap%
\pgfsetroundjoin%
\definecolor{currentfill}{rgb}{0.959574,0.964553,0.971523}%
\pgfsetfillcolor{currentfill}%
\pgfsetlinewidth{0.000000pt}%
\definecolor{currentstroke}{rgb}{0.000000,0.000000,0.000000}%
\pgfsetstrokecolor{currentstroke}%
\pgfsetdash{}{0pt}%
\pgfpathmoveto{\pgfqpoint{1.837535in}{0.834662in}}%
\pgfpathlineto{\pgfqpoint{1.874550in}{0.854421in}}%
\pgfpathlineto{\pgfqpoint{1.838411in}{0.941479in}}%
\pgfpathlineto{\pgfqpoint{1.801154in}{0.921786in}}%
\pgfpathclose%
\pgfusepath{fill}%
\end{pgfscope}%
\begin{pgfscope}%
\pgfpathrectangle{\pgfqpoint{0.150000in}{0.150000in}}{\pgfqpoint{2.700000in}{1.950000in}}%
\pgfusepath{clip}%
\pgfsetbuttcap%
\pgfsetroundjoin%
\definecolor{currentfill}{rgb}{0.586412,0.637347,0.708655}%
\pgfsetfillcolor{currentfill}%
\pgfsetlinewidth{0.000000pt}%
\definecolor{currentstroke}{rgb}{0.000000,0.000000,0.000000}%
\pgfsetstrokecolor{currentstroke}%
\pgfsetdash{}{0pt}%
\pgfpathmoveto{\pgfqpoint{2.026188in}{1.422161in}}%
\pgfpathlineto{\pgfqpoint{2.062896in}{1.440870in}}%
\pgfpathlineto{\pgfqpoint{2.027075in}{1.518851in}}%
\pgfpathlineto{\pgfqpoint{1.990167in}{1.500217in}}%
\pgfpathclose%
\pgfusepath{fill}%
\end{pgfscope}%
\begin{pgfscope}%
\pgfpathrectangle{\pgfqpoint{0.150000in}{0.150000in}}{\pgfqpoint{2.700000in}{1.950000in}}%
\pgfusepath{clip}%
\pgfsetbuttcap%
\pgfsetroundjoin%
\definecolor{currentfill}{rgb}{0.661045,0.702788,0.761229}%
\pgfsetfillcolor{currentfill}%
\pgfsetlinewidth{0.000000pt}%
\definecolor{currentstroke}{rgb}{0.000000,0.000000,0.000000}%
\pgfsetstrokecolor{currentstroke}%
\pgfsetdash{}{0pt}%
\pgfpathmoveto{\pgfqpoint{1.988525in}{1.306901in}}%
\pgfpathlineto{\pgfqpoint{2.025303in}{1.325820in}}%
\pgfpathlineto{\pgfqpoint{1.989345in}{1.403384in}}%
\pgfpathlineto{\pgfqpoint{1.952367in}{1.384538in}}%
\pgfpathclose%
\pgfusepath{fill}%
\end{pgfscope}%
\begin{pgfscope}%
\pgfpathrectangle{\pgfqpoint{0.150000in}{0.150000in}}{\pgfqpoint{2.700000in}{1.950000in}}%
\pgfusepath{clip}%
\pgfsetbuttcap%
\pgfsetroundjoin%
\definecolor{currentfill}{rgb}{0.729458,0.762776,0.809421}%
\pgfsetfillcolor{currentfill}%
\pgfsetlinewidth{0.000000pt}%
\definecolor{currentstroke}{rgb}{0.000000,0.000000,0.000000}%
\pgfsetstrokecolor{currentstroke}%
\pgfsetdash{}{0pt}%
\pgfpathmoveto{\pgfqpoint{1.950862in}{1.191638in}}%
\pgfpathlineto{\pgfqpoint{1.987709in}{1.210767in}}%
\pgfpathlineto{\pgfqpoint{1.951613in}{1.287913in}}%
\pgfpathlineto{\pgfqpoint{1.914566in}{1.268855in}}%
\pgfpathclose%
\pgfusepath{fill}%
\end{pgfscope}%
\begin{pgfscope}%
\pgfpathrectangle{\pgfqpoint{0.150000in}{0.150000in}}{\pgfqpoint{2.700000in}{1.950000in}}%
\pgfusepath{clip}%
\pgfsetbuttcap%
\pgfsetroundjoin%
\definecolor{currentfill}{rgb}{0.804090,0.828217,0.861994}%
\pgfsetfillcolor{currentfill}%
\pgfsetlinewidth{0.000000pt}%
\definecolor{currentstroke}{rgb}{0.000000,0.000000,0.000000}%
\pgfsetstrokecolor{currentstroke}%
\pgfsetdash{}{0pt}%
\pgfpathmoveto{\pgfqpoint{1.913197in}{1.076371in}}%
\pgfpathlineto{\pgfqpoint{1.950113in}{1.095711in}}%
\pgfpathlineto{\pgfqpoint{1.913880in}{1.172438in}}%
\pgfpathlineto{\pgfqpoint{1.876763in}{1.153169in}}%
\pgfpathclose%
\pgfusepath{fill}%
\end{pgfscope}%
\begin{pgfscope}%
\pgfpathrectangle{\pgfqpoint{0.150000in}{0.150000in}}{\pgfqpoint{2.700000in}{1.950000in}}%
\pgfusepath{clip}%
\pgfsetbuttcap%
\pgfsetroundjoin%
\definecolor{currentfill}{rgb}{0.878722,0.893658,0.914568}%
\pgfsetfillcolor{currentfill}%
\pgfsetlinewidth{0.000000pt}%
\definecolor{currentstroke}{rgb}{0.000000,0.000000,0.000000}%
\pgfsetstrokecolor{currentstroke}%
\pgfsetdash{}{0pt}%
\pgfpathmoveto{\pgfqpoint{1.875531in}{0.961101in}}%
\pgfpathlineto{\pgfqpoint{1.912516in}{0.980651in}}%
\pgfpathlineto{\pgfqpoint{1.876146in}{1.056961in}}%
\pgfpathlineto{\pgfqpoint{1.838959in}{1.037479in}}%
\pgfpathclose%
\pgfusepath{fill}%
\end{pgfscope}%
\begin{pgfscope}%
\pgfpathrectangle{\pgfqpoint{0.150000in}{0.150000in}}{\pgfqpoint{2.700000in}{1.950000in}}%
\pgfusepath{clip}%
\pgfsetbuttcap%
\pgfsetroundjoin%
\definecolor{currentfill}{rgb}{0.536657,0.593719,0.673606}%
\pgfsetfillcolor{currentfill}%
\pgfsetlinewidth{0.000000pt}%
\definecolor{currentstroke}{rgb}{0.000000,0.000000,0.000000}%
\pgfsetstrokecolor{currentstroke}%
\pgfsetdash{}{0pt}%
\pgfpathmoveto{\pgfqpoint{2.027075in}{1.518851in}}%
\pgfpathlineto{\pgfqpoint{2.063849in}{1.537417in}}%
\pgfpathlineto{\pgfqpoint{2.026356in}{1.579828in}}%
\pgfpathlineto{\pgfqpoint{1.989501in}{1.561332in}}%
\pgfpathclose%
\pgfusepath{fill}%
\end{pgfscope}%
\begin{pgfscope}%
\pgfpathrectangle{\pgfqpoint{0.150000in}{0.150000in}}{\pgfqpoint{2.700000in}{1.950000in}}%
\pgfusepath{clip}%
\pgfsetbuttcap%
\pgfsetroundjoin%
\definecolor{currentfill}{rgb}{0.979105,0.962086,0.963434}%
\pgfsetfillcolor{currentfill}%
\pgfsetlinewidth{0.000000pt}%
\definecolor{currentstroke}{rgb}{0.000000,0.000000,0.000000}%
\pgfsetstrokecolor{currentstroke}%
\pgfsetdash{}{0pt}%
\pgfpathmoveto{\pgfqpoint{1.762689in}{0.699468in}}%
\pgfpathlineto{\pgfqpoint{1.799908in}{0.719510in}}%
\pgfpathlineto{\pgfqpoint{1.763099in}{0.794926in}}%
\pgfpathlineto{\pgfqpoint{1.725677in}{0.774949in}}%
\pgfpathclose%
\pgfusepath{fill}%
\end{pgfscope}%
\begin{pgfscope}%
\pgfpathrectangle{\pgfqpoint{0.150000in}{0.150000in}}{\pgfqpoint{2.700000in}{1.950000in}}%
\pgfusepath{clip}%
\pgfsetbuttcap%
\pgfsetroundjoin%
\definecolor{currentfill}{rgb}{0.933517,0.879366,0.883655}%
\pgfsetfillcolor{currentfill}%
\pgfsetlinewidth{0.000000pt}%
\definecolor{currentstroke}{rgb}{0.000000,0.000000,0.000000}%
\pgfsetstrokecolor{currentstroke}%
\pgfsetdash{}{0pt}%
\pgfpathmoveto{\pgfqpoint{1.724991in}{0.584102in}}%
\pgfpathlineto{\pgfqpoint{1.762280in}{0.604354in}}%
\pgfpathlineto{\pgfqpoint{1.725334in}{0.679352in}}%
\pgfpathlineto{\pgfqpoint{1.687841in}{0.659163in}}%
\pgfpathclose%
\pgfusepath{fill}%
\end{pgfscope}%
\begin{pgfscope}%
\pgfpathrectangle{\pgfqpoint{0.150000in}{0.150000in}}{\pgfqpoint{2.700000in}{1.950000in}}%
\pgfusepath{clip}%
\pgfsetbuttcap%
\pgfsetroundjoin%
\definecolor{currentfill}{rgb}{0.887929,0.796645,0.803876}%
\pgfsetfillcolor{currentfill}%
\pgfsetlinewidth{0.000000pt}%
\definecolor{currentstroke}{rgb}{0.000000,0.000000,0.000000}%
\pgfsetstrokecolor{currentstroke}%
\pgfsetdash{}{0pt}%
\pgfpathmoveto{\pgfqpoint{1.687293in}{0.468732in}}%
\pgfpathlineto{\pgfqpoint{1.724651in}{0.489195in}}%
\pgfpathlineto{\pgfqpoint{1.687566in}{0.563775in}}%
\pgfpathlineto{\pgfqpoint{1.650004in}{0.543374in}}%
\pgfpathclose%
\pgfusepath{fill}%
\end{pgfscope}%
\begin{pgfscope}%
\pgfpathrectangle{\pgfqpoint{0.150000in}{0.150000in}}{\pgfqpoint{2.700000in}{1.950000in}}%
\pgfusepath{clip}%
\pgfsetbuttcap%
\pgfsetroundjoin%
\definecolor{currentfill}{rgb}{0.846140,0.720818,0.730744}%
\pgfsetfillcolor{currentfill}%
\pgfsetlinewidth{0.000000pt}%
\definecolor{currentstroke}{rgb}{0.000000,0.000000,0.000000}%
\pgfsetstrokecolor{currentstroke}%
\pgfsetdash{}{0pt}%
\pgfpathmoveto{\pgfqpoint{1.649593in}{0.353359in}}%
\pgfpathlineto{\pgfqpoint{1.687020in}{0.374033in}}%
\pgfpathlineto{\pgfqpoint{1.649798in}{0.448194in}}%
\pgfpathlineto{\pgfqpoint{1.612166in}{0.427580in}}%
\pgfpathclose%
\pgfusepath{fill}%
\end{pgfscope}%
\begin{pgfscope}%
\pgfpathrectangle{\pgfqpoint{0.150000in}{0.150000in}}{\pgfqpoint{2.700000in}{1.950000in}}%
\pgfusepath{clip}%
\pgfsetbuttcap%
\pgfsetroundjoin%
\definecolor{currentfill}{rgb}{0.524219,0.582812,0.664844}%
\pgfsetfillcolor{currentfill}%
\pgfsetlinewidth{0.000000pt}%
\definecolor{currentstroke}{rgb}{0.000000,0.000000,0.000000}%
\pgfsetstrokecolor{currentstroke}%
\pgfsetdash{}{0pt}%
\pgfpathmoveto{\pgfqpoint{1.952510in}{1.542769in}}%
\pgfpathlineto{\pgfqpoint{1.989501in}{1.561332in}}%
\pgfpathlineto{\pgfqpoint{1.950991in}{1.579828in}}%
\pgfpathlineto{\pgfqpoint{1.913998in}{1.561332in}}%
\pgfpathclose%
\pgfusepath{fill}%
\end{pgfscope}%
\begin{pgfscope}%
\pgfpathrectangle{\pgfqpoint{0.150000in}{0.150000in}}{\pgfqpoint{2.700000in}{1.950000in}}%
\pgfusepath{clip}%
\pgfsetbuttcap%
\pgfsetroundjoin%
\definecolor{currentfill}{rgb}{0.524219,0.582812,0.664844}%
\pgfsetfillcolor{currentfill}%
\pgfsetlinewidth{0.000000pt}%
\definecolor{currentstroke}{rgb}{0.000000,0.000000,0.000000}%
\pgfsetstrokecolor{currentstroke}%
\pgfsetdash{}{0pt}%
\pgfpathmoveto{\pgfqpoint{1.876870in}{1.542769in}}%
\pgfpathlineto{\pgfqpoint{1.913998in}{1.561332in}}%
\pgfpathlineto{\pgfqpoint{1.875627in}{1.579828in}}%
\pgfpathlineto{\pgfqpoint{1.838496in}{1.561332in}}%
\pgfpathclose%
\pgfusepath{fill}%
\end{pgfscope}%
\begin{pgfscope}%
\pgfpathrectangle{\pgfqpoint{0.150000in}{0.150000in}}{\pgfqpoint{2.700000in}{1.950000in}}%
\pgfusepath{clip}%
\pgfsetbuttcap%
\pgfsetroundjoin%
\definecolor{currentfill}{rgb}{0.524219,0.582812,0.664844}%
\pgfsetfillcolor{currentfill}%
\pgfsetlinewidth{0.000000pt}%
\definecolor{currentstroke}{rgb}{0.000000,0.000000,0.000000}%
\pgfsetstrokecolor{currentstroke}%
\pgfsetdash{}{0pt}%
\pgfpathmoveto{\pgfqpoint{1.801229in}{1.542769in}}%
\pgfpathlineto{\pgfqpoint{1.838496in}{1.561332in}}%
\pgfpathlineto{\pgfqpoint{1.800262in}{1.579828in}}%
\pgfpathlineto{\pgfqpoint{1.762994in}{1.561332in}}%
\pgfpathclose%
\pgfusepath{fill}%
\end{pgfscope}%
\begin{pgfscope}%
\pgfpathrectangle{\pgfqpoint{0.150000in}{0.150000in}}{\pgfqpoint{2.700000in}{1.950000in}}%
\pgfusepath{clip}%
\pgfsetbuttcap%
\pgfsetroundjoin%
\definecolor{currentfill}{rgb}{0.524219,0.582812,0.664844}%
\pgfsetfillcolor{currentfill}%
\pgfsetlinewidth{0.000000pt}%
\definecolor{currentstroke}{rgb}{0.000000,0.000000,0.000000}%
\pgfsetstrokecolor{currentstroke}%
\pgfsetdash{}{0pt}%
\pgfpathmoveto{\pgfqpoint{1.725588in}{1.542769in}}%
\pgfpathlineto{\pgfqpoint{1.762994in}{1.561332in}}%
\pgfpathlineto{\pgfqpoint{1.724898in}{1.579828in}}%
\pgfpathlineto{\pgfqpoint{1.687491in}{1.561332in}}%
\pgfpathclose%
\pgfusepath{fill}%
\end{pgfscope}%
\begin{pgfscope}%
\pgfpathrectangle{\pgfqpoint{0.150000in}{0.150000in}}{\pgfqpoint{2.700000in}{1.950000in}}%
\pgfusepath{clip}%
\pgfsetbuttcap%
\pgfsetroundjoin%
\definecolor{currentfill}{rgb}{0.524219,0.582812,0.664844}%
\pgfsetfillcolor{currentfill}%
\pgfsetlinewidth{0.000000pt}%
\definecolor{currentstroke}{rgb}{0.000000,0.000000,0.000000}%
\pgfsetstrokecolor{currentstroke}%
\pgfsetdash{}{0pt}%
\pgfpathmoveto{\pgfqpoint{1.649948in}{1.542769in}}%
\pgfpathlineto{\pgfqpoint{1.687491in}{1.561332in}}%
\pgfpathlineto{\pgfqpoint{1.649533in}{1.579828in}}%
\pgfpathlineto{\pgfqpoint{1.611989in}{1.561332in}}%
\pgfpathclose%
\pgfusepath{fill}%
\end{pgfscope}%
\begin{pgfscope}%
\pgfpathrectangle{\pgfqpoint{0.150000in}{0.150000in}}{\pgfqpoint{2.700000in}{1.950000in}}%
\pgfusepath{clip}%
\pgfsetbuttcap%
\pgfsetroundjoin%
\definecolor{currentfill}{rgb}{0.524219,0.582812,0.664844}%
\pgfsetfillcolor{currentfill}%
\pgfsetlinewidth{0.000000pt}%
\definecolor{currentstroke}{rgb}{0.000000,0.000000,0.000000}%
\pgfsetstrokecolor{currentstroke}%
\pgfsetdash{}{0pt}%
\pgfpathmoveto{\pgfqpoint{1.574307in}{1.542769in}}%
\pgfpathlineto{\pgfqpoint{1.611989in}{1.561332in}}%
\pgfpathlineto{\pgfqpoint{1.574169in}{1.579828in}}%
\pgfpathlineto{\pgfqpoint{1.536486in}{1.561332in}}%
\pgfpathclose%
\pgfusepath{fill}%
\end{pgfscope}%
\begin{pgfscope}%
\pgfpathrectangle{\pgfqpoint{0.150000in}{0.150000in}}{\pgfqpoint{2.700000in}{1.950000in}}%
\pgfusepath{clip}%
\pgfsetbuttcap%
\pgfsetroundjoin%
\definecolor{currentfill}{rgb}{0.524219,0.582812,0.664844}%
\pgfsetfillcolor{currentfill}%
\pgfsetlinewidth{0.000000pt}%
\definecolor{currentstroke}{rgb}{0.000000,0.000000,0.000000}%
\pgfsetstrokecolor{currentstroke}%
\pgfsetdash{}{0pt}%
\pgfpathmoveto{\pgfqpoint{1.498666in}{1.542769in}}%
\pgfpathlineto{\pgfqpoint{1.536486in}{1.561332in}}%
\pgfpathlineto{\pgfqpoint{1.498804in}{1.579828in}}%
\pgfpathlineto{\pgfqpoint{1.460984in}{1.561332in}}%
\pgfpathclose%
\pgfusepath{fill}%
\end{pgfscope}%
\begin{pgfscope}%
\pgfpathrectangle{\pgfqpoint{0.150000in}{0.150000in}}{\pgfqpoint{2.700000in}{1.950000in}}%
\pgfusepath{clip}%
\pgfsetbuttcap%
\pgfsetroundjoin%
\definecolor{currentfill}{rgb}{0.524219,0.582812,0.664844}%
\pgfsetfillcolor{currentfill}%
\pgfsetlinewidth{0.000000pt}%
\definecolor{currentstroke}{rgb}{0.000000,0.000000,0.000000}%
\pgfsetstrokecolor{currentstroke}%
\pgfsetdash{}{0pt}%
\pgfpathmoveto{\pgfqpoint{1.423025in}{1.542769in}}%
\pgfpathlineto{\pgfqpoint{1.460984in}{1.561332in}}%
\pgfpathlineto{\pgfqpoint{1.423440in}{1.579828in}}%
\pgfpathlineto{\pgfqpoint{1.385482in}{1.561332in}}%
\pgfpathclose%
\pgfusepath{fill}%
\end{pgfscope}%
\begin{pgfscope}%
\pgfpathrectangle{\pgfqpoint{0.150000in}{0.150000in}}{\pgfqpoint{2.700000in}{1.950000in}}%
\pgfusepath{clip}%
\pgfsetbuttcap%
\pgfsetroundjoin%
\definecolor{currentfill}{rgb}{0.524219,0.582812,0.664844}%
\pgfsetfillcolor{currentfill}%
\pgfsetlinewidth{0.000000pt}%
\definecolor{currentstroke}{rgb}{0.000000,0.000000,0.000000}%
\pgfsetstrokecolor{currentstroke}%
\pgfsetdash{}{0pt}%
\pgfpathmoveto{\pgfqpoint{1.347385in}{1.542769in}}%
\pgfpathlineto{\pgfqpoint{1.385482in}{1.561332in}}%
\pgfpathlineto{\pgfqpoint{1.348075in}{1.579828in}}%
\pgfpathlineto{\pgfqpoint{1.309979in}{1.561332in}}%
\pgfpathclose%
\pgfusepath{fill}%
\end{pgfscope}%
\begin{pgfscope}%
\pgfpathrectangle{\pgfqpoint{0.150000in}{0.150000in}}{\pgfqpoint{2.700000in}{1.950000in}}%
\pgfusepath{clip}%
\pgfsetbuttcap%
\pgfsetroundjoin%
\definecolor{currentfill}{rgb}{0.524219,0.582812,0.664844}%
\pgfsetfillcolor{currentfill}%
\pgfsetlinewidth{0.000000pt}%
\definecolor{currentstroke}{rgb}{0.000000,0.000000,0.000000}%
\pgfsetstrokecolor{currentstroke}%
\pgfsetdash{}{0pt}%
\pgfpathmoveto{\pgfqpoint{1.271744in}{1.542769in}}%
\pgfpathlineto{\pgfqpoint{1.309979in}{1.561332in}}%
\pgfpathlineto{\pgfqpoint{1.272711in}{1.579828in}}%
\pgfpathlineto{\pgfqpoint{1.234477in}{1.561332in}}%
\pgfpathclose%
\pgfusepath{fill}%
\end{pgfscope}%
\begin{pgfscope}%
\pgfpathrectangle{\pgfqpoint{0.150000in}{0.150000in}}{\pgfqpoint{2.700000in}{1.950000in}}%
\pgfusepath{clip}%
\pgfsetbuttcap%
\pgfsetroundjoin%
\definecolor{currentfill}{rgb}{0.524219,0.582812,0.664844}%
\pgfsetfillcolor{currentfill}%
\pgfsetlinewidth{0.000000pt}%
\definecolor{currentstroke}{rgb}{0.000000,0.000000,0.000000}%
\pgfsetstrokecolor{currentstroke}%
\pgfsetdash{}{0pt}%
\pgfpathmoveto{\pgfqpoint{1.196103in}{1.542769in}}%
\pgfpathlineto{\pgfqpoint{1.234477in}{1.561332in}}%
\pgfpathlineto{\pgfqpoint{1.197346in}{1.579828in}}%
\pgfpathlineto{\pgfqpoint{1.158975in}{1.561332in}}%
\pgfpathclose%
\pgfusepath{fill}%
\end{pgfscope}%
\begin{pgfscope}%
\pgfpathrectangle{\pgfqpoint{0.150000in}{0.150000in}}{\pgfqpoint{2.700000in}{1.950000in}}%
\pgfusepath{clip}%
\pgfsetbuttcap%
\pgfsetroundjoin%
\definecolor{currentfill}{rgb}{0.524219,0.582812,0.664844}%
\pgfsetfillcolor{currentfill}%
\pgfsetlinewidth{0.000000pt}%
\definecolor{currentstroke}{rgb}{0.000000,0.000000,0.000000}%
\pgfsetstrokecolor{currentstroke}%
\pgfsetdash{}{0pt}%
\pgfpathmoveto{\pgfqpoint{1.120462in}{1.542769in}}%
\pgfpathlineto{\pgfqpoint{1.158975in}{1.561332in}}%
\pgfpathlineto{\pgfqpoint{1.121982in}{1.579828in}}%
\pgfpathlineto{\pgfqpoint{1.083472in}{1.561332in}}%
\pgfpathclose%
\pgfusepath{fill}%
\end{pgfscope}%
\begin{pgfscope}%
\pgfpathrectangle{\pgfqpoint{0.150000in}{0.150000in}}{\pgfqpoint{2.700000in}{1.950000in}}%
\pgfusepath{clip}%
\pgfsetbuttcap%
\pgfsetroundjoin%
\definecolor{currentfill}{rgb}{0.524219,0.582812,0.664844}%
\pgfsetfillcolor{currentfill}%
\pgfsetlinewidth{0.000000pt}%
\definecolor{currentstroke}{rgb}{0.000000,0.000000,0.000000}%
\pgfsetstrokecolor{currentstroke}%
\pgfsetdash{}{0pt}%
\pgfpathmoveto{\pgfqpoint{1.044822in}{1.542769in}}%
\pgfpathlineto{\pgfqpoint{1.083472in}{1.561332in}}%
\pgfpathlineto{\pgfqpoint{1.046617in}{1.579828in}}%
\pgfpathlineto{\pgfqpoint{1.007970in}{1.561332in}}%
\pgfpathclose%
\pgfusepath{fill}%
\end{pgfscope}%
\begin{pgfscope}%
\pgfpathrectangle{\pgfqpoint{0.150000in}{0.150000in}}{\pgfqpoint{2.700000in}{1.950000in}}%
\pgfusepath{clip}%
\pgfsetbuttcap%
\pgfsetroundjoin%
\definecolor{currentfill}{rgb}{0.524219,0.582812,0.664844}%
\pgfsetfillcolor{currentfill}%
\pgfsetlinewidth{0.000000pt}%
\definecolor{currentstroke}{rgb}{0.000000,0.000000,0.000000}%
\pgfsetstrokecolor{currentstroke}%
\pgfsetdash{}{0pt}%
\pgfpathmoveto{\pgfqpoint{0.969181in}{1.542769in}}%
\pgfpathlineto{\pgfqpoint{1.007970in}{1.561332in}}%
\pgfpathlineto{\pgfqpoint{0.971253in}{1.579828in}}%
\pgfpathlineto{\pgfqpoint{0.932467in}{1.561332in}}%
\pgfpathclose%
\pgfusepath{fill}%
\end{pgfscope}%
\begin{pgfscope}%
\pgfpathrectangle{\pgfqpoint{0.150000in}{0.150000in}}{\pgfqpoint{2.700000in}{1.950000in}}%
\pgfusepath{clip}%
\pgfsetbuttcap%
\pgfsetroundjoin%
\definecolor{currentfill}{rgb}{0.959574,0.964553,0.971523}%
\pgfsetfillcolor{currentfill}%
\pgfsetlinewidth{0.000000pt}%
\definecolor{currentstroke}{rgb}{0.000000,0.000000,0.000000}%
\pgfsetstrokecolor{currentstroke}%
\pgfsetdash{}{0pt}%
\pgfpathmoveto{\pgfqpoint{1.800385in}{0.814830in}}%
\pgfpathlineto{\pgfqpoint{1.837535in}{0.834662in}}%
\pgfpathlineto{\pgfqpoint{1.801154in}{0.921786in}}%
\pgfpathlineto{\pgfqpoint{1.763761in}{0.902020in}}%
\pgfpathclose%
\pgfusepath{fill}%
\end{pgfscope}%
\begin{pgfscope}%
\pgfpathrectangle{\pgfqpoint{0.150000in}{0.150000in}}{\pgfqpoint{2.700000in}{1.950000in}}%
\pgfusepath{clip}%
\pgfsetbuttcap%
\pgfsetroundjoin%
\definecolor{currentfill}{rgb}{0.586412,0.637347,0.708655}%
\pgfsetfillcolor{currentfill}%
\pgfsetlinewidth{0.000000pt}%
\definecolor{currentstroke}{rgb}{0.000000,0.000000,0.000000}%
\pgfsetstrokecolor{currentstroke}%
\pgfsetdash{}{0pt}%
\pgfpathmoveto{\pgfqpoint{1.989345in}{1.403384in}}%
\pgfpathlineto{\pgfqpoint{2.026188in}{1.422161in}}%
\pgfpathlineto{\pgfqpoint{1.990167in}{1.500217in}}%
\pgfpathlineto{\pgfqpoint{1.953124in}{1.481514in}}%
\pgfpathclose%
\pgfusepath{fill}%
\end{pgfscope}%
\begin{pgfscope}%
\pgfpathrectangle{\pgfqpoint{0.150000in}{0.150000in}}{\pgfqpoint{2.700000in}{1.950000in}}%
\pgfusepath{clip}%
\pgfsetbuttcap%
\pgfsetroundjoin%
\definecolor{currentfill}{rgb}{0.661045,0.702788,0.761229}%
\pgfsetfillcolor{currentfill}%
\pgfsetlinewidth{0.000000pt}%
\definecolor{currentstroke}{rgb}{0.000000,0.000000,0.000000}%
\pgfsetstrokecolor{currentstroke}%
\pgfsetdash{}{0pt}%
\pgfpathmoveto{\pgfqpoint{1.951613in}{1.287913in}}%
\pgfpathlineto{\pgfqpoint{1.988525in}{1.306901in}}%
\pgfpathlineto{\pgfqpoint{1.952367in}{1.384538in}}%
\pgfpathlineto{\pgfqpoint{1.915254in}{1.365622in}}%
\pgfpathclose%
\pgfusepath{fill}%
\end{pgfscope}%
\begin{pgfscope}%
\pgfpathrectangle{\pgfqpoint{0.150000in}{0.150000in}}{\pgfqpoint{2.700000in}{1.950000in}}%
\pgfusepath{clip}%
\pgfsetbuttcap%
\pgfsetroundjoin%
\definecolor{currentfill}{rgb}{0.729458,0.762776,0.809421}%
\pgfsetfillcolor{currentfill}%
\pgfsetlinewidth{0.000000pt}%
\definecolor{currentstroke}{rgb}{0.000000,0.000000,0.000000}%
\pgfsetstrokecolor{currentstroke}%
\pgfsetdash{}{0pt}%
\pgfpathmoveto{\pgfqpoint{1.913880in}{1.172438in}}%
\pgfpathlineto{\pgfqpoint{1.950862in}{1.191638in}}%
\pgfpathlineto{\pgfqpoint{1.914566in}{1.268855in}}%
\pgfpathlineto{\pgfqpoint{1.877382in}{1.249727in}}%
\pgfpathclose%
\pgfusepath{fill}%
\end{pgfscope}%
\begin{pgfscope}%
\pgfpathrectangle{\pgfqpoint{0.150000in}{0.150000in}}{\pgfqpoint{2.700000in}{1.950000in}}%
\pgfusepath{clip}%
\pgfsetbuttcap%
\pgfsetroundjoin%
\definecolor{currentfill}{rgb}{0.804090,0.828217,0.861994}%
\pgfsetfillcolor{currentfill}%
\pgfsetlinewidth{0.000000pt}%
\definecolor{currentstroke}{rgb}{0.000000,0.000000,0.000000}%
\pgfsetstrokecolor{currentstroke}%
\pgfsetdash{}{0pt}%
\pgfpathmoveto{\pgfqpoint{1.876146in}{1.056961in}}%
\pgfpathlineto{\pgfqpoint{1.913197in}{1.076371in}}%
\pgfpathlineto{\pgfqpoint{1.876763in}{1.153169in}}%
\pgfpathlineto{\pgfqpoint{1.839509in}{1.133828in}}%
\pgfpathclose%
\pgfusepath{fill}%
\end{pgfscope}%
\begin{pgfscope}%
\pgfpathrectangle{\pgfqpoint{0.150000in}{0.150000in}}{\pgfqpoint{2.700000in}{1.950000in}}%
\pgfusepath{clip}%
\pgfsetbuttcap%
\pgfsetroundjoin%
\definecolor{currentfill}{rgb}{0.878722,0.893658,0.914568}%
\pgfsetfillcolor{currentfill}%
\pgfsetlinewidth{0.000000pt}%
\definecolor{currentstroke}{rgb}{0.000000,0.000000,0.000000}%
\pgfsetstrokecolor{currentstroke}%
\pgfsetdash{}{0pt}%
\pgfpathmoveto{\pgfqpoint{1.838411in}{0.941479in}}%
\pgfpathlineto{\pgfqpoint{1.875531in}{0.961101in}}%
\pgfpathlineto{\pgfqpoint{1.838959in}{1.037479in}}%
\pgfpathlineto{\pgfqpoint{1.801636in}{1.017926in}}%
\pgfpathclose%
\pgfusepath{fill}%
\end{pgfscope}%
\begin{pgfscope}%
\pgfpathrectangle{\pgfqpoint{0.150000in}{0.150000in}}{\pgfqpoint{2.700000in}{1.950000in}}%
\pgfusepath{clip}%
\pgfsetbuttcap%
\pgfsetroundjoin%
\definecolor{currentfill}{rgb}{0.536657,0.593719,0.673606}%
\pgfsetfillcolor{currentfill}%
\pgfsetlinewidth{0.000000pt}%
\definecolor{currentstroke}{rgb}{0.000000,0.000000,0.000000}%
\pgfsetstrokecolor{currentstroke}%
\pgfsetdash{}{0pt}%
\pgfpathmoveto{\pgfqpoint{1.990167in}{1.500217in}}%
\pgfpathlineto{\pgfqpoint{2.027075in}{1.518851in}}%
\pgfpathlineto{\pgfqpoint{1.989501in}{1.561332in}}%
\pgfpathlineto{\pgfqpoint{1.952510in}{1.542769in}}%
\pgfpathclose%
\pgfusepath{fill}%
\end{pgfscope}%
\begin{pgfscope}%
\pgfpathrectangle{\pgfqpoint{0.150000in}{0.150000in}}{\pgfqpoint{2.700000in}{1.950000in}}%
\pgfusepath{clip}%
\pgfsetbuttcap%
\pgfsetroundjoin%
\definecolor{currentfill}{rgb}{0.979105,0.962086,0.963434}%
\pgfsetfillcolor{currentfill}%
\pgfsetlinewidth{0.000000pt}%
\definecolor{currentstroke}{rgb}{0.000000,0.000000,0.000000}%
\pgfsetstrokecolor{currentstroke}%
\pgfsetdash{}{0pt}%
\pgfpathmoveto{\pgfqpoint{1.725334in}{0.679352in}}%
\pgfpathlineto{\pgfqpoint{1.762689in}{0.699468in}}%
\pgfpathlineto{\pgfqpoint{1.725677in}{0.774949in}}%
\pgfpathlineto{\pgfqpoint{1.688117in}{0.754899in}}%
\pgfpathclose%
\pgfusepath{fill}%
\end{pgfscope}%
\begin{pgfscope}%
\pgfpathrectangle{\pgfqpoint{0.150000in}{0.150000in}}{\pgfqpoint{2.700000in}{1.950000in}}%
\pgfusepath{clip}%
\pgfsetbuttcap%
\pgfsetroundjoin%
\definecolor{currentfill}{rgb}{0.933517,0.879366,0.883655}%
\pgfsetfillcolor{currentfill}%
\pgfsetlinewidth{0.000000pt}%
\definecolor{currentstroke}{rgb}{0.000000,0.000000,0.000000}%
\pgfsetstrokecolor{currentstroke}%
\pgfsetdash{}{0pt}%
\pgfpathmoveto{\pgfqpoint{1.687566in}{0.563775in}}%
\pgfpathlineto{\pgfqpoint{1.724991in}{0.584102in}}%
\pgfpathlineto{\pgfqpoint{1.687841in}{0.659163in}}%
\pgfpathlineto{\pgfqpoint{1.650211in}{0.638900in}}%
\pgfpathclose%
\pgfusepath{fill}%
\end{pgfscope}%
\begin{pgfscope}%
\pgfpathrectangle{\pgfqpoint{0.150000in}{0.150000in}}{\pgfqpoint{2.700000in}{1.950000in}}%
\pgfusepath{clip}%
\pgfsetbuttcap%
\pgfsetroundjoin%
\definecolor{currentfill}{rgb}{0.887929,0.796645,0.803876}%
\pgfsetfillcolor{currentfill}%
\pgfsetlinewidth{0.000000pt}%
\definecolor{currentstroke}{rgb}{0.000000,0.000000,0.000000}%
\pgfsetstrokecolor{currentstroke}%
\pgfsetdash{}{0pt}%
\pgfpathmoveto{\pgfqpoint{1.649798in}{0.448194in}}%
\pgfpathlineto{\pgfqpoint{1.687293in}{0.468732in}}%
\pgfpathlineto{\pgfqpoint{1.650004in}{0.543374in}}%
\pgfpathlineto{\pgfqpoint{1.612304in}{0.522897in}}%
\pgfpathclose%
\pgfusepath{fill}%
\end{pgfscope}%
\begin{pgfscope}%
\pgfpathrectangle{\pgfqpoint{0.150000in}{0.150000in}}{\pgfqpoint{2.700000in}{1.950000in}}%
\pgfusepath{clip}%
\pgfsetbuttcap%
\pgfsetroundjoin%
\definecolor{currentfill}{rgb}{0.846140,0.720818,0.730744}%
\pgfsetfillcolor{currentfill}%
\pgfsetlinewidth{0.000000pt}%
\definecolor{currentstroke}{rgb}{0.000000,0.000000,0.000000}%
\pgfsetstrokecolor{currentstroke}%
\pgfsetdash{}{0pt}%
\pgfpathmoveto{\pgfqpoint{1.612029in}{0.332609in}}%
\pgfpathlineto{\pgfqpoint{1.649593in}{0.353359in}}%
\pgfpathlineto{\pgfqpoint{1.612166in}{0.427580in}}%
\pgfpathlineto{\pgfqpoint{1.574396in}{0.406891in}}%
\pgfpathclose%
\pgfusepath{fill}%
\end{pgfscope}%
\begin{pgfscope}%
\pgfpathrectangle{\pgfqpoint{0.150000in}{0.150000in}}{\pgfqpoint{2.700000in}{1.950000in}}%
\pgfusepath{clip}%
\pgfsetbuttcap%
\pgfsetroundjoin%
\definecolor{currentfill}{rgb}{0.524219,0.582812,0.664844}%
\pgfsetfillcolor{currentfill}%
\pgfsetlinewidth{0.000000pt}%
\definecolor{currentstroke}{rgb}{0.000000,0.000000,0.000000}%
\pgfsetstrokecolor{currentstroke}%
\pgfsetdash{}{0pt}%
\pgfpathmoveto{\pgfqpoint{1.915384in}{1.524138in}}%
\pgfpathlineto{\pgfqpoint{1.952510in}{1.542769in}}%
\pgfpathlineto{\pgfqpoint{1.913998in}{1.561332in}}%
\pgfpathlineto{\pgfqpoint{1.876870in}{1.542769in}}%
\pgfpathclose%
\pgfusepath{fill}%
\end{pgfscope}%
\begin{pgfscope}%
\pgfpathrectangle{\pgfqpoint{0.150000in}{0.150000in}}{\pgfqpoint{2.700000in}{1.950000in}}%
\pgfusepath{clip}%
\pgfsetbuttcap%
\pgfsetroundjoin%
\definecolor{currentfill}{rgb}{0.524219,0.582812,0.664844}%
\pgfsetfillcolor{currentfill}%
\pgfsetlinewidth{0.000000pt}%
\definecolor{currentstroke}{rgb}{0.000000,0.000000,0.000000}%
\pgfsetstrokecolor{currentstroke}%
\pgfsetdash{}{0pt}%
\pgfpathmoveto{\pgfqpoint{1.839605in}{1.524138in}}%
\pgfpathlineto{\pgfqpoint{1.876870in}{1.542769in}}%
\pgfpathlineto{\pgfqpoint{1.838496in}{1.561332in}}%
\pgfpathlineto{\pgfqpoint{1.801229in}{1.542769in}}%
\pgfpathclose%
\pgfusepath{fill}%
\end{pgfscope}%
\begin{pgfscope}%
\pgfpathrectangle{\pgfqpoint{0.150000in}{0.150000in}}{\pgfqpoint{2.700000in}{1.950000in}}%
\pgfusepath{clip}%
\pgfsetbuttcap%
\pgfsetroundjoin%
\definecolor{currentfill}{rgb}{0.524219,0.582812,0.664844}%
\pgfsetfillcolor{currentfill}%
\pgfsetlinewidth{0.000000pt}%
\definecolor{currentstroke}{rgb}{0.000000,0.000000,0.000000}%
\pgfsetstrokecolor{currentstroke}%
\pgfsetdash{}{0pt}%
\pgfpathmoveto{\pgfqpoint{1.763825in}{1.524138in}}%
\pgfpathlineto{\pgfqpoint{1.801229in}{1.542769in}}%
\pgfpathlineto{\pgfqpoint{1.762994in}{1.561332in}}%
\pgfpathlineto{\pgfqpoint{1.725588in}{1.542769in}}%
\pgfpathclose%
\pgfusepath{fill}%
\end{pgfscope}%
\begin{pgfscope}%
\pgfpathrectangle{\pgfqpoint{0.150000in}{0.150000in}}{\pgfqpoint{2.700000in}{1.950000in}}%
\pgfusepath{clip}%
\pgfsetbuttcap%
\pgfsetroundjoin%
\definecolor{currentfill}{rgb}{0.524219,0.582812,0.664844}%
\pgfsetfillcolor{currentfill}%
\pgfsetlinewidth{0.000000pt}%
\definecolor{currentstroke}{rgb}{0.000000,0.000000,0.000000}%
\pgfsetstrokecolor{currentstroke}%
\pgfsetdash{}{0pt}%
\pgfpathmoveto{\pgfqpoint{1.688046in}{1.524138in}}%
\pgfpathlineto{\pgfqpoint{1.725588in}{1.542769in}}%
\pgfpathlineto{\pgfqpoint{1.687491in}{1.561332in}}%
\pgfpathlineto{\pgfqpoint{1.649948in}{1.542769in}}%
\pgfpathclose%
\pgfusepath{fill}%
\end{pgfscope}%
\begin{pgfscope}%
\pgfpathrectangle{\pgfqpoint{0.150000in}{0.150000in}}{\pgfqpoint{2.700000in}{1.950000in}}%
\pgfusepath{clip}%
\pgfsetbuttcap%
\pgfsetroundjoin%
\definecolor{currentfill}{rgb}{0.524219,0.582812,0.664844}%
\pgfsetfillcolor{currentfill}%
\pgfsetlinewidth{0.000000pt}%
\definecolor{currentstroke}{rgb}{0.000000,0.000000,0.000000}%
\pgfsetstrokecolor{currentstroke}%
\pgfsetdash{}{0pt}%
\pgfpathmoveto{\pgfqpoint{1.612266in}{1.524138in}}%
\pgfpathlineto{\pgfqpoint{1.649948in}{1.542769in}}%
\pgfpathlineto{\pgfqpoint{1.611989in}{1.561332in}}%
\pgfpathlineto{\pgfqpoint{1.574307in}{1.542769in}}%
\pgfpathclose%
\pgfusepath{fill}%
\end{pgfscope}%
\begin{pgfscope}%
\pgfpathrectangle{\pgfqpoint{0.150000in}{0.150000in}}{\pgfqpoint{2.700000in}{1.950000in}}%
\pgfusepath{clip}%
\pgfsetbuttcap%
\pgfsetroundjoin%
\definecolor{currentfill}{rgb}{0.524219,0.582812,0.664844}%
\pgfsetfillcolor{currentfill}%
\pgfsetlinewidth{0.000000pt}%
\definecolor{currentstroke}{rgb}{0.000000,0.000000,0.000000}%
\pgfsetstrokecolor{currentstroke}%
\pgfsetdash{}{0pt}%
\pgfpathmoveto{\pgfqpoint{1.536486in}{1.524138in}}%
\pgfpathlineto{\pgfqpoint{1.574307in}{1.542769in}}%
\pgfpathlineto{\pgfqpoint{1.536486in}{1.561332in}}%
\pgfpathlineto{\pgfqpoint{1.498666in}{1.542769in}}%
\pgfpathclose%
\pgfusepath{fill}%
\end{pgfscope}%
\begin{pgfscope}%
\pgfpathrectangle{\pgfqpoint{0.150000in}{0.150000in}}{\pgfqpoint{2.700000in}{1.950000in}}%
\pgfusepath{clip}%
\pgfsetbuttcap%
\pgfsetroundjoin%
\definecolor{currentfill}{rgb}{0.524219,0.582812,0.664844}%
\pgfsetfillcolor{currentfill}%
\pgfsetlinewidth{0.000000pt}%
\definecolor{currentstroke}{rgb}{0.000000,0.000000,0.000000}%
\pgfsetstrokecolor{currentstroke}%
\pgfsetdash{}{0pt}%
\pgfpathmoveto{\pgfqpoint{1.460707in}{1.524138in}}%
\pgfpathlineto{\pgfqpoint{1.498666in}{1.542769in}}%
\pgfpathlineto{\pgfqpoint{1.460984in}{1.561332in}}%
\pgfpathlineto{\pgfqpoint{1.423025in}{1.542769in}}%
\pgfpathclose%
\pgfusepath{fill}%
\end{pgfscope}%
\begin{pgfscope}%
\pgfpathrectangle{\pgfqpoint{0.150000in}{0.150000in}}{\pgfqpoint{2.700000in}{1.950000in}}%
\pgfusepath{clip}%
\pgfsetbuttcap%
\pgfsetroundjoin%
\definecolor{currentfill}{rgb}{0.524219,0.582812,0.664844}%
\pgfsetfillcolor{currentfill}%
\pgfsetlinewidth{0.000000pt}%
\definecolor{currentstroke}{rgb}{0.000000,0.000000,0.000000}%
\pgfsetstrokecolor{currentstroke}%
\pgfsetdash{}{0pt}%
\pgfpathmoveto{\pgfqpoint{1.384927in}{1.524138in}}%
\pgfpathlineto{\pgfqpoint{1.423025in}{1.542769in}}%
\pgfpathlineto{\pgfqpoint{1.385482in}{1.561332in}}%
\pgfpathlineto{\pgfqpoint{1.347385in}{1.542769in}}%
\pgfpathclose%
\pgfusepath{fill}%
\end{pgfscope}%
\begin{pgfscope}%
\pgfpathrectangle{\pgfqpoint{0.150000in}{0.150000in}}{\pgfqpoint{2.700000in}{1.950000in}}%
\pgfusepath{clip}%
\pgfsetbuttcap%
\pgfsetroundjoin%
\definecolor{currentfill}{rgb}{0.524219,0.582812,0.664844}%
\pgfsetfillcolor{currentfill}%
\pgfsetlinewidth{0.000000pt}%
\definecolor{currentstroke}{rgb}{0.000000,0.000000,0.000000}%
\pgfsetstrokecolor{currentstroke}%
\pgfsetdash{}{0pt}%
\pgfpathmoveto{\pgfqpoint{1.309148in}{1.524138in}}%
\pgfpathlineto{\pgfqpoint{1.347385in}{1.542769in}}%
\pgfpathlineto{\pgfqpoint{1.309979in}{1.561332in}}%
\pgfpathlineto{\pgfqpoint{1.271744in}{1.542769in}}%
\pgfpathclose%
\pgfusepath{fill}%
\end{pgfscope}%
\begin{pgfscope}%
\pgfpathrectangle{\pgfqpoint{0.150000in}{0.150000in}}{\pgfqpoint{2.700000in}{1.950000in}}%
\pgfusepath{clip}%
\pgfsetbuttcap%
\pgfsetroundjoin%
\definecolor{currentfill}{rgb}{0.524219,0.582812,0.664844}%
\pgfsetfillcolor{currentfill}%
\pgfsetlinewidth{0.000000pt}%
\definecolor{currentstroke}{rgb}{0.000000,0.000000,0.000000}%
\pgfsetstrokecolor{currentstroke}%
\pgfsetdash{}{0pt}%
\pgfpathmoveto{\pgfqpoint{1.233368in}{1.524138in}}%
\pgfpathlineto{\pgfqpoint{1.271744in}{1.542769in}}%
\pgfpathlineto{\pgfqpoint{1.234477in}{1.561332in}}%
\pgfpathlineto{\pgfqpoint{1.196103in}{1.542769in}}%
\pgfpathclose%
\pgfusepath{fill}%
\end{pgfscope}%
\begin{pgfscope}%
\pgfpathrectangle{\pgfqpoint{0.150000in}{0.150000in}}{\pgfqpoint{2.700000in}{1.950000in}}%
\pgfusepath{clip}%
\pgfsetbuttcap%
\pgfsetroundjoin%
\definecolor{currentfill}{rgb}{0.524219,0.582812,0.664844}%
\pgfsetfillcolor{currentfill}%
\pgfsetlinewidth{0.000000pt}%
\definecolor{currentstroke}{rgb}{0.000000,0.000000,0.000000}%
\pgfsetstrokecolor{currentstroke}%
\pgfsetdash{}{0pt}%
\pgfpathmoveto{\pgfqpoint{1.157589in}{1.524138in}}%
\pgfpathlineto{\pgfqpoint{1.196103in}{1.542769in}}%
\pgfpathlineto{\pgfqpoint{1.158975in}{1.561332in}}%
\pgfpathlineto{\pgfqpoint{1.120462in}{1.542769in}}%
\pgfpathclose%
\pgfusepath{fill}%
\end{pgfscope}%
\begin{pgfscope}%
\pgfpathrectangle{\pgfqpoint{0.150000in}{0.150000in}}{\pgfqpoint{2.700000in}{1.950000in}}%
\pgfusepath{clip}%
\pgfsetbuttcap%
\pgfsetroundjoin%
\definecolor{currentfill}{rgb}{0.524219,0.582812,0.664844}%
\pgfsetfillcolor{currentfill}%
\pgfsetlinewidth{0.000000pt}%
\definecolor{currentstroke}{rgb}{0.000000,0.000000,0.000000}%
\pgfsetstrokecolor{currentstroke}%
\pgfsetdash{}{0pt}%
\pgfpathmoveto{\pgfqpoint{1.081809in}{1.524138in}}%
\pgfpathlineto{\pgfqpoint{1.120462in}{1.542769in}}%
\pgfpathlineto{\pgfqpoint{1.083472in}{1.561332in}}%
\pgfpathlineto{\pgfqpoint{1.044822in}{1.542769in}}%
\pgfpathclose%
\pgfusepath{fill}%
\end{pgfscope}%
\begin{pgfscope}%
\pgfpathrectangle{\pgfqpoint{0.150000in}{0.150000in}}{\pgfqpoint{2.700000in}{1.950000in}}%
\pgfusepath{clip}%
\pgfsetbuttcap%
\pgfsetroundjoin%
\definecolor{currentfill}{rgb}{0.524219,0.582812,0.664844}%
\pgfsetfillcolor{currentfill}%
\pgfsetlinewidth{0.000000pt}%
\definecolor{currentstroke}{rgb}{0.000000,0.000000,0.000000}%
\pgfsetstrokecolor{currentstroke}%
\pgfsetdash{}{0pt}%
\pgfpathmoveto{\pgfqpoint{1.006029in}{1.524138in}}%
\pgfpathlineto{\pgfqpoint{1.044822in}{1.542769in}}%
\pgfpathlineto{\pgfqpoint{1.007970in}{1.561332in}}%
\pgfpathlineto{\pgfqpoint{0.969181in}{1.542769in}}%
\pgfpathclose%
\pgfusepath{fill}%
\end{pgfscope}%
\begin{pgfscope}%
\pgfpathrectangle{\pgfqpoint{0.150000in}{0.150000in}}{\pgfqpoint{2.700000in}{1.950000in}}%
\pgfusepath{clip}%
\pgfsetbuttcap%
\pgfsetroundjoin%
\definecolor{currentfill}{rgb}{0.524219,0.582812,0.664844}%
\pgfsetfillcolor{currentfill}%
\pgfsetlinewidth{0.000000pt}%
\definecolor{currentstroke}{rgb}{0.000000,0.000000,0.000000}%
\pgfsetstrokecolor{currentstroke}%
\pgfsetdash{}{0pt}%
\pgfpathmoveto{\pgfqpoint{0.930250in}{1.524138in}}%
\pgfpathlineto{\pgfqpoint{0.969181in}{1.542769in}}%
\pgfpathlineto{\pgfqpoint{0.932467in}{1.561332in}}%
\pgfpathlineto{\pgfqpoint{0.893540in}{1.542769in}}%
\pgfpathclose%
\pgfusepath{fill}%
\end{pgfscope}%
\begin{pgfscope}%
\pgfpathrectangle{\pgfqpoint{0.150000in}{0.150000in}}{\pgfqpoint{2.700000in}{1.950000in}}%
\pgfusepath{clip}%
\pgfsetbuttcap%
\pgfsetroundjoin%
\definecolor{currentfill}{rgb}{0.959574,0.964553,0.971523}%
\pgfsetfillcolor{currentfill}%
\pgfsetlinewidth{0.000000pt}%
\definecolor{currentstroke}{rgb}{0.000000,0.000000,0.000000}%
\pgfsetstrokecolor{currentstroke}%
\pgfsetdash{}{0pt}%
\pgfpathmoveto{\pgfqpoint{1.763099in}{0.794926in}}%
\pgfpathlineto{\pgfqpoint{1.800385in}{0.814830in}}%
\pgfpathlineto{\pgfqpoint{1.763761in}{0.902020in}}%
\pgfpathlineto{\pgfqpoint{1.726230in}{0.882182in}}%
\pgfpathclose%
\pgfusepath{fill}%
\end{pgfscope}%
\begin{pgfscope}%
\pgfpathrectangle{\pgfqpoint{0.150000in}{0.150000in}}{\pgfqpoint{2.700000in}{1.950000in}}%
\pgfusepath{clip}%
\pgfsetbuttcap%
\pgfsetroundjoin%
\definecolor{currentfill}{rgb}{0.586412,0.637347,0.708655}%
\pgfsetfillcolor{currentfill}%
\pgfsetlinewidth{0.000000pt}%
\definecolor{currentstroke}{rgb}{0.000000,0.000000,0.000000}%
\pgfsetstrokecolor{currentstroke}%
\pgfsetdash{}{0pt}%
\pgfpathmoveto{\pgfqpoint{1.952367in}{1.384538in}}%
\pgfpathlineto{\pgfqpoint{1.989345in}{1.403384in}}%
\pgfpathlineto{\pgfqpoint{1.953124in}{1.481514in}}%
\pgfpathlineto{\pgfqpoint{1.915944in}{1.462743in}}%
\pgfpathclose%
\pgfusepath{fill}%
\end{pgfscope}%
\begin{pgfscope}%
\pgfpathrectangle{\pgfqpoint{0.150000in}{0.150000in}}{\pgfqpoint{2.700000in}{1.950000in}}%
\pgfusepath{clip}%
\pgfsetbuttcap%
\pgfsetroundjoin%
\definecolor{currentfill}{rgb}{0.661045,0.702788,0.761229}%
\pgfsetfillcolor{currentfill}%
\pgfsetlinewidth{0.000000pt}%
\definecolor{currentstroke}{rgb}{0.000000,0.000000,0.000000}%
\pgfsetstrokecolor{currentstroke}%
\pgfsetdash{}{0pt}%
\pgfpathmoveto{\pgfqpoint{1.914566in}{1.268855in}}%
\pgfpathlineto{\pgfqpoint{1.951613in}{1.287913in}}%
\pgfpathlineto{\pgfqpoint{1.915254in}{1.365622in}}%
\pgfpathlineto{\pgfqpoint{1.878004in}{1.346637in}}%
\pgfpathclose%
\pgfusepath{fill}%
\end{pgfscope}%
\begin{pgfscope}%
\pgfpathrectangle{\pgfqpoint{0.150000in}{0.150000in}}{\pgfqpoint{2.700000in}{1.950000in}}%
\pgfusepath{clip}%
\pgfsetbuttcap%
\pgfsetroundjoin%
\definecolor{currentfill}{rgb}{0.729458,0.762776,0.809421}%
\pgfsetfillcolor{currentfill}%
\pgfsetlinewidth{0.000000pt}%
\definecolor{currentstroke}{rgb}{0.000000,0.000000,0.000000}%
\pgfsetstrokecolor{currentstroke}%
\pgfsetdash{}{0pt}%
\pgfpathmoveto{\pgfqpoint{1.876763in}{1.153169in}}%
\pgfpathlineto{\pgfqpoint{1.913880in}{1.172438in}}%
\pgfpathlineto{\pgfqpoint{1.877382in}{1.249727in}}%
\pgfpathlineto{\pgfqpoint{1.840062in}{1.230529in}}%
\pgfpathclose%
\pgfusepath{fill}%
\end{pgfscope}%
\begin{pgfscope}%
\pgfpathrectangle{\pgfqpoint{0.150000in}{0.150000in}}{\pgfqpoint{2.700000in}{1.950000in}}%
\pgfusepath{clip}%
\pgfsetbuttcap%
\pgfsetroundjoin%
\definecolor{currentfill}{rgb}{0.804090,0.828217,0.861994}%
\pgfsetfillcolor{currentfill}%
\pgfsetlinewidth{0.000000pt}%
\definecolor{currentstroke}{rgb}{0.000000,0.000000,0.000000}%
\pgfsetstrokecolor{currentstroke}%
\pgfsetdash{}{0pt}%
\pgfpathmoveto{\pgfqpoint{1.838959in}{1.037479in}}%
\pgfpathlineto{\pgfqpoint{1.876146in}{1.056961in}}%
\pgfpathlineto{\pgfqpoint{1.839509in}{1.133828in}}%
\pgfpathlineto{\pgfqpoint{1.802119in}{1.114417in}}%
\pgfpathclose%
\pgfusepath{fill}%
\end{pgfscope}%
\begin{pgfscope}%
\pgfpathrectangle{\pgfqpoint{0.150000in}{0.150000in}}{\pgfqpoint{2.700000in}{1.950000in}}%
\pgfusepath{clip}%
\pgfsetbuttcap%
\pgfsetroundjoin%
\definecolor{currentfill}{rgb}{0.878722,0.893658,0.914568}%
\pgfsetfillcolor{currentfill}%
\pgfsetlinewidth{0.000000pt}%
\definecolor{currentstroke}{rgb}{0.000000,0.000000,0.000000}%
\pgfsetstrokecolor{currentstroke}%
\pgfsetdash{}{0pt}%
\pgfpathmoveto{\pgfqpoint{1.801154in}{0.921786in}}%
\pgfpathlineto{\pgfqpoint{1.838411in}{0.941479in}}%
\pgfpathlineto{\pgfqpoint{1.801636in}{1.017926in}}%
\pgfpathlineto{\pgfqpoint{1.764175in}{0.998301in}}%
\pgfpathclose%
\pgfusepath{fill}%
\end{pgfscope}%
\begin{pgfscope}%
\pgfpathrectangle{\pgfqpoint{0.150000in}{0.150000in}}{\pgfqpoint{2.700000in}{1.950000in}}%
\pgfusepath{clip}%
\pgfsetbuttcap%
\pgfsetroundjoin%
\definecolor{currentfill}{rgb}{0.536657,0.593719,0.673606}%
\pgfsetfillcolor{currentfill}%
\pgfsetlinewidth{0.000000pt}%
\definecolor{currentstroke}{rgb}{0.000000,0.000000,0.000000}%
\pgfsetstrokecolor{currentstroke}%
\pgfsetdash{}{0pt}%
\pgfpathmoveto{\pgfqpoint{1.953124in}{1.481514in}}%
\pgfpathlineto{\pgfqpoint{1.990167in}{1.500217in}}%
\pgfpathlineto{\pgfqpoint{1.952510in}{1.542769in}}%
\pgfpathlineto{\pgfqpoint{1.915384in}{1.524138in}}%
\pgfpathclose%
\pgfusepath{fill}%
\end{pgfscope}%
\begin{pgfscope}%
\pgfpathrectangle{\pgfqpoint{0.150000in}{0.150000in}}{\pgfqpoint{2.700000in}{1.950000in}}%
\pgfusepath{clip}%
\pgfsetbuttcap%
\pgfsetroundjoin%
\definecolor{currentfill}{rgb}{0.979105,0.962086,0.963434}%
\pgfsetfillcolor{currentfill}%
\pgfsetlinewidth{0.000000pt}%
\definecolor{currentstroke}{rgb}{0.000000,0.000000,0.000000}%
\pgfsetstrokecolor{currentstroke}%
\pgfsetdash{}{0pt}%
\pgfpathmoveto{\pgfqpoint{1.687841in}{0.659163in}}%
\pgfpathlineto{\pgfqpoint{1.725334in}{0.679352in}}%
\pgfpathlineto{\pgfqpoint{1.688117in}{0.754899in}}%
\pgfpathlineto{\pgfqpoint{1.650418in}{0.734775in}}%
\pgfpathclose%
\pgfusepath{fill}%
\end{pgfscope}%
\begin{pgfscope}%
\pgfpathrectangle{\pgfqpoint{0.150000in}{0.150000in}}{\pgfqpoint{2.700000in}{1.950000in}}%
\pgfusepath{clip}%
\pgfsetbuttcap%
\pgfsetroundjoin%
\definecolor{currentfill}{rgb}{0.933517,0.879366,0.883655}%
\pgfsetfillcolor{currentfill}%
\pgfsetlinewidth{0.000000pt}%
\definecolor{currentstroke}{rgb}{0.000000,0.000000,0.000000}%
\pgfsetstrokecolor{currentstroke}%
\pgfsetdash{}{0pt}%
\pgfpathmoveto{\pgfqpoint{1.650004in}{0.543374in}}%
\pgfpathlineto{\pgfqpoint{1.687566in}{0.563775in}}%
\pgfpathlineto{\pgfqpoint{1.650211in}{0.638900in}}%
\pgfpathlineto{\pgfqpoint{1.612442in}{0.618562in}}%
\pgfpathclose%
\pgfusepath{fill}%
\end{pgfscope}%
\begin{pgfscope}%
\pgfpathrectangle{\pgfqpoint{0.150000in}{0.150000in}}{\pgfqpoint{2.700000in}{1.950000in}}%
\pgfusepath{clip}%
\pgfsetbuttcap%
\pgfsetroundjoin%
\definecolor{currentfill}{rgb}{0.887929,0.796645,0.803876}%
\pgfsetfillcolor{currentfill}%
\pgfsetlinewidth{0.000000pt}%
\definecolor{currentstroke}{rgb}{0.000000,0.000000,0.000000}%
\pgfsetstrokecolor{currentstroke}%
\pgfsetdash{}{0pt}%
\pgfpathmoveto{\pgfqpoint{1.612166in}{0.427580in}}%
\pgfpathlineto{\pgfqpoint{1.649798in}{0.448194in}}%
\pgfpathlineto{\pgfqpoint{1.612304in}{0.522897in}}%
\pgfpathlineto{\pgfqpoint{1.574465in}{0.502346in}}%
\pgfpathclose%
\pgfusepath{fill}%
\end{pgfscope}%
\begin{pgfscope}%
\pgfpathrectangle{\pgfqpoint{0.150000in}{0.150000in}}{\pgfqpoint{2.700000in}{1.950000in}}%
\pgfusepath{clip}%
\pgfsetbuttcap%
\pgfsetroundjoin%
\definecolor{currentfill}{rgb}{0.846140,0.720818,0.730744}%
\pgfsetfillcolor{currentfill}%
\pgfsetlinewidth{0.000000pt}%
\definecolor{currentstroke}{rgb}{0.000000,0.000000,0.000000}%
\pgfsetstrokecolor{currentstroke}%
\pgfsetdash{}{0pt}%
\pgfpathmoveto{\pgfqpoint{1.574327in}{0.311784in}}%
\pgfpathlineto{\pgfqpoint{1.612029in}{0.332609in}}%
\pgfpathlineto{\pgfqpoint{1.574396in}{0.406891in}}%
\pgfpathlineto{\pgfqpoint{1.536486in}{0.386126in}}%
\pgfpathclose%
\pgfusepath{fill}%
\end{pgfscope}%
\begin{pgfscope}%
\pgfpathrectangle{\pgfqpoint{0.150000in}{0.150000in}}{\pgfqpoint{2.700000in}{1.950000in}}%
\pgfusepath{clip}%
\pgfsetbuttcap%
\pgfsetroundjoin%
\definecolor{currentfill}{rgb}{0.524219,0.582812,0.664844}%
\pgfsetfillcolor{currentfill}%
\pgfsetlinewidth{0.000000pt}%
\definecolor{currentstroke}{rgb}{0.000000,0.000000,0.000000}%
\pgfsetstrokecolor{currentstroke}%
\pgfsetdash{}{0pt}%
\pgfpathmoveto{\pgfqpoint{1.878122in}{1.505438in}}%
\pgfpathlineto{\pgfqpoint{1.915384in}{1.524138in}}%
\pgfpathlineto{\pgfqpoint{1.876870in}{1.542769in}}%
\pgfpathlineto{\pgfqpoint{1.839605in}{1.524138in}}%
\pgfpathclose%
\pgfusepath{fill}%
\end{pgfscope}%
\begin{pgfscope}%
\pgfpathrectangle{\pgfqpoint{0.150000in}{0.150000in}}{\pgfqpoint{2.700000in}{1.950000in}}%
\pgfusepath{clip}%
\pgfsetbuttcap%
\pgfsetroundjoin%
\definecolor{currentfill}{rgb}{0.524219,0.582812,0.664844}%
\pgfsetfillcolor{currentfill}%
\pgfsetlinewidth{0.000000pt}%
\definecolor{currentstroke}{rgb}{0.000000,0.000000,0.000000}%
\pgfsetstrokecolor{currentstroke}%
\pgfsetdash{}{0pt}%
\pgfpathmoveto{\pgfqpoint{1.802203in}{1.505438in}}%
\pgfpathlineto{\pgfqpoint{1.839605in}{1.524138in}}%
\pgfpathlineto{\pgfqpoint{1.801229in}{1.542769in}}%
\pgfpathlineto{\pgfqpoint{1.763825in}{1.524138in}}%
\pgfpathclose%
\pgfusepath{fill}%
\end{pgfscope}%
\begin{pgfscope}%
\pgfpathrectangle{\pgfqpoint{0.150000in}{0.150000in}}{\pgfqpoint{2.700000in}{1.950000in}}%
\pgfusepath{clip}%
\pgfsetbuttcap%
\pgfsetroundjoin%
\definecolor{currentfill}{rgb}{0.524219,0.582812,0.664844}%
\pgfsetfillcolor{currentfill}%
\pgfsetlinewidth{0.000000pt}%
\definecolor{currentstroke}{rgb}{0.000000,0.000000,0.000000}%
\pgfsetstrokecolor{currentstroke}%
\pgfsetdash{}{0pt}%
\pgfpathmoveto{\pgfqpoint{1.726284in}{1.505438in}}%
\pgfpathlineto{\pgfqpoint{1.763825in}{1.524138in}}%
\pgfpathlineto{\pgfqpoint{1.725588in}{1.542769in}}%
\pgfpathlineto{\pgfqpoint{1.688046in}{1.524138in}}%
\pgfpathclose%
\pgfusepath{fill}%
\end{pgfscope}%
\begin{pgfscope}%
\pgfpathrectangle{\pgfqpoint{0.150000in}{0.150000in}}{\pgfqpoint{2.700000in}{1.950000in}}%
\pgfusepath{clip}%
\pgfsetbuttcap%
\pgfsetroundjoin%
\definecolor{currentfill}{rgb}{0.524219,0.582812,0.664844}%
\pgfsetfillcolor{currentfill}%
\pgfsetlinewidth{0.000000pt}%
\definecolor{currentstroke}{rgb}{0.000000,0.000000,0.000000}%
\pgfsetstrokecolor{currentstroke}%
\pgfsetdash{}{0pt}%
\pgfpathmoveto{\pgfqpoint{1.650365in}{1.505438in}}%
\pgfpathlineto{\pgfqpoint{1.688046in}{1.524138in}}%
\pgfpathlineto{\pgfqpoint{1.649948in}{1.542769in}}%
\pgfpathlineto{\pgfqpoint{1.612266in}{1.524138in}}%
\pgfpathclose%
\pgfusepath{fill}%
\end{pgfscope}%
\begin{pgfscope}%
\pgfpathrectangle{\pgfqpoint{0.150000in}{0.150000in}}{\pgfqpoint{2.700000in}{1.950000in}}%
\pgfusepath{clip}%
\pgfsetbuttcap%
\pgfsetroundjoin%
\definecolor{currentfill}{rgb}{0.524219,0.582812,0.664844}%
\pgfsetfillcolor{currentfill}%
\pgfsetlinewidth{0.000000pt}%
\definecolor{currentstroke}{rgb}{0.000000,0.000000,0.000000}%
\pgfsetstrokecolor{currentstroke}%
\pgfsetdash{}{0pt}%
\pgfpathmoveto{\pgfqpoint{1.574446in}{1.505438in}}%
\pgfpathlineto{\pgfqpoint{1.612266in}{1.524138in}}%
\pgfpathlineto{\pgfqpoint{1.574307in}{1.542769in}}%
\pgfpathlineto{\pgfqpoint{1.536486in}{1.524138in}}%
\pgfpathclose%
\pgfusepath{fill}%
\end{pgfscope}%
\begin{pgfscope}%
\pgfpathrectangle{\pgfqpoint{0.150000in}{0.150000in}}{\pgfqpoint{2.700000in}{1.950000in}}%
\pgfusepath{clip}%
\pgfsetbuttcap%
\pgfsetroundjoin%
\definecolor{currentfill}{rgb}{0.524219,0.582812,0.664844}%
\pgfsetfillcolor{currentfill}%
\pgfsetlinewidth{0.000000pt}%
\definecolor{currentstroke}{rgb}{0.000000,0.000000,0.000000}%
\pgfsetstrokecolor{currentstroke}%
\pgfsetdash{}{0pt}%
\pgfpathmoveto{\pgfqpoint{1.498527in}{1.505438in}}%
\pgfpathlineto{\pgfqpoint{1.536486in}{1.524138in}}%
\pgfpathlineto{\pgfqpoint{1.498666in}{1.542769in}}%
\pgfpathlineto{\pgfqpoint{1.460707in}{1.524138in}}%
\pgfpathclose%
\pgfusepath{fill}%
\end{pgfscope}%
\begin{pgfscope}%
\pgfpathrectangle{\pgfqpoint{0.150000in}{0.150000in}}{\pgfqpoint{2.700000in}{1.950000in}}%
\pgfusepath{clip}%
\pgfsetbuttcap%
\pgfsetroundjoin%
\definecolor{currentfill}{rgb}{0.524219,0.582812,0.664844}%
\pgfsetfillcolor{currentfill}%
\pgfsetlinewidth{0.000000pt}%
\definecolor{currentstroke}{rgb}{0.000000,0.000000,0.000000}%
\pgfsetstrokecolor{currentstroke}%
\pgfsetdash{}{0pt}%
\pgfpathmoveto{\pgfqpoint{1.422608in}{1.505438in}}%
\pgfpathlineto{\pgfqpoint{1.460707in}{1.524138in}}%
\pgfpathlineto{\pgfqpoint{1.423025in}{1.542769in}}%
\pgfpathlineto{\pgfqpoint{1.384927in}{1.524138in}}%
\pgfpathclose%
\pgfusepath{fill}%
\end{pgfscope}%
\begin{pgfscope}%
\pgfpathrectangle{\pgfqpoint{0.150000in}{0.150000in}}{\pgfqpoint{2.700000in}{1.950000in}}%
\pgfusepath{clip}%
\pgfsetbuttcap%
\pgfsetroundjoin%
\definecolor{currentfill}{rgb}{0.524219,0.582812,0.664844}%
\pgfsetfillcolor{currentfill}%
\pgfsetlinewidth{0.000000pt}%
\definecolor{currentstroke}{rgb}{0.000000,0.000000,0.000000}%
\pgfsetstrokecolor{currentstroke}%
\pgfsetdash{}{0pt}%
\pgfpathmoveto{\pgfqpoint{1.346689in}{1.505438in}}%
\pgfpathlineto{\pgfqpoint{1.384927in}{1.524138in}}%
\pgfpathlineto{\pgfqpoint{1.347385in}{1.542769in}}%
\pgfpathlineto{\pgfqpoint{1.309148in}{1.524138in}}%
\pgfpathclose%
\pgfusepath{fill}%
\end{pgfscope}%
\begin{pgfscope}%
\pgfpathrectangle{\pgfqpoint{0.150000in}{0.150000in}}{\pgfqpoint{2.700000in}{1.950000in}}%
\pgfusepath{clip}%
\pgfsetbuttcap%
\pgfsetroundjoin%
\definecolor{currentfill}{rgb}{0.524219,0.582812,0.664844}%
\pgfsetfillcolor{currentfill}%
\pgfsetlinewidth{0.000000pt}%
\definecolor{currentstroke}{rgb}{0.000000,0.000000,0.000000}%
\pgfsetstrokecolor{currentstroke}%
\pgfsetdash{}{0pt}%
\pgfpathmoveto{\pgfqpoint{1.270770in}{1.505438in}}%
\pgfpathlineto{\pgfqpoint{1.309148in}{1.524138in}}%
\pgfpathlineto{\pgfqpoint{1.271744in}{1.542769in}}%
\pgfpathlineto{\pgfqpoint{1.233368in}{1.524138in}}%
\pgfpathclose%
\pgfusepath{fill}%
\end{pgfscope}%
\begin{pgfscope}%
\pgfpathrectangle{\pgfqpoint{0.150000in}{0.150000in}}{\pgfqpoint{2.700000in}{1.950000in}}%
\pgfusepath{clip}%
\pgfsetbuttcap%
\pgfsetroundjoin%
\definecolor{currentfill}{rgb}{0.524219,0.582812,0.664844}%
\pgfsetfillcolor{currentfill}%
\pgfsetlinewidth{0.000000pt}%
\definecolor{currentstroke}{rgb}{0.000000,0.000000,0.000000}%
\pgfsetstrokecolor{currentstroke}%
\pgfsetdash{}{0pt}%
\pgfpathmoveto{\pgfqpoint{1.194851in}{1.505438in}}%
\pgfpathlineto{\pgfqpoint{1.233368in}{1.524138in}}%
\pgfpathlineto{\pgfqpoint{1.196103in}{1.542769in}}%
\pgfpathlineto{\pgfqpoint{1.157589in}{1.524138in}}%
\pgfpathclose%
\pgfusepath{fill}%
\end{pgfscope}%
\begin{pgfscope}%
\pgfpathrectangle{\pgfqpoint{0.150000in}{0.150000in}}{\pgfqpoint{2.700000in}{1.950000in}}%
\pgfusepath{clip}%
\pgfsetbuttcap%
\pgfsetroundjoin%
\definecolor{currentfill}{rgb}{0.524219,0.582812,0.664844}%
\pgfsetfillcolor{currentfill}%
\pgfsetlinewidth{0.000000pt}%
\definecolor{currentstroke}{rgb}{0.000000,0.000000,0.000000}%
\pgfsetstrokecolor{currentstroke}%
\pgfsetdash{}{0pt}%
\pgfpathmoveto{\pgfqpoint{1.118932in}{1.505438in}}%
\pgfpathlineto{\pgfqpoint{1.157589in}{1.524138in}}%
\pgfpathlineto{\pgfqpoint{1.120462in}{1.542769in}}%
\pgfpathlineto{\pgfqpoint{1.081809in}{1.524138in}}%
\pgfpathclose%
\pgfusepath{fill}%
\end{pgfscope}%
\begin{pgfscope}%
\pgfpathrectangle{\pgfqpoint{0.150000in}{0.150000in}}{\pgfqpoint{2.700000in}{1.950000in}}%
\pgfusepath{clip}%
\pgfsetbuttcap%
\pgfsetroundjoin%
\definecolor{currentfill}{rgb}{0.524219,0.582812,0.664844}%
\pgfsetfillcolor{currentfill}%
\pgfsetlinewidth{0.000000pt}%
\definecolor{currentstroke}{rgb}{0.000000,0.000000,0.000000}%
\pgfsetstrokecolor{currentstroke}%
\pgfsetdash{}{0pt}%
\pgfpathmoveto{\pgfqpoint{1.043013in}{1.505438in}}%
\pgfpathlineto{\pgfqpoint{1.081809in}{1.524138in}}%
\pgfpathlineto{\pgfqpoint{1.044822in}{1.542769in}}%
\pgfpathlineto{\pgfqpoint{1.006029in}{1.524138in}}%
\pgfpathclose%
\pgfusepath{fill}%
\end{pgfscope}%
\begin{pgfscope}%
\pgfpathrectangle{\pgfqpoint{0.150000in}{0.150000in}}{\pgfqpoint{2.700000in}{1.950000in}}%
\pgfusepath{clip}%
\pgfsetbuttcap%
\pgfsetroundjoin%
\definecolor{currentfill}{rgb}{0.524219,0.582812,0.664844}%
\pgfsetfillcolor{currentfill}%
\pgfsetlinewidth{0.000000pt}%
\definecolor{currentstroke}{rgb}{0.000000,0.000000,0.000000}%
\pgfsetstrokecolor{currentstroke}%
\pgfsetdash{}{0pt}%
\pgfpathmoveto{\pgfqpoint{0.967094in}{1.505438in}}%
\pgfpathlineto{\pgfqpoint{1.006029in}{1.524138in}}%
\pgfpathlineto{\pgfqpoint{0.969181in}{1.542769in}}%
\pgfpathlineto{\pgfqpoint{0.930250in}{1.524138in}}%
\pgfpathclose%
\pgfusepath{fill}%
\end{pgfscope}%
\begin{pgfscope}%
\pgfpathrectangle{\pgfqpoint{0.150000in}{0.150000in}}{\pgfqpoint{2.700000in}{1.950000in}}%
\pgfusepath{clip}%
\pgfsetbuttcap%
\pgfsetroundjoin%
\definecolor{currentfill}{rgb}{0.524219,0.582812,0.664844}%
\pgfsetfillcolor{currentfill}%
\pgfsetlinewidth{0.000000pt}%
\definecolor{currentstroke}{rgb}{0.000000,0.000000,0.000000}%
\pgfsetstrokecolor{currentstroke}%
\pgfsetdash{}{0pt}%
\pgfpathmoveto{\pgfqpoint{0.891175in}{1.505438in}}%
\pgfpathlineto{\pgfqpoint{0.930250in}{1.524138in}}%
\pgfpathlineto{\pgfqpoint{0.893540in}{1.542769in}}%
\pgfpathlineto{\pgfqpoint{0.854470in}{1.524138in}}%
\pgfpathclose%
\pgfusepath{fill}%
\end{pgfscope}%
\begin{pgfscope}%
\pgfpathrectangle{\pgfqpoint{0.150000in}{0.150000in}}{\pgfqpoint{2.700000in}{1.950000in}}%
\pgfusepath{clip}%
\pgfsetbuttcap%
\pgfsetroundjoin%
\definecolor{currentfill}{rgb}{0.959574,0.964553,0.971523}%
\pgfsetfillcolor{currentfill}%
\pgfsetlinewidth{0.000000pt}%
\definecolor{currentstroke}{rgb}{0.000000,0.000000,0.000000}%
\pgfsetstrokecolor{currentstroke}%
\pgfsetdash{}{0pt}%
\pgfpathmoveto{\pgfqpoint{1.725677in}{0.774949in}}%
\pgfpathlineto{\pgfqpoint{1.763099in}{0.794926in}}%
\pgfpathlineto{\pgfqpoint{1.726230in}{0.882182in}}%
\pgfpathlineto{\pgfqpoint{1.688561in}{0.862270in}}%
\pgfpathclose%
\pgfusepath{fill}%
\end{pgfscope}%
\begin{pgfscope}%
\pgfpathrectangle{\pgfqpoint{0.150000in}{0.150000in}}{\pgfqpoint{2.700000in}{1.950000in}}%
\pgfusepath{clip}%
\pgfsetbuttcap%
\pgfsetroundjoin%
\definecolor{currentfill}{rgb}{0.586412,0.637347,0.708655}%
\pgfsetfillcolor{currentfill}%
\pgfsetlinewidth{0.000000pt}%
\definecolor{currentstroke}{rgb}{0.000000,0.000000,0.000000}%
\pgfsetstrokecolor{currentstroke}%
\pgfsetdash{}{0pt}%
\pgfpathmoveto{\pgfqpoint{1.915254in}{1.365622in}}%
\pgfpathlineto{\pgfqpoint{1.952367in}{1.384538in}}%
\pgfpathlineto{\pgfqpoint{1.915944in}{1.462743in}}%
\pgfpathlineto{\pgfqpoint{1.878627in}{1.443902in}}%
\pgfpathclose%
\pgfusepath{fill}%
\end{pgfscope}%
\begin{pgfscope}%
\pgfpathrectangle{\pgfqpoint{0.150000in}{0.150000in}}{\pgfqpoint{2.700000in}{1.950000in}}%
\pgfusepath{clip}%
\pgfsetbuttcap%
\pgfsetroundjoin%
\definecolor{currentfill}{rgb}{0.661045,0.702788,0.761229}%
\pgfsetfillcolor{currentfill}%
\pgfsetlinewidth{0.000000pt}%
\definecolor{currentstroke}{rgb}{0.000000,0.000000,0.000000}%
\pgfsetstrokecolor{currentstroke}%
\pgfsetdash{}{0pt}%
\pgfpathmoveto{\pgfqpoint{1.877382in}{1.249727in}}%
\pgfpathlineto{\pgfqpoint{1.914566in}{1.268855in}}%
\pgfpathlineto{\pgfqpoint{1.878004in}{1.346637in}}%
\pgfpathlineto{\pgfqpoint{1.840616in}{1.327583in}}%
\pgfpathclose%
\pgfusepath{fill}%
\end{pgfscope}%
\begin{pgfscope}%
\pgfpathrectangle{\pgfqpoint{0.150000in}{0.150000in}}{\pgfqpoint{2.700000in}{1.950000in}}%
\pgfusepath{clip}%
\pgfsetbuttcap%
\pgfsetroundjoin%
\definecolor{currentfill}{rgb}{0.729458,0.762776,0.809421}%
\pgfsetfillcolor{currentfill}%
\pgfsetlinewidth{0.000000pt}%
\definecolor{currentstroke}{rgb}{0.000000,0.000000,0.000000}%
\pgfsetstrokecolor{currentstroke}%
\pgfsetdash{}{0pt}%
\pgfpathmoveto{\pgfqpoint{1.839509in}{1.133828in}}%
\pgfpathlineto{\pgfqpoint{1.876763in}{1.153169in}}%
\pgfpathlineto{\pgfqpoint{1.840062in}{1.230529in}}%
\pgfpathlineto{\pgfqpoint{1.802604in}{1.211260in}}%
\pgfpathclose%
\pgfusepath{fill}%
\end{pgfscope}%
\begin{pgfscope}%
\pgfpathrectangle{\pgfqpoint{0.150000in}{0.150000in}}{\pgfqpoint{2.700000in}{1.950000in}}%
\pgfusepath{clip}%
\pgfsetbuttcap%
\pgfsetroundjoin%
\definecolor{currentfill}{rgb}{0.804090,0.828217,0.861994}%
\pgfsetfillcolor{currentfill}%
\pgfsetlinewidth{0.000000pt}%
\definecolor{currentstroke}{rgb}{0.000000,0.000000,0.000000}%
\pgfsetstrokecolor{currentstroke}%
\pgfsetdash{}{0pt}%
\pgfpathmoveto{\pgfqpoint{1.801636in}{1.017926in}}%
\pgfpathlineto{\pgfqpoint{1.838959in}{1.037479in}}%
\pgfpathlineto{\pgfqpoint{1.802119in}{1.114417in}}%
\pgfpathlineto{\pgfqpoint{1.764591in}{1.094934in}}%
\pgfpathclose%
\pgfusepath{fill}%
\end{pgfscope}%
\begin{pgfscope}%
\pgfpathrectangle{\pgfqpoint{0.150000in}{0.150000in}}{\pgfqpoint{2.700000in}{1.950000in}}%
\pgfusepath{clip}%
\pgfsetbuttcap%
\pgfsetroundjoin%
\definecolor{currentfill}{rgb}{0.878722,0.893658,0.914568}%
\pgfsetfillcolor{currentfill}%
\pgfsetlinewidth{0.000000pt}%
\definecolor{currentstroke}{rgb}{0.000000,0.000000,0.000000}%
\pgfsetstrokecolor{currentstroke}%
\pgfsetdash{}{0pt}%
\pgfpathmoveto{\pgfqpoint{1.763761in}{0.902020in}}%
\pgfpathlineto{\pgfqpoint{1.801154in}{0.921786in}}%
\pgfpathlineto{\pgfqpoint{1.764175in}{0.998301in}}%
\pgfpathlineto{\pgfqpoint{1.726576in}{0.978604in}}%
\pgfpathclose%
\pgfusepath{fill}%
\end{pgfscope}%
\begin{pgfscope}%
\pgfpathrectangle{\pgfqpoint{0.150000in}{0.150000in}}{\pgfqpoint{2.700000in}{1.950000in}}%
\pgfusepath{clip}%
\pgfsetbuttcap%
\pgfsetroundjoin%
\definecolor{currentfill}{rgb}{0.536657,0.593719,0.673606}%
\pgfsetfillcolor{currentfill}%
\pgfsetlinewidth{0.000000pt}%
\definecolor{currentstroke}{rgb}{0.000000,0.000000,0.000000}%
\pgfsetstrokecolor{currentstroke}%
\pgfsetdash{}{0pt}%
\pgfpathmoveto{\pgfqpoint{1.915944in}{1.462743in}}%
\pgfpathlineto{\pgfqpoint{1.953124in}{1.481514in}}%
\pgfpathlineto{\pgfqpoint{1.915384in}{1.524138in}}%
\pgfpathlineto{\pgfqpoint{1.878122in}{1.505438in}}%
\pgfpathclose%
\pgfusepath{fill}%
\end{pgfscope}%
\begin{pgfscope}%
\pgfpathrectangle{\pgfqpoint{0.150000in}{0.150000in}}{\pgfqpoint{2.700000in}{1.950000in}}%
\pgfusepath{clip}%
\pgfsetbuttcap%
\pgfsetroundjoin%
\definecolor{currentfill}{rgb}{0.979105,0.962086,0.963434}%
\pgfsetfillcolor{currentfill}%
\pgfsetlinewidth{0.000000pt}%
\definecolor{currentstroke}{rgb}{0.000000,0.000000,0.000000}%
\pgfsetstrokecolor{currentstroke}%
\pgfsetdash{}{0pt}%
\pgfpathmoveto{\pgfqpoint{1.650211in}{0.638900in}}%
\pgfpathlineto{\pgfqpoint{1.687841in}{0.659163in}}%
\pgfpathlineto{\pgfqpoint{1.650418in}{0.734775in}}%
\pgfpathlineto{\pgfqpoint{1.612581in}{0.714577in}}%
\pgfpathclose%
\pgfusepath{fill}%
\end{pgfscope}%
\begin{pgfscope}%
\pgfpathrectangle{\pgfqpoint{0.150000in}{0.150000in}}{\pgfqpoint{2.700000in}{1.950000in}}%
\pgfusepath{clip}%
\pgfsetbuttcap%
\pgfsetroundjoin%
\definecolor{currentfill}{rgb}{0.933517,0.879366,0.883655}%
\pgfsetfillcolor{currentfill}%
\pgfsetlinewidth{0.000000pt}%
\definecolor{currentstroke}{rgb}{0.000000,0.000000,0.000000}%
\pgfsetstrokecolor{currentstroke}%
\pgfsetdash{}{0pt}%
\pgfpathmoveto{\pgfqpoint{1.612304in}{0.522897in}}%
\pgfpathlineto{\pgfqpoint{1.650004in}{0.543374in}}%
\pgfpathlineto{\pgfqpoint{1.612442in}{0.618562in}}%
\pgfpathlineto{\pgfqpoint{1.574534in}{0.598150in}}%
\pgfpathclose%
\pgfusepath{fill}%
\end{pgfscope}%
\begin{pgfscope}%
\pgfpathrectangle{\pgfqpoint{0.150000in}{0.150000in}}{\pgfqpoint{2.700000in}{1.950000in}}%
\pgfusepath{clip}%
\pgfsetbuttcap%
\pgfsetroundjoin%
\definecolor{currentfill}{rgb}{0.887929,0.796645,0.803876}%
\pgfsetfillcolor{currentfill}%
\pgfsetlinewidth{0.000000pt}%
\definecolor{currentstroke}{rgb}{0.000000,0.000000,0.000000}%
\pgfsetstrokecolor{currentstroke}%
\pgfsetdash{}{0pt}%
\pgfpathmoveto{\pgfqpoint{1.574396in}{0.406891in}}%
\pgfpathlineto{\pgfqpoint{1.612166in}{0.427580in}}%
\pgfpathlineto{\pgfqpoint{1.574465in}{0.502346in}}%
\pgfpathlineto{\pgfqpoint{1.536486in}{0.481719in}}%
\pgfpathclose%
\pgfusepath{fill}%
\end{pgfscope}%
\begin{pgfscope}%
\pgfpathrectangle{\pgfqpoint{0.150000in}{0.150000in}}{\pgfqpoint{2.700000in}{1.950000in}}%
\pgfusepath{clip}%
\pgfsetbuttcap%
\pgfsetroundjoin%
\definecolor{currentfill}{rgb}{0.846140,0.720818,0.730744}%
\pgfsetfillcolor{currentfill}%
\pgfsetlinewidth{0.000000pt}%
\definecolor{currentstroke}{rgb}{0.000000,0.000000,0.000000}%
\pgfsetstrokecolor{currentstroke}%
\pgfsetdash{}{0pt}%
\pgfpathmoveto{\pgfqpoint{1.536486in}{0.290881in}}%
\pgfpathlineto{\pgfqpoint{1.574327in}{0.311784in}}%
\pgfpathlineto{\pgfqpoint{1.536486in}{0.386126in}}%
\pgfpathlineto{\pgfqpoint{1.498437in}{0.365284in}}%
\pgfpathclose%
\pgfusepath{fill}%
\end{pgfscope}%
\begin{pgfscope}%
\pgfpathrectangle{\pgfqpoint{0.150000in}{0.150000in}}{\pgfqpoint{2.700000in}{1.950000in}}%
\pgfusepath{clip}%
\pgfsetbuttcap%
\pgfsetroundjoin%
\definecolor{currentfill}{rgb}{0.524219,0.582812,0.664844}%
\pgfsetfillcolor{currentfill}%
\pgfsetlinewidth{0.000000pt}%
\definecolor{currentstroke}{rgb}{0.000000,0.000000,0.000000}%
\pgfsetstrokecolor{currentstroke}%
\pgfsetdash{}{0pt}%
\pgfpathmoveto{\pgfqpoint{1.840722in}{1.486669in}}%
\pgfpathlineto{\pgfqpoint{1.878122in}{1.505438in}}%
\pgfpathlineto{\pgfqpoint{1.839605in}{1.524138in}}%
\pgfpathlineto{\pgfqpoint{1.802203in}{1.505438in}}%
\pgfpathclose%
\pgfusepath{fill}%
\end{pgfscope}%
\begin{pgfscope}%
\pgfpathrectangle{\pgfqpoint{0.150000in}{0.150000in}}{\pgfqpoint{2.700000in}{1.950000in}}%
\pgfusepath{clip}%
\pgfsetbuttcap%
\pgfsetroundjoin%
\definecolor{currentfill}{rgb}{0.524219,0.582812,0.664844}%
\pgfsetfillcolor{currentfill}%
\pgfsetlinewidth{0.000000pt}%
\definecolor{currentstroke}{rgb}{0.000000,0.000000,0.000000}%
\pgfsetstrokecolor{currentstroke}%
\pgfsetdash{}{0pt}%
\pgfpathmoveto{\pgfqpoint{1.764663in}{1.486669in}}%
\pgfpathlineto{\pgfqpoint{1.802203in}{1.505438in}}%
\pgfpathlineto{\pgfqpoint{1.763825in}{1.524138in}}%
\pgfpathlineto{\pgfqpoint{1.726284in}{1.505438in}}%
\pgfpathclose%
\pgfusepath{fill}%
\end{pgfscope}%
\begin{pgfscope}%
\pgfpathrectangle{\pgfqpoint{0.150000in}{0.150000in}}{\pgfqpoint{2.700000in}{1.950000in}}%
\pgfusepath{clip}%
\pgfsetbuttcap%
\pgfsetroundjoin%
\definecolor{currentfill}{rgb}{0.524219,0.582812,0.664844}%
\pgfsetfillcolor{currentfill}%
\pgfsetlinewidth{0.000000pt}%
\definecolor{currentstroke}{rgb}{0.000000,0.000000,0.000000}%
\pgfsetstrokecolor{currentstroke}%
\pgfsetdash{}{0pt}%
\pgfpathmoveto{\pgfqpoint{1.688604in}{1.486669in}}%
\pgfpathlineto{\pgfqpoint{1.726284in}{1.505438in}}%
\pgfpathlineto{\pgfqpoint{1.688046in}{1.524138in}}%
\pgfpathlineto{\pgfqpoint{1.650365in}{1.505438in}}%
\pgfpathclose%
\pgfusepath{fill}%
\end{pgfscope}%
\begin{pgfscope}%
\pgfpathrectangle{\pgfqpoint{0.150000in}{0.150000in}}{\pgfqpoint{2.700000in}{1.950000in}}%
\pgfusepath{clip}%
\pgfsetbuttcap%
\pgfsetroundjoin%
\definecolor{currentfill}{rgb}{0.524219,0.582812,0.664844}%
\pgfsetfillcolor{currentfill}%
\pgfsetlinewidth{0.000000pt}%
\definecolor{currentstroke}{rgb}{0.000000,0.000000,0.000000}%
\pgfsetstrokecolor{currentstroke}%
\pgfsetdash{}{0pt}%
\pgfpathmoveto{\pgfqpoint{1.612545in}{1.486669in}}%
\pgfpathlineto{\pgfqpoint{1.650365in}{1.505438in}}%
\pgfpathlineto{\pgfqpoint{1.612266in}{1.524138in}}%
\pgfpathlineto{\pgfqpoint{1.574446in}{1.505438in}}%
\pgfpathclose%
\pgfusepath{fill}%
\end{pgfscope}%
\begin{pgfscope}%
\pgfpathrectangle{\pgfqpoint{0.150000in}{0.150000in}}{\pgfqpoint{2.700000in}{1.950000in}}%
\pgfusepath{clip}%
\pgfsetbuttcap%
\pgfsetroundjoin%
\definecolor{currentfill}{rgb}{0.524219,0.582812,0.664844}%
\pgfsetfillcolor{currentfill}%
\pgfsetlinewidth{0.000000pt}%
\definecolor{currentstroke}{rgb}{0.000000,0.000000,0.000000}%
\pgfsetstrokecolor{currentstroke}%
\pgfsetdash{}{0pt}%
\pgfpathmoveto{\pgfqpoint{1.536486in}{1.486669in}}%
\pgfpathlineto{\pgfqpoint{1.574446in}{1.505438in}}%
\pgfpathlineto{\pgfqpoint{1.536486in}{1.524138in}}%
\pgfpathlineto{\pgfqpoint{1.498527in}{1.505438in}}%
\pgfpathclose%
\pgfusepath{fill}%
\end{pgfscope}%
\begin{pgfscope}%
\pgfpathrectangle{\pgfqpoint{0.150000in}{0.150000in}}{\pgfqpoint{2.700000in}{1.950000in}}%
\pgfusepath{clip}%
\pgfsetbuttcap%
\pgfsetroundjoin%
\definecolor{currentfill}{rgb}{0.524219,0.582812,0.664844}%
\pgfsetfillcolor{currentfill}%
\pgfsetlinewidth{0.000000pt}%
\definecolor{currentstroke}{rgb}{0.000000,0.000000,0.000000}%
\pgfsetstrokecolor{currentstroke}%
\pgfsetdash{}{0pt}%
\pgfpathmoveto{\pgfqpoint{1.460428in}{1.486669in}}%
\pgfpathlineto{\pgfqpoint{1.498527in}{1.505438in}}%
\pgfpathlineto{\pgfqpoint{1.460707in}{1.524138in}}%
\pgfpathlineto{\pgfqpoint{1.422608in}{1.505438in}}%
\pgfpathclose%
\pgfusepath{fill}%
\end{pgfscope}%
\begin{pgfscope}%
\pgfpathrectangle{\pgfqpoint{0.150000in}{0.150000in}}{\pgfqpoint{2.700000in}{1.950000in}}%
\pgfusepath{clip}%
\pgfsetbuttcap%
\pgfsetroundjoin%
\definecolor{currentfill}{rgb}{0.524219,0.582812,0.664844}%
\pgfsetfillcolor{currentfill}%
\pgfsetlinewidth{0.000000pt}%
\definecolor{currentstroke}{rgb}{0.000000,0.000000,0.000000}%
\pgfsetstrokecolor{currentstroke}%
\pgfsetdash{}{0pt}%
\pgfpathmoveto{\pgfqpoint{1.384369in}{1.486669in}}%
\pgfpathlineto{\pgfqpoint{1.422608in}{1.505438in}}%
\pgfpathlineto{\pgfqpoint{1.384927in}{1.524138in}}%
\pgfpathlineto{\pgfqpoint{1.346689in}{1.505438in}}%
\pgfpathclose%
\pgfusepath{fill}%
\end{pgfscope}%
\begin{pgfscope}%
\pgfpathrectangle{\pgfqpoint{0.150000in}{0.150000in}}{\pgfqpoint{2.700000in}{1.950000in}}%
\pgfusepath{clip}%
\pgfsetbuttcap%
\pgfsetroundjoin%
\definecolor{currentfill}{rgb}{0.524219,0.582812,0.664844}%
\pgfsetfillcolor{currentfill}%
\pgfsetlinewidth{0.000000pt}%
\definecolor{currentstroke}{rgb}{0.000000,0.000000,0.000000}%
\pgfsetstrokecolor{currentstroke}%
\pgfsetdash{}{0pt}%
\pgfpathmoveto{\pgfqpoint{1.308310in}{1.486669in}}%
\pgfpathlineto{\pgfqpoint{1.346689in}{1.505438in}}%
\pgfpathlineto{\pgfqpoint{1.309148in}{1.524138in}}%
\pgfpathlineto{\pgfqpoint{1.270770in}{1.505438in}}%
\pgfpathclose%
\pgfusepath{fill}%
\end{pgfscope}%
\begin{pgfscope}%
\pgfpathrectangle{\pgfqpoint{0.150000in}{0.150000in}}{\pgfqpoint{2.700000in}{1.950000in}}%
\pgfusepath{clip}%
\pgfsetbuttcap%
\pgfsetroundjoin%
\definecolor{currentfill}{rgb}{0.524219,0.582812,0.664844}%
\pgfsetfillcolor{currentfill}%
\pgfsetlinewidth{0.000000pt}%
\definecolor{currentstroke}{rgb}{0.000000,0.000000,0.000000}%
\pgfsetstrokecolor{currentstroke}%
\pgfsetdash{}{0pt}%
\pgfpathmoveto{\pgfqpoint{1.232251in}{1.486669in}}%
\pgfpathlineto{\pgfqpoint{1.270770in}{1.505438in}}%
\pgfpathlineto{\pgfqpoint{1.233368in}{1.524138in}}%
\pgfpathlineto{\pgfqpoint{1.194851in}{1.505438in}}%
\pgfpathclose%
\pgfusepath{fill}%
\end{pgfscope}%
\begin{pgfscope}%
\pgfpathrectangle{\pgfqpoint{0.150000in}{0.150000in}}{\pgfqpoint{2.700000in}{1.950000in}}%
\pgfusepath{clip}%
\pgfsetbuttcap%
\pgfsetroundjoin%
\definecolor{currentfill}{rgb}{0.524219,0.582812,0.664844}%
\pgfsetfillcolor{currentfill}%
\pgfsetlinewidth{0.000000pt}%
\definecolor{currentstroke}{rgb}{0.000000,0.000000,0.000000}%
\pgfsetstrokecolor{currentstroke}%
\pgfsetdash{}{0pt}%
\pgfpathmoveto{\pgfqpoint{1.156192in}{1.486669in}}%
\pgfpathlineto{\pgfqpoint{1.194851in}{1.505438in}}%
\pgfpathlineto{\pgfqpoint{1.157589in}{1.524138in}}%
\pgfpathlineto{\pgfqpoint{1.118932in}{1.505438in}}%
\pgfpathclose%
\pgfusepath{fill}%
\end{pgfscope}%
\begin{pgfscope}%
\pgfpathrectangle{\pgfqpoint{0.150000in}{0.150000in}}{\pgfqpoint{2.700000in}{1.950000in}}%
\pgfusepath{clip}%
\pgfsetbuttcap%
\pgfsetroundjoin%
\definecolor{currentfill}{rgb}{0.524219,0.582812,0.664844}%
\pgfsetfillcolor{currentfill}%
\pgfsetlinewidth{0.000000pt}%
\definecolor{currentstroke}{rgb}{0.000000,0.000000,0.000000}%
\pgfsetstrokecolor{currentstroke}%
\pgfsetdash{}{0pt}%
\pgfpathmoveto{\pgfqpoint{1.080133in}{1.486669in}}%
\pgfpathlineto{\pgfqpoint{1.118932in}{1.505438in}}%
\pgfpathlineto{\pgfqpoint{1.081809in}{1.524138in}}%
\pgfpathlineto{\pgfqpoint{1.043013in}{1.505438in}}%
\pgfpathclose%
\pgfusepath{fill}%
\end{pgfscope}%
\begin{pgfscope}%
\pgfpathrectangle{\pgfqpoint{0.150000in}{0.150000in}}{\pgfqpoint{2.700000in}{1.950000in}}%
\pgfusepath{clip}%
\pgfsetbuttcap%
\pgfsetroundjoin%
\definecolor{currentfill}{rgb}{0.524219,0.582812,0.664844}%
\pgfsetfillcolor{currentfill}%
\pgfsetlinewidth{0.000000pt}%
\definecolor{currentstroke}{rgb}{0.000000,0.000000,0.000000}%
\pgfsetstrokecolor{currentstroke}%
\pgfsetdash{}{0pt}%
\pgfpathmoveto{\pgfqpoint{1.004075in}{1.486669in}}%
\pgfpathlineto{\pgfqpoint{1.043013in}{1.505438in}}%
\pgfpathlineto{\pgfqpoint{1.006029in}{1.524138in}}%
\pgfpathlineto{\pgfqpoint{0.967094in}{1.505438in}}%
\pgfpathclose%
\pgfusepath{fill}%
\end{pgfscope}%
\begin{pgfscope}%
\pgfpathrectangle{\pgfqpoint{0.150000in}{0.150000in}}{\pgfqpoint{2.700000in}{1.950000in}}%
\pgfusepath{clip}%
\pgfsetbuttcap%
\pgfsetroundjoin%
\definecolor{currentfill}{rgb}{0.524219,0.582812,0.664844}%
\pgfsetfillcolor{currentfill}%
\pgfsetlinewidth{0.000000pt}%
\definecolor{currentstroke}{rgb}{0.000000,0.000000,0.000000}%
\pgfsetstrokecolor{currentstroke}%
\pgfsetdash{}{0pt}%
\pgfpathmoveto{\pgfqpoint{0.928016in}{1.486669in}}%
\pgfpathlineto{\pgfqpoint{0.967094in}{1.505438in}}%
\pgfpathlineto{\pgfqpoint{0.930250in}{1.524138in}}%
\pgfpathlineto{\pgfqpoint{0.891175in}{1.505438in}}%
\pgfpathclose%
\pgfusepath{fill}%
\end{pgfscope}%
\begin{pgfscope}%
\pgfpathrectangle{\pgfqpoint{0.150000in}{0.150000in}}{\pgfqpoint{2.700000in}{1.950000in}}%
\pgfusepath{clip}%
\pgfsetbuttcap%
\pgfsetroundjoin%
\definecolor{currentfill}{rgb}{0.524219,0.582812,0.664844}%
\pgfsetfillcolor{currentfill}%
\pgfsetlinewidth{0.000000pt}%
\definecolor{currentstroke}{rgb}{0.000000,0.000000,0.000000}%
\pgfsetstrokecolor{currentstroke}%
\pgfsetdash{}{0pt}%
\pgfpathmoveto{\pgfqpoint{0.851957in}{1.486669in}}%
\pgfpathlineto{\pgfqpoint{0.891175in}{1.505438in}}%
\pgfpathlineto{\pgfqpoint{0.854470in}{1.524138in}}%
\pgfpathlineto{\pgfqpoint{0.815256in}{1.505438in}}%
\pgfpathclose%
\pgfusepath{fill}%
\end{pgfscope}%
\begin{pgfscope}%
\pgfpathrectangle{\pgfqpoint{0.150000in}{0.150000in}}{\pgfqpoint{2.700000in}{1.950000in}}%
\pgfusepath{clip}%
\pgfsetbuttcap%
\pgfsetroundjoin%
\definecolor{currentfill}{rgb}{0.959574,0.964553,0.971523}%
\pgfsetfillcolor{currentfill}%
\pgfsetlinewidth{0.000000pt}%
\definecolor{currentstroke}{rgb}{0.000000,0.000000,0.000000}%
\pgfsetstrokecolor{currentstroke}%
\pgfsetdash{}{0pt}%
\pgfpathmoveto{\pgfqpoint{1.688117in}{0.754899in}}%
\pgfpathlineto{\pgfqpoint{1.725677in}{0.774949in}}%
\pgfpathlineto{\pgfqpoint{1.688561in}{0.862270in}}%
\pgfpathlineto{\pgfqpoint{1.650753in}{0.842285in}}%
\pgfpathclose%
\pgfusepath{fill}%
\end{pgfscope}%
\begin{pgfscope}%
\pgfpathrectangle{\pgfqpoint{0.150000in}{0.150000in}}{\pgfqpoint{2.700000in}{1.950000in}}%
\pgfusepath{clip}%
\pgfsetbuttcap%
\pgfsetroundjoin%
\definecolor{currentfill}{rgb}{0.586412,0.637347,0.708655}%
\pgfsetfillcolor{currentfill}%
\pgfsetlinewidth{0.000000pt}%
\definecolor{currentstroke}{rgb}{0.000000,0.000000,0.000000}%
\pgfsetstrokecolor{currentstroke}%
\pgfsetdash{}{0pt}%
\pgfpathmoveto{\pgfqpoint{1.878004in}{1.346637in}}%
\pgfpathlineto{\pgfqpoint{1.915254in}{1.365622in}}%
\pgfpathlineto{\pgfqpoint{1.878627in}{1.443902in}}%
\pgfpathlineto{\pgfqpoint{1.841173in}{1.424992in}}%
\pgfpathclose%
\pgfusepath{fill}%
\end{pgfscope}%
\begin{pgfscope}%
\pgfpathrectangle{\pgfqpoint{0.150000in}{0.150000in}}{\pgfqpoint{2.700000in}{1.950000in}}%
\pgfusepath{clip}%
\pgfsetbuttcap%
\pgfsetroundjoin%
\definecolor{currentfill}{rgb}{0.661045,0.702788,0.761229}%
\pgfsetfillcolor{currentfill}%
\pgfsetlinewidth{0.000000pt}%
\definecolor{currentstroke}{rgb}{0.000000,0.000000,0.000000}%
\pgfsetstrokecolor{currentstroke}%
\pgfsetdash{}{0pt}%
\pgfpathmoveto{\pgfqpoint{1.840062in}{1.230529in}}%
\pgfpathlineto{\pgfqpoint{1.877382in}{1.249727in}}%
\pgfpathlineto{\pgfqpoint{1.840616in}{1.327583in}}%
\pgfpathlineto{\pgfqpoint{1.803091in}{1.308458in}}%
\pgfpathclose%
\pgfusepath{fill}%
\end{pgfscope}%
\begin{pgfscope}%
\pgfpathrectangle{\pgfqpoint{0.150000in}{0.150000in}}{\pgfqpoint{2.700000in}{1.950000in}}%
\pgfusepath{clip}%
\pgfsetbuttcap%
\pgfsetroundjoin%
\definecolor{currentfill}{rgb}{0.729458,0.762776,0.809421}%
\pgfsetfillcolor{currentfill}%
\pgfsetlinewidth{0.000000pt}%
\definecolor{currentstroke}{rgb}{0.000000,0.000000,0.000000}%
\pgfsetstrokecolor{currentstroke}%
\pgfsetdash{}{0pt}%
\pgfpathmoveto{\pgfqpoint{1.802119in}{1.114417in}}%
\pgfpathlineto{\pgfqpoint{1.839509in}{1.133828in}}%
\pgfpathlineto{\pgfqpoint{1.802604in}{1.211260in}}%
\pgfpathlineto{\pgfqpoint{1.765008in}{1.191920in}}%
\pgfpathclose%
\pgfusepath{fill}%
\end{pgfscope}%
\begin{pgfscope}%
\pgfpathrectangle{\pgfqpoint{0.150000in}{0.150000in}}{\pgfqpoint{2.700000in}{1.950000in}}%
\pgfusepath{clip}%
\pgfsetbuttcap%
\pgfsetroundjoin%
\definecolor{currentfill}{rgb}{0.804090,0.828217,0.861994}%
\pgfsetfillcolor{currentfill}%
\pgfsetlinewidth{0.000000pt}%
\definecolor{currentstroke}{rgb}{0.000000,0.000000,0.000000}%
\pgfsetstrokecolor{currentstroke}%
\pgfsetdash{}{0pt}%
\pgfpathmoveto{\pgfqpoint{1.764175in}{0.998301in}}%
\pgfpathlineto{\pgfqpoint{1.801636in}{1.017926in}}%
\pgfpathlineto{\pgfqpoint{1.764591in}{1.094934in}}%
\pgfpathlineto{\pgfqpoint{1.726924in}{1.075379in}}%
\pgfpathclose%
\pgfusepath{fill}%
\end{pgfscope}%
\begin{pgfscope}%
\pgfpathrectangle{\pgfqpoint{0.150000in}{0.150000in}}{\pgfqpoint{2.700000in}{1.950000in}}%
\pgfusepath{clip}%
\pgfsetbuttcap%
\pgfsetroundjoin%
\definecolor{currentfill}{rgb}{0.878722,0.893658,0.914568}%
\pgfsetfillcolor{currentfill}%
\pgfsetlinewidth{0.000000pt}%
\definecolor{currentstroke}{rgb}{0.000000,0.000000,0.000000}%
\pgfsetstrokecolor{currentstroke}%
\pgfsetdash{}{0pt}%
\pgfpathmoveto{\pgfqpoint{1.726230in}{0.882182in}}%
\pgfpathlineto{\pgfqpoint{1.763761in}{0.902020in}}%
\pgfpathlineto{\pgfqpoint{1.726576in}{0.978604in}}%
\pgfpathlineto{\pgfqpoint{1.688839in}{0.958834in}}%
\pgfpathclose%
\pgfusepath{fill}%
\end{pgfscope}%
\begin{pgfscope}%
\pgfpathrectangle{\pgfqpoint{0.150000in}{0.150000in}}{\pgfqpoint{2.700000in}{1.950000in}}%
\pgfusepath{clip}%
\pgfsetbuttcap%
\pgfsetroundjoin%
\definecolor{currentfill}{rgb}{0.536657,0.593719,0.673606}%
\pgfsetfillcolor{currentfill}%
\pgfsetlinewidth{0.000000pt}%
\definecolor{currentstroke}{rgb}{0.000000,0.000000,0.000000}%
\pgfsetstrokecolor{currentstroke}%
\pgfsetdash{}{0pt}%
\pgfpathmoveto{\pgfqpoint{1.878627in}{1.443902in}}%
\pgfpathlineto{\pgfqpoint{1.915944in}{1.462743in}}%
\pgfpathlineto{\pgfqpoint{1.878122in}{1.505438in}}%
\pgfpathlineto{\pgfqpoint{1.840722in}{1.486669in}}%
\pgfpathclose%
\pgfusepath{fill}%
\end{pgfscope}%
\begin{pgfscope}%
\pgfpathrectangle{\pgfqpoint{0.150000in}{0.150000in}}{\pgfqpoint{2.700000in}{1.950000in}}%
\pgfusepath{clip}%
\pgfsetbuttcap%
\pgfsetroundjoin%
\definecolor{currentfill}{rgb}{0.979105,0.962086,0.963434}%
\pgfsetfillcolor{currentfill}%
\pgfsetlinewidth{0.000000pt}%
\definecolor{currentstroke}{rgb}{0.000000,0.000000,0.000000}%
\pgfsetstrokecolor{currentstroke}%
\pgfsetdash{}{0pt}%
\pgfpathmoveto{\pgfqpoint{1.612442in}{0.618562in}}%
\pgfpathlineto{\pgfqpoint{1.650211in}{0.638900in}}%
\pgfpathlineto{\pgfqpoint{1.612581in}{0.714577in}}%
\pgfpathlineto{\pgfqpoint{1.574604in}{0.694304in}}%
\pgfpathclose%
\pgfusepath{fill}%
\end{pgfscope}%
\begin{pgfscope}%
\pgfpathrectangle{\pgfqpoint{0.150000in}{0.150000in}}{\pgfqpoint{2.700000in}{1.950000in}}%
\pgfusepath{clip}%
\pgfsetbuttcap%
\pgfsetroundjoin%
\definecolor{currentfill}{rgb}{0.933517,0.879366,0.883655}%
\pgfsetfillcolor{currentfill}%
\pgfsetlinewidth{0.000000pt}%
\definecolor{currentstroke}{rgb}{0.000000,0.000000,0.000000}%
\pgfsetstrokecolor{currentstroke}%
\pgfsetdash{}{0pt}%
\pgfpathmoveto{\pgfqpoint{1.574465in}{0.502346in}}%
\pgfpathlineto{\pgfqpoint{1.612304in}{0.522897in}}%
\pgfpathlineto{\pgfqpoint{1.574534in}{0.598150in}}%
\pgfpathlineto{\pgfqpoint{1.536486in}{0.577661in}}%
\pgfpathclose%
\pgfusepath{fill}%
\end{pgfscope}%
\begin{pgfscope}%
\pgfpathrectangle{\pgfqpoint{0.150000in}{0.150000in}}{\pgfqpoint{2.700000in}{1.950000in}}%
\pgfusepath{clip}%
\pgfsetbuttcap%
\pgfsetroundjoin%
\definecolor{currentfill}{rgb}{0.887929,0.796645,0.803876}%
\pgfsetfillcolor{currentfill}%
\pgfsetlinewidth{0.000000pt}%
\definecolor{currentstroke}{rgb}{0.000000,0.000000,0.000000}%
\pgfsetstrokecolor{currentstroke}%
\pgfsetdash{}{0pt}%
\pgfpathmoveto{\pgfqpoint{1.536486in}{0.386126in}}%
\pgfpathlineto{\pgfqpoint{1.574396in}{0.406891in}}%
\pgfpathlineto{\pgfqpoint{1.536486in}{0.481719in}}%
\pgfpathlineto{\pgfqpoint{1.498368in}{0.461015in}}%
\pgfpathclose%
\pgfusepath{fill}%
\end{pgfscope}%
\begin{pgfscope}%
\pgfpathrectangle{\pgfqpoint{0.150000in}{0.150000in}}{\pgfqpoint{2.700000in}{1.950000in}}%
\pgfusepath{clip}%
\pgfsetbuttcap%
\pgfsetroundjoin%
\definecolor{currentfill}{rgb}{0.524219,0.582812,0.664844}%
\pgfsetfillcolor{currentfill}%
\pgfsetlinewidth{0.000000pt}%
\definecolor{currentstroke}{rgb}{0.000000,0.000000,0.000000}%
\pgfsetstrokecolor{currentstroke}%
\pgfsetdash{}{0pt}%
\pgfpathmoveto{\pgfqpoint{1.803184in}{1.467831in}}%
\pgfpathlineto{\pgfqpoint{1.840722in}{1.486669in}}%
\pgfpathlineto{\pgfqpoint{1.802203in}{1.505438in}}%
\pgfpathlineto{\pgfqpoint{1.764663in}{1.486669in}}%
\pgfpathclose%
\pgfusepath{fill}%
\end{pgfscope}%
\begin{pgfscope}%
\pgfpathrectangle{\pgfqpoint{0.150000in}{0.150000in}}{\pgfqpoint{2.700000in}{1.950000in}}%
\pgfusepath{clip}%
\pgfsetbuttcap%
\pgfsetroundjoin%
\definecolor{currentfill}{rgb}{0.524219,0.582812,0.664844}%
\pgfsetfillcolor{currentfill}%
\pgfsetlinewidth{0.000000pt}%
\definecolor{currentstroke}{rgb}{0.000000,0.000000,0.000000}%
\pgfsetstrokecolor{currentstroke}%
\pgfsetdash{}{0pt}%
\pgfpathmoveto{\pgfqpoint{1.726985in}{1.467831in}}%
\pgfpathlineto{\pgfqpoint{1.764663in}{1.486669in}}%
\pgfpathlineto{\pgfqpoint{1.726284in}{1.505438in}}%
\pgfpathlineto{\pgfqpoint{1.688604in}{1.486669in}}%
\pgfpathclose%
\pgfusepath{fill}%
\end{pgfscope}%
\begin{pgfscope}%
\pgfpathrectangle{\pgfqpoint{0.150000in}{0.150000in}}{\pgfqpoint{2.700000in}{1.950000in}}%
\pgfusepath{clip}%
\pgfsetbuttcap%
\pgfsetroundjoin%
\definecolor{currentfill}{rgb}{0.524219,0.582812,0.664844}%
\pgfsetfillcolor{currentfill}%
\pgfsetlinewidth{0.000000pt}%
\definecolor{currentstroke}{rgb}{0.000000,0.000000,0.000000}%
\pgfsetstrokecolor{currentstroke}%
\pgfsetdash{}{0pt}%
\pgfpathmoveto{\pgfqpoint{1.650785in}{1.467831in}}%
\pgfpathlineto{\pgfqpoint{1.688604in}{1.486669in}}%
\pgfpathlineto{\pgfqpoint{1.650365in}{1.505438in}}%
\pgfpathlineto{\pgfqpoint{1.612545in}{1.486669in}}%
\pgfpathclose%
\pgfusepath{fill}%
\end{pgfscope}%
\begin{pgfscope}%
\pgfpathrectangle{\pgfqpoint{0.150000in}{0.150000in}}{\pgfqpoint{2.700000in}{1.950000in}}%
\pgfusepath{clip}%
\pgfsetbuttcap%
\pgfsetroundjoin%
\definecolor{currentfill}{rgb}{0.524219,0.582812,0.664844}%
\pgfsetfillcolor{currentfill}%
\pgfsetlinewidth{0.000000pt}%
\definecolor{currentstroke}{rgb}{0.000000,0.000000,0.000000}%
\pgfsetstrokecolor{currentstroke}%
\pgfsetdash{}{0pt}%
\pgfpathmoveto{\pgfqpoint{1.574586in}{1.467831in}}%
\pgfpathlineto{\pgfqpoint{1.612545in}{1.486669in}}%
\pgfpathlineto{\pgfqpoint{1.574446in}{1.505438in}}%
\pgfpathlineto{\pgfqpoint{1.536486in}{1.486669in}}%
\pgfpathclose%
\pgfusepath{fill}%
\end{pgfscope}%
\begin{pgfscope}%
\pgfpathrectangle{\pgfqpoint{0.150000in}{0.150000in}}{\pgfqpoint{2.700000in}{1.950000in}}%
\pgfusepath{clip}%
\pgfsetbuttcap%
\pgfsetroundjoin%
\definecolor{currentfill}{rgb}{0.524219,0.582812,0.664844}%
\pgfsetfillcolor{currentfill}%
\pgfsetlinewidth{0.000000pt}%
\definecolor{currentstroke}{rgb}{0.000000,0.000000,0.000000}%
\pgfsetstrokecolor{currentstroke}%
\pgfsetdash{}{0pt}%
\pgfpathmoveto{\pgfqpoint{1.498387in}{1.467831in}}%
\pgfpathlineto{\pgfqpoint{1.536486in}{1.486669in}}%
\pgfpathlineto{\pgfqpoint{1.498527in}{1.505438in}}%
\pgfpathlineto{\pgfqpoint{1.460428in}{1.486669in}}%
\pgfpathclose%
\pgfusepath{fill}%
\end{pgfscope}%
\begin{pgfscope}%
\pgfpathrectangle{\pgfqpoint{0.150000in}{0.150000in}}{\pgfqpoint{2.700000in}{1.950000in}}%
\pgfusepath{clip}%
\pgfsetbuttcap%
\pgfsetroundjoin%
\definecolor{currentfill}{rgb}{0.524219,0.582812,0.664844}%
\pgfsetfillcolor{currentfill}%
\pgfsetlinewidth{0.000000pt}%
\definecolor{currentstroke}{rgb}{0.000000,0.000000,0.000000}%
\pgfsetstrokecolor{currentstroke}%
\pgfsetdash{}{0pt}%
\pgfpathmoveto{\pgfqpoint{1.422188in}{1.467831in}}%
\pgfpathlineto{\pgfqpoint{1.460428in}{1.486669in}}%
\pgfpathlineto{\pgfqpoint{1.422608in}{1.505438in}}%
\pgfpathlineto{\pgfqpoint{1.384369in}{1.486669in}}%
\pgfpathclose%
\pgfusepath{fill}%
\end{pgfscope}%
\begin{pgfscope}%
\pgfpathrectangle{\pgfqpoint{0.150000in}{0.150000in}}{\pgfqpoint{2.700000in}{1.950000in}}%
\pgfusepath{clip}%
\pgfsetbuttcap%
\pgfsetroundjoin%
\definecolor{currentfill}{rgb}{0.524219,0.582812,0.664844}%
\pgfsetfillcolor{currentfill}%
\pgfsetlinewidth{0.000000pt}%
\definecolor{currentstroke}{rgb}{0.000000,0.000000,0.000000}%
\pgfsetstrokecolor{currentstroke}%
\pgfsetdash{}{0pt}%
\pgfpathmoveto{\pgfqpoint{1.345988in}{1.467831in}}%
\pgfpathlineto{\pgfqpoint{1.384369in}{1.486669in}}%
\pgfpathlineto{\pgfqpoint{1.346689in}{1.505438in}}%
\pgfpathlineto{\pgfqpoint{1.308310in}{1.486669in}}%
\pgfpathclose%
\pgfusepath{fill}%
\end{pgfscope}%
\begin{pgfscope}%
\pgfpathrectangle{\pgfqpoint{0.150000in}{0.150000in}}{\pgfqpoint{2.700000in}{1.950000in}}%
\pgfusepath{clip}%
\pgfsetbuttcap%
\pgfsetroundjoin%
\definecolor{currentfill}{rgb}{0.524219,0.582812,0.664844}%
\pgfsetfillcolor{currentfill}%
\pgfsetlinewidth{0.000000pt}%
\definecolor{currentstroke}{rgb}{0.000000,0.000000,0.000000}%
\pgfsetstrokecolor{currentstroke}%
\pgfsetdash{}{0pt}%
\pgfpathmoveto{\pgfqpoint{1.269789in}{1.467831in}}%
\pgfpathlineto{\pgfqpoint{1.308310in}{1.486669in}}%
\pgfpathlineto{\pgfqpoint{1.270770in}{1.505438in}}%
\pgfpathlineto{\pgfqpoint{1.232251in}{1.486669in}}%
\pgfpathclose%
\pgfusepath{fill}%
\end{pgfscope}%
\begin{pgfscope}%
\pgfpathrectangle{\pgfqpoint{0.150000in}{0.150000in}}{\pgfqpoint{2.700000in}{1.950000in}}%
\pgfusepath{clip}%
\pgfsetbuttcap%
\pgfsetroundjoin%
\definecolor{currentfill}{rgb}{0.524219,0.582812,0.664844}%
\pgfsetfillcolor{currentfill}%
\pgfsetlinewidth{0.000000pt}%
\definecolor{currentstroke}{rgb}{0.000000,0.000000,0.000000}%
\pgfsetstrokecolor{currentstroke}%
\pgfsetdash{}{0pt}%
\pgfpathmoveto{\pgfqpoint{1.193590in}{1.467831in}}%
\pgfpathlineto{\pgfqpoint{1.232251in}{1.486669in}}%
\pgfpathlineto{\pgfqpoint{1.194851in}{1.505438in}}%
\pgfpathlineto{\pgfqpoint{1.156192in}{1.486669in}}%
\pgfpathclose%
\pgfusepath{fill}%
\end{pgfscope}%
\begin{pgfscope}%
\pgfpathrectangle{\pgfqpoint{0.150000in}{0.150000in}}{\pgfqpoint{2.700000in}{1.950000in}}%
\pgfusepath{clip}%
\pgfsetbuttcap%
\pgfsetroundjoin%
\definecolor{currentfill}{rgb}{0.524219,0.582812,0.664844}%
\pgfsetfillcolor{currentfill}%
\pgfsetlinewidth{0.000000pt}%
\definecolor{currentstroke}{rgb}{0.000000,0.000000,0.000000}%
\pgfsetstrokecolor{currentstroke}%
\pgfsetdash{}{0pt}%
\pgfpathmoveto{\pgfqpoint{1.117391in}{1.467831in}}%
\pgfpathlineto{\pgfqpoint{1.156192in}{1.486669in}}%
\pgfpathlineto{\pgfqpoint{1.118932in}{1.505438in}}%
\pgfpathlineto{\pgfqpoint{1.080133in}{1.486669in}}%
\pgfpathclose%
\pgfusepath{fill}%
\end{pgfscope}%
\begin{pgfscope}%
\pgfpathrectangle{\pgfqpoint{0.150000in}{0.150000in}}{\pgfqpoint{2.700000in}{1.950000in}}%
\pgfusepath{clip}%
\pgfsetbuttcap%
\pgfsetroundjoin%
\definecolor{currentfill}{rgb}{0.524219,0.582812,0.664844}%
\pgfsetfillcolor{currentfill}%
\pgfsetlinewidth{0.000000pt}%
\definecolor{currentstroke}{rgb}{0.000000,0.000000,0.000000}%
\pgfsetstrokecolor{currentstroke}%
\pgfsetdash{}{0pt}%
\pgfpathmoveto{\pgfqpoint{1.041191in}{1.467831in}}%
\pgfpathlineto{\pgfqpoint{1.080133in}{1.486669in}}%
\pgfpathlineto{\pgfqpoint{1.043013in}{1.505438in}}%
\pgfpathlineto{\pgfqpoint{1.004075in}{1.486669in}}%
\pgfpathclose%
\pgfusepath{fill}%
\end{pgfscope}%
\begin{pgfscope}%
\pgfpathrectangle{\pgfqpoint{0.150000in}{0.150000in}}{\pgfqpoint{2.700000in}{1.950000in}}%
\pgfusepath{clip}%
\pgfsetbuttcap%
\pgfsetroundjoin%
\definecolor{currentfill}{rgb}{0.524219,0.582812,0.664844}%
\pgfsetfillcolor{currentfill}%
\pgfsetlinewidth{0.000000pt}%
\definecolor{currentstroke}{rgb}{0.000000,0.000000,0.000000}%
\pgfsetstrokecolor{currentstroke}%
\pgfsetdash{}{0pt}%
\pgfpathmoveto{\pgfqpoint{0.964992in}{1.467831in}}%
\pgfpathlineto{\pgfqpoint{1.004075in}{1.486669in}}%
\pgfpathlineto{\pgfqpoint{0.967094in}{1.505438in}}%
\pgfpathlineto{\pgfqpoint{0.928016in}{1.486669in}}%
\pgfpathclose%
\pgfusepath{fill}%
\end{pgfscope}%
\begin{pgfscope}%
\pgfpathrectangle{\pgfqpoint{0.150000in}{0.150000in}}{\pgfqpoint{2.700000in}{1.950000in}}%
\pgfusepath{clip}%
\pgfsetbuttcap%
\pgfsetroundjoin%
\definecolor{currentfill}{rgb}{0.524219,0.582812,0.664844}%
\pgfsetfillcolor{currentfill}%
\pgfsetlinewidth{0.000000pt}%
\definecolor{currentstroke}{rgb}{0.000000,0.000000,0.000000}%
\pgfsetstrokecolor{currentstroke}%
\pgfsetdash{}{0pt}%
\pgfpathmoveto{\pgfqpoint{0.888793in}{1.467831in}}%
\pgfpathlineto{\pgfqpoint{0.928016in}{1.486669in}}%
\pgfpathlineto{\pgfqpoint{0.891175in}{1.505438in}}%
\pgfpathlineto{\pgfqpoint{0.851957in}{1.486669in}}%
\pgfpathclose%
\pgfusepath{fill}%
\end{pgfscope}%
\begin{pgfscope}%
\pgfpathrectangle{\pgfqpoint{0.150000in}{0.150000in}}{\pgfqpoint{2.700000in}{1.950000in}}%
\pgfusepath{clip}%
\pgfsetbuttcap%
\pgfsetroundjoin%
\definecolor{currentfill}{rgb}{0.524219,0.582812,0.664844}%
\pgfsetfillcolor{currentfill}%
\pgfsetlinewidth{0.000000pt}%
\definecolor{currentstroke}{rgb}{0.000000,0.000000,0.000000}%
\pgfsetstrokecolor{currentstroke}%
\pgfsetdash{}{0pt}%
\pgfpathmoveto{\pgfqpoint{0.812594in}{1.467831in}}%
\pgfpathlineto{\pgfqpoint{0.851957in}{1.486669in}}%
\pgfpathlineto{\pgfqpoint{0.815256in}{1.505438in}}%
\pgfpathlineto{\pgfqpoint{0.775898in}{1.486669in}}%
\pgfpathclose%
\pgfusepath{fill}%
\end{pgfscope}%
\begin{pgfscope}%
\pgfpathrectangle{\pgfqpoint{0.150000in}{0.150000in}}{\pgfqpoint{2.700000in}{1.950000in}}%
\pgfusepath{clip}%
\pgfsetbuttcap%
\pgfsetroundjoin%
\definecolor{currentfill}{rgb}{0.959574,0.964553,0.971523}%
\pgfsetfillcolor{currentfill}%
\pgfsetlinewidth{0.000000pt}%
\definecolor{currentstroke}{rgb}{0.000000,0.000000,0.000000}%
\pgfsetstrokecolor{currentstroke}%
\pgfsetdash{}{0pt}%
\pgfpathmoveto{\pgfqpoint{1.650418in}{0.734775in}}%
\pgfpathlineto{\pgfqpoint{1.688117in}{0.754899in}}%
\pgfpathlineto{\pgfqpoint{1.650753in}{0.842285in}}%
\pgfpathlineto{\pgfqpoint{1.612805in}{0.822226in}}%
\pgfpathclose%
\pgfusepath{fill}%
\end{pgfscope}%
\begin{pgfscope}%
\pgfpathrectangle{\pgfqpoint{0.150000in}{0.150000in}}{\pgfqpoint{2.700000in}{1.950000in}}%
\pgfusepath{clip}%
\pgfsetbuttcap%
\pgfsetroundjoin%
\definecolor{currentfill}{rgb}{0.586412,0.637347,0.708655}%
\pgfsetfillcolor{currentfill}%
\pgfsetlinewidth{0.000000pt}%
\definecolor{currentstroke}{rgb}{0.000000,0.000000,0.000000}%
\pgfsetstrokecolor{currentstroke}%
\pgfsetdash{}{0pt}%
\pgfpathmoveto{\pgfqpoint{1.840616in}{1.327583in}}%
\pgfpathlineto{\pgfqpoint{1.878004in}{1.346637in}}%
\pgfpathlineto{\pgfqpoint{1.841173in}{1.424992in}}%
\pgfpathlineto{\pgfqpoint{1.803580in}{1.406012in}}%
\pgfpathclose%
\pgfusepath{fill}%
\end{pgfscope}%
\begin{pgfscope}%
\pgfpathrectangle{\pgfqpoint{0.150000in}{0.150000in}}{\pgfqpoint{2.700000in}{1.950000in}}%
\pgfusepath{clip}%
\pgfsetbuttcap%
\pgfsetroundjoin%
\definecolor{currentfill}{rgb}{0.661045,0.702788,0.761229}%
\pgfsetfillcolor{currentfill}%
\pgfsetlinewidth{0.000000pt}%
\definecolor{currentstroke}{rgb}{0.000000,0.000000,0.000000}%
\pgfsetstrokecolor{currentstroke}%
\pgfsetdash{}{0pt}%
\pgfpathmoveto{\pgfqpoint{1.802604in}{1.211260in}}%
\pgfpathlineto{\pgfqpoint{1.840062in}{1.230529in}}%
\pgfpathlineto{\pgfqpoint{1.803091in}{1.308458in}}%
\pgfpathlineto{\pgfqpoint{1.765427in}{1.289262in}}%
\pgfpathclose%
\pgfusepath{fill}%
\end{pgfscope}%
\begin{pgfscope}%
\pgfpathrectangle{\pgfqpoint{0.150000in}{0.150000in}}{\pgfqpoint{2.700000in}{1.950000in}}%
\pgfusepath{clip}%
\pgfsetbuttcap%
\pgfsetroundjoin%
\definecolor{currentfill}{rgb}{0.729458,0.762776,0.809421}%
\pgfsetfillcolor{currentfill}%
\pgfsetlinewidth{0.000000pt}%
\definecolor{currentstroke}{rgb}{0.000000,0.000000,0.000000}%
\pgfsetstrokecolor{currentstroke}%
\pgfsetdash{}{0pt}%
\pgfpathmoveto{\pgfqpoint{1.764591in}{1.094934in}}%
\pgfpathlineto{\pgfqpoint{1.802119in}{1.114417in}}%
\pgfpathlineto{\pgfqpoint{1.765008in}{1.191920in}}%
\pgfpathlineto{\pgfqpoint{1.727273in}{1.172508in}}%
\pgfpathclose%
\pgfusepath{fill}%
\end{pgfscope}%
\begin{pgfscope}%
\pgfpathrectangle{\pgfqpoint{0.150000in}{0.150000in}}{\pgfqpoint{2.700000in}{1.950000in}}%
\pgfusepath{clip}%
\pgfsetbuttcap%
\pgfsetroundjoin%
\definecolor{currentfill}{rgb}{0.804090,0.828217,0.861994}%
\pgfsetfillcolor{currentfill}%
\pgfsetlinewidth{0.000000pt}%
\definecolor{currentstroke}{rgb}{0.000000,0.000000,0.000000}%
\pgfsetstrokecolor{currentstroke}%
\pgfsetdash{}{0pt}%
\pgfpathmoveto{\pgfqpoint{1.726576in}{0.978604in}}%
\pgfpathlineto{\pgfqpoint{1.764175in}{0.998301in}}%
\pgfpathlineto{\pgfqpoint{1.726924in}{1.075379in}}%
\pgfpathlineto{\pgfqpoint{1.689118in}{1.055751in}}%
\pgfpathclose%
\pgfusepath{fill}%
\end{pgfscope}%
\begin{pgfscope}%
\pgfpathrectangle{\pgfqpoint{0.150000in}{0.150000in}}{\pgfqpoint{2.700000in}{1.950000in}}%
\pgfusepath{clip}%
\pgfsetbuttcap%
\pgfsetroundjoin%
\definecolor{currentfill}{rgb}{0.878722,0.893658,0.914568}%
\pgfsetfillcolor{currentfill}%
\pgfsetlinewidth{0.000000pt}%
\definecolor{currentstroke}{rgb}{0.000000,0.000000,0.000000}%
\pgfsetstrokecolor{currentstroke}%
\pgfsetdash{}{0pt}%
\pgfpathmoveto{\pgfqpoint{1.688561in}{0.862270in}}%
\pgfpathlineto{\pgfqpoint{1.726230in}{0.882182in}}%
\pgfpathlineto{\pgfqpoint{1.688839in}{0.958834in}}%
\pgfpathlineto{\pgfqpoint{1.650962in}{0.938991in}}%
\pgfpathclose%
\pgfusepath{fill}%
\end{pgfscope}%
\begin{pgfscope}%
\pgfpathrectangle{\pgfqpoint{0.150000in}{0.150000in}}{\pgfqpoint{2.700000in}{1.950000in}}%
\pgfusepath{clip}%
\pgfsetbuttcap%
\pgfsetroundjoin%
\definecolor{currentfill}{rgb}{0.536657,0.593719,0.673606}%
\pgfsetfillcolor{currentfill}%
\pgfsetlinewidth{0.000000pt}%
\definecolor{currentstroke}{rgb}{0.000000,0.000000,0.000000}%
\pgfsetstrokecolor{currentstroke}%
\pgfsetdash{}{0pt}%
\pgfpathmoveto{\pgfqpoint{1.841173in}{1.424992in}}%
\pgfpathlineto{\pgfqpoint{1.878627in}{1.443902in}}%
\pgfpathlineto{\pgfqpoint{1.840722in}{1.486669in}}%
\pgfpathlineto{\pgfqpoint{1.803184in}{1.467831in}}%
\pgfpathclose%
\pgfusepath{fill}%
\end{pgfscope}%
\begin{pgfscope}%
\pgfpathrectangle{\pgfqpoint{0.150000in}{0.150000in}}{\pgfqpoint{2.700000in}{1.950000in}}%
\pgfusepath{clip}%
\pgfsetbuttcap%
\pgfsetroundjoin%
\definecolor{currentfill}{rgb}{0.979105,0.962086,0.963434}%
\pgfsetfillcolor{currentfill}%
\pgfsetlinewidth{0.000000pt}%
\definecolor{currentstroke}{rgb}{0.000000,0.000000,0.000000}%
\pgfsetstrokecolor{currentstroke}%
\pgfsetdash{}{0pt}%
\pgfpathmoveto{\pgfqpoint{1.574534in}{0.598150in}}%
\pgfpathlineto{\pgfqpoint{1.612442in}{0.618562in}}%
\pgfpathlineto{\pgfqpoint{1.574604in}{0.694304in}}%
\pgfpathlineto{\pgfqpoint{1.536486in}{0.673956in}}%
\pgfpathclose%
\pgfusepath{fill}%
\end{pgfscope}%
\begin{pgfscope}%
\pgfpathrectangle{\pgfqpoint{0.150000in}{0.150000in}}{\pgfqpoint{2.700000in}{1.950000in}}%
\pgfusepath{clip}%
\pgfsetbuttcap%
\pgfsetroundjoin%
\definecolor{currentfill}{rgb}{0.933517,0.879366,0.883655}%
\pgfsetfillcolor{currentfill}%
\pgfsetlinewidth{0.000000pt}%
\definecolor{currentstroke}{rgb}{0.000000,0.000000,0.000000}%
\pgfsetstrokecolor{currentstroke}%
\pgfsetdash{}{0pt}%
\pgfpathmoveto{\pgfqpoint{1.536486in}{0.481719in}}%
\pgfpathlineto{\pgfqpoint{1.574465in}{0.502346in}}%
\pgfpathlineto{\pgfqpoint{1.536486in}{0.577661in}}%
\pgfpathlineto{\pgfqpoint{1.498298in}{0.557097in}}%
\pgfpathclose%
\pgfusepath{fill}%
\end{pgfscope}%
\begin{pgfscope}%
\pgfpathrectangle{\pgfqpoint{0.150000in}{0.150000in}}{\pgfqpoint{2.700000in}{1.950000in}}%
\pgfusepath{clip}%
\pgfsetbuttcap%
\pgfsetroundjoin%
\definecolor{currentfill}{rgb}{0.887929,0.796645,0.803876}%
\pgfsetfillcolor{currentfill}%
\pgfsetlinewidth{0.000000pt}%
\definecolor{currentstroke}{rgb}{0.000000,0.000000,0.000000}%
\pgfsetstrokecolor{currentstroke}%
\pgfsetdash{}{0pt}%
\pgfpathmoveto{\pgfqpoint{1.498437in}{0.365284in}}%
\pgfpathlineto{\pgfqpoint{1.536486in}{0.386126in}}%
\pgfpathlineto{\pgfqpoint{1.498368in}{0.461015in}}%
\pgfpathlineto{\pgfqpoint{1.460108in}{0.440235in}}%
\pgfpathclose%
\pgfusepath{fill}%
\end{pgfscope}%
\begin{pgfscope}%
\pgfpathrectangle{\pgfqpoint{0.150000in}{0.150000in}}{\pgfqpoint{2.700000in}{1.950000in}}%
\pgfusepath{clip}%
\pgfsetbuttcap%
\pgfsetroundjoin%
\definecolor{currentfill}{rgb}{0.524219,0.582812,0.664844}%
\pgfsetfillcolor{currentfill}%
\pgfsetlinewidth{0.000000pt}%
\definecolor{currentstroke}{rgb}{0.000000,0.000000,0.000000}%
\pgfsetstrokecolor{currentstroke}%
\pgfsetdash{}{0pt}%
\pgfpathmoveto{\pgfqpoint{1.765507in}{1.448923in}}%
\pgfpathlineto{\pgfqpoint{1.803184in}{1.467831in}}%
\pgfpathlineto{\pgfqpoint{1.764663in}{1.486669in}}%
\pgfpathlineto{\pgfqpoint{1.726985in}{1.467831in}}%
\pgfpathclose%
\pgfusepath{fill}%
\end{pgfscope}%
\begin{pgfscope}%
\pgfpathrectangle{\pgfqpoint{0.150000in}{0.150000in}}{\pgfqpoint{2.700000in}{1.950000in}}%
\pgfusepath{clip}%
\pgfsetbuttcap%
\pgfsetroundjoin%
\definecolor{currentfill}{rgb}{0.524219,0.582812,0.664844}%
\pgfsetfillcolor{currentfill}%
\pgfsetlinewidth{0.000000pt}%
\definecolor{currentstroke}{rgb}{0.000000,0.000000,0.000000}%
\pgfsetstrokecolor{currentstroke}%
\pgfsetdash{}{0pt}%
\pgfpathmoveto{\pgfqpoint{1.689167in}{1.448923in}}%
\pgfpathlineto{\pgfqpoint{1.726985in}{1.467831in}}%
\pgfpathlineto{\pgfqpoint{1.688604in}{1.486669in}}%
\pgfpathlineto{\pgfqpoint{1.650785in}{1.467831in}}%
\pgfpathclose%
\pgfusepath{fill}%
\end{pgfscope}%
\begin{pgfscope}%
\pgfpathrectangle{\pgfqpoint{0.150000in}{0.150000in}}{\pgfqpoint{2.700000in}{1.950000in}}%
\pgfusepath{clip}%
\pgfsetbuttcap%
\pgfsetroundjoin%
\definecolor{currentfill}{rgb}{0.524219,0.582812,0.664844}%
\pgfsetfillcolor{currentfill}%
\pgfsetlinewidth{0.000000pt}%
\definecolor{currentstroke}{rgb}{0.000000,0.000000,0.000000}%
\pgfsetstrokecolor{currentstroke}%
\pgfsetdash{}{0pt}%
\pgfpathmoveto{\pgfqpoint{1.612827in}{1.448923in}}%
\pgfpathlineto{\pgfqpoint{1.650785in}{1.467831in}}%
\pgfpathlineto{\pgfqpoint{1.612545in}{1.486669in}}%
\pgfpathlineto{\pgfqpoint{1.574586in}{1.467831in}}%
\pgfpathclose%
\pgfusepath{fill}%
\end{pgfscope}%
\begin{pgfscope}%
\pgfpathrectangle{\pgfqpoint{0.150000in}{0.150000in}}{\pgfqpoint{2.700000in}{1.950000in}}%
\pgfusepath{clip}%
\pgfsetbuttcap%
\pgfsetroundjoin%
\definecolor{currentfill}{rgb}{0.524219,0.582812,0.664844}%
\pgfsetfillcolor{currentfill}%
\pgfsetlinewidth{0.000000pt}%
\definecolor{currentstroke}{rgb}{0.000000,0.000000,0.000000}%
\pgfsetstrokecolor{currentstroke}%
\pgfsetdash{}{0pt}%
\pgfpathmoveto{\pgfqpoint{1.536486in}{1.448923in}}%
\pgfpathlineto{\pgfqpoint{1.574586in}{1.467831in}}%
\pgfpathlineto{\pgfqpoint{1.536486in}{1.486669in}}%
\pgfpathlineto{\pgfqpoint{1.498387in}{1.467831in}}%
\pgfpathclose%
\pgfusepath{fill}%
\end{pgfscope}%
\begin{pgfscope}%
\pgfpathrectangle{\pgfqpoint{0.150000in}{0.150000in}}{\pgfqpoint{2.700000in}{1.950000in}}%
\pgfusepath{clip}%
\pgfsetbuttcap%
\pgfsetroundjoin%
\definecolor{currentfill}{rgb}{0.524219,0.582812,0.664844}%
\pgfsetfillcolor{currentfill}%
\pgfsetlinewidth{0.000000pt}%
\definecolor{currentstroke}{rgb}{0.000000,0.000000,0.000000}%
\pgfsetstrokecolor{currentstroke}%
\pgfsetdash{}{0pt}%
\pgfpathmoveto{\pgfqpoint{1.460146in}{1.448923in}}%
\pgfpathlineto{\pgfqpoint{1.498387in}{1.467831in}}%
\pgfpathlineto{\pgfqpoint{1.460428in}{1.486669in}}%
\pgfpathlineto{\pgfqpoint{1.422188in}{1.467831in}}%
\pgfpathclose%
\pgfusepath{fill}%
\end{pgfscope}%
\begin{pgfscope}%
\pgfpathrectangle{\pgfqpoint{0.150000in}{0.150000in}}{\pgfqpoint{2.700000in}{1.950000in}}%
\pgfusepath{clip}%
\pgfsetbuttcap%
\pgfsetroundjoin%
\definecolor{currentfill}{rgb}{0.524219,0.582812,0.664844}%
\pgfsetfillcolor{currentfill}%
\pgfsetlinewidth{0.000000pt}%
\definecolor{currentstroke}{rgb}{0.000000,0.000000,0.000000}%
\pgfsetstrokecolor{currentstroke}%
\pgfsetdash{}{0pt}%
\pgfpathmoveto{\pgfqpoint{1.383806in}{1.448923in}}%
\pgfpathlineto{\pgfqpoint{1.422188in}{1.467831in}}%
\pgfpathlineto{\pgfqpoint{1.384369in}{1.486669in}}%
\pgfpathlineto{\pgfqpoint{1.345988in}{1.467831in}}%
\pgfpathclose%
\pgfusepath{fill}%
\end{pgfscope}%
\begin{pgfscope}%
\pgfpathrectangle{\pgfqpoint{0.150000in}{0.150000in}}{\pgfqpoint{2.700000in}{1.950000in}}%
\pgfusepath{clip}%
\pgfsetbuttcap%
\pgfsetroundjoin%
\definecolor{currentfill}{rgb}{0.524219,0.582812,0.664844}%
\pgfsetfillcolor{currentfill}%
\pgfsetlinewidth{0.000000pt}%
\definecolor{currentstroke}{rgb}{0.000000,0.000000,0.000000}%
\pgfsetstrokecolor{currentstroke}%
\pgfsetdash{}{0pt}%
\pgfpathmoveto{\pgfqpoint{1.307466in}{1.448923in}}%
\pgfpathlineto{\pgfqpoint{1.345988in}{1.467831in}}%
\pgfpathlineto{\pgfqpoint{1.308310in}{1.486669in}}%
\pgfpathlineto{\pgfqpoint{1.269789in}{1.467831in}}%
\pgfpathclose%
\pgfusepath{fill}%
\end{pgfscope}%
\begin{pgfscope}%
\pgfpathrectangle{\pgfqpoint{0.150000in}{0.150000in}}{\pgfqpoint{2.700000in}{1.950000in}}%
\pgfusepath{clip}%
\pgfsetbuttcap%
\pgfsetroundjoin%
\definecolor{currentfill}{rgb}{0.524219,0.582812,0.664844}%
\pgfsetfillcolor{currentfill}%
\pgfsetlinewidth{0.000000pt}%
\definecolor{currentstroke}{rgb}{0.000000,0.000000,0.000000}%
\pgfsetstrokecolor{currentstroke}%
\pgfsetdash{}{0pt}%
\pgfpathmoveto{\pgfqpoint{1.231126in}{1.448923in}}%
\pgfpathlineto{\pgfqpoint{1.269789in}{1.467831in}}%
\pgfpathlineto{\pgfqpoint{1.232251in}{1.486669in}}%
\pgfpathlineto{\pgfqpoint{1.193590in}{1.467831in}}%
\pgfpathclose%
\pgfusepath{fill}%
\end{pgfscope}%
\begin{pgfscope}%
\pgfpathrectangle{\pgfqpoint{0.150000in}{0.150000in}}{\pgfqpoint{2.700000in}{1.950000in}}%
\pgfusepath{clip}%
\pgfsetbuttcap%
\pgfsetroundjoin%
\definecolor{currentfill}{rgb}{0.524219,0.582812,0.664844}%
\pgfsetfillcolor{currentfill}%
\pgfsetlinewidth{0.000000pt}%
\definecolor{currentstroke}{rgb}{0.000000,0.000000,0.000000}%
\pgfsetstrokecolor{currentstroke}%
\pgfsetdash{}{0pt}%
\pgfpathmoveto{\pgfqpoint{1.154786in}{1.448923in}}%
\pgfpathlineto{\pgfqpoint{1.193590in}{1.467831in}}%
\pgfpathlineto{\pgfqpoint{1.156192in}{1.486669in}}%
\pgfpathlineto{\pgfqpoint{1.117391in}{1.467831in}}%
\pgfpathclose%
\pgfusepath{fill}%
\end{pgfscope}%
\begin{pgfscope}%
\pgfpathrectangle{\pgfqpoint{0.150000in}{0.150000in}}{\pgfqpoint{2.700000in}{1.950000in}}%
\pgfusepath{clip}%
\pgfsetbuttcap%
\pgfsetroundjoin%
\definecolor{currentfill}{rgb}{0.524219,0.582812,0.664844}%
\pgfsetfillcolor{currentfill}%
\pgfsetlinewidth{0.000000pt}%
\definecolor{currentstroke}{rgb}{0.000000,0.000000,0.000000}%
\pgfsetstrokecolor{currentstroke}%
\pgfsetdash{}{0pt}%
\pgfpathmoveto{\pgfqpoint{1.078446in}{1.448923in}}%
\pgfpathlineto{\pgfqpoint{1.117391in}{1.467831in}}%
\pgfpathlineto{\pgfqpoint{1.080133in}{1.486669in}}%
\pgfpathlineto{\pgfqpoint{1.041191in}{1.467831in}}%
\pgfpathclose%
\pgfusepath{fill}%
\end{pgfscope}%
\begin{pgfscope}%
\pgfpathrectangle{\pgfqpoint{0.150000in}{0.150000in}}{\pgfqpoint{2.700000in}{1.950000in}}%
\pgfusepath{clip}%
\pgfsetbuttcap%
\pgfsetroundjoin%
\definecolor{currentfill}{rgb}{0.524219,0.582812,0.664844}%
\pgfsetfillcolor{currentfill}%
\pgfsetlinewidth{0.000000pt}%
\definecolor{currentstroke}{rgb}{0.000000,0.000000,0.000000}%
\pgfsetstrokecolor{currentstroke}%
\pgfsetdash{}{0pt}%
\pgfpathmoveto{\pgfqpoint{1.002105in}{1.448923in}}%
\pgfpathlineto{\pgfqpoint{1.041191in}{1.467831in}}%
\pgfpathlineto{\pgfqpoint{1.004075in}{1.486669in}}%
\pgfpathlineto{\pgfqpoint{0.964992in}{1.467831in}}%
\pgfpathclose%
\pgfusepath{fill}%
\end{pgfscope}%
\begin{pgfscope}%
\pgfpathrectangle{\pgfqpoint{0.150000in}{0.150000in}}{\pgfqpoint{2.700000in}{1.950000in}}%
\pgfusepath{clip}%
\pgfsetbuttcap%
\pgfsetroundjoin%
\definecolor{currentfill}{rgb}{0.524219,0.582812,0.664844}%
\pgfsetfillcolor{currentfill}%
\pgfsetlinewidth{0.000000pt}%
\definecolor{currentstroke}{rgb}{0.000000,0.000000,0.000000}%
\pgfsetstrokecolor{currentstroke}%
\pgfsetdash{}{0pt}%
\pgfpathmoveto{\pgfqpoint{0.925765in}{1.448923in}}%
\pgfpathlineto{\pgfqpoint{0.964992in}{1.467831in}}%
\pgfpathlineto{\pgfqpoint{0.928016in}{1.486669in}}%
\pgfpathlineto{\pgfqpoint{0.888793in}{1.467831in}}%
\pgfpathclose%
\pgfusepath{fill}%
\end{pgfscope}%
\begin{pgfscope}%
\pgfpathrectangle{\pgfqpoint{0.150000in}{0.150000in}}{\pgfqpoint{2.700000in}{1.950000in}}%
\pgfusepath{clip}%
\pgfsetbuttcap%
\pgfsetroundjoin%
\definecolor{currentfill}{rgb}{0.524219,0.582812,0.664844}%
\pgfsetfillcolor{currentfill}%
\pgfsetlinewidth{0.000000pt}%
\definecolor{currentstroke}{rgb}{0.000000,0.000000,0.000000}%
\pgfsetstrokecolor{currentstroke}%
\pgfsetdash{}{0pt}%
\pgfpathmoveto{\pgfqpoint{0.849425in}{1.448923in}}%
\pgfpathlineto{\pgfqpoint{0.888793in}{1.467831in}}%
\pgfpathlineto{\pgfqpoint{0.851957in}{1.486669in}}%
\pgfpathlineto{\pgfqpoint{0.812594in}{1.467831in}}%
\pgfpathclose%
\pgfusepath{fill}%
\end{pgfscope}%
\begin{pgfscope}%
\pgfpathrectangle{\pgfqpoint{0.150000in}{0.150000in}}{\pgfqpoint{2.700000in}{1.950000in}}%
\pgfusepath{clip}%
\pgfsetbuttcap%
\pgfsetroundjoin%
\definecolor{currentfill}{rgb}{0.959574,0.964553,0.971523}%
\pgfsetfillcolor{currentfill}%
\pgfsetlinewidth{0.000000pt}%
\definecolor{currentstroke}{rgb}{0.000000,0.000000,0.000000}%
\pgfsetstrokecolor{currentstroke}%
\pgfsetdash{}{0pt}%
\pgfpathmoveto{\pgfqpoint{1.612581in}{0.714577in}}%
\pgfpathlineto{\pgfqpoint{1.650418in}{0.734775in}}%
\pgfpathlineto{\pgfqpoint{1.612805in}{0.822226in}}%
\pgfpathlineto{\pgfqpoint{1.574716in}{0.802093in}}%
\pgfpathclose%
\pgfusepath{fill}%
\end{pgfscope}%
\begin{pgfscope}%
\pgfpathrectangle{\pgfqpoint{0.150000in}{0.150000in}}{\pgfqpoint{2.700000in}{1.950000in}}%
\pgfusepath{clip}%
\pgfsetbuttcap%
\pgfsetroundjoin%
\definecolor{currentfill}{rgb}{0.586412,0.637347,0.708655}%
\pgfsetfillcolor{currentfill}%
\pgfsetlinewidth{0.000000pt}%
\definecolor{currentstroke}{rgb}{0.000000,0.000000,0.000000}%
\pgfsetstrokecolor{currentstroke}%
\pgfsetdash{}{0pt}%
\pgfpathmoveto{\pgfqpoint{1.803091in}{1.308458in}}%
\pgfpathlineto{\pgfqpoint{1.840616in}{1.327583in}}%
\pgfpathlineto{\pgfqpoint{1.803580in}{1.406012in}}%
\pgfpathlineto{\pgfqpoint{1.765848in}{1.386961in}}%
\pgfpathclose%
\pgfusepath{fill}%
\end{pgfscope}%
\begin{pgfscope}%
\pgfpathrectangle{\pgfqpoint{0.150000in}{0.150000in}}{\pgfqpoint{2.700000in}{1.950000in}}%
\pgfusepath{clip}%
\pgfsetbuttcap%
\pgfsetroundjoin%
\definecolor{currentfill}{rgb}{0.661045,0.702788,0.761229}%
\pgfsetfillcolor{currentfill}%
\pgfsetlinewidth{0.000000pt}%
\definecolor{currentstroke}{rgb}{0.000000,0.000000,0.000000}%
\pgfsetstrokecolor{currentstroke}%
\pgfsetdash{}{0pt}%
\pgfpathmoveto{\pgfqpoint{1.765008in}{1.191920in}}%
\pgfpathlineto{\pgfqpoint{1.802604in}{1.211260in}}%
\pgfpathlineto{\pgfqpoint{1.765427in}{1.289262in}}%
\pgfpathlineto{\pgfqpoint{1.727624in}{1.269995in}}%
\pgfpathclose%
\pgfusepath{fill}%
\end{pgfscope}%
\begin{pgfscope}%
\pgfpathrectangle{\pgfqpoint{0.150000in}{0.150000in}}{\pgfqpoint{2.700000in}{1.950000in}}%
\pgfusepath{clip}%
\pgfsetbuttcap%
\pgfsetroundjoin%
\definecolor{currentfill}{rgb}{0.729458,0.762776,0.809421}%
\pgfsetfillcolor{currentfill}%
\pgfsetlinewidth{0.000000pt}%
\definecolor{currentstroke}{rgb}{0.000000,0.000000,0.000000}%
\pgfsetstrokecolor{currentstroke}%
\pgfsetdash{}{0pt}%
\pgfpathmoveto{\pgfqpoint{1.726924in}{1.075379in}}%
\pgfpathlineto{\pgfqpoint{1.764591in}{1.094934in}}%
\pgfpathlineto{\pgfqpoint{1.727273in}{1.172508in}}%
\pgfpathlineto{\pgfqpoint{1.689399in}{1.153025in}}%
\pgfpathclose%
\pgfusepath{fill}%
\end{pgfscope}%
\begin{pgfscope}%
\pgfpathrectangle{\pgfqpoint{0.150000in}{0.150000in}}{\pgfqpoint{2.700000in}{1.950000in}}%
\pgfusepath{clip}%
\pgfsetbuttcap%
\pgfsetroundjoin%
\definecolor{currentfill}{rgb}{0.804090,0.828217,0.861994}%
\pgfsetfillcolor{currentfill}%
\pgfsetlinewidth{0.000000pt}%
\definecolor{currentstroke}{rgb}{0.000000,0.000000,0.000000}%
\pgfsetstrokecolor{currentstroke}%
\pgfsetdash{}{0pt}%
\pgfpathmoveto{\pgfqpoint{1.688839in}{0.958834in}}%
\pgfpathlineto{\pgfqpoint{1.726576in}{0.978604in}}%
\pgfpathlineto{\pgfqpoint{1.689118in}{1.055751in}}%
\pgfpathlineto{\pgfqpoint{1.651172in}{1.036051in}}%
\pgfpathclose%
\pgfusepath{fill}%
\end{pgfscope}%
\begin{pgfscope}%
\pgfpathrectangle{\pgfqpoint{0.150000in}{0.150000in}}{\pgfqpoint{2.700000in}{1.950000in}}%
\pgfusepath{clip}%
\pgfsetbuttcap%
\pgfsetroundjoin%
\definecolor{currentfill}{rgb}{0.878722,0.893658,0.914568}%
\pgfsetfillcolor{currentfill}%
\pgfsetlinewidth{0.000000pt}%
\definecolor{currentstroke}{rgb}{0.000000,0.000000,0.000000}%
\pgfsetstrokecolor{currentstroke}%
\pgfsetdash{}{0pt}%
\pgfpathmoveto{\pgfqpoint{1.650753in}{0.842285in}}%
\pgfpathlineto{\pgfqpoint{1.688561in}{0.862270in}}%
\pgfpathlineto{\pgfqpoint{1.650962in}{0.938991in}}%
\pgfpathlineto{\pgfqpoint{1.612945in}{0.919074in}}%
\pgfpathclose%
\pgfusepath{fill}%
\end{pgfscope}%
\begin{pgfscope}%
\pgfpathrectangle{\pgfqpoint{0.150000in}{0.150000in}}{\pgfqpoint{2.700000in}{1.950000in}}%
\pgfusepath{clip}%
\pgfsetbuttcap%
\pgfsetroundjoin%
\definecolor{currentfill}{rgb}{0.536657,0.593719,0.673606}%
\pgfsetfillcolor{currentfill}%
\pgfsetlinewidth{0.000000pt}%
\definecolor{currentstroke}{rgb}{0.000000,0.000000,0.000000}%
\pgfsetstrokecolor{currentstroke}%
\pgfsetdash{}{0pt}%
\pgfpathmoveto{\pgfqpoint{1.803580in}{1.406012in}}%
\pgfpathlineto{\pgfqpoint{1.841173in}{1.424992in}}%
\pgfpathlineto{\pgfqpoint{1.803184in}{1.467831in}}%
\pgfpathlineto{\pgfqpoint{1.765507in}{1.448923in}}%
\pgfpathclose%
\pgfusepath{fill}%
\end{pgfscope}%
\begin{pgfscope}%
\pgfpathrectangle{\pgfqpoint{0.150000in}{0.150000in}}{\pgfqpoint{2.700000in}{1.950000in}}%
\pgfusepath{clip}%
\pgfsetbuttcap%
\pgfsetroundjoin%
\definecolor{currentfill}{rgb}{0.505561,0.566452,0.651700}%
\pgfsetfillcolor{currentfill}%
\pgfsetlinewidth{0.000000pt}%
\definecolor{currentstroke}{rgb}{0.000000,0.000000,0.000000}%
\pgfsetstrokecolor{currentstroke}%
\pgfsetdash{}{0pt}%
\pgfpathmoveto{\pgfqpoint{0.771392in}{1.472961in}}%
\pgfpathlineto{\pgfqpoint{0.812594in}{1.467831in}}%
\pgfpathlineto{\pgfqpoint{0.775898in}{1.486669in}}%
\pgfpathlineto{\pgfqpoint{0.734623in}{1.491866in}}%
\pgfpathclose%
\pgfusepath{fill}%
\end{pgfscope}%
\begin{pgfscope}%
\pgfpathrectangle{\pgfqpoint{0.150000in}{0.150000in}}{\pgfqpoint{2.700000in}{1.950000in}}%
\pgfusepath{clip}%
\pgfsetbuttcap%
\pgfsetroundjoin%
\definecolor{currentfill}{rgb}{0.979105,0.962086,0.963434}%
\pgfsetfillcolor{currentfill}%
\pgfsetlinewidth{0.000000pt}%
\definecolor{currentstroke}{rgb}{0.000000,0.000000,0.000000}%
\pgfsetstrokecolor{currentstroke}%
\pgfsetdash{}{0pt}%
\pgfpathmoveto{\pgfqpoint{1.536486in}{0.577661in}}%
\pgfpathlineto{\pgfqpoint{1.574534in}{0.598150in}}%
\pgfpathlineto{\pgfqpoint{1.536486in}{0.673956in}}%
\pgfpathlineto{\pgfqpoint{1.498228in}{0.653533in}}%
\pgfpathclose%
\pgfusepath{fill}%
\end{pgfscope}%
\begin{pgfscope}%
\pgfpathrectangle{\pgfqpoint{0.150000in}{0.150000in}}{\pgfqpoint{2.700000in}{1.950000in}}%
\pgfusepath{clip}%
\pgfsetbuttcap%
\pgfsetroundjoin%
\definecolor{currentfill}{rgb}{0.933517,0.879366,0.883655}%
\pgfsetfillcolor{currentfill}%
\pgfsetlinewidth{0.000000pt}%
\definecolor{currentstroke}{rgb}{0.000000,0.000000,0.000000}%
\pgfsetstrokecolor{currentstroke}%
\pgfsetdash{}{0pt}%
\pgfpathmoveto{\pgfqpoint{1.498368in}{0.461015in}}%
\pgfpathlineto{\pgfqpoint{1.536486in}{0.481719in}}%
\pgfpathlineto{\pgfqpoint{1.498298in}{0.557097in}}%
\pgfpathlineto{\pgfqpoint{1.459967in}{0.536457in}}%
\pgfpathclose%
\pgfusepath{fill}%
\end{pgfscope}%
\begin{pgfscope}%
\pgfpathrectangle{\pgfqpoint{0.150000in}{0.150000in}}{\pgfqpoint{2.700000in}{1.950000in}}%
\pgfusepath{clip}%
\pgfsetbuttcap%
\pgfsetroundjoin%
\definecolor{currentfill}{rgb}{0.524219,0.582812,0.664844}%
\pgfsetfillcolor{currentfill}%
\pgfsetlinewidth{0.000000pt}%
\definecolor{currentstroke}{rgb}{0.000000,0.000000,0.000000}%
\pgfsetstrokecolor{currentstroke}%
\pgfsetdash{}{0pt}%
\pgfpathmoveto{\pgfqpoint{1.727690in}{1.429946in}}%
\pgfpathlineto{\pgfqpoint{1.765507in}{1.448923in}}%
\pgfpathlineto{\pgfqpoint{1.726985in}{1.467831in}}%
\pgfpathlineto{\pgfqpoint{1.689167in}{1.448923in}}%
\pgfpathclose%
\pgfusepath{fill}%
\end{pgfscope}%
\begin{pgfscope}%
\pgfpathrectangle{\pgfqpoint{0.150000in}{0.150000in}}{\pgfqpoint{2.700000in}{1.950000in}}%
\pgfusepath{clip}%
\pgfsetbuttcap%
\pgfsetroundjoin%
\definecolor{currentfill}{rgb}{0.524219,0.582812,0.664844}%
\pgfsetfillcolor{currentfill}%
\pgfsetlinewidth{0.000000pt}%
\definecolor{currentstroke}{rgb}{0.000000,0.000000,0.000000}%
\pgfsetstrokecolor{currentstroke}%
\pgfsetdash{}{0pt}%
\pgfpathmoveto{\pgfqpoint{1.651209in}{1.429946in}}%
\pgfpathlineto{\pgfqpoint{1.689167in}{1.448923in}}%
\pgfpathlineto{\pgfqpoint{1.650785in}{1.467831in}}%
\pgfpathlineto{\pgfqpoint{1.612827in}{1.448923in}}%
\pgfpathclose%
\pgfusepath{fill}%
\end{pgfscope}%
\begin{pgfscope}%
\pgfpathrectangle{\pgfqpoint{0.150000in}{0.150000in}}{\pgfqpoint{2.700000in}{1.950000in}}%
\pgfusepath{clip}%
\pgfsetbuttcap%
\pgfsetroundjoin%
\definecolor{currentfill}{rgb}{0.524219,0.582812,0.664844}%
\pgfsetfillcolor{currentfill}%
\pgfsetlinewidth{0.000000pt}%
\definecolor{currentstroke}{rgb}{0.000000,0.000000,0.000000}%
\pgfsetstrokecolor{currentstroke}%
\pgfsetdash{}{0pt}%
\pgfpathmoveto{\pgfqpoint{1.574727in}{1.429946in}}%
\pgfpathlineto{\pgfqpoint{1.612827in}{1.448923in}}%
\pgfpathlineto{\pgfqpoint{1.574586in}{1.467831in}}%
\pgfpathlineto{\pgfqpoint{1.536486in}{1.448923in}}%
\pgfpathclose%
\pgfusepath{fill}%
\end{pgfscope}%
\begin{pgfscope}%
\pgfpathrectangle{\pgfqpoint{0.150000in}{0.150000in}}{\pgfqpoint{2.700000in}{1.950000in}}%
\pgfusepath{clip}%
\pgfsetbuttcap%
\pgfsetroundjoin%
\definecolor{currentfill}{rgb}{0.524219,0.582812,0.664844}%
\pgfsetfillcolor{currentfill}%
\pgfsetlinewidth{0.000000pt}%
\definecolor{currentstroke}{rgb}{0.000000,0.000000,0.000000}%
\pgfsetstrokecolor{currentstroke}%
\pgfsetdash{}{0pt}%
\pgfpathmoveto{\pgfqpoint{1.498246in}{1.429946in}}%
\pgfpathlineto{\pgfqpoint{1.536486in}{1.448923in}}%
\pgfpathlineto{\pgfqpoint{1.498387in}{1.467831in}}%
\pgfpathlineto{\pgfqpoint{1.460146in}{1.448923in}}%
\pgfpathclose%
\pgfusepath{fill}%
\end{pgfscope}%
\begin{pgfscope}%
\pgfpathrectangle{\pgfqpoint{0.150000in}{0.150000in}}{\pgfqpoint{2.700000in}{1.950000in}}%
\pgfusepath{clip}%
\pgfsetbuttcap%
\pgfsetroundjoin%
\definecolor{currentfill}{rgb}{0.524219,0.582812,0.664844}%
\pgfsetfillcolor{currentfill}%
\pgfsetlinewidth{0.000000pt}%
\definecolor{currentstroke}{rgb}{0.000000,0.000000,0.000000}%
\pgfsetstrokecolor{currentstroke}%
\pgfsetdash{}{0pt}%
\pgfpathmoveto{\pgfqpoint{1.421764in}{1.429946in}}%
\pgfpathlineto{\pgfqpoint{1.460146in}{1.448923in}}%
\pgfpathlineto{\pgfqpoint{1.422188in}{1.467831in}}%
\pgfpathlineto{\pgfqpoint{1.383806in}{1.448923in}}%
\pgfpathclose%
\pgfusepath{fill}%
\end{pgfscope}%
\begin{pgfscope}%
\pgfpathrectangle{\pgfqpoint{0.150000in}{0.150000in}}{\pgfqpoint{2.700000in}{1.950000in}}%
\pgfusepath{clip}%
\pgfsetbuttcap%
\pgfsetroundjoin%
\definecolor{currentfill}{rgb}{0.524219,0.582812,0.664844}%
\pgfsetfillcolor{currentfill}%
\pgfsetlinewidth{0.000000pt}%
\definecolor{currentstroke}{rgb}{0.000000,0.000000,0.000000}%
\pgfsetstrokecolor{currentstroke}%
\pgfsetdash{}{0pt}%
\pgfpathmoveto{\pgfqpoint{1.345283in}{1.429946in}}%
\pgfpathlineto{\pgfqpoint{1.383806in}{1.448923in}}%
\pgfpathlineto{\pgfqpoint{1.345988in}{1.467831in}}%
\pgfpathlineto{\pgfqpoint{1.307466in}{1.448923in}}%
\pgfpathclose%
\pgfusepath{fill}%
\end{pgfscope}%
\begin{pgfscope}%
\pgfpathrectangle{\pgfqpoint{0.150000in}{0.150000in}}{\pgfqpoint{2.700000in}{1.950000in}}%
\pgfusepath{clip}%
\pgfsetbuttcap%
\pgfsetroundjoin%
\definecolor{currentfill}{rgb}{0.524219,0.582812,0.664844}%
\pgfsetfillcolor{currentfill}%
\pgfsetlinewidth{0.000000pt}%
\definecolor{currentstroke}{rgb}{0.000000,0.000000,0.000000}%
\pgfsetstrokecolor{currentstroke}%
\pgfsetdash{}{0pt}%
\pgfpathmoveto{\pgfqpoint{1.268801in}{1.429946in}}%
\pgfpathlineto{\pgfqpoint{1.307466in}{1.448923in}}%
\pgfpathlineto{\pgfqpoint{1.269789in}{1.467831in}}%
\pgfpathlineto{\pgfqpoint{1.231126in}{1.448923in}}%
\pgfpathclose%
\pgfusepath{fill}%
\end{pgfscope}%
\begin{pgfscope}%
\pgfpathrectangle{\pgfqpoint{0.150000in}{0.150000in}}{\pgfqpoint{2.700000in}{1.950000in}}%
\pgfusepath{clip}%
\pgfsetbuttcap%
\pgfsetroundjoin%
\definecolor{currentfill}{rgb}{0.524219,0.582812,0.664844}%
\pgfsetfillcolor{currentfill}%
\pgfsetlinewidth{0.000000pt}%
\definecolor{currentstroke}{rgb}{0.000000,0.000000,0.000000}%
\pgfsetstrokecolor{currentstroke}%
\pgfsetdash{}{0pt}%
\pgfpathmoveto{\pgfqpoint{1.192319in}{1.429946in}}%
\pgfpathlineto{\pgfqpoint{1.231126in}{1.448923in}}%
\pgfpathlineto{\pgfqpoint{1.193590in}{1.467831in}}%
\pgfpathlineto{\pgfqpoint{1.154786in}{1.448923in}}%
\pgfpathclose%
\pgfusepath{fill}%
\end{pgfscope}%
\begin{pgfscope}%
\pgfpathrectangle{\pgfqpoint{0.150000in}{0.150000in}}{\pgfqpoint{2.700000in}{1.950000in}}%
\pgfusepath{clip}%
\pgfsetbuttcap%
\pgfsetroundjoin%
\definecolor{currentfill}{rgb}{0.524219,0.582812,0.664844}%
\pgfsetfillcolor{currentfill}%
\pgfsetlinewidth{0.000000pt}%
\definecolor{currentstroke}{rgb}{0.000000,0.000000,0.000000}%
\pgfsetstrokecolor{currentstroke}%
\pgfsetdash{}{0pt}%
\pgfpathmoveto{\pgfqpoint{1.115838in}{1.429946in}}%
\pgfpathlineto{\pgfqpoint{1.154786in}{1.448923in}}%
\pgfpathlineto{\pgfqpoint{1.117391in}{1.467831in}}%
\pgfpathlineto{\pgfqpoint{1.078446in}{1.448923in}}%
\pgfpathclose%
\pgfusepath{fill}%
\end{pgfscope}%
\begin{pgfscope}%
\pgfpathrectangle{\pgfqpoint{0.150000in}{0.150000in}}{\pgfqpoint{2.700000in}{1.950000in}}%
\pgfusepath{clip}%
\pgfsetbuttcap%
\pgfsetroundjoin%
\definecolor{currentfill}{rgb}{0.524219,0.582812,0.664844}%
\pgfsetfillcolor{currentfill}%
\pgfsetlinewidth{0.000000pt}%
\definecolor{currentstroke}{rgb}{0.000000,0.000000,0.000000}%
\pgfsetstrokecolor{currentstroke}%
\pgfsetdash{}{0pt}%
\pgfpathmoveto{\pgfqpoint{1.039356in}{1.429946in}}%
\pgfpathlineto{\pgfqpoint{1.078446in}{1.448923in}}%
\pgfpathlineto{\pgfqpoint{1.041191in}{1.467831in}}%
\pgfpathlineto{\pgfqpoint{1.002105in}{1.448923in}}%
\pgfpathclose%
\pgfusepath{fill}%
\end{pgfscope}%
\begin{pgfscope}%
\pgfpathrectangle{\pgfqpoint{0.150000in}{0.150000in}}{\pgfqpoint{2.700000in}{1.950000in}}%
\pgfusepath{clip}%
\pgfsetbuttcap%
\pgfsetroundjoin%
\definecolor{currentfill}{rgb}{0.524219,0.582812,0.664844}%
\pgfsetfillcolor{currentfill}%
\pgfsetlinewidth{0.000000pt}%
\definecolor{currentstroke}{rgb}{0.000000,0.000000,0.000000}%
\pgfsetstrokecolor{currentstroke}%
\pgfsetdash{}{0pt}%
\pgfpathmoveto{\pgfqpoint{0.962875in}{1.429946in}}%
\pgfpathlineto{\pgfqpoint{1.002105in}{1.448923in}}%
\pgfpathlineto{\pgfqpoint{0.964992in}{1.467831in}}%
\pgfpathlineto{\pgfqpoint{0.925765in}{1.448923in}}%
\pgfpathclose%
\pgfusepath{fill}%
\end{pgfscope}%
\begin{pgfscope}%
\pgfpathrectangle{\pgfqpoint{0.150000in}{0.150000in}}{\pgfqpoint{2.700000in}{1.950000in}}%
\pgfusepath{clip}%
\pgfsetbuttcap%
\pgfsetroundjoin%
\definecolor{currentfill}{rgb}{0.524219,0.582812,0.664844}%
\pgfsetfillcolor{currentfill}%
\pgfsetlinewidth{0.000000pt}%
\definecolor{currentstroke}{rgb}{0.000000,0.000000,0.000000}%
\pgfsetstrokecolor{currentstroke}%
\pgfsetdash{}{0pt}%
\pgfpathmoveto{\pgfqpoint{0.886393in}{1.429946in}}%
\pgfpathlineto{\pgfqpoint{0.925765in}{1.448923in}}%
\pgfpathlineto{\pgfqpoint{0.888793in}{1.467831in}}%
\pgfpathlineto{\pgfqpoint{0.849425in}{1.448923in}}%
\pgfpathclose%
\pgfusepath{fill}%
\end{pgfscope}%
\begin{pgfscope}%
\pgfpathrectangle{\pgfqpoint{0.150000in}{0.150000in}}{\pgfqpoint{2.700000in}{1.950000in}}%
\pgfusepath{clip}%
\pgfsetbuttcap%
\pgfsetroundjoin%
\definecolor{currentfill}{rgb}{0.959574,0.964553,0.971523}%
\pgfsetfillcolor{currentfill}%
\pgfsetlinewidth{0.000000pt}%
\definecolor{currentstroke}{rgb}{0.000000,0.000000,0.000000}%
\pgfsetstrokecolor{currentstroke}%
\pgfsetdash{}{0pt}%
\pgfpathmoveto{\pgfqpoint{1.574604in}{0.694304in}}%
\pgfpathlineto{\pgfqpoint{1.612581in}{0.714577in}}%
\pgfpathlineto{\pgfqpoint{1.574716in}{0.802093in}}%
\pgfpathlineto{\pgfqpoint{1.536486in}{0.781885in}}%
\pgfpathclose%
\pgfusepath{fill}%
\end{pgfscope}%
\begin{pgfscope}%
\pgfpathrectangle{\pgfqpoint{0.150000in}{0.150000in}}{\pgfqpoint{2.700000in}{1.950000in}}%
\pgfusepath{clip}%
\pgfsetbuttcap%
\pgfsetroundjoin%
\definecolor{currentfill}{rgb}{0.586412,0.637347,0.708655}%
\pgfsetfillcolor{currentfill}%
\pgfsetlinewidth{0.000000pt}%
\definecolor{currentstroke}{rgb}{0.000000,0.000000,0.000000}%
\pgfsetstrokecolor{currentstroke}%
\pgfsetdash{}{0pt}%
\pgfpathmoveto{\pgfqpoint{1.765427in}{1.289262in}}%
\pgfpathlineto{\pgfqpoint{1.803091in}{1.308458in}}%
\pgfpathlineto{\pgfqpoint{1.765848in}{1.386961in}}%
\pgfpathlineto{\pgfqpoint{1.727975in}{1.367840in}}%
\pgfpathclose%
\pgfusepath{fill}%
\end{pgfscope}%
\begin{pgfscope}%
\pgfpathrectangle{\pgfqpoint{0.150000in}{0.150000in}}{\pgfqpoint{2.700000in}{1.950000in}}%
\pgfusepath{clip}%
\pgfsetbuttcap%
\pgfsetroundjoin%
\definecolor{currentfill}{rgb}{0.661045,0.702788,0.761229}%
\pgfsetfillcolor{currentfill}%
\pgfsetlinewidth{0.000000pt}%
\definecolor{currentstroke}{rgb}{0.000000,0.000000,0.000000}%
\pgfsetstrokecolor{currentstroke}%
\pgfsetdash{}{0pt}%
\pgfpathmoveto{\pgfqpoint{1.727273in}{1.172508in}}%
\pgfpathlineto{\pgfqpoint{1.765008in}{1.191920in}}%
\pgfpathlineto{\pgfqpoint{1.727624in}{1.269995in}}%
\pgfpathlineto{\pgfqpoint{1.689680in}{1.250657in}}%
\pgfpathclose%
\pgfusepath{fill}%
\end{pgfscope}%
\begin{pgfscope}%
\pgfpathrectangle{\pgfqpoint{0.150000in}{0.150000in}}{\pgfqpoint{2.700000in}{1.950000in}}%
\pgfusepath{clip}%
\pgfsetbuttcap%
\pgfsetroundjoin%
\definecolor{currentfill}{rgb}{0.729458,0.762776,0.809421}%
\pgfsetfillcolor{currentfill}%
\pgfsetlinewidth{0.000000pt}%
\definecolor{currentstroke}{rgb}{0.000000,0.000000,0.000000}%
\pgfsetstrokecolor{currentstroke}%
\pgfsetdash{}{0pt}%
\pgfpathmoveto{\pgfqpoint{1.689118in}{1.055751in}}%
\pgfpathlineto{\pgfqpoint{1.726924in}{1.075379in}}%
\pgfpathlineto{\pgfqpoint{1.689399in}{1.153025in}}%
\pgfpathlineto{\pgfqpoint{1.651383in}{1.133469in}}%
\pgfpathclose%
\pgfusepath{fill}%
\end{pgfscope}%
\begin{pgfscope}%
\pgfpathrectangle{\pgfqpoint{0.150000in}{0.150000in}}{\pgfqpoint{2.700000in}{1.950000in}}%
\pgfusepath{clip}%
\pgfsetbuttcap%
\pgfsetroundjoin%
\definecolor{currentfill}{rgb}{0.804090,0.828217,0.861994}%
\pgfsetfillcolor{currentfill}%
\pgfsetlinewidth{0.000000pt}%
\definecolor{currentstroke}{rgb}{0.000000,0.000000,0.000000}%
\pgfsetstrokecolor{currentstroke}%
\pgfsetdash{}{0pt}%
\pgfpathmoveto{\pgfqpoint{1.650962in}{0.938991in}}%
\pgfpathlineto{\pgfqpoint{1.688839in}{0.958834in}}%
\pgfpathlineto{\pgfqpoint{1.651172in}{1.036051in}}%
\pgfpathlineto{\pgfqpoint{1.613086in}{1.016278in}}%
\pgfpathclose%
\pgfusepath{fill}%
\end{pgfscope}%
\begin{pgfscope}%
\pgfpathrectangle{\pgfqpoint{0.150000in}{0.150000in}}{\pgfqpoint{2.700000in}{1.950000in}}%
\pgfusepath{clip}%
\pgfsetbuttcap%
\pgfsetroundjoin%
\definecolor{currentfill}{rgb}{0.878722,0.893658,0.914568}%
\pgfsetfillcolor{currentfill}%
\pgfsetlinewidth{0.000000pt}%
\definecolor{currentstroke}{rgb}{0.000000,0.000000,0.000000}%
\pgfsetstrokecolor{currentstroke}%
\pgfsetdash{}{0pt}%
\pgfpathmoveto{\pgfqpoint{1.612805in}{0.822226in}}%
\pgfpathlineto{\pgfqpoint{1.650753in}{0.842285in}}%
\pgfpathlineto{\pgfqpoint{1.612945in}{0.919074in}}%
\pgfpathlineto{\pgfqpoint{1.574787in}{0.899084in}}%
\pgfpathclose%
\pgfusepath{fill}%
\end{pgfscope}%
\begin{pgfscope}%
\pgfpathrectangle{\pgfqpoint{0.150000in}{0.150000in}}{\pgfqpoint{2.700000in}{1.950000in}}%
\pgfusepath{clip}%
\pgfsetbuttcap%
\pgfsetroundjoin%
\definecolor{currentfill}{rgb}{0.536657,0.593719,0.673606}%
\pgfsetfillcolor{currentfill}%
\pgfsetlinewidth{0.000000pt}%
\definecolor{currentstroke}{rgb}{0.000000,0.000000,0.000000}%
\pgfsetstrokecolor{currentstroke}%
\pgfsetdash{}{0pt}%
\pgfpathmoveto{\pgfqpoint{1.765848in}{1.386961in}}%
\pgfpathlineto{\pgfqpoint{1.803580in}{1.406012in}}%
\pgfpathlineto{\pgfqpoint{1.765507in}{1.448923in}}%
\pgfpathlineto{\pgfqpoint{1.727690in}{1.429946in}}%
\pgfpathclose%
\pgfusepath{fill}%
\end{pgfscope}%
\begin{pgfscope}%
\pgfpathrectangle{\pgfqpoint{0.150000in}{0.150000in}}{\pgfqpoint{2.700000in}{1.950000in}}%
\pgfusepath{clip}%
\pgfsetbuttcap%
\pgfsetroundjoin%
\definecolor{currentfill}{rgb}{0.505561,0.566452,0.651700}%
\pgfsetfillcolor{currentfill}%
\pgfsetlinewidth{0.000000pt}%
\definecolor{currentstroke}{rgb}{0.000000,0.000000,0.000000}%
\pgfsetstrokecolor{currentstroke}%
\pgfsetdash{}{0pt}%
\pgfpathmoveto{\pgfqpoint{0.808297in}{1.453986in}}%
\pgfpathlineto{\pgfqpoint{0.849425in}{1.448923in}}%
\pgfpathlineto{\pgfqpoint{0.812594in}{1.467831in}}%
\pgfpathlineto{\pgfqpoint{0.771392in}{1.472961in}}%
\pgfpathclose%
\pgfusepath{fill}%
\end{pgfscope}%
\begin{pgfscope}%
\pgfpathrectangle{\pgfqpoint{0.150000in}{0.150000in}}{\pgfqpoint{2.700000in}{1.950000in}}%
\pgfusepath{clip}%
\pgfsetbuttcap%
\pgfsetroundjoin%
\definecolor{currentfill}{rgb}{0.979105,0.962086,0.963434}%
\pgfsetfillcolor{currentfill}%
\pgfsetlinewidth{0.000000pt}%
\definecolor{currentstroke}{rgb}{0.000000,0.000000,0.000000}%
\pgfsetstrokecolor{currentstroke}%
\pgfsetdash{}{0pt}%
\pgfpathmoveto{\pgfqpoint{1.498298in}{0.557097in}}%
\pgfpathlineto{\pgfqpoint{1.536486in}{0.577661in}}%
\pgfpathlineto{\pgfqpoint{1.498228in}{0.653533in}}%
\pgfpathlineto{\pgfqpoint{1.459827in}{0.633034in}}%
\pgfpathclose%
\pgfusepath{fill}%
\end{pgfscope}%
\begin{pgfscope}%
\pgfpathrectangle{\pgfqpoint{0.150000in}{0.150000in}}{\pgfqpoint{2.700000in}{1.950000in}}%
\pgfusepath{clip}%
\pgfsetbuttcap%
\pgfsetroundjoin%
\definecolor{currentfill}{rgb}{0.933517,0.879366,0.883655}%
\pgfsetfillcolor{currentfill}%
\pgfsetlinewidth{0.000000pt}%
\definecolor{currentstroke}{rgb}{0.000000,0.000000,0.000000}%
\pgfsetstrokecolor{currentstroke}%
\pgfsetdash{}{0pt}%
\pgfpathmoveto{\pgfqpoint{1.460108in}{0.440235in}}%
\pgfpathlineto{\pgfqpoint{1.498368in}{0.461015in}}%
\pgfpathlineto{\pgfqpoint{1.459967in}{0.536457in}}%
\pgfpathlineto{\pgfqpoint{1.421495in}{0.515740in}}%
\pgfpathclose%
\pgfusepath{fill}%
\end{pgfscope}%
\begin{pgfscope}%
\pgfpathrectangle{\pgfqpoint{0.150000in}{0.150000in}}{\pgfqpoint{2.700000in}{1.950000in}}%
\pgfusepath{clip}%
\pgfsetbuttcap%
\pgfsetroundjoin%
\definecolor{currentfill}{rgb}{0.524219,0.582812,0.664844}%
\pgfsetfillcolor{currentfill}%
\pgfsetlinewidth{0.000000pt}%
\definecolor{currentstroke}{rgb}{0.000000,0.000000,0.000000}%
\pgfsetstrokecolor{currentstroke}%
\pgfsetdash{}{0pt}%
\pgfpathmoveto{\pgfqpoint{1.689734in}{1.410897in}}%
\pgfpathlineto{\pgfqpoint{1.727690in}{1.429946in}}%
\pgfpathlineto{\pgfqpoint{1.689167in}{1.448923in}}%
\pgfpathlineto{\pgfqpoint{1.651209in}{1.429946in}}%
\pgfpathclose%
\pgfusepath{fill}%
\end{pgfscope}%
\begin{pgfscope}%
\pgfpathrectangle{\pgfqpoint{0.150000in}{0.150000in}}{\pgfqpoint{2.700000in}{1.950000in}}%
\pgfusepath{clip}%
\pgfsetbuttcap%
\pgfsetroundjoin%
\definecolor{currentfill}{rgb}{0.524219,0.582812,0.664844}%
\pgfsetfillcolor{currentfill}%
\pgfsetlinewidth{0.000000pt}%
\definecolor{currentstroke}{rgb}{0.000000,0.000000,0.000000}%
\pgfsetstrokecolor{currentstroke}%
\pgfsetdash{}{0pt}%
\pgfpathmoveto{\pgfqpoint{1.613110in}{1.410897in}}%
\pgfpathlineto{\pgfqpoint{1.651209in}{1.429946in}}%
\pgfpathlineto{\pgfqpoint{1.612827in}{1.448923in}}%
\pgfpathlineto{\pgfqpoint{1.574727in}{1.429946in}}%
\pgfpathclose%
\pgfusepath{fill}%
\end{pgfscope}%
\begin{pgfscope}%
\pgfpathrectangle{\pgfqpoint{0.150000in}{0.150000in}}{\pgfqpoint{2.700000in}{1.950000in}}%
\pgfusepath{clip}%
\pgfsetbuttcap%
\pgfsetroundjoin%
\definecolor{currentfill}{rgb}{0.524219,0.582812,0.664844}%
\pgfsetfillcolor{currentfill}%
\pgfsetlinewidth{0.000000pt}%
\definecolor{currentstroke}{rgb}{0.000000,0.000000,0.000000}%
\pgfsetstrokecolor{currentstroke}%
\pgfsetdash{}{0pt}%
\pgfpathmoveto{\pgfqpoint{1.536486in}{1.410897in}}%
\pgfpathlineto{\pgfqpoint{1.574727in}{1.429946in}}%
\pgfpathlineto{\pgfqpoint{1.536486in}{1.448923in}}%
\pgfpathlineto{\pgfqpoint{1.498246in}{1.429946in}}%
\pgfpathclose%
\pgfusepath{fill}%
\end{pgfscope}%
\begin{pgfscope}%
\pgfpathrectangle{\pgfqpoint{0.150000in}{0.150000in}}{\pgfqpoint{2.700000in}{1.950000in}}%
\pgfusepath{clip}%
\pgfsetbuttcap%
\pgfsetroundjoin%
\definecolor{currentfill}{rgb}{0.524219,0.582812,0.664844}%
\pgfsetfillcolor{currentfill}%
\pgfsetlinewidth{0.000000pt}%
\definecolor{currentstroke}{rgb}{0.000000,0.000000,0.000000}%
\pgfsetstrokecolor{currentstroke}%
\pgfsetdash{}{0pt}%
\pgfpathmoveto{\pgfqpoint{1.459863in}{1.410897in}}%
\pgfpathlineto{\pgfqpoint{1.498246in}{1.429946in}}%
\pgfpathlineto{\pgfqpoint{1.460146in}{1.448923in}}%
\pgfpathlineto{\pgfqpoint{1.421764in}{1.429946in}}%
\pgfpathclose%
\pgfusepath{fill}%
\end{pgfscope}%
\begin{pgfscope}%
\pgfpathrectangle{\pgfqpoint{0.150000in}{0.150000in}}{\pgfqpoint{2.700000in}{1.950000in}}%
\pgfusepath{clip}%
\pgfsetbuttcap%
\pgfsetroundjoin%
\definecolor{currentfill}{rgb}{0.524219,0.582812,0.664844}%
\pgfsetfillcolor{currentfill}%
\pgfsetlinewidth{0.000000pt}%
\definecolor{currentstroke}{rgb}{0.000000,0.000000,0.000000}%
\pgfsetstrokecolor{currentstroke}%
\pgfsetdash{}{0pt}%
\pgfpathmoveto{\pgfqpoint{1.383239in}{1.410897in}}%
\pgfpathlineto{\pgfqpoint{1.421764in}{1.429946in}}%
\pgfpathlineto{\pgfqpoint{1.383806in}{1.448923in}}%
\pgfpathlineto{\pgfqpoint{1.345283in}{1.429946in}}%
\pgfpathclose%
\pgfusepath{fill}%
\end{pgfscope}%
\begin{pgfscope}%
\pgfpathrectangle{\pgfqpoint{0.150000in}{0.150000in}}{\pgfqpoint{2.700000in}{1.950000in}}%
\pgfusepath{clip}%
\pgfsetbuttcap%
\pgfsetroundjoin%
\definecolor{currentfill}{rgb}{0.524219,0.582812,0.664844}%
\pgfsetfillcolor{currentfill}%
\pgfsetlinewidth{0.000000pt}%
\definecolor{currentstroke}{rgb}{0.000000,0.000000,0.000000}%
\pgfsetstrokecolor{currentstroke}%
\pgfsetdash{}{0pt}%
\pgfpathmoveto{\pgfqpoint{1.306616in}{1.410897in}}%
\pgfpathlineto{\pgfqpoint{1.345283in}{1.429946in}}%
\pgfpathlineto{\pgfqpoint{1.307466in}{1.448923in}}%
\pgfpathlineto{\pgfqpoint{1.268801in}{1.429946in}}%
\pgfpathclose%
\pgfusepath{fill}%
\end{pgfscope}%
\begin{pgfscope}%
\pgfpathrectangle{\pgfqpoint{0.150000in}{0.150000in}}{\pgfqpoint{2.700000in}{1.950000in}}%
\pgfusepath{clip}%
\pgfsetbuttcap%
\pgfsetroundjoin%
\definecolor{currentfill}{rgb}{0.524219,0.582812,0.664844}%
\pgfsetfillcolor{currentfill}%
\pgfsetlinewidth{0.000000pt}%
\definecolor{currentstroke}{rgb}{0.000000,0.000000,0.000000}%
\pgfsetstrokecolor{currentstroke}%
\pgfsetdash{}{0pt}%
\pgfpathmoveto{\pgfqpoint{1.229992in}{1.410897in}}%
\pgfpathlineto{\pgfqpoint{1.268801in}{1.429946in}}%
\pgfpathlineto{\pgfqpoint{1.231126in}{1.448923in}}%
\pgfpathlineto{\pgfqpoint{1.192319in}{1.429946in}}%
\pgfpathclose%
\pgfusepath{fill}%
\end{pgfscope}%
\begin{pgfscope}%
\pgfpathrectangle{\pgfqpoint{0.150000in}{0.150000in}}{\pgfqpoint{2.700000in}{1.950000in}}%
\pgfusepath{clip}%
\pgfsetbuttcap%
\pgfsetroundjoin%
\definecolor{currentfill}{rgb}{0.524219,0.582812,0.664844}%
\pgfsetfillcolor{currentfill}%
\pgfsetlinewidth{0.000000pt}%
\definecolor{currentstroke}{rgb}{0.000000,0.000000,0.000000}%
\pgfsetstrokecolor{currentstroke}%
\pgfsetdash{}{0pt}%
\pgfpathmoveto{\pgfqpoint{1.153369in}{1.410897in}}%
\pgfpathlineto{\pgfqpoint{1.192319in}{1.429946in}}%
\pgfpathlineto{\pgfqpoint{1.154786in}{1.448923in}}%
\pgfpathlineto{\pgfqpoint{1.115838in}{1.429946in}}%
\pgfpathclose%
\pgfusepath{fill}%
\end{pgfscope}%
\begin{pgfscope}%
\pgfpathrectangle{\pgfqpoint{0.150000in}{0.150000in}}{\pgfqpoint{2.700000in}{1.950000in}}%
\pgfusepath{clip}%
\pgfsetbuttcap%
\pgfsetroundjoin%
\definecolor{currentfill}{rgb}{0.524219,0.582812,0.664844}%
\pgfsetfillcolor{currentfill}%
\pgfsetlinewidth{0.000000pt}%
\definecolor{currentstroke}{rgb}{0.000000,0.000000,0.000000}%
\pgfsetstrokecolor{currentstroke}%
\pgfsetdash{}{0pt}%
\pgfpathmoveto{\pgfqpoint{1.076745in}{1.410897in}}%
\pgfpathlineto{\pgfqpoint{1.115838in}{1.429946in}}%
\pgfpathlineto{\pgfqpoint{1.078446in}{1.448923in}}%
\pgfpathlineto{\pgfqpoint{1.039356in}{1.429946in}}%
\pgfpathclose%
\pgfusepath{fill}%
\end{pgfscope}%
\begin{pgfscope}%
\pgfpathrectangle{\pgfqpoint{0.150000in}{0.150000in}}{\pgfqpoint{2.700000in}{1.950000in}}%
\pgfusepath{clip}%
\pgfsetbuttcap%
\pgfsetroundjoin%
\definecolor{currentfill}{rgb}{0.524219,0.582812,0.664844}%
\pgfsetfillcolor{currentfill}%
\pgfsetlinewidth{0.000000pt}%
\definecolor{currentstroke}{rgb}{0.000000,0.000000,0.000000}%
\pgfsetstrokecolor{currentstroke}%
\pgfsetdash{}{0pt}%
\pgfpathmoveto{\pgfqpoint{1.000122in}{1.410897in}}%
\pgfpathlineto{\pgfqpoint{1.039356in}{1.429946in}}%
\pgfpathlineto{\pgfqpoint{1.002105in}{1.448923in}}%
\pgfpathlineto{\pgfqpoint{0.962875in}{1.429946in}}%
\pgfpathclose%
\pgfusepath{fill}%
\end{pgfscope}%
\begin{pgfscope}%
\pgfpathrectangle{\pgfqpoint{0.150000in}{0.150000in}}{\pgfqpoint{2.700000in}{1.950000in}}%
\pgfusepath{clip}%
\pgfsetbuttcap%
\pgfsetroundjoin%
\definecolor{currentfill}{rgb}{0.524219,0.582812,0.664844}%
\pgfsetfillcolor{currentfill}%
\pgfsetlinewidth{0.000000pt}%
\definecolor{currentstroke}{rgb}{0.000000,0.000000,0.000000}%
\pgfsetstrokecolor{currentstroke}%
\pgfsetdash{}{0pt}%
\pgfpathmoveto{\pgfqpoint{0.923498in}{1.410897in}}%
\pgfpathlineto{\pgfqpoint{0.962875in}{1.429946in}}%
\pgfpathlineto{\pgfqpoint{0.925765in}{1.448923in}}%
\pgfpathlineto{\pgfqpoint{0.886393in}{1.429946in}}%
\pgfpathclose%
\pgfusepath{fill}%
\end{pgfscope}%
\begin{pgfscope}%
\pgfpathrectangle{\pgfqpoint{0.150000in}{0.150000in}}{\pgfqpoint{2.700000in}{1.950000in}}%
\pgfusepath{clip}%
\pgfsetbuttcap%
\pgfsetroundjoin%
\definecolor{currentfill}{rgb}{0.959574,0.964553,0.971523}%
\pgfsetfillcolor{currentfill}%
\pgfsetlinewidth{0.000000pt}%
\definecolor{currentstroke}{rgb}{0.000000,0.000000,0.000000}%
\pgfsetstrokecolor{currentstroke}%
\pgfsetdash{}{0pt}%
\pgfpathmoveto{\pgfqpoint{1.536486in}{0.673956in}}%
\pgfpathlineto{\pgfqpoint{1.574604in}{0.694304in}}%
\pgfpathlineto{\pgfqpoint{1.536486in}{0.781885in}}%
\pgfpathlineto{\pgfqpoint{1.498114in}{0.761602in}}%
\pgfpathclose%
\pgfusepath{fill}%
\end{pgfscope}%
\begin{pgfscope}%
\pgfpathrectangle{\pgfqpoint{0.150000in}{0.150000in}}{\pgfqpoint{2.700000in}{1.950000in}}%
\pgfusepath{clip}%
\pgfsetbuttcap%
\pgfsetroundjoin%
\definecolor{currentfill}{rgb}{0.586412,0.637347,0.708655}%
\pgfsetfillcolor{currentfill}%
\pgfsetlinewidth{0.000000pt}%
\definecolor{currentstroke}{rgb}{0.000000,0.000000,0.000000}%
\pgfsetstrokecolor{currentstroke}%
\pgfsetdash{}{0pt}%
\pgfpathmoveto{\pgfqpoint{1.727624in}{1.269995in}}%
\pgfpathlineto{\pgfqpoint{1.765427in}{1.289262in}}%
\pgfpathlineto{\pgfqpoint{1.727975in}{1.367840in}}%
\pgfpathlineto{\pgfqpoint{1.689962in}{1.348648in}}%
\pgfpathclose%
\pgfusepath{fill}%
\end{pgfscope}%
\begin{pgfscope}%
\pgfpathrectangle{\pgfqpoint{0.150000in}{0.150000in}}{\pgfqpoint{2.700000in}{1.950000in}}%
\pgfusepath{clip}%
\pgfsetbuttcap%
\pgfsetroundjoin%
\definecolor{currentfill}{rgb}{0.661045,0.702788,0.761229}%
\pgfsetfillcolor{currentfill}%
\pgfsetlinewidth{0.000000pt}%
\definecolor{currentstroke}{rgb}{0.000000,0.000000,0.000000}%
\pgfsetstrokecolor{currentstroke}%
\pgfsetdash{}{0pt}%
\pgfpathmoveto{\pgfqpoint{1.689399in}{1.153025in}}%
\pgfpathlineto{\pgfqpoint{1.727273in}{1.172508in}}%
\pgfpathlineto{\pgfqpoint{1.689680in}{1.250657in}}%
\pgfpathlineto{\pgfqpoint{1.651595in}{1.231246in}}%
\pgfpathclose%
\pgfusepath{fill}%
\end{pgfscope}%
\begin{pgfscope}%
\pgfpathrectangle{\pgfqpoint{0.150000in}{0.150000in}}{\pgfqpoint{2.700000in}{1.950000in}}%
\pgfusepath{clip}%
\pgfsetbuttcap%
\pgfsetroundjoin%
\definecolor{currentfill}{rgb}{0.729458,0.762776,0.809421}%
\pgfsetfillcolor{currentfill}%
\pgfsetlinewidth{0.000000pt}%
\definecolor{currentstroke}{rgb}{0.000000,0.000000,0.000000}%
\pgfsetstrokecolor{currentstroke}%
\pgfsetdash{}{0pt}%
\pgfpathmoveto{\pgfqpoint{1.651172in}{1.036051in}}%
\pgfpathlineto{\pgfqpoint{1.689118in}{1.055751in}}%
\pgfpathlineto{\pgfqpoint{1.651383in}{1.133469in}}%
\pgfpathlineto{\pgfqpoint{1.613227in}{1.113841in}}%
\pgfpathclose%
\pgfusepath{fill}%
\end{pgfscope}%
\begin{pgfscope}%
\pgfpathrectangle{\pgfqpoint{0.150000in}{0.150000in}}{\pgfqpoint{2.700000in}{1.950000in}}%
\pgfusepath{clip}%
\pgfsetbuttcap%
\pgfsetroundjoin%
\definecolor{currentfill}{rgb}{0.804090,0.828217,0.861994}%
\pgfsetfillcolor{currentfill}%
\pgfsetlinewidth{0.000000pt}%
\definecolor{currentstroke}{rgb}{0.000000,0.000000,0.000000}%
\pgfsetstrokecolor{currentstroke}%
\pgfsetdash{}{0pt}%
\pgfpathmoveto{\pgfqpoint{1.612945in}{0.919074in}}%
\pgfpathlineto{\pgfqpoint{1.650962in}{0.938991in}}%
\pgfpathlineto{\pgfqpoint{1.613086in}{1.016278in}}%
\pgfpathlineto{\pgfqpoint{1.574857in}{0.996432in}}%
\pgfpathclose%
\pgfusepath{fill}%
\end{pgfscope}%
\begin{pgfscope}%
\pgfpathrectangle{\pgfqpoint{0.150000in}{0.150000in}}{\pgfqpoint{2.700000in}{1.950000in}}%
\pgfusepath{clip}%
\pgfsetbuttcap%
\pgfsetroundjoin%
\definecolor{currentfill}{rgb}{0.878722,0.893658,0.914568}%
\pgfsetfillcolor{currentfill}%
\pgfsetlinewidth{0.000000pt}%
\definecolor{currentstroke}{rgb}{0.000000,0.000000,0.000000}%
\pgfsetstrokecolor{currentstroke}%
\pgfsetdash{}{0pt}%
\pgfpathmoveto{\pgfqpoint{1.574716in}{0.802093in}}%
\pgfpathlineto{\pgfqpoint{1.612805in}{0.822226in}}%
\pgfpathlineto{\pgfqpoint{1.574787in}{0.899084in}}%
\pgfpathlineto{\pgfqpoint{1.536486in}{0.879019in}}%
\pgfpathclose%
\pgfusepath{fill}%
\end{pgfscope}%
\begin{pgfscope}%
\pgfpathrectangle{\pgfqpoint{0.150000in}{0.150000in}}{\pgfqpoint{2.700000in}{1.950000in}}%
\pgfusepath{clip}%
\pgfsetbuttcap%
\pgfsetroundjoin%
\definecolor{currentfill}{rgb}{0.536657,0.593719,0.673606}%
\pgfsetfillcolor{currentfill}%
\pgfsetlinewidth{0.000000pt}%
\definecolor{currentstroke}{rgb}{0.000000,0.000000,0.000000}%
\pgfsetstrokecolor{currentstroke}%
\pgfsetdash{}{0pt}%
\pgfpathmoveto{\pgfqpoint{1.727975in}{1.367840in}}%
\pgfpathlineto{\pgfqpoint{1.765848in}{1.386961in}}%
\pgfpathlineto{\pgfqpoint{1.727690in}{1.429946in}}%
\pgfpathlineto{\pgfqpoint{1.689734in}{1.410897in}}%
\pgfpathclose%
\pgfusepath{fill}%
\end{pgfscope}%
\begin{pgfscope}%
\pgfpathrectangle{\pgfqpoint{0.150000in}{0.150000in}}{\pgfqpoint{2.700000in}{1.950000in}}%
\pgfusepath{clip}%
\pgfsetbuttcap%
\pgfsetroundjoin%
\definecolor{currentfill}{rgb}{0.505561,0.566452,0.651700}%
\pgfsetfillcolor{currentfill}%
\pgfsetlinewidth{0.000000pt}%
\definecolor{currentstroke}{rgb}{0.000000,0.000000,0.000000}%
\pgfsetstrokecolor{currentstroke}%
\pgfsetdash{}{0pt}%
\pgfpathmoveto{\pgfqpoint{0.845339in}{1.434940in}}%
\pgfpathlineto{\pgfqpoint{0.886393in}{1.429946in}}%
\pgfpathlineto{\pgfqpoint{0.849425in}{1.448923in}}%
\pgfpathlineto{\pgfqpoint{0.808297in}{1.453986in}}%
\pgfpathclose%
\pgfusepath{fill}%
\end{pgfscope}%
\begin{pgfscope}%
\pgfpathrectangle{\pgfqpoint{0.150000in}{0.150000in}}{\pgfqpoint{2.700000in}{1.950000in}}%
\pgfusepath{clip}%
\pgfsetbuttcap%
\pgfsetroundjoin%
\definecolor{currentfill}{rgb}{0.979105,0.962086,0.963434}%
\pgfsetfillcolor{currentfill}%
\pgfsetlinewidth{0.000000pt}%
\definecolor{currentstroke}{rgb}{0.000000,0.000000,0.000000}%
\pgfsetstrokecolor{currentstroke}%
\pgfsetdash{}{0pt}%
\pgfpathmoveto{\pgfqpoint{1.459967in}{0.536457in}}%
\pgfpathlineto{\pgfqpoint{1.498298in}{0.557097in}}%
\pgfpathlineto{\pgfqpoint{1.459827in}{0.633034in}}%
\pgfpathlineto{\pgfqpoint{1.421283in}{0.612458in}}%
\pgfpathclose%
\pgfusepath{fill}%
\end{pgfscope}%
\begin{pgfscope}%
\pgfpathrectangle{\pgfqpoint{0.150000in}{0.150000in}}{\pgfqpoint{2.700000in}{1.950000in}}%
\pgfusepath{clip}%
\pgfsetbuttcap%
\pgfsetroundjoin%
\definecolor{currentfill}{rgb}{0.480683,0.544638,0.634176}%
\pgfsetfillcolor{currentfill}%
\pgfsetlinewidth{0.000000pt}%
\definecolor{currentstroke}{rgb}{0.000000,0.000000,0.000000}%
\pgfsetstrokecolor{currentstroke}%
\pgfsetdash{}{0pt}%
\pgfpathmoveto{\pgfqpoint{0.729853in}{1.478133in}}%
\pgfpathlineto{\pgfqpoint{0.771392in}{1.472961in}}%
\pgfpathlineto{\pgfqpoint{0.734623in}{1.491866in}}%
\pgfpathlineto{\pgfqpoint{0.693012in}{1.497106in}}%
\pgfpathclose%
\pgfusepath{fill}%
\end{pgfscope}%
\begin{pgfscope}%
\pgfpathrectangle{\pgfqpoint{0.150000in}{0.150000in}}{\pgfqpoint{2.700000in}{1.950000in}}%
\pgfusepath{clip}%
\pgfsetbuttcap%
\pgfsetroundjoin%
\definecolor{currentfill}{rgb}{0.524219,0.582812,0.664844}%
\pgfsetfillcolor{currentfill}%
\pgfsetlinewidth{0.000000pt}%
\definecolor{currentstroke}{rgb}{0.000000,0.000000,0.000000}%
\pgfsetstrokecolor{currentstroke}%
\pgfsetdash{}{0pt}%
\pgfpathmoveto{\pgfqpoint{1.651636in}{1.391778in}}%
\pgfpathlineto{\pgfqpoint{1.689734in}{1.410897in}}%
\pgfpathlineto{\pgfqpoint{1.651209in}{1.429946in}}%
\pgfpathlineto{\pgfqpoint{1.613110in}{1.410897in}}%
\pgfpathclose%
\pgfusepath{fill}%
\end{pgfscope}%
\begin{pgfscope}%
\pgfpathrectangle{\pgfqpoint{0.150000in}{0.150000in}}{\pgfqpoint{2.700000in}{1.950000in}}%
\pgfusepath{clip}%
\pgfsetbuttcap%
\pgfsetroundjoin%
\definecolor{currentfill}{rgb}{0.524219,0.582812,0.664844}%
\pgfsetfillcolor{currentfill}%
\pgfsetlinewidth{0.000000pt}%
\definecolor{currentstroke}{rgb}{0.000000,0.000000,0.000000}%
\pgfsetstrokecolor{currentstroke}%
\pgfsetdash{}{0pt}%
\pgfpathmoveto{\pgfqpoint{1.574870in}{1.391778in}}%
\pgfpathlineto{\pgfqpoint{1.613110in}{1.410897in}}%
\pgfpathlineto{\pgfqpoint{1.574727in}{1.429946in}}%
\pgfpathlineto{\pgfqpoint{1.536486in}{1.410897in}}%
\pgfpathclose%
\pgfusepath{fill}%
\end{pgfscope}%
\begin{pgfscope}%
\pgfpathrectangle{\pgfqpoint{0.150000in}{0.150000in}}{\pgfqpoint{2.700000in}{1.950000in}}%
\pgfusepath{clip}%
\pgfsetbuttcap%
\pgfsetroundjoin%
\definecolor{currentfill}{rgb}{0.524219,0.582812,0.664844}%
\pgfsetfillcolor{currentfill}%
\pgfsetlinewidth{0.000000pt}%
\definecolor{currentstroke}{rgb}{0.000000,0.000000,0.000000}%
\pgfsetstrokecolor{currentstroke}%
\pgfsetdash{}{0pt}%
\pgfpathmoveto{\pgfqpoint{1.498103in}{1.391778in}}%
\pgfpathlineto{\pgfqpoint{1.536486in}{1.410897in}}%
\pgfpathlineto{\pgfqpoint{1.498246in}{1.429946in}}%
\pgfpathlineto{\pgfqpoint{1.459863in}{1.410897in}}%
\pgfpathclose%
\pgfusepath{fill}%
\end{pgfscope}%
\begin{pgfscope}%
\pgfpathrectangle{\pgfqpoint{0.150000in}{0.150000in}}{\pgfqpoint{2.700000in}{1.950000in}}%
\pgfusepath{clip}%
\pgfsetbuttcap%
\pgfsetroundjoin%
\definecolor{currentfill}{rgb}{0.524219,0.582812,0.664844}%
\pgfsetfillcolor{currentfill}%
\pgfsetlinewidth{0.000000pt}%
\definecolor{currentstroke}{rgb}{0.000000,0.000000,0.000000}%
\pgfsetstrokecolor{currentstroke}%
\pgfsetdash{}{0pt}%
\pgfpathmoveto{\pgfqpoint{1.421337in}{1.391778in}}%
\pgfpathlineto{\pgfqpoint{1.459863in}{1.410897in}}%
\pgfpathlineto{\pgfqpoint{1.421764in}{1.429946in}}%
\pgfpathlineto{\pgfqpoint{1.383239in}{1.410897in}}%
\pgfpathclose%
\pgfusepath{fill}%
\end{pgfscope}%
\begin{pgfscope}%
\pgfpathrectangle{\pgfqpoint{0.150000in}{0.150000in}}{\pgfqpoint{2.700000in}{1.950000in}}%
\pgfusepath{clip}%
\pgfsetbuttcap%
\pgfsetroundjoin%
\definecolor{currentfill}{rgb}{0.524219,0.582812,0.664844}%
\pgfsetfillcolor{currentfill}%
\pgfsetlinewidth{0.000000pt}%
\definecolor{currentstroke}{rgb}{0.000000,0.000000,0.000000}%
\pgfsetstrokecolor{currentstroke}%
\pgfsetdash{}{0pt}%
\pgfpathmoveto{\pgfqpoint{1.344571in}{1.391778in}}%
\pgfpathlineto{\pgfqpoint{1.383239in}{1.410897in}}%
\pgfpathlineto{\pgfqpoint{1.345283in}{1.429946in}}%
\pgfpathlineto{\pgfqpoint{1.306616in}{1.410897in}}%
\pgfpathclose%
\pgfusepath{fill}%
\end{pgfscope}%
\begin{pgfscope}%
\pgfpathrectangle{\pgfqpoint{0.150000in}{0.150000in}}{\pgfqpoint{2.700000in}{1.950000in}}%
\pgfusepath{clip}%
\pgfsetbuttcap%
\pgfsetroundjoin%
\definecolor{currentfill}{rgb}{0.524219,0.582812,0.664844}%
\pgfsetfillcolor{currentfill}%
\pgfsetlinewidth{0.000000pt}%
\definecolor{currentstroke}{rgb}{0.000000,0.000000,0.000000}%
\pgfsetstrokecolor{currentstroke}%
\pgfsetdash{}{0pt}%
\pgfpathmoveto{\pgfqpoint{1.267805in}{1.391778in}}%
\pgfpathlineto{\pgfqpoint{1.306616in}{1.410897in}}%
\pgfpathlineto{\pgfqpoint{1.268801in}{1.429946in}}%
\pgfpathlineto{\pgfqpoint{1.229992in}{1.410897in}}%
\pgfpathclose%
\pgfusepath{fill}%
\end{pgfscope}%
\begin{pgfscope}%
\pgfpathrectangle{\pgfqpoint{0.150000in}{0.150000in}}{\pgfqpoint{2.700000in}{1.950000in}}%
\pgfusepath{clip}%
\pgfsetbuttcap%
\pgfsetroundjoin%
\definecolor{currentfill}{rgb}{0.524219,0.582812,0.664844}%
\pgfsetfillcolor{currentfill}%
\pgfsetlinewidth{0.000000pt}%
\definecolor{currentstroke}{rgb}{0.000000,0.000000,0.000000}%
\pgfsetstrokecolor{currentstroke}%
\pgfsetdash{}{0pt}%
\pgfpathmoveto{\pgfqpoint{1.191039in}{1.391778in}}%
\pgfpathlineto{\pgfqpoint{1.229992in}{1.410897in}}%
\pgfpathlineto{\pgfqpoint{1.192319in}{1.429946in}}%
\pgfpathlineto{\pgfqpoint{1.153369in}{1.410897in}}%
\pgfpathclose%
\pgfusepath{fill}%
\end{pgfscope}%
\begin{pgfscope}%
\pgfpathrectangle{\pgfqpoint{0.150000in}{0.150000in}}{\pgfqpoint{2.700000in}{1.950000in}}%
\pgfusepath{clip}%
\pgfsetbuttcap%
\pgfsetroundjoin%
\definecolor{currentfill}{rgb}{0.524219,0.582812,0.664844}%
\pgfsetfillcolor{currentfill}%
\pgfsetlinewidth{0.000000pt}%
\definecolor{currentstroke}{rgb}{0.000000,0.000000,0.000000}%
\pgfsetstrokecolor{currentstroke}%
\pgfsetdash{}{0pt}%
\pgfpathmoveto{\pgfqpoint{1.114273in}{1.391778in}}%
\pgfpathlineto{\pgfqpoint{1.153369in}{1.410897in}}%
\pgfpathlineto{\pgfqpoint{1.115838in}{1.429946in}}%
\pgfpathlineto{\pgfqpoint{1.076745in}{1.410897in}}%
\pgfpathclose%
\pgfusepath{fill}%
\end{pgfscope}%
\begin{pgfscope}%
\pgfpathrectangle{\pgfqpoint{0.150000in}{0.150000in}}{\pgfqpoint{2.700000in}{1.950000in}}%
\pgfusepath{clip}%
\pgfsetbuttcap%
\pgfsetroundjoin%
\definecolor{currentfill}{rgb}{0.524219,0.582812,0.664844}%
\pgfsetfillcolor{currentfill}%
\pgfsetlinewidth{0.000000pt}%
\definecolor{currentstroke}{rgb}{0.000000,0.000000,0.000000}%
\pgfsetstrokecolor{currentstroke}%
\pgfsetdash{}{0pt}%
\pgfpathmoveto{\pgfqpoint{1.037507in}{1.391778in}}%
\pgfpathlineto{\pgfqpoint{1.076745in}{1.410897in}}%
\pgfpathlineto{\pgfqpoint{1.039356in}{1.429946in}}%
\pgfpathlineto{\pgfqpoint{1.000122in}{1.410897in}}%
\pgfpathclose%
\pgfusepath{fill}%
\end{pgfscope}%
\begin{pgfscope}%
\pgfpathrectangle{\pgfqpoint{0.150000in}{0.150000in}}{\pgfqpoint{2.700000in}{1.950000in}}%
\pgfusepath{clip}%
\pgfsetbuttcap%
\pgfsetroundjoin%
\definecolor{currentfill}{rgb}{0.524219,0.582812,0.664844}%
\pgfsetfillcolor{currentfill}%
\pgfsetlinewidth{0.000000pt}%
\definecolor{currentstroke}{rgb}{0.000000,0.000000,0.000000}%
\pgfsetstrokecolor{currentstroke}%
\pgfsetdash{}{0pt}%
\pgfpathmoveto{\pgfqpoint{0.960741in}{1.391778in}}%
\pgfpathlineto{\pgfqpoint{1.000122in}{1.410897in}}%
\pgfpathlineto{\pgfqpoint{0.962875in}{1.429946in}}%
\pgfpathlineto{\pgfqpoint{0.923498in}{1.410897in}}%
\pgfpathclose%
\pgfusepath{fill}%
\end{pgfscope}%
\begin{pgfscope}%
\pgfpathrectangle{\pgfqpoint{0.150000in}{0.150000in}}{\pgfqpoint{2.700000in}{1.950000in}}%
\pgfusepath{clip}%
\pgfsetbuttcap%
\pgfsetroundjoin%
\definecolor{currentfill}{rgb}{0.959574,0.964553,0.971523}%
\pgfsetfillcolor{currentfill}%
\pgfsetlinewidth{0.000000pt}%
\definecolor{currentstroke}{rgb}{0.000000,0.000000,0.000000}%
\pgfsetstrokecolor{currentstroke}%
\pgfsetdash{}{0pt}%
\pgfpathmoveto{\pgfqpoint{1.498228in}{0.653533in}}%
\pgfpathlineto{\pgfqpoint{1.536486in}{0.673956in}}%
\pgfpathlineto{\pgfqpoint{1.498114in}{0.761602in}}%
\pgfpathlineto{\pgfqpoint{1.459600in}{0.741244in}}%
\pgfpathclose%
\pgfusepath{fill}%
\end{pgfscope}%
\begin{pgfscope}%
\pgfpathrectangle{\pgfqpoint{0.150000in}{0.150000in}}{\pgfqpoint{2.700000in}{1.950000in}}%
\pgfusepath{clip}%
\pgfsetbuttcap%
\pgfsetroundjoin%
\definecolor{currentfill}{rgb}{0.586412,0.637347,0.708655}%
\pgfsetfillcolor{currentfill}%
\pgfsetlinewidth{0.000000pt}%
\definecolor{currentstroke}{rgb}{0.000000,0.000000,0.000000}%
\pgfsetstrokecolor{currentstroke}%
\pgfsetdash{}{0pt}%
\pgfpathmoveto{\pgfqpoint{1.689680in}{1.250657in}}%
\pgfpathlineto{\pgfqpoint{1.727624in}{1.269995in}}%
\pgfpathlineto{\pgfqpoint{1.689962in}{1.348648in}}%
\pgfpathlineto{\pgfqpoint{1.651808in}{1.329385in}}%
\pgfpathclose%
\pgfusepath{fill}%
\end{pgfscope}%
\begin{pgfscope}%
\pgfpathrectangle{\pgfqpoint{0.150000in}{0.150000in}}{\pgfqpoint{2.700000in}{1.950000in}}%
\pgfusepath{clip}%
\pgfsetbuttcap%
\pgfsetroundjoin%
\definecolor{currentfill}{rgb}{0.661045,0.702788,0.761229}%
\pgfsetfillcolor{currentfill}%
\pgfsetlinewidth{0.000000pt}%
\definecolor{currentstroke}{rgb}{0.000000,0.000000,0.000000}%
\pgfsetstrokecolor{currentstroke}%
\pgfsetdash{}{0pt}%
\pgfpathmoveto{\pgfqpoint{1.651383in}{1.133469in}}%
\pgfpathlineto{\pgfqpoint{1.689399in}{1.153025in}}%
\pgfpathlineto{\pgfqpoint{1.651595in}{1.231246in}}%
\pgfpathlineto{\pgfqpoint{1.613369in}{1.211764in}}%
\pgfpathclose%
\pgfusepath{fill}%
\end{pgfscope}%
\begin{pgfscope}%
\pgfpathrectangle{\pgfqpoint{0.150000in}{0.150000in}}{\pgfqpoint{2.700000in}{1.950000in}}%
\pgfusepath{clip}%
\pgfsetbuttcap%
\pgfsetroundjoin%
\definecolor{currentfill}{rgb}{0.729458,0.762776,0.809421}%
\pgfsetfillcolor{currentfill}%
\pgfsetlinewidth{0.000000pt}%
\definecolor{currentstroke}{rgb}{0.000000,0.000000,0.000000}%
\pgfsetstrokecolor{currentstroke}%
\pgfsetdash{}{0pt}%
\pgfpathmoveto{\pgfqpoint{1.613086in}{1.016278in}}%
\pgfpathlineto{\pgfqpoint{1.651172in}{1.036051in}}%
\pgfpathlineto{\pgfqpoint{1.613227in}{1.113841in}}%
\pgfpathlineto{\pgfqpoint{1.574928in}{1.094139in}}%
\pgfpathclose%
\pgfusepath{fill}%
\end{pgfscope}%
\begin{pgfscope}%
\pgfpathrectangle{\pgfqpoint{0.150000in}{0.150000in}}{\pgfqpoint{2.700000in}{1.950000in}}%
\pgfusepath{clip}%
\pgfsetbuttcap%
\pgfsetroundjoin%
\definecolor{currentfill}{rgb}{0.804090,0.828217,0.861994}%
\pgfsetfillcolor{currentfill}%
\pgfsetlinewidth{0.000000pt}%
\definecolor{currentstroke}{rgb}{0.000000,0.000000,0.000000}%
\pgfsetstrokecolor{currentstroke}%
\pgfsetdash{}{0pt}%
\pgfpathmoveto{\pgfqpoint{1.574787in}{0.899084in}}%
\pgfpathlineto{\pgfqpoint{1.612945in}{0.919074in}}%
\pgfpathlineto{\pgfqpoint{1.574857in}{0.996432in}}%
\pgfpathlineto{\pgfqpoint{1.536486in}{0.976511in}}%
\pgfpathclose%
\pgfusepath{fill}%
\end{pgfscope}%
\begin{pgfscope}%
\pgfpathrectangle{\pgfqpoint{0.150000in}{0.150000in}}{\pgfqpoint{2.700000in}{1.950000in}}%
\pgfusepath{clip}%
\pgfsetbuttcap%
\pgfsetroundjoin%
\definecolor{currentfill}{rgb}{0.878722,0.893658,0.914568}%
\pgfsetfillcolor{currentfill}%
\pgfsetlinewidth{0.000000pt}%
\definecolor{currentstroke}{rgb}{0.000000,0.000000,0.000000}%
\pgfsetstrokecolor{currentstroke}%
\pgfsetdash{}{0pt}%
\pgfpathmoveto{\pgfqpoint{1.536486in}{0.781885in}}%
\pgfpathlineto{\pgfqpoint{1.574716in}{0.802093in}}%
\pgfpathlineto{\pgfqpoint{1.536486in}{0.879019in}}%
\pgfpathlineto{\pgfqpoint{1.498044in}{0.858879in}}%
\pgfpathclose%
\pgfusepath{fill}%
\end{pgfscope}%
\begin{pgfscope}%
\pgfpathrectangle{\pgfqpoint{0.150000in}{0.150000in}}{\pgfqpoint{2.700000in}{1.950000in}}%
\pgfusepath{clip}%
\pgfsetbuttcap%
\pgfsetroundjoin%
\definecolor{currentfill}{rgb}{0.536657,0.593719,0.673606}%
\pgfsetfillcolor{currentfill}%
\pgfsetlinewidth{0.000000pt}%
\definecolor{currentstroke}{rgb}{0.000000,0.000000,0.000000}%
\pgfsetstrokecolor{currentstroke}%
\pgfsetdash{}{0pt}%
\pgfpathmoveto{\pgfqpoint{1.689962in}{1.348648in}}%
\pgfpathlineto{\pgfqpoint{1.727975in}{1.367840in}}%
\pgfpathlineto{\pgfqpoint{1.689734in}{1.410897in}}%
\pgfpathlineto{\pgfqpoint{1.651636in}{1.391778in}}%
\pgfpathclose%
\pgfusepath{fill}%
\end{pgfscope}%
\begin{pgfscope}%
\pgfpathrectangle{\pgfqpoint{0.150000in}{0.150000in}}{\pgfqpoint{2.700000in}{1.950000in}}%
\pgfusepath{clip}%
\pgfsetbuttcap%
\pgfsetroundjoin%
\definecolor{currentfill}{rgb}{0.511780,0.571906,0.656081}%
\pgfsetfillcolor{currentfill}%
\pgfsetlinewidth{0.000000pt}%
\definecolor{currentstroke}{rgb}{0.000000,0.000000,0.000000}%
\pgfsetstrokecolor{currentstroke}%
\pgfsetdash{}{0pt}%
\pgfpathmoveto{\pgfqpoint{0.883248in}{1.403787in}}%
\pgfpathlineto{\pgfqpoint{0.923498in}{1.410897in}}%
\pgfpathlineto{\pgfqpoint{0.886393in}{1.429946in}}%
\pgfpathlineto{\pgfqpoint{0.845339in}{1.434940in}}%
\pgfpathclose%
\pgfusepath{fill}%
\end{pgfscope}%
\begin{pgfscope}%
\pgfpathrectangle{\pgfqpoint{0.150000in}{0.150000in}}{\pgfqpoint{2.700000in}{1.950000in}}%
\pgfusepath{clip}%
\pgfsetbuttcap%
\pgfsetroundjoin%
\definecolor{currentfill}{rgb}{0.979105,0.962086,0.963434}%
\pgfsetfillcolor{currentfill}%
\pgfsetlinewidth{0.000000pt}%
\definecolor{currentstroke}{rgb}{0.000000,0.000000,0.000000}%
\pgfsetstrokecolor{currentstroke}%
\pgfsetdash{}{0pt}%
\pgfpathmoveto{\pgfqpoint{1.421495in}{0.515740in}}%
\pgfpathlineto{\pgfqpoint{1.459967in}{0.536457in}}%
\pgfpathlineto{\pgfqpoint{1.421283in}{0.612458in}}%
\pgfpathlineto{\pgfqpoint{1.382595in}{0.591806in}}%
\pgfpathclose%
\pgfusepath{fill}%
\end{pgfscope}%
\begin{pgfscope}%
\pgfpathrectangle{\pgfqpoint{0.150000in}{0.150000in}}{\pgfqpoint{2.700000in}{1.950000in}}%
\pgfusepath{clip}%
\pgfsetbuttcap%
\pgfsetroundjoin%
\definecolor{currentfill}{rgb}{0.480683,0.544638,0.634176}%
\pgfsetfillcolor{currentfill}%
\pgfsetlinewidth{0.000000pt}%
\definecolor{currentstroke}{rgb}{0.000000,0.000000,0.000000}%
\pgfsetstrokecolor{currentstroke}%
\pgfsetdash{}{0pt}%
\pgfpathmoveto{\pgfqpoint{0.766832in}{1.459090in}}%
\pgfpathlineto{\pgfqpoint{0.808297in}{1.453986in}}%
\pgfpathlineto{\pgfqpoint{0.771392in}{1.472961in}}%
\pgfpathlineto{\pgfqpoint{0.729853in}{1.478133in}}%
\pgfpathclose%
\pgfusepath{fill}%
\end{pgfscope}%
\begin{pgfscope}%
\pgfpathrectangle{\pgfqpoint{0.150000in}{0.150000in}}{\pgfqpoint{2.700000in}{1.950000in}}%
\pgfusepath{clip}%
\pgfsetbuttcap%
\pgfsetroundjoin%
\definecolor{currentfill}{rgb}{0.524219,0.582812,0.664844}%
\pgfsetfillcolor{currentfill}%
\pgfsetlinewidth{0.000000pt}%
\definecolor{currentstroke}{rgb}{0.000000,0.000000,0.000000}%
\pgfsetstrokecolor{currentstroke}%
\pgfsetdash{}{0pt}%
\pgfpathmoveto{\pgfqpoint{1.613396in}{1.372588in}}%
\pgfpathlineto{\pgfqpoint{1.651636in}{1.391778in}}%
\pgfpathlineto{\pgfqpoint{1.613110in}{1.410897in}}%
\pgfpathlineto{\pgfqpoint{1.574870in}{1.391778in}}%
\pgfpathclose%
\pgfusepath{fill}%
\end{pgfscope}%
\begin{pgfscope}%
\pgfpathrectangle{\pgfqpoint{0.150000in}{0.150000in}}{\pgfqpoint{2.700000in}{1.950000in}}%
\pgfusepath{clip}%
\pgfsetbuttcap%
\pgfsetroundjoin%
\definecolor{currentfill}{rgb}{0.524219,0.582812,0.664844}%
\pgfsetfillcolor{currentfill}%
\pgfsetlinewidth{0.000000pt}%
\definecolor{currentstroke}{rgb}{0.000000,0.000000,0.000000}%
\pgfsetstrokecolor{currentstroke}%
\pgfsetdash{}{0pt}%
\pgfpathmoveto{\pgfqpoint{1.536486in}{1.372588in}}%
\pgfpathlineto{\pgfqpoint{1.574870in}{1.391778in}}%
\pgfpathlineto{\pgfqpoint{1.536486in}{1.410897in}}%
\pgfpathlineto{\pgfqpoint{1.498103in}{1.391778in}}%
\pgfpathclose%
\pgfusepath{fill}%
\end{pgfscope}%
\begin{pgfscope}%
\pgfpathrectangle{\pgfqpoint{0.150000in}{0.150000in}}{\pgfqpoint{2.700000in}{1.950000in}}%
\pgfusepath{clip}%
\pgfsetbuttcap%
\pgfsetroundjoin%
\definecolor{currentfill}{rgb}{0.524219,0.582812,0.664844}%
\pgfsetfillcolor{currentfill}%
\pgfsetlinewidth{0.000000pt}%
\definecolor{currentstroke}{rgb}{0.000000,0.000000,0.000000}%
\pgfsetstrokecolor{currentstroke}%
\pgfsetdash{}{0pt}%
\pgfpathmoveto{\pgfqpoint{1.459577in}{1.372588in}}%
\pgfpathlineto{\pgfqpoint{1.498103in}{1.391778in}}%
\pgfpathlineto{\pgfqpoint{1.459863in}{1.410897in}}%
\pgfpathlineto{\pgfqpoint{1.421337in}{1.391778in}}%
\pgfpathclose%
\pgfusepath{fill}%
\end{pgfscope}%
\begin{pgfscope}%
\pgfpathrectangle{\pgfqpoint{0.150000in}{0.150000in}}{\pgfqpoint{2.700000in}{1.950000in}}%
\pgfusepath{clip}%
\pgfsetbuttcap%
\pgfsetroundjoin%
\definecolor{currentfill}{rgb}{0.524219,0.582812,0.664844}%
\pgfsetfillcolor{currentfill}%
\pgfsetlinewidth{0.000000pt}%
\definecolor{currentstroke}{rgb}{0.000000,0.000000,0.000000}%
\pgfsetstrokecolor{currentstroke}%
\pgfsetdash{}{0pt}%
\pgfpathmoveto{\pgfqpoint{1.382668in}{1.372588in}}%
\pgfpathlineto{\pgfqpoint{1.421337in}{1.391778in}}%
\pgfpathlineto{\pgfqpoint{1.383239in}{1.410897in}}%
\pgfpathlineto{\pgfqpoint{1.344571in}{1.391778in}}%
\pgfpathclose%
\pgfusepath{fill}%
\end{pgfscope}%
\begin{pgfscope}%
\pgfpathrectangle{\pgfqpoint{0.150000in}{0.150000in}}{\pgfqpoint{2.700000in}{1.950000in}}%
\pgfusepath{clip}%
\pgfsetbuttcap%
\pgfsetroundjoin%
\definecolor{currentfill}{rgb}{0.524219,0.582812,0.664844}%
\pgfsetfillcolor{currentfill}%
\pgfsetlinewidth{0.000000pt}%
\definecolor{currentstroke}{rgb}{0.000000,0.000000,0.000000}%
\pgfsetstrokecolor{currentstroke}%
\pgfsetdash{}{0pt}%
\pgfpathmoveto{\pgfqpoint{1.305759in}{1.372588in}}%
\pgfpathlineto{\pgfqpoint{1.344571in}{1.391778in}}%
\pgfpathlineto{\pgfqpoint{1.306616in}{1.410897in}}%
\pgfpathlineto{\pgfqpoint{1.267805in}{1.391778in}}%
\pgfpathclose%
\pgfusepath{fill}%
\end{pgfscope}%
\begin{pgfscope}%
\pgfpathrectangle{\pgfqpoint{0.150000in}{0.150000in}}{\pgfqpoint{2.700000in}{1.950000in}}%
\pgfusepath{clip}%
\pgfsetbuttcap%
\pgfsetroundjoin%
\definecolor{currentfill}{rgb}{0.524219,0.582812,0.664844}%
\pgfsetfillcolor{currentfill}%
\pgfsetlinewidth{0.000000pt}%
\definecolor{currentstroke}{rgb}{0.000000,0.000000,0.000000}%
\pgfsetstrokecolor{currentstroke}%
\pgfsetdash{}{0pt}%
\pgfpathmoveto{\pgfqpoint{1.228850in}{1.372588in}}%
\pgfpathlineto{\pgfqpoint{1.267805in}{1.391778in}}%
\pgfpathlineto{\pgfqpoint{1.229992in}{1.410897in}}%
\pgfpathlineto{\pgfqpoint{1.191039in}{1.391778in}}%
\pgfpathclose%
\pgfusepath{fill}%
\end{pgfscope}%
\begin{pgfscope}%
\pgfpathrectangle{\pgfqpoint{0.150000in}{0.150000in}}{\pgfqpoint{2.700000in}{1.950000in}}%
\pgfusepath{clip}%
\pgfsetbuttcap%
\pgfsetroundjoin%
\definecolor{currentfill}{rgb}{0.524219,0.582812,0.664844}%
\pgfsetfillcolor{currentfill}%
\pgfsetlinewidth{0.000000pt}%
\definecolor{currentstroke}{rgb}{0.000000,0.000000,0.000000}%
\pgfsetstrokecolor{currentstroke}%
\pgfsetdash{}{0pt}%
\pgfpathmoveto{\pgfqpoint{1.151941in}{1.372588in}}%
\pgfpathlineto{\pgfqpoint{1.191039in}{1.391778in}}%
\pgfpathlineto{\pgfqpoint{1.153369in}{1.410897in}}%
\pgfpathlineto{\pgfqpoint{1.114273in}{1.391778in}}%
\pgfpathclose%
\pgfusepath{fill}%
\end{pgfscope}%
\begin{pgfscope}%
\pgfpathrectangle{\pgfqpoint{0.150000in}{0.150000in}}{\pgfqpoint{2.700000in}{1.950000in}}%
\pgfusepath{clip}%
\pgfsetbuttcap%
\pgfsetroundjoin%
\definecolor{currentfill}{rgb}{0.524219,0.582812,0.664844}%
\pgfsetfillcolor{currentfill}%
\pgfsetlinewidth{0.000000pt}%
\definecolor{currentstroke}{rgb}{0.000000,0.000000,0.000000}%
\pgfsetstrokecolor{currentstroke}%
\pgfsetdash{}{0pt}%
\pgfpathmoveto{\pgfqpoint{1.075032in}{1.372588in}}%
\pgfpathlineto{\pgfqpoint{1.114273in}{1.391778in}}%
\pgfpathlineto{\pgfqpoint{1.076745in}{1.410897in}}%
\pgfpathlineto{\pgfqpoint{1.037507in}{1.391778in}}%
\pgfpathclose%
\pgfusepath{fill}%
\end{pgfscope}%
\begin{pgfscope}%
\pgfpathrectangle{\pgfqpoint{0.150000in}{0.150000in}}{\pgfqpoint{2.700000in}{1.950000in}}%
\pgfusepath{clip}%
\pgfsetbuttcap%
\pgfsetroundjoin%
\definecolor{currentfill}{rgb}{0.524219,0.582812,0.664844}%
\pgfsetfillcolor{currentfill}%
\pgfsetlinewidth{0.000000pt}%
\definecolor{currentstroke}{rgb}{0.000000,0.000000,0.000000}%
\pgfsetstrokecolor{currentstroke}%
\pgfsetdash{}{0pt}%
\pgfpathmoveto{\pgfqpoint{0.998123in}{1.372588in}}%
\pgfpathlineto{\pgfqpoint{1.037507in}{1.391778in}}%
\pgfpathlineto{\pgfqpoint{1.000122in}{1.410897in}}%
\pgfpathlineto{\pgfqpoint{0.960741in}{1.391778in}}%
\pgfpathclose%
\pgfusepath{fill}%
\end{pgfscope}%
\begin{pgfscope}%
\pgfpathrectangle{\pgfqpoint{0.150000in}{0.150000in}}{\pgfqpoint{2.700000in}{1.950000in}}%
\pgfusepath{clip}%
\pgfsetbuttcap%
\pgfsetroundjoin%
\definecolor{currentfill}{rgb}{0.959574,0.964553,0.971523}%
\pgfsetfillcolor{currentfill}%
\pgfsetlinewidth{0.000000pt}%
\definecolor{currentstroke}{rgb}{0.000000,0.000000,0.000000}%
\pgfsetstrokecolor{currentstroke}%
\pgfsetdash{}{0pt}%
\pgfpathmoveto{\pgfqpoint{1.459827in}{0.633034in}}%
\pgfpathlineto{\pgfqpoint{1.498228in}{0.653533in}}%
\pgfpathlineto{\pgfqpoint{1.459600in}{0.741244in}}%
\pgfpathlineto{\pgfqpoint{1.420941in}{0.720809in}}%
\pgfpathclose%
\pgfusepath{fill}%
\end{pgfscope}%
\begin{pgfscope}%
\pgfpathrectangle{\pgfqpoint{0.150000in}{0.150000in}}{\pgfqpoint{2.700000in}{1.950000in}}%
\pgfusepath{clip}%
\pgfsetbuttcap%
\pgfsetroundjoin%
\definecolor{currentfill}{rgb}{0.517999,0.577359,0.660463}%
\pgfsetfillcolor{currentfill}%
\pgfsetlinewidth{0.000000pt}%
\definecolor{currentstroke}{rgb}{0.000000,0.000000,0.000000}%
\pgfsetstrokecolor{currentstroke}%
\pgfsetdash{}{0pt}%
\pgfpathmoveto{\pgfqpoint{0.920527in}{1.384598in}}%
\pgfpathlineto{\pgfqpoint{0.960741in}{1.391778in}}%
\pgfpathlineto{\pgfqpoint{0.923498in}{1.410897in}}%
\pgfpathlineto{\pgfqpoint{0.883248in}{1.403787in}}%
\pgfpathclose%
\pgfusepath{fill}%
\end{pgfscope}%
\begin{pgfscope}%
\pgfpathrectangle{\pgfqpoint{0.150000in}{0.150000in}}{\pgfqpoint{2.700000in}{1.950000in}}%
\pgfusepath{clip}%
\pgfsetbuttcap%
\pgfsetroundjoin%
\definecolor{currentfill}{rgb}{0.586412,0.637347,0.708655}%
\pgfsetfillcolor{currentfill}%
\pgfsetlinewidth{0.000000pt}%
\definecolor{currentstroke}{rgb}{0.000000,0.000000,0.000000}%
\pgfsetstrokecolor{currentstroke}%
\pgfsetdash{}{0pt}%
\pgfpathmoveto{\pgfqpoint{1.651595in}{1.231246in}}%
\pgfpathlineto{\pgfqpoint{1.689680in}{1.250657in}}%
\pgfpathlineto{\pgfqpoint{1.651808in}{1.329385in}}%
\pgfpathlineto{\pgfqpoint{1.613511in}{1.310049in}}%
\pgfpathclose%
\pgfusepath{fill}%
\end{pgfscope}%
\begin{pgfscope}%
\pgfpathrectangle{\pgfqpoint{0.150000in}{0.150000in}}{\pgfqpoint{2.700000in}{1.950000in}}%
\pgfusepath{clip}%
\pgfsetbuttcap%
\pgfsetroundjoin%
\definecolor{currentfill}{rgb}{0.661045,0.702788,0.761229}%
\pgfsetfillcolor{currentfill}%
\pgfsetlinewidth{0.000000pt}%
\definecolor{currentstroke}{rgb}{0.000000,0.000000,0.000000}%
\pgfsetstrokecolor{currentstroke}%
\pgfsetdash{}{0pt}%
\pgfpathmoveto{\pgfqpoint{1.613227in}{1.113841in}}%
\pgfpathlineto{\pgfqpoint{1.651383in}{1.133469in}}%
\pgfpathlineto{\pgfqpoint{1.613369in}{1.211764in}}%
\pgfpathlineto{\pgfqpoint{1.574999in}{1.192208in}}%
\pgfpathclose%
\pgfusepath{fill}%
\end{pgfscope}%
\begin{pgfscope}%
\pgfpathrectangle{\pgfqpoint{0.150000in}{0.150000in}}{\pgfqpoint{2.700000in}{1.950000in}}%
\pgfusepath{clip}%
\pgfsetbuttcap%
\pgfsetroundjoin%
\definecolor{currentfill}{rgb}{0.729458,0.762776,0.809421}%
\pgfsetfillcolor{currentfill}%
\pgfsetlinewidth{0.000000pt}%
\definecolor{currentstroke}{rgb}{0.000000,0.000000,0.000000}%
\pgfsetstrokecolor{currentstroke}%
\pgfsetdash{}{0pt}%
\pgfpathmoveto{\pgfqpoint{1.574857in}{0.996432in}}%
\pgfpathlineto{\pgfqpoint{1.613086in}{1.016278in}}%
\pgfpathlineto{\pgfqpoint{1.574928in}{1.094139in}}%
\pgfpathlineto{\pgfqpoint{1.536486in}{1.074364in}}%
\pgfpathclose%
\pgfusepath{fill}%
\end{pgfscope}%
\begin{pgfscope}%
\pgfpathrectangle{\pgfqpoint{0.150000in}{0.150000in}}{\pgfqpoint{2.700000in}{1.950000in}}%
\pgfusepath{clip}%
\pgfsetbuttcap%
\pgfsetroundjoin%
\definecolor{currentfill}{rgb}{0.804090,0.828217,0.861994}%
\pgfsetfillcolor{currentfill}%
\pgfsetlinewidth{0.000000pt}%
\definecolor{currentstroke}{rgb}{0.000000,0.000000,0.000000}%
\pgfsetstrokecolor{currentstroke}%
\pgfsetdash{}{0pt}%
\pgfpathmoveto{\pgfqpoint{1.536486in}{0.879019in}}%
\pgfpathlineto{\pgfqpoint{1.574787in}{0.899084in}}%
\pgfpathlineto{\pgfqpoint{1.536486in}{0.976511in}}%
\pgfpathlineto{\pgfqpoint{1.497972in}{0.956516in}}%
\pgfpathclose%
\pgfusepath{fill}%
\end{pgfscope}%
\begin{pgfscope}%
\pgfpathrectangle{\pgfqpoint{0.150000in}{0.150000in}}{\pgfqpoint{2.700000in}{1.950000in}}%
\pgfusepath{clip}%
\pgfsetbuttcap%
\pgfsetroundjoin%
\definecolor{currentfill}{rgb}{0.878722,0.893658,0.914568}%
\pgfsetfillcolor{currentfill}%
\pgfsetlinewidth{0.000000pt}%
\definecolor{currentstroke}{rgb}{0.000000,0.000000,0.000000}%
\pgfsetstrokecolor{currentstroke}%
\pgfsetdash{}{0pt}%
\pgfpathmoveto{\pgfqpoint{1.498114in}{0.761602in}}%
\pgfpathlineto{\pgfqpoint{1.536486in}{0.781885in}}%
\pgfpathlineto{\pgfqpoint{1.498044in}{0.858879in}}%
\pgfpathlineto{\pgfqpoint{1.459457in}{0.838665in}}%
\pgfpathclose%
\pgfusepath{fill}%
\end{pgfscope}%
\begin{pgfscope}%
\pgfpathrectangle{\pgfqpoint{0.150000in}{0.150000in}}{\pgfqpoint{2.700000in}{1.950000in}}%
\pgfusepath{clip}%
\pgfsetbuttcap%
\pgfsetroundjoin%
\definecolor{currentfill}{rgb}{0.536657,0.593719,0.673606}%
\pgfsetfillcolor{currentfill}%
\pgfsetlinewidth{0.000000pt}%
\definecolor{currentstroke}{rgb}{0.000000,0.000000,0.000000}%
\pgfsetstrokecolor{currentstroke}%
\pgfsetdash{}{0pt}%
\pgfpathmoveto{\pgfqpoint{1.651808in}{1.329385in}}%
\pgfpathlineto{\pgfqpoint{1.689962in}{1.348648in}}%
\pgfpathlineto{\pgfqpoint{1.651636in}{1.391778in}}%
\pgfpathlineto{\pgfqpoint{1.613396in}{1.372588in}}%
\pgfpathclose%
\pgfusepath{fill}%
\end{pgfscope}%
\begin{pgfscope}%
\pgfpathrectangle{\pgfqpoint{0.150000in}{0.150000in}}{\pgfqpoint{2.700000in}{1.950000in}}%
\pgfusepath{clip}%
\pgfsetbuttcap%
\pgfsetroundjoin%
\definecolor{currentfill}{rgb}{0.480683,0.544638,0.634176}%
\pgfsetfillcolor{currentfill}%
\pgfsetlinewidth{0.000000pt}%
\definecolor{currentstroke}{rgb}{0.000000,0.000000,0.000000}%
\pgfsetstrokecolor{currentstroke}%
\pgfsetdash{}{0pt}%
\pgfpathmoveto{\pgfqpoint{0.803949in}{1.439976in}}%
\pgfpathlineto{\pgfqpoint{0.845339in}{1.434940in}}%
\pgfpathlineto{\pgfqpoint{0.808297in}{1.453986in}}%
\pgfpathlineto{\pgfqpoint{0.766832in}{1.459090in}}%
\pgfpathclose%
\pgfusepath{fill}%
\end{pgfscope}%
\begin{pgfscope}%
\pgfpathrectangle{\pgfqpoint{0.150000in}{0.150000in}}{\pgfqpoint{2.700000in}{1.950000in}}%
\pgfusepath{clip}%
\pgfsetbuttcap%
\pgfsetroundjoin%
\definecolor{currentfill}{rgb}{0.524219,0.582812,0.664844}%
\pgfsetfillcolor{currentfill}%
\pgfsetlinewidth{0.000000pt}%
\definecolor{currentstroke}{rgb}{0.000000,0.000000,0.000000}%
\pgfsetstrokecolor{currentstroke}%
\pgfsetdash{}{0pt}%
\pgfpathmoveto{\pgfqpoint{1.575013in}{1.353326in}}%
\pgfpathlineto{\pgfqpoint{1.613396in}{1.372588in}}%
\pgfpathlineto{\pgfqpoint{1.574870in}{1.391778in}}%
\pgfpathlineto{\pgfqpoint{1.536486in}{1.372588in}}%
\pgfpathclose%
\pgfusepath{fill}%
\end{pgfscope}%
\begin{pgfscope}%
\pgfpathrectangle{\pgfqpoint{0.150000in}{0.150000in}}{\pgfqpoint{2.700000in}{1.950000in}}%
\pgfusepath{clip}%
\pgfsetbuttcap%
\pgfsetroundjoin%
\definecolor{currentfill}{rgb}{0.524219,0.582812,0.664844}%
\pgfsetfillcolor{currentfill}%
\pgfsetlinewidth{0.000000pt}%
\definecolor{currentstroke}{rgb}{0.000000,0.000000,0.000000}%
\pgfsetstrokecolor{currentstroke}%
\pgfsetdash{}{0pt}%
\pgfpathmoveto{\pgfqpoint{1.497960in}{1.353326in}}%
\pgfpathlineto{\pgfqpoint{1.536486in}{1.372588in}}%
\pgfpathlineto{\pgfqpoint{1.498103in}{1.391778in}}%
\pgfpathlineto{\pgfqpoint{1.459577in}{1.372588in}}%
\pgfpathclose%
\pgfusepath{fill}%
\end{pgfscope}%
\begin{pgfscope}%
\pgfpathrectangle{\pgfqpoint{0.150000in}{0.150000in}}{\pgfqpoint{2.700000in}{1.950000in}}%
\pgfusepath{clip}%
\pgfsetbuttcap%
\pgfsetroundjoin%
\definecolor{currentfill}{rgb}{0.524219,0.582812,0.664844}%
\pgfsetfillcolor{currentfill}%
\pgfsetlinewidth{0.000000pt}%
\definecolor{currentstroke}{rgb}{0.000000,0.000000,0.000000}%
\pgfsetstrokecolor{currentstroke}%
\pgfsetdash{}{0pt}%
\pgfpathmoveto{\pgfqpoint{1.420908in}{1.353326in}}%
\pgfpathlineto{\pgfqpoint{1.459577in}{1.372588in}}%
\pgfpathlineto{\pgfqpoint{1.421337in}{1.391778in}}%
\pgfpathlineto{\pgfqpoint{1.382668in}{1.372588in}}%
\pgfpathclose%
\pgfusepath{fill}%
\end{pgfscope}%
\begin{pgfscope}%
\pgfpathrectangle{\pgfqpoint{0.150000in}{0.150000in}}{\pgfqpoint{2.700000in}{1.950000in}}%
\pgfusepath{clip}%
\pgfsetbuttcap%
\pgfsetroundjoin%
\definecolor{currentfill}{rgb}{0.524219,0.582812,0.664844}%
\pgfsetfillcolor{currentfill}%
\pgfsetlinewidth{0.000000pt}%
\definecolor{currentstroke}{rgb}{0.000000,0.000000,0.000000}%
\pgfsetstrokecolor{currentstroke}%
\pgfsetdash{}{0pt}%
\pgfpathmoveto{\pgfqpoint{1.343855in}{1.353326in}}%
\pgfpathlineto{\pgfqpoint{1.382668in}{1.372588in}}%
\pgfpathlineto{\pgfqpoint{1.344571in}{1.391778in}}%
\pgfpathlineto{\pgfqpoint{1.305759in}{1.372588in}}%
\pgfpathclose%
\pgfusepath{fill}%
\end{pgfscope}%
\begin{pgfscope}%
\pgfpathrectangle{\pgfqpoint{0.150000in}{0.150000in}}{\pgfqpoint{2.700000in}{1.950000in}}%
\pgfusepath{clip}%
\pgfsetbuttcap%
\pgfsetroundjoin%
\definecolor{currentfill}{rgb}{0.524219,0.582812,0.664844}%
\pgfsetfillcolor{currentfill}%
\pgfsetlinewidth{0.000000pt}%
\definecolor{currentstroke}{rgb}{0.000000,0.000000,0.000000}%
\pgfsetstrokecolor{currentstroke}%
\pgfsetdash{}{0pt}%
\pgfpathmoveto{\pgfqpoint{1.266802in}{1.353326in}}%
\pgfpathlineto{\pgfqpoint{1.305759in}{1.372588in}}%
\pgfpathlineto{\pgfqpoint{1.267805in}{1.391778in}}%
\pgfpathlineto{\pgfqpoint{1.228850in}{1.372588in}}%
\pgfpathclose%
\pgfusepath{fill}%
\end{pgfscope}%
\begin{pgfscope}%
\pgfpathrectangle{\pgfqpoint{0.150000in}{0.150000in}}{\pgfqpoint{2.700000in}{1.950000in}}%
\pgfusepath{clip}%
\pgfsetbuttcap%
\pgfsetroundjoin%
\definecolor{currentfill}{rgb}{0.524219,0.582812,0.664844}%
\pgfsetfillcolor{currentfill}%
\pgfsetlinewidth{0.000000pt}%
\definecolor{currentstroke}{rgb}{0.000000,0.000000,0.000000}%
\pgfsetstrokecolor{currentstroke}%
\pgfsetdash{}{0pt}%
\pgfpathmoveto{\pgfqpoint{1.189750in}{1.353326in}}%
\pgfpathlineto{\pgfqpoint{1.228850in}{1.372588in}}%
\pgfpathlineto{\pgfqpoint{1.191039in}{1.391778in}}%
\pgfpathlineto{\pgfqpoint{1.151941in}{1.372588in}}%
\pgfpathclose%
\pgfusepath{fill}%
\end{pgfscope}%
\begin{pgfscope}%
\pgfpathrectangle{\pgfqpoint{0.150000in}{0.150000in}}{\pgfqpoint{2.700000in}{1.950000in}}%
\pgfusepath{clip}%
\pgfsetbuttcap%
\pgfsetroundjoin%
\definecolor{currentfill}{rgb}{0.524219,0.582812,0.664844}%
\pgfsetfillcolor{currentfill}%
\pgfsetlinewidth{0.000000pt}%
\definecolor{currentstroke}{rgb}{0.000000,0.000000,0.000000}%
\pgfsetstrokecolor{currentstroke}%
\pgfsetdash{}{0pt}%
\pgfpathmoveto{\pgfqpoint{1.112697in}{1.353326in}}%
\pgfpathlineto{\pgfqpoint{1.151941in}{1.372588in}}%
\pgfpathlineto{\pgfqpoint{1.114273in}{1.391778in}}%
\pgfpathlineto{\pgfqpoint{1.075032in}{1.372588in}}%
\pgfpathclose%
\pgfusepath{fill}%
\end{pgfscope}%
\begin{pgfscope}%
\pgfpathrectangle{\pgfqpoint{0.150000in}{0.150000in}}{\pgfqpoint{2.700000in}{1.950000in}}%
\pgfusepath{clip}%
\pgfsetbuttcap%
\pgfsetroundjoin%
\definecolor{currentfill}{rgb}{0.524219,0.582812,0.664844}%
\pgfsetfillcolor{currentfill}%
\pgfsetlinewidth{0.000000pt}%
\definecolor{currentstroke}{rgb}{0.000000,0.000000,0.000000}%
\pgfsetstrokecolor{currentstroke}%
\pgfsetdash{}{0pt}%
\pgfpathmoveto{\pgfqpoint{1.035644in}{1.353326in}}%
\pgfpathlineto{\pgfqpoint{1.075032in}{1.372588in}}%
\pgfpathlineto{\pgfqpoint{1.037507in}{1.391778in}}%
\pgfpathlineto{\pgfqpoint{0.998123in}{1.372588in}}%
\pgfpathclose%
\pgfusepath{fill}%
\end{pgfscope}%
\begin{pgfscope}%
\pgfpathrectangle{\pgfqpoint{0.150000in}{0.150000in}}{\pgfqpoint{2.700000in}{1.950000in}}%
\pgfusepath{clip}%
\pgfsetbuttcap%
\pgfsetroundjoin%
\definecolor{currentfill}{rgb}{0.959574,0.964553,0.971523}%
\pgfsetfillcolor{currentfill}%
\pgfsetlinewidth{0.000000pt}%
\definecolor{currentstroke}{rgb}{0.000000,0.000000,0.000000}%
\pgfsetstrokecolor{currentstroke}%
\pgfsetdash{}{0pt}%
\pgfpathmoveto{\pgfqpoint{1.421283in}{0.612458in}}%
\pgfpathlineto{\pgfqpoint{1.459827in}{0.633034in}}%
\pgfpathlineto{\pgfqpoint{1.420941in}{0.720809in}}%
\pgfpathlineto{\pgfqpoint{1.382138in}{0.700298in}}%
\pgfpathclose%
\pgfusepath{fill}%
\end{pgfscope}%
\begin{pgfscope}%
\pgfpathrectangle{\pgfqpoint{0.150000in}{0.150000in}}{\pgfqpoint{2.700000in}{1.950000in}}%
\pgfusepath{clip}%
\pgfsetbuttcap%
\pgfsetroundjoin%
\definecolor{currentfill}{rgb}{0.443367,0.511918,0.607889}%
\pgfsetfillcolor{currentfill}%
\pgfsetlinewidth{0.000000pt}%
\definecolor{currentstroke}{rgb}{0.000000,0.000000,0.000000}%
\pgfsetstrokecolor{currentstroke}%
\pgfsetdash{}{0pt}%
\pgfpathmoveto{\pgfqpoint{0.687023in}{1.495518in}}%
\pgfpathlineto{\pgfqpoint{0.729853in}{1.478133in}}%
\pgfpathlineto{\pgfqpoint{0.693012in}{1.497106in}}%
\pgfpathlineto{\pgfqpoint{0.650070in}{1.514557in}}%
\pgfpathclose%
\pgfusepath{fill}%
\end{pgfscope}%
\begin{pgfscope}%
\pgfpathrectangle{\pgfqpoint{0.150000in}{0.150000in}}{\pgfqpoint{2.700000in}{1.950000in}}%
\pgfusepath{clip}%
\pgfsetbuttcap%
\pgfsetroundjoin%
\definecolor{currentfill}{rgb}{0.517999,0.577359,0.660463}%
\pgfsetfillcolor{currentfill}%
\pgfsetlinewidth{0.000000pt}%
\definecolor{currentstroke}{rgb}{0.000000,0.000000,0.000000}%
\pgfsetstrokecolor{currentstroke}%
\pgfsetdash{}{0pt}%
\pgfpathmoveto{\pgfqpoint{0.957946in}{1.365337in}}%
\pgfpathlineto{\pgfqpoint{0.998123in}{1.372588in}}%
\pgfpathlineto{\pgfqpoint{0.960741in}{1.391778in}}%
\pgfpathlineto{\pgfqpoint{0.920527in}{1.384598in}}%
\pgfpathclose%
\pgfusepath{fill}%
\end{pgfscope}%
\begin{pgfscope}%
\pgfpathrectangle{\pgfqpoint{0.150000in}{0.150000in}}{\pgfqpoint{2.700000in}{1.950000in}}%
\pgfusepath{clip}%
\pgfsetbuttcap%
\pgfsetroundjoin%
\definecolor{currentfill}{rgb}{0.586412,0.637347,0.708655}%
\pgfsetfillcolor{currentfill}%
\pgfsetlinewidth{0.000000pt}%
\definecolor{currentstroke}{rgb}{0.000000,0.000000,0.000000}%
\pgfsetstrokecolor{currentstroke}%
\pgfsetdash{}{0pt}%
\pgfpathmoveto{\pgfqpoint{1.613369in}{1.211764in}}%
\pgfpathlineto{\pgfqpoint{1.651595in}{1.231246in}}%
\pgfpathlineto{\pgfqpoint{1.613511in}{1.310049in}}%
\pgfpathlineto{\pgfqpoint{1.575071in}{1.290641in}}%
\pgfpathclose%
\pgfusepath{fill}%
\end{pgfscope}%
\begin{pgfscope}%
\pgfpathrectangle{\pgfqpoint{0.150000in}{0.150000in}}{\pgfqpoint{2.700000in}{1.950000in}}%
\pgfusepath{clip}%
\pgfsetbuttcap%
\pgfsetroundjoin%
\definecolor{currentfill}{rgb}{0.661045,0.702788,0.761229}%
\pgfsetfillcolor{currentfill}%
\pgfsetlinewidth{0.000000pt}%
\definecolor{currentstroke}{rgb}{0.000000,0.000000,0.000000}%
\pgfsetstrokecolor{currentstroke}%
\pgfsetdash{}{0pt}%
\pgfpathmoveto{\pgfqpoint{1.574928in}{1.094139in}}%
\pgfpathlineto{\pgfqpoint{1.613227in}{1.113841in}}%
\pgfpathlineto{\pgfqpoint{1.574999in}{1.192208in}}%
\pgfpathlineto{\pgfqpoint{1.536486in}{1.172580in}}%
\pgfpathclose%
\pgfusepath{fill}%
\end{pgfscope}%
\begin{pgfscope}%
\pgfpathrectangle{\pgfqpoint{0.150000in}{0.150000in}}{\pgfqpoint{2.700000in}{1.950000in}}%
\pgfusepath{clip}%
\pgfsetbuttcap%
\pgfsetroundjoin%
\definecolor{currentfill}{rgb}{0.729458,0.762776,0.809421}%
\pgfsetfillcolor{currentfill}%
\pgfsetlinewidth{0.000000pt}%
\definecolor{currentstroke}{rgb}{0.000000,0.000000,0.000000}%
\pgfsetstrokecolor{currentstroke}%
\pgfsetdash{}{0pt}%
\pgfpathmoveto{\pgfqpoint{1.536486in}{0.976511in}}%
\pgfpathlineto{\pgfqpoint{1.574857in}{0.996432in}}%
\pgfpathlineto{\pgfqpoint{1.536486in}{1.074364in}}%
\pgfpathlineto{\pgfqpoint{1.497901in}{1.054515in}}%
\pgfpathclose%
\pgfusepath{fill}%
\end{pgfscope}%
\begin{pgfscope}%
\pgfpathrectangle{\pgfqpoint{0.150000in}{0.150000in}}{\pgfqpoint{2.700000in}{1.950000in}}%
\pgfusepath{clip}%
\pgfsetbuttcap%
\pgfsetroundjoin%
\definecolor{currentfill}{rgb}{0.804090,0.828217,0.861994}%
\pgfsetfillcolor{currentfill}%
\pgfsetlinewidth{0.000000pt}%
\definecolor{currentstroke}{rgb}{0.000000,0.000000,0.000000}%
\pgfsetstrokecolor{currentstroke}%
\pgfsetdash{}{0pt}%
\pgfpathmoveto{\pgfqpoint{1.498044in}{0.858879in}}%
\pgfpathlineto{\pgfqpoint{1.536486in}{0.879019in}}%
\pgfpathlineto{\pgfqpoint{1.497972in}{0.956516in}}%
\pgfpathlineto{\pgfqpoint{1.459315in}{0.936447in}}%
\pgfpathclose%
\pgfusepath{fill}%
\end{pgfscope}%
\begin{pgfscope}%
\pgfpathrectangle{\pgfqpoint{0.150000in}{0.150000in}}{\pgfqpoint{2.700000in}{1.950000in}}%
\pgfusepath{clip}%
\pgfsetbuttcap%
\pgfsetroundjoin%
\definecolor{currentfill}{rgb}{0.878722,0.893658,0.914568}%
\pgfsetfillcolor{currentfill}%
\pgfsetlinewidth{0.000000pt}%
\definecolor{currentstroke}{rgb}{0.000000,0.000000,0.000000}%
\pgfsetstrokecolor{currentstroke}%
\pgfsetdash{}{0pt}%
\pgfpathmoveto{\pgfqpoint{1.459600in}{0.741244in}}%
\pgfpathlineto{\pgfqpoint{1.498114in}{0.761602in}}%
\pgfpathlineto{\pgfqpoint{1.459457in}{0.838665in}}%
\pgfpathlineto{\pgfqpoint{1.420727in}{0.818374in}}%
\pgfpathclose%
\pgfusepath{fill}%
\end{pgfscope}%
\begin{pgfscope}%
\pgfpathrectangle{\pgfqpoint{0.150000in}{0.150000in}}{\pgfqpoint{2.700000in}{1.950000in}}%
\pgfusepath{clip}%
\pgfsetbuttcap%
\pgfsetroundjoin%
\definecolor{currentfill}{rgb}{0.536657,0.593719,0.673606}%
\pgfsetfillcolor{currentfill}%
\pgfsetlinewidth{0.000000pt}%
\definecolor{currentstroke}{rgb}{0.000000,0.000000,0.000000}%
\pgfsetstrokecolor{currentstroke}%
\pgfsetdash{}{0pt}%
\pgfpathmoveto{\pgfqpoint{1.613511in}{1.310049in}}%
\pgfpathlineto{\pgfqpoint{1.651808in}{1.329385in}}%
\pgfpathlineto{\pgfqpoint{1.613396in}{1.372588in}}%
\pgfpathlineto{\pgfqpoint{1.575013in}{1.353326in}}%
\pgfpathclose%
\pgfusepath{fill}%
\end{pgfscope}%
\begin{pgfscope}%
\pgfpathrectangle{\pgfqpoint{0.150000in}{0.150000in}}{\pgfqpoint{2.700000in}{1.950000in}}%
\pgfusepath{clip}%
\pgfsetbuttcap%
\pgfsetroundjoin%
\definecolor{currentfill}{rgb}{0.486903,0.550092,0.638557}%
\pgfsetfillcolor{currentfill}%
\pgfsetlinewidth{0.000000pt}%
\definecolor{currentstroke}{rgb}{0.000000,0.000000,0.000000}%
\pgfsetstrokecolor{currentstroke}%
\pgfsetdash{}{0pt}%
\pgfpathmoveto{\pgfqpoint{0.841982in}{1.408699in}}%
\pgfpathlineto{\pgfqpoint{0.883248in}{1.403787in}}%
\pgfpathlineto{\pgfqpoint{0.845339in}{1.434940in}}%
\pgfpathlineto{\pgfqpoint{0.803949in}{1.439976in}}%
\pgfpathclose%
\pgfusepath{fill}%
\end{pgfscope}%
\begin{pgfscope}%
\pgfpathrectangle{\pgfqpoint{0.150000in}{0.150000in}}{\pgfqpoint{2.700000in}{1.950000in}}%
\pgfusepath{clip}%
\pgfsetbuttcap%
\pgfsetroundjoin%
\definecolor{currentfill}{rgb}{0.524219,0.582812,0.664844}%
\pgfsetfillcolor{currentfill}%
\pgfsetlinewidth{0.000000pt}%
\definecolor{currentstroke}{rgb}{0.000000,0.000000,0.000000}%
\pgfsetstrokecolor{currentstroke}%
\pgfsetdash{}{0pt}%
\pgfpathmoveto{\pgfqpoint{1.536486in}{1.333992in}}%
\pgfpathlineto{\pgfqpoint{1.575013in}{1.353326in}}%
\pgfpathlineto{\pgfqpoint{1.536486in}{1.372588in}}%
\pgfpathlineto{\pgfqpoint{1.497960in}{1.353326in}}%
\pgfpathclose%
\pgfusepath{fill}%
\end{pgfscope}%
\begin{pgfscope}%
\pgfpathrectangle{\pgfqpoint{0.150000in}{0.150000in}}{\pgfqpoint{2.700000in}{1.950000in}}%
\pgfusepath{clip}%
\pgfsetbuttcap%
\pgfsetroundjoin%
\definecolor{currentfill}{rgb}{0.524219,0.582812,0.664844}%
\pgfsetfillcolor{currentfill}%
\pgfsetlinewidth{0.000000pt}%
\definecolor{currentstroke}{rgb}{0.000000,0.000000,0.000000}%
\pgfsetstrokecolor{currentstroke}%
\pgfsetdash{}{0pt}%
\pgfpathmoveto{\pgfqpoint{1.459290in}{1.333992in}}%
\pgfpathlineto{\pgfqpoint{1.497960in}{1.353326in}}%
\pgfpathlineto{\pgfqpoint{1.459577in}{1.372588in}}%
\pgfpathlineto{\pgfqpoint{1.420908in}{1.353326in}}%
\pgfpathclose%
\pgfusepath{fill}%
\end{pgfscope}%
\begin{pgfscope}%
\pgfpathrectangle{\pgfqpoint{0.150000in}{0.150000in}}{\pgfqpoint{2.700000in}{1.950000in}}%
\pgfusepath{clip}%
\pgfsetbuttcap%
\pgfsetroundjoin%
\definecolor{currentfill}{rgb}{0.524219,0.582812,0.664844}%
\pgfsetfillcolor{currentfill}%
\pgfsetlinewidth{0.000000pt}%
\definecolor{currentstroke}{rgb}{0.000000,0.000000,0.000000}%
\pgfsetstrokecolor{currentstroke}%
\pgfsetdash{}{0pt}%
\pgfpathmoveto{\pgfqpoint{1.382093in}{1.333992in}}%
\pgfpathlineto{\pgfqpoint{1.420908in}{1.353326in}}%
\pgfpathlineto{\pgfqpoint{1.382668in}{1.372588in}}%
\pgfpathlineto{\pgfqpoint{1.343855in}{1.353326in}}%
\pgfpathclose%
\pgfusepath{fill}%
\end{pgfscope}%
\begin{pgfscope}%
\pgfpathrectangle{\pgfqpoint{0.150000in}{0.150000in}}{\pgfqpoint{2.700000in}{1.950000in}}%
\pgfusepath{clip}%
\pgfsetbuttcap%
\pgfsetroundjoin%
\definecolor{currentfill}{rgb}{0.524219,0.582812,0.664844}%
\pgfsetfillcolor{currentfill}%
\pgfsetlinewidth{0.000000pt}%
\definecolor{currentstroke}{rgb}{0.000000,0.000000,0.000000}%
\pgfsetstrokecolor{currentstroke}%
\pgfsetdash{}{0pt}%
\pgfpathmoveto{\pgfqpoint{1.304896in}{1.333992in}}%
\pgfpathlineto{\pgfqpoint{1.343855in}{1.353326in}}%
\pgfpathlineto{\pgfqpoint{1.305759in}{1.372588in}}%
\pgfpathlineto{\pgfqpoint{1.266802in}{1.353326in}}%
\pgfpathclose%
\pgfusepath{fill}%
\end{pgfscope}%
\begin{pgfscope}%
\pgfpathrectangle{\pgfqpoint{0.150000in}{0.150000in}}{\pgfqpoint{2.700000in}{1.950000in}}%
\pgfusepath{clip}%
\pgfsetbuttcap%
\pgfsetroundjoin%
\definecolor{currentfill}{rgb}{0.524219,0.582812,0.664844}%
\pgfsetfillcolor{currentfill}%
\pgfsetlinewidth{0.000000pt}%
\definecolor{currentstroke}{rgb}{0.000000,0.000000,0.000000}%
\pgfsetstrokecolor{currentstroke}%
\pgfsetdash{}{0pt}%
\pgfpathmoveto{\pgfqpoint{1.227700in}{1.333992in}}%
\pgfpathlineto{\pgfqpoint{1.266802in}{1.353326in}}%
\pgfpathlineto{\pgfqpoint{1.228850in}{1.372588in}}%
\pgfpathlineto{\pgfqpoint{1.189750in}{1.353326in}}%
\pgfpathclose%
\pgfusepath{fill}%
\end{pgfscope}%
\begin{pgfscope}%
\pgfpathrectangle{\pgfqpoint{0.150000in}{0.150000in}}{\pgfqpoint{2.700000in}{1.950000in}}%
\pgfusepath{clip}%
\pgfsetbuttcap%
\pgfsetroundjoin%
\definecolor{currentfill}{rgb}{0.524219,0.582812,0.664844}%
\pgfsetfillcolor{currentfill}%
\pgfsetlinewidth{0.000000pt}%
\definecolor{currentstroke}{rgb}{0.000000,0.000000,0.000000}%
\pgfsetstrokecolor{currentstroke}%
\pgfsetdash{}{0pt}%
\pgfpathmoveto{\pgfqpoint{1.150503in}{1.333992in}}%
\pgfpathlineto{\pgfqpoint{1.189750in}{1.353326in}}%
\pgfpathlineto{\pgfqpoint{1.151941in}{1.372588in}}%
\pgfpathlineto{\pgfqpoint{1.112697in}{1.353326in}}%
\pgfpathclose%
\pgfusepath{fill}%
\end{pgfscope}%
\begin{pgfscope}%
\pgfpathrectangle{\pgfqpoint{0.150000in}{0.150000in}}{\pgfqpoint{2.700000in}{1.950000in}}%
\pgfusepath{clip}%
\pgfsetbuttcap%
\pgfsetroundjoin%
\definecolor{currentfill}{rgb}{0.524219,0.582812,0.664844}%
\pgfsetfillcolor{currentfill}%
\pgfsetlinewidth{0.000000pt}%
\definecolor{currentstroke}{rgb}{0.000000,0.000000,0.000000}%
\pgfsetstrokecolor{currentstroke}%
\pgfsetdash{}{0pt}%
\pgfpathmoveto{\pgfqpoint{1.073306in}{1.333992in}}%
\pgfpathlineto{\pgfqpoint{1.112697in}{1.353326in}}%
\pgfpathlineto{\pgfqpoint{1.075032in}{1.372588in}}%
\pgfpathlineto{\pgfqpoint{1.035644in}{1.353326in}}%
\pgfpathclose%
\pgfusepath{fill}%
\end{pgfscope}%
\begin{pgfscope}%
\pgfpathrectangle{\pgfqpoint{0.150000in}{0.150000in}}{\pgfqpoint{2.700000in}{1.950000in}}%
\pgfusepath{clip}%
\pgfsetbuttcap%
\pgfsetroundjoin%
\definecolor{currentfill}{rgb}{0.959574,0.964553,0.971523}%
\pgfsetfillcolor{currentfill}%
\pgfsetlinewidth{0.000000pt}%
\definecolor{currentstroke}{rgb}{0.000000,0.000000,0.000000}%
\pgfsetstrokecolor{currentstroke}%
\pgfsetdash{}{0pt}%
\pgfpathmoveto{\pgfqpoint{1.382595in}{0.591806in}}%
\pgfpathlineto{\pgfqpoint{1.421283in}{0.612458in}}%
\pgfpathlineto{\pgfqpoint{1.382138in}{0.700298in}}%
\pgfpathlineto{\pgfqpoint{1.343189in}{0.679711in}}%
\pgfpathclose%
\pgfusepath{fill}%
\end{pgfscope}%
\begin{pgfscope}%
\pgfpathrectangle{\pgfqpoint{0.150000in}{0.150000in}}{\pgfqpoint{2.700000in}{1.950000in}}%
\pgfusepath{clip}%
\pgfsetbuttcap%
\pgfsetroundjoin%
\definecolor{currentfill}{rgb}{0.443367,0.511918,0.607889}%
\pgfsetfillcolor{currentfill}%
\pgfsetlinewidth{0.000000pt}%
\definecolor{currentstroke}{rgb}{0.000000,0.000000,0.000000}%
\pgfsetstrokecolor{currentstroke}%
\pgfsetdash{}{0pt}%
\pgfpathmoveto{\pgfqpoint{0.725026in}{1.464236in}}%
\pgfpathlineto{\pgfqpoint{0.766832in}{1.459090in}}%
\pgfpathlineto{\pgfqpoint{0.729853in}{1.478133in}}%
\pgfpathlineto{\pgfqpoint{0.687023in}{1.495518in}}%
\pgfpathclose%
\pgfusepath{fill}%
\end{pgfscope}%
\begin{pgfscope}%
\pgfpathrectangle{\pgfqpoint{0.150000in}{0.150000in}}{\pgfqpoint{2.700000in}{1.950000in}}%
\pgfusepath{clip}%
\pgfsetbuttcap%
\pgfsetroundjoin%
\definecolor{currentfill}{rgb}{0.493122,0.555545,0.642938}%
\pgfsetfillcolor{currentfill}%
\pgfsetlinewidth{0.000000pt}%
\definecolor{currentstroke}{rgb}{0.000000,0.000000,0.000000}%
\pgfsetstrokecolor{currentstroke}%
\pgfsetdash{}{0pt}%
\pgfpathmoveto{\pgfqpoint{0.879337in}{1.389440in}}%
\pgfpathlineto{\pgfqpoint{0.920527in}{1.384598in}}%
\pgfpathlineto{\pgfqpoint{0.883248in}{1.403787in}}%
\pgfpathlineto{\pgfqpoint{0.841982in}{1.408699in}}%
\pgfpathclose%
\pgfusepath{fill}%
\end{pgfscope}%
\begin{pgfscope}%
\pgfpathrectangle{\pgfqpoint{0.150000in}{0.150000in}}{\pgfqpoint{2.700000in}{1.950000in}}%
\pgfusepath{clip}%
\pgfsetbuttcap%
\pgfsetroundjoin%
\definecolor{currentfill}{rgb}{0.517999,0.577359,0.660463}%
\pgfsetfillcolor{currentfill}%
\pgfsetlinewidth{0.000000pt}%
\definecolor{currentstroke}{rgb}{0.000000,0.000000,0.000000}%
\pgfsetstrokecolor{currentstroke}%
\pgfsetdash{}{0pt}%
\pgfpathmoveto{\pgfqpoint{0.995504in}{1.346004in}}%
\pgfpathlineto{\pgfqpoint{1.035644in}{1.353326in}}%
\pgfpathlineto{\pgfqpoint{0.998123in}{1.372588in}}%
\pgfpathlineto{\pgfqpoint{0.957946in}{1.365337in}}%
\pgfpathclose%
\pgfusepath{fill}%
\end{pgfscope}%
\begin{pgfscope}%
\pgfpathrectangle{\pgfqpoint{0.150000in}{0.150000in}}{\pgfqpoint{2.700000in}{1.950000in}}%
\pgfusepath{clip}%
\pgfsetbuttcap%
\pgfsetroundjoin%
\definecolor{currentfill}{rgb}{0.586412,0.637347,0.708655}%
\pgfsetfillcolor{currentfill}%
\pgfsetlinewidth{0.000000pt}%
\definecolor{currentstroke}{rgb}{0.000000,0.000000,0.000000}%
\pgfsetstrokecolor{currentstroke}%
\pgfsetdash{}{0pt}%
\pgfpathmoveto{\pgfqpoint{1.574999in}{1.192208in}}%
\pgfpathlineto{\pgfqpoint{1.613369in}{1.211764in}}%
\pgfpathlineto{\pgfqpoint{1.575071in}{1.290641in}}%
\pgfpathlineto{\pgfqpoint{1.536486in}{1.271161in}}%
\pgfpathclose%
\pgfusepath{fill}%
\end{pgfscope}%
\begin{pgfscope}%
\pgfpathrectangle{\pgfqpoint{0.150000in}{0.150000in}}{\pgfqpoint{2.700000in}{1.950000in}}%
\pgfusepath{clip}%
\pgfsetbuttcap%
\pgfsetroundjoin%
\definecolor{currentfill}{rgb}{0.661045,0.702788,0.761229}%
\pgfsetfillcolor{currentfill}%
\pgfsetlinewidth{0.000000pt}%
\definecolor{currentstroke}{rgb}{0.000000,0.000000,0.000000}%
\pgfsetstrokecolor{currentstroke}%
\pgfsetdash{}{0pt}%
\pgfpathmoveto{\pgfqpoint{1.536486in}{1.074364in}}%
\pgfpathlineto{\pgfqpoint{1.574928in}{1.094139in}}%
\pgfpathlineto{\pgfqpoint{1.536486in}{1.172580in}}%
\pgfpathlineto{\pgfqpoint{1.497829in}{1.152878in}}%
\pgfpathclose%
\pgfusepath{fill}%
\end{pgfscope}%
\begin{pgfscope}%
\pgfpathrectangle{\pgfqpoint{0.150000in}{0.150000in}}{\pgfqpoint{2.700000in}{1.950000in}}%
\pgfusepath{clip}%
\pgfsetbuttcap%
\pgfsetroundjoin%
\definecolor{currentfill}{rgb}{0.729458,0.762776,0.809421}%
\pgfsetfillcolor{currentfill}%
\pgfsetlinewidth{0.000000pt}%
\definecolor{currentstroke}{rgb}{0.000000,0.000000,0.000000}%
\pgfsetstrokecolor{currentstroke}%
\pgfsetdash{}{0pt}%
\pgfpathmoveto{\pgfqpoint{1.497972in}{0.956516in}}%
\pgfpathlineto{\pgfqpoint{1.536486in}{0.976511in}}%
\pgfpathlineto{\pgfqpoint{1.497901in}{1.054515in}}%
\pgfpathlineto{\pgfqpoint{1.459171in}{1.034592in}}%
\pgfpathclose%
\pgfusepath{fill}%
\end{pgfscope}%
\begin{pgfscope}%
\pgfpathrectangle{\pgfqpoint{0.150000in}{0.150000in}}{\pgfqpoint{2.700000in}{1.950000in}}%
\pgfusepath{clip}%
\pgfsetbuttcap%
\pgfsetroundjoin%
\definecolor{currentfill}{rgb}{0.804090,0.828217,0.861994}%
\pgfsetfillcolor{currentfill}%
\pgfsetlinewidth{0.000000pt}%
\definecolor{currentstroke}{rgb}{0.000000,0.000000,0.000000}%
\pgfsetstrokecolor{currentstroke}%
\pgfsetdash{}{0pt}%
\pgfpathmoveto{\pgfqpoint{1.459457in}{0.838665in}}%
\pgfpathlineto{\pgfqpoint{1.498044in}{0.858879in}}%
\pgfpathlineto{\pgfqpoint{1.459315in}{0.936447in}}%
\pgfpathlineto{\pgfqpoint{1.420512in}{0.916302in}}%
\pgfpathclose%
\pgfusepath{fill}%
\end{pgfscope}%
\begin{pgfscope}%
\pgfpathrectangle{\pgfqpoint{0.150000in}{0.150000in}}{\pgfqpoint{2.700000in}{1.950000in}}%
\pgfusepath{clip}%
\pgfsetbuttcap%
\pgfsetroundjoin%
\definecolor{currentfill}{rgb}{0.878722,0.893658,0.914568}%
\pgfsetfillcolor{currentfill}%
\pgfsetlinewidth{0.000000pt}%
\definecolor{currentstroke}{rgb}{0.000000,0.000000,0.000000}%
\pgfsetstrokecolor{currentstroke}%
\pgfsetdash{}{0pt}%
\pgfpathmoveto{\pgfqpoint{1.420941in}{0.720809in}}%
\pgfpathlineto{\pgfqpoint{1.459600in}{0.741244in}}%
\pgfpathlineto{\pgfqpoint{1.420727in}{0.818374in}}%
\pgfpathlineto{\pgfqpoint{1.381851in}{0.798008in}}%
\pgfpathclose%
\pgfusepath{fill}%
\end{pgfscope}%
\begin{pgfscope}%
\pgfpathrectangle{\pgfqpoint{0.150000in}{0.150000in}}{\pgfqpoint{2.700000in}{1.950000in}}%
\pgfusepath{clip}%
\pgfsetbuttcap%
\pgfsetroundjoin%
\definecolor{currentfill}{rgb}{0.536657,0.593719,0.673606}%
\pgfsetfillcolor{currentfill}%
\pgfsetlinewidth{0.000000pt}%
\definecolor{currentstroke}{rgb}{0.000000,0.000000,0.000000}%
\pgfsetstrokecolor{currentstroke}%
\pgfsetdash{}{0pt}%
\pgfpathmoveto{\pgfqpoint{1.575071in}{1.290641in}}%
\pgfpathlineto{\pgfqpoint{1.613511in}{1.310049in}}%
\pgfpathlineto{\pgfqpoint{1.575013in}{1.353326in}}%
\pgfpathlineto{\pgfqpoint{1.536486in}{1.333992in}}%
\pgfpathclose%
\pgfusepath{fill}%
\end{pgfscope}%
\begin{pgfscope}%
\pgfpathrectangle{\pgfqpoint{0.150000in}{0.150000in}}{\pgfqpoint{2.700000in}{1.950000in}}%
\pgfusepath{clip}%
\pgfsetbuttcap%
\pgfsetroundjoin%
\definecolor{currentfill}{rgb}{0.449586,0.517371,0.612270}%
\pgfsetfillcolor{currentfill}%
\pgfsetlinewidth{0.000000pt}%
\definecolor{currentstroke}{rgb}{0.000000,0.000000,0.000000}%
\pgfsetstrokecolor{currentstroke}%
\pgfsetdash{}{0pt}%
\pgfpathmoveto{\pgfqpoint{0.762217in}{1.445052in}}%
\pgfpathlineto{\pgfqpoint{0.803949in}{1.439976in}}%
\pgfpathlineto{\pgfqpoint{0.766832in}{1.459090in}}%
\pgfpathlineto{\pgfqpoint{0.725026in}{1.464236in}}%
\pgfpathclose%
\pgfusepath{fill}%
\end{pgfscope}%
\begin{pgfscope}%
\pgfpathrectangle{\pgfqpoint{0.150000in}{0.150000in}}{\pgfqpoint{2.700000in}{1.950000in}}%
\pgfusepath{clip}%
\pgfsetbuttcap%
\pgfsetroundjoin%
\definecolor{currentfill}{rgb}{0.524219,0.582812,0.664844}%
\pgfsetfillcolor{currentfill}%
\pgfsetlinewidth{0.000000pt}%
\definecolor{currentstroke}{rgb}{0.000000,0.000000,0.000000}%
\pgfsetstrokecolor{currentstroke}%
\pgfsetdash{}{0pt}%
\pgfpathmoveto{\pgfqpoint{1.497816in}{1.314586in}}%
\pgfpathlineto{\pgfqpoint{1.536486in}{1.333992in}}%
\pgfpathlineto{\pgfqpoint{1.497960in}{1.353326in}}%
\pgfpathlineto{\pgfqpoint{1.459290in}{1.333992in}}%
\pgfpathclose%
\pgfusepath{fill}%
\end{pgfscope}%
\begin{pgfscope}%
\pgfpathrectangle{\pgfqpoint{0.150000in}{0.150000in}}{\pgfqpoint{2.700000in}{1.950000in}}%
\pgfusepath{clip}%
\pgfsetbuttcap%
\pgfsetroundjoin%
\definecolor{currentfill}{rgb}{0.524219,0.582812,0.664844}%
\pgfsetfillcolor{currentfill}%
\pgfsetlinewidth{0.000000pt}%
\definecolor{currentstroke}{rgb}{0.000000,0.000000,0.000000}%
\pgfsetstrokecolor{currentstroke}%
\pgfsetdash{}{0pt}%
\pgfpathmoveto{\pgfqpoint{1.420474in}{1.314586in}}%
\pgfpathlineto{\pgfqpoint{1.459290in}{1.333992in}}%
\pgfpathlineto{\pgfqpoint{1.420908in}{1.353326in}}%
\pgfpathlineto{\pgfqpoint{1.382093in}{1.333992in}}%
\pgfpathclose%
\pgfusepath{fill}%
\end{pgfscope}%
\begin{pgfscope}%
\pgfpathrectangle{\pgfqpoint{0.150000in}{0.150000in}}{\pgfqpoint{2.700000in}{1.950000in}}%
\pgfusepath{clip}%
\pgfsetbuttcap%
\pgfsetroundjoin%
\definecolor{currentfill}{rgb}{0.524219,0.582812,0.664844}%
\pgfsetfillcolor{currentfill}%
\pgfsetlinewidth{0.000000pt}%
\definecolor{currentstroke}{rgb}{0.000000,0.000000,0.000000}%
\pgfsetstrokecolor{currentstroke}%
\pgfsetdash{}{0pt}%
\pgfpathmoveto{\pgfqpoint{1.343133in}{1.314586in}}%
\pgfpathlineto{\pgfqpoint{1.382093in}{1.333992in}}%
\pgfpathlineto{\pgfqpoint{1.343855in}{1.353326in}}%
\pgfpathlineto{\pgfqpoint{1.304896in}{1.333992in}}%
\pgfpathclose%
\pgfusepath{fill}%
\end{pgfscope}%
\begin{pgfscope}%
\pgfpathrectangle{\pgfqpoint{0.150000in}{0.150000in}}{\pgfqpoint{2.700000in}{1.950000in}}%
\pgfusepath{clip}%
\pgfsetbuttcap%
\pgfsetroundjoin%
\definecolor{currentfill}{rgb}{0.524219,0.582812,0.664844}%
\pgfsetfillcolor{currentfill}%
\pgfsetlinewidth{0.000000pt}%
\definecolor{currentstroke}{rgb}{0.000000,0.000000,0.000000}%
\pgfsetstrokecolor{currentstroke}%
\pgfsetdash{}{0pt}%
\pgfpathmoveto{\pgfqpoint{1.265792in}{1.314586in}}%
\pgfpathlineto{\pgfqpoint{1.304896in}{1.333992in}}%
\pgfpathlineto{\pgfqpoint{1.266802in}{1.353326in}}%
\pgfpathlineto{\pgfqpoint{1.227700in}{1.333992in}}%
\pgfpathclose%
\pgfusepath{fill}%
\end{pgfscope}%
\begin{pgfscope}%
\pgfpathrectangle{\pgfqpoint{0.150000in}{0.150000in}}{\pgfqpoint{2.700000in}{1.950000in}}%
\pgfusepath{clip}%
\pgfsetbuttcap%
\pgfsetroundjoin%
\definecolor{currentfill}{rgb}{0.524219,0.582812,0.664844}%
\pgfsetfillcolor{currentfill}%
\pgfsetlinewidth{0.000000pt}%
\definecolor{currentstroke}{rgb}{0.000000,0.000000,0.000000}%
\pgfsetstrokecolor{currentstroke}%
\pgfsetdash{}{0pt}%
\pgfpathmoveto{\pgfqpoint{1.188450in}{1.314586in}}%
\pgfpathlineto{\pgfqpoint{1.227700in}{1.333992in}}%
\pgfpathlineto{\pgfqpoint{1.189750in}{1.353326in}}%
\pgfpathlineto{\pgfqpoint{1.150503in}{1.333992in}}%
\pgfpathclose%
\pgfusepath{fill}%
\end{pgfscope}%
\begin{pgfscope}%
\pgfpathrectangle{\pgfqpoint{0.150000in}{0.150000in}}{\pgfqpoint{2.700000in}{1.950000in}}%
\pgfusepath{clip}%
\pgfsetbuttcap%
\pgfsetroundjoin%
\definecolor{currentfill}{rgb}{0.524219,0.582812,0.664844}%
\pgfsetfillcolor{currentfill}%
\pgfsetlinewidth{0.000000pt}%
\definecolor{currentstroke}{rgb}{0.000000,0.000000,0.000000}%
\pgfsetstrokecolor{currentstroke}%
\pgfsetdash{}{0pt}%
\pgfpathmoveto{\pgfqpoint{1.111109in}{1.314586in}}%
\pgfpathlineto{\pgfqpoint{1.150503in}{1.333992in}}%
\pgfpathlineto{\pgfqpoint{1.112697in}{1.353326in}}%
\pgfpathlineto{\pgfqpoint{1.073306in}{1.333992in}}%
\pgfpathclose%
\pgfusepath{fill}%
\end{pgfscope}%
\begin{pgfscope}%
\pgfpathrectangle{\pgfqpoint{0.150000in}{0.150000in}}{\pgfqpoint{2.700000in}{1.950000in}}%
\pgfusepath{clip}%
\pgfsetbuttcap%
\pgfsetroundjoin%
\definecolor{currentfill}{rgb}{0.493122,0.555545,0.642938}%
\pgfsetfillcolor{currentfill}%
\pgfsetlinewidth{0.000000pt}%
\definecolor{currentstroke}{rgb}{0.000000,0.000000,0.000000}%
\pgfsetstrokecolor{currentstroke}%
\pgfsetdash{}{0pt}%
\pgfpathmoveto{\pgfqpoint{0.916833in}{1.370108in}}%
\pgfpathlineto{\pgfqpoint{0.957946in}{1.365337in}}%
\pgfpathlineto{\pgfqpoint{0.920527in}{1.384598in}}%
\pgfpathlineto{\pgfqpoint{0.879337in}{1.389440in}}%
\pgfpathclose%
\pgfusepath{fill}%
\end{pgfscope}%
\begin{pgfscope}%
\pgfpathrectangle{\pgfqpoint{0.150000in}{0.150000in}}{\pgfqpoint{2.700000in}{1.950000in}}%
\pgfusepath{clip}%
\pgfsetbuttcap%
\pgfsetroundjoin%
\definecolor{currentfill}{rgb}{0.517999,0.577359,0.660463}%
\pgfsetfillcolor{currentfill}%
\pgfsetlinewidth{0.000000pt}%
\definecolor{currentstroke}{rgb}{0.000000,0.000000,0.000000}%
\pgfsetstrokecolor{currentstroke}%
\pgfsetdash{}{0pt}%
\pgfpathmoveto{\pgfqpoint{1.033203in}{1.326598in}}%
\pgfpathlineto{\pgfqpoint{1.073306in}{1.333992in}}%
\pgfpathlineto{\pgfqpoint{1.035644in}{1.353326in}}%
\pgfpathlineto{\pgfqpoint{0.995504in}{1.346004in}}%
\pgfpathclose%
\pgfusepath{fill}%
\end{pgfscope}%
\begin{pgfscope}%
\pgfpathrectangle{\pgfqpoint{0.150000in}{0.150000in}}{\pgfqpoint{2.700000in}{1.950000in}}%
\pgfusepath{clip}%
\pgfsetbuttcap%
\pgfsetroundjoin%
\definecolor{currentfill}{rgb}{0.586412,0.637347,0.708655}%
\pgfsetfillcolor{currentfill}%
\pgfsetlinewidth{0.000000pt}%
\definecolor{currentstroke}{rgb}{0.000000,0.000000,0.000000}%
\pgfsetstrokecolor{currentstroke}%
\pgfsetdash{}{0pt}%
\pgfpathmoveto{\pgfqpoint{1.536486in}{1.172580in}}%
\pgfpathlineto{\pgfqpoint{1.574999in}{1.192208in}}%
\pgfpathlineto{\pgfqpoint{1.536486in}{1.271161in}}%
\pgfpathlineto{\pgfqpoint{1.497758in}{1.251607in}}%
\pgfpathclose%
\pgfusepath{fill}%
\end{pgfscope}%
\begin{pgfscope}%
\pgfpathrectangle{\pgfqpoint{0.150000in}{0.150000in}}{\pgfqpoint{2.700000in}{1.950000in}}%
\pgfusepath{clip}%
\pgfsetbuttcap%
\pgfsetroundjoin%
\definecolor{currentfill}{rgb}{0.661045,0.702788,0.761229}%
\pgfsetfillcolor{currentfill}%
\pgfsetlinewidth{0.000000pt}%
\definecolor{currentstroke}{rgb}{0.000000,0.000000,0.000000}%
\pgfsetstrokecolor{currentstroke}%
\pgfsetdash{}{0pt}%
\pgfpathmoveto{\pgfqpoint{1.497901in}{1.054515in}}%
\pgfpathlineto{\pgfqpoint{1.536486in}{1.074364in}}%
\pgfpathlineto{\pgfqpoint{1.497829in}{1.152878in}}%
\pgfpathlineto{\pgfqpoint{1.459027in}{1.133102in}}%
\pgfpathclose%
\pgfusepath{fill}%
\end{pgfscope}%
\begin{pgfscope}%
\pgfpathrectangle{\pgfqpoint{0.150000in}{0.150000in}}{\pgfqpoint{2.700000in}{1.950000in}}%
\pgfusepath{clip}%
\pgfsetbuttcap%
\pgfsetroundjoin%
\definecolor{currentfill}{rgb}{0.729458,0.762776,0.809421}%
\pgfsetfillcolor{currentfill}%
\pgfsetlinewidth{0.000000pt}%
\definecolor{currentstroke}{rgb}{0.000000,0.000000,0.000000}%
\pgfsetstrokecolor{currentstroke}%
\pgfsetdash{}{0pt}%
\pgfpathmoveto{\pgfqpoint{1.459315in}{0.936447in}}%
\pgfpathlineto{\pgfqpoint{1.497972in}{0.956516in}}%
\pgfpathlineto{\pgfqpoint{1.459171in}{1.034592in}}%
\pgfpathlineto{\pgfqpoint{1.420296in}{1.014594in}}%
\pgfpathclose%
\pgfusepath{fill}%
\end{pgfscope}%
\begin{pgfscope}%
\pgfpathrectangle{\pgfqpoint{0.150000in}{0.150000in}}{\pgfqpoint{2.700000in}{1.950000in}}%
\pgfusepath{clip}%
\pgfsetbuttcap%
\pgfsetroundjoin%
\definecolor{currentfill}{rgb}{0.804090,0.828217,0.861994}%
\pgfsetfillcolor{currentfill}%
\pgfsetlinewidth{0.000000pt}%
\definecolor{currentstroke}{rgb}{0.000000,0.000000,0.000000}%
\pgfsetstrokecolor{currentstroke}%
\pgfsetdash{}{0pt}%
\pgfpathmoveto{\pgfqpoint{1.420727in}{0.818374in}}%
\pgfpathlineto{\pgfqpoint{1.459457in}{0.838665in}}%
\pgfpathlineto{\pgfqpoint{1.420512in}{0.916302in}}%
\pgfpathlineto{\pgfqpoint{1.381563in}{0.896081in}}%
\pgfpathclose%
\pgfusepath{fill}%
\end{pgfscope}%
\begin{pgfscope}%
\pgfpathrectangle{\pgfqpoint{0.150000in}{0.150000in}}{\pgfqpoint{2.700000in}{1.950000in}}%
\pgfusepath{clip}%
\pgfsetbuttcap%
\pgfsetroundjoin%
\definecolor{currentfill}{rgb}{0.878722,0.893658,0.914568}%
\pgfsetfillcolor{currentfill}%
\pgfsetlinewidth{0.000000pt}%
\definecolor{currentstroke}{rgb}{0.000000,0.000000,0.000000}%
\pgfsetstrokecolor{currentstroke}%
\pgfsetdash{}{0pt}%
\pgfpathmoveto{\pgfqpoint{1.382138in}{0.700298in}}%
\pgfpathlineto{\pgfqpoint{1.420941in}{0.720809in}}%
\pgfpathlineto{\pgfqpoint{1.381851in}{0.798008in}}%
\pgfpathlineto{\pgfqpoint{1.342829in}{0.777565in}}%
\pgfpathclose%
\pgfusepath{fill}%
\end{pgfscope}%
\begin{pgfscope}%
\pgfpathrectangle{\pgfqpoint{0.150000in}{0.150000in}}{\pgfqpoint{2.700000in}{1.950000in}}%
\pgfusepath{clip}%
\pgfsetbuttcap%
\pgfsetroundjoin%
\definecolor{currentfill}{rgb}{0.536657,0.593719,0.673606}%
\pgfsetfillcolor{currentfill}%
\pgfsetlinewidth{0.000000pt}%
\definecolor{currentstroke}{rgb}{0.000000,0.000000,0.000000}%
\pgfsetstrokecolor{currentstroke}%
\pgfsetdash{}{0pt}%
\pgfpathmoveto{\pgfqpoint{1.536486in}{1.271161in}}%
\pgfpathlineto{\pgfqpoint{1.575071in}{1.290641in}}%
\pgfpathlineto{\pgfqpoint{1.536486in}{1.333992in}}%
\pgfpathlineto{\pgfqpoint{1.497816in}{1.314586in}}%
\pgfpathclose%
\pgfusepath{fill}%
\end{pgfscope}%
\begin{pgfscope}%
\pgfpathrectangle{\pgfqpoint{0.150000in}{0.150000in}}{\pgfqpoint{2.700000in}{1.950000in}}%
\pgfusepath{clip}%
\pgfsetbuttcap%
\pgfsetroundjoin%
\definecolor{currentfill}{rgb}{0.455806,0.522825,0.616651}%
\pgfsetfillcolor{currentfill}%
\pgfsetlinewidth{0.000000pt}%
\definecolor{currentstroke}{rgb}{0.000000,0.000000,0.000000}%
\pgfsetstrokecolor{currentstroke}%
\pgfsetdash{}{0pt}%
\pgfpathmoveto{\pgfqpoint{0.800376in}{1.413650in}}%
\pgfpathlineto{\pgfqpoint{0.841982in}{1.408699in}}%
\pgfpathlineto{\pgfqpoint{0.803949in}{1.439976in}}%
\pgfpathlineto{\pgfqpoint{0.762217in}{1.445052in}}%
\pgfpathclose%
\pgfusepath{fill}%
\end{pgfscope}%
\begin{pgfscope}%
\pgfpathrectangle{\pgfqpoint{0.150000in}{0.150000in}}{\pgfqpoint{2.700000in}{1.950000in}}%
\pgfusepath{clip}%
\pgfsetbuttcap%
\pgfsetroundjoin%
\definecolor{currentfill}{rgb}{0.406051,0.479197,0.581602}%
\pgfsetfillcolor{currentfill}%
\pgfsetlinewidth{0.000000pt}%
\definecolor{currentstroke}{rgb}{0.000000,0.000000,0.000000}%
\pgfsetstrokecolor{currentstroke}%
\pgfsetdash{}{0pt}%
\pgfpathmoveto{\pgfqpoint{0.644747in}{1.500831in}}%
\pgfpathlineto{\pgfqpoint{0.687023in}{1.495518in}}%
\pgfpathlineto{\pgfqpoint{0.650070in}{1.514557in}}%
\pgfpathlineto{\pgfqpoint{0.607720in}{1.519939in}}%
\pgfpathclose%
\pgfusepath{fill}%
\end{pgfscope}%
\begin{pgfscope}%
\pgfpathrectangle{\pgfqpoint{0.150000in}{0.150000in}}{\pgfqpoint{2.700000in}{1.950000in}}%
\pgfusepath{clip}%
\pgfsetbuttcap%
\pgfsetroundjoin%
\definecolor{currentfill}{rgb}{0.524219,0.582812,0.664844}%
\pgfsetfillcolor{currentfill}%
\pgfsetlinewidth{0.000000pt}%
\definecolor{currentstroke}{rgb}{0.000000,0.000000,0.000000}%
\pgfsetstrokecolor{currentstroke}%
\pgfsetdash{}{0pt}%
\pgfpathmoveto{\pgfqpoint{1.459000in}{1.295106in}}%
\pgfpathlineto{\pgfqpoint{1.497816in}{1.314586in}}%
\pgfpathlineto{\pgfqpoint{1.459290in}{1.333992in}}%
\pgfpathlineto{\pgfqpoint{1.420474in}{1.314586in}}%
\pgfpathclose%
\pgfusepath{fill}%
\end{pgfscope}%
\begin{pgfscope}%
\pgfpathrectangle{\pgfqpoint{0.150000in}{0.150000in}}{\pgfqpoint{2.700000in}{1.950000in}}%
\pgfusepath{clip}%
\pgfsetbuttcap%
\pgfsetroundjoin%
\definecolor{currentfill}{rgb}{0.524219,0.582812,0.664844}%
\pgfsetfillcolor{currentfill}%
\pgfsetlinewidth{0.000000pt}%
\definecolor{currentstroke}{rgb}{0.000000,0.000000,0.000000}%
\pgfsetstrokecolor{currentstroke}%
\pgfsetdash{}{0pt}%
\pgfpathmoveto{\pgfqpoint{1.381513in}{1.295106in}}%
\pgfpathlineto{\pgfqpoint{1.420474in}{1.314586in}}%
\pgfpathlineto{\pgfqpoint{1.382093in}{1.333992in}}%
\pgfpathlineto{\pgfqpoint{1.343133in}{1.314586in}}%
\pgfpathclose%
\pgfusepath{fill}%
\end{pgfscope}%
\begin{pgfscope}%
\pgfpathrectangle{\pgfqpoint{0.150000in}{0.150000in}}{\pgfqpoint{2.700000in}{1.950000in}}%
\pgfusepath{clip}%
\pgfsetbuttcap%
\pgfsetroundjoin%
\definecolor{currentfill}{rgb}{0.524219,0.582812,0.664844}%
\pgfsetfillcolor{currentfill}%
\pgfsetlinewidth{0.000000pt}%
\definecolor{currentstroke}{rgb}{0.000000,0.000000,0.000000}%
\pgfsetstrokecolor{currentstroke}%
\pgfsetdash{}{0pt}%
\pgfpathmoveto{\pgfqpoint{1.304027in}{1.295106in}}%
\pgfpathlineto{\pgfqpoint{1.343133in}{1.314586in}}%
\pgfpathlineto{\pgfqpoint{1.304896in}{1.333992in}}%
\pgfpathlineto{\pgfqpoint{1.265792in}{1.314586in}}%
\pgfpathclose%
\pgfusepath{fill}%
\end{pgfscope}%
\begin{pgfscope}%
\pgfpathrectangle{\pgfqpoint{0.150000in}{0.150000in}}{\pgfqpoint{2.700000in}{1.950000in}}%
\pgfusepath{clip}%
\pgfsetbuttcap%
\pgfsetroundjoin%
\definecolor{currentfill}{rgb}{0.524219,0.582812,0.664844}%
\pgfsetfillcolor{currentfill}%
\pgfsetlinewidth{0.000000pt}%
\definecolor{currentstroke}{rgb}{0.000000,0.000000,0.000000}%
\pgfsetstrokecolor{currentstroke}%
\pgfsetdash{}{0pt}%
\pgfpathmoveto{\pgfqpoint{1.226540in}{1.295106in}}%
\pgfpathlineto{\pgfqpoint{1.265792in}{1.314586in}}%
\pgfpathlineto{\pgfqpoint{1.227700in}{1.333992in}}%
\pgfpathlineto{\pgfqpoint{1.188450in}{1.314586in}}%
\pgfpathclose%
\pgfusepath{fill}%
\end{pgfscope}%
\begin{pgfscope}%
\pgfpathrectangle{\pgfqpoint{0.150000in}{0.150000in}}{\pgfqpoint{2.700000in}{1.950000in}}%
\pgfusepath{clip}%
\pgfsetbuttcap%
\pgfsetroundjoin%
\definecolor{currentfill}{rgb}{0.524219,0.582812,0.664844}%
\pgfsetfillcolor{currentfill}%
\pgfsetlinewidth{0.000000pt}%
\definecolor{currentstroke}{rgb}{0.000000,0.000000,0.000000}%
\pgfsetstrokecolor{currentstroke}%
\pgfsetdash{}{0pt}%
\pgfpathmoveto{\pgfqpoint{1.149054in}{1.295106in}}%
\pgfpathlineto{\pgfqpoint{1.188450in}{1.314586in}}%
\pgfpathlineto{\pgfqpoint{1.150503in}{1.333992in}}%
\pgfpathlineto{\pgfqpoint{1.111109in}{1.314586in}}%
\pgfpathclose%
\pgfusepath{fill}%
\end{pgfscope}%
\begin{pgfscope}%
\pgfpathrectangle{\pgfqpoint{0.150000in}{0.150000in}}{\pgfqpoint{2.700000in}{1.950000in}}%
\pgfusepath{clip}%
\pgfsetbuttcap%
\pgfsetroundjoin%
\definecolor{currentfill}{rgb}{0.499341,0.560999,0.647319}%
\pgfsetfillcolor{currentfill}%
\pgfsetlinewidth{0.000000pt}%
\definecolor{currentstroke}{rgb}{0.000000,0.000000,0.000000}%
\pgfsetstrokecolor{currentstroke}%
\pgfsetdash{}{0pt}%
\pgfpathmoveto{\pgfqpoint{0.955123in}{1.338638in}}%
\pgfpathlineto{\pgfqpoint{0.995504in}{1.346004in}}%
\pgfpathlineto{\pgfqpoint{0.957946in}{1.365337in}}%
\pgfpathlineto{\pgfqpoint{0.916833in}{1.370108in}}%
\pgfpathclose%
\pgfusepath{fill}%
\end{pgfscope}%
\begin{pgfscope}%
\pgfpathrectangle{\pgfqpoint{0.150000in}{0.150000in}}{\pgfqpoint{2.700000in}{1.950000in}}%
\pgfusepath{clip}%
\pgfsetbuttcap%
\pgfsetroundjoin%
\definecolor{currentfill}{rgb}{0.517999,0.577359,0.660463}%
\pgfsetfillcolor{currentfill}%
\pgfsetlinewidth{0.000000pt}%
\definecolor{currentstroke}{rgb}{0.000000,0.000000,0.000000}%
\pgfsetstrokecolor{currentstroke}%
\pgfsetdash{}{0pt}%
\pgfpathmoveto{\pgfqpoint{1.071044in}{1.307119in}}%
\pgfpathlineto{\pgfqpoint{1.111109in}{1.314586in}}%
\pgfpathlineto{\pgfqpoint{1.073306in}{1.333992in}}%
\pgfpathlineto{\pgfqpoint{1.033203in}{1.326598in}}%
\pgfpathclose%
\pgfusepath{fill}%
\end{pgfscope}%
\begin{pgfscope}%
\pgfpathrectangle{\pgfqpoint{0.150000in}{0.150000in}}{\pgfqpoint{2.700000in}{1.950000in}}%
\pgfusepath{clip}%
\pgfsetbuttcap%
\pgfsetroundjoin%
\definecolor{currentfill}{rgb}{0.586412,0.637347,0.708655}%
\pgfsetfillcolor{currentfill}%
\pgfsetlinewidth{0.000000pt}%
\definecolor{currentstroke}{rgb}{0.000000,0.000000,0.000000}%
\pgfsetstrokecolor{currentstroke}%
\pgfsetdash{}{0pt}%
\pgfpathmoveto{\pgfqpoint{1.497829in}{1.152878in}}%
\pgfpathlineto{\pgfqpoint{1.536486in}{1.172580in}}%
\pgfpathlineto{\pgfqpoint{1.497758in}{1.251607in}}%
\pgfpathlineto{\pgfqpoint{1.458883in}{1.231980in}}%
\pgfpathclose%
\pgfusepath{fill}%
\end{pgfscope}%
\begin{pgfscope}%
\pgfpathrectangle{\pgfqpoint{0.150000in}{0.150000in}}{\pgfqpoint{2.700000in}{1.950000in}}%
\pgfusepath{clip}%
\pgfsetbuttcap%
\pgfsetroundjoin%
\definecolor{currentfill}{rgb}{0.661045,0.702788,0.761229}%
\pgfsetfillcolor{currentfill}%
\pgfsetlinewidth{0.000000pt}%
\definecolor{currentstroke}{rgb}{0.000000,0.000000,0.000000}%
\pgfsetstrokecolor{currentstroke}%
\pgfsetdash{}{0pt}%
\pgfpathmoveto{\pgfqpoint{1.459171in}{1.034592in}}%
\pgfpathlineto{\pgfqpoint{1.497901in}{1.054515in}}%
\pgfpathlineto{\pgfqpoint{1.459027in}{1.133102in}}%
\pgfpathlineto{\pgfqpoint{1.420079in}{1.113252in}}%
\pgfpathclose%
\pgfusepath{fill}%
\end{pgfscope}%
\begin{pgfscope}%
\pgfpathrectangle{\pgfqpoint{0.150000in}{0.150000in}}{\pgfqpoint{2.700000in}{1.950000in}}%
\pgfusepath{clip}%
\pgfsetbuttcap%
\pgfsetroundjoin%
\definecolor{currentfill}{rgb}{0.729458,0.762776,0.809421}%
\pgfsetfillcolor{currentfill}%
\pgfsetlinewidth{0.000000pt}%
\definecolor{currentstroke}{rgb}{0.000000,0.000000,0.000000}%
\pgfsetstrokecolor{currentstroke}%
\pgfsetdash{}{0pt}%
\pgfpathmoveto{\pgfqpoint{1.420512in}{0.916302in}}%
\pgfpathlineto{\pgfqpoint{1.459315in}{0.936447in}}%
\pgfpathlineto{\pgfqpoint{1.420296in}{1.014594in}}%
\pgfpathlineto{\pgfqpoint{1.381274in}{0.994520in}}%
\pgfpathclose%
\pgfusepath{fill}%
\end{pgfscope}%
\begin{pgfscope}%
\pgfpathrectangle{\pgfqpoint{0.150000in}{0.150000in}}{\pgfqpoint{2.700000in}{1.950000in}}%
\pgfusepath{clip}%
\pgfsetbuttcap%
\pgfsetroundjoin%
\definecolor{currentfill}{rgb}{0.804090,0.828217,0.861994}%
\pgfsetfillcolor{currentfill}%
\pgfsetlinewidth{0.000000pt}%
\definecolor{currentstroke}{rgb}{0.000000,0.000000,0.000000}%
\pgfsetstrokecolor{currentstroke}%
\pgfsetdash{}{0pt}%
\pgfpathmoveto{\pgfqpoint{1.381851in}{0.798008in}}%
\pgfpathlineto{\pgfqpoint{1.420727in}{0.818374in}}%
\pgfpathlineto{\pgfqpoint{1.381563in}{0.896081in}}%
\pgfpathlineto{\pgfqpoint{1.342468in}{0.875785in}}%
\pgfpathclose%
\pgfusepath{fill}%
\end{pgfscope}%
\begin{pgfscope}%
\pgfpathrectangle{\pgfqpoint{0.150000in}{0.150000in}}{\pgfqpoint{2.700000in}{1.950000in}}%
\pgfusepath{clip}%
\pgfsetbuttcap%
\pgfsetroundjoin%
\definecolor{currentfill}{rgb}{0.878722,0.893658,0.914568}%
\pgfsetfillcolor{currentfill}%
\pgfsetlinewidth{0.000000pt}%
\definecolor{currentstroke}{rgb}{0.000000,0.000000,0.000000}%
\pgfsetstrokecolor{currentstroke}%
\pgfsetdash{}{0pt}%
\pgfpathmoveto{\pgfqpoint{1.343189in}{0.679711in}}%
\pgfpathlineto{\pgfqpoint{1.382138in}{0.700298in}}%
\pgfpathlineto{\pgfqpoint{1.342829in}{0.777565in}}%
\pgfpathlineto{\pgfqpoint{1.303661in}{0.757046in}}%
\pgfpathclose%
\pgfusepath{fill}%
\end{pgfscope}%
\begin{pgfscope}%
\pgfpathrectangle{\pgfqpoint{0.150000in}{0.150000in}}{\pgfqpoint{2.700000in}{1.950000in}}%
\pgfusepath{clip}%
\pgfsetbuttcap%
\pgfsetroundjoin%
\definecolor{currentfill}{rgb}{0.462025,0.528278,0.621032}%
\pgfsetfillcolor{currentfill}%
\pgfsetlinewidth{0.000000pt}%
\definecolor{currentstroke}{rgb}{0.000000,0.000000,0.000000}%
\pgfsetstrokecolor{currentstroke}%
\pgfsetdash{}{0pt}%
\pgfpathmoveto{\pgfqpoint{0.837807in}{1.394321in}}%
\pgfpathlineto{\pgfqpoint{0.879337in}{1.389440in}}%
\pgfpathlineto{\pgfqpoint{0.841982in}{1.408699in}}%
\pgfpathlineto{\pgfqpoint{0.800376in}{1.413650in}}%
\pgfpathclose%
\pgfusepath{fill}%
\end{pgfscope}%
\begin{pgfscope}%
\pgfpathrectangle{\pgfqpoint{0.150000in}{0.150000in}}{\pgfqpoint{2.700000in}{1.950000in}}%
\pgfusepath{clip}%
\pgfsetbuttcap%
\pgfsetroundjoin%
\definecolor{currentfill}{rgb}{0.536657,0.593719,0.673606}%
\pgfsetfillcolor{currentfill}%
\pgfsetlinewidth{0.000000pt}%
\definecolor{currentstroke}{rgb}{0.000000,0.000000,0.000000}%
\pgfsetstrokecolor{currentstroke}%
\pgfsetdash{}{0pt}%
\pgfpathmoveto{\pgfqpoint{1.497758in}{1.251607in}}%
\pgfpathlineto{\pgfqpoint{1.536486in}{1.271161in}}%
\pgfpathlineto{\pgfqpoint{1.497816in}{1.314586in}}%
\pgfpathlineto{\pgfqpoint{1.459000in}{1.295106in}}%
\pgfpathclose%
\pgfusepath{fill}%
\end{pgfscope}%
\begin{pgfscope}%
\pgfpathrectangle{\pgfqpoint{0.150000in}{0.150000in}}{\pgfqpoint{2.700000in}{1.950000in}}%
\pgfusepath{clip}%
\pgfsetbuttcap%
\pgfsetroundjoin%
\definecolor{currentfill}{rgb}{0.412270,0.484651,0.585983}%
\pgfsetfillcolor{currentfill}%
\pgfsetlinewidth{0.000000pt}%
\definecolor{currentstroke}{rgb}{0.000000,0.000000,0.000000}%
\pgfsetstrokecolor{currentstroke}%
\pgfsetdash{}{0pt}%
\pgfpathmoveto{\pgfqpoint{0.682874in}{1.469425in}}%
\pgfpathlineto{\pgfqpoint{0.725026in}{1.464236in}}%
\pgfpathlineto{\pgfqpoint{0.687023in}{1.495518in}}%
\pgfpathlineto{\pgfqpoint{0.644747in}{1.500831in}}%
\pgfpathclose%
\pgfusepath{fill}%
\end{pgfscope}%
\begin{pgfscope}%
\pgfpathrectangle{\pgfqpoint{0.150000in}{0.150000in}}{\pgfqpoint{2.700000in}{1.950000in}}%
\pgfusepath{clip}%
\pgfsetbuttcap%
\pgfsetroundjoin%
\definecolor{currentfill}{rgb}{0.499341,0.560999,0.647319}%
\pgfsetfillcolor{currentfill}%
\pgfsetlinewidth{0.000000pt}%
\definecolor{currentstroke}{rgb}{0.000000,0.000000,0.000000}%
\pgfsetstrokecolor{currentstroke}%
\pgfsetdash{}{0pt}%
\pgfpathmoveto{\pgfqpoint{0.992860in}{1.319160in}}%
\pgfpathlineto{\pgfqpoint{1.033203in}{1.326598in}}%
\pgfpathlineto{\pgfqpoint{0.995504in}{1.346004in}}%
\pgfpathlineto{\pgfqpoint{0.955123in}{1.338638in}}%
\pgfpathclose%
\pgfusepath{fill}%
\end{pgfscope}%
\begin{pgfscope}%
\pgfpathrectangle{\pgfqpoint{0.150000in}{0.150000in}}{\pgfqpoint{2.700000in}{1.950000in}}%
\pgfusepath{clip}%
\pgfsetbuttcap%
\pgfsetroundjoin%
\definecolor{currentfill}{rgb}{0.524219,0.582812,0.664844}%
\pgfsetfillcolor{currentfill}%
\pgfsetlinewidth{0.000000pt}%
\definecolor{currentstroke}{rgb}{0.000000,0.000000,0.000000}%
\pgfsetstrokecolor{currentstroke}%
\pgfsetdash{}{0pt}%
\pgfpathmoveto{\pgfqpoint{1.420038in}{1.275554in}}%
\pgfpathlineto{\pgfqpoint{1.459000in}{1.295106in}}%
\pgfpathlineto{\pgfqpoint{1.420474in}{1.314586in}}%
\pgfpathlineto{\pgfqpoint{1.381513in}{1.295106in}}%
\pgfpathclose%
\pgfusepath{fill}%
\end{pgfscope}%
\begin{pgfscope}%
\pgfpathrectangle{\pgfqpoint{0.150000in}{0.150000in}}{\pgfqpoint{2.700000in}{1.950000in}}%
\pgfusepath{clip}%
\pgfsetbuttcap%
\pgfsetroundjoin%
\definecolor{currentfill}{rgb}{0.524219,0.582812,0.664844}%
\pgfsetfillcolor{currentfill}%
\pgfsetlinewidth{0.000000pt}%
\definecolor{currentstroke}{rgb}{0.000000,0.000000,0.000000}%
\pgfsetstrokecolor{currentstroke}%
\pgfsetdash{}{0pt}%
\pgfpathmoveto{\pgfqpoint{1.342406in}{1.275554in}}%
\pgfpathlineto{\pgfqpoint{1.381513in}{1.295106in}}%
\pgfpathlineto{\pgfqpoint{1.343133in}{1.314586in}}%
\pgfpathlineto{\pgfqpoint{1.304027in}{1.295106in}}%
\pgfpathclose%
\pgfusepath{fill}%
\end{pgfscope}%
\begin{pgfscope}%
\pgfpathrectangle{\pgfqpoint{0.150000in}{0.150000in}}{\pgfqpoint{2.700000in}{1.950000in}}%
\pgfusepath{clip}%
\pgfsetbuttcap%
\pgfsetroundjoin%
\definecolor{currentfill}{rgb}{0.524219,0.582812,0.664844}%
\pgfsetfillcolor{currentfill}%
\pgfsetlinewidth{0.000000pt}%
\definecolor{currentstroke}{rgb}{0.000000,0.000000,0.000000}%
\pgfsetstrokecolor{currentstroke}%
\pgfsetdash{}{0pt}%
\pgfpathmoveto{\pgfqpoint{1.264774in}{1.275554in}}%
\pgfpathlineto{\pgfqpoint{1.304027in}{1.295106in}}%
\pgfpathlineto{\pgfqpoint{1.265792in}{1.314586in}}%
\pgfpathlineto{\pgfqpoint{1.226540in}{1.295106in}}%
\pgfpathclose%
\pgfusepath{fill}%
\end{pgfscope}%
\begin{pgfscope}%
\pgfpathrectangle{\pgfqpoint{0.150000in}{0.150000in}}{\pgfqpoint{2.700000in}{1.950000in}}%
\pgfusepath{clip}%
\pgfsetbuttcap%
\pgfsetroundjoin%
\definecolor{currentfill}{rgb}{0.524219,0.582812,0.664844}%
\pgfsetfillcolor{currentfill}%
\pgfsetlinewidth{0.000000pt}%
\definecolor{currentstroke}{rgb}{0.000000,0.000000,0.000000}%
\pgfsetstrokecolor{currentstroke}%
\pgfsetdash{}{0pt}%
\pgfpathmoveto{\pgfqpoint{1.187141in}{1.275554in}}%
\pgfpathlineto{\pgfqpoint{1.226540in}{1.295106in}}%
\pgfpathlineto{\pgfqpoint{1.188450in}{1.314586in}}%
\pgfpathlineto{\pgfqpoint{1.149054in}{1.295106in}}%
\pgfpathclose%
\pgfusepath{fill}%
\end{pgfscope}%
\begin{pgfscope}%
\pgfpathrectangle{\pgfqpoint{0.150000in}{0.150000in}}{\pgfqpoint{2.700000in}{1.950000in}}%
\pgfusepath{clip}%
\pgfsetbuttcap%
\pgfsetroundjoin%
\definecolor{currentfill}{rgb}{0.418490,0.490104,0.590365}%
\pgfsetfillcolor{currentfill}%
\pgfsetlinewidth{0.000000pt}%
\definecolor{currentstroke}{rgb}{0.000000,0.000000,0.000000}%
\pgfsetstrokecolor{currentstroke}%
\pgfsetdash{}{0pt}%
\pgfpathmoveto{\pgfqpoint{0.720140in}{1.450171in}}%
\pgfpathlineto{\pgfqpoint{0.762217in}{1.445052in}}%
\pgfpathlineto{\pgfqpoint{0.725026in}{1.464236in}}%
\pgfpathlineto{\pgfqpoint{0.682874in}{1.469425in}}%
\pgfpathclose%
\pgfusepath{fill}%
\end{pgfscope}%
\begin{pgfscope}%
\pgfpathrectangle{\pgfqpoint{0.150000in}{0.150000in}}{\pgfqpoint{2.700000in}{1.950000in}}%
\pgfusepath{clip}%
\pgfsetbuttcap%
\pgfsetroundjoin%
\definecolor{currentfill}{rgb}{0.468244,0.533732,0.625414}%
\pgfsetfillcolor{currentfill}%
\pgfsetlinewidth{0.000000pt}%
\definecolor{currentstroke}{rgb}{0.000000,0.000000,0.000000}%
\pgfsetstrokecolor{currentstroke}%
\pgfsetdash{}{0pt}%
\pgfpathmoveto{\pgfqpoint{0.876124in}{1.362798in}}%
\pgfpathlineto{\pgfqpoint{0.916833in}{1.370108in}}%
\pgfpathlineto{\pgfqpoint{0.879337in}{1.389440in}}%
\pgfpathlineto{\pgfqpoint{0.837807in}{1.394321in}}%
\pgfpathclose%
\pgfusepath{fill}%
\end{pgfscope}%
\begin{pgfscope}%
\pgfpathrectangle{\pgfqpoint{0.150000in}{0.150000in}}{\pgfqpoint{2.700000in}{1.950000in}}%
\pgfusepath{clip}%
\pgfsetbuttcap%
\pgfsetroundjoin%
\definecolor{currentfill}{rgb}{0.517999,0.577359,0.660463}%
\pgfsetfillcolor{currentfill}%
\pgfsetlinewidth{0.000000pt}%
\definecolor{currentstroke}{rgb}{0.000000,0.000000,0.000000}%
\pgfsetstrokecolor{currentstroke}%
\pgfsetdash{}{0pt}%
\pgfpathmoveto{\pgfqpoint{1.109028in}{1.287567in}}%
\pgfpathlineto{\pgfqpoint{1.149054in}{1.295106in}}%
\pgfpathlineto{\pgfqpoint{1.111109in}{1.314586in}}%
\pgfpathlineto{\pgfqpoint{1.071044in}{1.307119in}}%
\pgfpathclose%
\pgfusepath{fill}%
\end{pgfscope}%
\begin{pgfscope}%
\pgfpathrectangle{\pgfqpoint{0.150000in}{0.150000in}}{\pgfqpoint{2.700000in}{1.950000in}}%
\pgfusepath{clip}%
\pgfsetbuttcap%
\pgfsetroundjoin%
\definecolor{currentfill}{rgb}{0.586412,0.637347,0.708655}%
\pgfsetfillcolor{currentfill}%
\pgfsetlinewidth{0.000000pt}%
\definecolor{currentstroke}{rgb}{0.000000,0.000000,0.000000}%
\pgfsetstrokecolor{currentstroke}%
\pgfsetdash{}{0pt}%
\pgfpathmoveto{\pgfqpoint{1.459027in}{1.133102in}}%
\pgfpathlineto{\pgfqpoint{1.497829in}{1.152878in}}%
\pgfpathlineto{\pgfqpoint{1.458883in}{1.231980in}}%
\pgfpathlineto{\pgfqpoint{1.419862in}{1.212279in}}%
\pgfpathclose%
\pgfusepath{fill}%
\end{pgfscope}%
\begin{pgfscope}%
\pgfpathrectangle{\pgfqpoint{0.150000in}{0.150000in}}{\pgfqpoint{2.700000in}{1.950000in}}%
\pgfusepath{clip}%
\pgfsetbuttcap%
\pgfsetroundjoin%
\definecolor{currentfill}{rgb}{0.661045,0.702788,0.761229}%
\pgfsetfillcolor{currentfill}%
\pgfsetlinewidth{0.000000pt}%
\definecolor{currentstroke}{rgb}{0.000000,0.000000,0.000000}%
\pgfsetstrokecolor{currentstroke}%
\pgfsetdash{}{0pt}%
\pgfpathmoveto{\pgfqpoint{1.420296in}{1.014594in}}%
\pgfpathlineto{\pgfqpoint{1.459171in}{1.034592in}}%
\pgfpathlineto{\pgfqpoint{1.420079in}{1.113252in}}%
\pgfpathlineto{\pgfqpoint{1.380985in}{1.093327in}}%
\pgfpathclose%
\pgfusepath{fill}%
\end{pgfscope}%
\begin{pgfscope}%
\pgfpathrectangle{\pgfqpoint{0.150000in}{0.150000in}}{\pgfqpoint{2.700000in}{1.950000in}}%
\pgfusepath{clip}%
\pgfsetbuttcap%
\pgfsetroundjoin%
\definecolor{currentfill}{rgb}{0.729458,0.762776,0.809421}%
\pgfsetfillcolor{currentfill}%
\pgfsetlinewidth{0.000000pt}%
\definecolor{currentstroke}{rgb}{0.000000,0.000000,0.000000}%
\pgfsetstrokecolor{currentstroke}%
\pgfsetdash{}{0pt}%
\pgfpathmoveto{\pgfqpoint{1.381563in}{0.896081in}}%
\pgfpathlineto{\pgfqpoint{1.420512in}{0.916302in}}%
\pgfpathlineto{\pgfqpoint{1.381274in}{0.994520in}}%
\pgfpathlineto{\pgfqpoint{1.342106in}{0.974371in}}%
\pgfpathclose%
\pgfusepath{fill}%
\end{pgfscope}%
\begin{pgfscope}%
\pgfpathrectangle{\pgfqpoint{0.150000in}{0.150000in}}{\pgfqpoint{2.700000in}{1.950000in}}%
\pgfusepath{clip}%
\pgfsetbuttcap%
\pgfsetroundjoin%
\definecolor{currentfill}{rgb}{0.804090,0.828217,0.861994}%
\pgfsetfillcolor{currentfill}%
\pgfsetlinewidth{0.000000pt}%
\definecolor{currentstroke}{rgb}{0.000000,0.000000,0.000000}%
\pgfsetstrokecolor{currentstroke}%
\pgfsetdash{}{0pt}%
\pgfpathmoveto{\pgfqpoint{1.342829in}{0.777565in}}%
\pgfpathlineto{\pgfqpoint{1.381851in}{0.798008in}}%
\pgfpathlineto{\pgfqpoint{1.342468in}{0.875785in}}%
\pgfpathlineto{\pgfqpoint{1.303226in}{0.855412in}}%
\pgfpathclose%
\pgfusepath{fill}%
\end{pgfscope}%
\begin{pgfscope}%
\pgfpathrectangle{\pgfqpoint{0.150000in}{0.150000in}}{\pgfqpoint{2.700000in}{1.950000in}}%
\pgfusepath{clip}%
\pgfsetbuttcap%
\pgfsetroundjoin%
\definecolor{currentfill}{rgb}{0.536657,0.593719,0.673606}%
\pgfsetfillcolor{currentfill}%
\pgfsetlinewidth{0.000000pt}%
\definecolor{currentstroke}{rgb}{0.000000,0.000000,0.000000}%
\pgfsetstrokecolor{currentstroke}%
\pgfsetdash{}{0pt}%
\pgfpathmoveto{\pgfqpoint{1.458883in}{1.231980in}}%
\pgfpathlineto{\pgfqpoint{1.497758in}{1.251607in}}%
\pgfpathlineto{\pgfqpoint{1.459000in}{1.295106in}}%
\pgfpathlineto{\pgfqpoint{1.420038in}{1.275554in}}%
\pgfpathclose%
\pgfusepath{fill}%
\end{pgfscope}%
\begin{pgfscope}%
\pgfpathrectangle{\pgfqpoint{0.150000in}{0.150000in}}{\pgfqpoint{2.700000in}{1.950000in}}%
\pgfusepath{clip}%
\pgfsetbuttcap%
\pgfsetroundjoin%
\definecolor{currentfill}{rgb}{0.430928,0.501011,0.599127}%
\pgfsetfillcolor{currentfill}%
\pgfsetlinewidth{0.000000pt}%
\definecolor{currentstroke}{rgb}{0.000000,0.000000,0.000000}%
\pgfsetstrokecolor{currentstroke}%
\pgfsetdash{}{0pt}%
\pgfpathmoveto{\pgfqpoint{0.758425in}{1.418643in}}%
\pgfpathlineto{\pgfqpoint{0.800376in}{1.413650in}}%
\pgfpathlineto{\pgfqpoint{0.762217in}{1.445052in}}%
\pgfpathlineto{\pgfqpoint{0.720140in}{1.450171in}}%
\pgfpathclose%
\pgfusepath{fill}%
\end{pgfscope}%
\begin{pgfscope}%
\pgfpathrectangle{\pgfqpoint{0.150000in}{0.150000in}}{\pgfqpoint{2.700000in}{1.950000in}}%
\pgfusepath{clip}%
\pgfsetbuttcap%
\pgfsetroundjoin%
\definecolor{currentfill}{rgb}{0.499341,0.560999,0.647319}%
\pgfsetfillcolor{currentfill}%
\pgfsetlinewidth{0.000000pt}%
\definecolor{currentstroke}{rgb}{0.000000,0.000000,0.000000}%
\pgfsetstrokecolor{currentstroke}%
\pgfsetdash{}{0pt}%
\pgfpathmoveto{\pgfqpoint{1.030739in}{1.299608in}}%
\pgfpathlineto{\pgfqpoint{1.071044in}{1.307119in}}%
\pgfpathlineto{\pgfqpoint{1.033203in}{1.326598in}}%
\pgfpathlineto{\pgfqpoint{0.992860in}{1.319160in}}%
\pgfpathclose%
\pgfusepath{fill}%
\end{pgfscope}%
\begin{pgfscope}%
\pgfpathrectangle{\pgfqpoint{0.150000in}{0.150000in}}{\pgfqpoint{2.700000in}{1.950000in}}%
\pgfusepath{clip}%
\pgfsetbuttcap%
\pgfsetroundjoin%
\definecolor{currentfill}{rgb}{0.474464,0.539185,0.629795}%
\pgfsetfillcolor{currentfill}%
\pgfsetlinewidth{0.000000pt}%
\definecolor{currentstroke}{rgb}{0.000000,0.000000,0.000000}%
\pgfsetstrokecolor{currentstroke}%
\pgfsetdash{}{0pt}%
\pgfpathmoveto{\pgfqpoint{0.913797in}{1.343322in}}%
\pgfpathlineto{\pgfqpoint{0.955123in}{1.338638in}}%
\pgfpathlineto{\pgfqpoint{0.916833in}{1.370108in}}%
\pgfpathlineto{\pgfqpoint{0.876124in}{1.362798in}}%
\pgfpathclose%
\pgfusepath{fill}%
\end{pgfscope}%
\begin{pgfscope}%
\pgfpathrectangle{\pgfqpoint{0.150000in}{0.150000in}}{\pgfqpoint{2.700000in}{1.950000in}}%
\pgfusepath{clip}%
\pgfsetbuttcap%
\pgfsetroundjoin%
\definecolor{currentfill}{rgb}{0.524219,0.582812,0.664844}%
\pgfsetfillcolor{currentfill}%
\pgfsetlinewidth{0.000000pt}%
\definecolor{currentstroke}{rgb}{0.000000,0.000000,0.000000}%
\pgfsetstrokecolor{currentstroke}%
\pgfsetdash{}{0pt}%
\pgfpathmoveto{\pgfqpoint{1.380929in}{1.255927in}}%
\pgfpathlineto{\pgfqpoint{1.420038in}{1.275554in}}%
\pgfpathlineto{\pgfqpoint{1.381513in}{1.295106in}}%
\pgfpathlineto{\pgfqpoint{1.342406in}{1.275554in}}%
\pgfpathclose%
\pgfusepath{fill}%
\end{pgfscope}%
\begin{pgfscope}%
\pgfpathrectangle{\pgfqpoint{0.150000in}{0.150000in}}{\pgfqpoint{2.700000in}{1.950000in}}%
\pgfusepath{clip}%
\pgfsetbuttcap%
\pgfsetroundjoin%
\definecolor{currentfill}{rgb}{0.524219,0.582812,0.664844}%
\pgfsetfillcolor{currentfill}%
\pgfsetlinewidth{0.000000pt}%
\definecolor{currentstroke}{rgb}{0.000000,0.000000,0.000000}%
\pgfsetstrokecolor{currentstroke}%
\pgfsetdash{}{0pt}%
\pgfpathmoveto{\pgfqpoint{1.303151in}{1.255927in}}%
\pgfpathlineto{\pgfqpoint{1.342406in}{1.275554in}}%
\pgfpathlineto{\pgfqpoint{1.304027in}{1.295106in}}%
\pgfpathlineto{\pgfqpoint{1.264774in}{1.275554in}}%
\pgfpathclose%
\pgfusepath{fill}%
\end{pgfscope}%
\begin{pgfscope}%
\pgfpathrectangle{\pgfqpoint{0.150000in}{0.150000in}}{\pgfqpoint{2.700000in}{1.950000in}}%
\pgfusepath{clip}%
\pgfsetbuttcap%
\pgfsetroundjoin%
\definecolor{currentfill}{rgb}{0.524219,0.582812,0.664844}%
\pgfsetfillcolor{currentfill}%
\pgfsetlinewidth{0.000000pt}%
\definecolor{currentstroke}{rgb}{0.000000,0.000000,0.000000}%
\pgfsetstrokecolor{currentstroke}%
\pgfsetdash{}{0pt}%
\pgfpathmoveto{\pgfqpoint{1.225372in}{1.255927in}}%
\pgfpathlineto{\pgfqpoint{1.264774in}{1.275554in}}%
\pgfpathlineto{\pgfqpoint{1.226540in}{1.295106in}}%
\pgfpathlineto{\pgfqpoint{1.187141in}{1.275554in}}%
\pgfpathclose%
\pgfusepath{fill}%
\end{pgfscope}%
\begin{pgfscope}%
\pgfpathrectangle{\pgfqpoint{0.150000in}{0.150000in}}{\pgfqpoint{2.700000in}{1.950000in}}%
\pgfusepath{clip}%
\pgfsetbuttcap%
\pgfsetroundjoin%
\definecolor{currentfill}{rgb}{0.374954,0.451930,0.559697}%
\pgfsetfillcolor{currentfill}%
\pgfsetlinewidth{0.000000pt}%
\definecolor{currentstroke}{rgb}{0.000000,0.000000,0.000000}%
\pgfsetstrokecolor{currentstroke}%
\pgfsetdash{}{0pt}%
\pgfpathmoveto{\pgfqpoint{0.602120in}{1.506189in}}%
\pgfpathlineto{\pgfqpoint{0.644747in}{1.500831in}}%
\pgfpathlineto{\pgfqpoint{0.607720in}{1.519939in}}%
\pgfpathlineto{\pgfqpoint{0.563925in}{1.537674in}}%
\pgfpathclose%
\pgfusepath{fill}%
\end{pgfscope}%
\begin{pgfscope}%
\pgfpathrectangle{\pgfqpoint{0.150000in}{0.150000in}}{\pgfqpoint{2.700000in}{1.950000in}}%
\pgfusepath{clip}%
\pgfsetbuttcap%
\pgfsetroundjoin%
\definecolor{currentfill}{rgb}{0.517999,0.577359,0.660463}%
\pgfsetfillcolor{currentfill}%
\pgfsetlinewidth{0.000000pt}%
\definecolor{currentstroke}{rgb}{0.000000,0.000000,0.000000}%
\pgfsetstrokecolor{currentstroke}%
\pgfsetdash{}{0pt}%
\pgfpathmoveto{\pgfqpoint{1.147155in}{1.267942in}}%
\pgfpathlineto{\pgfqpoint{1.187141in}{1.275554in}}%
\pgfpathlineto{\pgfqpoint{1.149054in}{1.295106in}}%
\pgfpathlineto{\pgfqpoint{1.109028in}{1.287567in}}%
\pgfpathclose%
\pgfusepath{fill}%
\end{pgfscope}%
\begin{pgfscope}%
\pgfpathrectangle{\pgfqpoint{0.150000in}{0.150000in}}{\pgfqpoint{2.700000in}{1.950000in}}%
\pgfusepath{clip}%
\pgfsetbuttcap%
\pgfsetroundjoin%
\definecolor{currentfill}{rgb}{0.586412,0.637347,0.708655}%
\pgfsetfillcolor{currentfill}%
\pgfsetlinewidth{0.000000pt}%
\definecolor{currentstroke}{rgb}{0.000000,0.000000,0.000000}%
\pgfsetstrokecolor{currentstroke}%
\pgfsetdash{}{0pt}%
\pgfpathmoveto{\pgfqpoint{1.420079in}{1.113252in}}%
\pgfpathlineto{\pgfqpoint{1.459027in}{1.133102in}}%
\pgfpathlineto{\pgfqpoint{1.419862in}{1.212279in}}%
\pgfpathlineto{\pgfqpoint{1.380694in}{1.192503in}}%
\pgfpathclose%
\pgfusepath{fill}%
\end{pgfscope}%
\begin{pgfscope}%
\pgfpathrectangle{\pgfqpoint{0.150000in}{0.150000in}}{\pgfqpoint{2.700000in}{1.950000in}}%
\pgfusepath{clip}%
\pgfsetbuttcap%
\pgfsetroundjoin%
\definecolor{currentfill}{rgb}{0.661045,0.702788,0.761229}%
\pgfsetfillcolor{currentfill}%
\pgfsetlinewidth{0.000000pt}%
\definecolor{currentstroke}{rgb}{0.000000,0.000000,0.000000}%
\pgfsetstrokecolor{currentstroke}%
\pgfsetdash{}{0pt}%
\pgfpathmoveto{\pgfqpoint{1.381274in}{0.994520in}}%
\pgfpathlineto{\pgfqpoint{1.420296in}{1.014594in}}%
\pgfpathlineto{\pgfqpoint{1.380985in}{1.093327in}}%
\pgfpathlineto{\pgfqpoint{1.341742in}{1.073327in}}%
\pgfpathclose%
\pgfusepath{fill}%
\end{pgfscope}%
\begin{pgfscope}%
\pgfpathrectangle{\pgfqpoint{0.150000in}{0.150000in}}{\pgfqpoint{2.700000in}{1.950000in}}%
\pgfusepath{clip}%
\pgfsetbuttcap%
\pgfsetroundjoin%
\definecolor{currentfill}{rgb}{0.729458,0.762776,0.809421}%
\pgfsetfillcolor{currentfill}%
\pgfsetlinewidth{0.000000pt}%
\definecolor{currentstroke}{rgb}{0.000000,0.000000,0.000000}%
\pgfsetstrokecolor{currentstroke}%
\pgfsetdash{}{0pt}%
\pgfpathmoveto{\pgfqpoint{1.342468in}{0.875785in}}%
\pgfpathlineto{\pgfqpoint{1.381563in}{0.896081in}}%
\pgfpathlineto{\pgfqpoint{1.342106in}{0.974371in}}%
\pgfpathlineto{\pgfqpoint{1.302790in}{0.954146in}}%
\pgfpathclose%
\pgfusepath{fill}%
\end{pgfscope}%
\begin{pgfscope}%
\pgfpathrectangle{\pgfqpoint{0.150000in}{0.150000in}}{\pgfqpoint{2.700000in}{1.950000in}}%
\pgfusepath{clip}%
\pgfsetbuttcap%
\pgfsetroundjoin%
\definecolor{currentfill}{rgb}{0.804090,0.828217,0.861994}%
\pgfsetfillcolor{currentfill}%
\pgfsetlinewidth{0.000000pt}%
\definecolor{currentstroke}{rgb}{0.000000,0.000000,0.000000}%
\pgfsetstrokecolor{currentstroke}%
\pgfsetdash{}{0pt}%
\pgfpathmoveto{\pgfqpoint{1.303661in}{0.757046in}}%
\pgfpathlineto{\pgfqpoint{1.342829in}{0.777565in}}%
\pgfpathlineto{\pgfqpoint{1.303226in}{0.855412in}}%
\pgfpathlineto{\pgfqpoint{1.263836in}{0.834962in}}%
\pgfpathclose%
\pgfusepath{fill}%
\end{pgfscope}%
\begin{pgfscope}%
\pgfpathrectangle{\pgfqpoint{0.150000in}{0.150000in}}{\pgfqpoint{2.700000in}{1.950000in}}%
\pgfusepath{clip}%
\pgfsetbuttcap%
\pgfsetroundjoin%
\definecolor{currentfill}{rgb}{0.536657,0.593719,0.673606}%
\pgfsetfillcolor{currentfill}%
\pgfsetlinewidth{0.000000pt}%
\definecolor{currentstroke}{rgb}{0.000000,0.000000,0.000000}%
\pgfsetstrokecolor{currentstroke}%
\pgfsetdash{}{0pt}%
\pgfpathmoveto{\pgfqpoint{1.419862in}{1.212279in}}%
\pgfpathlineto{\pgfqpoint{1.458883in}{1.231980in}}%
\pgfpathlineto{\pgfqpoint{1.420038in}{1.275554in}}%
\pgfpathlineto{\pgfqpoint{1.380929in}{1.255927in}}%
\pgfpathclose%
\pgfusepath{fill}%
\end{pgfscope}%
\begin{pgfscope}%
\pgfpathrectangle{\pgfqpoint{0.150000in}{0.150000in}}{\pgfqpoint{2.700000in}{1.950000in}}%
\pgfusepath{clip}%
\pgfsetbuttcap%
\pgfsetroundjoin%
\definecolor{currentfill}{rgb}{0.437148,0.506464,0.603508}%
\pgfsetfillcolor{currentfill}%
\pgfsetlinewidth{0.000000pt}%
\definecolor{currentstroke}{rgb}{0.000000,0.000000,0.000000}%
\pgfsetstrokecolor{currentstroke}%
\pgfsetdash{}{0pt}%
\pgfpathmoveto{\pgfqpoint{0.796768in}{1.387067in}}%
\pgfpathlineto{\pgfqpoint{0.837807in}{1.394321in}}%
\pgfpathlineto{\pgfqpoint{0.800376in}{1.413650in}}%
\pgfpathlineto{\pgfqpoint{0.758425in}{1.418643in}}%
\pgfpathclose%
\pgfusepath{fill}%
\end{pgfscope}%
\begin{pgfscope}%
\pgfpathrectangle{\pgfqpoint{0.150000in}{0.150000in}}{\pgfqpoint{2.700000in}{1.950000in}}%
\pgfusepath{clip}%
\pgfsetbuttcap%
\pgfsetroundjoin%
\definecolor{currentfill}{rgb}{0.480683,0.544638,0.634176}%
\pgfsetfillcolor{currentfill}%
\pgfsetlinewidth{0.000000pt}%
\definecolor{currentstroke}{rgb}{0.000000,0.000000,0.000000}%
\pgfsetstrokecolor{currentstroke}%
\pgfsetdash{}{0pt}%
\pgfpathmoveto{\pgfqpoint{0.952272in}{1.311676in}}%
\pgfpathlineto{\pgfqpoint{0.992860in}{1.319160in}}%
\pgfpathlineto{\pgfqpoint{0.955123in}{1.338638in}}%
\pgfpathlineto{\pgfqpoint{0.913797in}{1.343322in}}%
\pgfpathclose%
\pgfusepath{fill}%
\end{pgfscope}%
\begin{pgfscope}%
\pgfpathrectangle{\pgfqpoint{0.150000in}{0.150000in}}{\pgfqpoint{2.700000in}{1.950000in}}%
\pgfusepath{clip}%
\pgfsetbuttcap%
\pgfsetroundjoin%
\definecolor{currentfill}{rgb}{0.387393,0.462837,0.568459}%
\pgfsetfillcolor{currentfill}%
\pgfsetlinewidth{0.000000pt}%
\definecolor{currentstroke}{rgb}{0.000000,0.000000,0.000000}%
\pgfsetstrokecolor{currentstroke}%
\pgfsetdash{}{0pt}%
\pgfpathmoveto{\pgfqpoint{0.640373in}{1.474656in}}%
\pgfpathlineto{\pgfqpoint{0.682874in}{1.469425in}}%
\pgfpathlineto{\pgfqpoint{0.644747in}{1.500831in}}%
\pgfpathlineto{\pgfqpoint{0.602120in}{1.506189in}}%
\pgfpathclose%
\pgfusepath{fill}%
\end{pgfscope}%
\begin{pgfscope}%
\pgfpathrectangle{\pgfqpoint{0.150000in}{0.150000in}}{\pgfqpoint{2.700000in}{1.950000in}}%
\pgfusepath{clip}%
\pgfsetbuttcap%
\pgfsetroundjoin%
\definecolor{currentfill}{rgb}{0.499341,0.560999,0.647319}%
\pgfsetfillcolor{currentfill}%
\pgfsetlinewidth{0.000000pt}%
\definecolor{currentstroke}{rgb}{0.000000,0.000000,0.000000}%
\pgfsetstrokecolor{currentstroke}%
\pgfsetdash{}{0pt}%
\pgfpathmoveto{\pgfqpoint{1.068761in}{1.279983in}}%
\pgfpathlineto{\pgfqpoint{1.109028in}{1.287567in}}%
\pgfpathlineto{\pgfqpoint{1.071044in}{1.307119in}}%
\pgfpathlineto{\pgfqpoint{1.030739in}{1.299608in}}%
\pgfpathclose%
\pgfusepath{fill}%
\end{pgfscope}%
\begin{pgfscope}%
\pgfpathrectangle{\pgfqpoint{0.150000in}{0.150000in}}{\pgfqpoint{2.700000in}{1.950000in}}%
\pgfusepath{clip}%
\pgfsetbuttcap%
\pgfsetroundjoin%
\definecolor{currentfill}{rgb}{0.524219,0.582812,0.664844}%
\pgfsetfillcolor{currentfill}%
\pgfsetlinewidth{0.000000pt}%
\definecolor{currentstroke}{rgb}{0.000000,0.000000,0.000000}%
\pgfsetstrokecolor{currentstroke}%
\pgfsetdash{}{0pt}%
\pgfpathmoveto{\pgfqpoint{1.341673in}{1.236227in}}%
\pgfpathlineto{\pgfqpoint{1.380929in}{1.255927in}}%
\pgfpathlineto{\pgfqpoint{1.342406in}{1.275554in}}%
\pgfpathlineto{\pgfqpoint{1.303151in}{1.255927in}}%
\pgfpathclose%
\pgfusepath{fill}%
\end{pgfscope}%
\begin{pgfscope}%
\pgfpathrectangle{\pgfqpoint{0.150000in}{0.150000in}}{\pgfqpoint{2.700000in}{1.950000in}}%
\pgfusepath{clip}%
\pgfsetbuttcap%
\pgfsetroundjoin%
\definecolor{currentfill}{rgb}{0.524219,0.582812,0.664844}%
\pgfsetfillcolor{currentfill}%
\pgfsetlinewidth{0.000000pt}%
\definecolor{currentstroke}{rgb}{0.000000,0.000000,0.000000}%
\pgfsetstrokecolor{currentstroke}%
\pgfsetdash{}{0pt}%
\pgfpathmoveto{\pgfqpoint{1.263748in}{1.236227in}}%
\pgfpathlineto{\pgfqpoint{1.303151in}{1.255927in}}%
\pgfpathlineto{\pgfqpoint{1.264774in}{1.275554in}}%
\pgfpathlineto{\pgfqpoint{1.225372in}{1.255927in}}%
\pgfpathclose%
\pgfusepath{fill}%
\end{pgfscope}%
\begin{pgfscope}%
\pgfpathrectangle{\pgfqpoint{0.150000in}{0.150000in}}{\pgfqpoint{2.700000in}{1.950000in}}%
\pgfusepath{clip}%
\pgfsetbuttcap%
\pgfsetroundjoin%
\definecolor{currentfill}{rgb}{0.443367,0.511918,0.607889}%
\pgfsetfillcolor{currentfill}%
\pgfsetlinewidth{0.000000pt}%
\definecolor{currentstroke}{rgb}{0.000000,0.000000,0.000000}%
\pgfsetstrokecolor{currentstroke}%
\pgfsetdash{}{0pt}%
\pgfpathmoveto{\pgfqpoint{0.834376in}{1.367593in}}%
\pgfpathlineto{\pgfqpoint{0.876124in}{1.362798in}}%
\pgfpathlineto{\pgfqpoint{0.837807in}{1.394321in}}%
\pgfpathlineto{\pgfqpoint{0.796768in}{1.387067in}}%
\pgfpathclose%
\pgfusepath{fill}%
\end{pgfscope}%
\begin{pgfscope}%
\pgfpathrectangle{\pgfqpoint{0.150000in}{0.150000in}}{\pgfqpoint{2.700000in}{1.950000in}}%
\pgfusepath{clip}%
\pgfsetbuttcap%
\pgfsetroundjoin%
\definecolor{currentfill}{rgb}{0.517999,0.577359,0.660463}%
\pgfsetfillcolor{currentfill}%
\pgfsetlinewidth{0.000000pt}%
\definecolor{currentstroke}{rgb}{0.000000,0.000000,0.000000}%
\pgfsetstrokecolor{currentstroke}%
\pgfsetdash{}{0pt}%
\pgfpathmoveto{\pgfqpoint{1.185426in}{1.248242in}}%
\pgfpathlineto{\pgfqpoint{1.225372in}{1.255927in}}%
\pgfpathlineto{\pgfqpoint{1.187141in}{1.275554in}}%
\pgfpathlineto{\pgfqpoint{1.147155in}{1.267942in}}%
\pgfpathclose%
\pgfusepath{fill}%
\end{pgfscope}%
\begin{pgfscope}%
\pgfpathrectangle{\pgfqpoint{0.150000in}{0.150000in}}{\pgfqpoint{2.700000in}{1.950000in}}%
\pgfusepath{clip}%
\pgfsetbuttcap%
\pgfsetroundjoin%
\definecolor{currentfill}{rgb}{0.586412,0.637347,0.708655}%
\pgfsetfillcolor{currentfill}%
\pgfsetlinewidth{0.000000pt}%
\definecolor{currentstroke}{rgb}{0.000000,0.000000,0.000000}%
\pgfsetstrokecolor{currentstroke}%
\pgfsetdash{}{0pt}%
\pgfpathmoveto{\pgfqpoint{1.380985in}{1.093327in}}%
\pgfpathlineto{\pgfqpoint{1.420079in}{1.113252in}}%
\pgfpathlineto{\pgfqpoint{1.380694in}{1.192503in}}%
\pgfpathlineto{\pgfqpoint{1.341377in}{1.172653in}}%
\pgfpathclose%
\pgfusepath{fill}%
\end{pgfscope}%
\begin{pgfscope}%
\pgfpathrectangle{\pgfqpoint{0.150000in}{0.150000in}}{\pgfqpoint{2.700000in}{1.950000in}}%
\pgfusepath{clip}%
\pgfsetbuttcap%
\pgfsetroundjoin%
\definecolor{currentfill}{rgb}{0.661045,0.702788,0.761229}%
\pgfsetfillcolor{currentfill}%
\pgfsetlinewidth{0.000000pt}%
\definecolor{currentstroke}{rgb}{0.000000,0.000000,0.000000}%
\pgfsetstrokecolor{currentstroke}%
\pgfsetdash{}{0pt}%
\pgfpathmoveto{\pgfqpoint{1.342106in}{0.974371in}}%
\pgfpathlineto{\pgfqpoint{1.381274in}{0.994520in}}%
\pgfpathlineto{\pgfqpoint{1.341742in}{1.073327in}}%
\pgfpathlineto{\pgfqpoint{1.302352in}{1.053251in}}%
\pgfpathclose%
\pgfusepath{fill}%
\end{pgfscope}%
\begin{pgfscope}%
\pgfpathrectangle{\pgfqpoint{0.150000in}{0.150000in}}{\pgfqpoint{2.700000in}{1.950000in}}%
\pgfusepath{clip}%
\pgfsetbuttcap%
\pgfsetroundjoin%
\definecolor{currentfill}{rgb}{0.729458,0.762776,0.809421}%
\pgfsetfillcolor{currentfill}%
\pgfsetlinewidth{0.000000pt}%
\definecolor{currentstroke}{rgb}{0.000000,0.000000,0.000000}%
\pgfsetstrokecolor{currentstroke}%
\pgfsetdash{}{0pt}%
\pgfpathmoveto{\pgfqpoint{1.303226in}{0.855412in}}%
\pgfpathlineto{\pgfqpoint{1.342468in}{0.875785in}}%
\pgfpathlineto{\pgfqpoint{1.302790in}{0.954146in}}%
\pgfpathlineto{\pgfqpoint{1.263325in}{0.933845in}}%
\pgfpathclose%
\pgfusepath{fill}%
\end{pgfscope}%
\begin{pgfscope}%
\pgfpathrectangle{\pgfqpoint{0.150000in}{0.150000in}}{\pgfqpoint{2.700000in}{1.950000in}}%
\pgfusepath{clip}%
\pgfsetbuttcap%
\pgfsetroundjoin%
\definecolor{currentfill}{rgb}{0.393612,0.468290,0.572840}%
\pgfsetfillcolor{currentfill}%
\pgfsetlinewidth{0.000000pt}%
\definecolor{currentstroke}{rgb}{0.000000,0.000000,0.000000}%
\pgfsetstrokecolor{currentstroke}%
\pgfsetdash{}{0pt}%
\pgfpathmoveto{\pgfqpoint{0.677714in}{1.455333in}}%
\pgfpathlineto{\pgfqpoint{0.720140in}{1.450171in}}%
\pgfpathlineto{\pgfqpoint{0.682874in}{1.469425in}}%
\pgfpathlineto{\pgfqpoint{0.640373in}{1.474656in}}%
\pgfpathclose%
\pgfusepath{fill}%
\end{pgfscope}%
\begin{pgfscope}%
\pgfpathrectangle{\pgfqpoint{0.150000in}{0.150000in}}{\pgfqpoint{2.700000in}{1.950000in}}%
\pgfusepath{clip}%
\pgfsetbuttcap%
\pgfsetroundjoin%
\definecolor{currentfill}{rgb}{0.536657,0.593719,0.673606}%
\pgfsetfillcolor{currentfill}%
\pgfsetlinewidth{0.000000pt}%
\definecolor{currentstroke}{rgb}{0.000000,0.000000,0.000000}%
\pgfsetstrokecolor{currentstroke}%
\pgfsetdash{}{0pt}%
\pgfpathmoveto{\pgfqpoint{1.380694in}{1.192503in}}%
\pgfpathlineto{\pgfqpoint{1.419862in}{1.212279in}}%
\pgfpathlineto{\pgfqpoint{1.380929in}{1.255927in}}%
\pgfpathlineto{\pgfqpoint{1.341673in}{1.236227in}}%
\pgfpathclose%
\pgfusepath{fill}%
\end{pgfscope}%
\begin{pgfscope}%
\pgfpathrectangle{\pgfqpoint{0.150000in}{0.150000in}}{\pgfqpoint{2.700000in}{1.950000in}}%
\pgfusepath{clip}%
\pgfsetbuttcap%
\pgfsetroundjoin%
\definecolor{currentfill}{rgb}{0.486903,0.550092,0.638557}%
\pgfsetfillcolor{currentfill}%
\pgfsetlinewidth{0.000000pt}%
\definecolor{currentstroke}{rgb}{0.000000,0.000000,0.000000}%
\pgfsetstrokecolor{currentstroke}%
\pgfsetdash{}{0pt}%
\pgfpathmoveto{\pgfqpoint{0.990189in}{1.292052in}}%
\pgfpathlineto{\pgfqpoint{1.030739in}{1.299608in}}%
\pgfpathlineto{\pgfqpoint{0.992860in}{1.319160in}}%
\pgfpathlineto{\pgfqpoint{0.952272in}{1.311676in}}%
\pgfpathclose%
\pgfusepath{fill}%
\end{pgfscope}%
\begin{pgfscope}%
\pgfpathrectangle{\pgfqpoint{0.150000in}{0.150000in}}{\pgfqpoint{2.700000in}{1.950000in}}%
\pgfusepath{clip}%
\pgfsetbuttcap%
\pgfsetroundjoin%
\definecolor{currentfill}{rgb}{0.505561,0.566452,0.651700}%
\pgfsetfillcolor{currentfill}%
\pgfsetlinewidth{0.000000pt}%
\definecolor{currentstroke}{rgb}{0.000000,0.000000,0.000000}%
\pgfsetstrokecolor{currentstroke}%
\pgfsetdash{}{0pt}%
\pgfpathmoveto{\pgfqpoint{1.107412in}{1.248242in}}%
\pgfpathlineto{\pgfqpoint{1.147155in}{1.267942in}}%
\pgfpathlineto{\pgfqpoint{1.109028in}{1.287567in}}%
\pgfpathlineto{\pgfqpoint{1.068761in}{1.279983in}}%
\pgfpathclose%
\pgfusepath{fill}%
\end{pgfscope}%
\begin{pgfscope}%
\pgfpathrectangle{\pgfqpoint{0.150000in}{0.150000in}}{\pgfqpoint{2.700000in}{1.950000in}}%
\pgfusepath{clip}%
\pgfsetbuttcap%
\pgfsetroundjoin%
\definecolor{currentfill}{rgb}{0.455806,0.522825,0.616651}%
\pgfsetfillcolor{currentfill}%
\pgfsetlinewidth{0.000000pt}%
\definecolor{currentstroke}{rgb}{0.000000,0.000000,0.000000}%
\pgfsetstrokecolor{currentstroke}%
\pgfsetdash{}{0pt}%
\pgfpathmoveto{\pgfqpoint{0.872879in}{1.335894in}}%
\pgfpathlineto{\pgfqpoint{0.913797in}{1.343322in}}%
\pgfpathlineto{\pgfqpoint{0.876124in}{1.362798in}}%
\pgfpathlineto{\pgfqpoint{0.834376in}{1.367593in}}%
\pgfpathclose%
\pgfusepath{fill}%
\end{pgfscope}%
\begin{pgfscope}%
\pgfpathrectangle{\pgfqpoint{0.150000in}{0.150000in}}{\pgfqpoint{2.700000in}{1.950000in}}%
\pgfusepath{clip}%
\pgfsetbuttcap%
\pgfsetroundjoin%
\definecolor{currentfill}{rgb}{0.524219,0.582812,0.664844}%
\pgfsetfillcolor{currentfill}%
\pgfsetlinewidth{0.000000pt}%
\definecolor{currentstroke}{rgb}{0.000000,0.000000,0.000000}%
\pgfsetstrokecolor{currentstroke}%
\pgfsetdash{}{0pt}%
\pgfpathmoveto{\pgfqpoint{1.302268in}{1.216452in}}%
\pgfpathlineto{\pgfqpoint{1.341673in}{1.236227in}}%
\pgfpathlineto{\pgfqpoint{1.303151in}{1.255927in}}%
\pgfpathlineto{\pgfqpoint{1.263748in}{1.236227in}}%
\pgfpathclose%
\pgfusepath{fill}%
\end{pgfscope}%
\begin{pgfscope}%
\pgfpathrectangle{\pgfqpoint{0.150000in}{0.150000in}}{\pgfqpoint{2.700000in}{1.950000in}}%
\pgfusepath{clip}%
\pgfsetbuttcap%
\pgfsetroundjoin%
\definecolor{currentfill}{rgb}{0.399831,0.473744,0.577221}%
\pgfsetfillcolor{currentfill}%
\pgfsetlinewidth{0.000000pt}%
\definecolor{currentstroke}{rgb}{0.000000,0.000000,0.000000}%
\pgfsetstrokecolor{currentstroke}%
\pgfsetdash{}{0pt}%
\pgfpathmoveto{\pgfqpoint{0.716125in}{1.423677in}}%
\pgfpathlineto{\pgfqpoint{0.758425in}{1.418643in}}%
\pgfpathlineto{\pgfqpoint{0.720140in}{1.450171in}}%
\pgfpathlineto{\pgfqpoint{0.677714in}{1.455333in}}%
\pgfpathclose%
\pgfusepath{fill}%
\end{pgfscope}%
\begin{pgfscope}%
\pgfpathrectangle{\pgfqpoint{0.150000in}{0.150000in}}{\pgfqpoint{2.700000in}{1.950000in}}%
\pgfusepath{clip}%
\pgfsetbuttcap%
\pgfsetroundjoin%
\definecolor{currentfill}{rgb}{0.517999,0.577359,0.660463}%
\pgfsetfillcolor{currentfill}%
\pgfsetlinewidth{0.000000pt}%
\definecolor{currentstroke}{rgb}{0.000000,0.000000,0.000000}%
\pgfsetstrokecolor{currentstroke}%
\pgfsetdash{}{0pt}%
\pgfpathmoveto{\pgfqpoint{1.223842in}{1.228467in}}%
\pgfpathlineto{\pgfqpoint{1.263748in}{1.236227in}}%
\pgfpathlineto{\pgfqpoint{1.225372in}{1.255927in}}%
\pgfpathlineto{\pgfqpoint{1.185426in}{1.248242in}}%
\pgfpathclose%
\pgfusepath{fill}%
\end{pgfscope}%
\begin{pgfscope}%
\pgfpathrectangle{\pgfqpoint{0.150000in}{0.150000in}}{\pgfqpoint{2.700000in}{1.950000in}}%
\pgfusepath{clip}%
\pgfsetbuttcap%
\pgfsetroundjoin%
\definecolor{currentfill}{rgb}{0.586412,0.637347,0.708655}%
\pgfsetfillcolor{currentfill}%
\pgfsetlinewidth{0.000000pt}%
\definecolor{currentstroke}{rgb}{0.000000,0.000000,0.000000}%
\pgfsetstrokecolor{currentstroke}%
\pgfsetdash{}{0pt}%
\pgfpathmoveto{\pgfqpoint{1.341742in}{1.073327in}}%
\pgfpathlineto{\pgfqpoint{1.380985in}{1.093327in}}%
\pgfpathlineto{\pgfqpoint{1.341377in}{1.172653in}}%
\pgfpathlineto{\pgfqpoint{1.301912in}{1.152728in}}%
\pgfpathclose%
\pgfusepath{fill}%
\end{pgfscope}%
\begin{pgfscope}%
\pgfpathrectangle{\pgfqpoint{0.150000in}{0.150000in}}{\pgfqpoint{2.700000in}{1.950000in}}%
\pgfusepath{clip}%
\pgfsetbuttcap%
\pgfsetroundjoin%
\definecolor{currentfill}{rgb}{0.661045,0.702788,0.761229}%
\pgfsetfillcolor{currentfill}%
\pgfsetlinewidth{0.000000pt}%
\definecolor{currentstroke}{rgb}{0.000000,0.000000,0.000000}%
\pgfsetstrokecolor{currentstroke}%
\pgfsetdash{}{0pt}%
\pgfpathmoveto{\pgfqpoint{1.302790in}{0.954146in}}%
\pgfpathlineto{\pgfqpoint{1.342106in}{0.974371in}}%
\pgfpathlineto{\pgfqpoint{1.302352in}{1.053251in}}%
\pgfpathlineto{\pgfqpoint{1.262812in}{1.033099in}}%
\pgfpathclose%
\pgfusepath{fill}%
\end{pgfscope}%
\begin{pgfscope}%
\pgfpathrectangle{\pgfqpoint{0.150000in}{0.150000in}}{\pgfqpoint{2.700000in}{1.950000in}}%
\pgfusepath{clip}%
\pgfsetbuttcap%
\pgfsetroundjoin%
\definecolor{currentfill}{rgb}{0.729458,0.762776,0.809421}%
\pgfsetfillcolor{currentfill}%
\pgfsetlinewidth{0.000000pt}%
\definecolor{currentstroke}{rgb}{0.000000,0.000000,0.000000}%
\pgfsetstrokecolor{currentstroke}%
\pgfsetdash{}{0pt}%
\pgfpathmoveto{\pgfqpoint{1.263836in}{0.834962in}}%
\pgfpathlineto{\pgfqpoint{1.303226in}{0.855412in}}%
\pgfpathlineto{\pgfqpoint{1.263325in}{0.933845in}}%
\pgfpathlineto{\pgfqpoint{1.223710in}{0.913466in}}%
\pgfpathclose%
\pgfusepath{fill}%
\end{pgfscope}%
\begin{pgfscope}%
\pgfpathrectangle{\pgfqpoint{0.150000in}{0.150000in}}{\pgfqpoint{2.700000in}{1.950000in}}%
\pgfusepath{clip}%
\pgfsetbuttcap%
\pgfsetroundjoin%
\definecolor{currentfill}{rgb}{0.486903,0.550092,0.638557}%
\pgfsetfillcolor{currentfill}%
\pgfsetlinewidth{0.000000pt}%
\definecolor{currentstroke}{rgb}{0.000000,0.000000,0.000000}%
\pgfsetstrokecolor{currentstroke}%
\pgfsetdash{}{0pt}%
\pgfpathmoveto{\pgfqpoint{1.028825in}{1.260284in}}%
\pgfpathlineto{\pgfqpoint{1.068761in}{1.279983in}}%
\pgfpathlineto{\pgfqpoint{1.030739in}{1.299608in}}%
\pgfpathlineto{\pgfqpoint{0.990189in}{1.292052in}}%
\pgfpathclose%
\pgfusepath{fill}%
\end{pgfscope}%
\begin{pgfscope}%
\pgfpathrectangle{\pgfqpoint{0.150000in}{0.150000in}}{\pgfqpoint{2.700000in}{1.950000in}}%
\pgfusepath{clip}%
\pgfsetbuttcap%
\pgfsetroundjoin%
\definecolor{currentfill}{rgb}{0.462025,0.528278,0.621032}%
\pgfsetfillcolor{currentfill}%
\pgfsetlinewidth{0.000000pt}%
\definecolor{currentstroke}{rgb}{0.000000,0.000000,0.000000}%
\pgfsetstrokecolor{currentstroke}%
\pgfsetdash{}{0pt}%
\pgfpathmoveto{\pgfqpoint{0.911440in}{1.304148in}}%
\pgfpathlineto{\pgfqpoint{0.952272in}{1.311676in}}%
\pgfpathlineto{\pgfqpoint{0.913797in}{1.343322in}}%
\pgfpathlineto{\pgfqpoint{0.872879in}{1.335894in}}%
\pgfpathclose%
\pgfusepath{fill}%
\end{pgfscope}%
\begin{pgfscope}%
\pgfpathrectangle{\pgfqpoint{0.150000in}{0.150000in}}{\pgfqpoint{2.700000in}{1.950000in}}%
\pgfusepath{clip}%
\pgfsetbuttcap%
\pgfsetroundjoin%
\definecolor{currentfill}{rgb}{0.536657,0.593719,0.673606}%
\pgfsetfillcolor{currentfill}%
\pgfsetlinewidth{0.000000pt}%
\definecolor{currentstroke}{rgb}{0.000000,0.000000,0.000000}%
\pgfsetstrokecolor{currentstroke}%
\pgfsetdash{}{0pt}%
\pgfpathmoveto{\pgfqpoint{1.341377in}{1.172653in}}%
\pgfpathlineto{\pgfqpoint{1.380694in}{1.192503in}}%
\pgfpathlineto{\pgfqpoint{1.341673in}{1.236227in}}%
\pgfpathlineto{\pgfqpoint{1.302268in}{1.216452in}}%
\pgfpathclose%
\pgfusepath{fill}%
\end{pgfscope}%
\begin{pgfscope}%
\pgfpathrectangle{\pgfqpoint{0.150000in}{0.150000in}}{\pgfqpoint{2.700000in}{1.950000in}}%
\pgfusepath{clip}%
\pgfsetbuttcap%
\pgfsetroundjoin%
\definecolor{currentfill}{rgb}{0.412270,0.484651,0.585983}%
\pgfsetfillcolor{currentfill}%
\pgfsetlinewidth{0.000000pt}%
\definecolor{currentstroke}{rgb}{0.000000,0.000000,0.000000}%
\pgfsetstrokecolor{currentstroke}%
\pgfsetdash{}{0pt}%
\pgfpathmoveto{\pgfqpoint{0.754596in}{1.391974in}}%
\pgfpathlineto{\pgfqpoint{0.796768in}{1.387067in}}%
\pgfpathlineto{\pgfqpoint{0.758425in}{1.418643in}}%
\pgfpathlineto{\pgfqpoint{0.716125in}{1.423677in}}%
\pgfpathclose%
\pgfusepath{fill}%
\end{pgfscope}%
\begin{pgfscope}%
\pgfpathrectangle{\pgfqpoint{0.150000in}{0.150000in}}{\pgfqpoint{2.700000in}{1.950000in}}%
\pgfusepath{clip}%
\pgfsetbuttcap%
\pgfsetroundjoin%
\definecolor{currentfill}{rgb}{0.505561,0.566452,0.651700}%
\pgfsetfillcolor{currentfill}%
\pgfsetlinewidth{0.000000pt}%
\definecolor{currentstroke}{rgb}{0.000000,0.000000,0.000000}%
\pgfsetstrokecolor{currentstroke}%
\pgfsetdash{}{0pt}%
\pgfpathmoveto{\pgfqpoint{1.145681in}{1.228467in}}%
\pgfpathlineto{\pgfqpoint{1.185426in}{1.248242in}}%
\pgfpathlineto{\pgfqpoint{1.147155in}{1.267942in}}%
\pgfpathlineto{\pgfqpoint{1.107412in}{1.248242in}}%
\pgfpathclose%
\pgfusepath{fill}%
\end{pgfscope}%
\begin{pgfscope}%
\pgfpathrectangle{\pgfqpoint{0.150000in}{0.150000in}}{\pgfqpoint{2.700000in}{1.950000in}}%
\pgfusepath{clip}%
\pgfsetbuttcap%
\pgfsetroundjoin%
\definecolor{currentfill}{rgb}{0.424709,0.495558,0.594746}%
\pgfsetfillcolor{currentfill}%
\pgfsetlinewidth{0.000000pt}%
\definecolor{currentstroke}{rgb}{0.000000,0.000000,0.000000}%
\pgfsetstrokecolor{currentstroke}%
\pgfsetdash{}{0pt}%
\pgfpathmoveto{\pgfqpoint{0.793125in}{1.360222in}}%
\pgfpathlineto{\pgfqpoint{0.834376in}{1.367593in}}%
\pgfpathlineto{\pgfqpoint{0.796768in}{1.387067in}}%
\pgfpathlineto{\pgfqpoint{0.754596in}{1.391974in}}%
\pgfpathclose%
\pgfusepath{fill}%
\end{pgfscope}%
\begin{pgfscope}%
\pgfpathrectangle{\pgfqpoint{0.150000in}{0.150000in}}{\pgfqpoint{2.700000in}{1.950000in}}%
\pgfusepath{clip}%
\pgfsetbuttcap%
\pgfsetroundjoin%
\definecolor{currentfill}{rgb}{0.499341,0.560999,0.647319}%
\pgfsetfillcolor{currentfill}%
\pgfsetlinewidth{0.000000pt}%
\definecolor{currentstroke}{rgb}{0.000000,0.000000,0.000000}%
\pgfsetstrokecolor{currentstroke}%
\pgfsetdash{}{0pt}%
\pgfpathmoveto{\pgfqpoint{1.066987in}{1.240509in}}%
\pgfpathlineto{\pgfqpoint{1.107412in}{1.248242in}}%
\pgfpathlineto{\pgfqpoint{1.068761in}{1.279983in}}%
\pgfpathlineto{\pgfqpoint{1.028825in}{1.260284in}}%
\pgfpathclose%
\pgfusepath{fill}%
\end{pgfscope}%
\begin{pgfscope}%
\pgfpathrectangle{\pgfqpoint{0.150000in}{0.150000in}}{\pgfqpoint{2.700000in}{1.950000in}}%
\pgfusepath{clip}%
\pgfsetbuttcap%
\pgfsetroundjoin%
\definecolor{currentfill}{rgb}{0.474464,0.539185,0.629795}%
\pgfsetfillcolor{currentfill}%
\pgfsetlinewidth{0.000000pt}%
\definecolor{currentstroke}{rgb}{0.000000,0.000000,0.000000}%
\pgfsetstrokecolor{currentstroke}%
\pgfsetdash{}{0pt}%
\pgfpathmoveto{\pgfqpoint{0.949394in}{1.284449in}}%
\pgfpathlineto{\pgfqpoint{0.990189in}{1.292052in}}%
\pgfpathlineto{\pgfqpoint{0.952272in}{1.311676in}}%
\pgfpathlineto{\pgfqpoint{0.911440in}{1.304148in}}%
\pgfpathclose%
\pgfusepath{fill}%
\end{pgfscope}%
\begin{pgfscope}%
\pgfpathrectangle{\pgfqpoint{0.150000in}{0.150000in}}{\pgfqpoint{2.700000in}{1.950000in}}%
\pgfusepath{clip}%
\pgfsetbuttcap%
\pgfsetroundjoin%
\definecolor{currentfill}{rgb}{0.517999,0.577359,0.660463}%
\pgfsetfillcolor{currentfill}%
\pgfsetlinewidth{0.000000pt}%
\definecolor{currentstroke}{rgb}{0.000000,0.000000,0.000000}%
\pgfsetstrokecolor{currentstroke}%
\pgfsetdash{}{0pt}%
\pgfpathmoveto{\pgfqpoint{1.262714in}{1.196602in}}%
\pgfpathlineto{\pgfqpoint{1.302268in}{1.216452in}}%
\pgfpathlineto{\pgfqpoint{1.263748in}{1.236227in}}%
\pgfpathlineto{\pgfqpoint{1.223842in}{1.228467in}}%
\pgfpathclose%
\pgfusepath{fill}%
\end{pgfscope}%
\begin{pgfscope}%
\pgfpathrectangle{\pgfqpoint{0.150000in}{0.150000in}}{\pgfqpoint{2.700000in}{1.950000in}}%
\pgfusepath{clip}%
\pgfsetbuttcap%
\pgfsetroundjoin%
\definecolor{currentfill}{rgb}{0.586412,0.637347,0.708655}%
\pgfsetfillcolor{currentfill}%
\pgfsetlinewidth{0.000000pt}%
\definecolor{currentstroke}{rgb}{0.000000,0.000000,0.000000}%
\pgfsetstrokecolor{currentstroke}%
\pgfsetdash{}{0pt}%
\pgfpathmoveto{\pgfqpoint{1.302352in}{1.053251in}}%
\pgfpathlineto{\pgfqpoint{1.341742in}{1.073327in}}%
\pgfpathlineto{\pgfqpoint{1.301912in}{1.152728in}}%
\pgfpathlineto{\pgfqpoint{1.262297in}{1.132727in}}%
\pgfpathclose%
\pgfusepath{fill}%
\end{pgfscope}%
\begin{pgfscope}%
\pgfpathrectangle{\pgfqpoint{0.150000in}{0.150000in}}{\pgfqpoint{2.700000in}{1.950000in}}%
\pgfusepath{clip}%
\pgfsetbuttcap%
\pgfsetroundjoin%
\definecolor{currentfill}{rgb}{0.661045,0.702788,0.761229}%
\pgfsetfillcolor{currentfill}%
\pgfsetlinewidth{0.000000pt}%
\definecolor{currentstroke}{rgb}{0.000000,0.000000,0.000000}%
\pgfsetstrokecolor{currentstroke}%
\pgfsetdash{}{0pt}%
\pgfpathmoveto{\pgfqpoint{1.263325in}{0.933845in}}%
\pgfpathlineto{\pgfqpoint{1.302790in}{0.954146in}}%
\pgfpathlineto{\pgfqpoint{1.262812in}{1.033099in}}%
\pgfpathlineto{\pgfqpoint{1.223122in}{1.012871in}}%
\pgfpathclose%
\pgfusepath{fill}%
\end{pgfscope}%
\begin{pgfscope}%
\pgfpathrectangle{\pgfqpoint{0.150000in}{0.150000in}}{\pgfqpoint{2.700000in}{1.950000in}}%
\pgfusepath{clip}%
\pgfsetbuttcap%
\pgfsetroundjoin%
\definecolor{currentfill}{rgb}{0.536657,0.593719,0.673606}%
\pgfsetfillcolor{currentfill}%
\pgfsetlinewidth{0.000000pt}%
\definecolor{currentstroke}{rgb}{0.000000,0.000000,0.000000}%
\pgfsetstrokecolor{currentstroke}%
\pgfsetdash{}{0pt}%
\pgfpathmoveto{\pgfqpoint{1.301912in}{1.152728in}}%
\pgfpathlineto{\pgfqpoint{1.341377in}{1.172653in}}%
\pgfpathlineto{\pgfqpoint{1.302268in}{1.216452in}}%
\pgfpathlineto{\pgfqpoint{1.262714in}{1.196602in}}%
\pgfpathclose%
\pgfusepath{fill}%
\end{pgfscope}%
\begin{pgfscope}%
\pgfpathrectangle{\pgfqpoint{0.150000in}{0.150000in}}{\pgfqpoint{2.700000in}{1.950000in}}%
\pgfusepath{clip}%
\pgfsetbuttcap%
\pgfsetroundjoin%
\definecolor{currentfill}{rgb}{0.437148,0.506464,0.603508}%
\pgfsetfillcolor{currentfill}%
\pgfsetlinewidth{0.000000pt}%
\definecolor{currentstroke}{rgb}{0.000000,0.000000,0.000000}%
\pgfsetstrokecolor{currentstroke}%
\pgfsetdash{}{0pt}%
\pgfpathmoveto{\pgfqpoint{0.831712in}{1.328422in}}%
\pgfpathlineto{\pgfqpoint{0.872879in}{1.335894in}}%
\pgfpathlineto{\pgfqpoint{0.834376in}{1.367593in}}%
\pgfpathlineto{\pgfqpoint{0.793125in}{1.360222in}}%
\pgfpathclose%
\pgfusepath{fill}%
\end{pgfscope}%
\begin{pgfscope}%
\pgfpathrectangle{\pgfqpoint{0.150000in}{0.150000in}}{\pgfqpoint{2.700000in}{1.950000in}}%
\pgfusepath{clip}%
\pgfsetbuttcap%
\pgfsetroundjoin%
\definecolor{currentfill}{rgb}{0.505561,0.566452,0.651700}%
\pgfsetfillcolor{currentfill}%
\pgfsetlinewidth{0.000000pt}%
\definecolor{currentstroke}{rgb}{0.000000,0.000000,0.000000}%
\pgfsetstrokecolor{currentstroke}%
\pgfsetdash{}{0pt}%
\pgfpathmoveto{\pgfqpoint{1.184094in}{1.208618in}}%
\pgfpathlineto{\pgfqpoint{1.223842in}{1.228467in}}%
\pgfpathlineto{\pgfqpoint{1.185426in}{1.248242in}}%
\pgfpathlineto{\pgfqpoint{1.145681in}{1.228467in}}%
\pgfpathclose%
\pgfusepath{fill}%
\end{pgfscope}%
\begin{pgfscope}%
\pgfpathrectangle{\pgfqpoint{0.150000in}{0.150000in}}{\pgfqpoint{2.700000in}{1.950000in}}%
\pgfusepath{clip}%
\pgfsetbuttcap%
\pgfsetroundjoin%
\definecolor{currentfill}{rgb}{0.480683,0.544638,0.634176}%
\pgfsetfillcolor{currentfill}%
\pgfsetlinewidth{0.000000pt}%
\definecolor{currentstroke}{rgb}{0.000000,0.000000,0.000000}%
\pgfsetstrokecolor{currentstroke}%
\pgfsetdash{}{0pt}%
\pgfpathmoveto{\pgfqpoint{0.988116in}{1.252579in}}%
\pgfpathlineto{\pgfqpoint{1.028825in}{1.260284in}}%
\pgfpathlineto{\pgfqpoint{0.990189in}{1.292052in}}%
\pgfpathlineto{\pgfqpoint{0.949394in}{1.284449in}}%
\pgfpathclose%
\pgfusepath{fill}%
\end{pgfscope}%
\begin{pgfscope}%
\pgfpathrectangle{\pgfqpoint{0.150000in}{0.150000in}}{\pgfqpoint{2.700000in}{1.950000in}}%
\pgfusepath{clip}%
\pgfsetbuttcap%
\pgfsetroundjoin%
\definecolor{currentfill}{rgb}{0.499341,0.560999,0.647319}%
\pgfsetfillcolor{currentfill}%
\pgfsetlinewidth{0.000000pt}%
\definecolor{currentstroke}{rgb}{0.000000,0.000000,0.000000}%
\pgfsetstrokecolor{currentstroke}%
\pgfsetdash{}{0pt}%
\pgfpathmoveto{\pgfqpoint{1.105295in}{1.220660in}}%
\pgfpathlineto{\pgfqpoint{1.145681in}{1.228467in}}%
\pgfpathlineto{\pgfqpoint{1.107412in}{1.248242in}}%
\pgfpathlineto{\pgfqpoint{1.066987in}{1.240509in}}%
\pgfpathclose%
\pgfusepath{fill}%
\end{pgfscope}%
\begin{pgfscope}%
\pgfpathrectangle{\pgfqpoint{0.150000in}{0.150000in}}{\pgfqpoint{2.700000in}{1.950000in}}%
\pgfusepath{clip}%
\pgfsetbuttcap%
\pgfsetroundjoin%
\definecolor{currentfill}{rgb}{0.443367,0.511918,0.607889}%
\pgfsetfillcolor{currentfill}%
\pgfsetlinewidth{0.000000pt}%
\definecolor{currentstroke}{rgb}{0.000000,0.000000,0.000000}%
\pgfsetstrokecolor{currentstroke}%
\pgfsetdash{}{0pt}%
\pgfpathmoveto{\pgfqpoint{0.869602in}{1.308725in}}%
\pgfpathlineto{\pgfqpoint{0.911440in}{1.304148in}}%
\pgfpathlineto{\pgfqpoint{0.872879in}{1.335894in}}%
\pgfpathlineto{\pgfqpoint{0.831712in}{1.328422in}}%
\pgfpathclose%
\pgfusepath{fill}%
\end{pgfscope}%
\begin{pgfscope}%
\pgfpathrectangle{\pgfqpoint{0.150000in}{0.150000in}}{\pgfqpoint{2.700000in}{1.950000in}}%
\pgfusepath{clip}%
\pgfsetbuttcap%
\pgfsetroundjoin%
\definecolor{currentfill}{rgb}{0.517999,0.577359,0.660463}%
\pgfsetfillcolor{currentfill}%
\pgfsetlinewidth{0.000000pt}%
\definecolor{currentstroke}{rgb}{0.000000,0.000000,0.000000}%
\pgfsetstrokecolor{currentstroke}%
\pgfsetdash{}{0pt}%
\pgfpathmoveto{\pgfqpoint{1.223010in}{1.176677in}}%
\pgfpathlineto{\pgfqpoint{1.262714in}{1.196602in}}%
\pgfpathlineto{\pgfqpoint{1.223842in}{1.228467in}}%
\pgfpathlineto{\pgfqpoint{1.184094in}{1.208618in}}%
\pgfpathclose%
\pgfusepath{fill}%
\end{pgfscope}%
\begin{pgfscope}%
\pgfpathrectangle{\pgfqpoint{0.150000in}{0.150000in}}{\pgfqpoint{2.700000in}{1.950000in}}%
\pgfusepath{clip}%
\pgfsetbuttcap%
\pgfsetroundjoin%
\definecolor{currentfill}{rgb}{0.586412,0.637347,0.708655}%
\pgfsetfillcolor{currentfill}%
\pgfsetlinewidth{0.000000pt}%
\definecolor{currentstroke}{rgb}{0.000000,0.000000,0.000000}%
\pgfsetstrokecolor{currentstroke}%
\pgfsetdash{}{0pt}%
\pgfpathmoveto{\pgfqpoint{1.262812in}{1.033099in}}%
\pgfpathlineto{\pgfqpoint{1.302352in}{1.053251in}}%
\pgfpathlineto{\pgfqpoint{1.262297in}{1.132727in}}%
\pgfpathlineto{\pgfqpoint{1.222531in}{1.112650in}}%
\pgfpathclose%
\pgfusepath{fill}%
\end{pgfscope}%
\begin{pgfscope}%
\pgfpathrectangle{\pgfqpoint{0.150000in}{0.150000in}}{\pgfqpoint{2.700000in}{1.950000in}}%
\pgfusepath{clip}%
\pgfsetbuttcap%
\pgfsetroundjoin%
\definecolor{currentfill}{rgb}{0.661045,0.702788,0.761229}%
\pgfsetfillcolor{currentfill}%
\pgfsetlinewidth{0.000000pt}%
\definecolor{currentstroke}{rgb}{0.000000,0.000000,0.000000}%
\pgfsetstrokecolor{currentstroke}%
\pgfsetdash{}{0pt}%
\pgfpathmoveto{\pgfqpoint{1.223710in}{0.913466in}}%
\pgfpathlineto{\pgfqpoint{1.263325in}{0.933845in}}%
\pgfpathlineto{\pgfqpoint{1.223122in}{1.012871in}}%
\pgfpathlineto{\pgfqpoint{1.183281in}{0.992565in}}%
\pgfpathclose%
\pgfusepath{fill}%
\end{pgfscope}%
\begin{pgfscope}%
\pgfpathrectangle{\pgfqpoint{0.150000in}{0.150000in}}{\pgfqpoint{2.700000in}{1.950000in}}%
\pgfusepath{clip}%
\pgfsetbuttcap%
\pgfsetroundjoin%
\definecolor{currentfill}{rgb}{0.536657,0.593719,0.673606}%
\pgfsetfillcolor{currentfill}%
\pgfsetlinewidth{0.000000pt}%
\definecolor{currentstroke}{rgb}{0.000000,0.000000,0.000000}%
\pgfsetstrokecolor{currentstroke}%
\pgfsetdash{}{0pt}%
\pgfpathmoveto{\pgfqpoint{1.262297in}{1.132727in}}%
\pgfpathlineto{\pgfqpoint{1.301912in}{1.152728in}}%
\pgfpathlineto{\pgfqpoint{1.262714in}{1.196602in}}%
\pgfpathlineto{\pgfqpoint{1.223010in}{1.176677in}}%
\pgfpathclose%
\pgfusepath{fill}%
\end{pgfscope}%
\begin{pgfscope}%
\pgfpathrectangle{\pgfqpoint{0.150000in}{0.150000in}}{\pgfqpoint{2.700000in}{1.950000in}}%
\pgfusepath{clip}%
\pgfsetbuttcap%
\pgfsetroundjoin%
\definecolor{currentfill}{rgb}{0.486903,0.550092,0.638557}%
\pgfsetfillcolor{currentfill}%
\pgfsetlinewidth{0.000000pt}%
\definecolor{currentstroke}{rgb}{0.000000,0.000000,0.000000}%
\pgfsetstrokecolor{currentstroke}%
\pgfsetdash{}{0pt}%
\pgfpathmoveto{\pgfqpoint{1.026317in}{1.232730in}}%
\pgfpathlineto{\pgfqpoint{1.066987in}{1.240509in}}%
\pgfpathlineto{\pgfqpoint{1.028825in}{1.260284in}}%
\pgfpathlineto{\pgfqpoint{0.988116in}{1.252579in}}%
\pgfpathclose%
\pgfusepath{fill}%
\end{pgfscope}%
\begin{pgfscope}%
\pgfpathrectangle{\pgfqpoint{0.150000in}{0.150000in}}{\pgfqpoint{2.700000in}{1.950000in}}%
\pgfusepath{clip}%
\pgfsetbuttcap%
\pgfsetroundjoin%
\definecolor{currentfill}{rgb}{0.505561,0.566452,0.651700}%
\pgfsetfillcolor{currentfill}%
\pgfsetlinewidth{0.000000pt}%
\definecolor{currentstroke}{rgb}{0.000000,0.000000,0.000000}%
\pgfsetstrokecolor{currentstroke}%
\pgfsetdash{}{0pt}%
\pgfpathmoveto{\pgfqpoint{1.144195in}{1.188693in}}%
\pgfpathlineto{\pgfqpoint{1.184094in}{1.208618in}}%
\pgfpathlineto{\pgfqpoint{1.145681in}{1.228467in}}%
\pgfpathlineto{\pgfqpoint{1.105295in}{1.220660in}}%
\pgfpathclose%
\pgfusepath{fill}%
\end{pgfscope}%
\begin{pgfscope}%
\pgfpathrectangle{\pgfqpoint{0.150000in}{0.150000in}}{\pgfqpoint{2.700000in}{1.950000in}}%
\pgfusepath{clip}%
\pgfsetbuttcap%
\pgfsetroundjoin%
\definecolor{currentfill}{rgb}{0.455806,0.522825,0.616651}%
\pgfsetfillcolor{currentfill}%
\pgfsetlinewidth{0.000000pt}%
\definecolor{currentstroke}{rgb}{0.000000,0.000000,0.000000}%
\pgfsetstrokecolor{currentstroke}%
\pgfsetdash{}{0pt}%
\pgfpathmoveto{\pgfqpoint{0.908351in}{1.276801in}}%
\pgfpathlineto{\pgfqpoint{0.949394in}{1.284449in}}%
\pgfpathlineto{\pgfqpoint{0.911440in}{1.304148in}}%
\pgfpathlineto{\pgfqpoint{0.869602in}{1.308725in}}%
\pgfpathclose%
\pgfusepath{fill}%
\end{pgfscope}%
\begin{pgfscope}%
\pgfpathrectangle{\pgfqpoint{0.150000in}{0.150000in}}{\pgfqpoint{2.700000in}{1.950000in}}%
\pgfusepath{clip}%
\pgfsetbuttcap%
\pgfsetroundjoin%
\definecolor{currentfill}{rgb}{0.586412,0.637347,0.708655}%
\pgfsetfillcolor{currentfill}%
\pgfsetlinewidth{0.000000pt}%
\definecolor{currentstroke}{rgb}{0.000000,0.000000,0.000000}%
\pgfsetstrokecolor{currentstroke}%
\pgfsetdash{}{0pt}%
\pgfpathmoveto{\pgfqpoint{1.223122in}{1.012871in}}%
\pgfpathlineto{\pgfqpoint{1.262812in}{1.033099in}}%
\pgfpathlineto{\pgfqpoint{1.222531in}{1.112650in}}%
\pgfpathlineto{\pgfqpoint{1.182614in}{1.092496in}}%
\pgfpathclose%
\pgfusepath{fill}%
\end{pgfscope}%
\begin{pgfscope}%
\pgfpathrectangle{\pgfqpoint{0.150000in}{0.150000in}}{\pgfqpoint{2.700000in}{1.950000in}}%
\pgfusepath{clip}%
\pgfsetbuttcap%
\pgfsetroundjoin%
\definecolor{currentfill}{rgb}{0.486903,0.550092,0.638557}%
\pgfsetfillcolor{currentfill}%
\pgfsetlinewidth{0.000000pt}%
\definecolor{currentstroke}{rgb}{0.000000,0.000000,0.000000}%
\pgfsetstrokecolor{currentstroke}%
\pgfsetdash{}{0pt}%
\pgfpathmoveto{\pgfqpoint{1.065201in}{1.200735in}}%
\pgfpathlineto{\pgfqpoint{1.105295in}{1.220660in}}%
\pgfpathlineto{\pgfqpoint{1.066987in}{1.240509in}}%
\pgfpathlineto{\pgfqpoint{1.026317in}{1.232730in}}%
\pgfpathclose%
\pgfusepath{fill}%
\end{pgfscope}%
\begin{pgfscope}%
\pgfpathrectangle{\pgfqpoint{0.150000in}{0.150000in}}{\pgfqpoint{2.700000in}{1.950000in}}%
\pgfusepath{clip}%
\pgfsetbuttcap%
\pgfsetroundjoin%
\definecolor{currentfill}{rgb}{0.462025,0.528278,0.621032}%
\pgfsetfillcolor{currentfill}%
\pgfsetlinewidth{0.000000pt}%
\definecolor{currentstroke}{rgb}{0.000000,0.000000,0.000000}%
\pgfsetstrokecolor{currentstroke}%
\pgfsetdash{}{0pt}%
\pgfpathmoveto{\pgfqpoint{0.947159in}{1.244828in}}%
\pgfpathlineto{\pgfqpoint{0.988116in}{1.252579in}}%
\pgfpathlineto{\pgfqpoint{0.949394in}{1.284449in}}%
\pgfpathlineto{\pgfqpoint{0.908351in}{1.276801in}}%
\pgfpathclose%
\pgfusepath{fill}%
\end{pgfscope}%
\begin{pgfscope}%
\pgfpathrectangle{\pgfqpoint{0.150000in}{0.150000in}}{\pgfqpoint{2.700000in}{1.950000in}}%
\pgfusepath{clip}%
\pgfsetbuttcap%
\pgfsetroundjoin%
\definecolor{currentfill}{rgb}{0.511780,0.571906,0.656081}%
\pgfsetfillcolor{currentfill}%
\pgfsetlinewidth{0.000000pt}%
\definecolor{currentstroke}{rgb}{0.000000,0.000000,0.000000}%
\pgfsetstrokecolor{currentstroke}%
\pgfsetdash{}{0pt}%
\pgfpathmoveto{\pgfqpoint{1.182752in}{1.168692in}}%
\pgfpathlineto{\pgfqpoint{1.223010in}{1.176677in}}%
\pgfpathlineto{\pgfqpoint{1.184094in}{1.208618in}}%
\pgfpathlineto{\pgfqpoint{1.144195in}{1.188693in}}%
\pgfpathclose%
\pgfusepath{fill}%
\end{pgfscope}%
\begin{pgfscope}%
\pgfpathrectangle{\pgfqpoint{0.150000in}{0.150000in}}{\pgfqpoint{2.700000in}{1.950000in}}%
\pgfusepath{clip}%
\pgfsetbuttcap%
\pgfsetroundjoin%
\definecolor{currentfill}{rgb}{0.530438,0.588266,0.669225}%
\pgfsetfillcolor{currentfill}%
\pgfsetlinewidth{0.000000pt}%
\definecolor{currentstroke}{rgb}{0.000000,0.000000,0.000000}%
\pgfsetstrokecolor{currentstroke}%
\pgfsetdash{}{0pt}%
\pgfpathmoveto{\pgfqpoint{1.222531in}{1.112650in}}%
\pgfpathlineto{\pgfqpoint{1.262297in}{1.132727in}}%
\pgfpathlineto{\pgfqpoint{1.223010in}{1.176677in}}%
\pgfpathlineto{\pgfqpoint{1.182752in}{1.168692in}}%
\pgfpathclose%
\pgfusepath{fill}%
\end{pgfscope}%
\begin{pgfscope}%
\pgfpathrectangle{\pgfqpoint{0.150000in}{0.150000in}}{\pgfqpoint{2.700000in}{1.950000in}}%
\pgfusepath{clip}%
\pgfsetbuttcap%
\pgfsetroundjoin%
\definecolor{currentfill}{rgb}{0.499341,0.560999,0.647319}%
\pgfsetfillcolor{currentfill}%
\pgfsetlinewidth{0.000000pt}%
\definecolor{currentstroke}{rgb}{0.000000,0.000000,0.000000}%
\pgfsetstrokecolor{currentstroke}%
\pgfsetdash{}{0pt}%
\pgfpathmoveto{\pgfqpoint{1.104144in}{1.168692in}}%
\pgfpathlineto{\pgfqpoint{1.144195in}{1.188693in}}%
\pgfpathlineto{\pgfqpoint{1.105295in}{1.220660in}}%
\pgfpathlineto{\pgfqpoint{1.065201in}{1.200735in}}%
\pgfpathclose%
\pgfusepath{fill}%
\end{pgfscope}%
\begin{pgfscope}%
\pgfpathrectangle{\pgfqpoint{0.150000in}{0.150000in}}{\pgfqpoint{2.700000in}{1.950000in}}%
\pgfusepath{clip}%
\pgfsetbuttcap%
\pgfsetroundjoin%
\definecolor{currentfill}{rgb}{0.474464,0.539185,0.629795}%
\pgfsetfillcolor{currentfill}%
\pgfsetlinewidth{0.000000pt}%
\definecolor{currentstroke}{rgb}{0.000000,0.000000,0.000000}%
\pgfsetstrokecolor{currentstroke}%
\pgfsetdash{}{0pt}%
\pgfpathmoveto{\pgfqpoint{0.986026in}{1.212806in}}%
\pgfpathlineto{\pgfqpoint{1.026317in}{1.232730in}}%
\pgfpathlineto{\pgfqpoint{0.988116in}{1.252579in}}%
\pgfpathlineto{\pgfqpoint{0.947159in}{1.244828in}}%
\pgfpathclose%
\pgfusepath{fill}%
\end{pgfscope}%
\begin{pgfscope}%
\pgfpathrectangle{\pgfqpoint{0.150000in}{0.150000in}}{\pgfqpoint{2.700000in}{1.950000in}}%
\pgfusepath{clip}%
\pgfsetbuttcap%
\pgfsetroundjoin%
\definecolor{currentfill}{rgb}{0.586412,0.637347,0.708655}%
\pgfsetfillcolor{currentfill}%
\pgfsetlinewidth{0.000000pt}%
\definecolor{currentstroke}{rgb}{0.000000,0.000000,0.000000}%
\pgfsetstrokecolor{currentstroke}%
\pgfsetdash{}{0pt}%
\pgfpathmoveto{\pgfqpoint{1.183281in}{0.992565in}}%
\pgfpathlineto{\pgfqpoint{1.223122in}{1.012871in}}%
\pgfpathlineto{\pgfqpoint{1.182614in}{1.092496in}}%
\pgfpathlineto{\pgfqpoint{1.142544in}{1.072266in}}%
\pgfpathclose%
\pgfusepath{fill}%
\end{pgfscope}%
\begin{pgfscope}%
\pgfpathrectangle{\pgfqpoint{0.150000in}{0.150000in}}{\pgfqpoint{2.700000in}{1.950000in}}%
\pgfusepath{clip}%
\pgfsetbuttcap%
\pgfsetroundjoin%
\definecolor{currentfill}{rgb}{0.486903,0.550092,0.638557}%
\pgfsetfillcolor{currentfill}%
\pgfsetlinewidth{0.000000pt}%
\definecolor{currentstroke}{rgb}{0.000000,0.000000,0.000000}%
\pgfsetstrokecolor{currentstroke}%
\pgfsetdash{}{0pt}%
\pgfpathmoveto{\pgfqpoint{1.024954in}{1.180735in}}%
\pgfpathlineto{\pgfqpoint{1.065201in}{1.200735in}}%
\pgfpathlineto{\pgfqpoint{1.026317in}{1.232730in}}%
\pgfpathlineto{\pgfqpoint{0.986026in}{1.212806in}}%
\pgfpathclose%
\pgfusepath{fill}%
\end{pgfscope}%
\begin{pgfscope}%
\pgfpathrectangle{\pgfqpoint{0.150000in}{0.150000in}}{\pgfqpoint{2.700000in}{1.950000in}}%
\pgfusepath{clip}%
\pgfsetbuttcap%
\pgfsetroundjoin%
\definecolor{currentfill}{rgb}{0.505561,0.566452,0.651700}%
\pgfsetfillcolor{currentfill}%
\pgfsetlinewidth{0.000000pt}%
\definecolor{currentstroke}{rgb}{0.000000,0.000000,0.000000}%
\pgfsetstrokecolor{currentstroke}%
\pgfsetdash{}{0pt}%
\pgfpathmoveto{\pgfqpoint{1.142698in}{1.148614in}}%
\pgfpathlineto{\pgfqpoint{1.182752in}{1.168692in}}%
\pgfpathlineto{\pgfqpoint{1.144195in}{1.188693in}}%
\pgfpathlineto{\pgfqpoint{1.104144in}{1.168692in}}%
\pgfpathclose%
\pgfusepath{fill}%
\end{pgfscope}%
\begin{pgfscope}%
\pgfpathrectangle{\pgfqpoint{0.150000in}{0.150000in}}{\pgfqpoint{2.700000in}{1.950000in}}%
\pgfusepath{clip}%
\pgfsetbuttcap%
\pgfsetroundjoin%
\definecolor{currentfill}{rgb}{0.530438,0.588266,0.669225}%
\pgfsetfillcolor{currentfill}%
\pgfsetlinewidth{0.000000pt}%
\definecolor{currentstroke}{rgb}{0.000000,0.000000,0.000000}%
\pgfsetstrokecolor{currentstroke}%
\pgfsetdash{}{0pt}%
\pgfpathmoveto{\pgfqpoint{1.182614in}{1.092496in}}%
\pgfpathlineto{\pgfqpoint{1.222531in}{1.112650in}}%
\pgfpathlineto{\pgfqpoint{1.182752in}{1.168692in}}%
\pgfpathlineto{\pgfqpoint{1.142698in}{1.148614in}}%
\pgfpathclose%
\pgfusepath{fill}%
\end{pgfscope}%
\begin{pgfscope}%
\pgfpathrectangle{\pgfqpoint{0.150000in}{0.150000in}}{\pgfqpoint{2.700000in}{1.950000in}}%
\pgfusepath{clip}%
\pgfsetbuttcap%
\pgfsetroundjoin%
\definecolor{currentfill}{rgb}{0.499341,0.560999,0.647319}%
\pgfsetfillcolor{currentfill}%
\pgfsetlinewidth{0.000000pt}%
\definecolor{currentstroke}{rgb}{0.000000,0.000000,0.000000}%
\pgfsetstrokecolor{currentstroke}%
\pgfsetdash{}{0pt}%
\pgfpathmoveto{\pgfqpoint{1.063400in}{1.160657in}}%
\pgfpathlineto{\pgfqpoint{1.104144in}{1.168692in}}%
\pgfpathlineto{\pgfqpoint{1.065201in}{1.200735in}}%
\pgfpathlineto{\pgfqpoint{1.024954in}{1.180735in}}%
\pgfpathclose%
\pgfusepath{fill}%
\end{pgfscope}%
\begin{pgfscope}%
\pgfpathrectangle{\pgfqpoint{0.150000in}{0.150000in}}{\pgfqpoint{2.700000in}{1.950000in}}%
\pgfusepath{clip}%
\pgfsetbuttcap%
\pgfsetroundjoin%
\definecolor{currentfill}{rgb}{0.530438,0.588266,0.669225}%
\pgfsetfillcolor{currentfill}%
\pgfsetlinewidth{0.000000pt}%
\definecolor{currentstroke}{rgb}{0.000000,0.000000,0.000000}%
\pgfsetstrokecolor{currentstroke}%
\pgfsetdash{}{0pt}%
\pgfpathmoveto{\pgfqpoint{1.142544in}{1.072266in}}%
\pgfpathlineto{\pgfqpoint{1.182614in}{1.092496in}}%
\pgfpathlineto{\pgfqpoint{1.142698in}{1.148614in}}%
\pgfpathlineto{\pgfqpoint{1.102491in}{1.128460in}}%
\pgfpathclose%
\pgfusepath{fill}%
\end{pgfscope}%
\begin{pgfscope}%
\pgfpathrectangle{\pgfqpoint{0.150000in}{0.150000in}}{\pgfqpoint{2.700000in}{1.950000in}}%
\pgfusepath{clip}%
\pgfsetbuttcap%
\pgfsetroundjoin%
\definecolor{currentfill}{rgb}{0.505561,0.566452,0.651700}%
\pgfsetfillcolor{currentfill}%
\pgfsetlinewidth{0.000000pt}%
\definecolor{currentstroke}{rgb}{0.000000,0.000000,0.000000}%
\pgfsetstrokecolor{currentstroke}%
\pgfsetdash{}{0pt}%
\pgfpathmoveto{\pgfqpoint{1.102491in}{1.128460in}}%
\pgfpathlineto{\pgfqpoint{1.142698in}{1.148614in}}%
\pgfpathlineto{\pgfqpoint{1.104144in}{1.168692in}}%
\pgfpathlineto{\pgfqpoint{1.063400in}{1.160657in}}%
\pgfpathclose%
\pgfusepath{fill}%
\end{pgfscope}%
\end{pgfpicture}%
\makeatother%
\endgroup%
}

            \subbottom[\label{fig:parameterised-incompetent-games-e}]%
                {%% Creator: Matplotlib, PGF backend
%%
%% To include the figure in your LaTeX document, write
%%   \input{<filename>.pgf}
%%
%% Make sure the required packages are loaded in your preamble
%%   \usepackage{pgf}
%%
%% Figures using additional raster images can only be included by \input if
%% they are in the same directory as the main LaTeX file. For loading figures
%% from other directories you can use the `import` package
%%   \usepackage{import}
%% and then include the figures with
%%   \import{<path to file>}{<filename>.pgf}
%%
%% Matplotlib used the following preamble
%%   \usepackage{fontspec}
%%   \setmainfont{DejaVuSerif.ttf}[Path=C:/Users/Thomas/anaconda3/lib/site-packages/matplotlib/mpl-data/fonts/ttf/]
%%   \setsansfont{DejaVuSans.ttf}[Path=C:/Users/Thomas/anaconda3/lib/site-packages/matplotlib/mpl-data/fonts/ttf/]
%%   \setmonofont{DejaVuSansMono.ttf}[Path=C:/Users/Thomas/anaconda3/lib/site-packages/matplotlib/mpl-data/fonts/ttf/]
%%
\begingroup%
\makeatletter%
\begin{pgfpicture}%
\pgfpathrectangle{\pgfpointorigin}{\pgfqpoint{3.000000in}{2.250000in}}%
\pgfusepath{use as bounding box, clip}%
\begin{pgfscope}%
\pgfsetbuttcap%
\pgfsetmiterjoin%
\definecolor{currentfill}{rgb}{1.000000,1.000000,1.000000}%
\pgfsetfillcolor{currentfill}%
\pgfsetlinewidth{0.000000pt}%
\definecolor{currentstroke}{rgb}{1.000000,1.000000,1.000000}%
\pgfsetstrokecolor{currentstroke}%
\pgfsetdash{}{0pt}%
\pgfpathmoveto{\pgfqpoint{0.000000in}{0.000000in}}%
\pgfpathlineto{\pgfqpoint{3.000000in}{0.000000in}}%
\pgfpathlineto{\pgfqpoint{3.000000in}{2.250000in}}%
\pgfpathlineto{\pgfqpoint{0.000000in}{2.250000in}}%
\pgfpathclose%
\pgfusepath{fill}%
\end{pgfscope}%
\begin{pgfscope}%
\pgfsetbuttcap%
\pgfsetmiterjoin%
\definecolor{currentfill}{rgb}{1.000000,1.000000,1.000000}%
\pgfsetfillcolor{currentfill}%
\pgfsetlinewidth{0.000000pt}%
\definecolor{currentstroke}{rgb}{0.000000,0.000000,0.000000}%
\pgfsetstrokecolor{currentstroke}%
\pgfsetstrokeopacity{0.000000}%
\pgfsetdash{}{0pt}%
\pgfpathmoveto{\pgfqpoint{0.150000in}{0.150000in}}%
\pgfpathlineto{\pgfqpoint{2.850000in}{0.150000in}}%
\pgfpathlineto{\pgfqpoint{2.850000in}{2.100000in}}%
\pgfpathlineto{\pgfqpoint{0.150000in}{2.100000in}}%
\pgfpathclose%
\pgfusepath{fill}%
\end{pgfscope}%
\begin{pgfscope}%
\pgfsetbuttcap%
\pgfsetmiterjoin%
\definecolor{currentfill}{rgb}{0.950000,0.950000,0.950000}%
\pgfsetfillcolor{currentfill}%
\pgfsetfillopacity{0.500000}%
\pgfsetlinewidth{1.003750pt}%
\definecolor{currentstroke}{rgb}{0.950000,0.950000,0.950000}%
\pgfsetstrokecolor{currentstroke}%
\pgfsetstrokeopacity{0.500000}%
\pgfsetdash{}{0pt}%
\pgfpathmoveto{\pgfqpoint{2.573296in}{0.776948in}}%
\pgfpathlineto{\pgfqpoint{1.536486in}{1.299017in}}%
\pgfpathlineto{\pgfqpoint{1.536486in}{2.074448in}}%
\pgfpathlineto{\pgfqpoint{2.652584in}{1.554387in}}%
\pgfusepath{stroke,fill}%
\end{pgfscope}%
\begin{pgfscope}%
\pgfsetbuttcap%
\pgfsetmiterjoin%
\definecolor{currentfill}{rgb}{0.900000,0.900000,0.900000}%
\pgfsetfillcolor{currentfill}%
\pgfsetfillopacity{0.500000}%
\pgfsetlinewidth{1.003750pt}%
\definecolor{currentstroke}{rgb}{0.900000,0.900000,0.900000}%
\pgfsetstrokecolor{currentstroke}%
\pgfsetstrokeopacity{0.500000}%
\pgfsetdash{}{0pt}%
\pgfpathmoveto{\pgfqpoint{0.499677in}{0.776948in}}%
\pgfpathlineto{\pgfqpoint{1.536486in}{1.299017in}}%
\pgfpathlineto{\pgfqpoint{1.536486in}{2.074448in}}%
\pgfpathlineto{\pgfqpoint{0.420389in}{1.554387in}}%
\pgfusepath{stroke,fill}%
\end{pgfscope}%
\begin{pgfscope}%
\pgfsetbuttcap%
\pgfsetmiterjoin%
\definecolor{currentfill}{rgb}{0.925000,0.925000,0.925000}%
\pgfsetfillcolor{currentfill}%
\pgfsetfillopacity{0.500000}%
\pgfsetlinewidth{1.003750pt}%
\definecolor{currentstroke}{rgb}{0.925000,0.925000,0.925000}%
\pgfsetstrokecolor{currentstroke}%
\pgfsetstrokeopacity{0.500000}%
\pgfsetdash{}{0pt}%
\pgfpathmoveto{\pgfqpoint{1.536486in}{0.199655in}}%
\pgfpathlineto{\pgfqpoint{2.573296in}{0.776948in}}%
\pgfpathlineto{\pgfqpoint{1.536486in}{1.299017in}}%
\pgfpathlineto{\pgfqpoint{0.499677in}{0.776948in}}%
\pgfusepath{stroke,fill}%
\end{pgfscope}%
\begin{pgfscope}%
\pgfsetrectcap%
\pgfsetroundjoin%
\pgfsetlinewidth{0.803000pt}%
\definecolor{currentstroke}{rgb}{0.000000,0.000000,0.000000}%
\pgfsetstrokecolor{currentstroke}%
\pgfsetdash{}{0pt}%
\pgfpathmoveto{\pgfqpoint{2.573296in}{0.776948in}}%
\pgfpathlineto{\pgfqpoint{1.536486in}{0.199655in}}%
\pgfusepath{stroke}%
\end{pgfscope}%
\begin{pgfscope}%
\definecolor{textcolor}{rgb}{0.000000,0.000000,0.000000}%
\pgfsetstrokecolor{textcolor}%
\pgfsetfillcolor{textcolor}%
\pgftext[x=2.017747in,y=0.045475in,left,base,rotate=29.108966]{\color{textcolor}\sffamily\fontsize{8.000000}{9.600000}\selectfont Player 2 (\(\displaystyle \mu\))}%
\end{pgfscope}%
\begin{pgfscope}%
\pgfsetbuttcap%
\pgfsetroundjoin%
\pgfsetlinewidth{0.803000pt}%
\definecolor{currentstroke}{rgb}{0.690196,0.690196,0.690196}%
\pgfsetstrokecolor{currentstroke}%
\pgfsetdash{}{0pt}%
\pgfpathmoveto{\pgfqpoint{1.605722in}{0.238205in}}%
\pgfpathlineto{\pgfqpoint{0.568749in}{0.811728in}}%
\pgfpathlineto{\pgfqpoint{0.494997in}{1.589151in}}%
\pgfusepath{stroke}%
\end{pgfscope}%
\begin{pgfscope}%
\pgfsetbuttcap%
\pgfsetroundjoin%
\pgfsetlinewidth{0.803000pt}%
\definecolor{currentstroke}{rgb}{0.690196,0.690196,0.690196}%
\pgfsetstrokecolor{currentstroke}%
\pgfsetdash{}{0pt}%
\pgfpathmoveto{\pgfqpoint{1.793262in}{0.342627in}}%
\pgfpathlineto{\pgfqpoint{0.755965in}{0.905998in}}%
\pgfpathlineto{\pgfqpoint{0.697035in}{1.683294in}}%
\pgfusepath{stroke}%
\end{pgfscope}%
\begin{pgfscope}%
\pgfsetbuttcap%
\pgfsetroundjoin%
\pgfsetlinewidth{0.803000pt}%
\definecolor{currentstroke}{rgb}{0.690196,0.690196,0.690196}%
\pgfsetstrokecolor{currentstroke}%
\pgfsetdash{}{0pt}%
\pgfpathmoveto{\pgfqpoint{1.977414in}{0.445162in}}%
\pgfpathlineto{\pgfqpoint{0.939964in}{0.998647in}}%
\pgfpathlineto{\pgfqpoint{0.895342in}{1.775698in}}%
\pgfusepath{stroke}%
\end{pgfscope}%
\begin{pgfscope}%
\pgfsetbuttcap%
\pgfsetroundjoin%
\pgfsetlinewidth{0.803000pt}%
\definecolor{currentstroke}{rgb}{0.690196,0.690196,0.690196}%
\pgfsetstrokecolor{currentstroke}%
\pgfsetdash{}{0pt}%
\pgfpathmoveto{\pgfqpoint{2.158267in}{0.545861in}}%
\pgfpathlineto{\pgfqpoint{1.120829in}{1.089719in}}%
\pgfpathlineto{\pgfqpoint{1.090021in}{1.866411in}}%
\pgfusepath{stroke}%
\end{pgfscope}%
\begin{pgfscope}%
\pgfsetbuttcap%
\pgfsetroundjoin%
\pgfsetlinewidth{0.803000pt}%
\definecolor{currentstroke}{rgb}{0.690196,0.690196,0.690196}%
\pgfsetstrokecolor{currentstroke}%
\pgfsetdash{}{0pt}%
\pgfpathmoveto{\pgfqpoint{2.335912in}{0.644773in}}%
\pgfpathlineto{\pgfqpoint{1.298639in}{1.179253in}}%
\pgfpathlineto{\pgfqpoint{1.281170in}{1.955480in}}%
\pgfusepath{stroke}%
\end{pgfscope}%
\begin{pgfscope}%
\pgfsetbuttcap%
\pgfsetroundjoin%
\pgfsetlinewidth{0.803000pt}%
\definecolor{currentstroke}{rgb}{0.690196,0.690196,0.690196}%
\pgfsetstrokecolor{currentstroke}%
\pgfsetdash{}{0pt}%
\pgfpathmoveto{\pgfqpoint{2.510430in}{0.741945in}}%
\pgfpathlineto{\pgfqpoint{1.473472in}{1.267287in}}%
\pgfpathlineto{\pgfqpoint{1.468885in}{2.042948in}}%
\pgfusepath{stroke}%
\end{pgfscope}%
\begin{pgfscope}%
\pgfsetrectcap%
\pgfsetroundjoin%
\pgfsetlinewidth{0.803000pt}%
\definecolor{currentstroke}{rgb}{0.000000,0.000000,0.000000}%
\pgfsetstrokecolor{currentstroke}%
\pgfsetdash{}{0pt}%
\pgfpathmoveto{\pgfqpoint{1.596992in}{0.243033in}}%
\pgfpathlineto{\pgfqpoint{1.623203in}{0.228537in}}%
\pgfusepath{stroke}%
\end{pgfscope}%
\begin{pgfscope}%
\definecolor{textcolor}{rgb}{0.000000,0.000000,0.000000}%
\pgfsetstrokecolor{textcolor}%
\pgfsetfillcolor{textcolor}%
\pgftext[x=1.680378in,y=0.147403in,,top]{\color{textcolor}\sffamily\fontsize{6.000000}{7.200000}\selectfont \(\displaystyle 0.0\)}%
\end{pgfscope}%
\begin{pgfscope}%
\pgfsetrectcap%
\pgfsetroundjoin%
\pgfsetlinewidth{0.803000pt}%
\definecolor{currentstroke}{rgb}{0.000000,0.000000,0.000000}%
\pgfsetstrokecolor{currentstroke}%
\pgfsetdash{}{0pt}%
\pgfpathmoveto{\pgfqpoint{1.784534in}{0.347367in}}%
\pgfpathlineto{\pgfqpoint{1.810740in}{0.333134in}}%
\pgfusepath{stroke}%
\end{pgfscope}%
\begin{pgfscope}%
\definecolor{textcolor}{rgb}{0.000000,0.000000,0.000000}%
\pgfsetstrokecolor{textcolor}%
\pgfsetfillcolor{textcolor}%
\pgftext[x=1.866959in,y=0.252496in,,top]{\color{textcolor}\sffamily\fontsize{6.000000}{7.200000}\selectfont \(\displaystyle 0.2\)}%
\end{pgfscope}%
\begin{pgfscope}%
\pgfsetrectcap%
\pgfsetroundjoin%
\pgfsetlinewidth{0.803000pt}%
\definecolor{currentstroke}{rgb}{0.000000,0.000000,0.000000}%
\pgfsetstrokecolor{currentstroke}%
\pgfsetdash{}{0pt}%
\pgfpathmoveto{\pgfqpoint{1.968688in}{0.449817in}}%
\pgfpathlineto{\pgfqpoint{1.994886in}{0.435840in}}%
\pgfusepath{stroke}%
\end{pgfscope}%
\begin{pgfscope}%
\definecolor{textcolor}{rgb}{0.000000,0.000000,0.000000}%
\pgfsetstrokecolor{textcolor}%
\pgfsetfillcolor{textcolor}%
\pgftext[x=2.050175in,y=0.355693in,,top]{\color{textcolor}\sffamily\fontsize{6.000000}{7.200000}\selectfont \(\displaystyle 0.4\)}%
\end{pgfscope}%
\begin{pgfscope}%
\pgfsetrectcap%
\pgfsetroundjoin%
\pgfsetlinewidth{0.803000pt}%
\definecolor{currentstroke}{rgb}{0.000000,0.000000,0.000000}%
\pgfsetstrokecolor{currentstroke}%
\pgfsetdash{}{0pt}%
\pgfpathmoveto{\pgfqpoint{2.149546in}{0.550433in}}%
\pgfpathlineto{\pgfqpoint{2.175732in}{0.536706in}}%
\pgfusepath{stroke}%
\end{pgfscope}%
\begin{pgfscope}%
\definecolor{textcolor}{rgb}{0.000000,0.000000,0.000000}%
\pgfsetstrokecolor{textcolor}%
\pgfsetfillcolor{textcolor}%
\pgftext[x=2.230114in,y=0.457045in,,top]{\color{textcolor}\sffamily\fontsize{6.000000}{7.200000}\selectfont \(\displaystyle 0.6\)}%
\end{pgfscope}%
\begin{pgfscope}%
\pgfsetrectcap%
\pgfsetroundjoin%
\pgfsetlinewidth{0.803000pt}%
\definecolor{currentstroke}{rgb}{0.000000,0.000000,0.000000}%
\pgfsetstrokecolor{currentstroke}%
\pgfsetdash{}{0pt}%
\pgfpathmoveto{\pgfqpoint{2.327195in}{0.649264in}}%
\pgfpathlineto{\pgfqpoint{2.353366in}{0.635779in}}%
\pgfusepath{stroke}%
\end{pgfscope}%
\begin{pgfscope}%
\definecolor{textcolor}{rgb}{0.000000,0.000000,0.000000}%
\pgfsetstrokecolor{textcolor}%
\pgfsetfillcolor{textcolor}%
\pgftext[x=2.406864in,y=0.556601in,,top]{\color{textcolor}\sffamily\fontsize{6.000000}{7.200000}\selectfont \(\displaystyle 0.8\)}%
\end{pgfscope}%
\begin{pgfscope}%
\pgfsetrectcap%
\pgfsetroundjoin%
\pgfsetlinewidth{0.803000pt}%
\definecolor{currentstroke}{rgb}{0.000000,0.000000,0.000000}%
\pgfsetstrokecolor{currentstroke}%
\pgfsetdash{}{0pt}%
\pgfpathmoveto{\pgfqpoint{2.501720in}{0.746357in}}%
\pgfpathlineto{\pgfqpoint{2.527872in}{0.733108in}}%
\pgfusepath{stroke}%
\end{pgfscope}%
\begin{pgfscope}%
\definecolor{textcolor}{rgb}{0.000000,0.000000,0.000000}%
\pgfsetstrokecolor{textcolor}%
\pgfsetfillcolor{textcolor}%
\pgftext[x=2.580510in,y=0.654408in,,top]{\color{textcolor}\sffamily\fontsize{6.000000}{7.200000}\selectfont \(\displaystyle 1.0\)}%
\end{pgfscope}%
\begin{pgfscope}%
\pgfsetrectcap%
\pgfsetroundjoin%
\pgfsetlinewidth{0.803000pt}%
\definecolor{currentstroke}{rgb}{0.000000,0.000000,0.000000}%
\pgfsetstrokecolor{currentstroke}%
\pgfsetdash{}{0pt}%
\pgfpathmoveto{\pgfqpoint{0.499677in}{0.776948in}}%
\pgfpathlineto{\pgfqpoint{1.536486in}{0.199655in}}%
\pgfusepath{stroke}%
\end{pgfscope}%
\begin{pgfscope}%
\definecolor{textcolor}{rgb}{0.000000,0.000000,0.000000}%
\pgfsetstrokecolor{textcolor}%
\pgfsetfillcolor{textcolor}%
\pgftext[x=0.492803in,y=0.358631in,left,base,rotate=330.891034]{\color{textcolor}\sffamily\fontsize{8.000000}{9.600000}\selectfont Player 1 (\(\displaystyle \lambda\))}%
\end{pgfscope}%
\begin{pgfscope}%
\pgfsetbuttcap%
\pgfsetroundjoin%
\pgfsetlinewidth{0.803000pt}%
\definecolor{currentstroke}{rgb}{0.690196,0.690196,0.690196}%
\pgfsetstrokecolor{currentstroke}%
\pgfsetdash{}{0pt}%
\pgfpathmoveto{\pgfqpoint{2.577976in}{1.589151in}}%
\pgfpathlineto{\pgfqpoint{2.504223in}{0.811728in}}%
\pgfpathlineto{\pgfqpoint{1.467251in}{0.238205in}}%
\pgfusepath{stroke}%
\end{pgfscope}%
\begin{pgfscope}%
\pgfsetbuttcap%
\pgfsetroundjoin%
\pgfsetlinewidth{0.803000pt}%
\definecolor{currentstroke}{rgb}{0.690196,0.690196,0.690196}%
\pgfsetstrokecolor{currentstroke}%
\pgfsetdash{}{0pt}%
\pgfpathmoveto{\pgfqpoint{2.375938in}{1.683294in}}%
\pgfpathlineto{\pgfqpoint{2.317008in}{0.905998in}}%
\pgfpathlineto{\pgfqpoint{1.279711in}{0.342627in}}%
\pgfusepath{stroke}%
\end{pgfscope}%
\begin{pgfscope}%
\pgfsetbuttcap%
\pgfsetroundjoin%
\pgfsetlinewidth{0.803000pt}%
\definecolor{currentstroke}{rgb}{0.690196,0.690196,0.690196}%
\pgfsetstrokecolor{currentstroke}%
\pgfsetdash{}{0pt}%
\pgfpathmoveto{\pgfqpoint{2.177631in}{1.775698in}}%
\pgfpathlineto{\pgfqpoint{2.133009in}{0.998647in}}%
\pgfpathlineto{\pgfqpoint{1.095559in}{0.445162in}}%
\pgfusepath{stroke}%
\end{pgfscope}%
\begin{pgfscope}%
\pgfsetbuttcap%
\pgfsetroundjoin%
\pgfsetlinewidth{0.803000pt}%
\definecolor{currentstroke}{rgb}{0.690196,0.690196,0.690196}%
\pgfsetstrokecolor{currentstroke}%
\pgfsetdash{}{0pt}%
\pgfpathmoveto{\pgfqpoint{1.982952in}{1.866411in}}%
\pgfpathlineto{\pgfqpoint{1.952144in}{1.089719in}}%
\pgfpathlineto{\pgfqpoint{0.914705in}{0.545861in}}%
\pgfusepath{stroke}%
\end{pgfscope}%
\begin{pgfscope}%
\pgfsetbuttcap%
\pgfsetroundjoin%
\pgfsetlinewidth{0.803000pt}%
\definecolor{currentstroke}{rgb}{0.690196,0.690196,0.690196}%
\pgfsetstrokecolor{currentstroke}%
\pgfsetdash{}{0pt}%
\pgfpathmoveto{\pgfqpoint{1.791803in}{1.955480in}}%
\pgfpathlineto{\pgfqpoint{1.774334in}{1.179253in}}%
\pgfpathlineto{\pgfqpoint{0.737061in}{0.644773in}}%
\pgfusepath{stroke}%
\end{pgfscope}%
\begin{pgfscope}%
\pgfsetbuttcap%
\pgfsetroundjoin%
\pgfsetlinewidth{0.803000pt}%
\definecolor{currentstroke}{rgb}{0.690196,0.690196,0.690196}%
\pgfsetstrokecolor{currentstroke}%
\pgfsetdash{}{0pt}%
\pgfpathmoveto{\pgfqpoint{1.604088in}{2.042948in}}%
\pgfpathlineto{\pgfqpoint{1.599501in}{1.267287in}}%
\pgfpathlineto{\pgfqpoint{0.562543in}{0.741945in}}%
\pgfusepath{stroke}%
\end{pgfscope}%
\begin{pgfscope}%
\pgfsetrectcap%
\pgfsetroundjoin%
\pgfsetlinewidth{0.803000pt}%
\definecolor{currentstroke}{rgb}{0.000000,0.000000,0.000000}%
\pgfsetstrokecolor{currentstroke}%
\pgfsetdash{}{0pt}%
\pgfpathmoveto{\pgfqpoint{1.475981in}{0.243033in}}%
\pgfpathlineto{\pgfqpoint{1.449770in}{0.228537in}}%
\pgfusepath{stroke}%
\end{pgfscope}%
\begin{pgfscope}%
\definecolor{textcolor}{rgb}{0.000000,0.000000,0.000000}%
\pgfsetstrokecolor{textcolor}%
\pgfsetfillcolor{textcolor}%
\pgftext[x=1.392595in,y=0.147403in,,top]{\color{textcolor}\sffamily\fontsize{6.000000}{7.200000}\selectfont \(\displaystyle 0.0\)}%
\end{pgfscope}%
\begin{pgfscope}%
\pgfsetrectcap%
\pgfsetroundjoin%
\pgfsetlinewidth{0.803000pt}%
\definecolor{currentstroke}{rgb}{0.000000,0.000000,0.000000}%
\pgfsetstrokecolor{currentstroke}%
\pgfsetdash{}{0pt}%
\pgfpathmoveto{\pgfqpoint{1.288439in}{0.347367in}}%
\pgfpathlineto{\pgfqpoint{1.262233in}{0.333134in}}%
\pgfusepath{stroke}%
\end{pgfscope}%
\begin{pgfscope}%
\definecolor{textcolor}{rgb}{0.000000,0.000000,0.000000}%
\pgfsetstrokecolor{textcolor}%
\pgfsetfillcolor{textcolor}%
\pgftext[x=1.206013in,y=0.252496in,,top]{\color{textcolor}\sffamily\fontsize{6.000000}{7.200000}\selectfont \(\displaystyle 0.2\)}%
\end{pgfscope}%
\begin{pgfscope}%
\pgfsetrectcap%
\pgfsetroundjoin%
\pgfsetlinewidth{0.803000pt}%
\definecolor{currentstroke}{rgb}{0.000000,0.000000,0.000000}%
\pgfsetstrokecolor{currentstroke}%
\pgfsetdash{}{0pt}%
\pgfpathmoveto{\pgfqpoint{1.104285in}{0.449817in}}%
\pgfpathlineto{\pgfqpoint{1.078087in}{0.435840in}}%
\pgfusepath{stroke}%
\end{pgfscope}%
\begin{pgfscope}%
\definecolor{textcolor}{rgb}{0.000000,0.000000,0.000000}%
\pgfsetstrokecolor{textcolor}%
\pgfsetfillcolor{textcolor}%
\pgftext[x=1.022798in,y=0.355693in,,top]{\color{textcolor}\sffamily\fontsize{6.000000}{7.200000}\selectfont \(\displaystyle 0.4\)}%
\end{pgfscope}%
\begin{pgfscope}%
\pgfsetrectcap%
\pgfsetroundjoin%
\pgfsetlinewidth{0.803000pt}%
\definecolor{currentstroke}{rgb}{0.000000,0.000000,0.000000}%
\pgfsetstrokecolor{currentstroke}%
\pgfsetdash{}{0pt}%
\pgfpathmoveto{\pgfqpoint{0.923427in}{0.550433in}}%
\pgfpathlineto{\pgfqpoint{0.897241in}{0.536706in}}%
\pgfusepath{stroke}%
\end{pgfscope}%
\begin{pgfscope}%
\definecolor{textcolor}{rgb}{0.000000,0.000000,0.000000}%
\pgfsetstrokecolor{textcolor}%
\pgfsetfillcolor{textcolor}%
\pgftext[x=0.842859in,y=0.457045in,,top]{\color{textcolor}\sffamily\fontsize{6.000000}{7.200000}\selectfont \(\displaystyle 0.6\)}%
\end{pgfscope}%
\begin{pgfscope}%
\pgfsetrectcap%
\pgfsetroundjoin%
\pgfsetlinewidth{0.803000pt}%
\definecolor{currentstroke}{rgb}{0.000000,0.000000,0.000000}%
\pgfsetstrokecolor{currentstroke}%
\pgfsetdash{}{0pt}%
\pgfpathmoveto{\pgfqpoint{0.745778in}{0.649264in}}%
\pgfpathlineto{\pgfqpoint{0.719607in}{0.635779in}}%
\pgfusepath{stroke}%
\end{pgfscope}%
\begin{pgfscope}%
\definecolor{textcolor}{rgb}{0.000000,0.000000,0.000000}%
\pgfsetstrokecolor{textcolor}%
\pgfsetfillcolor{textcolor}%
\pgftext[x=0.666109in,y=0.556601in,,top]{\color{textcolor}\sffamily\fontsize{6.000000}{7.200000}\selectfont \(\displaystyle 0.8\)}%
\end{pgfscope}%
\begin{pgfscope}%
\pgfsetrectcap%
\pgfsetroundjoin%
\pgfsetlinewidth{0.803000pt}%
\definecolor{currentstroke}{rgb}{0.000000,0.000000,0.000000}%
\pgfsetstrokecolor{currentstroke}%
\pgfsetdash{}{0pt}%
\pgfpathmoveto{\pgfqpoint{0.571253in}{0.746357in}}%
\pgfpathlineto{\pgfqpoint{0.545101in}{0.733108in}}%
\pgfusepath{stroke}%
\end{pgfscope}%
\begin{pgfscope}%
\definecolor{textcolor}{rgb}{0.000000,0.000000,0.000000}%
\pgfsetstrokecolor{textcolor}%
\pgfsetfillcolor{textcolor}%
\pgftext[x=0.492463in,y=0.654408in,,top]{\color{textcolor}\sffamily\fontsize{6.000000}{7.200000}\selectfont \(\displaystyle 1.0\)}%
\end{pgfscope}%
\begin{pgfscope}%
\pgfsetrectcap%
\pgfsetroundjoin%
\pgfsetlinewidth{0.803000pt}%
\definecolor{currentstroke}{rgb}{0.000000,0.000000,0.000000}%
\pgfsetstrokecolor{currentstroke}%
\pgfsetdash{}{0pt}%
\pgfpathmoveto{\pgfqpoint{0.499677in}{0.776948in}}%
\pgfpathlineto{\pgfqpoint{0.420389in}{1.554387in}}%
\pgfusepath{stroke}%
\end{pgfscope}%
\begin{pgfscope}%
\definecolor{textcolor}{rgb}{0.000000,0.000000,0.000000}%
\pgfsetstrokecolor{textcolor}%
\pgfsetfillcolor{textcolor}%
\pgftext[x=0.041630in,y=1.401767in,left,base,rotate=275.823265]{\color{textcolor}\sffamily\fontsize{8.000000}{9.600000}\selectfont \(\displaystyle \mathsf{val}(G_{\lambda, \mu}\))}%
\end{pgfscope}%
\begin{pgfscope}%
\pgfsetbuttcap%
\pgfsetroundjoin%
\pgfsetlinewidth{0.803000pt}%
\definecolor{currentstroke}{rgb}{0.690196,0.690196,0.690196}%
\pgfsetstrokecolor{currentstroke}%
\pgfsetdash{}{0pt}%
\pgfpathmoveto{\pgfqpoint{0.495833in}{0.814642in}}%
\pgfpathlineto{\pgfqpoint{1.536486in}{1.336744in}}%
\pgfpathlineto{\pgfqpoint{2.577140in}{0.814642in}}%
\pgfusepath{stroke}%
\end{pgfscope}%
\begin{pgfscope}%
\pgfsetbuttcap%
\pgfsetroundjoin%
\pgfsetlinewidth{0.803000pt}%
\definecolor{currentstroke}{rgb}{0.690196,0.690196,0.690196}%
\pgfsetstrokecolor{currentstroke}%
\pgfsetdash{}{0pt}%
\pgfpathmoveto{\pgfqpoint{0.483823in}{0.932397in}}%
\pgfpathlineto{\pgfqpoint{1.536486in}{1.454518in}}%
\pgfpathlineto{\pgfqpoint{2.589150in}{0.932397in}}%
\pgfusepath{stroke}%
\end{pgfscope}%
\begin{pgfscope}%
\pgfsetbuttcap%
\pgfsetroundjoin%
\pgfsetlinewidth{0.803000pt}%
\definecolor{currentstroke}{rgb}{0.690196,0.690196,0.690196}%
\pgfsetstrokecolor{currentstroke}%
\pgfsetdash{}{0pt}%
\pgfpathmoveto{\pgfqpoint{0.471534in}{1.052902in}}%
\pgfpathlineto{\pgfqpoint{1.536486in}{1.574907in}}%
\pgfpathlineto{\pgfqpoint{2.601439in}{1.052902in}}%
\pgfusepath{stroke}%
\end{pgfscope}%
\begin{pgfscope}%
\pgfsetbuttcap%
\pgfsetroundjoin%
\pgfsetlinewidth{0.803000pt}%
\definecolor{currentstroke}{rgb}{0.690196,0.690196,0.690196}%
\pgfsetstrokecolor{currentstroke}%
\pgfsetdash{}{0pt}%
\pgfpathmoveto{\pgfqpoint{0.458953in}{1.176254in}}%
\pgfpathlineto{\pgfqpoint{1.536486in}{1.697999in}}%
\pgfpathlineto{\pgfqpoint{2.614020in}{1.176254in}}%
\pgfusepath{stroke}%
\end{pgfscope}%
\begin{pgfscope}%
\pgfsetbuttcap%
\pgfsetroundjoin%
\pgfsetlinewidth{0.803000pt}%
\definecolor{currentstroke}{rgb}{0.690196,0.690196,0.690196}%
\pgfsetstrokecolor{currentstroke}%
\pgfsetdash{}{0pt}%
\pgfpathmoveto{\pgfqpoint{0.446072in}{1.302555in}}%
\pgfpathlineto{\pgfqpoint{1.536486in}{1.823887in}}%
\pgfpathlineto{\pgfqpoint{2.626901in}{1.302555in}}%
\pgfusepath{stroke}%
\end{pgfscope}%
\begin{pgfscope}%
\pgfsetbuttcap%
\pgfsetroundjoin%
\pgfsetlinewidth{0.803000pt}%
\definecolor{currentstroke}{rgb}{0.690196,0.690196,0.690196}%
\pgfsetstrokecolor{currentstroke}%
\pgfsetdash{}{0pt}%
\pgfpathmoveto{\pgfqpoint{0.432880in}{1.431913in}}%
\pgfpathlineto{\pgfqpoint{1.536486in}{1.952666in}}%
\pgfpathlineto{\pgfqpoint{2.640093in}{1.431913in}}%
\pgfusepath{stroke}%
\end{pgfscope}%
\begin{pgfscope}%
\pgfsetrectcap%
\pgfsetroundjoin%
\pgfsetlinewidth{0.803000pt}%
\definecolor{currentstroke}{rgb}{0.000000,0.000000,0.000000}%
\pgfsetstrokecolor{currentstroke}%
\pgfsetdash{}{0pt}%
\pgfpathmoveto{\pgfqpoint{0.504574in}{0.819028in}}%
\pgfpathlineto{\pgfqpoint{0.478329in}{0.805860in}}%
\pgfusepath{stroke}%
\end{pgfscope}%
\begin{pgfscope}%
\definecolor{textcolor}{rgb}{0.000000,0.000000,0.000000}%
\pgfsetstrokecolor{textcolor}%
\pgfsetfillcolor{textcolor}%
\pgftext[x=0.352603in,y=0.814642in,,top]{\color{textcolor}\sffamily\fontsize{6.000000}{7.200000}\selectfont \(\displaystyle -0.3\)}%
\end{pgfscope}%
\begin{pgfscope}%
\pgfsetrectcap%
\pgfsetroundjoin%
\pgfsetlinewidth{0.803000pt}%
\definecolor{currentstroke}{rgb}{0.000000,0.000000,0.000000}%
\pgfsetstrokecolor{currentstroke}%
\pgfsetdash{}{0pt}%
\pgfpathmoveto{\pgfqpoint{0.492671in}{0.936786in}}%
\pgfpathlineto{\pgfqpoint{0.466107in}{0.923610in}}%
\pgfusepath{stroke}%
\end{pgfscope}%
\begin{pgfscope}%
\definecolor{textcolor}{rgb}{0.000000,0.000000,0.000000}%
\pgfsetstrokecolor{textcolor}%
\pgfsetfillcolor{textcolor}%
\pgftext[x=0.338941in,y=0.932397in,,top]{\color{textcolor}\sffamily\fontsize{6.000000}{7.200000}\selectfont \(\displaystyle -0.2\)}%
\end{pgfscope}%
\begin{pgfscope}%
\pgfsetrectcap%
\pgfsetroundjoin%
\pgfsetlinewidth{0.803000pt}%
\definecolor{currentstroke}{rgb}{0.000000,0.000000,0.000000}%
\pgfsetstrokecolor{currentstroke}%
\pgfsetdash{}{0pt}%
\pgfpathmoveto{\pgfqpoint{0.480489in}{1.057292in}}%
\pgfpathlineto{\pgfqpoint{0.453600in}{1.044112in}}%
\pgfusepath{stroke}%
\end{pgfscope}%
\begin{pgfscope}%
\definecolor{textcolor}{rgb}{0.000000,0.000000,0.000000}%
\pgfsetstrokecolor{textcolor}%
\pgfsetfillcolor{textcolor}%
\pgftext[x=0.324959in,y=1.052902in,,top]{\color{textcolor}\sffamily\fontsize{6.000000}{7.200000}\selectfont \(\displaystyle -0.1\)}%
\end{pgfscope}%
\begin{pgfscope}%
\pgfsetrectcap%
\pgfsetroundjoin%
\pgfsetlinewidth{0.803000pt}%
\definecolor{currentstroke}{rgb}{0.000000,0.000000,0.000000}%
\pgfsetstrokecolor{currentstroke}%
\pgfsetdash{}{0pt}%
\pgfpathmoveto{\pgfqpoint{0.468020in}{1.180644in}}%
\pgfpathlineto{\pgfqpoint{0.440798in}{1.167463in}}%
\pgfusepath{stroke}%
\end{pgfscope}%
\begin{pgfscope}%
\definecolor{textcolor}{rgb}{0.000000,0.000000,0.000000}%
\pgfsetstrokecolor{textcolor}%
\pgfsetfillcolor{textcolor}%
\pgftext[x=0.310648in,y=1.176254in,,top]{\color{textcolor}\sffamily\fontsize{6.000000}{7.200000}\selectfont \(\displaystyle 0.0\)}%
\end{pgfscope}%
\begin{pgfscope}%
\pgfsetrectcap%
\pgfsetroundjoin%
\pgfsetlinewidth{0.803000pt}%
\definecolor{currentstroke}{rgb}{0.000000,0.000000,0.000000}%
\pgfsetstrokecolor{currentstroke}%
\pgfsetdash{}{0pt}%
\pgfpathmoveto{\pgfqpoint{0.455253in}{1.306945in}}%
\pgfpathlineto{\pgfqpoint{0.427688in}{1.293766in}}%
\pgfusepath{stroke}%
\end{pgfscope}%
\begin{pgfscope}%
\definecolor{textcolor}{rgb}{0.000000,0.000000,0.000000}%
\pgfsetstrokecolor{textcolor}%
\pgfsetfillcolor{textcolor}%
\pgftext[x=0.295994in,y=1.302555in,,top]{\color{textcolor}\sffamily\fontsize{6.000000}{7.200000}\selectfont \(\displaystyle 0.1\)}%
\end{pgfscope}%
\begin{pgfscope}%
\pgfsetrectcap%
\pgfsetroundjoin%
\pgfsetlinewidth{0.803000pt}%
\definecolor{currentstroke}{rgb}{0.000000,0.000000,0.000000}%
\pgfsetstrokecolor{currentstroke}%
\pgfsetdash{}{0pt}%
\pgfpathmoveto{\pgfqpoint{0.442177in}{1.436300in}}%
\pgfpathlineto{\pgfqpoint{0.414262in}{1.423128in}}%
\pgfusepath{stroke}%
\end{pgfscope}%
\begin{pgfscope}%
\definecolor{textcolor}{rgb}{0.000000,0.000000,0.000000}%
\pgfsetstrokecolor{textcolor}%
\pgfsetfillcolor{textcolor}%
\pgftext[x=0.280985in,y=1.431913in,,top]{\color{textcolor}\sffamily\fontsize{6.000000}{7.200000}\selectfont \(\displaystyle 0.2\)}%
\end{pgfscope}%
\begin{pgfscope}%
\pgfpathrectangle{\pgfqpoint{0.150000in}{0.150000in}}{\pgfqpoint{2.700000in}{1.950000in}}%
\pgfusepath{clip}%
\pgfsetbuttcap%
\pgfsetroundjoin%
\definecolor{currentfill}{rgb}{0.952512,0.913833,0.916896}%
\pgfsetfillcolor{currentfill}%
\pgfsetlinewidth{0.000000pt}%
\definecolor{currentstroke}{rgb}{0.000000,0.000000,0.000000}%
\pgfsetstrokecolor{currentstroke}%
\pgfsetdash{}{0pt}%
\pgfpathmoveto{\pgfqpoint{1.536486in}{1.549696in}}%
\pgfpathlineto{\pgfqpoint{1.572363in}{1.567326in}}%
\pgfpathlineto{\pgfqpoint{1.536486in}{1.584894in}}%
\pgfpathlineto{\pgfqpoint{1.500610in}{1.567326in}}%
\pgfpathclose%
\pgfusepath{fill}%
\end{pgfscope}%
\begin{pgfscope}%
\pgfpathrectangle{\pgfqpoint{0.150000in}{0.150000in}}{\pgfqpoint{2.700000in}{1.950000in}}%
\pgfusepath{clip}%
\pgfsetbuttcap%
\pgfsetroundjoin%
\definecolor{currentfill}{rgb}{0.952512,0.913833,0.916896}%
\pgfsetfillcolor{currentfill}%
\pgfsetlinewidth{0.000000pt}%
\definecolor{currentstroke}{rgb}{0.000000,0.000000,0.000000}%
\pgfsetstrokecolor{currentstroke}%
\pgfsetdash{}{0pt}%
\pgfpathmoveto{\pgfqpoint{1.572489in}{1.532005in}}%
\pgfpathlineto{\pgfqpoint{1.608365in}{1.549696in}}%
\pgfpathlineto{\pgfqpoint{1.572363in}{1.567326in}}%
\pgfpathlineto{\pgfqpoint{1.536486in}{1.549696in}}%
\pgfpathclose%
\pgfusepath{fill}%
\end{pgfscope}%
\begin{pgfscope}%
\pgfpathrectangle{\pgfqpoint{0.150000in}{0.150000in}}{\pgfqpoint{2.700000in}{1.950000in}}%
\pgfusepath{clip}%
\pgfsetbuttcap%
\pgfsetroundjoin%
\definecolor{currentfill}{rgb}{0.952512,0.913833,0.916896}%
\pgfsetfillcolor{currentfill}%
\pgfsetlinewidth{0.000000pt}%
\definecolor{currentstroke}{rgb}{0.000000,0.000000,0.000000}%
\pgfsetstrokecolor{currentstroke}%
\pgfsetdash{}{0pt}%
\pgfpathmoveto{\pgfqpoint{1.500484in}{1.532005in}}%
\pgfpathlineto{\pgfqpoint{1.536486in}{1.549696in}}%
\pgfpathlineto{\pgfqpoint{1.500610in}{1.567326in}}%
\pgfpathlineto{\pgfqpoint{1.464608in}{1.549696in}}%
\pgfpathclose%
\pgfusepath{fill}%
\end{pgfscope}%
\begin{pgfscope}%
\pgfpathrectangle{\pgfqpoint{0.150000in}{0.150000in}}{\pgfqpoint{2.700000in}{1.950000in}}%
\pgfusepath{clip}%
\pgfsetbuttcap%
\pgfsetroundjoin%
\definecolor{currentfill}{rgb}{0.656189,0.376149,0.398330}%
\pgfsetfillcolor{currentfill}%
\pgfsetlinewidth{0.000000pt}%
\definecolor{currentstroke}{rgb}{0.000000,0.000000,0.000000}%
\pgfsetstrokecolor{currentstroke}%
\pgfsetdash{}{0pt}%
\pgfpathmoveto{\pgfqpoint{2.184691in}{0.909532in}}%
\pgfpathlineto{\pgfqpoint{2.217972in}{0.905069in}}%
\pgfpathlineto{\pgfqpoint{2.183166in}{0.958688in}}%
\pgfpathlineto{\pgfqpoint{2.148996in}{0.951660in}}%
\pgfpathclose%
\pgfusepath{fill}%
\end{pgfscope}%
\begin{pgfscope}%
\pgfpathrectangle{\pgfqpoint{0.150000in}{0.150000in}}{\pgfqpoint{2.700000in}{1.950000in}}%
\pgfusepath{clip}%
\pgfsetbuttcap%
\pgfsetroundjoin%
\definecolor{currentfill}{rgb}{0.640993,0.348575,0.371737}%
\pgfsetfillcolor{currentfill}%
\pgfsetlinewidth{0.000000pt}%
\definecolor{currentstroke}{rgb}{0.000000,0.000000,0.000000}%
\pgfsetstrokecolor{currentstroke}%
\pgfsetdash{}{0pt}%
\pgfpathmoveto{\pgfqpoint{2.221898in}{0.890687in}}%
\pgfpathlineto{\pgfqpoint{2.255088in}{0.886291in}}%
\pgfpathlineto{\pgfqpoint{2.217972in}{0.905069in}}%
\pgfpathlineto{\pgfqpoint{2.184691in}{0.909532in}}%
\pgfpathclose%
\pgfusepath{fill}%
\end{pgfscope}%
\begin{pgfscope}%
\pgfpathrectangle{\pgfqpoint{0.150000in}{0.150000in}}{\pgfqpoint{2.700000in}{1.950000in}}%
\pgfusepath{clip}%
\pgfsetbuttcap%
\pgfsetroundjoin%
\definecolor{currentfill}{rgb}{0.952512,0.913833,0.916896}%
\pgfsetfillcolor{currentfill}%
\pgfsetlinewidth{0.000000pt}%
\definecolor{currentstroke}{rgb}{0.000000,0.000000,0.000000}%
\pgfsetstrokecolor{currentstroke}%
\pgfsetdash{}{0pt}%
\pgfpathmoveto{\pgfqpoint{1.608617in}{1.514252in}}%
\pgfpathlineto{\pgfqpoint{1.644493in}{1.532005in}}%
\pgfpathlineto{\pgfqpoint{1.608365in}{1.549696in}}%
\pgfpathlineto{\pgfqpoint{1.572489in}{1.532005in}}%
\pgfpathclose%
\pgfusepath{fill}%
\end{pgfscope}%
\begin{pgfscope}%
\pgfpathrectangle{\pgfqpoint{0.150000in}{0.150000in}}{\pgfqpoint{2.700000in}{1.950000in}}%
\pgfusepath{clip}%
\pgfsetbuttcap%
\pgfsetroundjoin%
\definecolor{currentfill}{rgb}{0.952512,0.913833,0.916896}%
\pgfsetfillcolor{currentfill}%
\pgfsetlinewidth{0.000000pt}%
\definecolor{currentstroke}{rgb}{0.000000,0.000000,0.000000}%
\pgfsetstrokecolor{currentstroke}%
\pgfsetdash{}{0pt}%
\pgfpathmoveto{\pgfqpoint{1.536486in}{1.514252in}}%
\pgfpathlineto{\pgfqpoint{1.572489in}{1.532005in}}%
\pgfpathlineto{\pgfqpoint{1.536486in}{1.549696in}}%
\pgfpathlineto{\pgfqpoint{1.500484in}{1.532005in}}%
\pgfpathclose%
\pgfusepath{fill}%
\end{pgfscope}%
\begin{pgfscope}%
\pgfpathrectangle{\pgfqpoint{0.150000in}{0.150000in}}{\pgfqpoint{2.700000in}{1.950000in}}%
\pgfusepath{clip}%
\pgfsetbuttcap%
\pgfsetroundjoin%
\definecolor{currentfill}{rgb}{0.952512,0.913833,0.916896}%
\pgfsetfillcolor{currentfill}%
\pgfsetlinewidth{0.000000pt}%
\definecolor{currentstroke}{rgb}{0.000000,0.000000,0.000000}%
\pgfsetstrokecolor{currentstroke}%
\pgfsetdash{}{0pt}%
\pgfpathmoveto{\pgfqpoint{1.464356in}{1.514252in}}%
\pgfpathlineto{\pgfqpoint{1.500484in}{1.532005in}}%
\pgfpathlineto{\pgfqpoint{1.464608in}{1.549696in}}%
\pgfpathlineto{\pgfqpoint{1.428480in}{1.532005in}}%
\pgfpathclose%
\pgfusepath{fill}%
\end{pgfscope}%
\begin{pgfscope}%
\pgfpathrectangle{\pgfqpoint{0.150000in}{0.150000in}}{\pgfqpoint{2.700000in}{1.950000in}}%
\pgfusepath{clip}%
\pgfsetbuttcap%
\pgfsetroundjoin%
\definecolor{currentfill}{rgb}{0.686581,0.431296,0.451517}%
\pgfsetfillcolor{currentfill}%
\pgfsetlinewidth{0.000000pt}%
\definecolor{currentstroke}{rgb}{0.000000,0.000000,0.000000}%
\pgfsetstrokecolor{currentstroke}%
\pgfsetdash{}{0pt}%
\pgfpathmoveto{\pgfqpoint{2.148996in}{0.951660in}}%
\pgfpathlineto{\pgfqpoint{2.183166in}{0.958688in}}%
\pgfpathlineto{\pgfqpoint{2.148245in}{1.012484in}}%
\pgfpathlineto{\pgfqpoint{2.113918in}{1.005600in}}%
\pgfpathclose%
\pgfusepath{fill}%
\end{pgfscope}%
\begin{pgfscope}%
\pgfpathrectangle{\pgfqpoint{0.150000in}{0.150000in}}{\pgfqpoint{2.700000in}{1.950000in}}%
\pgfusepath{clip}%
\pgfsetbuttcap%
\pgfsetroundjoin%
\definecolor{currentfill}{rgb}{0.952512,0.913833,0.916896}%
\pgfsetfillcolor{currentfill}%
\pgfsetlinewidth{0.000000pt}%
\definecolor{currentstroke}{rgb}{0.000000,0.000000,0.000000}%
\pgfsetstrokecolor{currentstroke}%
\pgfsetdash{}{0pt}%
\pgfpathmoveto{\pgfqpoint{1.644871in}{1.496437in}}%
\pgfpathlineto{\pgfqpoint{1.680747in}{1.514252in}}%
\pgfpathlineto{\pgfqpoint{1.644493in}{1.532005in}}%
\pgfpathlineto{\pgfqpoint{1.608617in}{1.514252in}}%
\pgfpathclose%
\pgfusepath{fill}%
\end{pgfscope}%
\begin{pgfscope}%
\pgfpathrectangle{\pgfqpoint{0.150000in}{0.150000in}}{\pgfqpoint{2.700000in}{1.950000in}}%
\pgfusepath{clip}%
\pgfsetbuttcap%
\pgfsetroundjoin%
\definecolor{currentfill}{rgb}{0.952512,0.913833,0.916896}%
\pgfsetfillcolor{currentfill}%
\pgfsetlinewidth{0.000000pt}%
\definecolor{currentstroke}{rgb}{0.000000,0.000000,0.000000}%
\pgfsetstrokecolor{currentstroke}%
\pgfsetdash{}{0pt}%
\pgfpathmoveto{\pgfqpoint{1.572615in}{1.496437in}}%
\pgfpathlineto{\pgfqpoint{1.608617in}{1.514252in}}%
\pgfpathlineto{\pgfqpoint{1.572489in}{1.532005in}}%
\pgfpathlineto{\pgfqpoint{1.536486in}{1.514252in}}%
\pgfpathclose%
\pgfusepath{fill}%
\end{pgfscope}%
\begin{pgfscope}%
\pgfpathrectangle{\pgfqpoint{0.150000in}{0.150000in}}{\pgfqpoint{2.700000in}{1.950000in}}%
\pgfusepath{clip}%
\pgfsetbuttcap%
\pgfsetroundjoin%
\definecolor{currentfill}{rgb}{0.952512,0.913833,0.916896}%
\pgfsetfillcolor{currentfill}%
\pgfsetlinewidth{0.000000pt}%
\definecolor{currentstroke}{rgb}{0.000000,0.000000,0.000000}%
\pgfsetstrokecolor{currentstroke}%
\pgfsetdash{}{0pt}%
\pgfpathmoveto{\pgfqpoint{1.500358in}{1.496437in}}%
\pgfpathlineto{\pgfqpoint{1.536486in}{1.514252in}}%
\pgfpathlineto{\pgfqpoint{1.500484in}{1.532005in}}%
\pgfpathlineto{\pgfqpoint{1.464356in}{1.514252in}}%
\pgfpathclose%
\pgfusepath{fill}%
\end{pgfscope}%
\begin{pgfscope}%
\pgfpathrectangle{\pgfqpoint{0.150000in}{0.150000in}}{\pgfqpoint{2.700000in}{1.950000in}}%
\pgfusepath{clip}%
\pgfsetbuttcap%
\pgfsetroundjoin%
\definecolor{currentfill}{rgb}{0.952512,0.913833,0.916896}%
\pgfsetfillcolor{currentfill}%
\pgfsetlinewidth{0.000000pt}%
\definecolor{currentstroke}{rgb}{0.000000,0.000000,0.000000}%
\pgfsetstrokecolor{currentstroke}%
\pgfsetdash{}{0pt}%
\pgfpathmoveto{\pgfqpoint{1.428102in}{1.496437in}}%
\pgfpathlineto{\pgfqpoint{1.464356in}{1.514252in}}%
\pgfpathlineto{\pgfqpoint{1.428480in}{1.532005in}}%
\pgfpathlineto{\pgfqpoint{1.392226in}{1.514252in}}%
\pgfpathclose%
\pgfusepath{fill}%
\end{pgfscope}%
\begin{pgfscope}%
\pgfpathrectangle{\pgfqpoint{0.150000in}{0.150000in}}{\pgfqpoint{2.700000in}{1.950000in}}%
\pgfusepath{clip}%
\pgfsetbuttcap%
\pgfsetroundjoin%
\definecolor{currentfill}{rgb}{0.720772,0.493336,0.511351}%
\pgfsetfillcolor{currentfill}%
\pgfsetlinewidth{0.000000pt}%
\definecolor{currentstroke}{rgb}{0.000000,0.000000,0.000000}%
\pgfsetstrokecolor{currentstroke}%
\pgfsetdash{}{0pt}%
\pgfpathmoveto{\pgfqpoint{2.113918in}{1.005600in}}%
\pgfpathlineto{\pgfqpoint{2.148245in}{1.012484in}}%
\pgfpathlineto{\pgfqpoint{2.113209in}{1.066458in}}%
\pgfpathlineto{\pgfqpoint{2.078724in}{1.059718in}}%
\pgfpathclose%
\pgfusepath{fill}%
\end{pgfscope}%
\begin{pgfscope}%
\pgfpathrectangle{\pgfqpoint{0.150000in}{0.150000in}}{\pgfqpoint{2.700000in}{1.950000in}}%
\pgfusepath{clip}%
\pgfsetbuttcap%
\pgfsetroundjoin%
\definecolor{currentfill}{rgb}{0.652390,0.369256,0.391682}%
\pgfsetfillcolor{currentfill}%
\pgfsetlinewidth{0.000000pt}%
\definecolor{currentstroke}{rgb}{0.000000,0.000000,0.000000}%
\pgfsetstrokecolor{currentstroke}%
\pgfsetdash{}{0pt}%
\pgfpathmoveto{\pgfqpoint{2.260880in}{0.895118in}}%
\pgfpathlineto{\pgfqpoint{2.294047in}{0.890687in}}%
\pgfpathlineto{\pgfqpoint{2.255088in}{0.886291in}}%
\pgfpathlineto{\pgfqpoint{2.221898in}{0.890687in}}%
\pgfpathclose%
\pgfusepath{fill}%
\end{pgfscope}%
\begin{pgfscope}%
\pgfpathrectangle{\pgfqpoint{0.150000in}{0.150000in}}{\pgfqpoint{2.700000in}{1.950000in}}%
\pgfusepath{clip}%
\pgfsetbuttcap%
\pgfsetroundjoin%
\definecolor{currentfill}{rgb}{0.671385,0.403722,0.424923}%
\pgfsetfillcolor{currentfill}%
\pgfsetlinewidth{0.000000pt}%
\definecolor{currentstroke}{rgb}{0.000000,0.000000,0.000000}%
\pgfsetstrokecolor{currentstroke}%
\pgfsetdash{}{0pt}%
\pgfpathmoveto{\pgfqpoint{2.150445in}{0.902346in}}%
\pgfpathlineto{\pgfqpoint{2.184691in}{0.909532in}}%
\pgfpathlineto{\pgfqpoint{2.148996in}{0.951660in}}%
\pgfpathlineto{\pgfqpoint{2.115286in}{0.956332in}}%
\pgfpathclose%
\pgfusepath{fill}%
\end{pgfscope}%
\begin{pgfscope}%
\pgfpathrectangle{\pgfqpoint{0.150000in}{0.150000in}}{\pgfqpoint{2.700000in}{1.950000in}}%
\pgfusepath{clip}%
\pgfsetbuttcap%
\pgfsetroundjoin%
\definecolor{currentfill}{rgb}{0.659988,0.383042,0.404979}%
\pgfsetfillcolor{currentfill}%
\pgfsetlinewidth{0.000000pt}%
\definecolor{currentstroke}{rgb}{0.000000,0.000000,0.000000}%
\pgfsetstrokecolor{currentstroke}%
\pgfsetdash{}{0pt}%
\pgfpathmoveto{\pgfqpoint{2.187700in}{0.883434in}}%
\pgfpathlineto{\pgfqpoint{2.221898in}{0.890687in}}%
\pgfpathlineto{\pgfqpoint{2.184691in}{0.909532in}}%
\pgfpathlineto{\pgfqpoint{2.150445in}{0.902346in}}%
\pgfpathclose%
\pgfusepath{fill}%
\end{pgfscope}%
\begin{pgfscope}%
\pgfpathrectangle{\pgfqpoint{0.150000in}{0.150000in}}{\pgfqpoint{2.700000in}{1.950000in}}%
\pgfusepath{clip}%
\pgfsetbuttcap%
\pgfsetroundjoin%
\definecolor{currentfill}{rgb}{0.758762,0.562270,0.577834}%
\pgfsetfillcolor{currentfill}%
\pgfsetlinewidth{0.000000pt}%
\definecolor{currentstroke}{rgb}{0.000000,0.000000,0.000000}%
\pgfsetstrokecolor{currentstroke}%
\pgfsetdash{}{0pt}%
\pgfpathmoveto{\pgfqpoint{2.078724in}{1.059718in}}%
\pgfpathlineto{\pgfqpoint{2.113209in}{1.066458in}}%
\pgfpathlineto{\pgfqpoint{2.078057in}{1.120610in}}%
\pgfpathlineto{\pgfqpoint{2.043414in}{1.114015in}}%
\pgfpathclose%
\pgfusepath{fill}%
\end{pgfscope}%
\begin{pgfscope}%
\pgfpathrectangle{\pgfqpoint{0.150000in}{0.150000in}}{\pgfqpoint{2.700000in}{1.950000in}}%
\pgfusepath{clip}%
\pgfsetbuttcap%
\pgfsetroundjoin%
\definecolor{currentfill}{rgb}{0.952512,0.913833,0.916896}%
\pgfsetfillcolor{currentfill}%
\pgfsetlinewidth{0.000000pt}%
\definecolor{currentstroke}{rgb}{0.000000,0.000000,0.000000}%
\pgfsetstrokecolor{currentstroke}%
\pgfsetdash{}{0pt}%
\pgfpathmoveto{\pgfqpoint{1.681252in}{1.478560in}}%
\pgfpathlineto{\pgfqpoint{1.717127in}{1.496437in}}%
\pgfpathlineto{\pgfqpoint{1.680747in}{1.514252in}}%
\pgfpathlineto{\pgfqpoint{1.644871in}{1.496437in}}%
\pgfpathclose%
\pgfusepath{fill}%
\end{pgfscope}%
\begin{pgfscope}%
\pgfpathrectangle{\pgfqpoint{0.150000in}{0.150000in}}{\pgfqpoint{2.700000in}{1.950000in}}%
\pgfusepath{clip}%
\pgfsetbuttcap%
\pgfsetroundjoin%
\definecolor{currentfill}{rgb}{0.952512,0.913833,0.916896}%
\pgfsetfillcolor{currentfill}%
\pgfsetlinewidth{0.000000pt}%
\definecolor{currentstroke}{rgb}{0.000000,0.000000,0.000000}%
\pgfsetstrokecolor{currentstroke}%
\pgfsetdash{}{0pt}%
\pgfpathmoveto{\pgfqpoint{1.608869in}{1.478560in}}%
\pgfpathlineto{\pgfqpoint{1.644871in}{1.496437in}}%
\pgfpathlineto{\pgfqpoint{1.608617in}{1.514252in}}%
\pgfpathlineto{\pgfqpoint{1.572615in}{1.496437in}}%
\pgfpathclose%
\pgfusepath{fill}%
\end{pgfscope}%
\begin{pgfscope}%
\pgfpathrectangle{\pgfqpoint{0.150000in}{0.150000in}}{\pgfqpoint{2.700000in}{1.950000in}}%
\pgfusepath{clip}%
\pgfsetbuttcap%
\pgfsetroundjoin%
\definecolor{currentfill}{rgb}{0.952512,0.913833,0.916896}%
\pgfsetfillcolor{currentfill}%
\pgfsetlinewidth{0.000000pt}%
\definecolor{currentstroke}{rgb}{0.000000,0.000000,0.000000}%
\pgfsetstrokecolor{currentstroke}%
\pgfsetdash{}{0pt}%
\pgfpathmoveto{\pgfqpoint{1.536486in}{1.478560in}}%
\pgfpathlineto{\pgfqpoint{1.572615in}{1.496437in}}%
\pgfpathlineto{\pgfqpoint{1.536486in}{1.514252in}}%
\pgfpathlineto{\pgfqpoint{1.500358in}{1.496437in}}%
\pgfpathclose%
\pgfusepath{fill}%
\end{pgfscope}%
\begin{pgfscope}%
\pgfpathrectangle{\pgfqpoint{0.150000in}{0.150000in}}{\pgfqpoint{2.700000in}{1.950000in}}%
\pgfusepath{clip}%
\pgfsetbuttcap%
\pgfsetroundjoin%
\definecolor{currentfill}{rgb}{0.952512,0.913833,0.916896}%
\pgfsetfillcolor{currentfill}%
\pgfsetlinewidth{0.000000pt}%
\definecolor{currentstroke}{rgb}{0.000000,0.000000,0.000000}%
\pgfsetstrokecolor{currentstroke}%
\pgfsetdash{}{0pt}%
\pgfpathmoveto{\pgfqpoint{1.464104in}{1.478560in}}%
\pgfpathlineto{\pgfqpoint{1.500358in}{1.496437in}}%
\pgfpathlineto{\pgfqpoint{1.464356in}{1.514252in}}%
\pgfpathlineto{\pgfqpoint{1.428102in}{1.496437in}}%
\pgfpathclose%
\pgfusepath{fill}%
\end{pgfscope}%
\begin{pgfscope}%
\pgfpathrectangle{\pgfqpoint{0.150000in}{0.150000in}}{\pgfqpoint{2.700000in}{1.950000in}}%
\pgfusepath{clip}%
\pgfsetbuttcap%
\pgfsetroundjoin%
\definecolor{currentfill}{rgb}{0.952512,0.913833,0.916896}%
\pgfsetfillcolor{currentfill}%
\pgfsetlinewidth{0.000000pt}%
\definecolor{currentstroke}{rgb}{0.000000,0.000000,0.000000}%
\pgfsetstrokecolor{currentstroke}%
\pgfsetdash{}{0pt}%
\pgfpathmoveto{\pgfqpoint{1.391721in}{1.478560in}}%
\pgfpathlineto{\pgfqpoint{1.428102in}{1.496437in}}%
\pgfpathlineto{\pgfqpoint{1.392226in}{1.514252in}}%
\pgfpathlineto{\pgfqpoint{1.355846in}{1.496437in}}%
\pgfpathclose%
\pgfusepath{fill}%
\end{pgfscope}%
\begin{pgfscope}%
\pgfpathrectangle{\pgfqpoint{0.150000in}{0.150000in}}{\pgfqpoint{2.700000in}{1.950000in}}%
\pgfusepath{clip}%
\pgfsetbuttcap%
\pgfsetroundjoin%
\definecolor{currentfill}{rgb}{0.705576,0.465763,0.484758}%
\pgfsetfillcolor{currentfill}%
\pgfsetlinewidth{0.000000pt}%
\definecolor{currentstroke}{rgb}{0.000000,0.000000,0.000000}%
\pgfsetstrokecolor{currentstroke}%
\pgfsetdash{}{0pt}%
\pgfpathmoveto{\pgfqpoint{2.115286in}{0.956332in}}%
\pgfpathlineto{\pgfqpoint{2.148996in}{0.951660in}}%
\pgfpathlineto{\pgfqpoint{2.113918in}{1.005600in}}%
\pgfpathlineto{\pgfqpoint{2.080011in}{1.010497in}}%
\pgfpathclose%
\pgfusepath{fill}%
\end{pgfscope}%
\begin{pgfscope}%
\pgfpathrectangle{\pgfqpoint{0.150000in}{0.150000in}}{\pgfqpoint{2.700000in}{1.950000in}}%
\pgfusepath{clip}%
\pgfsetbuttcap%
\pgfsetroundjoin%
\definecolor{currentfill}{rgb}{0.735968,0.520910,0.537944}%
\pgfsetfillcolor{currentfill}%
\pgfsetlinewidth{0.000000pt}%
\definecolor{currentstroke}{rgb}{0.000000,0.000000,0.000000}%
\pgfsetstrokecolor{currentstroke}%
\pgfsetdash{}{0pt}%
\pgfpathmoveto{\pgfqpoint{2.080011in}{1.010497in}}%
\pgfpathlineto{\pgfqpoint{2.113918in}{1.005600in}}%
\pgfpathlineto{\pgfqpoint{2.078724in}{1.059718in}}%
\pgfpathlineto{\pgfqpoint{2.044040in}{1.052939in}}%
\pgfpathclose%
\pgfusepath{fill}%
\end{pgfscope}%
\begin{pgfscope}%
\pgfpathrectangle{\pgfqpoint{0.150000in}{0.150000in}}{\pgfqpoint{2.700000in}{1.950000in}}%
\pgfusepath{clip}%
\pgfsetbuttcap%
\pgfsetroundjoin%
\definecolor{currentfill}{rgb}{0.792953,0.624311,0.637669}%
\pgfsetfillcolor{currentfill}%
\pgfsetlinewidth{0.000000pt}%
\definecolor{currentstroke}{rgb}{0.000000,0.000000,0.000000}%
\pgfsetstrokecolor{currentstroke}%
\pgfsetdash{}{0pt}%
\pgfpathmoveto{\pgfqpoint{2.043414in}{1.114015in}}%
\pgfpathlineto{\pgfqpoint{2.078057in}{1.120610in}}%
\pgfpathlineto{\pgfqpoint{2.042789in}{1.174941in}}%
\pgfpathlineto{\pgfqpoint{2.007986in}{1.168493in}}%
\pgfpathclose%
\pgfusepath{fill}%
\end{pgfscope}%
\begin{pgfscope}%
\pgfpathrectangle{\pgfqpoint{0.150000in}{0.150000in}}{\pgfqpoint{2.700000in}{1.950000in}}%
\pgfusepath{clip}%
\pgfsetbuttcap%
\pgfsetroundjoin%
\definecolor{currentfill}{rgb}{0.952512,0.913833,0.916896}%
\pgfsetfillcolor{currentfill}%
\pgfsetlinewidth{0.000000pt}%
\definecolor{currentstroke}{rgb}{0.000000,0.000000,0.000000}%
\pgfsetstrokecolor{currentstroke}%
\pgfsetdash{}{0pt}%
\pgfpathmoveto{\pgfqpoint{1.717762in}{1.460620in}}%
\pgfpathlineto{\pgfqpoint{1.753635in}{1.478560in}}%
\pgfpathlineto{\pgfqpoint{1.717127in}{1.496437in}}%
\pgfpathlineto{\pgfqpoint{1.681252in}{1.478560in}}%
\pgfpathclose%
\pgfusepath{fill}%
\end{pgfscope}%
\begin{pgfscope}%
\pgfpathrectangle{\pgfqpoint{0.150000in}{0.150000in}}{\pgfqpoint{2.700000in}{1.950000in}}%
\pgfusepath{clip}%
\pgfsetbuttcap%
\pgfsetroundjoin%
\definecolor{currentfill}{rgb}{0.952512,0.913833,0.916896}%
\pgfsetfillcolor{currentfill}%
\pgfsetlinewidth{0.000000pt}%
\definecolor{currentstroke}{rgb}{0.000000,0.000000,0.000000}%
\pgfsetstrokecolor{currentstroke}%
\pgfsetdash{}{0pt}%
\pgfpathmoveto{\pgfqpoint{1.645252in}{1.460620in}}%
\pgfpathlineto{\pgfqpoint{1.681252in}{1.478560in}}%
\pgfpathlineto{\pgfqpoint{1.644871in}{1.496437in}}%
\pgfpathlineto{\pgfqpoint{1.608869in}{1.478560in}}%
\pgfpathclose%
\pgfusepath{fill}%
\end{pgfscope}%
\begin{pgfscope}%
\pgfpathrectangle{\pgfqpoint{0.150000in}{0.150000in}}{\pgfqpoint{2.700000in}{1.950000in}}%
\pgfusepath{clip}%
\pgfsetbuttcap%
\pgfsetroundjoin%
\definecolor{currentfill}{rgb}{0.952512,0.913833,0.916896}%
\pgfsetfillcolor{currentfill}%
\pgfsetlinewidth{0.000000pt}%
\definecolor{currentstroke}{rgb}{0.000000,0.000000,0.000000}%
\pgfsetstrokecolor{currentstroke}%
\pgfsetdash{}{0pt}%
\pgfpathmoveto{\pgfqpoint{1.572742in}{1.460620in}}%
\pgfpathlineto{\pgfqpoint{1.608869in}{1.478560in}}%
\pgfpathlineto{\pgfqpoint{1.572615in}{1.496437in}}%
\pgfpathlineto{\pgfqpoint{1.536486in}{1.478560in}}%
\pgfpathclose%
\pgfusepath{fill}%
\end{pgfscope}%
\begin{pgfscope}%
\pgfpathrectangle{\pgfqpoint{0.150000in}{0.150000in}}{\pgfqpoint{2.700000in}{1.950000in}}%
\pgfusepath{clip}%
\pgfsetbuttcap%
\pgfsetroundjoin%
\definecolor{currentfill}{rgb}{0.952512,0.913833,0.916896}%
\pgfsetfillcolor{currentfill}%
\pgfsetlinewidth{0.000000pt}%
\definecolor{currentstroke}{rgb}{0.000000,0.000000,0.000000}%
\pgfsetstrokecolor{currentstroke}%
\pgfsetdash{}{0pt}%
\pgfpathmoveto{\pgfqpoint{1.500231in}{1.460620in}}%
\pgfpathlineto{\pgfqpoint{1.536486in}{1.478560in}}%
\pgfpathlineto{\pgfqpoint{1.500358in}{1.496437in}}%
\pgfpathlineto{\pgfqpoint{1.464104in}{1.478560in}}%
\pgfpathclose%
\pgfusepath{fill}%
\end{pgfscope}%
\begin{pgfscope}%
\pgfpathrectangle{\pgfqpoint{0.150000in}{0.150000in}}{\pgfqpoint{2.700000in}{1.950000in}}%
\pgfusepath{clip}%
\pgfsetbuttcap%
\pgfsetroundjoin%
\definecolor{currentfill}{rgb}{0.952512,0.913833,0.916896}%
\pgfsetfillcolor{currentfill}%
\pgfsetlinewidth{0.000000pt}%
\definecolor{currentstroke}{rgb}{0.000000,0.000000,0.000000}%
\pgfsetstrokecolor{currentstroke}%
\pgfsetdash{}{0pt}%
\pgfpathmoveto{\pgfqpoint{1.427721in}{1.460620in}}%
\pgfpathlineto{\pgfqpoint{1.464104in}{1.478560in}}%
\pgfpathlineto{\pgfqpoint{1.428102in}{1.496437in}}%
\pgfpathlineto{\pgfqpoint{1.391721in}{1.478560in}}%
\pgfpathclose%
\pgfusepath{fill}%
\end{pgfscope}%
\begin{pgfscope}%
\pgfpathrectangle{\pgfqpoint{0.150000in}{0.150000in}}{\pgfqpoint{2.700000in}{1.950000in}}%
\pgfusepath{clip}%
\pgfsetbuttcap%
\pgfsetroundjoin%
\definecolor{currentfill}{rgb}{0.952512,0.913833,0.916896}%
\pgfsetfillcolor{currentfill}%
\pgfsetlinewidth{0.000000pt}%
\definecolor{currentstroke}{rgb}{0.000000,0.000000,0.000000}%
\pgfsetstrokecolor{currentstroke}%
\pgfsetdash{}{0pt}%
\pgfpathmoveto{\pgfqpoint{1.355211in}{1.460620in}}%
\pgfpathlineto{\pgfqpoint{1.391721in}{1.478560in}}%
\pgfpathlineto{\pgfqpoint{1.355846in}{1.496437in}}%
\pgfpathlineto{\pgfqpoint{1.319338in}{1.478560in}}%
\pgfpathclose%
\pgfusepath{fill}%
\end{pgfscope}%
\begin{pgfscope}%
\pgfpathrectangle{\pgfqpoint{0.150000in}{0.150000in}}{\pgfqpoint{2.700000in}{1.950000in}}%
\pgfusepath{clip}%
\pgfsetbuttcap%
\pgfsetroundjoin%
\definecolor{currentfill}{rgb}{0.671385,0.403722,0.424923}%
\pgfsetfillcolor{currentfill}%
\pgfsetlinewidth{0.000000pt}%
\definecolor{currentstroke}{rgb}{0.000000,0.000000,0.000000}%
\pgfsetstrokecolor{currentstroke}%
\pgfsetdash{}{0pt}%
\pgfpathmoveto{\pgfqpoint{2.226656in}{0.887849in}}%
\pgfpathlineto{\pgfqpoint{2.260880in}{0.895118in}}%
\pgfpathlineto{\pgfqpoint{2.221898in}{0.890687in}}%
\pgfpathlineto{\pgfqpoint{2.187700in}{0.883434in}}%
\pgfpathclose%
\pgfusepath{fill}%
\end{pgfscope}%
\begin{pgfscope}%
\pgfpathrectangle{\pgfqpoint{0.150000in}{0.150000in}}{\pgfqpoint{2.700000in}{1.950000in}}%
\pgfusepath{clip}%
\pgfsetbuttcap%
\pgfsetroundjoin%
\definecolor{currentfill}{rgb}{0.690380,0.438189,0.458165}%
\pgfsetfillcolor{currentfill}%
\pgfsetlinewidth{0.000000pt}%
\definecolor{currentstroke}{rgb}{0.000000,0.000000,0.000000}%
\pgfsetstrokecolor{currentstroke}%
\pgfsetdash{}{0pt}%
\pgfpathmoveto{\pgfqpoint{2.116661in}{0.906830in}}%
\pgfpathlineto{\pgfqpoint{2.150445in}{0.902346in}}%
\pgfpathlineto{\pgfqpoint{2.115286in}{0.956332in}}%
\pgfpathlineto{\pgfqpoint{2.080683in}{0.949248in}}%
\pgfpathclose%
\pgfusepath{fill}%
\end{pgfscope}%
\begin{pgfscope}%
\pgfpathrectangle{\pgfqpoint{0.150000in}{0.150000in}}{\pgfqpoint{2.700000in}{1.950000in}}%
\pgfusepath{clip}%
\pgfsetbuttcap%
\pgfsetroundjoin%
\definecolor{currentfill}{rgb}{0.675184,0.410616,0.431572}%
\pgfsetfillcolor{currentfill}%
\pgfsetlinewidth{0.000000pt}%
\definecolor{currentstroke}{rgb}{0.000000,0.000000,0.000000}%
\pgfsetstrokecolor{currentstroke}%
\pgfsetdash{}{0pt}%
\pgfpathmoveto{\pgfqpoint{2.154007in}{0.887849in}}%
\pgfpathlineto{\pgfqpoint{2.187700in}{0.883434in}}%
\pgfpathlineto{\pgfqpoint{2.150445in}{0.902346in}}%
\pgfpathlineto{\pgfqpoint{2.116661in}{0.906830in}}%
\pgfpathclose%
\pgfusepath{fill}%
\end{pgfscope}%
\begin{pgfscope}%
\pgfpathrectangle{\pgfqpoint{0.150000in}{0.150000in}}{\pgfqpoint{2.700000in}{1.950000in}}%
\pgfusepath{clip}%
\pgfsetbuttcap%
\pgfsetroundjoin%
\definecolor{currentfill}{rgb}{0.678983,0.417509,0.438220}%
\pgfsetfillcolor{currentfill}%
\pgfsetlinewidth{0.000000pt}%
\definecolor{currentstroke}{rgb}{0.000000,0.000000,0.000000}%
\pgfsetstrokecolor{currentstroke}%
\pgfsetdash{}{0pt}%
\pgfpathmoveto{\pgfqpoint{2.300177in}{0.899586in}}%
\pgfpathlineto{\pgfqpoint{2.334226in}{0.906830in}}%
\pgfpathlineto{\pgfqpoint{2.294047in}{0.890687in}}%
\pgfpathlineto{\pgfqpoint{2.260880in}{0.895118in}}%
\pgfpathclose%
\pgfusepath{fill}%
\end{pgfscope}%
\begin{pgfscope}%
\pgfpathrectangle{\pgfqpoint{0.150000in}{0.150000in}}{\pgfqpoint{2.700000in}{1.950000in}}%
\pgfusepath{clip}%
\pgfsetbuttcap%
\pgfsetroundjoin%
\definecolor{currentfill}{rgb}{0.770159,0.582950,0.597779}%
\pgfsetfillcolor{currentfill}%
\pgfsetlinewidth{0.000000pt}%
\definecolor{currentstroke}{rgb}{0.000000,0.000000,0.000000}%
\pgfsetstrokecolor{currentstroke}%
\pgfsetdash{}{0pt}%
\pgfpathmoveto{\pgfqpoint{2.044040in}{1.052939in}}%
\pgfpathlineto{\pgfqpoint{2.078724in}{1.059718in}}%
\pgfpathlineto{\pgfqpoint{2.043414in}{1.114015in}}%
\pgfpathlineto{\pgfqpoint{2.008570in}{1.107383in}}%
\pgfpathclose%
\pgfusepath{fill}%
\end{pgfscope}%
\begin{pgfscope}%
\pgfpathrectangle{\pgfqpoint{0.150000in}{0.150000in}}{\pgfqpoint{2.700000in}{1.950000in}}%
\pgfusepath{clip}%
\pgfsetbuttcap%
\pgfsetroundjoin%
\definecolor{currentfill}{rgb}{0.827145,0.686351,0.697503}%
\pgfsetfillcolor{currentfill}%
\pgfsetlinewidth{0.000000pt}%
\definecolor{currentstroke}{rgb}{0.000000,0.000000,0.000000}%
\pgfsetstrokecolor{currentstroke}%
\pgfsetdash{}{0pt}%
\pgfpathmoveto{\pgfqpoint{2.007986in}{1.168493in}}%
\pgfpathlineto{\pgfqpoint{2.042789in}{1.174941in}}%
\pgfpathlineto{\pgfqpoint{2.007404in}{1.229452in}}%
\pgfpathlineto{\pgfqpoint{1.972440in}{1.223152in}}%
\pgfpathclose%
\pgfusepath{fill}%
\end{pgfscope}%
\begin{pgfscope}%
\pgfpathrectangle{\pgfqpoint{0.150000in}{0.150000in}}{\pgfqpoint{2.700000in}{1.950000in}}%
\pgfusepath{clip}%
\pgfsetbuttcap%
\pgfsetroundjoin%
\definecolor{currentfill}{rgb}{0.952512,0.913833,0.916896}%
\pgfsetfillcolor{currentfill}%
\pgfsetlinewidth{0.000000pt}%
\definecolor{currentstroke}{rgb}{0.000000,0.000000,0.000000}%
\pgfsetstrokecolor{currentstroke}%
\pgfsetdash{}{0pt}%
\pgfpathmoveto{\pgfqpoint{1.754400in}{1.442616in}}%
\pgfpathlineto{\pgfqpoint{1.790272in}{1.460620in}}%
\pgfpathlineto{\pgfqpoint{1.753635in}{1.478560in}}%
\pgfpathlineto{\pgfqpoint{1.717762in}{1.460620in}}%
\pgfpathclose%
\pgfusepath{fill}%
\end{pgfscope}%
\begin{pgfscope}%
\pgfpathrectangle{\pgfqpoint{0.150000in}{0.150000in}}{\pgfqpoint{2.700000in}{1.950000in}}%
\pgfusepath{clip}%
\pgfsetbuttcap%
\pgfsetroundjoin%
\definecolor{currentfill}{rgb}{0.952512,0.913833,0.916896}%
\pgfsetfillcolor{currentfill}%
\pgfsetlinewidth{0.000000pt}%
\definecolor{currentstroke}{rgb}{0.000000,0.000000,0.000000}%
\pgfsetstrokecolor{currentstroke}%
\pgfsetdash{}{0pt}%
\pgfpathmoveto{\pgfqpoint{1.681762in}{1.442616in}}%
\pgfpathlineto{\pgfqpoint{1.717762in}{1.460620in}}%
\pgfpathlineto{\pgfqpoint{1.681252in}{1.478560in}}%
\pgfpathlineto{\pgfqpoint{1.645252in}{1.460620in}}%
\pgfpathclose%
\pgfusepath{fill}%
\end{pgfscope}%
\begin{pgfscope}%
\pgfpathrectangle{\pgfqpoint{0.150000in}{0.150000in}}{\pgfqpoint{2.700000in}{1.950000in}}%
\pgfusepath{clip}%
\pgfsetbuttcap%
\pgfsetroundjoin%
\definecolor{currentfill}{rgb}{0.952512,0.913833,0.916896}%
\pgfsetfillcolor{currentfill}%
\pgfsetlinewidth{0.000000pt}%
\definecolor{currentstroke}{rgb}{0.000000,0.000000,0.000000}%
\pgfsetstrokecolor{currentstroke}%
\pgfsetdash{}{0pt}%
\pgfpathmoveto{\pgfqpoint{1.609124in}{1.442616in}}%
\pgfpathlineto{\pgfqpoint{1.645252in}{1.460620in}}%
\pgfpathlineto{\pgfqpoint{1.608869in}{1.478560in}}%
\pgfpathlineto{\pgfqpoint{1.572742in}{1.460620in}}%
\pgfpathclose%
\pgfusepath{fill}%
\end{pgfscope}%
\begin{pgfscope}%
\pgfpathrectangle{\pgfqpoint{0.150000in}{0.150000in}}{\pgfqpoint{2.700000in}{1.950000in}}%
\pgfusepath{clip}%
\pgfsetbuttcap%
\pgfsetroundjoin%
\definecolor{currentfill}{rgb}{0.952512,0.913833,0.916896}%
\pgfsetfillcolor{currentfill}%
\pgfsetlinewidth{0.000000pt}%
\definecolor{currentstroke}{rgb}{0.000000,0.000000,0.000000}%
\pgfsetstrokecolor{currentstroke}%
\pgfsetdash{}{0pt}%
\pgfpathmoveto{\pgfqpoint{1.536486in}{1.442616in}}%
\pgfpathlineto{\pgfqpoint{1.572742in}{1.460620in}}%
\pgfpathlineto{\pgfqpoint{1.536486in}{1.478560in}}%
\pgfpathlineto{\pgfqpoint{1.500231in}{1.460620in}}%
\pgfpathclose%
\pgfusepath{fill}%
\end{pgfscope}%
\begin{pgfscope}%
\pgfpathrectangle{\pgfqpoint{0.150000in}{0.150000in}}{\pgfqpoint{2.700000in}{1.950000in}}%
\pgfusepath{clip}%
\pgfsetbuttcap%
\pgfsetroundjoin%
\definecolor{currentfill}{rgb}{0.952512,0.913833,0.916896}%
\pgfsetfillcolor{currentfill}%
\pgfsetlinewidth{0.000000pt}%
\definecolor{currentstroke}{rgb}{0.000000,0.000000,0.000000}%
\pgfsetstrokecolor{currentstroke}%
\pgfsetdash{}{0pt}%
\pgfpathmoveto{\pgfqpoint{1.463849in}{1.442616in}}%
\pgfpathlineto{\pgfqpoint{1.500231in}{1.460620in}}%
\pgfpathlineto{\pgfqpoint{1.464104in}{1.478560in}}%
\pgfpathlineto{\pgfqpoint{1.427721in}{1.460620in}}%
\pgfpathclose%
\pgfusepath{fill}%
\end{pgfscope}%
\begin{pgfscope}%
\pgfpathrectangle{\pgfqpoint{0.150000in}{0.150000in}}{\pgfqpoint{2.700000in}{1.950000in}}%
\pgfusepath{clip}%
\pgfsetbuttcap%
\pgfsetroundjoin%
\definecolor{currentfill}{rgb}{0.952512,0.913833,0.916896}%
\pgfsetfillcolor{currentfill}%
\pgfsetlinewidth{0.000000pt}%
\definecolor{currentstroke}{rgb}{0.000000,0.000000,0.000000}%
\pgfsetstrokecolor{currentstroke}%
\pgfsetdash{}{0pt}%
\pgfpathmoveto{\pgfqpoint{1.391211in}{1.442616in}}%
\pgfpathlineto{\pgfqpoint{1.427721in}{1.460620in}}%
\pgfpathlineto{\pgfqpoint{1.391721in}{1.478560in}}%
\pgfpathlineto{\pgfqpoint{1.355211in}{1.460620in}}%
\pgfpathclose%
\pgfusepath{fill}%
\end{pgfscope}%
\begin{pgfscope}%
\pgfpathrectangle{\pgfqpoint{0.150000in}{0.150000in}}{\pgfqpoint{2.700000in}{1.950000in}}%
\pgfusepath{clip}%
\pgfsetbuttcap%
\pgfsetroundjoin%
\definecolor{currentfill}{rgb}{0.952512,0.913833,0.916896}%
\pgfsetfillcolor{currentfill}%
\pgfsetlinewidth{0.000000pt}%
\definecolor{currentstroke}{rgb}{0.000000,0.000000,0.000000}%
\pgfsetstrokecolor{currentstroke}%
\pgfsetdash{}{0pt}%
\pgfpathmoveto{\pgfqpoint{1.318573in}{1.442616in}}%
\pgfpathlineto{\pgfqpoint{1.355211in}{1.460620in}}%
\pgfpathlineto{\pgfqpoint{1.319338in}{1.478560in}}%
\pgfpathlineto{\pgfqpoint{1.282701in}{1.460620in}}%
\pgfpathclose%
\pgfusepath{fill}%
\end{pgfscope}%
\begin{pgfscope}%
\pgfpathrectangle{\pgfqpoint{0.150000in}{0.150000in}}{\pgfqpoint{2.700000in}{1.950000in}}%
\pgfusepath{clip}%
\pgfsetbuttcap%
\pgfsetroundjoin%
\definecolor{currentfill}{rgb}{0.720772,0.493336,0.511351}%
\pgfsetfillcolor{currentfill}%
\pgfsetlinewidth{0.000000pt}%
\definecolor{currentstroke}{rgb}{0.000000,0.000000,0.000000}%
\pgfsetstrokecolor{currentstroke}%
\pgfsetdash{}{0pt}%
\pgfpathmoveto{\pgfqpoint{2.080683in}{0.949248in}}%
\pgfpathlineto{\pgfqpoint{2.115286in}{0.956332in}}%
\pgfpathlineto{\pgfqpoint{2.080011in}{1.010497in}}%
\pgfpathlineto{\pgfqpoint{2.045248in}{1.003559in}}%
\pgfpathclose%
\pgfusepath{fill}%
\end{pgfscope}%
\begin{pgfscope}%
\pgfpathrectangle{\pgfqpoint{0.150000in}{0.150000in}}{\pgfqpoint{2.700000in}{1.950000in}}%
\pgfusepath{clip}%
\pgfsetbuttcap%
\pgfsetroundjoin%
\definecolor{currentfill}{rgb}{0.804350,0.644991,0.657613}%
\pgfsetfillcolor{currentfill}%
\pgfsetlinewidth{0.000000pt}%
\definecolor{currentstroke}{rgb}{0.000000,0.000000,0.000000}%
\pgfsetstrokecolor{currentstroke}%
\pgfsetdash{}{0pt}%
\pgfpathmoveto{\pgfqpoint{2.008570in}{1.107383in}}%
\pgfpathlineto{\pgfqpoint{2.043414in}{1.114015in}}%
\pgfpathlineto{\pgfqpoint{2.007986in}{1.168493in}}%
\pgfpathlineto{\pgfqpoint{1.972981in}{1.162007in}}%
\pgfpathclose%
\pgfusepath{fill}%
\end{pgfscope}%
\begin{pgfscope}%
\pgfpathrectangle{\pgfqpoint{0.150000in}{0.150000in}}{\pgfqpoint{2.700000in}{1.950000in}}%
\pgfusepath{clip}%
\pgfsetbuttcap%
\pgfsetroundjoin%
\definecolor{currentfill}{rgb}{0.857537,0.741498,0.750689}%
\pgfsetfillcolor{currentfill}%
\pgfsetlinewidth{0.000000pt}%
\definecolor{currentstroke}{rgb}{0.000000,0.000000,0.000000}%
\pgfsetstrokecolor{currentstroke}%
\pgfsetdash{}{0pt}%
\pgfpathmoveto{\pgfqpoint{1.972440in}{1.223152in}}%
\pgfpathlineto{\pgfqpoint{2.007404in}{1.229452in}}%
\pgfpathlineto{\pgfqpoint{1.971901in}{1.284145in}}%
\pgfpathlineto{\pgfqpoint{1.936319in}{1.265816in}}%
\pgfpathclose%
\pgfusepath{fill}%
\end{pgfscope}%
\begin{pgfscope}%
\pgfpathrectangle{\pgfqpoint{0.150000in}{0.150000in}}{\pgfqpoint{2.700000in}{1.950000in}}%
\pgfusepath{clip}%
\pgfsetbuttcap%
\pgfsetroundjoin%
\definecolor{currentfill}{rgb}{0.751164,0.548483,0.564537}%
\pgfsetfillcolor{currentfill}%
\pgfsetlinewidth{0.000000pt}%
\definecolor{currentstroke}{rgb}{0.000000,0.000000,0.000000}%
\pgfsetstrokecolor{currentstroke}%
\pgfsetdash{}{0pt}%
\pgfpathmoveto{\pgfqpoint{2.045248in}{1.003559in}}%
\pgfpathlineto{\pgfqpoint{2.080011in}{1.010497in}}%
\pgfpathlineto{\pgfqpoint{2.044040in}{1.052939in}}%
\pgfpathlineto{\pgfqpoint{2.009155in}{1.046121in}}%
\pgfpathclose%
\pgfusepath{fill}%
\end{pgfscope}%
\begin{pgfscope}%
\pgfpathrectangle{\pgfqpoint{0.150000in}{0.150000in}}{\pgfqpoint{2.700000in}{1.950000in}}%
\pgfusepath{clip}%
\pgfsetbuttcap%
\pgfsetroundjoin%
\definecolor{currentfill}{rgb}{0.952512,0.913833,0.916896}%
\pgfsetfillcolor{currentfill}%
\pgfsetlinewidth{0.000000pt}%
\definecolor{currentstroke}{rgb}{0.000000,0.000000,0.000000}%
\pgfsetstrokecolor{currentstroke}%
\pgfsetdash{}{0pt}%
\pgfpathmoveto{\pgfqpoint{1.791167in}{1.424549in}}%
\pgfpathlineto{\pgfqpoint{1.827037in}{1.442616in}}%
\pgfpathlineto{\pgfqpoint{1.790272in}{1.460620in}}%
\pgfpathlineto{\pgfqpoint{1.754400in}{1.442616in}}%
\pgfpathclose%
\pgfusepath{fill}%
\end{pgfscope}%
\begin{pgfscope}%
\pgfpathrectangle{\pgfqpoint{0.150000in}{0.150000in}}{\pgfqpoint{2.700000in}{1.950000in}}%
\pgfusepath{clip}%
\pgfsetbuttcap%
\pgfsetroundjoin%
\definecolor{currentfill}{rgb}{0.952512,0.913833,0.916896}%
\pgfsetfillcolor{currentfill}%
\pgfsetlinewidth{0.000000pt}%
\definecolor{currentstroke}{rgb}{0.000000,0.000000,0.000000}%
\pgfsetstrokecolor{currentstroke}%
\pgfsetdash{}{0pt}%
\pgfpathmoveto{\pgfqpoint{1.718401in}{1.424549in}}%
\pgfpathlineto{\pgfqpoint{1.754400in}{1.442616in}}%
\pgfpathlineto{\pgfqpoint{1.717762in}{1.460620in}}%
\pgfpathlineto{\pgfqpoint{1.681762in}{1.442616in}}%
\pgfpathclose%
\pgfusepath{fill}%
\end{pgfscope}%
\begin{pgfscope}%
\pgfpathrectangle{\pgfqpoint{0.150000in}{0.150000in}}{\pgfqpoint{2.700000in}{1.950000in}}%
\pgfusepath{clip}%
\pgfsetbuttcap%
\pgfsetroundjoin%
\definecolor{currentfill}{rgb}{0.952512,0.913833,0.916896}%
\pgfsetfillcolor{currentfill}%
\pgfsetlinewidth{0.000000pt}%
\definecolor{currentstroke}{rgb}{0.000000,0.000000,0.000000}%
\pgfsetstrokecolor{currentstroke}%
\pgfsetdash{}{0pt}%
\pgfpathmoveto{\pgfqpoint{1.645635in}{1.424549in}}%
\pgfpathlineto{\pgfqpoint{1.681762in}{1.442616in}}%
\pgfpathlineto{\pgfqpoint{1.645252in}{1.460620in}}%
\pgfpathlineto{\pgfqpoint{1.609124in}{1.442616in}}%
\pgfpathclose%
\pgfusepath{fill}%
\end{pgfscope}%
\begin{pgfscope}%
\pgfpathrectangle{\pgfqpoint{0.150000in}{0.150000in}}{\pgfqpoint{2.700000in}{1.950000in}}%
\pgfusepath{clip}%
\pgfsetbuttcap%
\pgfsetroundjoin%
\definecolor{currentfill}{rgb}{0.952512,0.913833,0.916896}%
\pgfsetfillcolor{currentfill}%
\pgfsetlinewidth{0.000000pt}%
\definecolor{currentstroke}{rgb}{0.000000,0.000000,0.000000}%
\pgfsetstrokecolor{currentstroke}%
\pgfsetdash{}{0pt}%
\pgfpathmoveto{\pgfqpoint{1.572869in}{1.424549in}}%
\pgfpathlineto{\pgfqpoint{1.609124in}{1.442616in}}%
\pgfpathlineto{\pgfqpoint{1.572742in}{1.460620in}}%
\pgfpathlineto{\pgfqpoint{1.536486in}{1.442616in}}%
\pgfpathclose%
\pgfusepath{fill}%
\end{pgfscope}%
\begin{pgfscope}%
\pgfpathrectangle{\pgfqpoint{0.150000in}{0.150000in}}{\pgfqpoint{2.700000in}{1.950000in}}%
\pgfusepath{clip}%
\pgfsetbuttcap%
\pgfsetroundjoin%
\definecolor{currentfill}{rgb}{0.952512,0.913833,0.916896}%
\pgfsetfillcolor{currentfill}%
\pgfsetlinewidth{0.000000pt}%
\definecolor{currentstroke}{rgb}{0.000000,0.000000,0.000000}%
\pgfsetstrokecolor{currentstroke}%
\pgfsetdash{}{0pt}%
\pgfpathmoveto{\pgfqpoint{1.500104in}{1.424549in}}%
\pgfpathlineto{\pgfqpoint{1.536486in}{1.442616in}}%
\pgfpathlineto{\pgfqpoint{1.500231in}{1.460620in}}%
\pgfpathlineto{\pgfqpoint{1.463849in}{1.442616in}}%
\pgfpathclose%
\pgfusepath{fill}%
\end{pgfscope}%
\begin{pgfscope}%
\pgfpathrectangle{\pgfqpoint{0.150000in}{0.150000in}}{\pgfqpoint{2.700000in}{1.950000in}}%
\pgfusepath{clip}%
\pgfsetbuttcap%
\pgfsetroundjoin%
\definecolor{currentfill}{rgb}{0.952512,0.913833,0.916896}%
\pgfsetfillcolor{currentfill}%
\pgfsetlinewidth{0.000000pt}%
\definecolor{currentstroke}{rgb}{0.000000,0.000000,0.000000}%
\pgfsetstrokecolor{currentstroke}%
\pgfsetdash{}{0pt}%
\pgfpathmoveto{\pgfqpoint{1.427338in}{1.424549in}}%
\pgfpathlineto{\pgfqpoint{1.463849in}{1.442616in}}%
\pgfpathlineto{\pgfqpoint{1.427721in}{1.460620in}}%
\pgfpathlineto{\pgfqpoint{1.391211in}{1.442616in}}%
\pgfpathclose%
\pgfusepath{fill}%
\end{pgfscope}%
\begin{pgfscope}%
\pgfpathrectangle{\pgfqpoint{0.150000in}{0.150000in}}{\pgfqpoint{2.700000in}{1.950000in}}%
\pgfusepath{clip}%
\pgfsetbuttcap%
\pgfsetroundjoin%
\definecolor{currentfill}{rgb}{0.952512,0.913833,0.916896}%
\pgfsetfillcolor{currentfill}%
\pgfsetlinewidth{0.000000pt}%
\definecolor{currentstroke}{rgb}{0.000000,0.000000,0.000000}%
\pgfsetstrokecolor{currentstroke}%
\pgfsetdash{}{0pt}%
\pgfpathmoveto{\pgfqpoint{1.354572in}{1.424549in}}%
\pgfpathlineto{\pgfqpoint{1.391211in}{1.442616in}}%
\pgfpathlineto{\pgfqpoint{1.355211in}{1.460620in}}%
\pgfpathlineto{\pgfqpoint{1.318573in}{1.442616in}}%
\pgfpathclose%
\pgfusepath{fill}%
\end{pgfscope}%
\begin{pgfscope}%
\pgfpathrectangle{\pgfqpoint{0.150000in}{0.150000in}}{\pgfqpoint{2.700000in}{1.950000in}}%
\pgfusepath{clip}%
\pgfsetbuttcap%
\pgfsetroundjoin%
\definecolor{currentfill}{rgb}{0.952512,0.913833,0.916896}%
\pgfsetfillcolor{currentfill}%
\pgfsetlinewidth{0.000000pt}%
\definecolor{currentstroke}{rgb}{0.000000,0.000000,0.000000}%
\pgfsetstrokecolor{currentstroke}%
\pgfsetdash{}{0pt}%
\pgfpathmoveto{\pgfqpoint{1.281806in}{1.424549in}}%
\pgfpathlineto{\pgfqpoint{1.318573in}{1.442616in}}%
\pgfpathlineto{\pgfqpoint{1.282701in}{1.460620in}}%
\pgfpathlineto{\pgfqpoint{1.245936in}{1.442616in}}%
\pgfpathclose%
\pgfusepath{fill}%
\end{pgfscope}%
\begin{pgfscope}%
\pgfpathrectangle{\pgfqpoint{0.150000in}{0.150000in}}{\pgfqpoint{2.700000in}{1.950000in}}%
\pgfusepath{clip}%
\pgfsetbuttcap%
\pgfsetroundjoin%
\definecolor{currentfill}{rgb}{0.686581,0.431296,0.451517}%
\pgfsetfillcolor{currentfill}%
\pgfsetlinewidth{0.000000pt}%
\definecolor{currentstroke}{rgb}{0.000000,0.000000,0.000000}%
\pgfsetstrokecolor{currentstroke}%
\pgfsetdash{}{0pt}%
\pgfpathmoveto{\pgfqpoint{2.192985in}{0.892299in}}%
\pgfpathlineto{\pgfqpoint{2.226656in}{0.887849in}}%
\pgfpathlineto{\pgfqpoint{2.187700in}{0.883434in}}%
\pgfpathlineto{\pgfqpoint{2.154007in}{0.887849in}}%
\pgfpathclose%
\pgfusepath{fill}%
\end{pgfscope}%
\begin{pgfscope}%
\pgfpathrectangle{\pgfqpoint{0.150000in}{0.150000in}}{\pgfqpoint{2.700000in}{1.950000in}}%
\pgfusepath{clip}%
\pgfsetbuttcap%
\pgfsetroundjoin%
\definecolor{currentfill}{rgb}{0.781556,0.603631,0.617724}%
\pgfsetfillcolor{currentfill}%
\pgfsetlinewidth{0.000000pt}%
\definecolor{currentstroke}{rgb}{0.000000,0.000000,0.000000}%
\pgfsetstrokecolor{currentstroke}%
\pgfsetdash{}{0pt}%
\pgfpathmoveto{\pgfqpoint{2.009155in}{1.046121in}}%
\pgfpathlineto{\pgfqpoint{2.044040in}{1.052939in}}%
\pgfpathlineto{\pgfqpoint{2.008570in}{1.107383in}}%
\pgfpathlineto{\pgfqpoint{1.973523in}{1.100711in}}%
\pgfpathclose%
\pgfusepath{fill}%
\end{pgfscope}%
\begin{pgfscope}%
\pgfpathrectangle{\pgfqpoint{0.150000in}{0.150000in}}{\pgfqpoint{2.700000in}{1.950000in}}%
\pgfusepath{clip}%
\pgfsetbuttcap%
\pgfsetroundjoin%
\definecolor{currentfill}{rgb}{0.834743,0.700138,0.710800}%
\pgfsetfillcolor{currentfill}%
\pgfsetlinewidth{0.000000pt}%
\definecolor{currentstroke}{rgb}{0.000000,0.000000,0.000000}%
\pgfsetstrokecolor{currentstroke}%
\pgfsetdash{}{0pt}%
\pgfpathmoveto{\pgfqpoint{1.972981in}{1.162007in}}%
\pgfpathlineto{\pgfqpoint{2.007986in}{1.168493in}}%
\pgfpathlineto{\pgfqpoint{1.972440in}{1.223152in}}%
\pgfpathlineto{\pgfqpoint{1.936816in}{1.204693in}}%
\pgfpathclose%
\pgfusepath{fill}%
\end{pgfscope}%
\begin{pgfscope}%
\pgfpathrectangle{\pgfqpoint{0.150000in}{0.150000in}}{\pgfqpoint{2.700000in}{1.950000in}}%
\pgfusepath{clip}%
\pgfsetbuttcap%
\pgfsetroundjoin%
\definecolor{currentfill}{rgb}{0.709375,0.472656,0.491406}%
\pgfsetfillcolor{currentfill}%
\pgfsetlinewidth{0.000000pt}%
\definecolor{currentstroke}{rgb}{0.000000,0.000000,0.000000}%
\pgfsetstrokecolor{currentstroke}%
\pgfsetdash{}{0pt}%
\pgfpathmoveto{\pgfqpoint{2.082603in}{0.911349in}}%
\pgfpathlineto{\pgfqpoint{2.116661in}{0.906830in}}%
\pgfpathlineto{\pgfqpoint{2.080683in}{0.949248in}}%
\pgfpathlineto{\pgfqpoint{2.046462in}{0.953943in}}%
\pgfpathclose%
\pgfusepath{fill}%
\end{pgfscope}%
\begin{pgfscope}%
\pgfpathrectangle{\pgfqpoint{0.150000in}{0.150000in}}{\pgfqpoint{2.700000in}{1.950000in}}%
\pgfusepath{clip}%
\pgfsetbuttcap%
\pgfsetroundjoin%
\definecolor{currentfill}{rgb}{0.891728,0.803539,0.810524}%
\pgfsetfillcolor{currentfill}%
\pgfsetlinewidth{0.000000pt}%
\definecolor{currentstroke}{rgb}{0.000000,0.000000,0.000000}%
\pgfsetstrokecolor{currentstroke}%
\pgfsetdash{}{0pt}%
\pgfpathmoveto{\pgfqpoint{1.936319in}{1.265816in}}%
\pgfpathlineto{\pgfqpoint{1.971901in}{1.284145in}}%
\pgfpathlineto{\pgfqpoint{1.936280in}{1.339019in}}%
\pgfpathlineto{\pgfqpoint{1.900577in}{1.320757in}}%
\pgfpathclose%
\pgfusepath{fill}%
\end{pgfscope}%
\begin{pgfscope}%
\pgfpathrectangle{\pgfqpoint{0.150000in}{0.150000in}}{\pgfqpoint{2.700000in}{1.950000in}}%
\pgfusepath{clip}%
\pgfsetbuttcap%
\pgfsetroundjoin%
\definecolor{currentfill}{rgb}{0.697978,0.451976,0.471461}%
\pgfsetfillcolor{currentfill}%
\pgfsetlinewidth{0.000000pt}%
\definecolor{currentstroke}{rgb}{0.000000,0.000000,0.000000}%
\pgfsetstrokecolor{currentstroke}%
\pgfsetdash{}{0pt}%
\pgfpathmoveto{\pgfqpoint{2.266765in}{0.904089in}}%
\pgfpathlineto{\pgfqpoint{2.300177in}{0.899586in}}%
\pgfpathlineto{\pgfqpoint{2.260880in}{0.895118in}}%
\pgfpathlineto{\pgfqpoint{2.226656in}{0.887849in}}%
\pgfpathclose%
\pgfusepath{fill}%
\end{pgfscope}%
\begin{pgfscope}%
\pgfpathrectangle{\pgfqpoint{0.150000in}{0.150000in}}{\pgfqpoint{2.700000in}{1.950000in}}%
\pgfusepath{clip}%
\pgfsetbuttcap%
\pgfsetroundjoin%
\definecolor{currentfill}{rgb}{0.697978,0.451976,0.471461}%
\pgfsetfillcolor{currentfill}%
\pgfsetlinewidth{0.000000pt}%
\definecolor{currentstroke}{rgb}{0.000000,0.000000,0.000000}%
\pgfsetstrokecolor{currentstroke}%
\pgfsetdash{}{0pt}%
\pgfpathmoveto{\pgfqpoint{2.120040in}{0.892299in}}%
\pgfpathlineto{\pgfqpoint{2.154007in}{0.887849in}}%
\pgfpathlineto{\pgfqpoint{2.116661in}{0.906830in}}%
\pgfpathlineto{\pgfqpoint{2.082603in}{0.911349in}}%
\pgfpathclose%
\pgfusepath{fill}%
\end{pgfscope}%
\begin{pgfscope}%
\pgfpathrectangle{\pgfqpoint{0.150000in}{0.150000in}}{\pgfqpoint{2.700000in}{1.950000in}}%
\pgfusepath{clip}%
\pgfsetbuttcap%
\pgfsetroundjoin%
\definecolor{currentfill}{rgb}{0.735968,0.520910,0.537944}%
\pgfsetfillcolor{currentfill}%
\pgfsetlinewidth{0.000000pt}%
\definecolor{currentstroke}{rgb}{0.000000,0.000000,0.000000}%
\pgfsetstrokecolor{currentstroke}%
\pgfsetdash{}{0pt}%
\pgfpathmoveto{\pgfqpoint{2.046462in}{0.953943in}}%
\pgfpathlineto{\pgfqpoint{2.080683in}{0.949248in}}%
\pgfpathlineto{\pgfqpoint{2.045248in}{1.003559in}}%
\pgfpathlineto{\pgfqpoint{2.010283in}{0.996580in}}%
\pgfpathclose%
\pgfusepath{fill}%
\end{pgfscope}%
\begin{pgfscope}%
\pgfpathrectangle{\pgfqpoint{0.150000in}{0.150000in}}{\pgfqpoint{2.700000in}{1.950000in}}%
\pgfusepath{clip}%
\pgfsetbuttcap%
\pgfsetroundjoin%
\definecolor{currentfill}{rgb}{0.865135,0.755285,0.763986}%
\pgfsetfillcolor{currentfill}%
\pgfsetlinewidth{0.000000pt}%
\definecolor{currentstroke}{rgb}{0.000000,0.000000,0.000000}%
\pgfsetstrokecolor{currentstroke}%
\pgfsetdash{}{0pt}%
\pgfpathmoveto{\pgfqpoint{1.936816in}{1.204693in}}%
\pgfpathlineto{\pgfqpoint{1.972440in}{1.223152in}}%
\pgfpathlineto{\pgfqpoint{1.936319in}{1.265816in}}%
\pgfpathlineto{\pgfqpoint{1.901029in}{1.259599in}}%
\pgfpathclose%
\pgfusepath{fill}%
\end{pgfscope}%
\begin{pgfscope}%
\pgfpathrectangle{\pgfqpoint{0.150000in}{0.150000in}}{\pgfqpoint{2.700000in}{1.950000in}}%
\pgfusepath{clip}%
\pgfsetbuttcap%
\pgfsetroundjoin%
\definecolor{currentfill}{rgb}{0.952512,0.913833,0.916896}%
\pgfsetfillcolor{currentfill}%
\pgfsetlinewidth{0.000000pt}%
\definecolor{currentstroke}{rgb}{0.000000,0.000000,0.000000}%
\pgfsetstrokecolor{currentstroke}%
\pgfsetdash{}{0pt}%
\pgfpathmoveto{\pgfqpoint{1.828063in}{1.406419in}}%
\pgfpathlineto{\pgfqpoint{1.863932in}{1.424549in}}%
\pgfpathlineto{\pgfqpoint{1.827037in}{1.442616in}}%
\pgfpathlineto{\pgfqpoint{1.791167in}{1.424549in}}%
\pgfpathclose%
\pgfusepath{fill}%
\end{pgfscope}%
\begin{pgfscope}%
\pgfpathrectangle{\pgfqpoint{0.150000in}{0.150000in}}{\pgfqpoint{2.700000in}{1.950000in}}%
\pgfusepath{clip}%
\pgfsetbuttcap%
\pgfsetroundjoin%
\definecolor{currentfill}{rgb}{0.952512,0.913833,0.916896}%
\pgfsetfillcolor{currentfill}%
\pgfsetlinewidth{0.000000pt}%
\definecolor{currentstroke}{rgb}{0.000000,0.000000,0.000000}%
\pgfsetstrokecolor{currentstroke}%
\pgfsetdash{}{0pt}%
\pgfpathmoveto{\pgfqpoint{1.755169in}{1.406419in}}%
\pgfpathlineto{\pgfqpoint{1.791167in}{1.424549in}}%
\pgfpathlineto{\pgfqpoint{1.754400in}{1.442616in}}%
\pgfpathlineto{\pgfqpoint{1.718401in}{1.424549in}}%
\pgfpathclose%
\pgfusepath{fill}%
\end{pgfscope}%
\begin{pgfscope}%
\pgfpathrectangle{\pgfqpoint{0.150000in}{0.150000in}}{\pgfqpoint{2.700000in}{1.950000in}}%
\pgfusepath{clip}%
\pgfsetbuttcap%
\pgfsetroundjoin%
\definecolor{currentfill}{rgb}{0.952512,0.913833,0.916896}%
\pgfsetfillcolor{currentfill}%
\pgfsetlinewidth{0.000000pt}%
\definecolor{currentstroke}{rgb}{0.000000,0.000000,0.000000}%
\pgfsetstrokecolor{currentstroke}%
\pgfsetdash{}{0pt}%
\pgfpathmoveto{\pgfqpoint{1.682275in}{1.406419in}}%
\pgfpathlineto{\pgfqpoint{1.718401in}{1.424549in}}%
\pgfpathlineto{\pgfqpoint{1.681762in}{1.442616in}}%
\pgfpathlineto{\pgfqpoint{1.645635in}{1.424549in}}%
\pgfpathclose%
\pgfusepath{fill}%
\end{pgfscope}%
\begin{pgfscope}%
\pgfpathrectangle{\pgfqpoint{0.150000in}{0.150000in}}{\pgfqpoint{2.700000in}{1.950000in}}%
\pgfusepath{clip}%
\pgfsetbuttcap%
\pgfsetroundjoin%
\definecolor{currentfill}{rgb}{0.952512,0.913833,0.916896}%
\pgfsetfillcolor{currentfill}%
\pgfsetlinewidth{0.000000pt}%
\definecolor{currentstroke}{rgb}{0.000000,0.000000,0.000000}%
\pgfsetstrokecolor{currentstroke}%
\pgfsetdash{}{0pt}%
\pgfpathmoveto{\pgfqpoint{1.609381in}{1.406419in}}%
\pgfpathlineto{\pgfqpoint{1.645635in}{1.424549in}}%
\pgfpathlineto{\pgfqpoint{1.609124in}{1.442616in}}%
\pgfpathlineto{\pgfqpoint{1.572869in}{1.424549in}}%
\pgfpathclose%
\pgfusepath{fill}%
\end{pgfscope}%
\begin{pgfscope}%
\pgfpathrectangle{\pgfqpoint{0.150000in}{0.150000in}}{\pgfqpoint{2.700000in}{1.950000in}}%
\pgfusepath{clip}%
\pgfsetbuttcap%
\pgfsetroundjoin%
\definecolor{currentfill}{rgb}{0.952512,0.913833,0.916896}%
\pgfsetfillcolor{currentfill}%
\pgfsetlinewidth{0.000000pt}%
\definecolor{currentstroke}{rgb}{0.000000,0.000000,0.000000}%
\pgfsetstrokecolor{currentstroke}%
\pgfsetdash{}{0pt}%
\pgfpathmoveto{\pgfqpoint{1.536486in}{1.406419in}}%
\pgfpathlineto{\pgfqpoint{1.572869in}{1.424549in}}%
\pgfpathlineto{\pgfqpoint{1.536486in}{1.442616in}}%
\pgfpathlineto{\pgfqpoint{1.500104in}{1.424549in}}%
\pgfpathclose%
\pgfusepath{fill}%
\end{pgfscope}%
\begin{pgfscope}%
\pgfpathrectangle{\pgfqpoint{0.150000in}{0.150000in}}{\pgfqpoint{2.700000in}{1.950000in}}%
\pgfusepath{clip}%
\pgfsetbuttcap%
\pgfsetroundjoin%
\definecolor{currentfill}{rgb}{0.952512,0.913833,0.916896}%
\pgfsetfillcolor{currentfill}%
\pgfsetlinewidth{0.000000pt}%
\definecolor{currentstroke}{rgb}{0.000000,0.000000,0.000000}%
\pgfsetstrokecolor{currentstroke}%
\pgfsetdash{}{0pt}%
\pgfpathmoveto{\pgfqpoint{1.463592in}{1.406419in}}%
\pgfpathlineto{\pgfqpoint{1.500104in}{1.424549in}}%
\pgfpathlineto{\pgfqpoint{1.463849in}{1.442616in}}%
\pgfpathlineto{\pgfqpoint{1.427338in}{1.424549in}}%
\pgfpathclose%
\pgfusepath{fill}%
\end{pgfscope}%
\begin{pgfscope}%
\pgfpathrectangle{\pgfqpoint{0.150000in}{0.150000in}}{\pgfqpoint{2.700000in}{1.950000in}}%
\pgfusepath{clip}%
\pgfsetbuttcap%
\pgfsetroundjoin%
\definecolor{currentfill}{rgb}{0.952512,0.913833,0.916896}%
\pgfsetfillcolor{currentfill}%
\pgfsetlinewidth{0.000000pt}%
\definecolor{currentstroke}{rgb}{0.000000,0.000000,0.000000}%
\pgfsetstrokecolor{currentstroke}%
\pgfsetdash{}{0pt}%
\pgfpathmoveto{\pgfqpoint{1.390698in}{1.406419in}}%
\pgfpathlineto{\pgfqpoint{1.427338in}{1.424549in}}%
\pgfpathlineto{\pgfqpoint{1.391211in}{1.442616in}}%
\pgfpathlineto{\pgfqpoint{1.354572in}{1.424549in}}%
\pgfpathclose%
\pgfusepath{fill}%
\end{pgfscope}%
\begin{pgfscope}%
\pgfpathrectangle{\pgfqpoint{0.150000in}{0.150000in}}{\pgfqpoint{2.700000in}{1.950000in}}%
\pgfusepath{clip}%
\pgfsetbuttcap%
\pgfsetroundjoin%
\definecolor{currentfill}{rgb}{0.952512,0.913833,0.916896}%
\pgfsetfillcolor{currentfill}%
\pgfsetlinewidth{0.000000pt}%
\definecolor{currentstroke}{rgb}{0.000000,0.000000,0.000000}%
\pgfsetstrokecolor{currentstroke}%
\pgfsetdash{}{0pt}%
\pgfpathmoveto{\pgfqpoint{1.317804in}{1.406419in}}%
\pgfpathlineto{\pgfqpoint{1.354572in}{1.424549in}}%
\pgfpathlineto{\pgfqpoint{1.318573in}{1.442616in}}%
\pgfpathlineto{\pgfqpoint{1.281806in}{1.424549in}}%
\pgfpathclose%
\pgfusepath{fill}%
\end{pgfscope}%
\begin{pgfscope}%
\pgfpathrectangle{\pgfqpoint{0.150000in}{0.150000in}}{\pgfqpoint{2.700000in}{1.950000in}}%
\pgfusepath{clip}%
\pgfsetbuttcap%
\pgfsetroundjoin%
\definecolor{currentfill}{rgb}{0.952512,0.913833,0.916896}%
\pgfsetfillcolor{currentfill}%
\pgfsetlinewidth{0.000000pt}%
\definecolor{currentstroke}{rgb}{0.000000,0.000000,0.000000}%
\pgfsetstrokecolor{currentstroke}%
\pgfsetdash{}{0pt}%
\pgfpathmoveto{\pgfqpoint{1.244909in}{1.406419in}}%
\pgfpathlineto{\pgfqpoint{1.281806in}{1.424549in}}%
\pgfpathlineto{\pgfqpoint{1.245936in}{1.442616in}}%
\pgfpathlineto{\pgfqpoint{1.209041in}{1.424549in}}%
\pgfpathclose%
\pgfusepath{fill}%
\end{pgfscope}%
\begin{pgfscope}%
\pgfpathrectangle{\pgfqpoint{0.150000in}{0.150000in}}{\pgfqpoint{2.700000in}{1.950000in}}%
\pgfusepath{clip}%
\pgfsetbuttcap%
\pgfsetroundjoin%
\definecolor{currentfill}{rgb}{0.811949,0.658778,0.670910}%
\pgfsetfillcolor{currentfill}%
\pgfsetlinewidth{0.000000pt}%
\definecolor{currentstroke}{rgb}{0.000000,0.000000,0.000000}%
\pgfsetstrokecolor{currentstroke}%
\pgfsetdash{}{0pt}%
\pgfpathmoveto{\pgfqpoint{1.973523in}{1.100711in}}%
\pgfpathlineto{\pgfqpoint{2.008570in}{1.107383in}}%
\pgfpathlineto{\pgfqpoint{1.972981in}{1.162007in}}%
\pgfpathlineto{\pgfqpoint{1.937313in}{1.143417in}}%
\pgfpathclose%
\pgfusepath{fill}%
\end{pgfscope}%
\begin{pgfscope}%
\pgfpathrectangle{\pgfqpoint{0.150000in}{0.150000in}}{\pgfqpoint{2.700000in}{1.950000in}}%
\pgfusepath{clip}%
\pgfsetbuttcap%
\pgfsetroundjoin%
\definecolor{currentfill}{rgb}{0.705576,0.465763,0.484758}%
\pgfsetfillcolor{currentfill}%
\pgfsetlinewidth{0.000000pt}%
\definecolor{currentstroke}{rgb}{0.000000,0.000000,0.000000}%
\pgfsetstrokecolor{currentstroke}%
\pgfsetdash{}{0pt}%
\pgfpathmoveto{\pgfqpoint{2.339793in}{0.904089in}}%
\pgfpathlineto{\pgfqpoint{2.373866in}{0.911349in}}%
\pgfpathlineto{\pgfqpoint{2.334226in}{0.906830in}}%
\pgfpathlineto{\pgfqpoint{2.300177in}{0.899586in}}%
\pgfpathclose%
\pgfusepath{fill}%
\end{pgfscope}%
\begin{pgfscope}%
\pgfpathrectangle{\pgfqpoint{0.150000in}{0.150000in}}{\pgfqpoint{2.700000in}{1.950000in}}%
\pgfusepath{clip}%
\pgfsetbuttcap%
\pgfsetroundjoin%
\definecolor{currentfill}{rgb}{0.762561,0.569164,0.584482}%
\pgfsetfillcolor{currentfill}%
\pgfsetlinewidth{0.000000pt}%
\definecolor{currentstroke}{rgb}{0.000000,0.000000,0.000000}%
\pgfsetstrokecolor{currentstroke}%
\pgfsetdash{}{0pt}%
\pgfpathmoveto{\pgfqpoint{2.010283in}{0.996580in}}%
\pgfpathlineto{\pgfqpoint{2.045248in}{1.003559in}}%
\pgfpathlineto{\pgfqpoint{2.009155in}{1.046121in}}%
\pgfpathlineto{\pgfqpoint{1.974066in}{1.039263in}}%
\pgfpathclose%
\pgfusepath{fill}%
\end{pgfscope}%
\begin{pgfscope}%
\pgfpathrectangle{\pgfqpoint{0.150000in}{0.150000in}}{\pgfqpoint{2.700000in}{1.950000in}}%
\pgfusepath{clip}%
\pgfsetbuttcap%
\pgfsetroundjoin%
\definecolor{currentfill}{rgb}{0.925919,0.865579,0.870358}%
\pgfsetfillcolor{currentfill}%
\pgfsetlinewidth{0.000000pt}%
\definecolor{currentstroke}{rgb}{0.000000,0.000000,0.000000}%
\pgfsetstrokecolor{currentstroke}%
\pgfsetdash{}{0pt}%
\pgfpathmoveto{\pgfqpoint{1.900577in}{1.320757in}}%
\pgfpathlineto{\pgfqpoint{1.936280in}{1.339019in}}%
\pgfpathlineto{\pgfqpoint{1.900541in}{1.394076in}}%
\pgfpathlineto{\pgfqpoint{1.864715in}{1.375880in}}%
\pgfpathclose%
\pgfusepath{fill}%
\end{pgfscope}%
\begin{pgfscope}%
\pgfpathrectangle{\pgfqpoint{0.150000in}{0.150000in}}{\pgfqpoint{2.700000in}{1.950000in}}%
\pgfusepath{clip}%
\pgfsetbuttcap%
\pgfsetroundjoin%
\definecolor{currentfill}{rgb}{0.891728,0.803539,0.810524}%
\pgfsetfillcolor{currentfill}%
\pgfsetlinewidth{0.000000pt}%
\definecolor{currentstroke}{rgb}{0.000000,0.000000,0.000000}%
\pgfsetstrokecolor{currentstroke}%
\pgfsetdash{}{0pt}%
\pgfpathmoveto{\pgfqpoint{1.901029in}{1.259599in}}%
\pgfpathlineto{\pgfqpoint{1.936319in}{1.265816in}}%
\pgfpathlineto{\pgfqpoint{1.900577in}{1.320757in}}%
\pgfpathlineto{\pgfqpoint{1.864747in}{1.302429in}}%
\pgfpathclose%
\pgfusepath{fill}%
\end{pgfscope}%
\begin{pgfscope}%
\pgfpathrectangle{\pgfqpoint{0.150000in}{0.150000in}}{\pgfqpoint{2.700000in}{1.950000in}}%
\pgfusepath{clip}%
\pgfsetbuttcap%
\pgfsetroundjoin%
\definecolor{currentfill}{rgb}{0.948713,0.906939,0.910248}%
\pgfsetfillcolor{currentfill}%
\pgfsetlinewidth{0.000000pt}%
\definecolor{currentstroke}{rgb}{0.000000,0.000000,0.000000}%
\pgfsetstrokecolor{currentstroke}%
\pgfsetdash{}{0pt}%
\pgfpathmoveto{\pgfqpoint{1.864715in}{1.375880in}}%
\pgfpathlineto{\pgfqpoint{1.900541in}{1.394076in}}%
\pgfpathlineto{\pgfqpoint{1.863932in}{1.424549in}}%
\pgfpathlineto{\pgfqpoint{1.828063in}{1.406419in}}%
\pgfpathclose%
\pgfusepath{fill}%
\end{pgfscope}%
\begin{pgfscope}%
\pgfpathrectangle{\pgfqpoint{0.150000in}{0.150000in}}{\pgfqpoint{2.700000in}{1.950000in}}%
\pgfusepath{clip}%
\pgfsetbuttcap%
\pgfsetroundjoin%
\definecolor{currentfill}{rgb}{0.842341,0.713925,0.724096}%
\pgfsetfillcolor{currentfill}%
\pgfsetlinewidth{0.000000pt}%
\definecolor{currentstroke}{rgb}{0.000000,0.000000,0.000000}%
\pgfsetstrokecolor{currentstroke}%
\pgfsetdash{}{0pt}%
\pgfpathmoveto{\pgfqpoint{1.937313in}{1.143417in}}%
\pgfpathlineto{\pgfqpoint{1.972981in}{1.162007in}}%
\pgfpathlineto{\pgfqpoint{1.936816in}{1.204693in}}%
\pgfpathlineto{\pgfqpoint{1.901483in}{1.198290in}}%
\pgfpathclose%
\pgfusepath{fill}%
\end{pgfscope}%
\begin{pgfscope}%
\pgfpathrectangle{\pgfqpoint{0.150000in}{0.150000in}}{\pgfqpoint{2.700000in}{1.950000in}}%
\pgfusepath{clip}%
\pgfsetbuttcap%
\pgfsetroundjoin%
\definecolor{currentfill}{rgb}{0.792953,0.624311,0.637669}%
\pgfsetfillcolor{currentfill}%
\pgfsetlinewidth{0.000000pt}%
\definecolor{currentstroke}{rgb}{0.000000,0.000000,0.000000}%
\pgfsetstrokecolor{currentstroke}%
\pgfsetdash{}{0pt}%
\pgfpathmoveto{\pgfqpoint{1.974066in}{1.039263in}}%
\pgfpathlineto{\pgfqpoint{2.009155in}{1.046121in}}%
\pgfpathlineto{\pgfqpoint{1.973523in}{1.100711in}}%
\pgfpathlineto{\pgfqpoint{1.938272in}{1.094001in}}%
\pgfpathclose%
\pgfusepath{fill}%
\end{pgfscope}%
\begin{pgfscope}%
\pgfpathrectangle{\pgfqpoint{0.150000in}{0.150000in}}{\pgfqpoint{2.700000in}{1.950000in}}%
\pgfusepath{clip}%
\pgfsetbuttcap%
\pgfsetroundjoin%
\definecolor{currentfill}{rgb}{0.952512,0.913833,0.916896}%
\pgfsetfillcolor{currentfill}%
\pgfsetlinewidth{0.000000pt}%
\definecolor{currentstroke}{rgb}{0.000000,0.000000,0.000000}%
\pgfsetstrokecolor{currentstroke}%
\pgfsetdash{}{0pt}%
\pgfpathmoveto{\pgfqpoint{1.792068in}{1.388224in}}%
\pgfpathlineto{\pgfqpoint{1.828063in}{1.406419in}}%
\pgfpathlineto{\pgfqpoint{1.791167in}{1.424549in}}%
\pgfpathlineto{\pgfqpoint{1.755169in}{1.406419in}}%
\pgfpathclose%
\pgfusepath{fill}%
\end{pgfscope}%
\begin{pgfscope}%
\pgfpathrectangle{\pgfqpoint{0.150000in}{0.150000in}}{\pgfqpoint{2.700000in}{1.950000in}}%
\pgfusepath{clip}%
\pgfsetbuttcap%
\pgfsetroundjoin%
\definecolor{currentfill}{rgb}{0.952512,0.913833,0.916896}%
\pgfsetfillcolor{currentfill}%
\pgfsetlinewidth{0.000000pt}%
\definecolor{currentstroke}{rgb}{0.000000,0.000000,0.000000}%
\pgfsetstrokecolor{currentstroke}%
\pgfsetdash{}{0pt}%
\pgfpathmoveto{\pgfqpoint{1.719044in}{1.388224in}}%
\pgfpathlineto{\pgfqpoint{1.755169in}{1.406419in}}%
\pgfpathlineto{\pgfqpoint{1.718401in}{1.424549in}}%
\pgfpathlineto{\pgfqpoint{1.682275in}{1.406419in}}%
\pgfpathclose%
\pgfusepath{fill}%
\end{pgfscope}%
\begin{pgfscope}%
\pgfpathrectangle{\pgfqpoint{0.150000in}{0.150000in}}{\pgfqpoint{2.700000in}{1.950000in}}%
\pgfusepath{clip}%
\pgfsetbuttcap%
\pgfsetroundjoin%
\definecolor{currentfill}{rgb}{0.952512,0.913833,0.916896}%
\pgfsetfillcolor{currentfill}%
\pgfsetlinewidth{0.000000pt}%
\definecolor{currentstroke}{rgb}{0.000000,0.000000,0.000000}%
\pgfsetstrokecolor{currentstroke}%
\pgfsetdash{}{0pt}%
\pgfpathmoveto{\pgfqpoint{1.646021in}{1.388224in}}%
\pgfpathlineto{\pgfqpoint{1.682275in}{1.406419in}}%
\pgfpathlineto{\pgfqpoint{1.645635in}{1.424549in}}%
\pgfpathlineto{\pgfqpoint{1.609381in}{1.406419in}}%
\pgfpathclose%
\pgfusepath{fill}%
\end{pgfscope}%
\begin{pgfscope}%
\pgfpathrectangle{\pgfqpoint{0.150000in}{0.150000in}}{\pgfqpoint{2.700000in}{1.950000in}}%
\pgfusepath{clip}%
\pgfsetbuttcap%
\pgfsetroundjoin%
\definecolor{currentfill}{rgb}{0.952512,0.913833,0.916896}%
\pgfsetfillcolor{currentfill}%
\pgfsetlinewidth{0.000000pt}%
\definecolor{currentstroke}{rgb}{0.000000,0.000000,0.000000}%
\pgfsetstrokecolor{currentstroke}%
\pgfsetdash{}{0pt}%
\pgfpathmoveto{\pgfqpoint{1.572998in}{1.388224in}}%
\pgfpathlineto{\pgfqpoint{1.609381in}{1.406419in}}%
\pgfpathlineto{\pgfqpoint{1.572869in}{1.424549in}}%
\pgfpathlineto{\pgfqpoint{1.536486in}{1.406419in}}%
\pgfpathclose%
\pgfusepath{fill}%
\end{pgfscope}%
\begin{pgfscope}%
\pgfpathrectangle{\pgfqpoint{0.150000in}{0.150000in}}{\pgfqpoint{2.700000in}{1.950000in}}%
\pgfusepath{clip}%
\pgfsetbuttcap%
\pgfsetroundjoin%
\definecolor{currentfill}{rgb}{0.952512,0.913833,0.916896}%
\pgfsetfillcolor{currentfill}%
\pgfsetlinewidth{0.000000pt}%
\definecolor{currentstroke}{rgb}{0.000000,0.000000,0.000000}%
\pgfsetstrokecolor{currentstroke}%
\pgfsetdash{}{0pt}%
\pgfpathmoveto{\pgfqpoint{1.499975in}{1.388224in}}%
\pgfpathlineto{\pgfqpoint{1.536486in}{1.406419in}}%
\pgfpathlineto{\pgfqpoint{1.500104in}{1.424549in}}%
\pgfpathlineto{\pgfqpoint{1.463592in}{1.406419in}}%
\pgfpathclose%
\pgfusepath{fill}%
\end{pgfscope}%
\begin{pgfscope}%
\pgfpathrectangle{\pgfqpoint{0.150000in}{0.150000in}}{\pgfqpoint{2.700000in}{1.950000in}}%
\pgfusepath{clip}%
\pgfsetbuttcap%
\pgfsetroundjoin%
\definecolor{currentfill}{rgb}{0.952512,0.913833,0.916896}%
\pgfsetfillcolor{currentfill}%
\pgfsetlinewidth{0.000000pt}%
\definecolor{currentstroke}{rgb}{0.000000,0.000000,0.000000}%
\pgfsetstrokecolor{currentstroke}%
\pgfsetdash{}{0pt}%
\pgfpathmoveto{\pgfqpoint{1.426952in}{1.388224in}}%
\pgfpathlineto{\pgfqpoint{1.463592in}{1.406419in}}%
\pgfpathlineto{\pgfqpoint{1.427338in}{1.424549in}}%
\pgfpathlineto{\pgfqpoint{1.390698in}{1.406419in}}%
\pgfpathclose%
\pgfusepath{fill}%
\end{pgfscope}%
\begin{pgfscope}%
\pgfpathrectangle{\pgfqpoint{0.150000in}{0.150000in}}{\pgfqpoint{2.700000in}{1.950000in}}%
\pgfusepath{clip}%
\pgfsetbuttcap%
\pgfsetroundjoin%
\definecolor{currentfill}{rgb}{0.952512,0.913833,0.916896}%
\pgfsetfillcolor{currentfill}%
\pgfsetlinewidth{0.000000pt}%
\definecolor{currentstroke}{rgb}{0.000000,0.000000,0.000000}%
\pgfsetstrokecolor{currentstroke}%
\pgfsetdash{}{0pt}%
\pgfpathmoveto{\pgfqpoint{1.353928in}{1.388224in}}%
\pgfpathlineto{\pgfqpoint{1.390698in}{1.406419in}}%
\pgfpathlineto{\pgfqpoint{1.354572in}{1.424549in}}%
\pgfpathlineto{\pgfqpoint{1.317804in}{1.406419in}}%
\pgfpathclose%
\pgfusepath{fill}%
\end{pgfscope}%
\begin{pgfscope}%
\pgfpathrectangle{\pgfqpoint{0.150000in}{0.150000in}}{\pgfqpoint{2.700000in}{1.950000in}}%
\pgfusepath{clip}%
\pgfsetbuttcap%
\pgfsetroundjoin%
\definecolor{currentfill}{rgb}{0.952512,0.913833,0.916896}%
\pgfsetfillcolor{currentfill}%
\pgfsetlinewidth{0.000000pt}%
\definecolor{currentstroke}{rgb}{0.000000,0.000000,0.000000}%
\pgfsetstrokecolor{currentstroke}%
\pgfsetdash{}{0pt}%
\pgfpathmoveto{\pgfqpoint{1.280905in}{1.388224in}}%
\pgfpathlineto{\pgfqpoint{1.317804in}{1.406419in}}%
\pgfpathlineto{\pgfqpoint{1.281806in}{1.424549in}}%
\pgfpathlineto{\pgfqpoint{1.244909in}{1.406419in}}%
\pgfpathclose%
\pgfusepath{fill}%
\end{pgfscope}%
\begin{pgfscope}%
\pgfpathrectangle{\pgfqpoint{0.150000in}{0.150000in}}{\pgfqpoint{2.700000in}{1.950000in}}%
\pgfusepath{clip}%
\pgfsetbuttcap%
\pgfsetroundjoin%
\definecolor{currentfill}{rgb}{0.952512,0.913833,0.916896}%
\pgfsetfillcolor{currentfill}%
\pgfsetlinewidth{0.000000pt}%
\definecolor{currentstroke}{rgb}{0.000000,0.000000,0.000000}%
\pgfsetstrokecolor{currentstroke}%
\pgfsetdash{}{0pt}%
\pgfpathmoveto{\pgfqpoint{1.207882in}{1.388224in}}%
\pgfpathlineto{\pgfqpoint{1.244909in}{1.406419in}}%
\pgfpathlineto{\pgfqpoint{1.209041in}{1.424549in}}%
\pgfpathlineto{\pgfqpoint{1.172015in}{1.406419in}}%
\pgfpathclose%
\pgfusepath{fill}%
\end{pgfscope}%
\begin{pgfscope}%
\pgfpathrectangle{\pgfqpoint{0.150000in}{0.150000in}}{\pgfqpoint{2.700000in}{1.950000in}}%
\pgfusepath{clip}%
\pgfsetbuttcap%
\pgfsetroundjoin%
\definecolor{currentfill}{rgb}{0.709375,0.472656,0.491406}%
\pgfsetfillcolor{currentfill}%
\pgfsetlinewidth{0.000000pt}%
\definecolor{currentstroke}{rgb}{0.000000,0.000000,0.000000}%
\pgfsetstrokecolor{currentstroke}%
\pgfsetdash{}{0pt}%
\pgfpathmoveto{\pgfqpoint{2.158324in}{0.884971in}}%
\pgfpathlineto{\pgfqpoint{2.192985in}{0.892299in}}%
\pgfpathlineto{\pgfqpoint{2.154007in}{0.887849in}}%
\pgfpathlineto{\pgfqpoint{2.120040in}{0.892299in}}%
\pgfpathclose%
\pgfusepath{fill}%
\end{pgfscope}%
\begin{pgfscope}%
\pgfpathrectangle{\pgfqpoint{0.150000in}{0.150000in}}{\pgfqpoint{2.700000in}{1.950000in}}%
\pgfusepath{clip}%
\pgfsetbuttcap%
\pgfsetroundjoin%
\definecolor{currentfill}{rgb}{0.868934,0.762178,0.770634}%
\pgfsetfillcolor{currentfill}%
\pgfsetlinewidth{0.000000pt}%
\definecolor{currentstroke}{rgb}{0.000000,0.000000,0.000000}%
\pgfsetstrokecolor{currentstroke}%
\pgfsetdash{}{0pt}%
\pgfpathmoveto{\pgfqpoint{1.901483in}{1.198290in}}%
\pgfpathlineto{\pgfqpoint{1.936816in}{1.204693in}}%
\pgfpathlineto{\pgfqpoint{1.901029in}{1.259599in}}%
\pgfpathlineto{\pgfqpoint{1.865155in}{1.241141in}}%
\pgfpathclose%
\pgfusepath{fill}%
\end{pgfscope}%
\begin{pgfscope}%
\pgfpathrectangle{\pgfqpoint{0.150000in}{0.150000in}}{\pgfqpoint{2.700000in}{1.950000in}}%
\pgfusepath{clip}%
\pgfsetbuttcap%
\pgfsetroundjoin%
\definecolor{currentfill}{rgb}{0.728370,0.507123,0.524648}%
\pgfsetfillcolor{currentfill}%
\pgfsetlinewidth{0.000000pt}%
\definecolor{currentstroke}{rgb}{0.000000,0.000000,0.000000}%
\pgfsetstrokecolor{currentstroke}%
\pgfsetdash{}{0pt}%
\pgfpathmoveto{\pgfqpoint{2.047682in}{0.904089in}}%
\pgfpathlineto{\pgfqpoint{2.082603in}{0.911349in}}%
\pgfpathlineto{\pgfqpoint{2.046462in}{0.953943in}}%
\pgfpathlineto{\pgfqpoint{2.011417in}{0.946803in}}%
\pgfpathclose%
\pgfusepath{fill}%
\end{pgfscope}%
\begin{pgfscope}%
\pgfpathrectangle{\pgfqpoint{0.150000in}{0.150000in}}{\pgfqpoint{2.700000in}{1.950000in}}%
\pgfusepath{clip}%
\pgfsetbuttcap%
\pgfsetroundjoin%
\definecolor{currentfill}{rgb}{0.925919,0.865579,0.870358}%
\pgfsetfillcolor{currentfill}%
\pgfsetlinewidth{0.000000pt}%
\definecolor{currentstroke}{rgb}{0.000000,0.000000,0.000000}%
\pgfsetstrokecolor{currentstroke}%
\pgfsetdash{}{0pt}%
\pgfpathmoveto{\pgfqpoint{1.864747in}{1.302429in}}%
\pgfpathlineto{\pgfqpoint{1.900577in}{1.320757in}}%
\pgfpathlineto{\pgfqpoint{1.864715in}{1.375880in}}%
\pgfpathlineto{\pgfqpoint{1.828761in}{1.357620in}}%
\pgfpathclose%
\pgfusepath{fill}%
\end{pgfscope}%
\begin{pgfscope}%
\pgfpathrectangle{\pgfqpoint{0.150000in}{0.150000in}}{\pgfqpoint{2.700000in}{1.950000in}}%
\pgfusepath{clip}%
\pgfsetbuttcap%
\pgfsetroundjoin%
\definecolor{currentfill}{rgb}{0.819547,0.672564,0.684206}%
\pgfsetfillcolor{currentfill}%
\pgfsetlinewidth{0.000000pt}%
\definecolor{currentstroke}{rgb}{0.000000,0.000000,0.000000}%
\pgfsetstrokecolor{currentstroke}%
\pgfsetdash{}{0pt}%
\pgfpathmoveto{\pgfqpoint{1.938272in}{1.094001in}}%
\pgfpathlineto{\pgfqpoint{1.973523in}{1.100711in}}%
\pgfpathlineto{\pgfqpoint{1.937313in}{1.143417in}}%
\pgfpathlineto{\pgfqpoint{1.901937in}{1.136828in}}%
\pgfpathclose%
\pgfusepath{fill}%
\end{pgfscope}%
\begin{pgfscope}%
\pgfpathrectangle{\pgfqpoint{0.150000in}{0.150000in}}{\pgfqpoint{2.700000in}{1.950000in}}%
\pgfusepath{clip}%
\pgfsetbuttcap%
\pgfsetroundjoin%
\definecolor{currentfill}{rgb}{0.713174,0.479550,0.498055}%
\pgfsetfillcolor{currentfill}%
\pgfsetlinewidth{0.000000pt}%
\definecolor{currentstroke}{rgb}{0.000000,0.000000,0.000000}%
\pgfsetstrokecolor{currentstroke}%
\pgfsetdash{}{0pt}%
\pgfpathmoveto{\pgfqpoint{2.232280in}{0.896786in}}%
\pgfpathlineto{\pgfqpoint{2.266765in}{0.904089in}}%
\pgfpathlineto{\pgfqpoint{2.226656in}{0.887849in}}%
\pgfpathlineto{\pgfqpoint{2.192985in}{0.892299in}}%
\pgfpathclose%
\pgfusepath{fill}%
\end{pgfscope}%
\begin{pgfscope}%
\pgfpathrectangle{\pgfqpoint{0.150000in}{0.150000in}}{\pgfqpoint{2.700000in}{1.950000in}}%
\pgfusepath{clip}%
\pgfsetbuttcap%
\pgfsetroundjoin%
\definecolor{currentfill}{rgb}{0.751164,0.548483,0.564537}%
\pgfsetfillcolor{currentfill}%
\pgfsetlinewidth{0.000000pt}%
\definecolor{currentstroke}{rgb}{0.000000,0.000000,0.000000}%
\pgfsetstrokecolor{currentstroke}%
\pgfsetdash{}{0pt}%
\pgfpathmoveto{\pgfqpoint{2.011417in}{0.946803in}}%
\pgfpathlineto{\pgfqpoint{2.046462in}{0.953943in}}%
\pgfpathlineto{\pgfqpoint{2.010283in}{0.996580in}}%
\pgfpathlineto{\pgfqpoint{1.975114in}{0.989561in}}%
\pgfpathclose%
\pgfusepath{fill}%
\end{pgfscope}%
\begin{pgfscope}%
\pgfpathrectangle{\pgfqpoint{0.150000in}{0.150000in}}{\pgfqpoint{2.700000in}{1.950000in}}%
\pgfusepath{clip}%
\pgfsetbuttcap%
\pgfsetroundjoin%
\definecolor{currentfill}{rgb}{0.948713,0.906939,0.910248}%
\pgfsetfillcolor{currentfill}%
\pgfsetlinewidth{0.000000pt}%
\definecolor{currentstroke}{rgb}{0.000000,0.000000,0.000000}%
\pgfsetstrokecolor{currentstroke}%
\pgfsetdash{}{0pt}%
\pgfpathmoveto{\pgfqpoint{1.828761in}{1.357620in}}%
\pgfpathlineto{\pgfqpoint{1.864715in}{1.375880in}}%
\pgfpathlineto{\pgfqpoint{1.828063in}{1.406419in}}%
\pgfpathlineto{\pgfqpoint{1.792068in}{1.388224in}}%
\pgfpathclose%
\pgfusepath{fill}%
\end{pgfscope}%
\begin{pgfscope}%
\pgfpathrectangle{\pgfqpoint{0.150000in}{0.150000in}}{\pgfqpoint{2.700000in}{1.950000in}}%
\pgfusepath{clip}%
\pgfsetbuttcap%
\pgfsetroundjoin%
\definecolor{currentfill}{rgb}{0.716973,0.486443,0.504703}%
\pgfsetfillcolor{currentfill}%
\pgfsetlinewidth{0.000000pt}%
\definecolor{currentstroke}{rgb}{0.000000,0.000000,0.000000}%
\pgfsetstrokecolor{currentstroke}%
\pgfsetdash{}{0pt}%
\pgfpathmoveto{\pgfqpoint{2.085166in}{0.884971in}}%
\pgfpathlineto{\pgfqpoint{2.120040in}{0.892299in}}%
\pgfpathlineto{\pgfqpoint{2.082603in}{0.911349in}}%
\pgfpathlineto{\pgfqpoint{2.047682in}{0.904089in}}%
\pgfpathclose%
\pgfusepath{fill}%
\end{pgfscope}%
\begin{pgfscope}%
\pgfpathrectangle{\pgfqpoint{0.150000in}{0.150000in}}{\pgfqpoint{2.700000in}{1.950000in}}%
\pgfusepath{clip}%
\pgfsetbuttcap%
\pgfsetroundjoin%
\definecolor{currentfill}{rgb}{0.899326,0.817325,0.823820}%
\pgfsetfillcolor{currentfill}%
\pgfsetlinewidth{0.000000pt}%
\definecolor{currentstroke}{rgb}{0.000000,0.000000,0.000000}%
\pgfsetstrokecolor{currentstroke}%
\pgfsetdash{}{0pt}%
\pgfpathmoveto{\pgfqpoint{1.865155in}{1.241141in}}%
\pgfpathlineto{\pgfqpoint{1.901029in}{1.259599in}}%
\pgfpathlineto{\pgfqpoint{1.864747in}{1.302429in}}%
\pgfpathlineto{\pgfqpoint{1.829126in}{1.296298in}}%
\pgfpathclose%
\pgfusepath{fill}%
\end{pgfscope}%
\begin{pgfscope}%
\pgfpathrectangle{\pgfqpoint{0.150000in}{0.150000in}}{\pgfqpoint{2.700000in}{1.950000in}}%
\pgfusepath{clip}%
\pgfsetbuttcap%
\pgfsetroundjoin%
\definecolor{currentfill}{rgb}{0.952512,0.913833,0.916896}%
\pgfsetfillcolor{currentfill}%
\pgfsetlinewidth{0.000000pt}%
\definecolor{currentstroke}{rgb}{0.000000,0.000000,0.000000}%
\pgfsetstrokecolor{currentstroke}%
\pgfsetdash{}{0pt}%
\pgfpathmoveto{\pgfqpoint{1.755944in}{1.369965in}}%
\pgfpathlineto{\pgfqpoint{1.792068in}{1.388224in}}%
\pgfpathlineto{\pgfqpoint{1.755169in}{1.406419in}}%
\pgfpathlineto{\pgfqpoint{1.719044in}{1.388224in}}%
\pgfpathclose%
\pgfusepath{fill}%
\end{pgfscope}%
\begin{pgfscope}%
\pgfpathrectangle{\pgfqpoint{0.150000in}{0.150000in}}{\pgfqpoint{2.700000in}{1.950000in}}%
\pgfusepath{clip}%
\pgfsetbuttcap%
\pgfsetroundjoin%
\definecolor{currentfill}{rgb}{0.952512,0.913833,0.916896}%
\pgfsetfillcolor{currentfill}%
\pgfsetlinewidth{0.000000pt}%
\definecolor{currentstroke}{rgb}{0.000000,0.000000,0.000000}%
\pgfsetstrokecolor{currentstroke}%
\pgfsetdash{}{0pt}%
\pgfpathmoveto{\pgfqpoint{1.682792in}{1.369965in}}%
\pgfpathlineto{\pgfqpoint{1.719044in}{1.388224in}}%
\pgfpathlineto{\pgfqpoint{1.682275in}{1.406419in}}%
\pgfpathlineto{\pgfqpoint{1.646021in}{1.388224in}}%
\pgfpathclose%
\pgfusepath{fill}%
\end{pgfscope}%
\begin{pgfscope}%
\pgfpathrectangle{\pgfqpoint{0.150000in}{0.150000in}}{\pgfqpoint{2.700000in}{1.950000in}}%
\pgfusepath{clip}%
\pgfsetbuttcap%
\pgfsetroundjoin%
\definecolor{currentfill}{rgb}{0.952512,0.913833,0.916896}%
\pgfsetfillcolor{currentfill}%
\pgfsetlinewidth{0.000000pt}%
\definecolor{currentstroke}{rgb}{0.000000,0.000000,0.000000}%
\pgfsetstrokecolor{currentstroke}%
\pgfsetdash{}{0pt}%
\pgfpathmoveto{\pgfqpoint{1.609639in}{1.369965in}}%
\pgfpathlineto{\pgfqpoint{1.646021in}{1.388224in}}%
\pgfpathlineto{\pgfqpoint{1.609381in}{1.406419in}}%
\pgfpathlineto{\pgfqpoint{1.572998in}{1.388224in}}%
\pgfpathclose%
\pgfusepath{fill}%
\end{pgfscope}%
\begin{pgfscope}%
\pgfpathrectangle{\pgfqpoint{0.150000in}{0.150000in}}{\pgfqpoint{2.700000in}{1.950000in}}%
\pgfusepath{clip}%
\pgfsetbuttcap%
\pgfsetroundjoin%
\definecolor{currentfill}{rgb}{0.952512,0.913833,0.916896}%
\pgfsetfillcolor{currentfill}%
\pgfsetlinewidth{0.000000pt}%
\definecolor{currentstroke}{rgb}{0.000000,0.000000,0.000000}%
\pgfsetstrokecolor{currentstroke}%
\pgfsetdash{}{0pt}%
\pgfpathmoveto{\pgfqpoint{1.536486in}{1.369965in}}%
\pgfpathlineto{\pgfqpoint{1.572998in}{1.388224in}}%
\pgfpathlineto{\pgfqpoint{1.536486in}{1.406419in}}%
\pgfpathlineto{\pgfqpoint{1.499975in}{1.388224in}}%
\pgfpathclose%
\pgfusepath{fill}%
\end{pgfscope}%
\begin{pgfscope}%
\pgfpathrectangle{\pgfqpoint{0.150000in}{0.150000in}}{\pgfqpoint{2.700000in}{1.950000in}}%
\pgfusepath{clip}%
\pgfsetbuttcap%
\pgfsetroundjoin%
\definecolor{currentfill}{rgb}{0.952512,0.913833,0.916896}%
\pgfsetfillcolor{currentfill}%
\pgfsetlinewidth{0.000000pt}%
\definecolor{currentstroke}{rgb}{0.000000,0.000000,0.000000}%
\pgfsetstrokecolor{currentstroke}%
\pgfsetdash{}{0pt}%
\pgfpathmoveto{\pgfqpoint{1.463334in}{1.369965in}}%
\pgfpathlineto{\pgfqpoint{1.499975in}{1.388224in}}%
\pgfpathlineto{\pgfqpoint{1.463592in}{1.406419in}}%
\pgfpathlineto{\pgfqpoint{1.426952in}{1.388224in}}%
\pgfpathclose%
\pgfusepath{fill}%
\end{pgfscope}%
\begin{pgfscope}%
\pgfpathrectangle{\pgfqpoint{0.150000in}{0.150000in}}{\pgfqpoint{2.700000in}{1.950000in}}%
\pgfusepath{clip}%
\pgfsetbuttcap%
\pgfsetroundjoin%
\definecolor{currentfill}{rgb}{0.952512,0.913833,0.916896}%
\pgfsetfillcolor{currentfill}%
\pgfsetlinewidth{0.000000pt}%
\definecolor{currentstroke}{rgb}{0.000000,0.000000,0.000000}%
\pgfsetstrokecolor{currentstroke}%
\pgfsetdash{}{0pt}%
\pgfpathmoveto{\pgfqpoint{1.390181in}{1.369965in}}%
\pgfpathlineto{\pgfqpoint{1.426952in}{1.388224in}}%
\pgfpathlineto{\pgfqpoint{1.390698in}{1.406419in}}%
\pgfpathlineto{\pgfqpoint{1.353928in}{1.388224in}}%
\pgfpathclose%
\pgfusepath{fill}%
\end{pgfscope}%
\begin{pgfscope}%
\pgfpathrectangle{\pgfqpoint{0.150000in}{0.150000in}}{\pgfqpoint{2.700000in}{1.950000in}}%
\pgfusepath{clip}%
\pgfsetbuttcap%
\pgfsetroundjoin%
\definecolor{currentfill}{rgb}{0.952512,0.913833,0.916896}%
\pgfsetfillcolor{currentfill}%
\pgfsetlinewidth{0.000000pt}%
\definecolor{currentstroke}{rgb}{0.000000,0.000000,0.000000}%
\pgfsetstrokecolor{currentstroke}%
\pgfsetdash{}{0pt}%
\pgfpathmoveto{\pgfqpoint{1.317029in}{1.369965in}}%
\pgfpathlineto{\pgfqpoint{1.353928in}{1.388224in}}%
\pgfpathlineto{\pgfqpoint{1.317804in}{1.406419in}}%
\pgfpathlineto{\pgfqpoint{1.280905in}{1.388224in}}%
\pgfpathclose%
\pgfusepath{fill}%
\end{pgfscope}%
\begin{pgfscope}%
\pgfpathrectangle{\pgfqpoint{0.150000in}{0.150000in}}{\pgfqpoint{2.700000in}{1.950000in}}%
\pgfusepath{clip}%
\pgfsetbuttcap%
\pgfsetroundjoin%
\definecolor{currentfill}{rgb}{0.952512,0.913833,0.916896}%
\pgfsetfillcolor{currentfill}%
\pgfsetlinewidth{0.000000pt}%
\definecolor{currentstroke}{rgb}{0.000000,0.000000,0.000000}%
\pgfsetstrokecolor{currentstroke}%
\pgfsetdash{}{0pt}%
\pgfpathmoveto{\pgfqpoint{1.243876in}{1.369965in}}%
\pgfpathlineto{\pgfqpoint{1.280905in}{1.388224in}}%
\pgfpathlineto{\pgfqpoint{1.244909in}{1.406419in}}%
\pgfpathlineto{\pgfqpoint{1.207882in}{1.388224in}}%
\pgfpathclose%
\pgfusepath{fill}%
\end{pgfscope}%
\begin{pgfscope}%
\pgfpathrectangle{\pgfqpoint{0.150000in}{0.150000in}}{\pgfqpoint{2.700000in}{1.950000in}}%
\pgfusepath{clip}%
\pgfsetbuttcap%
\pgfsetroundjoin%
\definecolor{currentfill}{rgb}{0.952512,0.913833,0.916896}%
\pgfsetfillcolor{currentfill}%
\pgfsetlinewidth{0.000000pt}%
\definecolor{currentstroke}{rgb}{0.000000,0.000000,0.000000}%
\pgfsetstrokecolor{currentstroke}%
\pgfsetdash{}{0pt}%
\pgfpathmoveto{\pgfqpoint{1.170723in}{1.369965in}}%
\pgfpathlineto{\pgfqpoint{1.207882in}{1.388224in}}%
\pgfpathlineto{\pgfqpoint{1.172015in}{1.406419in}}%
\pgfpathlineto{\pgfqpoint{1.134859in}{1.388224in}}%
\pgfpathclose%
\pgfusepath{fill}%
\end{pgfscope}%
\begin{pgfscope}%
\pgfpathrectangle{\pgfqpoint{0.150000in}{0.150000in}}{\pgfqpoint{2.700000in}{1.950000in}}%
\pgfusepath{clip}%
\pgfsetbuttcap%
\pgfsetroundjoin%
\definecolor{currentfill}{rgb}{0.846140,0.720818,0.730744}%
\pgfsetfillcolor{currentfill}%
\pgfsetlinewidth{0.000000pt}%
\definecolor{currentstroke}{rgb}{0.000000,0.000000,0.000000}%
\pgfsetstrokecolor{currentstroke}%
\pgfsetdash{}{0pt}%
\pgfpathmoveto{\pgfqpoint{1.901937in}{1.136828in}}%
\pgfpathlineto{\pgfqpoint{1.937313in}{1.143417in}}%
\pgfpathlineto{\pgfqpoint{1.901483in}{1.198290in}}%
\pgfpathlineto{\pgfqpoint{1.865565in}{1.179700in}}%
\pgfpathclose%
\pgfusepath{fill}%
\end{pgfscope}%
\begin{pgfscope}%
\pgfpathrectangle{\pgfqpoint{0.150000in}{0.150000in}}{\pgfqpoint{2.700000in}{1.950000in}}%
\pgfusepath{clip}%
\pgfsetbuttcap%
\pgfsetroundjoin%
\definecolor{currentfill}{rgb}{0.720772,0.493336,0.511351}%
\pgfsetfillcolor{currentfill}%
\pgfsetlinewidth{0.000000pt}%
\definecolor{currentstroke}{rgb}{0.000000,0.000000,0.000000}%
\pgfsetstrokecolor{currentstroke}%
\pgfsetdash{}{0pt}%
\pgfpathmoveto{\pgfqpoint{2.306407in}{0.908629in}}%
\pgfpathlineto{\pgfqpoint{2.339793in}{0.904089in}}%
\pgfpathlineto{\pgfqpoint{2.300177in}{0.899586in}}%
\pgfpathlineto{\pgfqpoint{2.266765in}{0.904089in}}%
\pgfpathclose%
\pgfusepath{fill}%
\end{pgfscope}%
\begin{pgfscope}%
\pgfpathrectangle{\pgfqpoint{0.150000in}{0.150000in}}{\pgfqpoint{2.700000in}{1.950000in}}%
\pgfusepath{clip}%
\pgfsetbuttcap%
\pgfsetroundjoin%
\definecolor{currentfill}{rgb}{0.777757,0.596737,0.611075}%
\pgfsetfillcolor{currentfill}%
\pgfsetlinewidth{0.000000pt}%
\definecolor{currentstroke}{rgb}{0.000000,0.000000,0.000000}%
\pgfsetstrokecolor{currentstroke}%
\pgfsetdash{}{0pt}%
\pgfpathmoveto{\pgfqpoint{1.975114in}{0.989561in}}%
\pgfpathlineto{\pgfqpoint{2.010283in}{0.996580in}}%
\pgfpathlineto{\pgfqpoint{1.974066in}{1.039263in}}%
\pgfpathlineto{\pgfqpoint{1.939235in}{1.044347in}}%
\pgfpathclose%
\pgfusepath{fill}%
\end{pgfscope}%
\begin{pgfscope}%
\pgfpathrectangle{\pgfqpoint{0.150000in}{0.150000in}}{\pgfqpoint{2.700000in}{1.950000in}}%
\pgfusepath{clip}%
\pgfsetbuttcap%
\pgfsetroundjoin%
\definecolor{currentfill}{rgb}{0.925919,0.865579,0.870358}%
\pgfsetfillcolor{currentfill}%
\pgfsetlinewidth{0.000000pt}%
\definecolor{currentstroke}{rgb}{0.000000,0.000000,0.000000}%
\pgfsetstrokecolor{currentstroke}%
\pgfsetdash{}{0pt}%
\pgfpathmoveto{\pgfqpoint{1.829126in}{1.296298in}}%
\pgfpathlineto{\pgfqpoint{1.864747in}{1.302429in}}%
\pgfpathlineto{\pgfqpoint{1.828761in}{1.357620in}}%
\pgfpathlineto{\pgfqpoint{1.792680in}{1.339295in}}%
\pgfpathclose%
\pgfusepath{fill}%
\end{pgfscope}%
\begin{pgfscope}%
\pgfpathrectangle{\pgfqpoint{0.150000in}{0.150000in}}{\pgfqpoint{2.700000in}{1.950000in}}%
\pgfusepath{clip}%
\pgfsetbuttcap%
\pgfsetroundjoin%
\definecolor{currentfill}{rgb}{0.876532,0.775965,0.783931}%
\pgfsetfillcolor{currentfill}%
\pgfsetlinewidth{0.000000pt}%
\definecolor{currentstroke}{rgb}{0.000000,0.000000,0.000000}%
\pgfsetstrokecolor{currentstroke}%
\pgfsetdash{}{0pt}%
\pgfpathmoveto{\pgfqpoint{1.865565in}{1.179700in}}%
\pgfpathlineto{\pgfqpoint{1.901483in}{1.198290in}}%
\pgfpathlineto{\pgfqpoint{1.865155in}{1.241141in}}%
\pgfpathlineto{\pgfqpoint{1.829491in}{1.234823in}}%
\pgfpathclose%
\pgfusepath{fill}%
\end{pgfscope}%
\begin{pgfscope}%
\pgfpathrectangle{\pgfqpoint{0.150000in}{0.150000in}}{\pgfqpoint{2.700000in}{1.950000in}}%
\pgfusepath{clip}%
\pgfsetbuttcap%
\pgfsetroundjoin%
\definecolor{currentfill}{rgb}{0.948713,0.906939,0.910248}%
\pgfsetfillcolor{currentfill}%
\pgfsetlinewidth{0.000000pt}%
\definecolor{currentstroke}{rgb}{0.000000,0.000000,0.000000}%
\pgfsetstrokecolor{currentstroke}%
\pgfsetdash{}{0pt}%
\pgfpathmoveto{\pgfqpoint{1.792680in}{1.339295in}}%
\pgfpathlineto{\pgfqpoint{1.828761in}{1.357620in}}%
\pgfpathlineto{\pgfqpoint{1.792068in}{1.388224in}}%
\pgfpathlineto{\pgfqpoint{1.755944in}{1.369965in}}%
\pgfpathclose%
\pgfusepath{fill}%
\end{pgfscope}%
\begin{pgfscope}%
\pgfpathrectangle{\pgfqpoint{0.150000in}{0.150000in}}{\pgfqpoint{2.700000in}{1.950000in}}%
\pgfusepath{clip}%
\pgfsetbuttcap%
\pgfsetroundjoin%
\definecolor{currentfill}{rgb}{0.732169,0.514017,0.531296}%
\pgfsetfillcolor{currentfill}%
\pgfsetlinewidth{0.000000pt}%
\definecolor{currentstroke}{rgb}{0.000000,0.000000,0.000000}%
\pgfsetstrokecolor{currentstroke}%
\pgfsetdash{}{0pt}%
\pgfpathmoveto{\pgfqpoint{2.380705in}{0.920500in}}%
\pgfpathlineto{\pgfqpoint{2.414836in}{0.927750in}}%
\pgfpathlineto{\pgfqpoint{2.373866in}{0.911349in}}%
\pgfpathlineto{\pgfqpoint{2.339793in}{0.904089in}}%
\pgfpathclose%
\pgfusepath{fill}%
\end{pgfscope}%
\begin{pgfscope}%
\pgfpathrectangle{\pgfqpoint{0.150000in}{0.150000in}}{\pgfqpoint{2.700000in}{1.950000in}}%
\pgfusepath{clip}%
\pgfsetbuttcap%
\pgfsetroundjoin%
\definecolor{currentfill}{rgb}{0.804350,0.644991,0.657613}%
\pgfsetfillcolor{currentfill}%
\pgfsetlinewidth{0.000000pt}%
\definecolor{currentstroke}{rgb}{0.000000,0.000000,0.000000}%
\pgfsetstrokecolor{currentstroke}%
\pgfsetdash{}{0pt}%
\pgfpathmoveto{\pgfqpoint{1.939235in}{1.044347in}}%
\pgfpathlineto{\pgfqpoint{1.974066in}{1.039263in}}%
\pgfpathlineto{\pgfqpoint{1.938272in}{1.094001in}}%
\pgfpathlineto{\pgfqpoint{1.902814in}{1.087251in}}%
\pgfpathclose%
\pgfusepath{fill}%
\end{pgfscope}%
\begin{pgfscope}%
\pgfpathrectangle{\pgfqpoint{0.150000in}{0.150000in}}{\pgfqpoint{2.700000in}{1.950000in}}%
\pgfusepath{clip}%
\pgfsetbuttcap%
\pgfsetroundjoin%
\definecolor{currentfill}{rgb}{0.952512,0.913833,0.916896}%
\pgfsetfillcolor{currentfill}%
\pgfsetlinewidth{0.000000pt}%
\definecolor{currentstroke}{rgb}{0.000000,0.000000,0.000000}%
\pgfsetstrokecolor{currentstroke}%
\pgfsetdash{}{0pt}%
\pgfpathmoveto{\pgfqpoint{1.719693in}{1.351640in}}%
\pgfpathlineto{\pgfqpoint{1.755944in}{1.369965in}}%
\pgfpathlineto{\pgfqpoint{1.719044in}{1.388224in}}%
\pgfpathlineto{\pgfqpoint{1.682792in}{1.369965in}}%
\pgfpathclose%
\pgfusepath{fill}%
\end{pgfscope}%
\begin{pgfscope}%
\pgfpathrectangle{\pgfqpoint{0.150000in}{0.150000in}}{\pgfqpoint{2.700000in}{1.950000in}}%
\pgfusepath{clip}%
\pgfsetbuttcap%
\pgfsetroundjoin%
\definecolor{currentfill}{rgb}{0.952512,0.913833,0.916896}%
\pgfsetfillcolor{currentfill}%
\pgfsetlinewidth{0.000000pt}%
\definecolor{currentstroke}{rgb}{0.000000,0.000000,0.000000}%
\pgfsetstrokecolor{currentstroke}%
\pgfsetdash{}{0pt}%
\pgfpathmoveto{\pgfqpoint{1.646410in}{1.351640in}}%
\pgfpathlineto{\pgfqpoint{1.682792in}{1.369965in}}%
\pgfpathlineto{\pgfqpoint{1.646021in}{1.388224in}}%
\pgfpathlineto{\pgfqpoint{1.609639in}{1.369965in}}%
\pgfpathclose%
\pgfusepath{fill}%
\end{pgfscope}%
\begin{pgfscope}%
\pgfpathrectangle{\pgfqpoint{0.150000in}{0.150000in}}{\pgfqpoint{2.700000in}{1.950000in}}%
\pgfusepath{clip}%
\pgfsetbuttcap%
\pgfsetroundjoin%
\definecolor{currentfill}{rgb}{0.952512,0.913833,0.916896}%
\pgfsetfillcolor{currentfill}%
\pgfsetlinewidth{0.000000pt}%
\definecolor{currentstroke}{rgb}{0.000000,0.000000,0.000000}%
\pgfsetstrokecolor{currentstroke}%
\pgfsetdash{}{0pt}%
\pgfpathmoveto{\pgfqpoint{1.573128in}{1.351640in}}%
\pgfpathlineto{\pgfqpoint{1.609639in}{1.369965in}}%
\pgfpathlineto{\pgfqpoint{1.572998in}{1.388224in}}%
\pgfpathlineto{\pgfqpoint{1.536486in}{1.369965in}}%
\pgfpathclose%
\pgfusepath{fill}%
\end{pgfscope}%
\begin{pgfscope}%
\pgfpathrectangle{\pgfqpoint{0.150000in}{0.150000in}}{\pgfqpoint{2.700000in}{1.950000in}}%
\pgfusepath{clip}%
\pgfsetbuttcap%
\pgfsetroundjoin%
\definecolor{currentfill}{rgb}{0.952512,0.913833,0.916896}%
\pgfsetfillcolor{currentfill}%
\pgfsetlinewidth{0.000000pt}%
\definecolor{currentstroke}{rgb}{0.000000,0.000000,0.000000}%
\pgfsetstrokecolor{currentstroke}%
\pgfsetdash{}{0pt}%
\pgfpathmoveto{\pgfqpoint{1.499845in}{1.351640in}}%
\pgfpathlineto{\pgfqpoint{1.536486in}{1.369965in}}%
\pgfpathlineto{\pgfqpoint{1.499975in}{1.388224in}}%
\pgfpathlineto{\pgfqpoint{1.463334in}{1.369965in}}%
\pgfpathclose%
\pgfusepath{fill}%
\end{pgfscope}%
\begin{pgfscope}%
\pgfpathrectangle{\pgfqpoint{0.150000in}{0.150000in}}{\pgfqpoint{2.700000in}{1.950000in}}%
\pgfusepath{clip}%
\pgfsetbuttcap%
\pgfsetroundjoin%
\definecolor{currentfill}{rgb}{0.952512,0.913833,0.916896}%
\pgfsetfillcolor{currentfill}%
\pgfsetlinewidth{0.000000pt}%
\definecolor{currentstroke}{rgb}{0.000000,0.000000,0.000000}%
\pgfsetstrokecolor{currentstroke}%
\pgfsetdash{}{0pt}%
\pgfpathmoveto{\pgfqpoint{1.426563in}{1.351640in}}%
\pgfpathlineto{\pgfqpoint{1.463334in}{1.369965in}}%
\pgfpathlineto{\pgfqpoint{1.426952in}{1.388224in}}%
\pgfpathlineto{\pgfqpoint{1.390181in}{1.369965in}}%
\pgfpathclose%
\pgfusepath{fill}%
\end{pgfscope}%
\begin{pgfscope}%
\pgfpathrectangle{\pgfqpoint{0.150000in}{0.150000in}}{\pgfqpoint{2.700000in}{1.950000in}}%
\pgfusepath{clip}%
\pgfsetbuttcap%
\pgfsetroundjoin%
\definecolor{currentfill}{rgb}{0.952512,0.913833,0.916896}%
\pgfsetfillcolor{currentfill}%
\pgfsetlinewidth{0.000000pt}%
\definecolor{currentstroke}{rgb}{0.000000,0.000000,0.000000}%
\pgfsetstrokecolor{currentstroke}%
\pgfsetdash{}{0pt}%
\pgfpathmoveto{\pgfqpoint{1.353280in}{1.351640in}}%
\pgfpathlineto{\pgfqpoint{1.390181in}{1.369965in}}%
\pgfpathlineto{\pgfqpoint{1.353928in}{1.388224in}}%
\pgfpathlineto{\pgfqpoint{1.317029in}{1.369965in}}%
\pgfpathclose%
\pgfusepath{fill}%
\end{pgfscope}%
\begin{pgfscope}%
\pgfpathrectangle{\pgfqpoint{0.150000in}{0.150000in}}{\pgfqpoint{2.700000in}{1.950000in}}%
\pgfusepath{clip}%
\pgfsetbuttcap%
\pgfsetroundjoin%
\definecolor{currentfill}{rgb}{0.952512,0.913833,0.916896}%
\pgfsetfillcolor{currentfill}%
\pgfsetlinewidth{0.000000pt}%
\definecolor{currentstroke}{rgb}{0.000000,0.000000,0.000000}%
\pgfsetstrokecolor{currentstroke}%
\pgfsetdash{}{0pt}%
\pgfpathmoveto{\pgfqpoint{1.279998in}{1.351640in}}%
\pgfpathlineto{\pgfqpoint{1.317029in}{1.369965in}}%
\pgfpathlineto{\pgfqpoint{1.280905in}{1.388224in}}%
\pgfpathlineto{\pgfqpoint{1.243876in}{1.369965in}}%
\pgfpathclose%
\pgfusepath{fill}%
\end{pgfscope}%
\begin{pgfscope}%
\pgfpathrectangle{\pgfqpoint{0.150000in}{0.150000in}}{\pgfqpoint{2.700000in}{1.950000in}}%
\pgfusepath{clip}%
\pgfsetbuttcap%
\pgfsetroundjoin%
\definecolor{currentfill}{rgb}{0.952512,0.913833,0.916896}%
\pgfsetfillcolor{currentfill}%
\pgfsetlinewidth{0.000000pt}%
\definecolor{currentstroke}{rgb}{0.000000,0.000000,0.000000}%
\pgfsetstrokecolor{currentstroke}%
\pgfsetdash{}{0pt}%
\pgfpathmoveto{\pgfqpoint{1.206715in}{1.351640in}}%
\pgfpathlineto{\pgfqpoint{1.243876in}{1.369965in}}%
\pgfpathlineto{\pgfqpoint{1.207882in}{1.388224in}}%
\pgfpathlineto{\pgfqpoint{1.170723in}{1.369965in}}%
\pgfpathclose%
\pgfusepath{fill}%
\end{pgfscope}%
\begin{pgfscope}%
\pgfpathrectangle{\pgfqpoint{0.150000in}{0.150000in}}{\pgfqpoint{2.700000in}{1.950000in}}%
\pgfusepath{clip}%
\pgfsetbuttcap%
\pgfsetroundjoin%
\definecolor{currentfill}{rgb}{0.952512,0.913833,0.916896}%
\pgfsetfillcolor{currentfill}%
\pgfsetlinewidth{0.000000pt}%
\definecolor{currentstroke}{rgb}{0.000000,0.000000,0.000000}%
\pgfsetstrokecolor{currentstroke}%
\pgfsetdash{}{0pt}%
\pgfpathmoveto{\pgfqpoint{1.133433in}{1.351640in}}%
\pgfpathlineto{\pgfqpoint{1.170723in}{1.369965in}}%
\pgfpathlineto{\pgfqpoint{1.134859in}{1.388224in}}%
\pgfpathlineto{\pgfqpoint{1.097571in}{1.369965in}}%
\pgfpathclose%
\pgfusepath{fill}%
\end{pgfscope}%
\begin{pgfscope}%
\pgfpathrectangle{\pgfqpoint{0.150000in}{0.150000in}}{\pgfqpoint{2.700000in}{1.950000in}}%
\pgfusepath{clip}%
\pgfsetbuttcap%
\pgfsetroundjoin%
\definecolor{currentfill}{rgb}{0.830944,0.693244,0.704151}%
\pgfsetfillcolor{currentfill}%
\pgfsetlinewidth{0.000000pt}%
\definecolor{currentstroke}{rgb}{0.000000,0.000000,0.000000}%
\pgfsetstrokecolor{currentstroke}%
\pgfsetdash{}{0pt}%
\pgfpathmoveto{\pgfqpoint{1.902814in}{1.087251in}}%
\pgfpathlineto{\pgfqpoint{1.938272in}{1.094001in}}%
\pgfpathlineto{\pgfqpoint{1.901937in}{1.136828in}}%
\pgfpathlineto{\pgfqpoint{1.866355in}{1.130200in}}%
\pgfpathclose%
\pgfusepath{fill}%
\end{pgfscope}%
\begin{pgfscope}%
\pgfpathrectangle{\pgfqpoint{0.150000in}{0.150000in}}{\pgfqpoint{2.700000in}{1.950000in}}%
\pgfusepath{clip}%
\pgfsetbuttcap%
\pgfsetroundjoin%
\definecolor{currentfill}{rgb}{0.724571,0.500230,0.517999}%
\pgfsetfillcolor{currentfill}%
\pgfsetlinewidth{0.000000pt}%
\definecolor{currentstroke}{rgb}{0.000000,0.000000,0.000000}%
\pgfsetstrokecolor{currentstroke}%
\pgfsetdash{}{0pt}%
\pgfpathmoveto{\pgfqpoint{2.124137in}{0.889441in}}%
\pgfpathlineto{\pgfqpoint{2.158324in}{0.884971in}}%
\pgfpathlineto{\pgfqpoint{2.120040in}{0.892299in}}%
\pgfpathlineto{\pgfqpoint{2.085166in}{0.884971in}}%
\pgfpathclose%
\pgfusepath{fill}%
\end{pgfscope}%
\begin{pgfscope}%
\pgfpathrectangle{\pgfqpoint{0.150000in}{0.150000in}}{\pgfqpoint{2.700000in}{1.950000in}}%
\pgfusepath{clip}%
\pgfsetbuttcap%
\pgfsetroundjoin%
\definecolor{currentfill}{rgb}{0.903125,0.824219,0.830469}%
\pgfsetfillcolor{currentfill}%
\pgfsetlinewidth{0.000000pt}%
\definecolor{currentstroke}{rgb}{0.000000,0.000000,0.000000}%
\pgfsetstrokecolor{currentstroke}%
\pgfsetdash{}{0pt}%
\pgfpathmoveto{\pgfqpoint{1.829491in}{1.234823in}}%
\pgfpathlineto{\pgfqpoint{1.865155in}{1.241141in}}%
\pgfpathlineto{\pgfqpoint{1.829126in}{1.296298in}}%
\pgfpathlineto{\pgfqpoint{1.793000in}{1.277842in}}%
\pgfpathclose%
\pgfusepath{fill}%
\end{pgfscope}%
\begin{pgfscope}%
\pgfpathrectangle{\pgfqpoint{0.150000in}{0.150000in}}{\pgfqpoint{2.700000in}{1.950000in}}%
\pgfusepath{clip}%
\pgfsetbuttcap%
\pgfsetroundjoin%
\definecolor{currentfill}{rgb}{0.743566,0.534697,0.551241}%
\pgfsetfillcolor{currentfill}%
\pgfsetlinewidth{0.000000pt}%
\definecolor{currentstroke}{rgb}{0.000000,0.000000,0.000000}%
\pgfsetstrokecolor{currentstroke}%
\pgfsetdash{}{0pt}%
\pgfpathmoveto{\pgfqpoint{2.013104in}{0.908629in}}%
\pgfpathlineto{\pgfqpoint{2.047682in}{0.904089in}}%
\pgfpathlineto{\pgfqpoint{2.011417in}{0.946803in}}%
\pgfpathlineto{\pgfqpoint{1.976673in}{0.951520in}}%
\pgfpathclose%
\pgfusepath{fill}%
\end{pgfscope}%
\begin{pgfscope}%
\pgfpathrectangle{\pgfqpoint{0.150000in}{0.150000in}}{\pgfqpoint{2.700000in}{1.950000in}}%
\pgfusepath{clip}%
\pgfsetbuttcap%
\pgfsetroundjoin%
\definecolor{currentfill}{rgb}{0.728370,0.507123,0.524648}%
\pgfsetfillcolor{currentfill}%
\pgfsetlinewidth{0.000000pt}%
\definecolor{currentstroke}{rgb}{0.000000,0.000000,0.000000}%
\pgfsetstrokecolor{currentstroke}%
\pgfsetdash{}{0pt}%
\pgfpathmoveto{\pgfqpoint{2.197593in}{0.889441in}}%
\pgfpathlineto{\pgfqpoint{2.232280in}{0.896786in}}%
\pgfpathlineto{\pgfqpoint{2.192985in}{0.892299in}}%
\pgfpathlineto{\pgfqpoint{2.158324in}{0.884971in}}%
\pgfpathclose%
\pgfusepath{fill}%
\end{pgfscope}%
\begin{pgfscope}%
\pgfpathrectangle{\pgfqpoint{0.150000in}{0.150000in}}{\pgfqpoint{2.700000in}{1.950000in}}%
\pgfusepath{clip}%
\pgfsetbuttcap%
\pgfsetroundjoin%
\definecolor{currentfill}{rgb}{0.853738,0.734605,0.744041}%
\pgfsetfillcolor{currentfill}%
\pgfsetlinewidth{0.000000pt}%
\definecolor{currentstroke}{rgb}{0.000000,0.000000,0.000000}%
\pgfsetstrokecolor{currentstroke}%
\pgfsetdash{}{0pt}%
\pgfpathmoveto{\pgfqpoint{1.866355in}{1.130200in}}%
\pgfpathlineto{\pgfqpoint{1.901937in}{1.136828in}}%
\pgfpathlineto{\pgfqpoint{1.865565in}{1.179700in}}%
\pgfpathlineto{\pgfqpoint{1.829857in}{1.173195in}}%
\pgfpathclose%
\pgfusepath{fill}%
\end{pgfscope}%
\begin{pgfscope}%
\pgfpathrectangle{\pgfqpoint{0.150000in}{0.150000in}}{\pgfqpoint{2.700000in}{1.950000in}}%
\pgfusepath{clip}%
\pgfsetbuttcap%
\pgfsetroundjoin%
\definecolor{currentfill}{rgb}{0.770159,0.582950,0.597779}%
\pgfsetfillcolor{currentfill}%
\pgfsetlinewidth{0.000000pt}%
\definecolor{currentstroke}{rgb}{0.000000,0.000000,0.000000}%
\pgfsetstrokecolor{currentstroke}%
\pgfsetdash{}{0pt}%
\pgfpathmoveto{\pgfqpoint{1.976673in}{0.951520in}}%
\pgfpathlineto{\pgfqpoint{2.011417in}{0.946803in}}%
\pgfpathlineto{\pgfqpoint{1.975114in}{0.989561in}}%
\pgfpathlineto{\pgfqpoint{1.940203in}{0.994455in}}%
\pgfpathclose%
\pgfusepath{fill}%
\end{pgfscope}%
\begin{pgfscope}%
\pgfpathrectangle{\pgfqpoint{0.150000in}{0.150000in}}{\pgfqpoint{2.700000in}{1.950000in}}%
\pgfusepath{clip}%
\pgfsetbuttcap%
\pgfsetroundjoin%
\definecolor{currentfill}{rgb}{0.948713,0.906939,0.910248}%
\pgfsetfillcolor{currentfill}%
\pgfsetlinewidth{0.000000pt}%
\definecolor{currentstroke}{rgb}{0.000000,0.000000,0.000000}%
\pgfsetstrokecolor{currentstroke}%
\pgfsetdash{}{0pt}%
\pgfpathmoveto{\pgfqpoint{1.756471in}{1.320905in}}%
\pgfpathlineto{\pgfqpoint{1.792680in}{1.339295in}}%
\pgfpathlineto{\pgfqpoint{1.755944in}{1.369965in}}%
\pgfpathlineto{\pgfqpoint{1.719693in}{1.351640in}}%
\pgfpathclose%
\pgfusepath{fill}%
\end{pgfscope}%
\begin{pgfscope}%
\pgfpathrectangle{\pgfqpoint{0.150000in}{0.150000in}}{\pgfqpoint{2.700000in}{1.950000in}}%
\pgfusepath{clip}%
\pgfsetbuttcap%
\pgfsetroundjoin%
\definecolor{currentfill}{rgb}{0.732169,0.514017,0.531296}%
\pgfsetfillcolor{currentfill}%
\pgfsetlinewidth{0.000000pt}%
\definecolor{currentstroke}{rgb}{0.000000,0.000000,0.000000}%
\pgfsetstrokecolor{currentstroke}%
\pgfsetdash{}{0pt}%
\pgfpathmoveto{\pgfqpoint{2.050680in}{0.889441in}}%
\pgfpathlineto{\pgfqpoint{2.085166in}{0.884971in}}%
\pgfpathlineto{\pgfqpoint{2.047682in}{0.904089in}}%
\pgfpathlineto{\pgfqpoint{2.013104in}{0.908629in}}%
\pgfpathclose%
\pgfusepath{fill}%
\end{pgfscope}%
\begin{pgfscope}%
\pgfpathrectangle{\pgfqpoint{0.150000in}{0.150000in}}{\pgfqpoint{2.700000in}{1.950000in}}%
\pgfusepath{clip}%
\pgfsetbuttcap%
\pgfsetroundjoin%
\definecolor{currentfill}{rgb}{0.929718,0.872472,0.877007}%
\pgfsetfillcolor{currentfill}%
\pgfsetlinewidth{0.000000pt}%
\definecolor{currentstroke}{rgb}{0.000000,0.000000,0.000000}%
\pgfsetstrokecolor{currentstroke}%
\pgfsetdash{}{0pt}%
\pgfpathmoveto{\pgfqpoint{1.793000in}{1.277842in}}%
\pgfpathlineto{\pgfqpoint{1.829126in}{1.296298in}}%
\pgfpathlineto{\pgfqpoint{1.792680in}{1.339295in}}%
\pgfpathlineto{\pgfqpoint{1.756471in}{1.320905in}}%
\pgfpathclose%
\pgfusepath{fill}%
\end{pgfscope}%
\begin{pgfscope}%
\pgfpathrectangle{\pgfqpoint{0.150000in}{0.150000in}}{\pgfqpoint{2.700000in}{1.950000in}}%
\pgfusepath{clip}%
\pgfsetbuttcap%
\pgfsetroundjoin%
\definecolor{currentfill}{rgb}{0.880331,0.782858,0.790579}%
\pgfsetfillcolor{currentfill}%
\pgfsetlinewidth{0.000000pt}%
\definecolor{currentstroke}{rgb}{0.000000,0.000000,0.000000}%
\pgfsetstrokecolor{currentstroke}%
\pgfsetdash{}{0pt}%
\pgfpathmoveto{\pgfqpoint{1.829857in}{1.173195in}}%
\pgfpathlineto{\pgfqpoint{1.865565in}{1.179700in}}%
\pgfpathlineto{\pgfqpoint{1.829491in}{1.234823in}}%
\pgfpathlineto{\pgfqpoint{1.793321in}{1.216234in}}%
\pgfpathclose%
\pgfusepath{fill}%
\end{pgfscope}%
\begin{pgfscope}%
\pgfpathrectangle{\pgfqpoint{0.150000in}{0.150000in}}{\pgfqpoint{2.700000in}{1.950000in}}%
\pgfusepath{clip}%
\pgfsetbuttcap%
\pgfsetroundjoin%
\definecolor{currentfill}{rgb}{0.952512,0.913833,0.916896}%
\pgfsetfillcolor{currentfill}%
\pgfsetlinewidth{0.000000pt}%
\definecolor{currentstroke}{rgb}{0.000000,0.000000,0.000000}%
\pgfsetstrokecolor{currentstroke}%
\pgfsetdash{}{0pt}%
\pgfpathmoveto{\pgfqpoint{1.683312in}{1.333251in}}%
\pgfpathlineto{\pgfqpoint{1.719693in}{1.351640in}}%
\pgfpathlineto{\pgfqpoint{1.682792in}{1.369965in}}%
\pgfpathlineto{\pgfqpoint{1.646410in}{1.351640in}}%
\pgfpathclose%
\pgfusepath{fill}%
\end{pgfscope}%
\begin{pgfscope}%
\pgfpathrectangle{\pgfqpoint{0.150000in}{0.150000in}}{\pgfqpoint{2.700000in}{1.950000in}}%
\pgfusepath{clip}%
\pgfsetbuttcap%
\pgfsetroundjoin%
\definecolor{currentfill}{rgb}{0.952512,0.913833,0.916896}%
\pgfsetfillcolor{currentfill}%
\pgfsetlinewidth{0.000000pt}%
\definecolor{currentstroke}{rgb}{0.000000,0.000000,0.000000}%
\pgfsetstrokecolor{currentstroke}%
\pgfsetdash{}{0pt}%
\pgfpathmoveto{\pgfqpoint{1.609899in}{1.333251in}}%
\pgfpathlineto{\pgfqpoint{1.646410in}{1.351640in}}%
\pgfpathlineto{\pgfqpoint{1.609639in}{1.369965in}}%
\pgfpathlineto{\pgfqpoint{1.573128in}{1.351640in}}%
\pgfpathclose%
\pgfusepath{fill}%
\end{pgfscope}%
\begin{pgfscope}%
\pgfpathrectangle{\pgfqpoint{0.150000in}{0.150000in}}{\pgfqpoint{2.700000in}{1.950000in}}%
\pgfusepath{clip}%
\pgfsetbuttcap%
\pgfsetroundjoin%
\definecolor{currentfill}{rgb}{0.952512,0.913833,0.916896}%
\pgfsetfillcolor{currentfill}%
\pgfsetlinewidth{0.000000pt}%
\definecolor{currentstroke}{rgb}{0.000000,0.000000,0.000000}%
\pgfsetstrokecolor{currentstroke}%
\pgfsetdash{}{0pt}%
\pgfpathmoveto{\pgfqpoint{1.536486in}{1.333251in}}%
\pgfpathlineto{\pgfqpoint{1.573128in}{1.351640in}}%
\pgfpathlineto{\pgfqpoint{1.536486in}{1.369965in}}%
\pgfpathlineto{\pgfqpoint{1.499845in}{1.351640in}}%
\pgfpathclose%
\pgfusepath{fill}%
\end{pgfscope}%
\begin{pgfscope}%
\pgfpathrectangle{\pgfqpoint{0.150000in}{0.150000in}}{\pgfqpoint{2.700000in}{1.950000in}}%
\pgfusepath{clip}%
\pgfsetbuttcap%
\pgfsetroundjoin%
\definecolor{currentfill}{rgb}{0.952512,0.913833,0.916896}%
\pgfsetfillcolor{currentfill}%
\pgfsetlinewidth{0.000000pt}%
\definecolor{currentstroke}{rgb}{0.000000,0.000000,0.000000}%
\pgfsetstrokecolor{currentstroke}%
\pgfsetdash{}{0pt}%
\pgfpathmoveto{\pgfqpoint{1.463074in}{1.333251in}}%
\pgfpathlineto{\pgfqpoint{1.499845in}{1.351640in}}%
\pgfpathlineto{\pgfqpoint{1.463334in}{1.369965in}}%
\pgfpathlineto{\pgfqpoint{1.426563in}{1.351640in}}%
\pgfpathclose%
\pgfusepath{fill}%
\end{pgfscope}%
\begin{pgfscope}%
\pgfpathrectangle{\pgfqpoint{0.150000in}{0.150000in}}{\pgfqpoint{2.700000in}{1.950000in}}%
\pgfusepath{clip}%
\pgfsetbuttcap%
\pgfsetroundjoin%
\definecolor{currentfill}{rgb}{0.952512,0.913833,0.916896}%
\pgfsetfillcolor{currentfill}%
\pgfsetlinewidth{0.000000pt}%
\definecolor{currentstroke}{rgb}{0.000000,0.000000,0.000000}%
\pgfsetstrokecolor{currentstroke}%
\pgfsetdash{}{0pt}%
\pgfpathmoveto{\pgfqpoint{1.389661in}{1.333251in}}%
\pgfpathlineto{\pgfqpoint{1.426563in}{1.351640in}}%
\pgfpathlineto{\pgfqpoint{1.390181in}{1.369965in}}%
\pgfpathlineto{\pgfqpoint{1.353280in}{1.351640in}}%
\pgfpathclose%
\pgfusepath{fill}%
\end{pgfscope}%
\begin{pgfscope}%
\pgfpathrectangle{\pgfqpoint{0.150000in}{0.150000in}}{\pgfqpoint{2.700000in}{1.950000in}}%
\pgfusepath{clip}%
\pgfsetbuttcap%
\pgfsetroundjoin%
\definecolor{currentfill}{rgb}{0.952512,0.913833,0.916896}%
\pgfsetfillcolor{currentfill}%
\pgfsetlinewidth{0.000000pt}%
\definecolor{currentstroke}{rgb}{0.000000,0.000000,0.000000}%
\pgfsetstrokecolor{currentstroke}%
\pgfsetdash{}{0pt}%
\pgfpathmoveto{\pgfqpoint{1.316248in}{1.333251in}}%
\pgfpathlineto{\pgfqpoint{1.353280in}{1.351640in}}%
\pgfpathlineto{\pgfqpoint{1.317029in}{1.369965in}}%
\pgfpathlineto{\pgfqpoint{1.279998in}{1.351640in}}%
\pgfpathclose%
\pgfusepath{fill}%
\end{pgfscope}%
\begin{pgfscope}%
\pgfpathrectangle{\pgfqpoint{0.150000in}{0.150000in}}{\pgfqpoint{2.700000in}{1.950000in}}%
\pgfusepath{clip}%
\pgfsetbuttcap%
\pgfsetroundjoin%
\definecolor{currentfill}{rgb}{0.952512,0.913833,0.916896}%
\pgfsetfillcolor{currentfill}%
\pgfsetlinewidth{0.000000pt}%
\definecolor{currentstroke}{rgb}{0.000000,0.000000,0.000000}%
\pgfsetstrokecolor{currentstroke}%
\pgfsetdash{}{0pt}%
\pgfpathmoveto{\pgfqpoint{1.242835in}{1.333251in}}%
\pgfpathlineto{\pgfqpoint{1.279998in}{1.351640in}}%
\pgfpathlineto{\pgfqpoint{1.243876in}{1.369965in}}%
\pgfpathlineto{\pgfqpoint{1.206715in}{1.351640in}}%
\pgfpathclose%
\pgfusepath{fill}%
\end{pgfscope}%
\begin{pgfscope}%
\pgfpathrectangle{\pgfqpoint{0.150000in}{0.150000in}}{\pgfqpoint{2.700000in}{1.950000in}}%
\pgfusepath{clip}%
\pgfsetbuttcap%
\pgfsetroundjoin%
\definecolor{currentfill}{rgb}{0.952512,0.913833,0.916896}%
\pgfsetfillcolor{currentfill}%
\pgfsetlinewidth{0.000000pt}%
\definecolor{currentstroke}{rgb}{0.000000,0.000000,0.000000}%
\pgfsetstrokecolor{currentstroke}%
\pgfsetdash{}{0pt}%
\pgfpathmoveto{\pgfqpoint{1.169423in}{1.333251in}}%
\pgfpathlineto{\pgfqpoint{1.206715in}{1.351640in}}%
\pgfpathlineto{\pgfqpoint{1.170723in}{1.369965in}}%
\pgfpathlineto{\pgfqpoint{1.133433in}{1.351640in}}%
\pgfpathclose%
\pgfusepath{fill}%
\end{pgfscope}%
\begin{pgfscope}%
\pgfpathrectangle{\pgfqpoint{0.150000in}{0.150000in}}{\pgfqpoint{2.700000in}{1.950000in}}%
\pgfusepath{clip}%
\pgfsetbuttcap%
\pgfsetroundjoin%
\definecolor{currentfill}{rgb}{0.952512,0.913833,0.916896}%
\pgfsetfillcolor{currentfill}%
\pgfsetlinewidth{0.000000pt}%
\definecolor{currentstroke}{rgb}{0.000000,0.000000,0.000000}%
\pgfsetstrokecolor{currentstroke}%
\pgfsetdash{}{0pt}%
\pgfpathmoveto{\pgfqpoint{1.096010in}{1.333251in}}%
\pgfpathlineto{\pgfqpoint{1.133433in}{1.351640in}}%
\pgfpathlineto{\pgfqpoint{1.097571in}{1.369965in}}%
\pgfpathlineto{\pgfqpoint{1.060150in}{1.351640in}}%
\pgfpathclose%
\pgfusepath{fill}%
\end{pgfscope}%
\begin{pgfscope}%
\pgfpathrectangle{\pgfqpoint{0.150000in}{0.150000in}}{\pgfqpoint{2.700000in}{1.950000in}}%
\pgfusepath{clip}%
\pgfsetbuttcap%
\pgfsetroundjoin%
\definecolor{currentfill}{rgb}{0.739767,0.527803,0.544593}%
\pgfsetfillcolor{currentfill}%
\pgfsetlinewidth{0.000000pt}%
\definecolor{currentstroke}{rgb}{0.000000,0.000000,0.000000}%
\pgfsetstrokecolor{currentstroke}%
\pgfsetdash{}{0pt}%
\pgfpathmoveto{\pgfqpoint{2.271897in}{0.901310in}}%
\pgfpathlineto{\pgfqpoint{2.306407in}{0.908629in}}%
\pgfpathlineto{\pgfqpoint{2.266765in}{0.904089in}}%
\pgfpathlineto{\pgfqpoint{2.232280in}{0.896786in}}%
\pgfpathclose%
\pgfusepath{fill}%
\end{pgfscope}%
\begin{pgfscope}%
\pgfpathrectangle{\pgfqpoint{0.150000in}{0.150000in}}{\pgfqpoint{2.700000in}{1.950000in}}%
\pgfusepath{clip}%
\pgfsetbuttcap%
\pgfsetroundjoin%
\definecolor{currentfill}{rgb}{0.792953,0.624311,0.637669}%
\pgfsetfillcolor{currentfill}%
\pgfsetlinewidth{0.000000pt}%
\definecolor{currentstroke}{rgb}{0.000000,0.000000,0.000000}%
\pgfsetstrokecolor{currentstroke}%
\pgfsetdash{}{0pt}%
\pgfpathmoveto{\pgfqpoint{1.940203in}{0.994455in}}%
\pgfpathlineto{\pgfqpoint{1.975114in}{0.989561in}}%
\pgfpathlineto{\pgfqpoint{1.939235in}{1.044347in}}%
\pgfpathlineto{\pgfqpoint{1.903695in}{1.037436in}}%
\pgfpathclose%
\pgfusepath{fill}%
\end{pgfscope}%
\begin{pgfscope}%
\pgfpathrectangle{\pgfqpoint{0.150000in}{0.150000in}}{\pgfqpoint{2.700000in}{1.950000in}}%
\pgfusepath{clip}%
\pgfsetbuttcap%
\pgfsetroundjoin%
\definecolor{currentfill}{rgb}{0.906924,0.831112,0.837117}%
\pgfsetfillcolor{currentfill}%
\pgfsetlinewidth{0.000000pt}%
\definecolor{currentstroke}{rgb}{0.000000,0.000000,0.000000}%
\pgfsetstrokecolor{currentstroke}%
\pgfsetdash{}{0pt}%
\pgfpathmoveto{\pgfqpoint{1.793321in}{1.216234in}}%
\pgfpathlineto{\pgfqpoint{1.829491in}{1.234823in}}%
\pgfpathlineto{\pgfqpoint{1.793000in}{1.277842in}}%
\pgfpathlineto{\pgfqpoint{1.756747in}{1.259319in}}%
\pgfpathclose%
\pgfusepath{fill}%
\end{pgfscope}%
\begin{pgfscope}%
\pgfpathrectangle{\pgfqpoint{0.150000in}{0.150000in}}{\pgfqpoint{2.700000in}{1.950000in}}%
\pgfusepath{clip}%
\pgfsetbuttcap%
\pgfsetroundjoin%
\definecolor{currentfill}{rgb}{0.747365,0.541590,0.557889}%
\pgfsetfillcolor{currentfill}%
\pgfsetlinewidth{0.000000pt}%
\definecolor{currentstroke}{rgb}{0.000000,0.000000,0.000000}%
\pgfsetstrokecolor{currentstroke}%
\pgfsetdash{}{0pt}%
\pgfpathmoveto{\pgfqpoint{2.346374in}{0.913207in}}%
\pgfpathlineto{\pgfqpoint{2.380705in}{0.920500in}}%
\pgfpathlineto{\pgfqpoint{2.339793in}{0.904089in}}%
\pgfpathlineto{\pgfqpoint{2.306407in}{0.908629in}}%
\pgfpathclose%
\pgfusepath{fill}%
\end{pgfscope}%
\begin{pgfscope}%
\pgfpathrectangle{\pgfqpoint{0.150000in}{0.150000in}}{\pgfqpoint{2.700000in}{1.950000in}}%
\pgfusepath{clip}%
\pgfsetbuttcap%
\pgfsetroundjoin%
\definecolor{currentfill}{rgb}{0.819547,0.672564,0.684206}%
\pgfsetfillcolor{currentfill}%
\pgfsetlinewidth{0.000000pt}%
\definecolor{currentstroke}{rgb}{0.000000,0.000000,0.000000}%
\pgfsetstrokecolor{currentstroke}%
\pgfsetdash{}{0pt}%
\pgfpathmoveto{\pgfqpoint{1.903695in}{1.037436in}}%
\pgfpathlineto{\pgfqpoint{1.939235in}{1.044347in}}%
\pgfpathlineto{\pgfqpoint{1.902814in}{1.087251in}}%
\pgfpathlineto{\pgfqpoint{1.867149in}{1.080462in}}%
\pgfpathclose%
\pgfusepath{fill}%
\end{pgfscope}%
\begin{pgfscope}%
\pgfpathrectangle{\pgfqpoint{0.150000in}{0.150000in}}{\pgfqpoint{2.700000in}{1.950000in}}%
\pgfusepath{clip}%
\pgfsetbuttcap%
\pgfsetroundjoin%
\definecolor{currentfill}{rgb}{0.948713,0.906939,0.910248}%
\pgfsetfillcolor{currentfill}%
\pgfsetlinewidth{0.000000pt}%
\definecolor{currentstroke}{rgb}{0.000000,0.000000,0.000000}%
\pgfsetstrokecolor{currentstroke}%
\pgfsetdash{}{0pt}%
\pgfpathmoveto{\pgfqpoint{1.720133in}{1.302450in}}%
\pgfpathlineto{\pgfqpoint{1.756471in}{1.320905in}}%
\pgfpathlineto{\pgfqpoint{1.719693in}{1.351640in}}%
\pgfpathlineto{\pgfqpoint{1.683312in}{1.333251in}}%
\pgfpathclose%
\pgfusepath{fill}%
\end{pgfscope}%
\begin{pgfscope}%
\pgfpathrectangle{\pgfqpoint{0.150000in}{0.150000in}}{\pgfqpoint{2.700000in}{1.950000in}}%
\pgfusepath{clip}%
\pgfsetbuttcap%
\pgfsetroundjoin%
\definecolor{currentfill}{rgb}{0.929718,0.872472,0.877007}%
\pgfsetfillcolor{currentfill}%
\pgfsetlinewidth{0.000000pt}%
\definecolor{currentstroke}{rgb}{0.000000,0.000000,0.000000}%
\pgfsetstrokecolor{currentstroke}%
\pgfsetdash{}{0pt}%
\pgfpathmoveto{\pgfqpoint{1.756747in}{1.259319in}}%
\pgfpathlineto{\pgfqpoint{1.793000in}{1.277842in}}%
\pgfpathlineto{\pgfqpoint{1.756471in}{1.320905in}}%
\pgfpathlineto{\pgfqpoint{1.720133in}{1.302450in}}%
\pgfpathclose%
\pgfusepath{fill}%
\end{pgfscope}%
\begin{pgfscope}%
\pgfpathrectangle{\pgfqpoint{0.150000in}{0.150000in}}{\pgfqpoint{2.700000in}{1.950000in}}%
\pgfusepath{clip}%
\pgfsetbuttcap%
\pgfsetroundjoin%
\definecolor{currentfill}{rgb}{0.842341,0.713925,0.724096}%
\pgfsetfillcolor{currentfill}%
\pgfsetlinewidth{0.000000pt}%
\definecolor{currentstroke}{rgb}{0.000000,0.000000,0.000000}%
\pgfsetstrokecolor{currentstroke}%
\pgfsetdash{}{0pt}%
\pgfpathmoveto{\pgfqpoint{1.867149in}{1.080462in}}%
\pgfpathlineto{\pgfqpoint{1.902814in}{1.087251in}}%
\pgfpathlineto{\pgfqpoint{1.866355in}{1.130200in}}%
\pgfpathlineto{\pgfqpoint{1.830564in}{1.123534in}}%
\pgfpathclose%
\pgfusepath{fill}%
\end{pgfscope}%
\begin{pgfscope}%
\pgfpathrectangle{\pgfqpoint{0.150000in}{0.150000in}}{\pgfqpoint{2.700000in}{1.950000in}}%
\pgfusepath{clip}%
\pgfsetbuttcap%
\pgfsetroundjoin%
\definecolor{currentfill}{rgb}{0.865135,0.755285,0.763986}%
\pgfsetfillcolor{currentfill}%
\pgfsetlinewidth{0.000000pt}%
\definecolor{currentstroke}{rgb}{0.000000,0.000000,0.000000}%
\pgfsetstrokecolor{currentstroke}%
\pgfsetdash{}{0pt}%
\pgfpathmoveto{\pgfqpoint{1.830564in}{1.123534in}}%
\pgfpathlineto{\pgfqpoint{1.866355in}{1.130200in}}%
\pgfpathlineto{\pgfqpoint{1.829857in}{1.173195in}}%
\pgfpathlineto{\pgfqpoint{1.793643in}{1.154473in}}%
\pgfpathclose%
\pgfusepath{fill}%
\end{pgfscope}%
\begin{pgfscope}%
\pgfpathrectangle{\pgfqpoint{0.150000in}{0.150000in}}{\pgfqpoint{2.700000in}{1.950000in}}%
\pgfusepath{clip}%
\pgfsetbuttcap%
\pgfsetroundjoin%
\definecolor{currentfill}{rgb}{0.952512,0.913833,0.916896}%
\pgfsetfillcolor{currentfill}%
\pgfsetlinewidth{0.000000pt}%
\definecolor{currentstroke}{rgb}{0.000000,0.000000,0.000000}%
\pgfsetstrokecolor{currentstroke}%
\pgfsetdash{}{0pt}%
\pgfpathmoveto{\pgfqpoint{1.646802in}{1.314796in}}%
\pgfpathlineto{\pgfqpoint{1.683312in}{1.333251in}}%
\pgfpathlineto{\pgfqpoint{1.646410in}{1.351640in}}%
\pgfpathlineto{\pgfqpoint{1.609899in}{1.333251in}}%
\pgfpathclose%
\pgfusepath{fill}%
\end{pgfscope}%
\begin{pgfscope}%
\pgfpathrectangle{\pgfqpoint{0.150000in}{0.150000in}}{\pgfqpoint{2.700000in}{1.950000in}}%
\pgfusepath{clip}%
\pgfsetbuttcap%
\pgfsetroundjoin%
\definecolor{currentfill}{rgb}{0.952512,0.913833,0.916896}%
\pgfsetfillcolor{currentfill}%
\pgfsetlinewidth{0.000000pt}%
\definecolor{currentstroke}{rgb}{0.000000,0.000000,0.000000}%
\pgfsetstrokecolor{currentstroke}%
\pgfsetdash{}{0pt}%
\pgfpathmoveto{\pgfqpoint{1.573258in}{1.314796in}}%
\pgfpathlineto{\pgfqpoint{1.609899in}{1.333251in}}%
\pgfpathlineto{\pgfqpoint{1.573128in}{1.351640in}}%
\pgfpathlineto{\pgfqpoint{1.536486in}{1.333251in}}%
\pgfpathclose%
\pgfusepath{fill}%
\end{pgfscope}%
\begin{pgfscope}%
\pgfpathrectangle{\pgfqpoint{0.150000in}{0.150000in}}{\pgfqpoint{2.700000in}{1.950000in}}%
\pgfusepath{clip}%
\pgfsetbuttcap%
\pgfsetroundjoin%
\definecolor{currentfill}{rgb}{0.952512,0.913833,0.916896}%
\pgfsetfillcolor{currentfill}%
\pgfsetlinewidth{0.000000pt}%
\definecolor{currentstroke}{rgb}{0.000000,0.000000,0.000000}%
\pgfsetstrokecolor{currentstroke}%
\pgfsetdash{}{0pt}%
\pgfpathmoveto{\pgfqpoint{1.499715in}{1.314796in}}%
\pgfpathlineto{\pgfqpoint{1.536486in}{1.333251in}}%
\pgfpathlineto{\pgfqpoint{1.499845in}{1.351640in}}%
\pgfpathlineto{\pgfqpoint{1.463074in}{1.333251in}}%
\pgfpathclose%
\pgfusepath{fill}%
\end{pgfscope}%
\begin{pgfscope}%
\pgfpathrectangle{\pgfqpoint{0.150000in}{0.150000in}}{\pgfqpoint{2.700000in}{1.950000in}}%
\pgfusepath{clip}%
\pgfsetbuttcap%
\pgfsetroundjoin%
\definecolor{currentfill}{rgb}{0.952512,0.913833,0.916896}%
\pgfsetfillcolor{currentfill}%
\pgfsetlinewidth{0.000000pt}%
\definecolor{currentstroke}{rgb}{0.000000,0.000000,0.000000}%
\pgfsetstrokecolor{currentstroke}%
\pgfsetdash{}{0pt}%
\pgfpathmoveto{\pgfqpoint{1.426171in}{1.314796in}}%
\pgfpathlineto{\pgfqpoint{1.463074in}{1.333251in}}%
\pgfpathlineto{\pgfqpoint{1.426563in}{1.351640in}}%
\pgfpathlineto{\pgfqpoint{1.389661in}{1.333251in}}%
\pgfpathclose%
\pgfusepath{fill}%
\end{pgfscope}%
\begin{pgfscope}%
\pgfpathrectangle{\pgfqpoint{0.150000in}{0.150000in}}{\pgfqpoint{2.700000in}{1.950000in}}%
\pgfusepath{clip}%
\pgfsetbuttcap%
\pgfsetroundjoin%
\definecolor{currentfill}{rgb}{0.952512,0.913833,0.916896}%
\pgfsetfillcolor{currentfill}%
\pgfsetlinewidth{0.000000pt}%
\definecolor{currentstroke}{rgb}{0.000000,0.000000,0.000000}%
\pgfsetstrokecolor{currentstroke}%
\pgfsetdash{}{0pt}%
\pgfpathmoveto{\pgfqpoint{1.352628in}{1.314796in}}%
\pgfpathlineto{\pgfqpoint{1.389661in}{1.333251in}}%
\pgfpathlineto{\pgfqpoint{1.353280in}{1.351640in}}%
\pgfpathlineto{\pgfqpoint{1.316248in}{1.333251in}}%
\pgfpathclose%
\pgfusepath{fill}%
\end{pgfscope}%
\begin{pgfscope}%
\pgfpathrectangle{\pgfqpoint{0.150000in}{0.150000in}}{\pgfqpoint{2.700000in}{1.950000in}}%
\pgfusepath{clip}%
\pgfsetbuttcap%
\pgfsetroundjoin%
\definecolor{currentfill}{rgb}{0.952512,0.913833,0.916896}%
\pgfsetfillcolor{currentfill}%
\pgfsetlinewidth{0.000000pt}%
\definecolor{currentstroke}{rgb}{0.000000,0.000000,0.000000}%
\pgfsetstrokecolor{currentstroke}%
\pgfsetdash{}{0pt}%
\pgfpathmoveto{\pgfqpoint{1.279084in}{1.314796in}}%
\pgfpathlineto{\pgfqpoint{1.316248in}{1.333251in}}%
\pgfpathlineto{\pgfqpoint{1.279998in}{1.351640in}}%
\pgfpathlineto{\pgfqpoint{1.242835in}{1.333251in}}%
\pgfpathclose%
\pgfusepath{fill}%
\end{pgfscope}%
\begin{pgfscope}%
\pgfpathrectangle{\pgfqpoint{0.150000in}{0.150000in}}{\pgfqpoint{2.700000in}{1.950000in}}%
\pgfusepath{clip}%
\pgfsetbuttcap%
\pgfsetroundjoin%
\definecolor{currentfill}{rgb}{0.952512,0.913833,0.916896}%
\pgfsetfillcolor{currentfill}%
\pgfsetlinewidth{0.000000pt}%
\definecolor{currentstroke}{rgb}{0.000000,0.000000,0.000000}%
\pgfsetstrokecolor{currentstroke}%
\pgfsetdash{}{0pt}%
\pgfpathmoveto{\pgfqpoint{1.205540in}{1.314796in}}%
\pgfpathlineto{\pgfqpoint{1.242835in}{1.333251in}}%
\pgfpathlineto{\pgfqpoint{1.206715in}{1.351640in}}%
\pgfpathlineto{\pgfqpoint{1.169423in}{1.333251in}}%
\pgfpathclose%
\pgfusepath{fill}%
\end{pgfscope}%
\begin{pgfscope}%
\pgfpathrectangle{\pgfqpoint{0.150000in}{0.150000in}}{\pgfqpoint{2.700000in}{1.950000in}}%
\pgfusepath{clip}%
\pgfsetbuttcap%
\pgfsetroundjoin%
\definecolor{currentfill}{rgb}{0.952512,0.913833,0.916896}%
\pgfsetfillcolor{currentfill}%
\pgfsetlinewidth{0.000000pt}%
\definecolor{currentstroke}{rgb}{0.000000,0.000000,0.000000}%
\pgfsetstrokecolor{currentstroke}%
\pgfsetdash{}{0pt}%
\pgfpathmoveto{\pgfqpoint{1.131997in}{1.314796in}}%
\pgfpathlineto{\pgfqpoint{1.169423in}{1.333251in}}%
\pgfpathlineto{\pgfqpoint{1.133433in}{1.351640in}}%
\pgfpathlineto{\pgfqpoint{1.096010in}{1.333251in}}%
\pgfpathclose%
\pgfusepath{fill}%
\end{pgfscope}%
\begin{pgfscope}%
\pgfpathrectangle{\pgfqpoint{0.150000in}{0.150000in}}{\pgfqpoint{2.700000in}{1.950000in}}%
\pgfusepath{clip}%
\pgfsetbuttcap%
\pgfsetroundjoin%
\definecolor{currentfill}{rgb}{0.952512,0.913833,0.916896}%
\pgfsetfillcolor{currentfill}%
\pgfsetlinewidth{0.000000pt}%
\definecolor{currentstroke}{rgb}{0.000000,0.000000,0.000000}%
\pgfsetstrokecolor{currentstroke}%
\pgfsetdash{}{0pt}%
\pgfpathmoveto{\pgfqpoint{1.058453in}{1.314796in}}%
\pgfpathlineto{\pgfqpoint{1.096010in}{1.333251in}}%
\pgfpathlineto{\pgfqpoint{1.060150in}{1.351640in}}%
\pgfpathlineto{\pgfqpoint{1.022597in}{1.333251in}}%
\pgfpathclose%
\pgfusepath{fill}%
\end{pgfscope}%
\begin{pgfscope}%
\pgfpathrectangle{\pgfqpoint{0.150000in}{0.150000in}}{\pgfqpoint{2.700000in}{1.950000in}}%
\pgfusepath{clip}%
\pgfsetbuttcap%
\pgfsetroundjoin%
\definecolor{currentfill}{rgb}{0.884130,0.789752,0.797227}%
\pgfsetfillcolor{currentfill}%
\pgfsetlinewidth{0.000000pt}%
\definecolor{currentstroke}{rgb}{0.000000,0.000000,0.000000}%
\pgfsetstrokecolor{currentstroke}%
\pgfsetdash{}{0pt}%
\pgfpathmoveto{\pgfqpoint{1.793643in}{1.154473in}}%
\pgfpathlineto{\pgfqpoint{1.829857in}{1.173195in}}%
\pgfpathlineto{\pgfqpoint{1.793321in}{1.216234in}}%
\pgfpathlineto{\pgfqpoint{1.757023in}{1.197579in}}%
\pgfpathclose%
\pgfusepath{fill}%
\end{pgfscope}%
\begin{pgfscope}%
\pgfpathrectangle{\pgfqpoint{0.150000in}{0.150000in}}{\pgfqpoint{2.700000in}{1.950000in}}%
\pgfusepath{clip}%
\pgfsetbuttcap%
\pgfsetroundjoin%
\definecolor{currentfill}{rgb}{0.743566,0.534697,0.551241}%
\pgfsetfillcolor{currentfill}%
\pgfsetlinewidth{0.000000pt}%
\definecolor{currentstroke}{rgb}{0.000000,0.000000,0.000000}%
\pgfsetstrokecolor{currentstroke}%
\pgfsetdash{}{0pt}%
\pgfpathmoveto{\pgfqpoint{2.089029in}{0.882052in}}%
\pgfpathlineto{\pgfqpoint{2.124137in}{0.889441in}}%
\pgfpathlineto{\pgfqpoint{2.085166in}{0.884971in}}%
\pgfpathlineto{\pgfqpoint{2.050680in}{0.889441in}}%
\pgfpathclose%
\pgfusepath{fill}%
\end{pgfscope}%
\begin{pgfscope}%
\pgfpathrectangle{\pgfqpoint{0.150000in}{0.150000in}}{\pgfqpoint{2.700000in}{1.950000in}}%
\pgfusepath{clip}%
\pgfsetbuttcap%
\pgfsetroundjoin%
\definecolor{currentfill}{rgb}{0.762561,0.569164,0.584482}%
\pgfsetfillcolor{currentfill}%
\pgfsetlinewidth{0.000000pt}%
\definecolor{currentstroke}{rgb}{0.000000,0.000000,0.000000}%
\pgfsetstrokecolor{currentstroke}%
\pgfsetdash{}{0pt}%
\pgfpathmoveto{\pgfqpoint{1.977733in}{0.901310in}}%
\pgfpathlineto{\pgfqpoint{2.013104in}{0.908629in}}%
\pgfpathlineto{\pgfqpoint{1.976673in}{0.951520in}}%
\pgfpathlineto{\pgfqpoint{1.941176in}{0.944323in}}%
\pgfpathclose%
\pgfusepath{fill}%
\end{pgfscope}%
\begin{pgfscope}%
\pgfpathrectangle{\pgfqpoint{0.150000in}{0.150000in}}{\pgfqpoint{2.700000in}{1.950000in}}%
\pgfusepath{clip}%
\pgfsetbuttcap%
\pgfsetroundjoin%
\definecolor{currentfill}{rgb}{0.762561,0.569164,0.584482}%
\pgfsetfillcolor{currentfill}%
\pgfsetlinewidth{0.000000pt}%
\definecolor{currentstroke}{rgb}{0.000000,0.000000,0.000000}%
\pgfsetstrokecolor{currentstroke}%
\pgfsetdash{}{0pt}%
\pgfpathmoveto{\pgfqpoint{2.421022in}{0.925131in}}%
\pgfpathlineto{\pgfqpoint{2.455174in}{0.932397in}}%
\pgfpathlineto{\pgfqpoint{2.414836in}{0.927750in}}%
\pgfpathlineto{\pgfqpoint{2.380705in}{0.920500in}}%
\pgfpathclose%
\pgfusepath{fill}%
\end{pgfscope}%
\begin{pgfscope}%
\pgfpathrectangle{\pgfqpoint{0.150000in}{0.150000in}}{\pgfqpoint{2.700000in}{1.950000in}}%
\pgfusepath{clip}%
\pgfsetbuttcap%
\pgfsetroundjoin%
\definecolor{currentfill}{rgb}{0.743566,0.534697,0.551241}%
\pgfsetfillcolor{currentfill}%
\pgfsetlinewidth{0.000000pt}%
\definecolor{currentstroke}{rgb}{0.000000,0.000000,0.000000}%
\pgfsetstrokecolor{currentstroke}%
\pgfsetdash{}{0pt}%
\pgfpathmoveto{\pgfqpoint{2.163427in}{0.893948in}}%
\pgfpathlineto{\pgfqpoint{2.197593in}{0.889441in}}%
\pgfpathlineto{\pgfqpoint{2.158324in}{0.884971in}}%
\pgfpathlineto{\pgfqpoint{2.124137in}{0.889441in}}%
\pgfpathclose%
\pgfusepath{fill}%
\end{pgfscope}%
\begin{pgfscope}%
\pgfpathrectangle{\pgfqpoint{0.150000in}{0.150000in}}{\pgfqpoint{2.700000in}{1.950000in}}%
\pgfusepath{clip}%
\pgfsetbuttcap%
\pgfsetroundjoin%
\definecolor{currentfill}{rgb}{0.906924,0.831112,0.837117}%
\pgfsetfillcolor{currentfill}%
\pgfsetlinewidth{0.000000pt}%
\definecolor{currentstroke}{rgb}{0.000000,0.000000,0.000000}%
\pgfsetstrokecolor{currentstroke}%
\pgfsetdash{}{0pt}%
\pgfpathmoveto{\pgfqpoint{1.757023in}{1.197579in}}%
\pgfpathlineto{\pgfqpoint{1.793321in}{1.216234in}}%
\pgfpathlineto{\pgfqpoint{1.756747in}{1.259319in}}%
\pgfpathlineto{\pgfqpoint{1.720364in}{1.240731in}}%
\pgfpathclose%
\pgfusepath{fill}%
\end{pgfscope}%
\begin{pgfscope}%
\pgfpathrectangle{\pgfqpoint{0.150000in}{0.150000in}}{\pgfqpoint{2.700000in}{1.950000in}}%
\pgfusepath{clip}%
\pgfsetbuttcap%
\pgfsetroundjoin%
\definecolor{currentfill}{rgb}{0.785355,0.610524,0.624372}%
\pgfsetfillcolor{currentfill}%
\pgfsetlinewidth{0.000000pt}%
\definecolor{currentstroke}{rgb}{0.000000,0.000000,0.000000}%
\pgfsetstrokecolor{currentstroke}%
\pgfsetdash{}{0pt}%
\pgfpathmoveto{\pgfqpoint{1.941176in}{0.944323in}}%
\pgfpathlineto{\pgfqpoint{1.976673in}{0.951520in}}%
\pgfpathlineto{\pgfqpoint{1.940203in}{0.994455in}}%
\pgfpathlineto{\pgfqpoint{1.904581in}{0.987381in}}%
\pgfpathclose%
\pgfusepath{fill}%
\end{pgfscope}%
\begin{pgfscope}%
\pgfpathrectangle{\pgfqpoint{0.150000in}{0.150000in}}{\pgfqpoint{2.700000in}{1.950000in}}%
\pgfusepath{clip}%
\pgfsetbuttcap%
\pgfsetroundjoin%
\definecolor{currentfill}{rgb}{0.751164,0.548483,0.564537}%
\pgfsetfillcolor{currentfill}%
\pgfsetlinewidth{0.000000pt}%
\definecolor{currentstroke}{rgb}{0.000000,0.000000,0.000000}%
\pgfsetstrokecolor{currentstroke}%
\pgfsetdash{}{0pt}%
\pgfpathmoveto{\pgfqpoint{2.015357in}{0.882052in}}%
\pgfpathlineto{\pgfqpoint{2.050680in}{0.889441in}}%
\pgfpathlineto{\pgfqpoint{2.013104in}{0.908629in}}%
\pgfpathlineto{\pgfqpoint{1.977733in}{0.901310in}}%
\pgfpathclose%
\pgfusepath{fill}%
\end{pgfscope}%
\begin{pgfscope}%
\pgfpathrectangle{\pgfqpoint{0.150000in}{0.150000in}}{\pgfqpoint{2.700000in}{1.950000in}}%
\pgfusepath{clip}%
\pgfsetbuttcap%
\pgfsetroundjoin%
\definecolor{currentfill}{rgb}{0.751164,0.548483,0.564537}%
\pgfsetfillcolor{currentfill}%
\pgfsetlinewidth{0.000000pt}%
\definecolor{currentstroke}{rgb}{0.000000,0.000000,0.000000}%
\pgfsetstrokecolor{currentstroke}%
\pgfsetdash{}{0pt}%
\pgfpathmoveto{\pgfqpoint{2.237184in}{0.893948in}}%
\pgfpathlineto{\pgfqpoint{2.271897in}{0.901310in}}%
\pgfpathlineto{\pgfqpoint{2.232280in}{0.896786in}}%
\pgfpathlineto{\pgfqpoint{2.197593in}{0.889441in}}%
\pgfpathclose%
\pgfusepath{fill}%
\end{pgfscope}%
\begin{pgfscope}%
\pgfpathrectangle{\pgfqpoint{0.150000in}{0.150000in}}{\pgfqpoint{2.700000in}{1.950000in}}%
\pgfusepath{clip}%
\pgfsetbuttcap%
\pgfsetroundjoin%
\definecolor{currentfill}{rgb}{0.948713,0.906939,0.910248}%
\pgfsetfillcolor{currentfill}%
\pgfsetlinewidth{0.000000pt}%
\definecolor{currentstroke}{rgb}{0.000000,0.000000,0.000000}%
\pgfsetstrokecolor{currentstroke}%
\pgfsetdash{}{0pt}%
\pgfpathmoveto{\pgfqpoint{1.683666in}{1.283928in}}%
\pgfpathlineto{\pgfqpoint{1.720133in}{1.302450in}}%
\pgfpathlineto{\pgfqpoint{1.683312in}{1.333251in}}%
\pgfpathlineto{\pgfqpoint{1.646802in}{1.314796in}}%
\pgfpathclose%
\pgfusepath{fill}%
\end{pgfscope}%
\begin{pgfscope}%
\pgfpathrectangle{\pgfqpoint{0.150000in}{0.150000in}}{\pgfqpoint{2.700000in}{1.950000in}}%
\pgfusepath{clip}%
\pgfsetbuttcap%
\pgfsetroundjoin%
\definecolor{currentfill}{rgb}{0.929718,0.872472,0.877007}%
\pgfsetfillcolor{currentfill}%
\pgfsetlinewidth{0.000000pt}%
\definecolor{currentstroke}{rgb}{0.000000,0.000000,0.000000}%
\pgfsetstrokecolor{currentstroke}%
\pgfsetdash{}{0pt}%
\pgfpathmoveto{\pgfqpoint{1.720364in}{1.240731in}}%
\pgfpathlineto{\pgfqpoint{1.756747in}{1.259319in}}%
\pgfpathlineto{\pgfqpoint{1.720133in}{1.302450in}}%
\pgfpathlineto{\pgfqpoint{1.683666in}{1.283928in}}%
\pgfpathclose%
\pgfusepath{fill}%
\end{pgfscope}%
\begin{pgfscope}%
\pgfpathrectangle{\pgfqpoint{0.150000in}{0.150000in}}{\pgfqpoint{2.700000in}{1.950000in}}%
\pgfusepath{clip}%
\pgfsetbuttcap%
\pgfsetroundjoin%
\definecolor{currentfill}{rgb}{0.808150,0.651884,0.664262}%
\pgfsetfillcolor{currentfill}%
\pgfsetlinewidth{0.000000pt}%
\definecolor{currentstroke}{rgb}{0.000000,0.000000,0.000000}%
\pgfsetstrokecolor{currentstroke}%
\pgfsetdash{}{0pt}%
\pgfpathmoveto{\pgfqpoint{1.904581in}{0.987381in}}%
\pgfpathlineto{\pgfqpoint{1.940203in}{0.994455in}}%
\pgfpathlineto{\pgfqpoint{1.903695in}{1.037436in}}%
\pgfpathlineto{\pgfqpoint{1.867946in}{1.030484in}}%
\pgfpathclose%
\pgfusepath{fill}%
\end{pgfscope}%
\begin{pgfscope}%
\pgfpathrectangle{\pgfqpoint{0.150000in}{0.150000in}}{\pgfqpoint{2.700000in}{1.950000in}}%
\pgfusepath{clip}%
\pgfsetbuttcap%
\pgfsetroundjoin%
\definecolor{currentfill}{rgb}{0.952512,0.913833,0.916896}%
\pgfsetfillcolor{currentfill}%
\pgfsetlinewidth{0.000000pt}%
\definecolor{currentstroke}{rgb}{0.000000,0.000000,0.000000}%
\pgfsetstrokecolor{currentstroke}%
\pgfsetdash{}{0pt}%
\pgfpathmoveto{\pgfqpoint{1.610161in}{1.296276in}}%
\pgfpathlineto{\pgfqpoint{1.646802in}{1.314796in}}%
\pgfpathlineto{\pgfqpoint{1.609899in}{1.333251in}}%
\pgfpathlineto{\pgfqpoint{1.573258in}{1.314796in}}%
\pgfpathclose%
\pgfusepath{fill}%
\end{pgfscope}%
\begin{pgfscope}%
\pgfpathrectangle{\pgfqpoint{0.150000in}{0.150000in}}{\pgfqpoint{2.700000in}{1.950000in}}%
\pgfusepath{clip}%
\pgfsetbuttcap%
\pgfsetroundjoin%
\definecolor{currentfill}{rgb}{0.952512,0.913833,0.916896}%
\pgfsetfillcolor{currentfill}%
\pgfsetlinewidth{0.000000pt}%
\definecolor{currentstroke}{rgb}{0.000000,0.000000,0.000000}%
\pgfsetstrokecolor{currentstroke}%
\pgfsetdash{}{0pt}%
\pgfpathmoveto{\pgfqpoint{1.536486in}{1.296276in}}%
\pgfpathlineto{\pgfqpoint{1.573258in}{1.314796in}}%
\pgfpathlineto{\pgfqpoint{1.536486in}{1.333251in}}%
\pgfpathlineto{\pgfqpoint{1.499715in}{1.314796in}}%
\pgfpathclose%
\pgfusepath{fill}%
\end{pgfscope}%
\begin{pgfscope}%
\pgfpathrectangle{\pgfqpoint{0.150000in}{0.150000in}}{\pgfqpoint{2.700000in}{1.950000in}}%
\pgfusepath{clip}%
\pgfsetbuttcap%
\pgfsetroundjoin%
\definecolor{currentfill}{rgb}{0.952512,0.913833,0.916896}%
\pgfsetfillcolor{currentfill}%
\pgfsetlinewidth{0.000000pt}%
\definecolor{currentstroke}{rgb}{0.000000,0.000000,0.000000}%
\pgfsetstrokecolor{currentstroke}%
\pgfsetdash{}{0pt}%
\pgfpathmoveto{\pgfqpoint{1.462812in}{1.296276in}}%
\pgfpathlineto{\pgfqpoint{1.499715in}{1.314796in}}%
\pgfpathlineto{\pgfqpoint{1.463074in}{1.333251in}}%
\pgfpathlineto{\pgfqpoint{1.426171in}{1.314796in}}%
\pgfpathclose%
\pgfusepath{fill}%
\end{pgfscope}%
\begin{pgfscope}%
\pgfpathrectangle{\pgfqpoint{0.150000in}{0.150000in}}{\pgfqpoint{2.700000in}{1.950000in}}%
\pgfusepath{clip}%
\pgfsetbuttcap%
\pgfsetroundjoin%
\definecolor{currentfill}{rgb}{0.952512,0.913833,0.916896}%
\pgfsetfillcolor{currentfill}%
\pgfsetlinewidth{0.000000pt}%
\definecolor{currentstroke}{rgb}{0.000000,0.000000,0.000000}%
\pgfsetstrokecolor{currentstroke}%
\pgfsetdash{}{0pt}%
\pgfpathmoveto{\pgfqpoint{1.389137in}{1.296276in}}%
\pgfpathlineto{\pgfqpoint{1.426171in}{1.314796in}}%
\pgfpathlineto{\pgfqpoint{1.389661in}{1.333251in}}%
\pgfpathlineto{\pgfqpoint{1.352628in}{1.314796in}}%
\pgfpathclose%
\pgfusepath{fill}%
\end{pgfscope}%
\begin{pgfscope}%
\pgfpathrectangle{\pgfqpoint{0.150000in}{0.150000in}}{\pgfqpoint{2.700000in}{1.950000in}}%
\pgfusepath{clip}%
\pgfsetbuttcap%
\pgfsetroundjoin%
\definecolor{currentfill}{rgb}{0.952512,0.913833,0.916896}%
\pgfsetfillcolor{currentfill}%
\pgfsetlinewidth{0.000000pt}%
\definecolor{currentstroke}{rgb}{0.000000,0.000000,0.000000}%
\pgfsetstrokecolor{currentstroke}%
\pgfsetdash{}{0pt}%
\pgfpathmoveto{\pgfqpoint{1.315462in}{1.296276in}}%
\pgfpathlineto{\pgfqpoint{1.352628in}{1.314796in}}%
\pgfpathlineto{\pgfqpoint{1.316248in}{1.333251in}}%
\pgfpathlineto{\pgfqpoint{1.279084in}{1.314796in}}%
\pgfpathclose%
\pgfusepath{fill}%
\end{pgfscope}%
\begin{pgfscope}%
\pgfpathrectangle{\pgfqpoint{0.150000in}{0.150000in}}{\pgfqpoint{2.700000in}{1.950000in}}%
\pgfusepath{clip}%
\pgfsetbuttcap%
\pgfsetroundjoin%
\definecolor{currentfill}{rgb}{0.952512,0.913833,0.916896}%
\pgfsetfillcolor{currentfill}%
\pgfsetlinewidth{0.000000pt}%
\definecolor{currentstroke}{rgb}{0.000000,0.000000,0.000000}%
\pgfsetstrokecolor{currentstroke}%
\pgfsetdash{}{0pt}%
\pgfpathmoveto{\pgfqpoint{1.241787in}{1.296276in}}%
\pgfpathlineto{\pgfqpoint{1.279084in}{1.314796in}}%
\pgfpathlineto{\pgfqpoint{1.242835in}{1.333251in}}%
\pgfpathlineto{\pgfqpoint{1.205540in}{1.314796in}}%
\pgfpathclose%
\pgfusepath{fill}%
\end{pgfscope}%
\begin{pgfscope}%
\pgfpathrectangle{\pgfqpoint{0.150000in}{0.150000in}}{\pgfqpoint{2.700000in}{1.950000in}}%
\pgfusepath{clip}%
\pgfsetbuttcap%
\pgfsetroundjoin%
\definecolor{currentfill}{rgb}{0.952512,0.913833,0.916896}%
\pgfsetfillcolor{currentfill}%
\pgfsetlinewidth{0.000000pt}%
\definecolor{currentstroke}{rgb}{0.000000,0.000000,0.000000}%
\pgfsetstrokecolor{currentstroke}%
\pgfsetdash{}{0pt}%
\pgfpathmoveto{\pgfqpoint{1.168112in}{1.296276in}}%
\pgfpathlineto{\pgfqpoint{1.205540in}{1.314796in}}%
\pgfpathlineto{\pgfqpoint{1.169423in}{1.333251in}}%
\pgfpathlineto{\pgfqpoint{1.131997in}{1.314796in}}%
\pgfpathclose%
\pgfusepath{fill}%
\end{pgfscope}%
\begin{pgfscope}%
\pgfpathrectangle{\pgfqpoint{0.150000in}{0.150000in}}{\pgfqpoint{2.700000in}{1.950000in}}%
\pgfusepath{clip}%
\pgfsetbuttcap%
\pgfsetroundjoin%
\definecolor{currentfill}{rgb}{0.952512,0.913833,0.916896}%
\pgfsetfillcolor{currentfill}%
\pgfsetlinewidth{0.000000pt}%
\definecolor{currentstroke}{rgb}{0.000000,0.000000,0.000000}%
\pgfsetstrokecolor{currentstroke}%
\pgfsetdash{}{0pt}%
\pgfpathmoveto{\pgfqpoint{1.094437in}{1.296276in}}%
\pgfpathlineto{\pgfqpoint{1.131997in}{1.314796in}}%
\pgfpathlineto{\pgfqpoint{1.096010in}{1.333251in}}%
\pgfpathlineto{\pgfqpoint{1.058453in}{1.314796in}}%
\pgfpathclose%
\pgfusepath{fill}%
\end{pgfscope}%
\begin{pgfscope}%
\pgfpathrectangle{\pgfqpoint{0.150000in}{0.150000in}}{\pgfqpoint{2.700000in}{1.950000in}}%
\pgfusepath{clip}%
\pgfsetbuttcap%
\pgfsetroundjoin%
\definecolor{currentfill}{rgb}{0.830944,0.693244,0.704151}%
\pgfsetfillcolor{currentfill}%
\pgfsetlinewidth{0.000000pt}%
\definecolor{currentstroke}{rgb}{0.000000,0.000000,0.000000}%
\pgfsetstrokecolor{currentstroke}%
\pgfsetdash{}{0pt}%
\pgfpathmoveto{\pgfqpoint{1.867946in}{1.030484in}}%
\pgfpathlineto{\pgfqpoint{1.903695in}{1.037436in}}%
\pgfpathlineto{\pgfqpoint{1.867149in}{1.080462in}}%
\pgfpathlineto{\pgfqpoint{1.831273in}{1.073633in}}%
\pgfpathclose%
\pgfusepath{fill}%
\end{pgfscope}%
\begin{pgfscope}%
\pgfpathrectangle{\pgfqpoint{0.150000in}{0.150000in}}{\pgfqpoint{2.700000in}{1.950000in}}%
\pgfusepath{clip}%
\pgfsetbuttcap%
\pgfsetroundjoin%
\definecolor{currentfill}{rgb}{0.868934,0.762178,0.770634}%
\pgfsetfillcolor{currentfill}%
\pgfsetlinewidth{0.000000pt}%
\definecolor{currentstroke}{rgb}{0.000000,0.000000,0.000000}%
\pgfsetstrokecolor{currentstroke}%
\pgfsetdash{}{0pt}%
\pgfpathmoveto{\pgfqpoint{1.794263in}{1.104678in}}%
\pgfpathlineto{\pgfqpoint{1.830564in}{1.123534in}}%
\pgfpathlineto{\pgfqpoint{1.793643in}{1.154473in}}%
\pgfpathlineto{\pgfqpoint{1.757555in}{1.147862in}}%
\pgfpathclose%
\pgfusepath{fill}%
\end{pgfscope}%
\begin{pgfscope}%
\pgfpathrectangle{\pgfqpoint{0.150000in}{0.150000in}}{\pgfqpoint{2.700000in}{1.950000in}}%
\pgfusepath{clip}%
\pgfsetbuttcap%
\pgfsetroundjoin%
\definecolor{currentfill}{rgb}{0.762561,0.569164,0.584482}%
\pgfsetfillcolor{currentfill}%
\pgfsetlinewidth{0.000000pt}%
\definecolor{currentstroke}{rgb}{0.000000,0.000000,0.000000}%
\pgfsetstrokecolor{currentstroke}%
\pgfsetdash{}{0pt}%
\pgfpathmoveto{\pgfqpoint{2.311840in}{0.905871in}}%
\pgfpathlineto{\pgfqpoint{2.346374in}{0.913207in}}%
\pgfpathlineto{\pgfqpoint{2.306407in}{0.908629in}}%
\pgfpathlineto{\pgfqpoint{2.271897in}{0.901310in}}%
\pgfpathclose%
\pgfusepath{fill}%
\end{pgfscope}%
\begin{pgfscope}%
\pgfpathrectangle{\pgfqpoint{0.150000in}{0.150000in}}{\pgfqpoint{2.700000in}{1.950000in}}%
\pgfusepath{clip}%
\pgfsetbuttcap%
\pgfsetroundjoin%
\definecolor{currentfill}{rgb}{0.960110,0.927619,0.930193}%
\pgfsetfillcolor{currentfill}%
\pgfsetlinewidth{0.000000pt}%
\definecolor{currentstroke}{rgb}{0.000000,0.000000,0.000000}%
\pgfsetstrokecolor{currentstroke}%
\pgfsetdash{}{0pt}%
\pgfpathmoveto{\pgfqpoint{1.020165in}{1.308651in}}%
\pgfpathlineto{\pgfqpoint{1.058453in}{1.314796in}}%
\pgfpathlineto{\pgfqpoint{1.022597in}{1.333251in}}%
\pgfpathlineto{\pgfqpoint{0.984272in}{1.327171in}}%
\pgfpathclose%
\pgfusepath{fill}%
\end{pgfscope}%
\begin{pgfscope}%
\pgfpathrectangle{\pgfqpoint{0.150000in}{0.150000in}}{\pgfqpoint{2.700000in}{1.950000in}}%
\pgfusepath{clip}%
\pgfsetbuttcap%
\pgfsetroundjoin%
\definecolor{currentfill}{rgb}{0.853738,0.734605,0.744041}%
\pgfsetfillcolor{currentfill}%
\pgfsetlinewidth{0.000000pt}%
\definecolor{currentstroke}{rgb}{0.000000,0.000000,0.000000}%
\pgfsetstrokecolor{currentstroke}%
\pgfsetdash{}{0pt}%
\pgfpathmoveto{\pgfqpoint{1.831273in}{1.073633in}}%
\pgfpathlineto{\pgfqpoint{1.867149in}{1.080462in}}%
\pgfpathlineto{\pgfqpoint{1.830564in}{1.123534in}}%
\pgfpathlineto{\pgfqpoint{1.794263in}{1.104678in}}%
\pgfpathclose%
\pgfusepath{fill}%
\end{pgfscope}%
\begin{pgfscope}%
\pgfpathrectangle{\pgfqpoint{0.150000in}{0.150000in}}{\pgfqpoint{2.700000in}{1.950000in}}%
\pgfusepath{clip}%
\pgfsetbuttcap%
\pgfsetroundjoin%
\definecolor{currentfill}{rgb}{0.891728,0.803539,0.810524}%
\pgfsetfillcolor{currentfill}%
\pgfsetlinewidth{0.000000pt}%
\definecolor{currentstroke}{rgb}{0.000000,0.000000,0.000000}%
\pgfsetstrokecolor{currentstroke}%
\pgfsetdash{}{0pt}%
\pgfpathmoveto{\pgfqpoint{1.757555in}{1.147862in}}%
\pgfpathlineto{\pgfqpoint{1.793643in}{1.154473in}}%
\pgfpathlineto{\pgfqpoint{1.757023in}{1.197579in}}%
\pgfpathlineto{\pgfqpoint{1.720808in}{1.191092in}}%
\pgfpathclose%
\pgfusepath{fill}%
\end{pgfscope}%
\begin{pgfscope}%
\pgfpathrectangle{\pgfqpoint{0.150000in}{0.150000in}}{\pgfqpoint{2.700000in}{1.950000in}}%
\pgfusepath{clip}%
\pgfsetbuttcap%
\pgfsetroundjoin%
\definecolor{currentfill}{rgb}{0.948713,0.906939,0.910248}%
\pgfsetfillcolor{currentfill}%
\pgfsetlinewidth{0.000000pt}%
\definecolor{currentstroke}{rgb}{0.000000,0.000000,0.000000}%
\pgfsetstrokecolor{currentstroke}%
\pgfsetdash{}{0pt}%
\pgfpathmoveto{\pgfqpoint{1.647068in}{1.265341in}}%
\pgfpathlineto{\pgfqpoint{1.683666in}{1.283928in}}%
\pgfpathlineto{\pgfqpoint{1.646802in}{1.314796in}}%
\pgfpathlineto{\pgfqpoint{1.610161in}{1.296276in}}%
\pgfpathclose%
\pgfusepath{fill}%
\end{pgfscope}%
\begin{pgfscope}%
\pgfpathrectangle{\pgfqpoint{0.150000in}{0.150000in}}{\pgfqpoint{2.700000in}{1.950000in}}%
\pgfusepath{clip}%
\pgfsetbuttcap%
\pgfsetroundjoin%
\definecolor{currentfill}{rgb}{0.914522,0.844899,0.850414}%
\pgfsetfillcolor{currentfill}%
\pgfsetlinewidth{0.000000pt}%
\definecolor{currentstroke}{rgb}{0.000000,0.000000,0.000000}%
\pgfsetstrokecolor{currentstroke}%
\pgfsetdash{}{0pt}%
\pgfpathmoveto{\pgfqpoint{1.720808in}{1.191092in}}%
\pgfpathlineto{\pgfqpoint{1.757023in}{1.197579in}}%
\pgfpathlineto{\pgfqpoint{1.720364in}{1.240731in}}%
\pgfpathlineto{\pgfqpoint{1.684021in}{1.234367in}}%
\pgfpathclose%
\pgfusepath{fill}%
\end{pgfscope}%
\begin{pgfscope}%
\pgfpathrectangle{\pgfqpoint{0.150000in}{0.150000in}}{\pgfqpoint{2.700000in}{1.950000in}}%
\pgfusepath{clip}%
\pgfsetbuttcap%
\pgfsetroundjoin%
\definecolor{currentfill}{rgb}{0.933517,0.879366,0.883655}%
\pgfsetfillcolor{currentfill}%
\pgfsetlinewidth{0.000000pt}%
\definecolor{currentstroke}{rgb}{0.000000,0.000000,0.000000}%
\pgfsetstrokecolor{currentstroke}%
\pgfsetdash{}{0pt}%
\pgfpathmoveto{\pgfqpoint{1.684021in}{1.234367in}}%
\pgfpathlineto{\pgfqpoint{1.720364in}{1.240731in}}%
\pgfpathlineto{\pgfqpoint{1.683666in}{1.283928in}}%
\pgfpathlineto{\pgfqpoint{1.647068in}{1.265341in}}%
\pgfpathclose%
\pgfusepath{fill}%
\end{pgfscope}%
\begin{pgfscope}%
\pgfpathrectangle{\pgfqpoint{0.150000in}{0.150000in}}{\pgfqpoint{2.700000in}{1.950000in}}%
\pgfusepath{clip}%
\pgfsetbuttcap%
\pgfsetroundjoin%
\definecolor{currentfill}{rgb}{0.773958,0.589844,0.604427}%
\pgfsetfillcolor{currentfill}%
\pgfsetlinewidth{0.000000pt}%
\definecolor{currentstroke}{rgb}{0.000000,0.000000,0.000000}%
\pgfsetstrokecolor{currentstroke}%
\pgfsetdash{}{0pt}%
\pgfpathmoveto{\pgfqpoint{2.386669in}{0.917822in}}%
\pgfpathlineto{\pgfqpoint{2.421022in}{0.925131in}}%
\pgfpathlineto{\pgfqpoint{2.380705in}{0.920500in}}%
\pgfpathlineto{\pgfqpoint{2.346374in}{0.913207in}}%
\pgfpathclose%
\pgfusepath{fill}%
\end{pgfscope}%
\begin{pgfscope}%
\pgfpathrectangle{\pgfqpoint{0.150000in}{0.150000in}}{\pgfqpoint{2.700000in}{1.950000in}}%
\pgfusepath{clip}%
\pgfsetbuttcap%
\pgfsetroundjoin%
\definecolor{currentfill}{rgb}{0.952512,0.913833,0.916896}%
\pgfsetfillcolor{currentfill}%
\pgfsetlinewidth{0.000000pt}%
\definecolor{currentstroke}{rgb}{0.000000,0.000000,0.000000}%
\pgfsetstrokecolor{currentstroke}%
\pgfsetdash{}{0pt}%
\pgfpathmoveto{\pgfqpoint{1.573390in}{1.277689in}}%
\pgfpathlineto{\pgfqpoint{1.610161in}{1.296276in}}%
\pgfpathlineto{\pgfqpoint{1.573258in}{1.314796in}}%
\pgfpathlineto{\pgfqpoint{1.536486in}{1.296276in}}%
\pgfpathclose%
\pgfusepath{fill}%
\end{pgfscope}%
\begin{pgfscope}%
\pgfpathrectangle{\pgfqpoint{0.150000in}{0.150000in}}{\pgfqpoint{2.700000in}{1.950000in}}%
\pgfusepath{clip}%
\pgfsetbuttcap%
\pgfsetroundjoin%
\definecolor{currentfill}{rgb}{0.952512,0.913833,0.916896}%
\pgfsetfillcolor{currentfill}%
\pgfsetlinewidth{0.000000pt}%
\definecolor{currentstroke}{rgb}{0.000000,0.000000,0.000000}%
\pgfsetstrokecolor{currentstroke}%
\pgfsetdash{}{0pt}%
\pgfpathmoveto{\pgfqpoint{1.499583in}{1.277689in}}%
\pgfpathlineto{\pgfqpoint{1.536486in}{1.296276in}}%
\pgfpathlineto{\pgfqpoint{1.499715in}{1.314796in}}%
\pgfpathlineto{\pgfqpoint{1.462812in}{1.296276in}}%
\pgfpathclose%
\pgfusepath{fill}%
\end{pgfscope}%
\begin{pgfscope}%
\pgfpathrectangle{\pgfqpoint{0.150000in}{0.150000in}}{\pgfqpoint{2.700000in}{1.950000in}}%
\pgfusepath{clip}%
\pgfsetbuttcap%
\pgfsetroundjoin%
\definecolor{currentfill}{rgb}{0.952512,0.913833,0.916896}%
\pgfsetfillcolor{currentfill}%
\pgfsetlinewidth{0.000000pt}%
\definecolor{currentstroke}{rgb}{0.000000,0.000000,0.000000}%
\pgfsetstrokecolor{currentstroke}%
\pgfsetdash{}{0pt}%
\pgfpathmoveto{\pgfqpoint{1.425777in}{1.277689in}}%
\pgfpathlineto{\pgfqpoint{1.462812in}{1.296276in}}%
\pgfpathlineto{\pgfqpoint{1.426171in}{1.314796in}}%
\pgfpathlineto{\pgfqpoint{1.389137in}{1.296276in}}%
\pgfpathclose%
\pgfusepath{fill}%
\end{pgfscope}%
\begin{pgfscope}%
\pgfpathrectangle{\pgfqpoint{0.150000in}{0.150000in}}{\pgfqpoint{2.700000in}{1.950000in}}%
\pgfusepath{clip}%
\pgfsetbuttcap%
\pgfsetroundjoin%
\definecolor{currentfill}{rgb}{0.952512,0.913833,0.916896}%
\pgfsetfillcolor{currentfill}%
\pgfsetlinewidth{0.000000pt}%
\definecolor{currentstroke}{rgb}{0.000000,0.000000,0.000000}%
\pgfsetstrokecolor{currentstroke}%
\pgfsetdash{}{0pt}%
\pgfpathmoveto{\pgfqpoint{1.351970in}{1.277689in}}%
\pgfpathlineto{\pgfqpoint{1.389137in}{1.296276in}}%
\pgfpathlineto{\pgfqpoint{1.352628in}{1.314796in}}%
\pgfpathlineto{\pgfqpoint{1.315462in}{1.296276in}}%
\pgfpathclose%
\pgfusepath{fill}%
\end{pgfscope}%
\begin{pgfscope}%
\pgfpathrectangle{\pgfqpoint{0.150000in}{0.150000in}}{\pgfqpoint{2.700000in}{1.950000in}}%
\pgfusepath{clip}%
\pgfsetbuttcap%
\pgfsetroundjoin%
\definecolor{currentfill}{rgb}{0.952512,0.913833,0.916896}%
\pgfsetfillcolor{currentfill}%
\pgfsetlinewidth{0.000000pt}%
\definecolor{currentstroke}{rgb}{0.000000,0.000000,0.000000}%
\pgfsetstrokecolor{currentstroke}%
\pgfsetdash{}{0pt}%
\pgfpathmoveto{\pgfqpoint{1.278163in}{1.277689in}}%
\pgfpathlineto{\pgfqpoint{1.315462in}{1.296276in}}%
\pgfpathlineto{\pgfqpoint{1.279084in}{1.314796in}}%
\pgfpathlineto{\pgfqpoint{1.241787in}{1.296276in}}%
\pgfpathclose%
\pgfusepath{fill}%
\end{pgfscope}%
\begin{pgfscope}%
\pgfpathrectangle{\pgfqpoint{0.150000in}{0.150000in}}{\pgfqpoint{2.700000in}{1.950000in}}%
\pgfusepath{clip}%
\pgfsetbuttcap%
\pgfsetroundjoin%
\definecolor{currentfill}{rgb}{0.952512,0.913833,0.916896}%
\pgfsetfillcolor{currentfill}%
\pgfsetlinewidth{0.000000pt}%
\definecolor{currentstroke}{rgb}{0.000000,0.000000,0.000000}%
\pgfsetstrokecolor{currentstroke}%
\pgfsetdash{}{0pt}%
\pgfpathmoveto{\pgfqpoint{1.204357in}{1.277689in}}%
\pgfpathlineto{\pgfqpoint{1.241787in}{1.296276in}}%
\pgfpathlineto{\pgfqpoint{1.205540in}{1.314796in}}%
\pgfpathlineto{\pgfqpoint{1.168112in}{1.296276in}}%
\pgfpathclose%
\pgfusepath{fill}%
\end{pgfscope}%
\begin{pgfscope}%
\pgfpathrectangle{\pgfqpoint{0.150000in}{0.150000in}}{\pgfqpoint{2.700000in}{1.950000in}}%
\pgfusepath{clip}%
\pgfsetbuttcap%
\pgfsetroundjoin%
\definecolor{currentfill}{rgb}{0.952512,0.913833,0.916896}%
\pgfsetfillcolor{currentfill}%
\pgfsetlinewidth{0.000000pt}%
\definecolor{currentstroke}{rgb}{0.000000,0.000000,0.000000}%
\pgfsetstrokecolor{currentstroke}%
\pgfsetdash{}{0pt}%
\pgfpathmoveto{\pgfqpoint{1.130550in}{1.277689in}}%
\pgfpathlineto{\pgfqpoint{1.168112in}{1.296276in}}%
\pgfpathlineto{\pgfqpoint{1.131997in}{1.314796in}}%
\pgfpathlineto{\pgfqpoint{1.094437in}{1.296276in}}%
\pgfpathclose%
\pgfusepath{fill}%
\end{pgfscope}%
\begin{pgfscope}%
\pgfpathrectangle{\pgfqpoint{0.150000in}{0.150000in}}{\pgfqpoint{2.700000in}{1.950000in}}%
\pgfusepath{clip}%
\pgfsetbuttcap%
\pgfsetroundjoin%
\definecolor{currentfill}{rgb}{0.758762,0.562270,0.577834}%
\pgfsetfillcolor{currentfill}%
\pgfsetlinewidth{0.000000pt}%
\definecolor{currentstroke}{rgb}{0.000000,0.000000,0.000000}%
\pgfsetstrokecolor{currentstroke}%
\pgfsetdash{}{0pt}%
\pgfpathmoveto{\pgfqpoint{2.054315in}{0.886542in}}%
\pgfpathlineto{\pgfqpoint{2.089029in}{0.882052in}}%
\pgfpathlineto{\pgfqpoint{2.050680in}{0.889441in}}%
\pgfpathlineto{\pgfqpoint{2.015357in}{0.882052in}}%
\pgfpathclose%
\pgfusepath{fill}%
\end{pgfscope}%
\begin{pgfscope}%
\pgfpathrectangle{\pgfqpoint{0.150000in}{0.150000in}}{\pgfqpoint{2.700000in}{1.950000in}}%
\pgfusepath{clip}%
\pgfsetbuttcap%
\pgfsetroundjoin%
\definecolor{currentfill}{rgb}{0.781556,0.603631,0.617724}%
\pgfsetfillcolor{currentfill}%
\pgfsetlinewidth{0.000000pt}%
\definecolor{currentstroke}{rgb}{0.000000,0.000000,0.000000}%
\pgfsetstrokecolor{currentstroke}%
\pgfsetdash{}{0pt}%
\pgfpathmoveto{\pgfqpoint{1.942624in}{0.905871in}}%
\pgfpathlineto{\pgfqpoint{1.977733in}{0.901310in}}%
\pgfpathlineto{\pgfqpoint{1.941176in}{0.944323in}}%
\pgfpathlineto{\pgfqpoint{1.905898in}{0.949063in}}%
\pgfpathclose%
\pgfusepath{fill}%
\end{pgfscope}%
\begin{pgfscope}%
\pgfpathrectangle{\pgfqpoint{0.150000in}{0.150000in}}{\pgfqpoint{2.700000in}{1.950000in}}%
\pgfusepath{clip}%
\pgfsetbuttcap%
\pgfsetroundjoin%
\definecolor{currentfill}{rgb}{0.762561,0.569164,0.584482}%
\pgfsetfillcolor{currentfill}%
\pgfsetlinewidth{0.000000pt}%
\definecolor{currentstroke}{rgb}{0.000000,0.000000,0.000000}%
\pgfsetstrokecolor{currentstroke}%
\pgfsetdash{}{0pt}%
\pgfpathmoveto{\pgfqpoint{2.128291in}{0.886542in}}%
\pgfpathlineto{\pgfqpoint{2.163427in}{0.893948in}}%
\pgfpathlineto{\pgfqpoint{2.124137in}{0.889441in}}%
\pgfpathlineto{\pgfqpoint{2.089029in}{0.882052in}}%
\pgfpathclose%
\pgfusepath{fill}%
\end{pgfscope}%
\begin{pgfscope}%
\pgfpathrectangle{\pgfqpoint{0.150000in}{0.150000in}}{\pgfqpoint{2.700000in}{1.950000in}}%
\pgfusepath{clip}%
\pgfsetbuttcap%
\pgfsetroundjoin%
\definecolor{currentfill}{rgb}{0.960110,0.927619,0.930193}%
\pgfsetfillcolor{currentfill}%
\pgfsetlinewidth{0.000000pt}%
\definecolor{currentstroke}{rgb}{0.000000,0.000000,0.000000}%
\pgfsetstrokecolor{currentstroke}%
\pgfsetdash{}{0pt}%
\pgfpathmoveto{\pgfqpoint{1.056187in}{1.290065in}}%
\pgfpathlineto{\pgfqpoint{1.094437in}{1.296276in}}%
\pgfpathlineto{\pgfqpoint{1.058453in}{1.314796in}}%
\pgfpathlineto{\pgfqpoint{1.020165in}{1.308651in}}%
\pgfpathclose%
\pgfusepath{fill}%
\end{pgfscope}%
\begin{pgfscope}%
\pgfpathrectangle{\pgfqpoint{0.150000in}{0.150000in}}{\pgfqpoint{2.700000in}{1.950000in}}%
\pgfusepath{clip}%
\pgfsetbuttcap%
\pgfsetroundjoin%
\definecolor{currentfill}{rgb}{0.800551,0.638097,0.650965}%
\pgfsetfillcolor{currentfill}%
\pgfsetlinewidth{0.000000pt}%
\definecolor{currentstroke}{rgb}{0.000000,0.000000,0.000000}%
\pgfsetstrokecolor{currentstroke}%
\pgfsetdash{}{0pt}%
\pgfpathmoveto{\pgfqpoint{1.905898in}{0.949063in}}%
\pgfpathlineto{\pgfqpoint{1.941176in}{0.944323in}}%
\pgfpathlineto{\pgfqpoint{1.904581in}{0.987381in}}%
\pgfpathlineto{\pgfqpoint{1.868748in}{0.980264in}}%
\pgfpathclose%
\pgfusepath{fill}%
\end{pgfscope}%
\begin{pgfscope}%
\pgfpathrectangle{\pgfqpoint{0.150000in}{0.150000in}}{\pgfqpoint{2.700000in}{1.950000in}}%
\pgfusepath{clip}%
\pgfsetbuttcap%
\pgfsetroundjoin%
\definecolor{currentfill}{rgb}{0.766360,0.576057,0.591131}%
\pgfsetfillcolor{currentfill}%
\pgfsetlinewidth{0.000000pt}%
\definecolor{currentstroke}{rgb}{0.000000,0.000000,0.000000}%
\pgfsetstrokecolor{currentstroke}%
\pgfsetdash{}{0pt}%
\pgfpathmoveto{\pgfqpoint{2.202266in}{0.886542in}}%
\pgfpathlineto{\pgfqpoint{2.237184in}{0.893948in}}%
\pgfpathlineto{\pgfqpoint{2.197593in}{0.889441in}}%
\pgfpathlineto{\pgfqpoint{2.163427in}{0.893948in}}%
\pgfpathclose%
\pgfusepath{fill}%
\end{pgfscope}%
\begin{pgfscope}%
\pgfpathrectangle{\pgfqpoint{0.150000in}{0.150000in}}{\pgfqpoint{2.700000in}{1.950000in}}%
\pgfusepath{clip}%
\pgfsetbuttcap%
\pgfsetroundjoin%
\definecolor{currentfill}{rgb}{0.819547,0.672564,0.684206}%
\pgfsetfillcolor{currentfill}%
\pgfsetlinewidth{0.000000pt}%
\definecolor{currentstroke}{rgb}{0.000000,0.000000,0.000000}%
\pgfsetstrokecolor{currentstroke}%
\pgfsetdash{}{0pt}%
\pgfpathmoveto{\pgfqpoint{1.868748in}{0.980264in}}%
\pgfpathlineto{\pgfqpoint{1.904581in}{0.987381in}}%
\pgfpathlineto{\pgfqpoint{1.867946in}{1.030484in}}%
\pgfpathlineto{\pgfqpoint{1.831987in}{1.023491in}}%
\pgfpathclose%
\pgfusepath{fill}%
\end{pgfscope}%
\begin{pgfscope}%
\pgfpathrectangle{\pgfqpoint{0.150000in}{0.150000in}}{\pgfqpoint{2.700000in}{1.950000in}}%
\pgfusepath{clip}%
\pgfsetbuttcap%
\pgfsetroundjoin%
\definecolor{currentfill}{rgb}{0.770159,0.582950,0.597779}%
\pgfsetfillcolor{currentfill}%
\pgfsetlinewidth{0.000000pt}%
\definecolor{currentstroke}{rgb}{0.000000,0.000000,0.000000}%
\pgfsetstrokecolor{currentstroke}%
\pgfsetdash{}{0pt}%
\pgfpathmoveto{\pgfqpoint{1.980340in}{0.886542in}}%
\pgfpathlineto{\pgfqpoint{2.015357in}{0.882052in}}%
\pgfpathlineto{\pgfqpoint{1.977733in}{0.901310in}}%
\pgfpathlineto{\pgfqpoint{1.942624in}{0.905871in}}%
\pgfpathclose%
\pgfusepath{fill}%
\end{pgfscope}%
\begin{pgfscope}%
\pgfpathrectangle{\pgfqpoint{0.150000in}{0.150000in}}{\pgfqpoint{2.700000in}{1.950000in}}%
\pgfusepath{clip}%
\pgfsetbuttcap%
\pgfsetroundjoin%
\definecolor{currentfill}{rgb}{0.857537,0.741498,0.750689}%
\pgfsetfillcolor{currentfill}%
\pgfsetlinewidth{0.000000pt}%
\definecolor{currentstroke}{rgb}{0.000000,0.000000,0.000000}%
\pgfsetstrokecolor{currentstroke}%
\pgfsetdash{}{0pt}%
\pgfpathmoveto{\pgfqpoint{1.794886in}{1.054642in}}%
\pgfpathlineto{\pgfqpoint{1.831273in}{1.073633in}}%
\pgfpathlineto{\pgfqpoint{1.794263in}{1.104678in}}%
\pgfpathlineto{\pgfqpoint{1.758090in}{1.097904in}}%
\pgfpathclose%
\pgfusepath{fill}%
\end{pgfscope}%
\begin{pgfscope}%
\pgfpathrectangle{\pgfqpoint{0.150000in}{0.150000in}}{\pgfqpoint{2.700000in}{1.950000in}}%
\pgfusepath{clip}%
\pgfsetbuttcap%
\pgfsetroundjoin%
\definecolor{currentfill}{rgb}{0.948713,0.906939,0.910248}%
\pgfsetfillcolor{currentfill}%
\pgfsetlinewidth{0.000000pt}%
\definecolor{currentstroke}{rgb}{0.000000,0.000000,0.000000}%
\pgfsetstrokecolor{currentstroke}%
\pgfsetdash{}{0pt}%
\pgfpathmoveto{\pgfqpoint{1.610340in}{1.246687in}}%
\pgfpathlineto{\pgfqpoint{1.647068in}{1.265341in}}%
\pgfpathlineto{\pgfqpoint{1.610161in}{1.296276in}}%
\pgfpathlineto{\pgfqpoint{1.573390in}{1.277689in}}%
\pgfpathclose%
\pgfusepath{fill}%
\end{pgfscope}%
\begin{pgfscope}%
\pgfpathrectangle{\pgfqpoint{0.150000in}{0.150000in}}{\pgfqpoint{2.700000in}{1.950000in}}%
\pgfusepath{clip}%
\pgfsetbuttcap%
\pgfsetroundjoin%
\definecolor{currentfill}{rgb}{0.842341,0.713925,0.724096}%
\pgfsetfillcolor{currentfill}%
\pgfsetlinewidth{0.000000pt}%
\definecolor{currentstroke}{rgb}{0.000000,0.000000,0.000000}%
\pgfsetstrokecolor{currentstroke}%
\pgfsetdash{}{0pt}%
\pgfpathmoveto{\pgfqpoint{1.831987in}{1.023491in}}%
\pgfpathlineto{\pgfqpoint{1.867946in}{1.030484in}}%
\pgfpathlineto{\pgfqpoint{1.831273in}{1.073633in}}%
\pgfpathlineto{\pgfqpoint{1.794886in}{1.054642in}}%
\pgfpathclose%
\pgfusepath{fill}%
\end{pgfscope}%
\begin{pgfscope}%
\pgfpathrectangle{\pgfqpoint{0.150000in}{0.150000in}}{\pgfqpoint{2.700000in}{1.950000in}}%
\pgfusepath{clip}%
\pgfsetbuttcap%
\pgfsetroundjoin%
\definecolor{currentfill}{rgb}{0.880331,0.782858,0.790579}%
\pgfsetfillcolor{currentfill}%
\pgfsetlinewidth{0.000000pt}%
\definecolor{currentstroke}{rgb}{0.000000,0.000000,0.000000}%
\pgfsetstrokecolor{currentstroke}%
\pgfsetdash{}{0pt}%
\pgfpathmoveto{\pgfqpoint{1.758090in}{1.097904in}}%
\pgfpathlineto{\pgfqpoint{1.794263in}{1.104678in}}%
\pgfpathlineto{\pgfqpoint{1.757555in}{1.147862in}}%
\pgfpathlineto{\pgfqpoint{1.721254in}{1.141212in}}%
\pgfpathclose%
\pgfusepath{fill}%
\end{pgfscope}%
\begin{pgfscope}%
\pgfpathrectangle{\pgfqpoint{0.150000in}{0.150000in}}{\pgfqpoint{2.700000in}{1.950000in}}%
\pgfusepath{clip}%
\pgfsetbuttcap%
\pgfsetroundjoin%
\definecolor{currentfill}{rgb}{0.899326,0.817325,0.823820}%
\pgfsetfillcolor{currentfill}%
\pgfsetlinewidth{0.000000pt}%
\definecolor{currentstroke}{rgb}{0.000000,0.000000,0.000000}%
\pgfsetstrokecolor{currentstroke}%
\pgfsetdash{}{0pt}%
\pgfpathmoveto{\pgfqpoint{1.721254in}{1.141212in}}%
\pgfpathlineto{\pgfqpoint{1.757555in}{1.147862in}}%
\pgfpathlineto{\pgfqpoint{1.720808in}{1.191092in}}%
\pgfpathlineto{\pgfqpoint{1.684207in}{1.172303in}}%
\pgfpathclose%
\pgfusepath{fill}%
\end{pgfscope}%
\begin{pgfscope}%
\pgfpathrectangle{\pgfqpoint{0.150000in}{0.150000in}}{\pgfqpoint{2.700000in}{1.950000in}}%
\pgfusepath{clip}%
\pgfsetbuttcap%
\pgfsetroundjoin%
\definecolor{currentfill}{rgb}{0.773958,0.589844,0.604427}%
\pgfsetfillcolor{currentfill}%
\pgfsetlinewidth{0.000000pt}%
\definecolor{currentstroke}{rgb}{0.000000,0.000000,0.000000}%
\pgfsetstrokecolor{currentstroke}%
\pgfsetdash{}{0pt}%
\pgfpathmoveto{\pgfqpoint{2.277102in}{0.898492in}}%
\pgfpathlineto{\pgfqpoint{2.311840in}{0.905871in}}%
\pgfpathlineto{\pgfqpoint{2.271897in}{0.901310in}}%
\pgfpathlineto{\pgfqpoint{2.237184in}{0.893948in}}%
\pgfpathclose%
\pgfusepath{fill}%
\end{pgfscope}%
\begin{pgfscope}%
\pgfpathrectangle{\pgfqpoint{0.150000in}{0.150000in}}{\pgfqpoint{2.700000in}{1.950000in}}%
\pgfusepath{clip}%
\pgfsetbuttcap%
\pgfsetroundjoin%
\definecolor{currentfill}{rgb}{0.952512,0.913833,0.916896}%
\pgfsetfillcolor{currentfill}%
\pgfsetlinewidth{0.000000pt}%
\definecolor{currentstroke}{rgb}{0.000000,0.000000,0.000000}%
\pgfsetstrokecolor{currentstroke}%
\pgfsetdash{}{0pt}%
\pgfpathmoveto{\pgfqpoint{1.536486in}{1.259035in}}%
\pgfpathlineto{\pgfqpoint{1.573390in}{1.277689in}}%
\pgfpathlineto{\pgfqpoint{1.536486in}{1.296276in}}%
\pgfpathlineto{\pgfqpoint{1.499583in}{1.277689in}}%
\pgfpathclose%
\pgfusepath{fill}%
\end{pgfscope}%
\begin{pgfscope}%
\pgfpathrectangle{\pgfqpoint{0.150000in}{0.150000in}}{\pgfqpoint{2.700000in}{1.950000in}}%
\pgfusepath{clip}%
\pgfsetbuttcap%
\pgfsetroundjoin%
\definecolor{currentfill}{rgb}{0.952512,0.913833,0.916896}%
\pgfsetfillcolor{currentfill}%
\pgfsetlinewidth{0.000000pt}%
\definecolor{currentstroke}{rgb}{0.000000,0.000000,0.000000}%
\pgfsetstrokecolor{currentstroke}%
\pgfsetdash{}{0pt}%
\pgfpathmoveto{\pgfqpoint{1.462548in}{1.259035in}}%
\pgfpathlineto{\pgfqpoint{1.499583in}{1.277689in}}%
\pgfpathlineto{\pgfqpoint{1.462812in}{1.296276in}}%
\pgfpathlineto{\pgfqpoint{1.425777in}{1.277689in}}%
\pgfpathclose%
\pgfusepath{fill}%
\end{pgfscope}%
\begin{pgfscope}%
\pgfpathrectangle{\pgfqpoint{0.150000in}{0.150000in}}{\pgfqpoint{2.700000in}{1.950000in}}%
\pgfusepath{clip}%
\pgfsetbuttcap%
\pgfsetroundjoin%
\definecolor{currentfill}{rgb}{0.952512,0.913833,0.916896}%
\pgfsetfillcolor{currentfill}%
\pgfsetlinewidth{0.000000pt}%
\definecolor{currentstroke}{rgb}{0.000000,0.000000,0.000000}%
\pgfsetstrokecolor{currentstroke}%
\pgfsetdash{}{0pt}%
\pgfpathmoveto{\pgfqpoint{1.388609in}{1.259035in}}%
\pgfpathlineto{\pgfqpoint{1.425777in}{1.277689in}}%
\pgfpathlineto{\pgfqpoint{1.389137in}{1.296276in}}%
\pgfpathlineto{\pgfqpoint{1.351970in}{1.277689in}}%
\pgfpathclose%
\pgfusepath{fill}%
\end{pgfscope}%
\begin{pgfscope}%
\pgfpathrectangle{\pgfqpoint{0.150000in}{0.150000in}}{\pgfqpoint{2.700000in}{1.950000in}}%
\pgfusepath{clip}%
\pgfsetbuttcap%
\pgfsetroundjoin%
\definecolor{currentfill}{rgb}{0.952512,0.913833,0.916896}%
\pgfsetfillcolor{currentfill}%
\pgfsetlinewidth{0.000000pt}%
\definecolor{currentstroke}{rgb}{0.000000,0.000000,0.000000}%
\pgfsetstrokecolor{currentstroke}%
\pgfsetdash{}{0pt}%
\pgfpathmoveto{\pgfqpoint{1.314670in}{1.259035in}}%
\pgfpathlineto{\pgfqpoint{1.351970in}{1.277689in}}%
\pgfpathlineto{\pgfqpoint{1.315462in}{1.296276in}}%
\pgfpathlineto{\pgfqpoint{1.278163in}{1.277689in}}%
\pgfpathclose%
\pgfusepath{fill}%
\end{pgfscope}%
\begin{pgfscope}%
\pgfpathrectangle{\pgfqpoint{0.150000in}{0.150000in}}{\pgfqpoint{2.700000in}{1.950000in}}%
\pgfusepath{clip}%
\pgfsetbuttcap%
\pgfsetroundjoin%
\definecolor{currentfill}{rgb}{0.952512,0.913833,0.916896}%
\pgfsetfillcolor{currentfill}%
\pgfsetlinewidth{0.000000pt}%
\definecolor{currentstroke}{rgb}{0.000000,0.000000,0.000000}%
\pgfsetstrokecolor{currentstroke}%
\pgfsetdash{}{0pt}%
\pgfpathmoveto{\pgfqpoint{1.240731in}{1.259035in}}%
\pgfpathlineto{\pgfqpoint{1.278163in}{1.277689in}}%
\pgfpathlineto{\pgfqpoint{1.241787in}{1.296276in}}%
\pgfpathlineto{\pgfqpoint{1.204357in}{1.277689in}}%
\pgfpathclose%
\pgfusepath{fill}%
\end{pgfscope}%
\begin{pgfscope}%
\pgfpathrectangle{\pgfqpoint{0.150000in}{0.150000in}}{\pgfqpoint{2.700000in}{1.950000in}}%
\pgfusepath{clip}%
\pgfsetbuttcap%
\pgfsetroundjoin%
\definecolor{currentfill}{rgb}{0.952512,0.913833,0.916896}%
\pgfsetfillcolor{currentfill}%
\pgfsetlinewidth{0.000000pt}%
\definecolor{currentstroke}{rgb}{0.000000,0.000000,0.000000}%
\pgfsetstrokecolor{currentstroke}%
\pgfsetdash{}{0pt}%
\pgfpathmoveto{\pgfqpoint{1.166793in}{1.259035in}}%
\pgfpathlineto{\pgfqpoint{1.204357in}{1.277689in}}%
\pgfpathlineto{\pgfqpoint{1.168112in}{1.296276in}}%
\pgfpathlineto{\pgfqpoint{1.130550in}{1.277689in}}%
\pgfpathclose%
\pgfusepath{fill}%
\end{pgfscope}%
\begin{pgfscope}%
\pgfpathrectangle{\pgfqpoint{0.150000in}{0.150000in}}{\pgfqpoint{2.700000in}{1.950000in}}%
\pgfusepath{clip}%
\pgfsetbuttcap%
\pgfsetroundjoin%
\definecolor{currentfill}{rgb}{0.937316,0.886259,0.890303}%
\pgfsetfillcolor{currentfill}%
\pgfsetlinewidth{0.000000pt}%
\definecolor{currentstroke}{rgb}{0.000000,0.000000,0.000000}%
\pgfsetstrokecolor{currentstroke}%
\pgfsetdash{}{0pt}%
\pgfpathmoveto{\pgfqpoint{1.647336in}{1.215646in}}%
\pgfpathlineto{\pgfqpoint{1.684021in}{1.234367in}}%
\pgfpathlineto{\pgfqpoint{1.647068in}{1.265341in}}%
\pgfpathlineto{\pgfqpoint{1.610340in}{1.246687in}}%
\pgfpathclose%
\pgfusepath{fill}%
\end{pgfscope}%
\begin{pgfscope}%
\pgfpathrectangle{\pgfqpoint{0.150000in}{0.150000in}}{\pgfqpoint{2.700000in}{1.950000in}}%
\pgfusepath{clip}%
\pgfsetbuttcap%
\pgfsetroundjoin%
\definecolor{currentfill}{rgb}{0.918321,0.851792,0.857062}%
\pgfsetfillcolor{currentfill}%
\pgfsetlinewidth{0.000000pt}%
\definecolor{currentstroke}{rgb}{0.000000,0.000000,0.000000}%
\pgfsetstrokecolor{currentstroke}%
\pgfsetdash{}{0pt}%
\pgfpathmoveto{\pgfqpoint{1.684207in}{1.172303in}}%
\pgfpathlineto{\pgfqpoint{1.720808in}{1.191092in}}%
\pgfpathlineto{\pgfqpoint{1.684021in}{1.234367in}}%
\pgfpathlineto{\pgfqpoint{1.647336in}{1.215646in}}%
\pgfpathclose%
\pgfusepath{fill}%
\end{pgfscope}%
\begin{pgfscope}%
\pgfpathrectangle{\pgfqpoint{0.150000in}{0.150000in}}{\pgfqpoint{2.700000in}{1.950000in}}%
\pgfusepath{clip}%
\pgfsetbuttcap%
\pgfsetroundjoin%
\definecolor{currentfill}{rgb}{0.975306,0.955193,0.956786}%
\pgfsetfillcolor{currentfill}%
\pgfsetlinewidth{0.000000pt}%
\definecolor{currentstroke}{rgb}{0.000000,0.000000,0.000000}%
\pgfsetstrokecolor{currentstroke}%
\pgfsetdash{}{0pt}%
\pgfpathmoveto{\pgfqpoint{0.981006in}{1.314904in}}%
\pgfpathlineto{\pgfqpoint{1.020165in}{1.308651in}}%
\pgfpathlineto{\pgfqpoint{0.984272in}{1.327171in}}%
\pgfpathlineto{\pgfqpoint{0.945035in}{1.333489in}}%
\pgfpathclose%
\pgfusepath{fill}%
\end{pgfscope}%
\begin{pgfscope}%
\pgfpathrectangle{\pgfqpoint{0.150000in}{0.150000in}}{\pgfqpoint{2.700000in}{1.950000in}}%
\pgfusepath{clip}%
\pgfsetbuttcap%
\pgfsetroundjoin%
\definecolor{currentfill}{rgb}{0.960110,0.927619,0.930193}%
\pgfsetfillcolor{currentfill}%
\pgfsetlinewidth{0.000000pt}%
\definecolor{currentstroke}{rgb}{0.000000,0.000000,0.000000}%
\pgfsetstrokecolor{currentstroke}%
\pgfsetdash{}{0pt}%
\pgfpathmoveto{\pgfqpoint{1.092338in}{1.271412in}}%
\pgfpathlineto{\pgfqpoint{1.130550in}{1.277689in}}%
\pgfpathlineto{\pgfqpoint{1.094437in}{1.296276in}}%
\pgfpathlineto{\pgfqpoint{1.056187in}{1.290065in}}%
\pgfpathclose%
\pgfusepath{fill}%
\end{pgfscope}%
\begin{pgfscope}%
\pgfpathrectangle{\pgfqpoint{0.150000in}{0.150000in}}{\pgfqpoint{2.700000in}{1.950000in}}%
\pgfusepath{clip}%
\pgfsetbuttcap%
\pgfsetroundjoin%
\definecolor{currentfill}{rgb}{0.785355,0.610524,0.624372}%
\pgfsetfillcolor{currentfill}%
\pgfsetlinewidth{0.000000pt}%
\definecolor{currentstroke}{rgb}{0.000000,0.000000,0.000000}%
\pgfsetstrokecolor{currentstroke}%
\pgfsetdash{}{0pt}%
\pgfpathmoveto{\pgfqpoint{2.352112in}{0.910470in}}%
\pgfpathlineto{\pgfqpoint{2.386669in}{0.917822in}}%
\pgfpathlineto{\pgfqpoint{2.346374in}{0.913207in}}%
\pgfpathlineto{\pgfqpoint{2.311840in}{0.905871in}}%
\pgfpathclose%
\pgfusepath{fill}%
\end{pgfscope}%
\begin{pgfscope}%
\pgfpathrectangle{\pgfqpoint{0.150000in}{0.150000in}}{\pgfqpoint{2.700000in}{1.950000in}}%
\pgfusepath{clip}%
\pgfsetbuttcap%
\pgfsetroundjoin%
\definecolor{currentfill}{rgb}{0.948713,0.906939,0.910248}%
\pgfsetfillcolor{currentfill}%
\pgfsetlinewidth{0.000000pt}%
\definecolor{currentstroke}{rgb}{0.000000,0.000000,0.000000}%
\pgfsetstrokecolor{currentstroke}%
\pgfsetdash{}{0pt}%
\pgfpathmoveto{\pgfqpoint{1.573479in}{1.227966in}}%
\pgfpathlineto{\pgfqpoint{1.610340in}{1.246687in}}%
\pgfpathlineto{\pgfqpoint{1.573390in}{1.277689in}}%
\pgfpathlineto{\pgfqpoint{1.536486in}{1.259035in}}%
\pgfpathclose%
\pgfusepath{fill}%
\end{pgfscope}%
\begin{pgfscope}%
\pgfpathrectangle{\pgfqpoint{0.150000in}{0.150000in}}{\pgfqpoint{2.700000in}{1.950000in}}%
\pgfusepath{clip}%
\pgfsetbuttcap%
\pgfsetroundjoin%
\definecolor{currentfill}{rgb}{0.952512,0.913833,0.916896}%
\pgfsetfillcolor{currentfill}%
\pgfsetlinewidth{0.000000pt}%
\definecolor{currentstroke}{rgb}{0.000000,0.000000,0.000000}%
\pgfsetstrokecolor{currentstroke}%
\pgfsetdash{}{0pt}%
\pgfpathmoveto{\pgfqpoint{1.351308in}{1.240315in}}%
\pgfpathlineto{\pgfqpoint{1.388609in}{1.259035in}}%
\pgfpathlineto{\pgfqpoint{1.351970in}{1.277689in}}%
\pgfpathlineto{\pgfqpoint{1.314670in}{1.259035in}}%
\pgfpathclose%
\pgfusepath{fill}%
\end{pgfscope}%
\begin{pgfscope}%
\pgfpathrectangle{\pgfqpoint{0.150000in}{0.150000in}}{\pgfqpoint{2.700000in}{1.950000in}}%
\pgfusepath{clip}%
\pgfsetbuttcap%
\pgfsetroundjoin%
\definecolor{currentfill}{rgb}{0.952512,0.913833,0.916896}%
\pgfsetfillcolor{currentfill}%
\pgfsetlinewidth{0.000000pt}%
\definecolor{currentstroke}{rgb}{0.000000,0.000000,0.000000}%
\pgfsetstrokecolor{currentstroke}%
\pgfsetdash{}{0pt}%
\pgfpathmoveto{\pgfqpoint{1.425379in}{1.240315in}}%
\pgfpathlineto{\pgfqpoint{1.462548in}{1.259035in}}%
\pgfpathlineto{\pgfqpoint{1.425777in}{1.277689in}}%
\pgfpathlineto{\pgfqpoint{1.388609in}{1.259035in}}%
\pgfpathclose%
\pgfusepath{fill}%
\end{pgfscope}%
\begin{pgfscope}%
\pgfpathrectangle{\pgfqpoint{0.150000in}{0.150000in}}{\pgfqpoint{2.700000in}{1.950000in}}%
\pgfusepath{clip}%
\pgfsetbuttcap%
\pgfsetroundjoin%
\definecolor{currentfill}{rgb}{0.952512,0.913833,0.916896}%
\pgfsetfillcolor{currentfill}%
\pgfsetlinewidth{0.000000pt}%
\definecolor{currentstroke}{rgb}{0.000000,0.000000,0.000000}%
\pgfsetstrokecolor{currentstroke}%
\pgfsetdash{}{0pt}%
\pgfpathmoveto{\pgfqpoint{1.499451in}{1.240315in}}%
\pgfpathlineto{\pgfqpoint{1.536486in}{1.259035in}}%
\pgfpathlineto{\pgfqpoint{1.499583in}{1.277689in}}%
\pgfpathlineto{\pgfqpoint{1.462548in}{1.259035in}}%
\pgfpathclose%
\pgfusepath{fill}%
\end{pgfscope}%
\begin{pgfscope}%
\pgfpathrectangle{\pgfqpoint{0.150000in}{0.150000in}}{\pgfqpoint{2.700000in}{1.950000in}}%
\pgfusepath{clip}%
\pgfsetbuttcap%
\pgfsetroundjoin%
\definecolor{currentfill}{rgb}{0.952512,0.913833,0.916896}%
\pgfsetfillcolor{currentfill}%
\pgfsetlinewidth{0.000000pt}%
\definecolor{currentstroke}{rgb}{0.000000,0.000000,0.000000}%
\pgfsetstrokecolor{currentstroke}%
\pgfsetdash{}{0pt}%
\pgfpathmoveto{\pgfqpoint{1.277236in}{1.240315in}}%
\pgfpathlineto{\pgfqpoint{1.314670in}{1.259035in}}%
\pgfpathlineto{\pgfqpoint{1.278163in}{1.277689in}}%
\pgfpathlineto{\pgfqpoint{1.240731in}{1.259035in}}%
\pgfpathclose%
\pgfusepath{fill}%
\end{pgfscope}%
\begin{pgfscope}%
\pgfpathrectangle{\pgfqpoint{0.150000in}{0.150000in}}{\pgfqpoint{2.700000in}{1.950000in}}%
\pgfusepath{clip}%
\pgfsetbuttcap%
\pgfsetroundjoin%
\definecolor{currentfill}{rgb}{0.952512,0.913833,0.916896}%
\pgfsetfillcolor{currentfill}%
\pgfsetlinewidth{0.000000pt}%
\definecolor{currentstroke}{rgb}{0.000000,0.000000,0.000000}%
\pgfsetstrokecolor{currentstroke}%
\pgfsetdash{}{0pt}%
\pgfpathmoveto{\pgfqpoint{1.203165in}{1.240315in}}%
\pgfpathlineto{\pgfqpoint{1.240731in}{1.259035in}}%
\pgfpathlineto{\pgfqpoint{1.204357in}{1.277689in}}%
\pgfpathlineto{\pgfqpoint{1.166793in}{1.259035in}}%
\pgfpathclose%
\pgfusepath{fill}%
\end{pgfscope}%
\begin{pgfscope}%
\pgfpathrectangle{\pgfqpoint{0.150000in}{0.150000in}}{\pgfqpoint{2.700000in}{1.950000in}}%
\pgfusepath{clip}%
\pgfsetbuttcap%
\pgfsetroundjoin%
\definecolor{currentfill}{rgb}{0.937316,0.886259,0.890303}%
\pgfsetfillcolor{currentfill}%
\pgfsetlinewidth{0.000000pt}%
\definecolor{currentstroke}{rgb}{0.000000,0.000000,0.000000}%
\pgfsetstrokecolor{currentstroke}%
\pgfsetdash{}{0pt}%
\pgfpathmoveto{\pgfqpoint{1.610519in}{1.196858in}}%
\pgfpathlineto{\pgfqpoint{1.647336in}{1.215646in}}%
\pgfpathlineto{\pgfqpoint{1.610340in}{1.246687in}}%
\pgfpathlineto{\pgfqpoint{1.573479in}{1.227966in}}%
\pgfpathclose%
\pgfusepath{fill}%
\end{pgfscope}%
\begin{pgfscope}%
\pgfpathrectangle{\pgfqpoint{0.150000in}{0.150000in}}{\pgfqpoint{2.700000in}{1.950000in}}%
\pgfusepath{clip}%
\pgfsetbuttcap%
\pgfsetroundjoin%
\definecolor{currentfill}{rgb}{0.777757,0.596737,0.611075}%
\pgfsetfillcolor{currentfill}%
\pgfsetlinewidth{0.000000pt}%
\definecolor{currentstroke}{rgb}{0.000000,0.000000,0.000000}%
\pgfsetstrokecolor{currentstroke}%
\pgfsetdash{}{0pt}%
\pgfpathmoveto{\pgfqpoint{2.092946in}{0.879092in}}%
\pgfpathlineto{\pgfqpoint{2.128291in}{0.886542in}}%
\pgfpathlineto{\pgfqpoint{2.089029in}{0.882052in}}%
\pgfpathlineto{\pgfqpoint{2.054315in}{0.886542in}}%
\pgfpathclose%
\pgfusepath{fill}%
\end{pgfscope}%
\begin{pgfscope}%
\pgfpathrectangle{\pgfqpoint{0.150000in}{0.150000in}}{\pgfqpoint{2.700000in}{1.950000in}}%
\pgfusepath{clip}%
\pgfsetbuttcap%
\pgfsetroundjoin%
\definecolor{currentfill}{rgb}{0.777757,0.596737,0.611075}%
\pgfsetfillcolor{currentfill}%
\pgfsetlinewidth{0.000000pt}%
\definecolor{currentstroke}{rgb}{0.000000,0.000000,0.000000}%
\pgfsetstrokecolor{currentstroke}%
\pgfsetdash{}{0pt}%
\pgfpathmoveto{\pgfqpoint{2.018752in}{0.879092in}}%
\pgfpathlineto{\pgfqpoint{2.054315in}{0.886542in}}%
\pgfpathlineto{\pgfqpoint{2.015357in}{0.882052in}}%
\pgfpathlineto{\pgfqpoint{1.980340in}{0.886542in}}%
\pgfpathclose%
\pgfusepath{fill}%
\end{pgfscope}%
\begin{pgfscope}%
\pgfpathrectangle{\pgfqpoint{0.150000in}{0.150000in}}{\pgfqpoint{2.700000in}{1.950000in}}%
\pgfusepath{clip}%
\pgfsetbuttcap%
\pgfsetroundjoin%
\definecolor{currentfill}{rgb}{0.975306,0.955193,0.956786}%
\pgfsetfillcolor{currentfill}%
\pgfsetlinewidth{0.000000pt}%
\definecolor{currentstroke}{rgb}{0.000000,0.000000,0.000000}%
\pgfsetstrokecolor{currentstroke}%
\pgfsetdash{}{0pt}%
\pgfpathmoveto{\pgfqpoint{1.017710in}{1.283818in}}%
\pgfpathlineto{\pgfqpoint{1.056187in}{1.290065in}}%
\pgfpathlineto{\pgfqpoint{1.020165in}{1.308651in}}%
\pgfpathlineto{\pgfqpoint{0.981006in}{1.314904in}}%
\pgfpathclose%
\pgfusepath{fill}%
\end{pgfscope}%
\begin{pgfscope}%
\pgfpathrectangle{\pgfqpoint{0.150000in}{0.150000in}}{\pgfqpoint{2.700000in}{1.950000in}}%
\pgfusepath{clip}%
\pgfsetbuttcap%
\pgfsetroundjoin%
\definecolor{currentfill}{rgb}{0.865135,0.755285,0.763986}%
\pgfsetfillcolor{currentfill}%
\pgfsetlinewidth{0.000000pt}%
\definecolor{currentstroke}{rgb}{0.000000,0.000000,0.000000}%
\pgfsetstrokecolor{currentstroke}%
\pgfsetdash{}{0pt}%
\pgfpathmoveto{\pgfqpoint{1.758627in}{1.047704in}}%
\pgfpathlineto{\pgfqpoint{1.794886in}{1.054642in}}%
\pgfpathlineto{\pgfqpoint{1.758090in}{1.097904in}}%
\pgfpathlineto{\pgfqpoint{1.721702in}{1.091091in}}%
\pgfpathclose%
\pgfusepath{fill}%
\end{pgfscope}%
\begin{pgfscope}%
\pgfpathrectangle{\pgfqpoint{0.150000in}{0.150000in}}{\pgfqpoint{2.700000in}{1.950000in}}%
\pgfusepath{clip}%
\pgfsetbuttcap%
\pgfsetroundjoin%
\definecolor{currentfill}{rgb}{0.956311,0.920726,0.923545}%
\pgfsetfillcolor{currentfill}%
\pgfsetlinewidth{0.000000pt}%
\definecolor{currentstroke}{rgb}{0.000000,0.000000,0.000000}%
\pgfsetstrokecolor{currentstroke}%
\pgfsetdash{}{0pt}%
\pgfpathmoveto{\pgfqpoint{1.129094in}{1.240315in}}%
\pgfpathlineto{\pgfqpoint{1.166793in}{1.259035in}}%
\pgfpathlineto{\pgfqpoint{1.130550in}{1.277689in}}%
\pgfpathlineto{\pgfqpoint{1.092338in}{1.271412in}}%
\pgfpathclose%
\pgfusepath{fill}%
\end{pgfscope}%
\begin{pgfscope}%
\pgfpathrectangle{\pgfqpoint{0.150000in}{0.150000in}}{\pgfqpoint{2.700000in}{1.950000in}}%
\pgfusepath{clip}%
\pgfsetbuttcap%
\pgfsetroundjoin%
\definecolor{currentfill}{rgb}{0.849939,0.727711,0.737393}%
\pgfsetfillcolor{currentfill}%
\pgfsetlinewidth{0.000000pt}%
\definecolor{currentstroke}{rgb}{0.000000,0.000000,0.000000}%
\pgfsetstrokecolor{currentstroke}%
\pgfsetdash{}{0pt}%
\pgfpathmoveto{\pgfqpoint{1.795814in}{1.016456in}}%
\pgfpathlineto{\pgfqpoint{1.831987in}{1.023491in}}%
\pgfpathlineto{\pgfqpoint{1.794886in}{1.054642in}}%
\pgfpathlineto{\pgfqpoint{1.758627in}{1.047704in}}%
\pgfpathclose%
\pgfusepath{fill}%
\end{pgfscope}%
\begin{pgfscope}%
\pgfpathrectangle{\pgfqpoint{0.150000in}{0.150000in}}{\pgfqpoint{2.700000in}{1.950000in}}%
\pgfusepath{clip}%
\pgfsetbuttcap%
\pgfsetroundjoin%
\definecolor{currentfill}{rgb}{0.922120,0.858686,0.863710}%
\pgfsetfillcolor{currentfill}%
\pgfsetlinewidth{0.000000pt}%
\definecolor{currentstroke}{rgb}{0.000000,0.000000,0.000000}%
\pgfsetstrokecolor{currentstroke}%
\pgfsetdash{}{0pt}%
\pgfpathmoveto{\pgfqpoint{1.647605in}{1.165710in}}%
\pgfpathlineto{\pgfqpoint{1.684207in}{1.172303in}}%
\pgfpathlineto{\pgfqpoint{1.647336in}{1.215646in}}%
\pgfpathlineto{\pgfqpoint{1.610519in}{1.196858in}}%
\pgfpathclose%
\pgfusepath{fill}%
\end{pgfscope}%
\begin{pgfscope}%
\pgfpathrectangle{\pgfqpoint{0.150000in}{0.150000in}}{\pgfqpoint{2.700000in}{1.950000in}}%
\pgfusepath{clip}%
\pgfsetbuttcap%
\pgfsetroundjoin%
\definecolor{currentfill}{rgb}{0.903125,0.824219,0.830469}%
\pgfsetfillcolor{currentfill}%
\pgfsetlinewidth{0.000000pt}%
\definecolor{currentstroke}{rgb}{0.000000,0.000000,0.000000}%
\pgfsetstrokecolor{currentstroke}%
\pgfsetdash{}{0pt}%
\pgfpathmoveto{\pgfqpoint{1.684565in}{1.122288in}}%
\pgfpathlineto{\pgfqpoint{1.721254in}{1.141212in}}%
\pgfpathlineto{\pgfqpoint{1.684207in}{1.172303in}}%
\pgfpathlineto{\pgfqpoint{1.647605in}{1.165710in}}%
\pgfpathclose%
\pgfusepath{fill}%
\end{pgfscope}%
\begin{pgfscope}%
\pgfpathrectangle{\pgfqpoint{0.150000in}{0.150000in}}{\pgfqpoint{2.700000in}{1.950000in}}%
\pgfusepath{clip}%
\pgfsetbuttcap%
\pgfsetroundjoin%
\definecolor{currentfill}{rgb}{0.887929,0.796645,0.803876}%
\pgfsetfillcolor{currentfill}%
\pgfsetlinewidth{0.000000pt}%
\definecolor{currentstroke}{rgb}{0.000000,0.000000,0.000000}%
\pgfsetstrokecolor{currentstroke}%
\pgfsetdash{}{0pt}%
\pgfpathmoveto{\pgfqpoint{1.721702in}{1.091091in}}%
\pgfpathlineto{\pgfqpoint{1.758090in}{1.097904in}}%
\pgfpathlineto{\pgfqpoint{1.721254in}{1.141212in}}%
\pgfpathlineto{\pgfqpoint{1.684565in}{1.122288in}}%
\pgfpathclose%
\pgfusepath{fill}%
\end{pgfscope}%
\begin{pgfscope}%
\pgfpathrectangle{\pgfqpoint{0.150000in}{0.150000in}}{\pgfqpoint{2.700000in}{1.950000in}}%
\pgfusepath{clip}%
\pgfsetbuttcap%
\pgfsetroundjoin%
\definecolor{currentfill}{rgb}{0.781556,0.603631,0.617724}%
\pgfsetfillcolor{currentfill}%
\pgfsetlinewidth{0.000000pt}%
\definecolor{currentstroke}{rgb}{0.000000,0.000000,0.000000}%
\pgfsetstrokecolor{currentstroke}%
\pgfsetdash{}{0pt}%
\pgfpathmoveto{\pgfqpoint{2.167141in}{0.879092in}}%
\pgfpathlineto{\pgfqpoint{2.202266in}{0.886542in}}%
\pgfpathlineto{\pgfqpoint{2.163427in}{0.893948in}}%
\pgfpathlineto{\pgfqpoint{2.128291in}{0.886542in}}%
\pgfpathclose%
\pgfusepath{fill}%
\end{pgfscope}%
\begin{pgfscope}%
\pgfpathrectangle{\pgfqpoint{0.150000in}{0.150000in}}{\pgfqpoint{2.700000in}{1.950000in}}%
\pgfusepath{clip}%
\pgfsetbuttcap%
\pgfsetroundjoin%
\definecolor{currentfill}{rgb}{0.834743,0.700138,0.710800}%
\pgfsetfillcolor{currentfill}%
\pgfsetlinewidth{0.000000pt}%
\definecolor{currentstroke}{rgb}{0.000000,0.000000,0.000000}%
\pgfsetstrokecolor{currentstroke}%
\pgfsetdash{}{0pt}%
\pgfpathmoveto{\pgfqpoint{1.833048in}{0.985169in}}%
\pgfpathlineto{\pgfqpoint{1.868748in}{0.980264in}}%
\pgfpathlineto{\pgfqpoint{1.831987in}{1.023491in}}%
\pgfpathlineto{\pgfqpoint{1.795814in}{1.016456in}}%
\pgfpathclose%
\pgfusepath{fill}%
\end{pgfscope}%
\begin{pgfscope}%
\pgfpathrectangle{\pgfqpoint{0.150000in}{0.150000in}}{\pgfqpoint{2.700000in}{1.950000in}}%
\pgfusepath{clip}%
\pgfsetbuttcap%
\pgfsetroundjoin%
\definecolor{currentfill}{rgb}{0.819547,0.672564,0.684206}%
\pgfsetfillcolor{currentfill}%
\pgfsetlinewidth{0.000000pt}%
\definecolor{currentstroke}{rgb}{0.000000,0.000000,0.000000}%
\pgfsetstrokecolor{currentstroke}%
\pgfsetdash{}{0pt}%
\pgfpathmoveto{\pgfqpoint{1.869941in}{0.941807in}}%
\pgfpathlineto{\pgfqpoint{1.905898in}{0.949063in}}%
\pgfpathlineto{\pgfqpoint{1.868748in}{0.980264in}}%
\pgfpathlineto{\pgfqpoint{1.833048in}{0.985169in}}%
\pgfpathclose%
\pgfusepath{fill}%
\end{pgfscope}%
\begin{pgfscope}%
\pgfpathrectangle{\pgfqpoint{0.150000in}{0.150000in}}{\pgfqpoint{2.700000in}{1.950000in}}%
\pgfusepath{clip}%
\pgfsetbuttcap%
\pgfsetroundjoin%
\definecolor{currentfill}{rgb}{0.800551,0.638097,0.650965}%
\pgfsetfillcolor{currentfill}%
\pgfsetlinewidth{0.000000pt}%
\definecolor{currentstroke}{rgb}{0.000000,0.000000,0.000000}%
\pgfsetstrokecolor{currentstroke}%
\pgfsetdash{}{0pt}%
\pgfpathmoveto{\pgfqpoint{1.907225in}{0.910470in}}%
\pgfpathlineto{\pgfqpoint{1.942624in}{0.905871in}}%
\pgfpathlineto{\pgfqpoint{1.905898in}{0.949063in}}%
\pgfpathlineto{\pgfqpoint{1.869941in}{0.941807in}}%
\pgfpathclose%
\pgfusepath{fill}%
\end{pgfscope}%
\begin{pgfscope}%
\pgfpathrectangle{\pgfqpoint{0.150000in}{0.150000in}}{\pgfqpoint{2.700000in}{1.950000in}}%
\pgfusepath{clip}%
\pgfsetbuttcap%
\pgfsetroundjoin%
\definecolor{currentfill}{rgb}{0.785355,0.610524,0.624372}%
\pgfsetfillcolor{currentfill}%
\pgfsetlinewidth{0.000000pt}%
\definecolor{currentstroke}{rgb}{0.000000,0.000000,0.000000}%
\pgfsetstrokecolor{currentstroke}%
\pgfsetdash{}{0pt}%
\pgfpathmoveto{\pgfqpoint{2.242158in}{0.891069in}}%
\pgfpathlineto{\pgfqpoint{2.277102in}{0.898492in}}%
\pgfpathlineto{\pgfqpoint{2.237184in}{0.893948in}}%
\pgfpathlineto{\pgfqpoint{2.202266in}{0.886542in}}%
\pgfpathclose%
\pgfusepath{fill}%
\end{pgfscope}%
\begin{pgfscope}%
\pgfpathrectangle{\pgfqpoint{0.150000in}{0.150000in}}{\pgfqpoint{2.700000in}{1.950000in}}%
\pgfusepath{clip}%
\pgfsetbuttcap%
\pgfsetroundjoin%
\definecolor{currentfill}{rgb}{0.948713,0.906939,0.910248}%
\pgfsetfillcolor{currentfill}%
\pgfsetlinewidth{0.000000pt}%
\definecolor{currentstroke}{rgb}{0.000000,0.000000,0.000000}%
\pgfsetstrokecolor{currentstroke}%
\pgfsetdash{}{0pt}%
\pgfpathmoveto{\pgfqpoint{1.536486in}{1.209178in}}%
\pgfpathlineto{\pgfqpoint{1.573479in}{1.227966in}}%
\pgfpathlineto{\pgfqpoint{1.536486in}{1.259035in}}%
\pgfpathlineto{\pgfqpoint{1.499451in}{1.240315in}}%
\pgfpathclose%
\pgfusepath{fill}%
\end{pgfscope}%
\begin{pgfscope}%
\pgfpathrectangle{\pgfqpoint{0.150000in}{0.150000in}}{\pgfqpoint{2.700000in}{1.950000in}}%
\pgfusepath{clip}%
\pgfsetbuttcap%
\pgfsetroundjoin%
\definecolor{currentfill}{rgb}{0.792953,0.624311,0.637669}%
\pgfsetfillcolor{currentfill}%
\pgfsetlinewidth{0.000000pt}%
\definecolor{currentstroke}{rgb}{0.000000,0.000000,0.000000}%
\pgfsetstrokecolor{currentstroke}%
\pgfsetdash{}{0pt}%
\pgfpathmoveto{\pgfqpoint{1.945033in}{0.891069in}}%
\pgfpathlineto{\pgfqpoint{1.980340in}{0.886542in}}%
\pgfpathlineto{\pgfqpoint{1.942624in}{0.905871in}}%
\pgfpathlineto{\pgfqpoint{1.907225in}{0.910470in}}%
\pgfpathclose%
\pgfusepath{fill}%
\end{pgfscope}%
\begin{pgfscope}%
\pgfpathrectangle{\pgfqpoint{0.150000in}{0.150000in}}{\pgfqpoint{2.700000in}{1.950000in}}%
\pgfusepath{clip}%
\pgfsetbuttcap%
\pgfsetroundjoin%
\definecolor{currentfill}{rgb}{0.971507,0.948300,0.950138}%
\pgfsetfillcolor{currentfill}%
\pgfsetlinewidth{0.000000pt}%
\definecolor{currentstroke}{rgb}{0.000000,0.000000,0.000000}%
\pgfsetstrokecolor{currentstroke}%
\pgfsetdash{}{0pt}%
\pgfpathmoveto{\pgfqpoint{1.053900in}{1.265098in}}%
\pgfpathlineto{\pgfqpoint{1.092338in}{1.271412in}}%
\pgfpathlineto{\pgfqpoint{1.056187in}{1.290065in}}%
\pgfpathlineto{\pgfqpoint{1.017710in}{1.283818in}}%
\pgfpathclose%
\pgfusepath{fill}%
\end{pgfscope}%
\begin{pgfscope}%
\pgfpathrectangle{\pgfqpoint{0.150000in}{0.150000in}}{\pgfqpoint{2.700000in}{1.950000in}}%
\pgfusepath{clip}%
\pgfsetbuttcap%
\pgfsetroundjoin%
\definecolor{currentfill}{rgb}{0.952512,0.913833,0.916896}%
\pgfsetfillcolor{currentfill}%
\pgfsetlinewidth{0.000000pt}%
\definecolor{currentstroke}{rgb}{0.000000,0.000000,0.000000}%
\pgfsetstrokecolor{currentstroke}%
\pgfsetdash{}{0pt}%
\pgfpathmoveto{\pgfqpoint{1.462282in}{1.221527in}}%
\pgfpathlineto{\pgfqpoint{1.499451in}{1.240315in}}%
\pgfpathlineto{\pgfqpoint{1.462548in}{1.259035in}}%
\pgfpathlineto{\pgfqpoint{1.425379in}{1.240315in}}%
\pgfpathclose%
\pgfusepath{fill}%
\end{pgfscope}%
\begin{pgfscope}%
\pgfpathrectangle{\pgfqpoint{0.150000in}{0.150000in}}{\pgfqpoint{2.700000in}{1.950000in}}%
\pgfusepath{clip}%
\pgfsetbuttcap%
\pgfsetroundjoin%
\definecolor{currentfill}{rgb}{0.952512,0.913833,0.916896}%
\pgfsetfillcolor{currentfill}%
\pgfsetlinewidth{0.000000pt}%
\definecolor{currentstroke}{rgb}{0.000000,0.000000,0.000000}%
\pgfsetstrokecolor{currentstroke}%
\pgfsetdash{}{0pt}%
\pgfpathmoveto{\pgfqpoint{1.388077in}{1.221527in}}%
\pgfpathlineto{\pgfqpoint{1.425379in}{1.240315in}}%
\pgfpathlineto{\pgfqpoint{1.388609in}{1.259035in}}%
\pgfpathlineto{\pgfqpoint{1.351308in}{1.240315in}}%
\pgfpathclose%
\pgfusepath{fill}%
\end{pgfscope}%
\begin{pgfscope}%
\pgfpathrectangle{\pgfqpoint{0.150000in}{0.150000in}}{\pgfqpoint{2.700000in}{1.950000in}}%
\pgfusepath{clip}%
\pgfsetbuttcap%
\pgfsetroundjoin%
\definecolor{currentfill}{rgb}{0.952512,0.913833,0.916896}%
\pgfsetfillcolor{currentfill}%
\pgfsetlinewidth{0.000000pt}%
\definecolor{currentstroke}{rgb}{0.000000,0.000000,0.000000}%
\pgfsetstrokecolor{currentstroke}%
\pgfsetdash{}{0pt}%
\pgfpathmoveto{\pgfqpoint{1.313873in}{1.221527in}}%
\pgfpathlineto{\pgfqpoint{1.351308in}{1.240315in}}%
\pgfpathlineto{\pgfqpoint{1.314670in}{1.259035in}}%
\pgfpathlineto{\pgfqpoint{1.277236in}{1.240315in}}%
\pgfpathclose%
\pgfusepath{fill}%
\end{pgfscope}%
\begin{pgfscope}%
\pgfpathrectangle{\pgfqpoint{0.150000in}{0.150000in}}{\pgfqpoint{2.700000in}{1.950000in}}%
\pgfusepath{clip}%
\pgfsetbuttcap%
\pgfsetroundjoin%
\definecolor{currentfill}{rgb}{0.952512,0.913833,0.916896}%
\pgfsetfillcolor{currentfill}%
\pgfsetlinewidth{0.000000pt}%
\definecolor{currentstroke}{rgb}{0.000000,0.000000,0.000000}%
\pgfsetstrokecolor{currentstroke}%
\pgfsetdash{}{0pt}%
\pgfpathmoveto{\pgfqpoint{1.239668in}{1.221527in}}%
\pgfpathlineto{\pgfqpoint{1.277236in}{1.240315in}}%
\pgfpathlineto{\pgfqpoint{1.240731in}{1.259035in}}%
\pgfpathlineto{\pgfqpoint{1.203165in}{1.240315in}}%
\pgfpathclose%
\pgfusepath{fill}%
\end{pgfscope}%
\begin{pgfscope}%
\pgfpathrectangle{\pgfqpoint{0.150000in}{0.150000in}}{\pgfqpoint{2.700000in}{1.950000in}}%
\pgfusepath{clip}%
\pgfsetbuttcap%
\pgfsetroundjoin%
\definecolor{currentfill}{rgb}{0.952512,0.913833,0.916896}%
\pgfsetfillcolor{currentfill}%
\pgfsetlinewidth{0.000000pt}%
\definecolor{currentstroke}{rgb}{0.000000,0.000000,0.000000}%
\pgfsetstrokecolor{currentstroke}%
\pgfsetdash{}{0pt}%
\pgfpathmoveto{\pgfqpoint{1.165464in}{1.221527in}}%
\pgfpathlineto{\pgfqpoint{1.203165in}{1.240315in}}%
\pgfpathlineto{\pgfqpoint{1.166793in}{1.259035in}}%
\pgfpathlineto{\pgfqpoint{1.129094in}{1.240315in}}%
\pgfpathclose%
\pgfusepath{fill}%
\end{pgfscope}%
\begin{pgfscope}%
\pgfpathrectangle{\pgfqpoint{0.150000in}{0.150000in}}{\pgfqpoint{2.700000in}{1.950000in}}%
\pgfusepath{clip}%
\pgfsetbuttcap%
\pgfsetroundjoin%
\definecolor{currentfill}{rgb}{0.937316,0.886259,0.890303}%
\pgfsetfillcolor{currentfill}%
\pgfsetlinewidth{0.000000pt}%
\definecolor{currentstroke}{rgb}{0.000000,0.000000,0.000000}%
\pgfsetstrokecolor{currentstroke}%
\pgfsetdash{}{0pt}%
\pgfpathmoveto{\pgfqpoint{1.573569in}{1.178002in}}%
\pgfpathlineto{\pgfqpoint{1.610519in}{1.196858in}}%
\pgfpathlineto{\pgfqpoint{1.573479in}{1.227966in}}%
\pgfpathlineto{\pgfqpoint{1.536486in}{1.209178in}}%
\pgfpathclose%
\pgfusepath{fill}%
\end{pgfscope}%
\begin{pgfscope}%
\pgfpathrectangle{\pgfqpoint{0.150000in}{0.150000in}}{\pgfqpoint{2.700000in}{1.950000in}}%
\pgfusepath{clip}%
\pgfsetbuttcap%
\pgfsetroundjoin%
\definecolor{currentfill}{rgb}{0.796752,0.631204,0.644317}%
\pgfsetfillcolor{currentfill}%
\pgfsetlinewidth{0.000000pt}%
\definecolor{currentstroke}{rgb}{0.000000,0.000000,0.000000}%
\pgfsetstrokecolor{currentstroke}%
\pgfsetdash{}{0pt}%
\pgfpathmoveto{\pgfqpoint{2.317350in}{0.903074in}}%
\pgfpathlineto{\pgfqpoint{2.352112in}{0.910470in}}%
\pgfpathlineto{\pgfqpoint{2.311840in}{0.905871in}}%
\pgfpathlineto{\pgfqpoint{2.277102in}{0.898492in}}%
\pgfpathclose%
\pgfusepath{fill}%
\end{pgfscope}%
\begin{pgfscope}%
\pgfpathrectangle{\pgfqpoint{0.150000in}{0.150000in}}{\pgfqpoint{2.700000in}{1.950000in}}%
\pgfusepath{clip}%
\pgfsetbuttcap%
\pgfsetroundjoin%
\definecolor{currentfill}{rgb}{0.925919,0.865579,0.870358}%
\pgfsetfillcolor{currentfill}%
\pgfsetlinewidth{0.000000pt}%
\definecolor{currentstroke}{rgb}{0.000000,0.000000,0.000000}%
\pgfsetstrokecolor{currentstroke}%
\pgfsetdash{}{0pt}%
\pgfpathmoveto{\pgfqpoint{1.610698in}{1.146786in}}%
\pgfpathlineto{\pgfqpoint{1.647605in}{1.165710in}}%
\pgfpathlineto{\pgfqpoint{1.610519in}{1.196858in}}%
\pgfpathlineto{\pgfqpoint{1.573569in}{1.178002in}}%
\pgfpathclose%
\pgfusepath{fill}%
\end{pgfscope}%
\begin{pgfscope}%
\pgfpathrectangle{\pgfqpoint{0.150000in}{0.150000in}}{\pgfqpoint{2.700000in}{1.950000in}}%
\pgfusepath{clip}%
\pgfsetbuttcap%
\pgfsetroundjoin%
\definecolor{currentfill}{rgb}{0.910723,0.838006,0.843765}%
\pgfsetfillcolor{currentfill}%
\pgfsetlinewidth{0.000000pt}%
\definecolor{currentstroke}{rgb}{0.000000,0.000000,0.000000}%
\pgfsetstrokecolor{currentstroke}%
\pgfsetdash{}{0pt}%
\pgfpathmoveto{\pgfqpoint{1.647875in}{1.115531in}}%
\pgfpathlineto{\pgfqpoint{1.684565in}{1.122288in}}%
\pgfpathlineto{\pgfqpoint{1.647605in}{1.165710in}}%
\pgfpathlineto{\pgfqpoint{1.610698in}{1.146786in}}%
\pgfpathclose%
\pgfusepath{fill}%
\end{pgfscope}%
\begin{pgfscope}%
\pgfpathrectangle{\pgfqpoint{0.150000in}{0.150000in}}{\pgfqpoint{2.700000in}{1.950000in}}%
\pgfusepath{clip}%
\pgfsetbuttcap%
\pgfsetroundjoin%
\definecolor{currentfill}{rgb}{0.895527,0.810432,0.817172}%
\pgfsetfillcolor{currentfill}%
\pgfsetlinewidth{0.000000pt}%
\definecolor{currentstroke}{rgb}{0.000000,0.000000,0.000000}%
\pgfsetstrokecolor{currentstroke}%
\pgfsetdash{}{0pt}%
\pgfpathmoveto{\pgfqpoint{1.685098in}{1.084237in}}%
\pgfpathlineto{\pgfqpoint{1.721702in}{1.091091in}}%
\pgfpathlineto{\pgfqpoint{1.684565in}{1.122288in}}%
\pgfpathlineto{\pgfqpoint{1.647875in}{1.115531in}}%
\pgfpathclose%
\pgfusepath{fill}%
\end{pgfscope}%
\begin{pgfscope}%
\pgfpathrectangle{\pgfqpoint{0.150000in}{0.150000in}}{\pgfqpoint{2.700000in}{1.950000in}}%
\pgfusepath{clip}%
\pgfsetbuttcap%
\pgfsetroundjoin%
\definecolor{currentfill}{rgb}{0.880331,0.782858,0.790579}%
\pgfsetfillcolor{currentfill}%
\pgfsetlinewidth{0.000000pt}%
\definecolor{currentstroke}{rgb}{0.000000,0.000000,0.000000}%
\pgfsetstrokecolor{currentstroke}%
\pgfsetdash{}{0pt}%
\pgfpathmoveto{\pgfqpoint{1.722152in}{1.040725in}}%
\pgfpathlineto{\pgfqpoint{1.758627in}{1.047704in}}%
\pgfpathlineto{\pgfqpoint{1.721702in}{1.091091in}}%
\pgfpathlineto{\pgfqpoint{1.685098in}{1.084237in}}%
\pgfpathclose%
\pgfusepath{fill}%
\end{pgfscope}%
\begin{pgfscope}%
\pgfpathrectangle{\pgfqpoint{0.150000in}{0.150000in}}{\pgfqpoint{2.700000in}{1.950000in}}%
\pgfusepath{clip}%
\pgfsetbuttcap%
\pgfsetroundjoin%
\definecolor{currentfill}{rgb}{0.967708,0.941406,0.943490}%
\pgfsetfillcolor{currentfill}%
\pgfsetlinewidth{0.000000pt}%
\definecolor{currentstroke}{rgb}{0.000000,0.000000,0.000000}%
\pgfsetstrokecolor{currentstroke}%
\pgfsetdash{}{0pt}%
\pgfpathmoveto{\pgfqpoint{1.090219in}{1.246311in}}%
\pgfpathlineto{\pgfqpoint{1.129094in}{1.240315in}}%
\pgfpathlineto{\pgfqpoint{1.092338in}{1.271412in}}%
\pgfpathlineto{\pgfqpoint{1.053900in}{1.265098in}}%
\pgfpathclose%
\pgfusepath{fill}%
\end{pgfscope}%
\begin{pgfscope}%
\pgfpathrectangle{\pgfqpoint{0.150000in}{0.150000in}}{\pgfqpoint{2.700000in}{1.950000in}}%
\pgfusepath{clip}%
\pgfsetbuttcap%
\pgfsetroundjoin%
\definecolor{currentfill}{rgb}{0.861336,0.748392,0.757338}%
\pgfsetfillcolor{currentfill}%
\pgfsetlinewidth{0.000000pt}%
\definecolor{currentstroke}{rgb}{0.000000,0.000000,0.000000}%
\pgfsetstrokecolor{currentstroke}%
\pgfsetdash{}{0pt}%
\pgfpathmoveto{\pgfqpoint{1.759427in}{1.009380in}}%
\pgfpathlineto{\pgfqpoint{1.795814in}{1.016456in}}%
\pgfpathlineto{\pgfqpoint{1.758627in}{1.047704in}}%
\pgfpathlineto{\pgfqpoint{1.722152in}{1.040725in}}%
\pgfpathclose%
\pgfusepath{fill}%
\end{pgfscope}%
\begin{pgfscope}%
\pgfpathrectangle{\pgfqpoint{0.150000in}{0.150000in}}{\pgfqpoint{2.700000in}{1.950000in}}%
\pgfusepath{clip}%
\pgfsetbuttcap%
\pgfsetroundjoin%
\definecolor{currentfill}{rgb}{0.952512,0.913833,0.916896}%
\pgfsetfillcolor{currentfill}%
\pgfsetlinewidth{0.000000pt}%
\definecolor{currentstroke}{rgb}{0.000000,0.000000,0.000000}%
\pgfsetstrokecolor{currentstroke}%
\pgfsetdash{}{0pt}%
\pgfpathmoveto{\pgfqpoint{1.499317in}{1.202671in}}%
\pgfpathlineto{\pgfqpoint{1.536486in}{1.209178in}}%
\pgfpathlineto{\pgfqpoint{1.499451in}{1.240315in}}%
\pgfpathlineto{\pgfqpoint{1.462282in}{1.221527in}}%
\pgfpathclose%
\pgfusepath{fill}%
\end{pgfscope}%
\begin{pgfscope}%
\pgfpathrectangle{\pgfqpoint{0.150000in}{0.150000in}}{\pgfqpoint{2.700000in}{1.950000in}}%
\pgfusepath{clip}%
\pgfsetbuttcap%
\pgfsetroundjoin%
\definecolor{currentfill}{rgb}{0.952512,0.913833,0.916896}%
\pgfsetfillcolor{currentfill}%
\pgfsetlinewidth{0.000000pt}%
\definecolor{currentstroke}{rgb}{0.000000,0.000000,0.000000}%
\pgfsetstrokecolor{currentstroke}%
\pgfsetdash{}{0pt}%
\pgfpathmoveto{\pgfqpoint{1.424979in}{1.202671in}}%
\pgfpathlineto{\pgfqpoint{1.462282in}{1.221527in}}%
\pgfpathlineto{\pgfqpoint{1.425379in}{1.240315in}}%
\pgfpathlineto{\pgfqpoint{1.388077in}{1.221527in}}%
\pgfpathclose%
\pgfusepath{fill}%
\end{pgfscope}%
\begin{pgfscope}%
\pgfpathrectangle{\pgfqpoint{0.150000in}{0.150000in}}{\pgfqpoint{2.700000in}{1.950000in}}%
\pgfusepath{clip}%
\pgfsetbuttcap%
\pgfsetroundjoin%
\definecolor{currentfill}{rgb}{0.952512,0.913833,0.916896}%
\pgfsetfillcolor{currentfill}%
\pgfsetlinewidth{0.000000pt}%
\definecolor{currentstroke}{rgb}{0.000000,0.000000,0.000000}%
\pgfsetstrokecolor{currentstroke}%
\pgfsetdash{}{0pt}%
\pgfpathmoveto{\pgfqpoint{1.350641in}{1.202671in}}%
\pgfpathlineto{\pgfqpoint{1.388077in}{1.221527in}}%
\pgfpathlineto{\pgfqpoint{1.351308in}{1.240315in}}%
\pgfpathlineto{\pgfqpoint{1.313873in}{1.221527in}}%
\pgfpathclose%
\pgfusepath{fill}%
\end{pgfscope}%
\begin{pgfscope}%
\pgfpathrectangle{\pgfqpoint{0.150000in}{0.150000in}}{\pgfqpoint{2.700000in}{1.950000in}}%
\pgfusepath{clip}%
\pgfsetbuttcap%
\pgfsetroundjoin%
\definecolor{currentfill}{rgb}{0.952512,0.913833,0.916896}%
\pgfsetfillcolor{currentfill}%
\pgfsetlinewidth{0.000000pt}%
\definecolor{currentstroke}{rgb}{0.000000,0.000000,0.000000}%
\pgfsetstrokecolor{currentstroke}%
\pgfsetdash{}{0pt}%
\pgfpathmoveto{\pgfqpoint{1.276303in}{1.202671in}}%
\pgfpathlineto{\pgfqpoint{1.313873in}{1.221527in}}%
\pgfpathlineto{\pgfqpoint{1.277236in}{1.240315in}}%
\pgfpathlineto{\pgfqpoint{1.239668in}{1.221527in}}%
\pgfpathclose%
\pgfusepath{fill}%
\end{pgfscope}%
\begin{pgfscope}%
\pgfpathrectangle{\pgfqpoint{0.150000in}{0.150000in}}{\pgfqpoint{2.700000in}{1.950000in}}%
\pgfusepath{clip}%
\pgfsetbuttcap%
\pgfsetroundjoin%
\definecolor{currentfill}{rgb}{0.952512,0.913833,0.916896}%
\pgfsetfillcolor{currentfill}%
\pgfsetlinewidth{0.000000pt}%
\definecolor{currentstroke}{rgb}{0.000000,0.000000,0.000000}%
\pgfsetstrokecolor{currentstroke}%
\pgfsetdash{}{0pt}%
\pgfpathmoveto{\pgfqpoint{1.201964in}{1.202671in}}%
\pgfpathlineto{\pgfqpoint{1.239668in}{1.221527in}}%
\pgfpathlineto{\pgfqpoint{1.203165in}{1.240315in}}%
\pgfpathlineto{\pgfqpoint{1.165464in}{1.221527in}}%
\pgfpathclose%
\pgfusepath{fill}%
\end{pgfscope}%
\begin{pgfscope}%
\pgfpathrectangle{\pgfqpoint{0.150000in}{0.150000in}}{\pgfqpoint{2.700000in}{1.950000in}}%
\pgfusepath{clip}%
\pgfsetbuttcap%
\pgfsetroundjoin%
\definecolor{currentfill}{rgb}{0.792953,0.624311,0.637669}%
\pgfsetfillcolor{currentfill}%
\pgfsetlinewidth{0.000000pt}%
\definecolor{currentstroke}{rgb}{0.000000,0.000000,0.000000}%
\pgfsetstrokecolor{currentstroke}%
\pgfsetdash{}{0pt}%
\pgfpathmoveto{\pgfqpoint{2.132504in}{0.883602in}}%
\pgfpathlineto{\pgfqpoint{2.167141in}{0.879092in}}%
\pgfpathlineto{\pgfqpoint{2.128291in}{0.886542in}}%
\pgfpathlineto{\pgfqpoint{2.092946in}{0.879092in}}%
\pgfpathclose%
\pgfusepath{fill}%
\end{pgfscope}%
\begin{pgfscope}%
\pgfpathrectangle{\pgfqpoint{0.150000in}{0.150000in}}{\pgfqpoint{2.700000in}{1.950000in}}%
\pgfusepath{clip}%
\pgfsetbuttcap%
\pgfsetroundjoin%
\definecolor{currentfill}{rgb}{0.792953,0.624311,0.637669}%
\pgfsetfillcolor{currentfill}%
\pgfsetlinewidth{0.000000pt}%
\definecolor{currentstroke}{rgb}{0.000000,0.000000,0.000000}%
\pgfsetstrokecolor{currentstroke}%
\pgfsetdash{}{0pt}%
\pgfpathmoveto{\pgfqpoint{2.058002in}{0.883602in}}%
\pgfpathlineto{\pgfqpoint{2.092946in}{0.879092in}}%
\pgfpathlineto{\pgfqpoint{2.054315in}{0.886542in}}%
\pgfpathlineto{\pgfqpoint{2.018752in}{0.879092in}}%
\pgfpathclose%
\pgfusepath{fill}%
\end{pgfscope}%
\begin{pgfscope}%
\pgfpathrectangle{\pgfqpoint{0.150000in}{0.150000in}}{\pgfqpoint{2.700000in}{1.950000in}}%
\pgfusepath{clip}%
\pgfsetbuttcap%
\pgfsetroundjoin%
\definecolor{currentfill}{rgb}{0.849939,0.727711,0.737393}%
\pgfsetfillcolor{currentfill}%
\pgfsetlinewidth{0.000000pt}%
\definecolor{currentstroke}{rgb}{0.000000,0.000000,0.000000}%
\pgfsetstrokecolor{currentstroke}%
\pgfsetdash{}{0pt}%
\pgfpathmoveto{\pgfqpoint{1.796748in}{0.977995in}}%
\pgfpathlineto{\pgfqpoint{1.833048in}{0.985169in}}%
\pgfpathlineto{\pgfqpoint{1.795814in}{1.016456in}}%
\pgfpathlineto{\pgfqpoint{1.759427in}{1.009380in}}%
\pgfpathclose%
\pgfusepath{fill}%
\end{pgfscope}%
\begin{pgfscope}%
\pgfpathrectangle{\pgfqpoint{0.150000in}{0.150000in}}{\pgfqpoint{2.700000in}{1.950000in}}%
\pgfusepath{clip}%
\pgfsetbuttcap%
\pgfsetroundjoin%
\definecolor{currentfill}{rgb}{0.941115,0.893153,0.896952}%
\pgfsetfillcolor{currentfill}%
\pgfsetlinewidth{0.000000pt}%
\definecolor{currentstroke}{rgb}{0.000000,0.000000,0.000000}%
\pgfsetstrokecolor{currentstroke}%
\pgfsetdash{}{0pt}%
\pgfpathmoveto{\pgfqpoint{1.536486in}{1.159078in}}%
\pgfpathlineto{\pgfqpoint{1.573569in}{1.178002in}}%
\pgfpathlineto{\pgfqpoint{1.536486in}{1.209178in}}%
\pgfpathlineto{\pgfqpoint{1.499317in}{1.202671in}}%
\pgfpathclose%
\pgfusepath{fill}%
\end{pgfscope}%
\begin{pgfscope}%
\pgfpathrectangle{\pgfqpoint{0.150000in}{0.150000in}}{\pgfqpoint{2.700000in}{1.950000in}}%
\pgfusepath{clip}%
\pgfsetbuttcap%
\pgfsetroundjoin%
\definecolor{currentfill}{rgb}{0.796752,0.631204,0.644317}%
\pgfsetfillcolor{currentfill}%
\pgfsetlinewidth{0.000000pt}%
\definecolor{currentstroke}{rgb}{0.000000,0.000000,0.000000}%
\pgfsetstrokecolor{currentstroke}%
\pgfsetdash{}{0pt}%
\pgfpathmoveto{\pgfqpoint{2.207006in}{0.883602in}}%
\pgfpathlineto{\pgfqpoint{2.242158in}{0.891069in}}%
\pgfpathlineto{\pgfqpoint{2.202266in}{0.886542in}}%
\pgfpathlineto{\pgfqpoint{2.167141in}{0.879092in}}%
\pgfpathclose%
\pgfusepath{fill}%
\end{pgfscope}%
\begin{pgfscope}%
\pgfpathrectangle{\pgfqpoint{0.150000in}{0.150000in}}{\pgfqpoint{2.700000in}{1.950000in}}%
\pgfusepath{clip}%
\pgfsetbuttcap%
\pgfsetroundjoin%
\definecolor{currentfill}{rgb}{0.796752,0.631204,0.644317}%
\pgfsetfillcolor{currentfill}%
\pgfsetlinewidth{0.000000pt}%
\definecolor{currentstroke}{rgb}{0.000000,0.000000,0.000000}%
\pgfsetstrokecolor{currentstroke}%
\pgfsetdash{}{0pt}%
\pgfpathmoveto{\pgfqpoint{1.983499in}{0.883602in}}%
\pgfpathlineto{\pgfqpoint{2.018752in}{0.879092in}}%
\pgfpathlineto{\pgfqpoint{1.980340in}{0.886542in}}%
\pgfpathlineto{\pgfqpoint{1.945033in}{0.891069in}}%
\pgfpathclose%
\pgfusepath{fill}%
\end{pgfscope}%
\begin{pgfscope}%
\pgfpathrectangle{\pgfqpoint{0.150000in}{0.150000in}}{\pgfqpoint{2.700000in}{1.950000in}}%
\pgfusepath{clip}%
\pgfsetbuttcap%
\pgfsetroundjoin%
\definecolor{currentfill}{rgb}{0.834743,0.700138,0.710800}%
\pgfsetfillcolor{currentfill}%
\pgfsetlinewidth{0.000000pt}%
\definecolor{currentstroke}{rgb}{0.000000,0.000000,0.000000}%
\pgfsetstrokecolor{currentstroke}%
\pgfsetdash{}{0pt}%
\pgfpathmoveto{\pgfqpoint{1.834117in}{0.946571in}}%
\pgfpathlineto{\pgfqpoint{1.869941in}{0.941807in}}%
\pgfpathlineto{\pgfqpoint{1.833048in}{0.985169in}}%
\pgfpathlineto{\pgfqpoint{1.796748in}{0.977995in}}%
\pgfpathclose%
\pgfusepath{fill}%
\end{pgfscope}%
\begin{pgfscope}%
\pgfpathrectangle{\pgfqpoint{0.150000in}{0.150000in}}{\pgfqpoint{2.700000in}{1.950000in}}%
\pgfusepath{clip}%
\pgfsetbuttcap%
\pgfsetroundjoin%
\definecolor{currentfill}{rgb}{0.925919,0.865579,0.870358}%
\pgfsetfillcolor{currentfill}%
\pgfsetlinewidth{0.000000pt}%
\definecolor{currentstroke}{rgb}{0.000000,0.000000,0.000000}%
\pgfsetstrokecolor{currentstroke}%
\pgfsetdash{}{0pt}%
\pgfpathmoveto{\pgfqpoint{1.573659in}{1.127794in}}%
\pgfpathlineto{\pgfqpoint{1.610698in}{1.146786in}}%
\pgfpathlineto{\pgfqpoint{1.573569in}{1.178002in}}%
\pgfpathlineto{\pgfqpoint{1.536486in}{1.159078in}}%
\pgfpathclose%
\pgfusepath{fill}%
\end{pgfscope}%
\begin{pgfscope}%
\pgfpathrectangle{\pgfqpoint{0.150000in}{0.150000in}}{\pgfqpoint{2.700000in}{1.950000in}}%
\pgfusepath{clip}%
\pgfsetbuttcap%
\pgfsetroundjoin%
\definecolor{currentfill}{rgb}{0.819547,0.672564,0.684206}%
\pgfsetfillcolor{currentfill}%
\pgfsetlinewidth{0.000000pt}%
\definecolor{currentstroke}{rgb}{0.000000,0.000000,0.000000}%
\pgfsetstrokecolor{currentstroke}%
\pgfsetdash{}{0pt}%
\pgfpathmoveto{\pgfqpoint{1.871142in}{0.903074in}}%
\pgfpathlineto{\pgfqpoint{1.907225in}{0.910470in}}%
\pgfpathlineto{\pgfqpoint{1.869941in}{0.941807in}}%
\pgfpathlineto{\pgfqpoint{1.834117in}{0.946571in}}%
\pgfpathclose%
\pgfusepath{fill}%
\end{pgfscope}%
\begin{pgfscope}%
\pgfpathrectangle{\pgfqpoint{0.150000in}{0.150000in}}{\pgfqpoint{2.700000in}{1.950000in}}%
\pgfusepath{clip}%
\pgfsetbuttcap%
\pgfsetroundjoin%
\definecolor{currentfill}{rgb}{0.972013,0.975460,0.980285}%
\pgfsetfillcolor{currentfill}%
\pgfsetlinewidth{0.000000pt}%
\definecolor{currentstroke}{rgb}{0.000000,0.000000,0.000000}%
\pgfsetstrokecolor{currentstroke}%
\pgfsetdash{}{0pt}%
\pgfpathmoveto{\pgfqpoint{0.939431in}{1.358863in}}%
\pgfpathlineto{\pgfqpoint{0.981006in}{1.314904in}}%
\pgfpathlineto{\pgfqpoint{0.945035in}{1.333489in}}%
\pgfpathlineto{\pgfqpoint{0.903260in}{1.377511in}}%
\pgfpathclose%
\pgfusepath{fill}%
\end{pgfscope}%
\begin{pgfscope}%
\pgfpathrectangle{\pgfqpoint{0.150000in}{0.150000in}}{\pgfqpoint{2.700000in}{1.950000in}}%
\pgfusepath{clip}%
\pgfsetbuttcap%
\pgfsetroundjoin%
\definecolor{currentfill}{rgb}{0.963909,0.934513,0.936841}%
\pgfsetfillcolor{currentfill}%
\pgfsetlinewidth{0.000000pt}%
\definecolor{currentstroke}{rgb}{0.000000,0.000000,0.000000}%
\pgfsetstrokecolor{currentstroke}%
\pgfsetdash{}{0pt}%
\pgfpathmoveto{\pgfqpoint{1.126670in}{1.227456in}}%
\pgfpathlineto{\pgfqpoint{1.165464in}{1.221527in}}%
\pgfpathlineto{\pgfqpoint{1.129094in}{1.240315in}}%
\pgfpathlineto{\pgfqpoint{1.090219in}{1.246311in}}%
\pgfpathclose%
\pgfusepath{fill}%
\end{pgfscope}%
\begin{pgfscope}%
\pgfpathrectangle{\pgfqpoint{0.150000in}{0.150000in}}{\pgfqpoint{2.700000in}{1.950000in}}%
\pgfusepath{clip}%
\pgfsetbuttcap%
\pgfsetroundjoin%
\definecolor{currentfill}{rgb}{0.914522,0.844899,0.850414}%
\pgfsetfillcolor{currentfill}%
\pgfsetlinewidth{0.000000pt}%
\definecolor{currentstroke}{rgb}{0.000000,0.000000,0.000000}%
\pgfsetstrokecolor{currentstroke}%
\pgfsetdash{}{0pt}%
\pgfpathmoveto{\pgfqpoint{1.610879in}{1.096471in}}%
\pgfpathlineto{\pgfqpoint{1.647875in}{1.115531in}}%
\pgfpathlineto{\pgfqpoint{1.610698in}{1.146786in}}%
\pgfpathlineto{\pgfqpoint{1.573659in}{1.127794in}}%
\pgfpathclose%
\pgfusepath{fill}%
\end{pgfscope}%
\begin{pgfscope}%
\pgfpathrectangle{\pgfqpoint{0.150000in}{0.150000in}}{\pgfqpoint{2.700000in}{1.950000in}}%
\pgfusepath{clip}%
\pgfsetbuttcap%
\pgfsetroundjoin%
\definecolor{currentfill}{rgb}{0.808150,0.651884,0.664262}%
\pgfsetfillcolor{currentfill}%
\pgfsetlinewidth{0.000000pt}%
\definecolor{currentstroke}{rgb}{0.000000,0.000000,0.000000}%
\pgfsetstrokecolor{currentstroke}%
\pgfsetdash{}{0pt}%
\pgfpathmoveto{\pgfqpoint{1.908997in}{0.883602in}}%
\pgfpathlineto{\pgfqpoint{1.945033in}{0.891069in}}%
\pgfpathlineto{\pgfqpoint{1.907225in}{0.910470in}}%
\pgfpathlineto{\pgfqpoint{1.871142in}{0.903074in}}%
\pgfpathclose%
\pgfusepath{fill}%
\end{pgfscope}%
\begin{pgfscope}%
\pgfpathrectangle{\pgfqpoint{0.150000in}{0.150000in}}{\pgfqpoint{2.700000in}{1.950000in}}%
\pgfusepath{clip}%
\pgfsetbuttcap%
\pgfsetroundjoin%
\definecolor{currentfill}{rgb}{0.808150,0.651884,0.664262}%
\pgfsetfillcolor{currentfill}%
\pgfsetlinewidth{0.000000pt}%
\definecolor{currentstroke}{rgb}{0.000000,0.000000,0.000000}%
\pgfsetstrokecolor{currentstroke}%
\pgfsetdash{}{0pt}%
\pgfpathmoveto{\pgfqpoint{2.282381in}{0.895633in}}%
\pgfpathlineto{\pgfqpoint{2.317350in}{0.903074in}}%
\pgfpathlineto{\pgfqpoint{2.277102in}{0.898492in}}%
\pgfpathlineto{\pgfqpoint{2.242158in}{0.891069in}}%
\pgfpathclose%
\pgfusepath{fill}%
\end{pgfscope}%
\begin{pgfscope}%
\pgfpathrectangle{\pgfqpoint{0.150000in}{0.150000in}}{\pgfqpoint{2.700000in}{1.950000in}}%
\pgfusepath{clip}%
\pgfsetbuttcap%
\pgfsetroundjoin%
\definecolor{currentfill}{rgb}{0.952512,0.913833,0.916896}%
\pgfsetfillcolor{currentfill}%
\pgfsetlinewidth{0.000000pt}%
\definecolor{currentstroke}{rgb}{0.000000,0.000000,0.000000}%
\pgfsetstrokecolor{currentstroke}%
\pgfsetdash{}{0pt}%
\pgfpathmoveto{\pgfqpoint{1.462014in}{1.183748in}}%
\pgfpathlineto{\pgfqpoint{1.499317in}{1.202671in}}%
\pgfpathlineto{\pgfqpoint{1.462282in}{1.221527in}}%
\pgfpathlineto{\pgfqpoint{1.424979in}{1.202671in}}%
\pgfpathclose%
\pgfusepath{fill}%
\end{pgfscope}%
\begin{pgfscope}%
\pgfpathrectangle{\pgfqpoint{0.150000in}{0.150000in}}{\pgfqpoint{2.700000in}{1.950000in}}%
\pgfusepath{clip}%
\pgfsetbuttcap%
\pgfsetroundjoin%
\definecolor{currentfill}{rgb}{0.952512,0.913833,0.916896}%
\pgfsetfillcolor{currentfill}%
\pgfsetlinewidth{0.000000pt}%
\definecolor{currentstroke}{rgb}{0.000000,0.000000,0.000000}%
\pgfsetstrokecolor{currentstroke}%
\pgfsetdash{}{0pt}%
\pgfpathmoveto{\pgfqpoint{1.387542in}{1.183748in}}%
\pgfpathlineto{\pgfqpoint{1.424979in}{1.202671in}}%
\pgfpathlineto{\pgfqpoint{1.388077in}{1.221527in}}%
\pgfpathlineto{\pgfqpoint{1.350641in}{1.202671in}}%
\pgfpathclose%
\pgfusepath{fill}%
\end{pgfscope}%
\begin{pgfscope}%
\pgfpathrectangle{\pgfqpoint{0.150000in}{0.150000in}}{\pgfqpoint{2.700000in}{1.950000in}}%
\pgfusepath{clip}%
\pgfsetbuttcap%
\pgfsetroundjoin%
\definecolor{currentfill}{rgb}{0.952512,0.913833,0.916896}%
\pgfsetfillcolor{currentfill}%
\pgfsetlinewidth{0.000000pt}%
\definecolor{currentstroke}{rgb}{0.000000,0.000000,0.000000}%
\pgfsetstrokecolor{currentstroke}%
\pgfsetdash{}{0pt}%
\pgfpathmoveto{\pgfqpoint{1.313069in}{1.183748in}}%
\pgfpathlineto{\pgfqpoint{1.350641in}{1.202671in}}%
\pgfpathlineto{\pgfqpoint{1.313873in}{1.221527in}}%
\pgfpathlineto{\pgfqpoint{1.276303in}{1.202671in}}%
\pgfpathclose%
\pgfusepath{fill}%
\end{pgfscope}%
\begin{pgfscope}%
\pgfpathrectangle{\pgfqpoint{0.150000in}{0.150000in}}{\pgfqpoint{2.700000in}{1.950000in}}%
\pgfusepath{clip}%
\pgfsetbuttcap%
\pgfsetroundjoin%
\definecolor{currentfill}{rgb}{0.952512,0.913833,0.916896}%
\pgfsetfillcolor{currentfill}%
\pgfsetlinewidth{0.000000pt}%
\definecolor{currentstroke}{rgb}{0.000000,0.000000,0.000000}%
\pgfsetstrokecolor{currentstroke}%
\pgfsetdash{}{0pt}%
\pgfpathmoveto{\pgfqpoint{1.238597in}{1.183748in}}%
\pgfpathlineto{\pgfqpoint{1.276303in}{1.202671in}}%
\pgfpathlineto{\pgfqpoint{1.239668in}{1.221527in}}%
\pgfpathlineto{\pgfqpoint{1.201964in}{1.202671in}}%
\pgfpathclose%
\pgfusepath{fill}%
\end{pgfscope}%
\begin{pgfscope}%
\pgfpathrectangle{\pgfqpoint{0.150000in}{0.150000in}}{\pgfqpoint{2.700000in}{1.950000in}}%
\pgfusepath{clip}%
\pgfsetbuttcap%
\pgfsetroundjoin%
\definecolor{currentfill}{rgb}{0.903125,0.824219,0.830469}%
\pgfsetfillcolor{currentfill}%
\pgfsetlinewidth{0.000000pt}%
\definecolor{currentstroke}{rgb}{0.000000,0.000000,0.000000}%
\pgfsetstrokecolor{currentstroke}%
\pgfsetdash{}{0pt}%
\pgfpathmoveto{\pgfqpoint{1.648146in}{1.065108in}}%
\pgfpathlineto{\pgfqpoint{1.685098in}{1.084237in}}%
\pgfpathlineto{\pgfqpoint{1.647875in}{1.115531in}}%
\pgfpathlineto{\pgfqpoint{1.610879in}{1.096471in}}%
\pgfpathclose%
\pgfusepath{fill}%
\end{pgfscope}%
\begin{pgfscope}%
\pgfpathrectangle{\pgfqpoint{0.150000in}{0.150000in}}{\pgfqpoint{2.700000in}{1.950000in}}%
\pgfusepath{clip}%
\pgfsetbuttcap%
\pgfsetroundjoin%
\definecolor{currentfill}{rgb}{0.978232,0.980913,0.984666}%
\pgfsetfillcolor{currentfill}%
\pgfsetlinewidth{0.000000pt}%
\definecolor{currentstroke}{rgb}{0.000000,0.000000,0.000000}%
\pgfsetstrokecolor{currentstroke}%
\pgfsetdash{}{0pt}%
\pgfpathmoveto{\pgfqpoint{0.976391in}{1.327566in}}%
\pgfpathlineto{\pgfqpoint{1.017710in}{1.283818in}}%
\pgfpathlineto{\pgfqpoint{0.981006in}{1.314904in}}%
\pgfpathlineto{\pgfqpoint{0.939431in}{1.358863in}}%
\pgfpathclose%
\pgfusepath{fill}%
\end{pgfscope}%
\begin{pgfscope}%
\pgfpathrectangle{\pgfqpoint{0.150000in}{0.150000in}}{\pgfqpoint{2.700000in}{1.950000in}}%
\pgfusepath{clip}%
\pgfsetbuttcap%
\pgfsetroundjoin%
\definecolor{currentfill}{rgb}{0.887929,0.796645,0.803876}%
\pgfsetfillcolor{currentfill}%
\pgfsetlinewidth{0.000000pt}%
\definecolor{currentstroke}{rgb}{0.000000,0.000000,0.000000}%
\pgfsetstrokecolor{currentstroke}%
\pgfsetdash{}{0pt}%
\pgfpathmoveto{\pgfqpoint{1.685461in}{1.033705in}}%
\pgfpathlineto{\pgfqpoint{1.722152in}{1.040725in}}%
\pgfpathlineto{\pgfqpoint{1.685098in}{1.084237in}}%
\pgfpathlineto{\pgfqpoint{1.648146in}{1.065108in}}%
\pgfpathclose%
\pgfusepath{fill}%
\end{pgfscope}%
\begin{pgfscope}%
\pgfpathrectangle{\pgfqpoint{0.150000in}{0.150000in}}{\pgfqpoint{2.700000in}{1.950000in}}%
\pgfusepath{clip}%
\pgfsetbuttcap%
\pgfsetroundjoin%
\definecolor{currentfill}{rgb}{0.872733,0.769072,0.777282}%
\pgfsetfillcolor{currentfill}%
\pgfsetlinewidth{0.000000pt}%
\definecolor{currentstroke}{rgb}{0.000000,0.000000,0.000000}%
\pgfsetstrokecolor{currentstroke}%
\pgfsetdash{}{0pt}%
\pgfpathmoveto{\pgfqpoint{1.722823in}{1.002262in}}%
\pgfpathlineto{\pgfqpoint{1.759427in}{1.009380in}}%
\pgfpathlineto{\pgfqpoint{1.722152in}{1.040725in}}%
\pgfpathlineto{\pgfqpoint{1.685461in}{1.033705in}}%
\pgfpathclose%
\pgfusepath{fill}%
\end{pgfscope}%
\begin{pgfscope}%
\pgfpathrectangle{\pgfqpoint{0.150000in}{0.150000in}}{\pgfqpoint{2.700000in}{1.950000in}}%
\pgfusepath{clip}%
\pgfsetbuttcap%
\pgfsetroundjoin%
\definecolor{currentfill}{rgb}{0.944914,0.900046,0.903600}%
\pgfsetfillcolor{currentfill}%
\pgfsetlinewidth{0.000000pt}%
\definecolor{currentstroke}{rgb}{0.000000,0.000000,0.000000}%
\pgfsetstrokecolor{currentstroke}%
\pgfsetdash{}{0pt}%
\pgfpathmoveto{\pgfqpoint{1.499227in}{1.152407in}}%
\pgfpathlineto{\pgfqpoint{1.536486in}{1.159078in}}%
\pgfpathlineto{\pgfqpoint{1.499317in}{1.202671in}}%
\pgfpathlineto{\pgfqpoint{1.462014in}{1.183748in}}%
\pgfpathclose%
\pgfusepath{fill}%
\end{pgfscope}%
\begin{pgfscope}%
\pgfpathrectangle{\pgfqpoint{0.150000in}{0.150000in}}{\pgfqpoint{2.700000in}{1.950000in}}%
\pgfusepath{clip}%
\pgfsetbuttcap%
\pgfsetroundjoin%
\definecolor{currentfill}{rgb}{0.963909,0.934513,0.936841}%
\pgfsetfillcolor{currentfill}%
\pgfsetlinewidth{0.000000pt}%
\definecolor{currentstroke}{rgb}{0.000000,0.000000,0.000000}%
\pgfsetstrokecolor{currentstroke}%
\pgfsetdash{}{0pt}%
\pgfpathmoveto{\pgfqpoint{1.163252in}{1.208533in}}%
\pgfpathlineto{\pgfqpoint{1.201964in}{1.202671in}}%
\pgfpathlineto{\pgfqpoint{1.165464in}{1.221527in}}%
\pgfpathlineto{\pgfqpoint{1.126670in}{1.227456in}}%
\pgfpathclose%
\pgfusepath{fill}%
\end{pgfscope}%
\begin{pgfscope}%
\pgfpathrectangle{\pgfqpoint{0.150000in}{0.150000in}}{\pgfqpoint{2.700000in}{1.950000in}}%
\pgfusepath{clip}%
\pgfsetbuttcap%
\pgfsetroundjoin%
\definecolor{currentfill}{rgb}{0.929718,0.872472,0.877007}%
\pgfsetfillcolor{currentfill}%
\pgfsetlinewidth{0.000000pt}%
\definecolor{currentstroke}{rgb}{0.000000,0.000000,0.000000}%
\pgfsetstrokecolor{currentstroke}%
\pgfsetdash{}{0pt}%
\pgfpathmoveto{\pgfqpoint{1.536486in}{1.121025in}}%
\pgfpathlineto{\pgfqpoint{1.573659in}{1.127794in}}%
\pgfpathlineto{\pgfqpoint{1.536486in}{1.159078in}}%
\pgfpathlineto{\pgfqpoint{1.499227in}{1.152407in}}%
\pgfpathclose%
\pgfusepath{fill}%
\end{pgfscope}%
\begin{pgfscope}%
\pgfpathrectangle{\pgfqpoint{0.150000in}{0.150000in}}{\pgfqpoint{2.700000in}{1.950000in}}%
\pgfusepath{clip}%
\pgfsetbuttcap%
\pgfsetroundjoin%
\definecolor{currentfill}{rgb}{0.861336,0.748392,0.757338}%
\pgfsetfillcolor{currentfill}%
\pgfsetlinewidth{0.000000pt}%
\definecolor{currentstroke}{rgb}{0.000000,0.000000,0.000000}%
\pgfsetstrokecolor{currentstroke}%
\pgfsetdash{}{0pt}%
\pgfpathmoveto{\pgfqpoint{1.760232in}{0.970779in}}%
\pgfpathlineto{\pgfqpoint{1.796748in}{0.977995in}}%
\pgfpathlineto{\pgfqpoint{1.759427in}{1.009380in}}%
\pgfpathlineto{\pgfqpoint{1.722823in}{1.002262in}}%
\pgfpathclose%
\pgfusepath{fill}%
\end{pgfscope}%
\begin{pgfscope}%
\pgfpathrectangle{\pgfqpoint{0.150000in}{0.150000in}}{\pgfqpoint{2.700000in}{1.950000in}}%
\pgfusepath{clip}%
\pgfsetbuttcap%
\pgfsetroundjoin%
\definecolor{currentfill}{rgb}{0.990671,0.991820,0.993428}%
\pgfsetfillcolor{currentfill}%
\pgfsetlinewidth{0.000000pt}%
\definecolor{currentstroke}{rgb}{0.000000,0.000000,0.000000}%
\pgfsetstrokecolor{currentstroke}%
\pgfsetdash{}{0pt}%
\pgfpathmoveto{\pgfqpoint{1.012783in}{1.308782in}}%
\pgfpathlineto{\pgfqpoint{1.053900in}{1.265098in}}%
\pgfpathlineto{\pgfqpoint{1.017710in}{1.283818in}}%
\pgfpathlineto{\pgfqpoint{0.976391in}{1.327566in}}%
\pgfpathclose%
\pgfusepath{fill}%
\end{pgfscope}%
\begin{pgfscope}%
\pgfpathrectangle{\pgfqpoint{0.150000in}{0.150000in}}{\pgfqpoint{2.700000in}{1.950000in}}%
\pgfusepath{clip}%
\pgfsetbuttcap%
\pgfsetroundjoin%
\definecolor{currentfill}{rgb}{0.952512,0.913833,0.916896}%
\pgfsetfillcolor{currentfill}%
\pgfsetlinewidth{0.000000pt}%
\definecolor{currentstroke}{rgb}{0.000000,0.000000,0.000000}%
\pgfsetstrokecolor{currentstroke}%
\pgfsetdash{}{0pt}%
\pgfpathmoveto{\pgfqpoint{1.424576in}{1.164756in}}%
\pgfpathlineto{\pgfqpoint{1.462014in}{1.183748in}}%
\pgfpathlineto{\pgfqpoint{1.424979in}{1.202671in}}%
\pgfpathlineto{\pgfqpoint{1.387542in}{1.183748in}}%
\pgfpathclose%
\pgfusepath{fill}%
\end{pgfscope}%
\begin{pgfscope}%
\pgfpathrectangle{\pgfqpoint{0.150000in}{0.150000in}}{\pgfqpoint{2.700000in}{1.950000in}}%
\pgfusepath{clip}%
\pgfsetbuttcap%
\pgfsetroundjoin%
\definecolor{currentfill}{rgb}{0.952512,0.913833,0.916896}%
\pgfsetfillcolor{currentfill}%
\pgfsetlinewidth{0.000000pt}%
\definecolor{currentstroke}{rgb}{0.000000,0.000000,0.000000}%
\pgfsetstrokecolor{currentstroke}%
\pgfsetdash{}{0pt}%
\pgfpathmoveto{\pgfqpoint{1.349969in}{1.164756in}}%
\pgfpathlineto{\pgfqpoint{1.387542in}{1.183748in}}%
\pgfpathlineto{\pgfqpoint{1.350641in}{1.202671in}}%
\pgfpathlineto{\pgfqpoint{1.313069in}{1.183748in}}%
\pgfpathclose%
\pgfusepath{fill}%
\end{pgfscope}%
\begin{pgfscope}%
\pgfpathrectangle{\pgfqpoint{0.150000in}{0.150000in}}{\pgfqpoint{2.700000in}{1.950000in}}%
\pgfusepath{clip}%
\pgfsetbuttcap%
\pgfsetroundjoin%
\definecolor{currentfill}{rgb}{0.952512,0.913833,0.916896}%
\pgfsetfillcolor{currentfill}%
\pgfsetlinewidth{0.000000pt}%
\definecolor{currentstroke}{rgb}{0.000000,0.000000,0.000000}%
\pgfsetstrokecolor{currentstroke}%
\pgfsetdash{}{0pt}%
\pgfpathmoveto{\pgfqpoint{1.275362in}{1.164756in}}%
\pgfpathlineto{\pgfqpoint{1.313069in}{1.183748in}}%
\pgfpathlineto{\pgfqpoint{1.276303in}{1.202671in}}%
\pgfpathlineto{\pgfqpoint{1.238597in}{1.183748in}}%
\pgfpathclose%
\pgfusepath{fill}%
\end{pgfscope}%
\begin{pgfscope}%
\pgfpathrectangle{\pgfqpoint{0.150000in}{0.150000in}}{\pgfqpoint{2.700000in}{1.950000in}}%
\pgfusepath{clip}%
\pgfsetbuttcap%
\pgfsetroundjoin%
\definecolor{currentfill}{rgb}{0.918321,0.851792,0.857062}%
\pgfsetfillcolor{currentfill}%
\pgfsetlinewidth{0.000000pt}%
\definecolor{currentstroke}{rgb}{0.000000,0.000000,0.000000}%
\pgfsetstrokecolor{currentstroke}%
\pgfsetdash{}{0pt}%
\pgfpathmoveto{\pgfqpoint{1.573794in}{1.089604in}}%
\pgfpathlineto{\pgfqpoint{1.610879in}{1.096471in}}%
\pgfpathlineto{\pgfqpoint{1.573659in}{1.127794in}}%
\pgfpathlineto{\pgfqpoint{1.536486in}{1.121025in}}%
\pgfpathclose%
\pgfusepath{fill}%
\end{pgfscope}%
\begin{pgfscope}%
\pgfpathrectangle{\pgfqpoint{0.150000in}{0.150000in}}{\pgfqpoint{2.700000in}{1.950000in}}%
\pgfusepath{clip}%
\pgfsetbuttcap%
\pgfsetroundjoin%
\definecolor{currentfill}{rgb}{0.811949,0.658778,0.670910}%
\pgfsetfillcolor{currentfill}%
\pgfsetlinewidth{0.000000pt}%
\definecolor{currentstroke}{rgb}{0.000000,0.000000,0.000000}%
\pgfsetstrokecolor{currentstroke}%
\pgfsetdash{}{0pt}%
\pgfpathmoveto{\pgfqpoint{2.171644in}{0.876090in}}%
\pgfpathlineto{\pgfqpoint{2.207006in}{0.883602in}}%
\pgfpathlineto{\pgfqpoint{2.167141in}{0.879092in}}%
\pgfpathlineto{\pgfqpoint{2.132504in}{0.883602in}}%
\pgfpathclose%
\pgfusepath{fill}%
\end{pgfscope}%
\begin{pgfscope}%
\pgfpathrectangle{\pgfqpoint{0.150000in}{0.150000in}}{\pgfqpoint{2.700000in}{1.950000in}}%
\pgfusepath{clip}%
\pgfsetbuttcap%
\pgfsetroundjoin%
\definecolor{currentfill}{rgb}{0.811949,0.658778,0.670910}%
\pgfsetfillcolor{currentfill}%
\pgfsetlinewidth{0.000000pt}%
\definecolor{currentstroke}{rgb}{0.000000,0.000000,0.000000}%
\pgfsetstrokecolor{currentstroke}%
\pgfsetdash{}{0pt}%
\pgfpathmoveto{\pgfqpoint{2.096920in}{0.876090in}}%
\pgfpathlineto{\pgfqpoint{2.132504in}{0.883602in}}%
\pgfpathlineto{\pgfqpoint{2.092946in}{0.879092in}}%
\pgfpathlineto{\pgfqpoint{2.058002in}{0.883602in}}%
\pgfpathclose%
\pgfusepath{fill}%
\end{pgfscope}%
\begin{pgfscope}%
\pgfpathrectangle{\pgfqpoint{0.150000in}{0.150000in}}{\pgfqpoint{2.700000in}{1.950000in}}%
\pgfusepath{clip}%
\pgfsetbuttcap%
\pgfsetroundjoin%
\definecolor{currentfill}{rgb}{0.811949,0.658778,0.670910}%
\pgfsetfillcolor{currentfill}%
\pgfsetlinewidth{0.000000pt}%
\definecolor{currentstroke}{rgb}{0.000000,0.000000,0.000000}%
\pgfsetstrokecolor{currentstroke}%
\pgfsetdash{}{0pt}%
\pgfpathmoveto{\pgfqpoint{2.022195in}{0.876090in}}%
\pgfpathlineto{\pgfqpoint{2.058002in}{0.883602in}}%
\pgfpathlineto{\pgfqpoint{2.018752in}{0.879092in}}%
\pgfpathlineto{\pgfqpoint{1.983499in}{0.883602in}}%
\pgfpathclose%
\pgfusepath{fill}%
\end{pgfscope}%
\begin{pgfscope}%
\pgfpathrectangle{\pgfqpoint{0.150000in}{0.150000in}}{\pgfqpoint{2.700000in}{1.950000in}}%
\pgfusepath{clip}%
\pgfsetbuttcap%
\pgfsetroundjoin%
\definecolor{currentfill}{rgb}{0.849939,0.727711,0.737393}%
\pgfsetfillcolor{currentfill}%
\pgfsetlinewidth{0.000000pt}%
\definecolor{currentstroke}{rgb}{0.000000,0.000000,0.000000}%
\pgfsetstrokecolor{currentstroke}%
\pgfsetdash{}{0pt}%
\pgfpathmoveto{\pgfqpoint{1.797689in}{0.939256in}}%
\pgfpathlineto{\pgfqpoint{1.834117in}{0.946571in}}%
\pgfpathlineto{\pgfqpoint{1.796748in}{0.977995in}}%
\pgfpathlineto{\pgfqpoint{1.760232in}{0.970779in}}%
\pgfpathclose%
\pgfusepath{fill}%
\end{pgfscope}%
\begin{pgfscope}%
\pgfpathrectangle{\pgfqpoint{0.150000in}{0.150000in}}{\pgfqpoint{2.700000in}{1.950000in}}%
\pgfusepath{clip}%
\pgfsetbuttcap%
\pgfsetroundjoin%
\definecolor{currentfill}{rgb}{0.815748,0.665671,0.677558}%
\pgfsetfillcolor{currentfill}%
\pgfsetlinewidth{0.000000pt}%
\definecolor{currentstroke}{rgb}{0.000000,0.000000,0.000000}%
\pgfsetstrokecolor{currentstroke}%
\pgfsetdash{}{0pt}%
\pgfpathmoveto{\pgfqpoint{1.947471in}{0.876090in}}%
\pgfpathlineto{\pgfqpoint{1.983499in}{0.883602in}}%
\pgfpathlineto{\pgfqpoint{1.945033in}{0.891069in}}%
\pgfpathlineto{\pgfqpoint{1.908997in}{0.883602in}}%
\pgfpathclose%
\pgfusepath{fill}%
\end{pgfscope}%
\begin{pgfscope}%
\pgfpathrectangle{\pgfqpoint{0.150000in}{0.150000in}}{\pgfqpoint{2.700000in}{1.950000in}}%
\pgfusepath{clip}%
\pgfsetbuttcap%
\pgfsetroundjoin%
\definecolor{currentfill}{rgb}{0.834743,0.700138,0.710800}%
\pgfsetfillcolor{currentfill}%
\pgfsetlinewidth{0.000000pt}%
\definecolor{currentstroke}{rgb}{0.000000,0.000000,0.000000}%
\pgfsetstrokecolor{currentstroke}%
\pgfsetdash{}{0pt}%
\pgfpathmoveto{\pgfqpoint{1.835194in}{0.907693in}}%
\pgfpathlineto{\pgfqpoint{1.871142in}{0.903074in}}%
\pgfpathlineto{\pgfqpoint{1.834117in}{0.946571in}}%
\pgfpathlineto{\pgfqpoint{1.797689in}{0.939256in}}%
\pgfpathclose%
\pgfusepath{fill}%
\end{pgfscope}%
\begin{pgfscope}%
\pgfpathrectangle{\pgfqpoint{0.150000in}{0.150000in}}{\pgfqpoint{2.700000in}{1.950000in}}%
\pgfusepath{clip}%
\pgfsetbuttcap%
\pgfsetroundjoin%
\definecolor{currentfill}{rgb}{0.906924,0.831112,0.837117}%
\pgfsetfillcolor{currentfill}%
\pgfsetlinewidth{0.000000pt}%
\definecolor{currentstroke}{rgb}{0.000000,0.000000,0.000000}%
\pgfsetstrokecolor{currentstroke}%
\pgfsetdash{}{0pt}%
\pgfpathmoveto{\pgfqpoint{1.611148in}{1.058143in}}%
\pgfpathlineto{\pgfqpoint{1.648146in}{1.065108in}}%
\pgfpathlineto{\pgfqpoint{1.610879in}{1.096471in}}%
\pgfpathlineto{\pgfqpoint{1.573794in}{1.089604in}}%
\pgfpathclose%
\pgfusepath{fill}%
\end{pgfscope}%
\begin{pgfscope}%
\pgfpathrectangle{\pgfqpoint{0.150000in}{0.150000in}}{\pgfqpoint{2.700000in}{1.950000in}}%
\pgfusepath{clip}%
\pgfsetbuttcap%
\pgfsetroundjoin%
\definecolor{currentfill}{rgb}{0.819547,0.672564,0.684206}%
\pgfsetfillcolor{currentfill}%
\pgfsetlinewidth{0.000000pt}%
\definecolor{currentstroke}{rgb}{0.000000,0.000000,0.000000}%
\pgfsetstrokecolor{currentstroke}%
\pgfsetdash{}{0pt}%
\pgfpathmoveto{\pgfqpoint{2.247203in}{0.888149in}}%
\pgfpathlineto{\pgfqpoint{2.282381in}{0.895633in}}%
\pgfpathlineto{\pgfqpoint{2.242158in}{0.891069in}}%
\pgfpathlineto{\pgfqpoint{2.207006in}{0.883602in}}%
\pgfpathclose%
\pgfusepath{fill}%
\end{pgfscope}%
\begin{pgfscope}%
\pgfpathrectangle{\pgfqpoint{0.150000in}{0.150000in}}{\pgfqpoint{2.700000in}{1.950000in}}%
\pgfusepath{clip}%
\pgfsetbuttcap%
\pgfsetroundjoin%
\definecolor{currentfill}{rgb}{0.963909,0.934513,0.936841}%
\pgfsetfillcolor{currentfill}%
\pgfsetlinewidth{0.000000pt}%
\definecolor{currentstroke}{rgb}{0.000000,0.000000,0.000000}%
\pgfsetstrokecolor{currentstroke}%
\pgfsetdash{}{0pt}%
\pgfpathmoveto{\pgfqpoint{1.199967in}{1.189542in}}%
\pgfpathlineto{\pgfqpoint{1.238597in}{1.183748in}}%
\pgfpathlineto{\pgfqpoint{1.201964in}{1.202671in}}%
\pgfpathlineto{\pgfqpoint{1.163252in}{1.208533in}}%
\pgfpathclose%
\pgfusepath{fill}%
\end{pgfscope}%
\begin{pgfscope}%
\pgfpathrectangle{\pgfqpoint{0.150000in}{0.150000in}}{\pgfqpoint{2.700000in}{1.950000in}}%
\pgfusepath{clip}%
\pgfsetbuttcap%
\pgfsetroundjoin%
\definecolor{currentfill}{rgb}{0.948713,0.906939,0.910248}%
\pgfsetfillcolor{currentfill}%
\pgfsetlinewidth{0.000000pt}%
\definecolor{currentstroke}{rgb}{0.000000,0.000000,0.000000}%
\pgfsetstrokecolor{currentstroke}%
\pgfsetdash{}{0pt}%
\pgfpathmoveto{\pgfqpoint{1.461832in}{1.133346in}}%
\pgfpathlineto{\pgfqpoint{1.499227in}{1.152407in}}%
\pgfpathlineto{\pgfqpoint{1.462014in}{1.183748in}}%
\pgfpathlineto{\pgfqpoint{1.424576in}{1.164756in}}%
\pgfpathclose%
\pgfusepath{fill}%
\end{pgfscope}%
\begin{pgfscope}%
\pgfpathrectangle{\pgfqpoint{0.150000in}{0.150000in}}{\pgfqpoint{2.700000in}{1.950000in}}%
\pgfusepath{clip}%
\pgfsetbuttcap%
\pgfsetroundjoin%
\definecolor{currentfill}{rgb}{0.895527,0.810432,0.817172}%
\pgfsetfillcolor{currentfill}%
\pgfsetlinewidth{0.000000pt}%
\definecolor{currentstroke}{rgb}{0.000000,0.000000,0.000000}%
\pgfsetstrokecolor{currentstroke}%
\pgfsetdash{}{0pt}%
\pgfpathmoveto{\pgfqpoint{1.648551in}{1.026642in}}%
\pgfpathlineto{\pgfqpoint{1.685461in}{1.033705in}}%
\pgfpathlineto{\pgfqpoint{1.648146in}{1.065108in}}%
\pgfpathlineto{\pgfqpoint{1.611148in}{1.058143in}}%
\pgfpathclose%
\pgfusepath{fill}%
\end{pgfscope}%
\begin{pgfscope}%
\pgfpathrectangle{\pgfqpoint{0.150000in}{0.150000in}}{\pgfqpoint{2.700000in}{1.950000in}}%
\pgfusepath{clip}%
\pgfsetbuttcap%
\pgfsetroundjoin%
\definecolor{currentfill}{rgb}{0.996890,0.997273,0.997809}%
\pgfsetfillcolor{currentfill}%
\pgfsetlinewidth{0.000000pt}%
\definecolor{currentstroke}{rgb}{0.000000,0.000000,0.000000}%
\pgfsetstrokecolor{currentstroke}%
\pgfsetdash{}{0pt}%
\pgfpathmoveto{\pgfqpoint{1.049880in}{1.277376in}}%
\pgfpathlineto{\pgfqpoint{1.090219in}{1.246311in}}%
\pgfpathlineto{\pgfqpoint{1.053900in}{1.265098in}}%
\pgfpathlineto{\pgfqpoint{1.012783in}{1.308782in}}%
\pgfpathclose%
\pgfusepath{fill}%
\end{pgfscope}%
\begin{pgfscope}%
\pgfpathrectangle{\pgfqpoint{0.150000in}{0.150000in}}{\pgfqpoint{2.700000in}{1.950000in}}%
\pgfusepath{clip}%
\pgfsetbuttcap%
\pgfsetroundjoin%
\definecolor{currentfill}{rgb}{0.827145,0.686351,0.697503}%
\pgfsetfillcolor{currentfill}%
\pgfsetlinewidth{0.000000pt}%
\definecolor{currentstroke}{rgb}{0.000000,0.000000,0.000000}%
\pgfsetstrokecolor{currentstroke}%
\pgfsetdash{}{0pt}%
\pgfpathmoveto{\pgfqpoint{1.873141in}{0.888149in}}%
\pgfpathlineto{\pgfqpoint{1.908997in}{0.883602in}}%
\pgfpathlineto{\pgfqpoint{1.871142in}{0.903074in}}%
\pgfpathlineto{\pgfqpoint{1.835194in}{0.907693in}}%
\pgfpathclose%
\pgfusepath{fill}%
\end{pgfscope}%
\begin{pgfscope}%
\pgfpathrectangle{\pgfqpoint{0.150000in}{0.150000in}}{\pgfqpoint{2.700000in}{1.950000in}}%
\pgfusepath{clip}%
\pgfsetbuttcap%
\pgfsetroundjoin%
\definecolor{currentfill}{rgb}{0.952512,0.913833,0.916896}%
\pgfsetfillcolor{currentfill}%
\pgfsetlinewidth{0.000000pt}%
\definecolor{currentstroke}{rgb}{0.000000,0.000000,0.000000}%
\pgfsetstrokecolor{currentstroke}%
\pgfsetdash{}{0pt}%
\pgfpathmoveto{\pgfqpoint{1.387002in}{1.145695in}}%
\pgfpathlineto{\pgfqpoint{1.424576in}{1.164756in}}%
\pgfpathlineto{\pgfqpoint{1.387542in}{1.183748in}}%
\pgfpathlineto{\pgfqpoint{1.349969in}{1.164756in}}%
\pgfpathclose%
\pgfusepath{fill}%
\end{pgfscope}%
\begin{pgfscope}%
\pgfpathrectangle{\pgfqpoint{0.150000in}{0.150000in}}{\pgfqpoint{2.700000in}{1.950000in}}%
\pgfusepath{clip}%
\pgfsetbuttcap%
\pgfsetroundjoin%
\definecolor{currentfill}{rgb}{0.952512,0.913833,0.916896}%
\pgfsetfillcolor{currentfill}%
\pgfsetlinewidth{0.000000pt}%
\definecolor{currentstroke}{rgb}{0.000000,0.000000,0.000000}%
\pgfsetstrokecolor{currentstroke}%
\pgfsetdash{}{0pt}%
\pgfpathmoveto{\pgfqpoint{1.312260in}{1.145695in}}%
\pgfpathlineto{\pgfqpoint{1.349969in}{1.164756in}}%
\pgfpathlineto{\pgfqpoint{1.313069in}{1.183748in}}%
\pgfpathlineto{\pgfqpoint{1.275362in}{1.164756in}}%
\pgfpathclose%
\pgfusepath{fill}%
\end{pgfscope}%
\begin{pgfscope}%
\pgfpathrectangle{\pgfqpoint{0.150000in}{0.150000in}}{\pgfqpoint{2.700000in}{1.950000in}}%
\pgfusepath{clip}%
\pgfsetbuttcap%
\pgfsetroundjoin%
\definecolor{currentfill}{rgb}{0.937316,0.886259,0.890303}%
\pgfsetfillcolor{currentfill}%
\pgfsetlinewidth{0.000000pt}%
\definecolor{currentstroke}{rgb}{0.000000,0.000000,0.000000}%
\pgfsetstrokecolor{currentstroke}%
\pgfsetdash{}{0pt}%
\pgfpathmoveto{\pgfqpoint{1.499136in}{1.101896in}}%
\pgfpathlineto{\pgfqpoint{1.536486in}{1.121025in}}%
\pgfpathlineto{\pgfqpoint{1.499227in}{1.152407in}}%
\pgfpathlineto{\pgfqpoint{1.461832in}{1.133346in}}%
\pgfpathclose%
\pgfusepath{fill}%
\end{pgfscope}%
\begin{pgfscope}%
\pgfpathrectangle{\pgfqpoint{0.150000in}{0.150000in}}{\pgfqpoint{2.700000in}{1.950000in}}%
\pgfusepath{clip}%
\pgfsetbuttcap%
\pgfsetroundjoin%
\definecolor{currentfill}{rgb}{0.884130,0.789752,0.797227}%
\pgfsetfillcolor{currentfill}%
\pgfsetlinewidth{0.000000pt}%
\definecolor{currentstroke}{rgb}{0.000000,0.000000,0.000000}%
\pgfsetstrokecolor{currentstroke}%
\pgfsetdash{}{0pt}%
\pgfpathmoveto{\pgfqpoint{1.686001in}{0.995101in}}%
\pgfpathlineto{\pgfqpoint{1.722823in}{1.002262in}}%
\pgfpathlineto{\pgfqpoint{1.685461in}{1.033705in}}%
\pgfpathlineto{\pgfqpoint{1.648551in}{1.026642in}}%
\pgfpathclose%
\pgfusepath{fill}%
\end{pgfscope}%
\begin{pgfscope}%
\pgfpathrectangle{\pgfqpoint{0.150000in}{0.150000in}}{\pgfqpoint{2.700000in}{1.950000in}}%
\pgfusepath{clip}%
\pgfsetbuttcap%
\pgfsetroundjoin%
\definecolor{currentfill}{rgb}{0.872733,0.769072,0.777282}%
\pgfsetfillcolor{currentfill}%
\pgfsetlinewidth{0.000000pt}%
\definecolor{currentstroke}{rgb}{0.000000,0.000000,0.000000}%
\pgfsetstrokecolor{currentstroke}%
\pgfsetdash{}{0pt}%
\pgfpathmoveto{\pgfqpoint{1.723498in}{0.963520in}}%
\pgfpathlineto{\pgfqpoint{1.760232in}{0.970779in}}%
\pgfpathlineto{\pgfqpoint{1.722823in}{1.002262in}}%
\pgfpathlineto{\pgfqpoint{1.686001in}{0.995101in}}%
\pgfpathclose%
\pgfusepath{fill}%
\end{pgfscope}%
\begin{pgfscope}%
\pgfpathrectangle{\pgfqpoint{0.150000in}{0.150000in}}{\pgfqpoint{2.700000in}{1.950000in}}%
\pgfusepath{clip}%
\pgfsetbuttcap%
\pgfsetroundjoin%
\definecolor{currentfill}{rgb}{0.963909,0.934513,0.936841}%
\pgfsetfillcolor{currentfill}%
\pgfsetlinewidth{0.000000pt}%
\definecolor{currentstroke}{rgb}{0.000000,0.000000,0.000000}%
\pgfsetstrokecolor{currentstroke}%
\pgfsetdash{}{0pt}%
\pgfpathmoveto{\pgfqpoint{1.237167in}{1.158074in}}%
\pgfpathlineto{\pgfqpoint{1.275362in}{1.164756in}}%
\pgfpathlineto{\pgfqpoint{1.238597in}{1.183748in}}%
\pgfpathlineto{\pgfqpoint{1.199967in}{1.189542in}}%
\pgfpathclose%
\pgfusepath{fill}%
\end{pgfscope}%
\begin{pgfscope}%
\pgfpathrectangle{\pgfqpoint{0.150000in}{0.150000in}}{\pgfqpoint{2.700000in}{1.950000in}}%
\pgfusepath{clip}%
\pgfsetbuttcap%
\pgfsetroundjoin%
\definecolor{currentfill}{rgb}{0.925919,0.865579,0.870358}%
\pgfsetfillcolor{currentfill}%
\pgfsetlinewidth{0.000000pt}%
\definecolor{currentstroke}{rgb}{0.000000,0.000000,0.000000}%
\pgfsetstrokecolor{currentstroke}%
\pgfsetdash{}{0pt}%
\pgfpathmoveto{\pgfqpoint{1.536486in}{1.082697in}}%
\pgfpathlineto{\pgfqpoint{1.573794in}{1.089604in}}%
\pgfpathlineto{\pgfqpoint{1.536486in}{1.121025in}}%
\pgfpathlineto{\pgfqpoint{1.499136in}{1.101896in}}%
\pgfpathclose%
\pgfusepath{fill}%
\end{pgfscope}%
\begin{pgfscope}%
\pgfpathrectangle{\pgfqpoint{0.150000in}{0.150000in}}{\pgfqpoint{2.700000in}{1.950000in}}%
\pgfusepath{clip}%
\pgfsetbuttcap%
\pgfsetroundjoin%
\definecolor{currentfill}{rgb}{0.996890,0.997273,0.997809}%
\pgfsetfillcolor{currentfill}%
\pgfsetlinewidth{0.000000pt}%
\definecolor{currentstroke}{rgb}{0.000000,0.000000,0.000000}%
\pgfsetstrokecolor{currentstroke}%
\pgfsetdash{}{0pt}%
\pgfpathmoveto{\pgfqpoint{1.086495in}{1.258454in}}%
\pgfpathlineto{\pgfqpoint{1.126670in}{1.227456in}}%
\pgfpathlineto{\pgfqpoint{1.090219in}{1.246311in}}%
\pgfpathlineto{\pgfqpoint{1.049880in}{1.277376in}}%
\pgfpathclose%
\pgfusepath{fill}%
\end{pgfscope}%
\begin{pgfscope}%
\pgfpathrectangle{\pgfqpoint{0.150000in}{0.150000in}}{\pgfqpoint{2.700000in}{1.950000in}}%
\pgfusepath{clip}%
\pgfsetbuttcap%
\pgfsetroundjoin%
\definecolor{currentfill}{rgb}{0.948713,0.906939,0.910248}%
\pgfsetfillcolor{currentfill}%
\pgfsetlinewidth{0.000000pt}%
\definecolor{currentstroke}{rgb}{0.000000,0.000000,0.000000}%
\pgfsetstrokecolor{currentstroke}%
\pgfsetdash{}{0pt}%
\pgfpathmoveto{\pgfqpoint{1.424302in}{1.114216in}}%
\pgfpathlineto{\pgfqpoint{1.461832in}{1.133346in}}%
\pgfpathlineto{\pgfqpoint{1.424576in}{1.164756in}}%
\pgfpathlineto{\pgfqpoint{1.387002in}{1.145695in}}%
\pgfpathclose%
\pgfusepath{fill}%
\end{pgfscope}%
\begin{pgfscope}%
\pgfpathrectangle{\pgfqpoint{0.150000in}{0.150000in}}{\pgfqpoint{2.700000in}{1.950000in}}%
\pgfusepath{clip}%
\pgfsetbuttcap%
\pgfsetroundjoin%
\definecolor{currentfill}{rgb}{0.827145,0.686351,0.697503}%
\pgfsetfillcolor{currentfill}%
\pgfsetlinewidth{0.000000pt}%
\definecolor{currentstroke}{rgb}{0.000000,0.000000,0.000000}%
\pgfsetstrokecolor{currentstroke}%
\pgfsetdash{}{0pt}%
\pgfpathmoveto{\pgfqpoint{2.136071in}{0.868533in}}%
\pgfpathlineto{\pgfqpoint{2.171644in}{0.876090in}}%
\pgfpathlineto{\pgfqpoint{2.132504in}{0.883602in}}%
\pgfpathlineto{\pgfqpoint{2.096920in}{0.876090in}}%
\pgfpathclose%
\pgfusepath{fill}%
\end{pgfscope}%
\begin{pgfscope}%
\pgfpathrectangle{\pgfqpoint{0.150000in}{0.150000in}}{\pgfqpoint{2.700000in}{1.950000in}}%
\pgfusepath{clip}%
\pgfsetbuttcap%
\pgfsetroundjoin%
\definecolor{currentfill}{rgb}{0.827145,0.686351,0.697503}%
\pgfsetfillcolor{currentfill}%
\pgfsetlinewidth{0.000000pt}%
\definecolor{currentstroke}{rgb}{0.000000,0.000000,0.000000}%
\pgfsetstrokecolor{currentstroke}%
\pgfsetdash{}{0pt}%
\pgfpathmoveto{\pgfqpoint{2.061123in}{0.868533in}}%
\pgfpathlineto{\pgfqpoint{2.096920in}{0.876090in}}%
\pgfpathlineto{\pgfqpoint{2.058002in}{0.883602in}}%
\pgfpathlineto{\pgfqpoint{2.022195in}{0.876090in}}%
\pgfpathclose%
\pgfusepath{fill}%
\end{pgfscope}%
\begin{pgfscope}%
\pgfpathrectangle{\pgfqpoint{0.150000in}{0.150000in}}{\pgfqpoint{2.700000in}{1.950000in}}%
\pgfusepath{clip}%
\pgfsetbuttcap%
\pgfsetroundjoin%
\definecolor{currentfill}{rgb}{0.827145,0.686351,0.697503}%
\pgfsetfillcolor{currentfill}%
\pgfsetlinewidth{0.000000pt}%
\definecolor{currentstroke}{rgb}{0.000000,0.000000,0.000000}%
\pgfsetstrokecolor{currentstroke}%
\pgfsetdash{}{0pt}%
\pgfpathmoveto{\pgfqpoint{1.986175in}{0.868533in}}%
\pgfpathlineto{\pgfqpoint{2.022195in}{0.876090in}}%
\pgfpathlineto{\pgfqpoint{1.983499in}{0.883602in}}%
\pgfpathlineto{\pgfqpoint{1.947471in}{0.876090in}}%
\pgfpathclose%
\pgfusepath{fill}%
\end{pgfscope}%
\begin{pgfscope}%
\pgfpathrectangle{\pgfqpoint{0.150000in}{0.150000in}}{\pgfqpoint{2.700000in}{1.950000in}}%
\pgfusepath{clip}%
\pgfsetbuttcap%
\pgfsetroundjoin%
\definecolor{currentfill}{rgb}{0.865135,0.755285,0.763986}%
\pgfsetfillcolor{currentfill}%
\pgfsetlinewidth{0.000000pt}%
\definecolor{currentstroke}{rgb}{0.000000,0.000000,0.000000}%
\pgfsetstrokecolor{currentstroke}%
\pgfsetdash{}{0pt}%
\pgfpathmoveto{\pgfqpoint{1.761308in}{0.944043in}}%
\pgfpathlineto{\pgfqpoint{1.797689in}{0.939256in}}%
\pgfpathlineto{\pgfqpoint{1.760232in}{0.970779in}}%
\pgfpathlineto{\pgfqpoint{1.723498in}{0.963520in}}%
\pgfpathclose%
\pgfusepath{fill}%
\end{pgfscope}%
\begin{pgfscope}%
\pgfpathrectangle{\pgfqpoint{0.150000in}{0.150000in}}{\pgfqpoint{2.700000in}{1.950000in}}%
\pgfusepath{clip}%
\pgfsetbuttcap%
\pgfsetroundjoin%
\definecolor{currentfill}{rgb}{0.952512,0.913833,0.916896}%
\pgfsetfillcolor{currentfill}%
\pgfsetlinewidth{0.000000pt}%
\definecolor{currentstroke}{rgb}{0.000000,0.000000,0.000000}%
\pgfsetstrokecolor{currentstroke}%
\pgfsetdash{}{0pt}%
\pgfpathmoveto{\pgfqpoint{1.349292in}{1.126565in}}%
\pgfpathlineto{\pgfqpoint{1.387002in}{1.145695in}}%
\pgfpathlineto{\pgfqpoint{1.349969in}{1.164756in}}%
\pgfpathlineto{\pgfqpoint{1.312260in}{1.145695in}}%
\pgfpathclose%
\pgfusepath{fill}%
\end{pgfscope}%
\begin{pgfscope}%
\pgfpathrectangle{\pgfqpoint{0.150000in}{0.150000in}}{\pgfqpoint{2.700000in}{1.950000in}}%
\pgfusepath{clip}%
\pgfsetbuttcap%
\pgfsetroundjoin%
\definecolor{currentfill}{rgb}{0.918321,0.851792,0.857062}%
\pgfsetfillcolor{currentfill}%
\pgfsetlinewidth{0.000000pt}%
\definecolor{currentstroke}{rgb}{0.000000,0.000000,0.000000}%
\pgfsetstrokecolor{currentstroke}%
\pgfsetdash{}{0pt}%
\pgfpathmoveto{\pgfqpoint{1.573929in}{1.051137in}}%
\pgfpathlineto{\pgfqpoint{1.611148in}{1.058143in}}%
\pgfpathlineto{\pgfqpoint{1.573794in}{1.089604in}}%
\pgfpathlineto{\pgfqpoint{1.536486in}{1.082697in}}%
\pgfpathclose%
\pgfusepath{fill}%
\end{pgfscope}%
\begin{pgfscope}%
\pgfpathrectangle{\pgfqpoint{0.150000in}{0.150000in}}{\pgfqpoint{2.700000in}{1.950000in}}%
\pgfusepath{clip}%
\pgfsetbuttcap%
\pgfsetroundjoin%
\definecolor{currentfill}{rgb}{0.941115,0.893153,0.896952}%
\pgfsetfillcolor{currentfill}%
\pgfsetlinewidth{0.000000pt}%
\definecolor{currentstroke}{rgb}{0.000000,0.000000,0.000000}%
\pgfsetstrokecolor{currentstroke}%
\pgfsetdash{}{0pt}%
\pgfpathmoveto{\pgfqpoint{1.461561in}{1.095017in}}%
\pgfpathlineto{\pgfqpoint{1.499136in}{1.101896in}}%
\pgfpathlineto{\pgfqpoint{1.461832in}{1.133346in}}%
\pgfpathlineto{\pgfqpoint{1.424302in}{1.114216in}}%
\pgfpathclose%
\pgfusepath{fill}%
\end{pgfscope}%
\begin{pgfscope}%
\pgfpathrectangle{\pgfqpoint{0.150000in}{0.150000in}}{\pgfqpoint{2.700000in}{1.950000in}}%
\pgfusepath{clip}%
\pgfsetbuttcap%
\pgfsetroundjoin%
\definecolor{currentfill}{rgb}{0.830944,0.693244,0.704151}%
\pgfsetfillcolor{currentfill}%
\pgfsetlinewidth{0.000000pt}%
\definecolor{currentstroke}{rgb}{0.000000,0.000000,0.000000}%
\pgfsetstrokecolor{currentstroke}%
\pgfsetdash{}{0pt}%
\pgfpathmoveto{\pgfqpoint{1.911668in}{0.880619in}}%
\pgfpathlineto{\pgfqpoint{1.947471in}{0.876090in}}%
\pgfpathlineto{\pgfqpoint{1.908997in}{0.883602in}}%
\pgfpathlineto{\pgfqpoint{1.873141in}{0.888149in}}%
\pgfpathclose%
\pgfusepath{fill}%
\end{pgfscope}%
\begin{pgfscope}%
\pgfpathrectangle{\pgfqpoint{0.150000in}{0.150000in}}{\pgfqpoint{2.700000in}{1.950000in}}%
\pgfusepath{clip}%
\pgfsetbuttcap%
\pgfsetroundjoin%
\definecolor{currentfill}{rgb}{0.830944,0.693244,0.704151}%
\pgfsetfillcolor{currentfill}%
\pgfsetlinewidth{0.000000pt}%
\definecolor{currentstroke}{rgb}{0.000000,0.000000,0.000000}%
\pgfsetstrokecolor{currentstroke}%
\pgfsetdash{}{0pt}%
\pgfpathmoveto{\pgfqpoint{2.211814in}{0.880619in}}%
\pgfpathlineto{\pgfqpoint{2.247203in}{0.888149in}}%
\pgfpathlineto{\pgfqpoint{2.207006in}{0.883602in}}%
\pgfpathlineto{\pgfqpoint{2.171644in}{0.876090in}}%
\pgfpathclose%
\pgfusepath{fill}%
\end{pgfscope}%
\begin{pgfscope}%
\pgfpathrectangle{\pgfqpoint{0.150000in}{0.150000in}}{\pgfqpoint{2.700000in}{1.950000in}}%
\pgfusepath{clip}%
\pgfsetbuttcap%
\pgfsetroundjoin%
\definecolor{currentfill}{rgb}{0.960110,0.927619,0.930193}%
\pgfsetfillcolor{currentfill}%
\pgfsetlinewidth{0.000000pt}%
\definecolor{currentstroke}{rgb}{0.000000,0.000000,0.000000}%
\pgfsetstrokecolor{currentstroke}%
\pgfsetdash{}{0pt}%
\pgfpathmoveto{\pgfqpoint{1.274106in}{1.138944in}}%
\pgfpathlineto{\pgfqpoint{1.312260in}{1.145695in}}%
\pgfpathlineto{\pgfqpoint{1.275362in}{1.164756in}}%
\pgfpathlineto{\pgfqpoint{1.237167in}{1.158074in}}%
\pgfpathclose%
\pgfusepath{fill}%
\end{pgfscope}%
\begin{pgfscope}%
\pgfpathrectangle{\pgfqpoint{0.150000in}{0.150000in}}{\pgfqpoint{2.700000in}{1.950000in}}%
\pgfusepath{clip}%
\pgfsetbuttcap%
\pgfsetroundjoin%
\definecolor{currentfill}{rgb}{0.906924,0.831112,0.837117}%
\pgfsetfillcolor{currentfill}%
\pgfsetlinewidth{0.000000pt}%
\definecolor{currentstroke}{rgb}{0.000000,0.000000,0.000000}%
\pgfsetstrokecolor{currentstroke}%
\pgfsetdash{}{0pt}%
\pgfpathmoveto{\pgfqpoint{1.611419in}{1.019538in}}%
\pgfpathlineto{\pgfqpoint{1.648551in}{1.026642in}}%
\pgfpathlineto{\pgfqpoint{1.611148in}{1.058143in}}%
\pgfpathlineto{\pgfqpoint{1.573929in}{1.051137in}}%
\pgfpathclose%
\pgfusepath{fill}%
\end{pgfscope}%
\begin{pgfscope}%
\pgfpathrectangle{\pgfqpoint{0.150000in}{0.150000in}}{\pgfqpoint{2.700000in}{1.950000in}}%
\pgfusepath{clip}%
\pgfsetbuttcap%
\pgfsetroundjoin%
\definecolor{currentfill}{rgb}{0.998100,0.996553,0.996676}%
\pgfsetfillcolor{currentfill}%
\pgfsetlinewidth{0.000000pt}%
\definecolor{currentstroke}{rgb}{0.000000,0.000000,0.000000}%
\pgfsetstrokecolor{currentstroke}%
\pgfsetdash{}{0pt}%
\pgfpathmoveto{\pgfqpoint{1.123731in}{1.226939in}}%
\pgfpathlineto{\pgfqpoint{1.163252in}{1.208533in}}%
\pgfpathlineto{\pgfqpoint{1.126670in}{1.227456in}}%
\pgfpathlineto{\pgfqpoint{1.086495in}{1.258454in}}%
\pgfpathclose%
\pgfusepath{fill}%
\end{pgfscope}%
\begin{pgfscope}%
\pgfpathrectangle{\pgfqpoint{0.150000in}{0.150000in}}{\pgfqpoint{2.700000in}{1.950000in}}%
\pgfusepath{clip}%
\pgfsetbuttcap%
\pgfsetroundjoin%
\definecolor{currentfill}{rgb}{0.853738,0.734605,0.744041}%
\pgfsetfillcolor{currentfill}%
\pgfsetlinewidth{0.000000pt}%
\definecolor{currentstroke}{rgb}{0.000000,0.000000,0.000000}%
\pgfsetstrokecolor{currentstroke}%
\pgfsetdash{}{0pt}%
\pgfpathmoveto{\pgfqpoint{1.798946in}{0.912352in}}%
\pgfpathlineto{\pgfqpoint{1.835194in}{0.907693in}}%
\pgfpathlineto{\pgfqpoint{1.797689in}{0.939256in}}%
\pgfpathlineto{\pgfqpoint{1.761308in}{0.944043in}}%
\pgfpathclose%
\pgfusepath{fill}%
\end{pgfscope}%
\begin{pgfscope}%
\pgfpathrectangle{\pgfqpoint{0.150000in}{0.150000in}}{\pgfqpoint{2.700000in}{1.950000in}}%
\pgfusepath{clip}%
\pgfsetbuttcap%
\pgfsetroundjoin%
\definecolor{currentfill}{rgb}{0.895527,0.810432,0.817172}%
\pgfsetfillcolor{currentfill}%
\pgfsetlinewidth{0.000000pt}%
\definecolor{currentstroke}{rgb}{0.000000,0.000000,0.000000}%
\pgfsetstrokecolor{currentstroke}%
\pgfsetdash{}{0pt}%
\pgfpathmoveto{\pgfqpoint{1.648958in}{0.987898in}}%
\pgfpathlineto{\pgfqpoint{1.686001in}{0.995101in}}%
\pgfpathlineto{\pgfqpoint{1.648551in}{1.026642in}}%
\pgfpathlineto{\pgfqpoint{1.611419in}{1.019538in}}%
\pgfpathclose%
\pgfusepath{fill}%
\end{pgfscope}%
\begin{pgfscope}%
\pgfpathrectangle{\pgfqpoint{0.150000in}{0.150000in}}{\pgfqpoint{2.700000in}{1.950000in}}%
\pgfusepath{clip}%
\pgfsetbuttcap%
\pgfsetroundjoin%
\definecolor{currentfill}{rgb}{0.952512,0.913833,0.916896}%
\pgfsetfillcolor{currentfill}%
\pgfsetlinewidth{0.000000pt}%
\definecolor{currentstroke}{rgb}{0.000000,0.000000,0.000000}%
\pgfsetstrokecolor{currentstroke}%
\pgfsetdash{}{0pt}%
\pgfpathmoveto{\pgfqpoint{1.386459in}{1.107366in}}%
\pgfpathlineto{\pgfqpoint{1.424302in}{1.114216in}}%
\pgfpathlineto{\pgfqpoint{1.387002in}{1.145695in}}%
\pgfpathlineto{\pgfqpoint{1.349292in}{1.126565in}}%
\pgfpathclose%
\pgfusepath{fill}%
\end{pgfscope}%
\begin{pgfscope}%
\pgfpathrectangle{\pgfqpoint{0.150000in}{0.150000in}}{\pgfqpoint{2.700000in}{1.950000in}}%
\pgfusepath{clip}%
\pgfsetbuttcap%
\pgfsetroundjoin%
\definecolor{currentfill}{rgb}{0.933517,0.879366,0.883655}%
\pgfsetfillcolor{currentfill}%
\pgfsetlinewidth{0.000000pt}%
\definecolor{currentstroke}{rgb}{0.000000,0.000000,0.000000}%
\pgfsetstrokecolor{currentstroke}%
\pgfsetdash{}{0pt}%
\pgfpathmoveto{\pgfqpoint{1.499000in}{1.063428in}}%
\pgfpathlineto{\pgfqpoint{1.536486in}{1.082697in}}%
\pgfpathlineto{\pgfqpoint{1.499136in}{1.101896in}}%
\pgfpathlineto{\pgfqpoint{1.461561in}{1.095017in}}%
\pgfpathclose%
\pgfusepath{fill}%
\end{pgfscope}%
\begin{pgfscope}%
\pgfpathrectangle{\pgfqpoint{0.150000in}{0.150000in}}{\pgfqpoint{2.700000in}{1.950000in}}%
\pgfusepath{clip}%
\pgfsetbuttcap%
\pgfsetroundjoin%
\definecolor{currentfill}{rgb}{0.849939,0.727711,0.737393}%
\pgfsetfillcolor{currentfill}%
\pgfsetlinewidth{0.000000pt}%
\definecolor{currentstroke}{rgb}{0.000000,0.000000,0.000000}%
\pgfsetstrokecolor{currentstroke}%
\pgfsetdash{}{0pt}%
\pgfpathmoveto{\pgfqpoint{1.836986in}{0.892734in}}%
\pgfpathlineto{\pgfqpoint{1.873141in}{0.888149in}}%
\pgfpathlineto{\pgfqpoint{1.835194in}{0.907693in}}%
\pgfpathlineto{\pgfqpoint{1.798946in}{0.912352in}}%
\pgfpathclose%
\pgfusepath{fill}%
\end{pgfscope}%
\begin{pgfscope}%
\pgfpathrectangle{\pgfqpoint{0.150000in}{0.150000in}}{\pgfqpoint{2.700000in}{1.950000in}}%
\pgfusepath{clip}%
\pgfsetbuttcap%
\pgfsetroundjoin%
\definecolor{currentfill}{rgb}{0.994301,0.989660,0.990028}%
\pgfsetfillcolor{currentfill}%
\pgfsetlinewidth{0.000000pt}%
\definecolor{currentstroke}{rgb}{0.000000,0.000000,0.000000}%
\pgfsetstrokecolor{currentstroke}%
\pgfsetdash{}{0pt}%
\pgfpathmoveto{\pgfqpoint{1.160570in}{1.207879in}}%
\pgfpathlineto{\pgfqpoint{1.199967in}{1.189542in}}%
\pgfpathlineto{\pgfqpoint{1.163252in}{1.208533in}}%
\pgfpathlineto{\pgfqpoint{1.123731in}{1.226939in}}%
\pgfpathclose%
\pgfusepath{fill}%
\end{pgfscope}%
\begin{pgfscope}%
\pgfpathrectangle{\pgfqpoint{0.150000in}{0.150000in}}{\pgfqpoint{2.700000in}{1.950000in}}%
\pgfusepath{clip}%
\pgfsetbuttcap%
\pgfsetroundjoin%
\definecolor{currentfill}{rgb}{0.887929,0.796645,0.803876}%
\pgfsetfillcolor{currentfill}%
\pgfsetlinewidth{0.000000pt}%
\definecolor{currentstroke}{rgb}{0.000000,0.000000,0.000000}%
\pgfsetstrokecolor{currentstroke}%
\pgfsetdash{}{0pt}%
\pgfpathmoveto{\pgfqpoint{1.686721in}{0.968419in}}%
\pgfpathlineto{\pgfqpoint{1.723498in}{0.963520in}}%
\pgfpathlineto{\pgfqpoint{1.686001in}{0.995101in}}%
\pgfpathlineto{\pgfqpoint{1.648958in}{0.987898in}}%
\pgfpathclose%
\pgfusepath{fill}%
\end{pgfscope}%
\begin{pgfscope}%
\pgfpathrectangle{\pgfqpoint{0.150000in}{0.150000in}}{\pgfqpoint{2.700000in}{1.950000in}}%
\pgfusepath{clip}%
\pgfsetbuttcap%
\pgfsetroundjoin%
\definecolor{currentfill}{rgb}{0.960110,0.927619,0.930193}%
\pgfsetfillcolor{currentfill}%
\pgfsetlinewidth{0.000000pt}%
\definecolor{currentstroke}{rgb}{0.000000,0.000000,0.000000}%
\pgfsetstrokecolor{currentstroke}%
\pgfsetdash{}{0pt}%
\pgfpathmoveto{\pgfqpoint{1.311180in}{1.119744in}}%
\pgfpathlineto{\pgfqpoint{1.349292in}{1.126565in}}%
\pgfpathlineto{\pgfqpoint{1.312260in}{1.145695in}}%
\pgfpathlineto{\pgfqpoint{1.274106in}{1.138944in}}%
\pgfpathclose%
\pgfusepath{fill}%
\end{pgfscope}%
\begin{pgfscope}%
\pgfpathrectangle{\pgfqpoint{0.150000in}{0.150000in}}{\pgfqpoint{2.700000in}{1.950000in}}%
\pgfusepath{clip}%
\pgfsetbuttcap%
\pgfsetroundjoin%
\definecolor{currentfill}{rgb}{0.944914,0.900046,0.903600}%
\pgfsetfillcolor{currentfill}%
\pgfsetlinewidth{0.000000pt}%
\definecolor{currentstroke}{rgb}{0.000000,0.000000,0.000000}%
\pgfsetstrokecolor{currentstroke}%
\pgfsetdash{}{0pt}%
\pgfpathmoveto{\pgfqpoint{1.423894in}{1.075748in}}%
\pgfpathlineto{\pgfqpoint{1.461561in}{1.095017in}}%
\pgfpathlineto{\pgfqpoint{1.424302in}{1.114216in}}%
\pgfpathlineto{\pgfqpoint{1.386459in}{1.107366in}}%
\pgfpathclose%
\pgfusepath{fill}%
\end{pgfscope}%
\begin{pgfscope}%
\pgfpathrectangle{\pgfqpoint{0.150000in}{0.150000in}}{\pgfqpoint{2.700000in}{1.950000in}}%
\pgfusepath{clip}%
\pgfsetbuttcap%
\pgfsetroundjoin%
\definecolor{currentfill}{rgb}{0.838542,0.707031,0.717448}%
\pgfsetfillcolor{currentfill}%
\pgfsetlinewidth{0.000000pt}%
\definecolor{currentstroke}{rgb}{0.000000,0.000000,0.000000}%
\pgfsetstrokecolor{currentstroke}%
\pgfsetdash{}{0pt}%
\pgfpathmoveto{\pgfqpoint{2.100284in}{0.860931in}}%
\pgfpathlineto{\pgfqpoint{2.136071in}{0.868533in}}%
\pgfpathlineto{\pgfqpoint{2.096920in}{0.876090in}}%
\pgfpathlineto{\pgfqpoint{2.061123in}{0.868533in}}%
\pgfpathclose%
\pgfusepath{fill}%
\end{pgfscope}%
\begin{pgfscope}%
\pgfpathrectangle{\pgfqpoint{0.150000in}{0.150000in}}{\pgfqpoint{2.700000in}{1.950000in}}%
\pgfusepath{clip}%
\pgfsetbuttcap%
\pgfsetroundjoin%
\definecolor{currentfill}{rgb}{0.838542,0.707031,0.717448}%
\pgfsetfillcolor{currentfill}%
\pgfsetlinewidth{0.000000pt}%
\definecolor{currentstroke}{rgb}{0.000000,0.000000,0.000000}%
\pgfsetstrokecolor{currentstroke}%
\pgfsetdash{}{0pt}%
\pgfpathmoveto{\pgfqpoint{2.025111in}{0.860931in}}%
\pgfpathlineto{\pgfqpoint{2.061123in}{0.868533in}}%
\pgfpathlineto{\pgfqpoint{2.022195in}{0.876090in}}%
\pgfpathlineto{\pgfqpoint{1.986175in}{0.868533in}}%
\pgfpathclose%
\pgfusepath{fill}%
\end{pgfscope}%
\begin{pgfscope}%
\pgfpathrectangle{\pgfqpoint{0.150000in}{0.150000in}}{\pgfqpoint{2.700000in}{1.950000in}}%
\pgfusepath{clip}%
\pgfsetbuttcap%
\pgfsetroundjoin%
\definecolor{currentfill}{rgb}{0.925919,0.865579,0.870358}%
\pgfsetfillcolor{currentfill}%
\pgfsetlinewidth{0.000000pt}%
\definecolor{currentstroke}{rgb}{0.000000,0.000000,0.000000}%
\pgfsetstrokecolor{currentstroke}%
\pgfsetdash{}{0pt}%
\pgfpathmoveto{\pgfqpoint{1.536486in}{1.044089in}}%
\pgfpathlineto{\pgfqpoint{1.573929in}{1.051137in}}%
\pgfpathlineto{\pgfqpoint{1.536486in}{1.082697in}}%
\pgfpathlineto{\pgfqpoint{1.499000in}{1.063428in}}%
\pgfpathclose%
\pgfusepath{fill}%
\end{pgfscope}%
\begin{pgfscope}%
\pgfpathrectangle{\pgfqpoint{0.150000in}{0.150000in}}{\pgfqpoint{2.700000in}{1.950000in}}%
\pgfusepath{clip}%
\pgfsetbuttcap%
\pgfsetroundjoin%
\definecolor{currentfill}{rgb}{0.842341,0.713925,0.724096}%
\pgfsetfillcolor{currentfill}%
\pgfsetlinewidth{0.000000pt}%
\definecolor{currentstroke}{rgb}{0.000000,0.000000,0.000000}%
\pgfsetstrokecolor{currentstroke}%
\pgfsetdash{}{0pt}%
\pgfpathmoveto{\pgfqpoint{2.175457in}{0.860931in}}%
\pgfpathlineto{\pgfqpoint{2.211814in}{0.880619in}}%
\pgfpathlineto{\pgfqpoint{2.171644in}{0.876090in}}%
\pgfpathlineto{\pgfqpoint{2.136071in}{0.868533in}}%
\pgfpathclose%
\pgfusepath{fill}%
\end{pgfscope}%
\begin{pgfscope}%
\pgfpathrectangle{\pgfqpoint{0.150000in}{0.150000in}}{\pgfqpoint{2.700000in}{1.950000in}}%
\pgfusepath{clip}%
\pgfsetbuttcap%
\pgfsetroundjoin%
\definecolor{currentfill}{rgb}{0.842341,0.713925,0.724096}%
\pgfsetfillcolor{currentfill}%
\pgfsetlinewidth{0.000000pt}%
\definecolor{currentstroke}{rgb}{0.000000,0.000000,0.000000}%
\pgfsetstrokecolor{currentstroke}%
\pgfsetdash{}{0pt}%
\pgfpathmoveto{\pgfqpoint{1.950427in}{0.873045in}}%
\pgfpathlineto{\pgfqpoint{1.986175in}{0.868533in}}%
\pgfpathlineto{\pgfqpoint{1.947471in}{0.876090in}}%
\pgfpathlineto{\pgfqpoint{1.911668in}{0.880619in}}%
\pgfpathclose%
\pgfusepath{fill}%
\end{pgfscope}%
\begin{pgfscope}%
\pgfpathrectangle{\pgfqpoint{0.150000in}{0.150000in}}{\pgfqpoint{2.700000in}{1.950000in}}%
\pgfusepath{clip}%
\pgfsetbuttcap%
\pgfsetroundjoin%
\definecolor{currentfill}{rgb}{0.880331,0.782858,0.790579}%
\pgfsetfillcolor{currentfill}%
\pgfsetlinewidth{0.000000pt}%
\definecolor{currentstroke}{rgb}{0.000000,0.000000,0.000000}%
\pgfsetstrokecolor{currentstroke}%
\pgfsetdash{}{0pt}%
\pgfpathmoveto{\pgfqpoint{1.724400in}{0.936669in}}%
\pgfpathlineto{\pgfqpoint{1.761308in}{0.944043in}}%
\pgfpathlineto{\pgfqpoint{1.723498in}{0.963520in}}%
\pgfpathlineto{\pgfqpoint{1.686721in}{0.968419in}}%
\pgfpathclose%
\pgfusepath{fill}%
\end{pgfscope}%
\begin{pgfscope}%
\pgfpathrectangle{\pgfqpoint{0.150000in}{0.150000in}}{\pgfqpoint{2.700000in}{1.950000in}}%
\pgfusepath{clip}%
\pgfsetbuttcap%
\pgfsetroundjoin%
\definecolor{currentfill}{rgb}{0.918321,0.851792,0.857062}%
\pgfsetfillcolor{currentfill}%
\pgfsetlinewidth{0.000000pt}%
\definecolor{currentstroke}{rgb}{0.000000,0.000000,0.000000}%
\pgfsetstrokecolor{currentstroke}%
\pgfsetdash{}{0pt}%
\pgfpathmoveto{\pgfqpoint{1.574065in}{1.012390in}}%
\pgfpathlineto{\pgfqpoint{1.611419in}{1.019538in}}%
\pgfpathlineto{\pgfqpoint{1.573929in}{1.051137in}}%
\pgfpathlineto{\pgfqpoint{1.536486in}{1.044089in}}%
\pgfpathclose%
\pgfusepath{fill}%
\end{pgfscope}%
\begin{pgfscope}%
\pgfpathrectangle{\pgfqpoint{0.150000in}{0.150000in}}{\pgfqpoint{2.700000in}{1.950000in}}%
\pgfusepath{clip}%
\pgfsetbuttcap%
\pgfsetroundjoin%
\definecolor{currentfill}{rgb}{0.990502,0.982767,0.983379}%
\pgfsetfillcolor{currentfill}%
\pgfsetlinewidth{0.000000pt}%
\definecolor{currentstroke}{rgb}{0.000000,0.000000,0.000000}%
\pgfsetstrokecolor{currentstroke}%
\pgfsetdash{}{0pt}%
\pgfpathmoveto{\pgfqpoint{1.197545in}{1.188750in}}%
\pgfpathlineto{\pgfqpoint{1.237167in}{1.158074in}}%
\pgfpathlineto{\pgfqpoint{1.199967in}{1.189542in}}%
\pgfpathlineto{\pgfqpoint{1.160570in}{1.207879in}}%
\pgfpathclose%
\pgfusepath{fill}%
\end{pgfscope}%
\begin{pgfscope}%
\pgfpathrectangle{\pgfqpoint{0.150000in}{0.150000in}}{\pgfqpoint{2.700000in}{1.950000in}}%
\pgfusepath{clip}%
\pgfsetbuttcap%
\pgfsetroundjoin%
\definecolor{currentfill}{rgb}{0.941115,0.893153,0.896952}%
\pgfsetfillcolor{currentfill}%
\pgfsetlinewidth{0.000000pt}%
\definecolor{currentstroke}{rgb}{0.000000,0.000000,0.000000}%
\pgfsetstrokecolor{currentstroke}%
\pgfsetdash{}{0pt}%
\pgfpathmoveto{\pgfqpoint{1.461288in}{1.056409in}}%
\pgfpathlineto{\pgfqpoint{1.499000in}{1.063428in}}%
\pgfpathlineto{\pgfqpoint{1.461561in}{1.095017in}}%
\pgfpathlineto{\pgfqpoint{1.423894in}{1.075748in}}%
\pgfpathclose%
\pgfusepath{fill}%
\end{pgfscope}%
\begin{pgfscope}%
\pgfpathrectangle{\pgfqpoint{0.150000in}{0.150000in}}{\pgfqpoint{2.700000in}{1.950000in}}%
\pgfusepath{clip}%
\pgfsetbuttcap%
\pgfsetroundjoin%
\definecolor{currentfill}{rgb}{0.960110,0.927619,0.930193}%
\pgfsetfillcolor{currentfill}%
\pgfsetlinewidth{0.000000pt}%
\definecolor{currentstroke}{rgb}{0.000000,0.000000,0.000000}%
\pgfsetstrokecolor{currentstroke}%
\pgfsetdash{}{0pt}%
\pgfpathmoveto{\pgfqpoint{1.348389in}{1.100475in}}%
\pgfpathlineto{\pgfqpoint{1.386459in}{1.107366in}}%
\pgfpathlineto{\pgfqpoint{1.349292in}{1.126565in}}%
\pgfpathlineto{\pgfqpoint{1.311180in}{1.119744in}}%
\pgfpathclose%
\pgfusepath{fill}%
\end{pgfscope}%
\begin{pgfscope}%
\pgfpathrectangle{\pgfqpoint{0.150000in}{0.150000in}}{\pgfqpoint{2.700000in}{1.950000in}}%
\pgfusepath{clip}%
\pgfsetbuttcap%
\pgfsetroundjoin%
\definecolor{currentfill}{rgb}{0.853738,0.734605,0.744041}%
\pgfsetfillcolor{currentfill}%
\pgfsetlinewidth{0.000000pt}%
\definecolor{currentstroke}{rgb}{0.000000,0.000000,0.000000}%
\pgfsetstrokecolor{currentstroke}%
\pgfsetdash{}{0pt}%
\pgfpathmoveto{\pgfqpoint{1.875165in}{0.873045in}}%
\pgfpathlineto{\pgfqpoint{1.911668in}{0.880619in}}%
\pgfpathlineto{\pgfqpoint{1.873141in}{0.888149in}}%
\pgfpathlineto{\pgfqpoint{1.836986in}{0.892734in}}%
\pgfpathclose%
\pgfusepath{fill}%
\end{pgfscope}%
\begin{pgfscope}%
\pgfpathrectangle{\pgfqpoint{0.150000in}{0.150000in}}{\pgfqpoint{2.700000in}{1.950000in}}%
\pgfusepath{clip}%
\pgfsetbuttcap%
\pgfsetroundjoin%
\definecolor{currentfill}{rgb}{0.906924,0.831112,0.837117}%
\pgfsetfillcolor{currentfill}%
\pgfsetlinewidth{0.000000pt}%
\definecolor{currentstroke}{rgb}{0.000000,0.000000,0.000000}%
\pgfsetstrokecolor{currentstroke}%
\pgfsetdash{}{0pt}%
\pgfpathmoveto{\pgfqpoint{1.611692in}{0.980651in}}%
\pgfpathlineto{\pgfqpoint{1.648958in}{0.987898in}}%
\pgfpathlineto{\pgfqpoint{1.611419in}{1.019538in}}%
\pgfpathlineto{\pgfqpoint{1.574065in}{1.012390in}}%
\pgfpathclose%
\pgfusepath{fill}%
\end{pgfscope}%
\begin{pgfscope}%
\pgfpathrectangle{\pgfqpoint{0.150000in}{0.150000in}}{\pgfqpoint{2.700000in}{1.950000in}}%
\pgfusepath{clip}%
\pgfsetbuttcap%
\pgfsetroundjoin%
\definecolor{currentfill}{rgb}{0.872733,0.769072,0.777282}%
\pgfsetfillcolor{currentfill}%
\pgfsetlinewidth{0.000000pt}%
\definecolor{currentstroke}{rgb}{0.000000,0.000000,0.000000}%
\pgfsetstrokecolor{currentstroke}%
\pgfsetdash{}{0pt}%
\pgfpathmoveto{\pgfqpoint{1.762127in}{0.904877in}}%
\pgfpathlineto{\pgfqpoint{1.798946in}{0.912352in}}%
\pgfpathlineto{\pgfqpoint{1.761308in}{0.944043in}}%
\pgfpathlineto{\pgfqpoint{1.724400in}{0.936669in}}%
\pgfpathclose%
\pgfusepath{fill}%
\end{pgfscope}%
\begin{pgfscope}%
\pgfpathrectangle{\pgfqpoint{0.150000in}{0.150000in}}{\pgfqpoint{2.700000in}{1.950000in}}%
\pgfusepath{clip}%
\pgfsetbuttcap%
\pgfsetroundjoin%
\definecolor{currentfill}{rgb}{0.986703,0.975873,0.976731}%
\pgfsetfillcolor{currentfill}%
\pgfsetlinewidth{0.000000pt}%
\definecolor{currentstroke}{rgb}{0.000000,0.000000,0.000000}%
\pgfsetstrokecolor{currentstroke}%
\pgfsetdash{}{0pt}%
\pgfpathmoveto{\pgfqpoint{1.235011in}{1.157055in}}%
\pgfpathlineto{\pgfqpoint{1.274106in}{1.138944in}}%
\pgfpathlineto{\pgfqpoint{1.237167in}{1.158074in}}%
\pgfpathlineto{\pgfqpoint{1.197545in}{1.188750in}}%
\pgfpathclose%
\pgfusepath{fill}%
\end{pgfscope}%
\begin{pgfscope}%
\pgfpathrectangle{\pgfqpoint{0.150000in}{0.150000in}}{\pgfqpoint{2.700000in}{1.950000in}}%
\pgfusepath{clip}%
\pgfsetbuttcap%
\pgfsetroundjoin%
\definecolor{currentfill}{rgb}{0.816529,0.839124,0.870757}%
\pgfsetfillcolor{currentfill}%
\pgfsetlinewidth{0.000000pt}%
\definecolor{currentstroke}{rgb}{0.000000,0.000000,0.000000}%
\pgfsetstrokecolor{currentstroke}%
\pgfsetdash{}{0pt}%
\pgfpathmoveto{\pgfqpoint{0.892651in}{1.480519in}}%
\pgfpathlineto{\pgfqpoint{0.939431in}{1.358863in}}%
\pgfpathlineto{\pgfqpoint{0.903260in}{1.377511in}}%
\pgfpathlineto{\pgfqpoint{0.855219in}{1.512164in}}%
\pgfpathclose%
\pgfusepath{fill}%
\end{pgfscope}%
\begin{pgfscope}%
\pgfpathrectangle{\pgfqpoint{0.150000in}{0.150000in}}{\pgfqpoint{2.700000in}{1.950000in}}%
\pgfusepath{clip}%
\pgfsetbuttcap%
\pgfsetroundjoin%
\definecolor{currentfill}{rgb}{0.952512,0.913833,0.916896}%
\pgfsetfillcolor{currentfill}%
\pgfsetlinewidth{0.000000pt}%
\definecolor{currentstroke}{rgb}{0.000000,0.000000,0.000000}%
\pgfsetstrokecolor{currentstroke}%
\pgfsetdash{}{0pt}%
\pgfpathmoveto{\pgfqpoint{1.385912in}{1.068757in}}%
\pgfpathlineto{\pgfqpoint{1.423894in}{1.075748in}}%
\pgfpathlineto{\pgfqpoint{1.386459in}{1.107366in}}%
\pgfpathlineto{\pgfqpoint{1.348389in}{1.100475in}}%
\pgfpathclose%
\pgfusepath{fill}%
\end{pgfscope}%
\begin{pgfscope}%
\pgfpathrectangle{\pgfqpoint{0.150000in}{0.150000in}}{\pgfqpoint{2.700000in}{1.950000in}}%
\pgfusepath{clip}%
\pgfsetbuttcap%
\pgfsetroundjoin%
\definecolor{currentfill}{rgb}{0.937316,0.886259,0.890303}%
\pgfsetfillcolor{currentfill}%
\pgfsetlinewidth{0.000000pt}%
\definecolor{currentstroke}{rgb}{0.000000,0.000000,0.000000}%
\pgfsetstrokecolor{currentstroke}%
\pgfsetdash{}{0pt}%
\pgfpathmoveto{\pgfqpoint{1.498819in}{1.036999in}}%
\pgfpathlineto{\pgfqpoint{1.536486in}{1.044089in}}%
\pgfpathlineto{\pgfqpoint{1.499000in}{1.063428in}}%
\pgfpathlineto{\pgfqpoint{1.461288in}{1.056409in}}%
\pgfpathclose%
\pgfusepath{fill}%
\end{pgfscope}%
\begin{pgfscope}%
\pgfpathrectangle{\pgfqpoint{0.150000in}{0.150000in}}{\pgfqpoint{2.700000in}{1.950000in}}%
\pgfusepath{clip}%
\pgfsetbuttcap%
\pgfsetroundjoin%
\definecolor{currentfill}{rgb}{0.903125,0.824219,0.830469}%
\pgfsetfillcolor{currentfill}%
\pgfsetlinewidth{0.000000pt}%
\definecolor{currentstroke}{rgb}{0.000000,0.000000,0.000000}%
\pgfsetstrokecolor{currentstroke}%
\pgfsetdash{}{0pt}%
\pgfpathmoveto{\pgfqpoint{1.649501in}{0.961101in}}%
\pgfpathlineto{\pgfqpoint{1.686721in}{0.968419in}}%
\pgfpathlineto{\pgfqpoint{1.648958in}{0.987898in}}%
\pgfpathlineto{\pgfqpoint{1.611692in}{0.980651in}}%
\pgfpathclose%
\pgfusepath{fill}%
\end{pgfscope}%
\begin{pgfscope}%
\pgfpathrectangle{\pgfqpoint{0.150000in}{0.150000in}}{\pgfqpoint{2.700000in}{1.950000in}}%
\pgfusepath{clip}%
\pgfsetbuttcap%
\pgfsetroundjoin%
\definecolor{currentfill}{rgb}{0.865135,0.755285,0.763986}%
\pgfsetfillcolor{currentfill}%
\pgfsetlinewidth{0.000000pt}%
\definecolor{currentstroke}{rgb}{0.000000,0.000000,0.000000}%
\pgfsetstrokecolor{currentstroke}%
\pgfsetdash{}{0pt}%
\pgfpathmoveto{\pgfqpoint{1.800215in}{0.885187in}}%
\pgfpathlineto{\pgfqpoint{1.836986in}{0.892734in}}%
\pgfpathlineto{\pgfqpoint{1.798946in}{0.912352in}}%
\pgfpathlineto{\pgfqpoint{1.762127in}{0.904877in}}%
\pgfpathclose%
\pgfusepath{fill}%
\end{pgfscope}%
\begin{pgfscope}%
\pgfpathrectangle{\pgfqpoint{0.150000in}{0.150000in}}{\pgfqpoint{2.700000in}{1.950000in}}%
\pgfusepath{clip}%
\pgfsetbuttcap%
\pgfsetroundjoin%
\definecolor{currentfill}{rgb}{0.849939,0.727711,0.737393}%
\pgfsetfillcolor{currentfill}%
\pgfsetlinewidth{0.000000pt}%
\definecolor{currentstroke}{rgb}{0.000000,0.000000,0.000000}%
\pgfsetstrokecolor{currentstroke}%
\pgfsetdash{}{0pt}%
\pgfpathmoveto{\pgfqpoint{2.139681in}{0.853284in}}%
\pgfpathlineto{\pgfqpoint{2.175457in}{0.860931in}}%
\pgfpathlineto{\pgfqpoint{2.136071in}{0.868533in}}%
\pgfpathlineto{\pgfqpoint{2.100284in}{0.860931in}}%
\pgfpathclose%
\pgfusepath{fill}%
\end{pgfscope}%
\begin{pgfscope}%
\pgfpathrectangle{\pgfqpoint{0.150000in}{0.150000in}}{\pgfqpoint{2.700000in}{1.950000in}}%
\pgfusepath{clip}%
\pgfsetbuttcap%
\pgfsetroundjoin%
\definecolor{currentfill}{rgb}{0.849939,0.727711,0.737393}%
\pgfsetfillcolor{currentfill}%
\pgfsetlinewidth{0.000000pt}%
\definecolor{currentstroke}{rgb}{0.000000,0.000000,0.000000}%
\pgfsetstrokecolor{currentstroke}%
\pgfsetdash{}{0pt}%
\pgfpathmoveto{\pgfqpoint{2.064282in}{0.853284in}}%
\pgfpathlineto{\pgfqpoint{2.100284in}{0.860931in}}%
\pgfpathlineto{\pgfqpoint{2.061123in}{0.868533in}}%
\pgfpathlineto{\pgfqpoint{2.025111in}{0.860931in}}%
\pgfpathclose%
\pgfusepath{fill}%
\end{pgfscope}%
\begin{pgfscope}%
\pgfpathrectangle{\pgfqpoint{0.150000in}{0.150000in}}{\pgfqpoint{2.700000in}{1.950000in}}%
\pgfusepath{clip}%
\pgfsetbuttcap%
\pgfsetroundjoin%
\definecolor{currentfill}{rgb}{0.828968,0.850031,0.879519}%
\pgfsetfillcolor{currentfill}%
\pgfsetlinewidth{0.000000pt}%
\definecolor{currentstroke}{rgb}{0.000000,0.000000,0.000000}%
\pgfsetstrokecolor{currentstroke}%
\pgfsetdash{}{0pt}%
\pgfpathmoveto{\pgfqpoint{0.930133in}{1.448834in}}%
\pgfpathlineto{\pgfqpoint{0.976391in}{1.327566in}}%
\pgfpathlineto{\pgfqpoint{0.939431in}{1.358863in}}%
\pgfpathlineto{\pgfqpoint{0.892651in}{1.480519in}}%
\pgfpathclose%
\pgfusepath{fill}%
\end{pgfscope}%
\begin{pgfscope}%
\pgfpathrectangle{\pgfqpoint{0.150000in}{0.150000in}}{\pgfqpoint{2.700000in}{1.950000in}}%
\pgfusepath{clip}%
\pgfsetbuttcap%
\pgfsetroundjoin%
\definecolor{currentfill}{rgb}{0.853738,0.734605,0.744041}%
\pgfsetfillcolor{currentfill}%
\pgfsetlinewidth{0.000000pt}%
\definecolor{currentstroke}{rgb}{0.000000,0.000000,0.000000}%
\pgfsetstrokecolor{currentstroke}%
\pgfsetdash{}{0pt}%
\pgfpathmoveto{\pgfqpoint{1.989419in}{0.865424in}}%
\pgfpathlineto{\pgfqpoint{2.025111in}{0.860931in}}%
\pgfpathlineto{\pgfqpoint{1.986175in}{0.868533in}}%
\pgfpathlineto{\pgfqpoint{1.950427in}{0.873045in}}%
\pgfpathclose%
\pgfusepath{fill}%
\end{pgfscope}%
\begin{pgfscope}%
\pgfpathrectangle{\pgfqpoint{0.150000in}{0.150000in}}{\pgfqpoint{2.700000in}{1.950000in}}%
\pgfusepath{clip}%
\pgfsetbuttcap%
\pgfsetroundjoin%
\definecolor{currentfill}{rgb}{0.982904,0.968980,0.970083}%
\pgfsetfillcolor{currentfill}%
\pgfsetlinewidth{0.000000pt}%
\definecolor{currentstroke}{rgb}{0.000000,0.000000,0.000000}%
\pgfsetstrokecolor{currentstroke}%
\pgfsetdash{}{0pt}%
\pgfpathmoveto{\pgfqpoint{1.272213in}{1.137785in}}%
\pgfpathlineto{\pgfqpoint{1.311180in}{1.119744in}}%
\pgfpathlineto{\pgfqpoint{1.274106in}{1.138944in}}%
\pgfpathlineto{\pgfqpoint{1.235011in}{1.157055in}}%
\pgfpathclose%
\pgfusepath{fill}%
\end{pgfscope}%
\begin{pgfscope}%
\pgfpathrectangle{\pgfqpoint{0.150000in}{0.150000in}}{\pgfqpoint{2.700000in}{1.950000in}}%
\pgfusepath{clip}%
\pgfsetbuttcap%
\pgfsetroundjoin%
\definecolor{currentfill}{rgb}{0.948713,0.906939,0.910248}%
\pgfsetfillcolor{currentfill}%
\pgfsetlinewidth{0.000000pt}%
\definecolor{currentstroke}{rgb}{0.000000,0.000000,0.000000}%
\pgfsetstrokecolor{currentstroke}%
\pgfsetdash{}{0pt}%
\pgfpathmoveto{\pgfqpoint{1.423349in}{1.049347in}}%
\pgfpathlineto{\pgfqpoint{1.461288in}{1.056409in}}%
\pgfpathlineto{\pgfqpoint{1.423894in}{1.075748in}}%
\pgfpathlineto{\pgfqpoint{1.385912in}{1.068757in}}%
\pgfpathclose%
\pgfusepath{fill}%
\end{pgfscope}%
\begin{pgfscope}%
\pgfpathrectangle{\pgfqpoint{0.150000in}{0.150000in}}{\pgfqpoint{2.700000in}{1.950000in}}%
\pgfusepath{clip}%
\pgfsetbuttcap%
\pgfsetroundjoin%
\definecolor{currentfill}{rgb}{0.929718,0.872472,0.877007}%
\pgfsetfillcolor{currentfill}%
\pgfsetlinewidth{0.000000pt}%
\definecolor{currentstroke}{rgb}{0.000000,0.000000,0.000000}%
\pgfsetstrokecolor{currentstroke}%
\pgfsetdash{}{0pt}%
\pgfpathmoveto{\pgfqpoint{1.536486in}{1.005200in}}%
\pgfpathlineto{\pgfqpoint{1.574065in}{1.012390in}}%
\pgfpathlineto{\pgfqpoint{1.536486in}{1.044089in}}%
\pgfpathlineto{\pgfqpoint{1.498819in}{1.036999in}}%
\pgfpathclose%
\pgfusepath{fill}%
\end{pgfscope}%
\begin{pgfscope}%
\pgfpathrectangle{\pgfqpoint{0.150000in}{0.150000in}}{\pgfqpoint{2.700000in}{1.950000in}}%
\pgfusepath{clip}%
\pgfsetbuttcap%
\pgfsetroundjoin%
\definecolor{currentfill}{rgb}{0.895527,0.810432,0.817172}%
\pgfsetfillcolor{currentfill}%
\pgfsetlinewidth{0.000000pt}%
\definecolor{currentstroke}{rgb}{0.000000,0.000000,0.000000}%
\pgfsetstrokecolor{currentstroke}%
\pgfsetdash{}{0pt}%
\pgfpathmoveto{\pgfqpoint{1.687270in}{0.929249in}}%
\pgfpathlineto{\pgfqpoint{1.724400in}{0.936669in}}%
\pgfpathlineto{\pgfqpoint{1.686721in}{0.968419in}}%
\pgfpathlineto{\pgfqpoint{1.649501in}{0.961101in}}%
\pgfpathclose%
\pgfusepath{fill}%
\end{pgfscope}%
\begin{pgfscope}%
\pgfpathrectangle{\pgfqpoint{0.150000in}{0.150000in}}{\pgfqpoint{2.700000in}{1.950000in}}%
\pgfusepath{clip}%
\pgfsetbuttcap%
\pgfsetroundjoin%
\definecolor{currentfill}{rgb}{0.861336,0.748392,0.757338}%
\pgfsetfillcolor{currentfill}%
\pgfsetlinewidth{0.000000pt}%
\definecolor{currentstroke}{rgb}{0.000000,0.000000,0.000000}%
\pgfsetstrokecolor{currentstroke}%
\pgfsetdash{}{0pt}%
\pgfpathmoveto{\pgfqpoint{1.913930in}{0.865424in}}%
\pgfpathlineto{\pgfqpoint{1.950427in}{0.873045in}}%
\pgfpathlineto{\pgfqpoint{1.911668in}{0.880619in}}%
\pgfpathlineto{\pgfqpoint{1.875165in}{0.873045in}}%
\pgfpathclose%
\pgfusepath{fill}%
\end{pgfscope}%
\begin{pgfscope}%
\pgfpathrectangle{\pgfqpoint{0.150000in}{0.150000in}}{\pgfqpoint{2.700000in}{1.950000in}}%
\pgfusepath{clip}%
\pgfsetbuttcap%
\pgfsetroundjoin%
\definecolor{currentfill}{rgb}{0.841406,0.860938,0.888281}%
\pgfsetfillcolor{currentfill}%
\pgfsetlinewidth{0.000000pt}%
\definecolor{currentstroke}{rgb}{0.000000,0.000000,0.000000}%
\pgfsetstrokecolor{currentstroke}%
\pgfsetdash{}{0pt}%
\pgfpathmoveto{\pgfqpoint{0.967662in}{1.417107in}}%
\pgfpathlineto{\pgfqpoint{1.012783in}{1.308782in}}%
\pgfpathlineto{\pgfqpoint{0.976391in}{1.327566in}}%
\pgfpathlineto{\pgfqpoint{0.930133in}{1.448834in}}%
\pgfpathclose%
\pgfusepath{fill}%
\end{pgfscope}%
\begin{pgfscope}%
\pgfpathrectangle{\pgfqpoint{0.150000in}{0.150000in}}{\pgfqpoint{2.700000in}{1.950000in}}%
\pgfusepath{clip}%
\pgfsetbuttcap%
\pgfsetroundjoin%
\definecolor{currentfill}{rgb}{0.922120,0.858686,0.863710}%
\pgfsetfillcolor{currentfill}%
\pgfsetlinewidth{0.000000pt}%
\definecolor{currentstroke}{rgb}{0.000000,0.000000,0.000000}%
\pgfsetstrokecolor{currentstroke}%
\pgfsetdash{}{0pt}%
\pgfpathmoveto{\pgfqpoint{1.574247in}{0.985649in}}%
\pgfpathlineto{\pgfqpoint{1.611692in}{0.980651in}}%
\pgfpathlineto{\pgfqpoint{1.574065in}{1.012390in}}%
\pgfpathlineto{\pgfqpoint{1.536486in}{1.005200in}}%
\pgfpathclose%
\pgfusepath{fill}%
\end{pgfscope}%
\begin{pgfscope}%
\pgfpathrectangle{\pgfqpoint{0.150000in}{0.150000in}}{\pgfqpoint{2.700000in}{1.950000in}}%
\pgfusepath{clip}%
\pgfsetbuttcap%
\pgfsetroundjoin%
\definecolor{currentfill}{rgb}{0.978232,0.980913,0.984666}%
\pgfsetfillcolor{currentfill}%
\pgfsetlinewidth{0.000000pt}%
\definecolor{currentstroke}{rgb}{0.000000,0.000000,0.000000}%
\pgfsetstrokecolor{currentstroke}%
\pgfsetdash{}{0pt}%
\pgfpathmoveto{\pgfqpoint{0.632061in}{1.157055in}}%
\pgfpathlineto{\pgfqpoint{0.669276in}{1.201275in}}%
\pgfpathlineto{\pgfqpoint{0.633222in}{1.220405in}}%
\pgfpathlineto{\pgfqpoint{0.596094in}{1.176254in}}%
\pgfpathclose%
\pgfusepath{fill}%
\end{pgfscope}%
\begin{pgfscope}%
\pgfpathrectangle{\pgfqpoint{0.150000in}{0.150000in}}{\pgfqpoint{2.700000in}{1.950000in}}%
\pgfusepath{clip}%
\pgfsetbuttcap%
\pgfsetroundjoin%
\definecolor{currentfill}{rgb}{0.887929,0.796645,0.803876}%
\pgfsetfillcolor{currentfill}%
\pgfsetlinewidth{0.000000pt}%
\definecolor{currentstroke}{rgb}{0.000000,0.000000,0.000000}%
\pgfsetstrokecolor{currentstroke}%
\pgfsetdash{}{0pt}%
\pgfpathmoveto{\pgfqpoint{1.725311in}{0.909557in}}%
\pgfpathlineto{\pgfqpoint{1.762127in}{0.904877in}}%
\pgfpathlineto{\pgfqpoint{1.724400in}{0.936669in}}%
\pgfpathlineto{\pgfqpoint{1.687270in}{0.929249in}}%
\pgfpathclose%
\pgfusepath{fill}%
\end{pgfscope}%
\begin{pgfscope}%
\pgfpathrectangle{\pgfqpoint{0.150000in}{0.150000in}}{\pgfqpoint{2.700000in}{1.950000in}}%
\pgfusepath{clip}%
\pgfsetbuttcap%
\pgfsetroundjoin%
\definecolor{currentfill}{rgb}{0.868934,0.762178,0.770634}%
\pgfsetfillcolor{currentfill}%
\pgfsetlinewidth{0.000000pt}%
\definecolor{currentstroke}{rgb}{0.000000,0.000000,0.000000}%
\pgfsetstrokecolor{currentstroke}%
\pgfsetdash{}{0pt}%
\pgfpathmoveto{\pgfqpoint{1.838800in}{0.877594in}}%
\pgfpathlineto{\pgfqpoint{1.875165in}{0.873045in}}%
\pgfpathlineto{\pgfqpoint{1.836986in}{0.892734in}}%
\pgfpathlineto{\pgfqpoint{1.800215in}{0.885187in}}%
\pgfpathclose%
\pgfusepath{fill}%
\end{pgfscope}%
\begin{pgfscope}%
\pgfpathrectangle{\pgfqpoint{0.150000in}{0.150000in}}{\pgfqpoint{2.700000in}{1.950000in}}%
\pgfusepath{clip}%
\pgfsetbuttcap%
\pgfsetroundjoin%
\definecolor{currentfill}{rgb}{0.979105,0.962086,0.963434}%
\pgfsetfillcolor{currentfill}%
\pgfsetlinewidth{0.000000pt}%
\definecolor{currentstroke}{rgb}{0.000000,0.000000,0.000000}%
\pgfsetstrokecolor{currentstroke}%
\pgfsetdash{}{0pt}%
\pgfpathmoveto{\pgfqpoint{1.309821in}{1.105979in}}%
\pgfpathlineto{\pgfqpoint{1.348389in}{1.100475in}}%
\pgfpathlineto{\pgfqpoint{1.311180in}{1.119744in}}%
\pgfpathlineto{\pgfqpoint{1.272213in}{1.137785in}}%
\pgfpathclose%
\pgfusepath{fill}%
\end{pgfscope}%
\begin{pgfscope}%
\pgfpathrectangle{\pgfqpoint{0.150000in}{0.150000in}}{\pgfqpoint{2.700000in}{1.950000in}}%
\pgfusepath{clip}%
\pgfsetbuttcap%
\pgfsetroundjoin%
\definecolor{currentfill}{rgb}{0.860064,0.877298,0.901425}%
\pgfsetfillcolor{currentfill}%
\pgfsetlinewidth{0.000000pt}%
\definecolor{currentstroke}{rgb}{0.000000,0.000000,0.000000}%
\pgfsetstrokecolor{currentstroke}%
\pgfsetdash{}{0pt}%
\pgfpathmoveto{\pgfqpoint{1.005240in}{1.385340in}}%
\pgfpathlineto{\pgfqpoint{1.049880in}{1.277376in}}%
\pgfpathlineto{\pgfqpoint{1.012783in}{1.308782in}}%
\pgfpathlineto{\pgfqpoint{0.967662in}{1.417107in}}%
\pgfpathclose%
\pgfusepath{fill}%
\end{pgfscope}%
\begin{pgfscope}%
\pgfpathrectangle{\pgfqpoint{0.150000in}{0.150000in}}{\pgfqpoint{2.700000in}{1.950000in}}%
\pgfusepath{clip}%
\pgfsetbuttcap%
\pgfsetroundjoin%
\definecolor{currentfill}{rgb}{0.948713,0.906939,0.910248}%
\pgfsetfillcolor{currentfill}%
\pgfsetlinewidth{0.000000pt}%
\definecolor{currentstroke}{rgb}{0.000000,0.000000,0.000000}%
\pgfsetstrokecolor{currentstroke}%
\pgfsetdash{}{0pt}%
\pgfpathmoveto{\pgfqpoint{1.460924in}{1.029866in}}%
\pgfpathlineto{\pgfqpoint{1.498819in}{1.036999in}}%
\pgfpathlineto{\pgfqpoint{1.461288in}{1.056409in}}%
\pgfpathlineto{\pgfqpoint{1.423349in}{1.049347in}}%
\pgfpathclose%
\pgfusepath{fill}%
\end{pgfscope}%
\begin{pgfscope}%
\pgfpathrectangle{\pgfqpoint{0.150000in}{0.150000in}}{\pgfqpoint{2.700000in}{1.950000in}}%
\pgfusepath{clip}%
\pgfsetbuttcap%
\pgfsetroundjoin%
\definecolor{currentfill}{rgb}{0.940916,0.948192,0.958379}%
\pgfsetfillcolor{currentfill}%
\pgfsetlinewidth{0.000000pt}%
\definecolor{currentstroke}{rgb}{0.000000,0.000000,0.000000}%
\pgfsetstrokecolor{currentstroke}%
\pgfsetdash{}{0pt}%
\pgfpathmoveto{\pgfqpoint{0.669276in}{1.201275in}}%
\pgfpathlineto{\pgfqpoint{0.706532in}{1.245544in}}%
\pgfpathlineto{\pgfqpoint{0.669364in}{1.277216in}}%
\pgfpathlineto{\pgfqpoint{0.633222in}{1.220405in}}%
\pgfpathclose%
\pgfusepath{fill}%
\end{pgfscope}%
\begin{pgfscope}%
\pgfpathrectangle{\pgfqpoint{0.150000in}{0.150000in}}{\pgfqpoint{2.700000in}{1.950000in}}%
\pgfusepath{clip}%
\pgfsetbuttcap%
\pgfsetroundjoin%
\definecolor{currentfill}{rgb}{0.861336,0.748392,0.757338}%
\pgfsetfillcolor{currentfill}%
\pgfsetlinewidth{0.000000pt}%
\definecolor{currentstroke}{rgb}{0.000000,0.000000,0.000000}%
\pgfsetstrokecolor{currentstroke}%
\pgfsetdash{}{0pt}%
\pgfpathmoveto{\pgfqpoint{2.103689in}{0.845590in}}%
\pgfpathlineto{\pgfqpoint{2.139681in}{0.853284in}}%
\pgfpathlineto{\pgfqpoint{2.100284in}{0.860931in}}%
\pgfpathlineto{\pgfqpoint{2.064282in}{0.853284in}}%
\pgfpathclose%
\pgfusepath{fill}%
\end{pgfscope}%
\begin{pgfscope}%
\pgfpathrectangle{\pgfqpoint{0.150000in}{0.150000in}}{\pgfqpoint{2.700000in}{1.950000in}}%
\pgfusepath{clip}%
\pgfsetbuttcap%
\pgfsetroundjoin%
\definecolor{currentfill}{rgb}{0.914522,0.844899,0.850414}%
\pgfsetfillcolor{currentfill}%
\pgfsetlinewidth{0.000000pt}%
\definecolor{currentstroke}{rgb}{0.000000,0.000000,0.000000}%
\pgfsetstrokecolor{currentstroke}%
\pgfsetdash{}{0pt}%
\pgfpathmoveto{\pgfqpoint{1.612057in}{0.953738in}}%
\pgfpathlineto{\pgfqpoint{1.649501in}{0.961101in}}%
\pgfpathlineto{\pgfqpoint{1.611692in}{0.980651in}}%
\pgfpathlineto{\pgfqpoint{1.574247in}{0.985649in}}%
\pgfpathclose%
\pgfusepath{fill}%
\end{pgfscope}%
\begin{pgfscope}%
\pgfpathrectangle{\pgfqpoint{0.150000in}{0.150000in}}{\pgfqpoint{2.700000in}{1.950000in}}%
\pgfusepath{clip}%
\pgfsetbuttcap%
\pgfsetroundjoin%
\definecolor{currentfill}{rgb}{0.865135,0.755285,0.763986}%
\pgfsetfillcolor{currentfill}%
\pgfsetlinewidth{0.000000pt}%
\definecolor{currentstroke}{rgb}{0.000000,0.000000,0.000000}%
\pgfsetstrokecolor{currentstroke}%
\pgfsetdash{}{0pt}%
\pgfpathmoveto{\pgfqpoint{2.028062in}{0.845590in}}%
\pgfpathlineto{\pgfqpoint{2.064282in}{0.853284in}}%
\pgfpathlineto{\pgfqpoint{2.025111in}{0.860931in}}%
\pgfpathlineto{\pgfqpoint{1.989419in}{0.865424in}}%
\pgfpathclose%
\pgfusepath{fill}%
\end{pgfscope}%
\begin{pgfscope}%
\pgfpathrectangle{\pgfqpoint{0.150000in}{0.150000in}}{\pgfqpoint{2.700000in}{1.950000in}}%
\pgfusepath{clip}%
\pgfsetbuttcap%
\pgfsetroundjoin%
\definecolor{currentfill}{rgb}{0.872503,0.888205,0.910187}%
\pgfsetfillcolor{currentfill}%
\pgfsetlinewidth{0.000000pt}%
\definecolor{currentstroke}{rgb}{0.000000,0.000000,0.000000}%
\pgfsetstrokecolor{currentstroke}%
\pgfsetdash{}{0pt}%
\pgfpathmoveto{\pgfqpoint{1.042867in}{1.353531in}}%
\pgfpathlineto{\pgfqpoint{1.086495in}{1.258454in}}%
\pgfpathlineto{\pgfqpoint{1.049880in}{1.277376in}}%
\pgfpathlineto{\pgfqpoint{1.005240in}{1.385340in}}%
\pgfpathclose%
\pgfusepath{fill}%
\end{pgfscope}%
\begin{pgfscope}%
\pgfpathrectangle{\pgfqpoint{0.150000in}{0.150000in}}{\pgfqpoint{2.700000in}{1.950000in}}%
\pgfusepath{clip}%
\pgfsetbuttcap%
\pgfsetroundjoin%
\definecolor{currentfill}{rgb}{0.884130,0.789752,0.797227}%
\pgfsetfillcolor{currentfill}%
\pgfsetlinewidth{0.000000pt}%
\definecolor{currentstroke}{rgb}{0.000000,0.000000,0.000000}%
\pgfsetstrokecolor{currentstroke}%
\pgfsetdash{}{0pt}%
\pgfpathmoveto{\pgfqpoint{1.763491in}{0.889792in}}%
\pgfpathlineto{\pgfqpoint{1.800215in}{0.885187in}}%
\pgfpathlineto{\pgfqpoint{1.762127in}{0.904877in}}%
\pgfpathlineto{\pgfqpoint{1.725311in}{0.909557in}}%
\pgfpathclose%
\pgfusepath{fill}%
\end{pgfscope}%
\begin{pgfscope}%
\pgfpathrectangle{\pgfqpoint{0.150000in}{0.150000in}}{\pgfqpoint{2.700000in}{1.950000in}}%
\pgfusepath{clip}%
\pgfsetbuttcap%
\pgfsetroundjoin%
\definecolor{currentfill}{rgb}{0.975306,0.955193,0.956786}%
\pgfsetfillcolor{currentfill}%
\pgfsetlinewidth{0.000000pt}%
\definecolor{currentstroke}{rgb}{0.000000,0.000000,0.000000}%
\pgfsetstrokecolor{currentstroke}%
\pgfsetdash{}{0pt}%
\pgfpathmoveto{\pgfqpoint{1.347252in}{1.086568in}}%
\pgfpathlineto{\pgfqpoint{1.385912in}{1.068757in}}%
\pgfpathlineto{\pgfqpoint{1.348389in}{1.100475in}}%
\pgfpathlineto{\pgfqpoint{1.309821in}{1.105979in}}%
\pgfpathclose%
\pgfusepath{fill}%
\end{pgfscope}%
\begin{pgfscope}%
\pgfpathrectangle{\pgfqpoint{0.150000in}{0.150000in}}{\pgfqpoint{2.700000in}{1.950000in}}%
\pgfusepath{clip}%
\pgfsetbuttcap%
\pgfsetroundjoin%
\definecolor{currentfill}{rgb}{0.978232,0.980913,0.984666}%
\pgfsetfillcolor{currentfill}%
\pgfsetlinewidth{0.000000pt}%
\definecolor{currentstroke}{rgb}{0.000000,0.000000,0.000000}%
\pgfsetstrokecolor{currentstroke}%
\pgfsetdash{}{0pt}%
\pgfpathmoveto{\pgfqpoint{0.668160in}{1.137785in}}%
\pgfpathlineto{\pgfqpoint{0.705463in}{1.182076in}}%
\pgfpathlineto{\pgfqpoint{0.669276in}{1.201275in}}%
\pgfpathlineto{\pgfqpoint{0.632061in}{1.157055in}}%
\pgfpathclose%
\pgfusepath{fill}%
\end{pgfscope}%
\begin{pgfscope}%
\pgfpathrectangle{\pgfqpoint{0.150000in}{0.150000in}}{\pgfqpoint{2.700000in}{1.950000in}}%
\pgfusepath{clip}%
\pgfsetbuttcap%
\pgfsetroundjoin%
\definecolor{currentfill}{rgb}{0.891161,0.904565,0.923330}%
\pgfsetfillcolor{currentfill}%
\pgfsetlinewidth{0.000000pt}%
\definecolor{currentstroke}{rgb}{0.000000,0.000000,0.000000}%
\pgfsetstrokecolor{currentstroke}%
\pgfsetdash{}{0pt}%
\pgfpathmoveto{\pgfqpoint{0.706532in}{1.245544in}}%
\pgfpathlineto{\pgfqpoint{0.742887in}{1.302534in}}%
\pgfpathlineto{\pgfqpoint{0.706616in}{1.321521in}}%
\pgfpathlineto{\pgfqpoint{0.669364in}{1.277216in}}%
\pgfpathclose%
\pgfusepath{fill}%
\end{pgfscope}%
\begin{pgfscope}%
\pgfpathrectangle{\pgfqpoint{0.150000in}{0.150000in}}{\pgfqpoint{2.700000in}{1.950000in}}%
\pgfusepath{clip}%
\pgfsetbuttcap%
\pgfsetroundjoin%
\definecolor{currentfill}{rgb}{0.941115,0.893153,0.896952}%
\pgfsetfillcolor{currentfill}%
\pgfsetlinewidth{0.000000pt}%
\definecolor{currentstroke}{rgb}{0.000000,0.000000,0.000000}%
\pgfsetstrokecolor{currentstroke}%
\pgfsetdash{}{0pt}%
\pgfpathmoveto{\pgfqpoint{1.498681in}{0.997966in}}%
\pgfpathlineto{\pgfqpoint{1.536486in}{1.005200in}}%
\pgfpathlineto{\pgfqpoint{1.498819in}{1.036999in}}%
\pgfpathlineto{\pgfqpoint{1.460924in}{1.029866in}}%
\pgfpathclose%
\pgfusepath{fill}%
\end{pgfscope}%
\begin{pgfscope}%
\pgfpathrectangle{\pgfqpoint{0.150000in}{0.150000in}}{\pgfqpoint{2.700000in}{1.950000in}}%
\pgfusepath{clip}%
\pgfsetbuttcap%
\pgfsetroundjoin%
\definecolor{currentfill}{rgb}{0.872733,0.769072,0.777282}%
\pgfsetfillcolor{currentfill}%
\pgfsetlinewidth{0.000000pt}%
\definecolor{currentstroke}{rgb}{0.000000,0.000000,0.000000}%
\pgfsetstrokecolor{currentstroke}%
\pgfsetdash{}{0pt}%
\pgfpathmoveto{\pgfqpoint{1.952929in}{0.857758in}}%
\pgfpathlineto{\pgfqpoint{1.989419in}{0.865424in}}%
\pgfpathlineto{\pgfqpoint{1.950427in}{0.873045in}}%
\pgfpathlineto{\pgfqpoint{1.913930in}{0.865424in}}%
\pgfpathclose%
\pgfusepath{fill}%
\end{pgfscope}%
\begin{pgfscope}%
\pgfpathrectangle{\pgfqpoint{0.150000in}{0.150000in}}{\pgfqpoint{2.700000in}{1.950000in}}%
\pgfusepath{clip}%
\pgfsetbuttcap%
\pgfsetroundjoin%
\definecolor{currentfill}{rgb}{0.884942,0.899112,0.918949}%
\pgfsetfillcolor{currentfill}%
\pgfsetlinewidth{0.000000pt}%
\definecolor{currentstroke}{rgb}{0.000000,0.000000,0.000000}%
\pgfsetstrokecolor{currentstroke}%
\pgfsetdash{}{0pt}%
\pgfpathmoveto{\pgfqpoint{1.080543in}{1.321681in}}%
\pgfpathlineto{\pgfqpoint{1.123731in}{1.226939in}}%
\pgfpathlineto{\pgfqpoint{1.086495in}{1.258454in}}%
\pgfpathlineto{\pgfqpoint{1.042867in}{1.353531in}}%
\pgfpathclose%
\pgfusepath{fill}%
\end{pgfscope}%
\begin{pgfscope}%
\pgfpathrectangle{\pgfqpoint{0.150000in}{0.150000in}}{\pgfqpoint{2.700000in}{1.950000in}}%
\pgfusepath{clip}%
\pgfsetbuttcap%
\pgfsetroundjoin%
\definecolor{currentfill}{rgb}{0.940916,0.948192,0.958379}%
\pgfsetfillcolor{currentfill}%
\pgfsetlinewidth{0.000000pt}%
\definecolor{currentstroke}{rgb}{0.000000,0.000000,0.000000}%
\pgfsetstrokecolor{currentstroke}%
\pgfsetdash{}{0pt}%
\pgfpathmoveto{\pgfqpoint{0.705463in}{1.182076in}}%
\pgfpathlineto{\pgfqpoint{0.742806in}{1.226415in}}%
\pgfpathlineto{\pgfqpoint{0.706532in}{1.245544in}}%
\pgfpathlineto{\pgfqpoint{0.669276in}{1.201275in}}%
\pgfpathclose%
\pgfusepath{fill}%
\end{pgfscope}%
\begin{pgfscope}%
\pgfpathrectangle{\pgfqpoint{0.150000in}{0.150000in}}{\pgfqpoint{2.700000in}{1.950000in}}%
\pgfusepath{clip}%
\pgfsetbuttcap%
\pgfsetroundjoin%
\definecolor{currentfill}{rgb}{0.910723,0.838006,0.843765}%
\pgfsetfillcolor{currentfill}%
\pgfsetlinewidth{0.000000pt}%
\definecolor{currentstroke}{rgb}{0.000000,0.000000,0.000000}%
\pgfsetstrokecolor{currentstroke}%
\pgfsetdash{}{0pt}%
\pgfpathmoveto{\pgfqpoint{1.650050in}{0.934043in}}%
\pgfpathlineto{\pgfqpoint{1.687270in}{0.929249in}}%
\pgfpathlineto{\pgfqpoint{1.649501in}{0.961101in}}%
\pgfpathlineto{\pgfqpoint{1.612057in}{0.953738in}}%
\pgfpathclose%
\pgfusepath{fill}%
\end{pgfscope}%
\begin{pgfscope}%
\pgfpathrectangle{\pgfqpoint{0.150000in}{0.150000in}}{\pgfqpoint{2.700000in}{1.950000in}}%
\pgfusepath{clip}%
\pgfsetbuttcap%
\pgfsetroundjoin%
\definecolor{currentfill}{rgb}{0.876532,0.775965,0.783931}%
\pgfsetfillcolor{currentfill}%
\pgfsetlinewidth{0.000000pt}%
\definecolor{currentstroke}{rgb}{0.000000,0.000000,0.000000}%
\pgfsetstrokecolor{currentstroke}%
\pgfsetdash{}{0pt}%
\pgfpathmoveto{\pgfqpoint{1.877212in}{0.857758in}}%
\pgfpathlineto{\pgfqpoint{1.913930in}{0.865424in}}%
\pgfpathlineto{\pgfqpoint{1.875165in}{0.873045in}}%
\pgfpathlineto{\pgfqpoint{1.838800in}{0.877594in}}%
\pgfpathclose%
\pgfusepath{fill}%
\end{pgfscope}%
\begin{pgfscope}%
\pgfpathrectangle{\pgfqpoint{0.150000in}{0.150000in}}{\pgfqpoint{2.700000in}{1.950000in}}%
\pgfusepath{clip}%
\pgfsetbuttcap%
\pgfsetroundjoin%
\definecolor{currentfill}{rgb}{0.990671,0.991820,0.993428}%
\pgfsetfillcolor{currentfill}%
\pgfsetlinewidth{0.000000pt}%
\definecolor{currentstroke}{rgb}{0.000000,0.000000,0.000000}%
\pgfsetstrokecolor{currentstroke}%
\pgfsetdash{}{0pt}%
\pgfpathmoveto{\pgfqpoint{0.705379in}{1.105979in}}%
\pgfpathlineto{\pgfqpoint{0.742726in}{1.150281in}}%
\pgfpathlineto{\pgfqpoint{0.705463in}{1.182076in}}%
\pgfpathlineto{\pgfqpoint{0.668160in}{1.137785in}}%
\pgfpathclose%
\pgfusepath{fill}%
\end{pgfscope}%
\begin{pgfscope}%
\pgfpathrectangle{\pgfqpoint{0.150000in}{0.150000in}}{\pgfqpoint{2.700000in}{1.950000in}}%
\pgfusepath{clip}%
\pgfsetbuttcap%
\pgfsetroundjoin%
\definecolor{currentfill}{rgb}{0.971507,0.948300,0.950138}%
\pgfsetfillcolor{currentfill}%
\pgfsetlinewidth{0.000000pt}%
\definecolor{currentstroke}{rgb}{0.000000,0.000000,0.000000}%
\pgfsetstrokecolor{currentstroke}%
\pgfsetdash{}{0pt}%
\pgfpathmoveto{\pgfqpoint{1.384821in}{1.067085in}}%
\pgfpathlineto{\pgfqpoint{1.423349in}{1.049347in}}%
\pgfpathlineto{\pgfqpoint{1.385912in}{1.068757in}}%
\pgfpathlineto{\pgfqpoint{1.347252in}{1.086568in}}%
\pgfpathclose%
\pgfusepath{fill}%
\end{pgfscope}%
\begin{pgfscope}%
\pgfpathrectangle{\pgfqpoint{0.150000in}{0.150000in}}{\pgfqpoint{2.700000in}{1.950000in}}%
\pgfusepath{clip}%
\pgfsetbuttcap%
\pgfsetroundjoin%
\definecolor{currentfill}{rgb}{0.897381,0.910018,0.927711}%
\pgfsetfillcolor{currentfill}%
\pgfsetlinewidth{0.000000pt}%
\definecolor{currentstroke}{rgb}{0.000000,0.000000,0.000000}%
\pgfsetstrokecolor{currentstroke}%
\pgfsetdash{}{0pt}%
\pgfpathmoveto{\pgfqpoint{0.742806in}{1.226415in}}%
\pgfpathlineto{\pgfqpoint{0.780190in}{1.270802in}}%
\pgfpathlineto{\pgfqpoint{0.742887in}{1.302534in}}%
\pgfpathlineto{\pgfqpoint{0.706532in}{1.245544in}}%
\pgfpathclose%
\pgfusepath{fill}%
\end{pgfscope}%
\begin{pgfscope}%
\pgfpathrectangle{\pgfqpoint{0.150000in}{0.150000in}}{\pgfqpoint{2.700000in}{1.950000in}}%
\pgfusepath{clip}%
\pgfsetbuttcap%
\pgfsetroundjoin%
\definecolor{currentfill}{rgb}{0.897381,0.910018,0.927711}%
\pgfsetfillcolor{currentfill}%
\pgfsetlinewidth{0.000000pt}%
\definecolor{currentstroke}{rgb}{0.000000,0.000000,0.000000}%
\pgfsetstrokecolor{currentstroke}%
\pgfsetdash{}{0pt}%
\pgfpathmoveto{\pgfqpoint{1.118267in}{1.289790in}}%
\pgfpathlineto{\pgfqpoint{1.160570in}{1.207879in}}%
\pgfpathlineto{\pgfqpoint{1.123731in}{1.226939in}}%
\pgfpathlineto{\pgfqpoint{1.080543in}{1.321681in}}%
\pgfpathclose%
\pgfusepath{fill}%
\end{pgfscope}%
\begin{pgfscope}%
\pgfpathrectangle{\pgfqpoint{0.150000in}{0.150000in}}{\pgfqpoint{2.700000in}{1.950000in}}%
\pgfusepath{clip}%
\pgfsetbuttcap%
\pgfsetroundjoin%
\definecolor{currentfill}{rgb}{0.937316,0.886259,0.890303}%
\pgfsetfillcolor{currentfill}%
\pgfsetlinewidth{0.000000pt}%
\definecolor{currentstroke}{rgb}{0.000000,0.000000,0.000000}%
\pgfsetstrokecolor{currentstroke}%
\pgfsetdash{}{0pt}%
\pgfpathmoveto{\pgfqpoint{1.536486in}{0.978342in}}%
\pgfpathlineto{\pgfqpoint{1.574247in}{0.985649in}}%
\pgfpathlineto{\pgfqpoint{1.536486in}{1.005200in}}%
\pgfpathlineto{\pgfqpoint{1.498681in}{0.997966in}}%
\pgfpathclose%
\pgfusepath{fill}%
\end{pgfscope}%
\begin{pgfscope}%
\pgfpathrectangle{\pgfqpoint{0.150000in}{0.150000in}}{\pgfqpoint{2.700000in}{1.950000in}}%
\pgfusepath{clip}%
\pgfsetbuttcap%
\pgfsetroundjoin%
\definecolor{currentfill}{rgb}{0.953355,0.959099,0.967142}%
\pgfsetfillcolor{currentfill}%
\pgfsetlinewidth{0.000000pt}%
\definecolor{currentstroke}{rgb}{0.000000,0.000000,0.000000}%
\pgfsetstrokecolor{currentstroke}%
\pgfsetdash{}{0pt}%
\pgfpathmoveto{\pgfqpoint{0.742726in}{1.150281in}}%
\pgfpathlineto{\pgfqpoint{0.780113in}{1.194631in}}%
\pgfpathlineto{\pgfqpoint{0.742806in}{1.226415in}}%
\pgfpathlineto{\pgfqpoint{0.705463in}{1.182076in}}%
\pgfpathclose%
\pgfusepath{fill}%
\end{pgfscope}%
\begin{pgfscope}%
\pgfpathrectangle{\pgfqpoint{0.150000in}{0.150000in}}{\pgfqpoint{2.700000in}{1.950000in}}%
\pgfusepath{clip}%
\pgfsetbuttcap%
\pgfsetroundjoin%
\definecolor{currentfill}{rgb}{0.887929,0.796645,0.803876}%
\pgfsetfillcolor{currentfill}%
\pgfsetlinewidth{0.000000pt}%
\definecolor{currentstroke}{rgb}{0.000000,0.000000,0.000000}%
\pgfsetstrokecolor{currentstroke}%
\pgfsetdash{}{0pt}%
\pgfpathmoveto{\pgfqpoint{1.801811in}{0.869955in}}%
\pgfpathlineto{\pgfqpoint{1.838800in}{0.877594in}}%
\pgfpathlineto{\pgfqpoint{1.800215in}{0.885187in}}%
\pgfpathlineto{\pgfqpoint{1.763491in}{0.889792in}}%
\pgfpathclose%
\pgfusepath{fill}%
\end{pgfscope}%
\begin{pgfscope}%
\pgfpathrectangle{\pgfqpoint{0.150000in}{0.150000in}}{\pgfqpoint{2.700000in}{1.950000in}}%
\pgfusepath{clip}%
\pgfsetbuttcap%
\pgfsetroundjoin%
\definecolor{currentfill}{rgb}{0.906924,0.831112,0.837117}%
\pgfsetfillcolor{currentfill}%
\pgfsetlinewidth{0.000000pt}%
\definecolor{currentstroke}{rgb}{0.000000,0.000000,0.000000}%
\pgfsetstrokecolor{currentstroke}%
\pgfsetdash{}{0pt}%
\pgfpathmoveto{\pgfqpoint{1.688183in}{0.914277in}}%
\pgfpathlineto{\pgfqpoint{1.725311in}{0.909557in}}%
\pgfpathlineto{\pgfqpoint{1.687270in}{0.929249in}}%
\pgfpathlineto{\pgfqpoint{1.650050in}{0.934043in}}%
\pgfpathclose%
\pgfusepath{fill}%
\end{pgfscope}%
\begin{pgfscope}%
\pgfpathrectangle{\pgfqpoint{0.150000in}{0.150000in}}{\pgfqpoint{2.700000in}{1.950000in}}%
\pgfusepath{clip}%
\pgfsetbuttcap%
\pgfsetroundjoin%
\definecolor{currentfill}{rgb}{0.872733,0.769072,0.777282}%
\pgfsetfillcolor{currentfill}%
\pgfsetlinewidth{0.000000pt}%
\definecolor{currentstroke}{rgb}{0.000000,0.000000,0.000000}%
\pgfsetstrokecolor{currentstroke}%
\pgfsetdash{}{0pt}%
\pgfpathmoveto{\pgfqpoint{2.067479in}{0.837849in}}%
\pgfpathlineto{\pgfqpoint{2.103689in}{0.845590in}}%
\pgfpathlineto{\pgfqpoint{2.064282in}{0.853284in}}%
\pgfpathlineto{\pgfqpoint{2.028062in}{0.845590in}}%
\pgfpathclose%
\pgfusepath{fill}%
\end{pgfscope}%
\begin{pgfscope}%
\pgfpathrectangle{\pgfqpoint{0.150000in}{0.150000in}}{\pgfqpoint{2.700000in}{1.950000in}}%
\pgfusepath{clip}%
\pgfsetbuttcap%
\pgfsetroundjoin%
\definecolor{currentfill}{rgb}{0.916039,0.926379,0.940855}%
\pgfsetfillcolor{currentfill}%
\pgfsetlinewidth{0.000000pt}%
\definecolor{currentstroke}{rgb}{0.000000,0.000000,0.000000}%
\pgfsetstrokecolor{currentstroke}%
\pgfsetdash{}{0pt}%
\pgfpathmoveto{\pgfqpoint{0.780113in}{1.194631in}}%
\pgfpathlineto{\pgfqpoint{0.817542in}{1.239030in}}%
\pgfpathlineto{\pgfqpoint{0.780190in}{1.270802in}}%
\pgfpathlineto{\pgfqpoint{0.742806in}{1.226415in}}%
\pgfpathclose%
\pgfusepath{fill}%
\end{pgfscope}%
\begin{pgfscope}%
\pgfpathrectangle{\pgfqpoint{0.150000in}{0.150000in}}{\pgfqpoint{2.700000in}{1.950000in}}%
\pgfusepath{clip}%
\pgfsetbuttcap%
\pgfsetroundjoin%
\definecolor{currentfill}{rgb}{0.822748,0.844577,0.875138}%
\pgfsetfillcolor{currentfill}%
\pgfsetlinewidth{0.000000pt}%
\definecolor{currentstroke}{rgb}{0.000000,0.000000,0.000000}%
\pgfsetstrokecolor{currentstroke}%
\pgfsetdash{}{0pt}%
\pgfpathmoveto{\pgfqpoint{0.742887in}{1.302534in}}%
\pgfpathlineto{\pgfqpoint{0.778467in}{1.372515in}}%
\pgfpathlineto{\pgfqpoint{0.741080in}{1.404253in}}%
\pgfpathlineto{\pgfqpoint{0.706616in}{1.321521in}}%
\pgfpathclose%
\pgfusepath{fill}%
\end{pgfscope}%
\begin{pgfscope}%
\pgfpathrectangle{\pgfqpoint{0.150000in}{0.150000in}}{\pgfqpoint{2.700000in}{1.950000in}}%
\pgfusepath{clip}%
\pgfsetbuttcap%
\pgfsetroundjoin%
\definecolor{currentfill}{rgb}{0.996890,0.997273,0.997809}%
\pgfsetfillcolor{currentfill}%
\pgfsetlinewidth{0.000000pt}%
\definecolor{currentstroke}{rgb}{0.000000,0.000000,0.000000}%
\pgfsetstrokecolor{currentstroke}%
\pgfsetdash{}{0pt}%
\pgfpathmoveto{\pgfqpoint{0.741702in}{1.086568in}}%
\pgfpathlineto{\pgfqpoint{0.779137in}{1.130940in}}%
\pgfpathlineto{\pgfqpoint{0.742726in}{1.150281in}}%
\pgfpathlineto{\pgfqpoint{0.705379in}{1.105979in}}%
\pgfpathclose%
\pgfusepath{fill}%
\end{pgfscope}%
\begin{pgfscope}%
\pgfpathrectangle{\pgfqpoint{0.150000in}{0.150000in}}{\pgfqpoint{2.700000in}{1.950000in}}%
\pgfusepath{clip}%
\pgfsetbuttcap%
\pgfsetroundjoin%
\definecolor{currentfill}{rgb}{0.909819,0.920925,0.936474}%
\pgfsetfillcolor{currentfill}%
\pgfsetlinewidth{0.000000pt}%
\definecolor{currentstroke}{rgb}{0.000000,0.000000,0.000000}%
\pgfsetstrokecolor{currentstroke}%
\pgfsetdash{}{0pt}%
\pgfpathmoveto{\pgfqpoint{1.156041in}{1.257857in}}%
\pgfpathlineto{\pgfqpoint{1.197545in}{1.188750in}}%
\pgfpathlineto{\pgfqpoint{1.160570in}{1.207879in}}%
\pgfpathlineto{\pgfqpoint{1.118267in}{1.289790in}}%
\pgfpathclose%
\pgfusepath{fill}%
\end{pgfscope}%
\begin{pgfscope}%
\pgfpathrectangle{\pgfqpoint{0.150000in}{0.150000in}}{\pgfqpoint{2.700000in}{1.950000in}}%
\pgfusepath{clip}%
\pgfsetbuttcap%
\pgfsetroundjoin%
\definecolor{currentfill}{rgb}{0.880331,0.782858,0.790579}%
\pgfsetfillcolor{currentfill}%
\pgfsetlinewidth{0.000000pt}%
\definecolor{currentstroke}{rgb}{0.000000,0.000000,0.000000}%
\pgfsetstrokecolor{currentstroke}%
\pgfsetdash{}{0pt}%
\pgfpathmoveto{\pgfqpoint{1.991623in}{0.837849in}}%
\pgfpathlineto{\pgfqpoint{2.028062in}{0.845590in}}%
\pgfpathlineto{\pgfqpoint{1.989419in}{0.865424in}}%
\pgfpathlineto{\pgfqpoint{1.952929in}{0.857758in}}%
\pgfpathclose%
\pgfusepath{fill}%
\end{pgfscope}%
\begin{pgfscope}%
\pgfpathrectangle{\pgfqpoint{0.150000in}{0.150000in}}{\pgfqpoint{2.700000in}{1.950000in}}%
\pgfusepath{clip}%
\pgfsetbuttcap%
\pgfsetroundjoin%
\definecolor{currentfill}{rgb}{0.967708,0.941406,0.943490}%
\pgfsetfillcolor{currentfill}%
\pgfsetlinewidth{0.000000pt}%
\definecolor{currentstroke}{rgb}{0.000000,0.000000,0.000000}%
\pgfsetstrokecolor{currentstroke}%
\pgfsetdash{}{0pt}%
\pgfpathmoveto{\pgfqpoint{1.422664in}{1.035095in}}%
\pgfpathlineto{\pgfqpoint{1.460924in}{1.029866in}}%
\pgfpathlineto{\pgfqpoint{1.423349in}{1.049347in}}%
\pgfpathlineto{\pgfqpoint{1.384821in}{1.067085in}}%
\pgfpathclose%
\pgfusepath{fill}%
\end{pgfscope}%
\begin{pgfscope}%
\pgfpathrectangle{\pgfqpoint{0.150000in}{0.150000in}}{\pgfqpoint{2.700000in}{1.950000in}}%
\pgfusepath{clip}%
\pgfsetbuttcap%
\pgfsetroundjoin%
\definecolor{currentfill}{rgb}{0.933517,0.879366,0.883655}%
\pgfsetfillcolor{currentfill}%
\pgfsetlinewidth{0.000000pt}%
\definecolor{currentstroke}{rgb}{0.000000,0.000000,0.000000}%
\pgfsetstrokecolor{currentstroke}%
\pgfsetdash{}{0pt}%
\pgfpathmoveto{\pgfqpoint{1.574431in}{0.958646in}}%
\pgfpathlineto{\pgfqpoint{1.612057in}{0.953738in}}%
\pgfpathlineto{\pgfqpoint{1.574247in}{0.985649in}}%
\pgfpathlineto{\pgfqpoint{1.536486in}{0.978342in}}%
\pgfpathclose%
\pgfusepath{fill}%
\end{pgfscope}%
\begin{pgfscope}%
\pgfpathrectangle{\pgfqpoint{0.150000in}{0.150000in}}{\pgfqpoint{2.700000in}{1.950000in}}%
\pgfusepath{clip}%
\pgfsetbuttcap%
\pgfsetroundjoin%
\definecolor{currentfill}{rgb}{0.959574,0.964553,0.971523}%
\pgfsetfillcolor{currentfill}%
\pgfsetlinewidth{0.000000pt}%
\definecolor{currentstroke}{rgb}{0.000000,0.000000,0.000000}%
\pgfsetstrokecolor{currentstroke}%
\pgfsetdash{}{0pt}%
\pgfpathmoveto{\pgfqpoint{0.779137in}{1.130940in}}%
\pgfpathlineto{\pgfqpoint{0.816613in}{1.175362in}}%
\pgfpathlineto{\pgfqpoint{0.780113in}{1.194631in}}%
\pgfpathlineto{\pgfqpoint{0.742726in}{1.150281in}}%
\pgfpathclose%
\pgfusepath{fill}%
\end{pgfscope}%
\begin{pgfscope}%
\pgfpathrectangle{\pgfqpoint{0.150000in}{0.150000in}}{\pgfqpoint{2.700000in}{1.950000in}}%
\pgfusepath{clip}%
\pgfsetbuttcap%
\pgfsetroundjoin%
\definecolor{currentfill}{rgb}{0.841406,0.860938,0.888281}%
\pgfsetfillcolor{currentfill}%
\pgfsetlinewidth{0.000000pt}%
\definecolor{currentstroke}{rgb}{0.000000,0.000000,0.000000}%
\pgfsetstrokecolor{currentstroke}%
\pgfsetdash{}{0pt}%
\pgfpathmoveto{\pgfqpoint{0.780190in}{1.270802in}}%
\pgfpathlineto{\pgfqpoint{0.815902in}{1.340736in}}%
\pgfpathlineto{\pgfqpoint{0.778467in}{1.372515in}}%
\pgfpathlineto{\pgfqpoint{0.742887in}{1.302534in}}%
\pgfpathclose%
\pgfusepath{fill}%
\end{pgfscope}%
\begin{pgfscope}%
\pgfpathrectangle{\pgfqpoint{0.150000in}{0.150000in}}{\pgfqpoint{2.700000in}{1.950000in}}%
\pgfusepath{clip}%
\pgfsetbuttcap%
\pgfsetroundjoin%
\definecolor{currentfill}{rgb}{0.884130,0.789752,0.797227}%
\pgfsetfillcolor{currentfill}%
\pgfsetlinewidth{0.000000pt}%
\definecolor{currentstroke}{rgb}{0.000000,0.000000,0.000000}%
\pgfsetstrokecolor{currentstroke}%
\pgfsetdash{}{0pt}%
\pgfpathmoveto{\pgfqpoint{1.916219in}{0.850045in}}%
\pgfpathlineto{\pgfqpoint{1.952929in}{0.857758in}}%
\pgfpathlineto{\pgfqpoint{1.913930in}{0.865424in}}%
\pgfpathlineto{\pgfqpoint{1.877212in}{0.857758in}}%
\pgfpathclose%
\pgfusepath{fill}%
\end{pgfscope}%
\begin{pgfscope}%
\pgfpathrectangle{\pgfqpoint{0.150000in}{0.150000in}}{\pgfqpoint{2.700000in}{1.950000in}}%
\pgfusepath{clip}%
\pgfsetbuttcap%
\pgfsetroundjoin%
\definecolor{currentfill}{rgb}{0.922258,0.931832,0.945236}%
\pgfsetfillcolor{currentfill}%
\pgfsetlinewidth{0.000000pt}%
\definecolor{currentstroke}{rgb}{0.000000,0.000000,0.000000}%
\pgfsetstrokecolor{currentstroke}%
\pgfsetdash{}{0pt}%
\pgfpathmoveto{\pgfqpoint{0.816613in}{1.175362in}}%
\pgfpathlineto{\pgfqpoint{0.854130in}{1.219831in}}%
\pgfpathlineto{\pgfqpoint{0.817542in}{1.239030in}}%
\pgfpathlineto{\pgfqpoint{0.780113in}{1.194631in}}%
\pgfpathclose%
\pgfusepath{fill}%
\end{pgfscope}%
\begin{pgfscope}%
\pgfpathrectangle{\pgfqpoint{0.150000in}{0.150000in}}{\pgfqpoint{2.700000in}{1.950000in}}%
\pgfusepath{clip}%
\pgfsetbuttcap%
\pgfsetroundjoin%
\definecolor{currentfill}{rgb}{0.922258,0.931832,0.945236}%
\pgfsetfillcolor{currentfill}%
\pgfsetlinewidth{0.000000pt}%
\definecolor{currentstroke}{rgb}{0.000000,0.000000,0.000000}%
\pgfsetstrokecolor{currentstroke}%
\pgfsetdash{}{0pt}%
\pgfpathmoveto{\pgfqpoint{1.193863in}{1.225883in}}%
\pgfpathlineto{\pgfqpoint{1.235011in}{1.157055in}}%
\pgfpathlineto{\pgfqpoint{1.197545in}{1.188750in}}%
\pgfpathlineto{\pgfqpoint{1.156041in}{1.257857in}}%
\pgfpathclose%
\pgfusepath{fill}%
\end{pgfscope}%
\begin{pgfscope}%
\pgfpathrectangle{\pgfqpoint{0.150000in}{0.150000in}}{\pgfqpoint{2.700000in}{1.950000in}}%
\pgfusepath{clip}%
\pgfsetbuttcap%
\pgfsetroundjoin%
\definecolor{currentfill}{rgb}{0.906924,0.831112,0.837117}%
\pgfsetfillcolor{currentfill}%
\pgfsetlinewidth{0.000000pt}%
\definecolor{currentstroke}{rgb}{0.000000,0.000000,0.000000}%
\pgfsetstrokecolor{currentstroke}%
\pgfsetdash{}{0pt}%
\pgfpathmoveto{\pgfqpoint{1.726456in}{0.894437in}}%
\pgfpathlineto{\pgfqpoint{1.763491in}{0.889792in}}%
\pgfpathlineto{\pgfqpoint{1.725311in}{0.909557in}}%
\pgfpathlineto{\pgfqpoint{1.688183in}{0.914277in}}%
\pgfpathclose%
\pgfusepath{fill}%
\end{pgfscope}%
\begin{pgfscope}%
\pgfpathrectangle{\pgfqpoint{0.150000in}{0.150000in}}{\pgfqpoint{2.700000in}{1.950000in}}%
\pgfusepath{clip}%
\pgfsetbuttcap%
\pgfsetroundjoin%
\definecolor{currentfill}{rgb}{0.998100,0.996553,0.996676}%
\pgfsetfillcolor{currentfill}%
\pgfsetlinewidth{0.000000pt}%
\definecolor{currentstroke}{rgb}{0.000000,0.000000,0.000000}%
\pgfsetstrokecolor{currentstroke}%
\pgfsetdash{}{0pt}%
\pgfpathmoveto{\pgfqpoint{0.778158in}{1.067085in}}%
\pgfpathlineto{\pgfqpoint{0.816540in}{1.099033in}}%
\pgfpathlineto{\pgfqpoint{0.779137in}{1.130940in}}%
\pgfpathlineto{\pgfqpoint{0.741702in}{1.086568in}}%
\pgfpathclose%
\pgfusepath{fill}%
\end{pgfscope}%
\begin{pgfscope}%
\pgfpathrectangle{\pgfqpoint{0.150000in}{0.150000in}}{\pgfqpoint{2.700000in}{1.950000in}}%
\pgfusepath{clip}%
\pgfsetbuttcap%
\pgfsetroundjoin%
\definecolor{currentfill}{rgb}{0.860064,0.877298,0.901425}%
\pgfsetfillcolor{currentfill}%
\pgfsetlinewidth{0.000000pt}%
\definecolor{currentstroke}{rgb}{0.000000,0.000000,0.000000}%
\pgfsetstrokecolor{currentstroke}%
\pgfsetdash{}{0pt}%
\pgfpathmoveto{\pgfqpoint{0.817542in}{1.239030in}}%
\pgfpathlineto{\pgfqpoint{0.853386in}{1.308916in}}%
\pgfpathlineto{\pgfqpoint{0.815902in}{1.340736in}}%
\pgfpathlineto{\pgfqpoint{0.780190in}{1.270802in}}%
\pgfpathclose%
\pgfusepath{fill}%
\end{pgfscope}%
\begin{pgfscope}%
\pgfpathrectangle{\pgfqpoint{0.150000in}{0.150000in}}{\pgfqpoint{2.700000in}{1.950000in}}%
\pgfusepath{clip}%
\pgfsetbuttcap%
\pgfsetroundjoin%
\definecolor{currentfill}{rgb}{0.963909,0.934513,0.936841}%
\pgfsetfillcolor{currentfill}%
\pgfsetlinewidth{0.000000pt}%
\definecolor{currentstroke}{rgb}{0.000000,0.000000,0.000000}%
\pgfsetstrokecolor{currentstroke}%
\pgfsetdash{}{0pt}%
\pgfpathmoveto{\pgfqpoint{1.460465in}{1.015470in}}%
\pgfpathlineto{\pgfqpoint{1.498681in}{0.997966in}}%
\pgfpathlineto{\pgfqpoint{1.460924in}{1.029866in}}%
\pgfpathlineto{\pgfqpoint{1.422664in}{1.035095in}}%
\pgfpathclose%
\pgfusepath{fill}%
\end{pgfscope}%
\begin{pgfscope}%
\pgfpathrectangle{\pgfqpoint{0.150000in}{0.150000in}}{\pgfqpoint{2.700000in}{1.950000in}}%
\pgfusepath{clip}%
\pgfsetbuttcap%
\pgfsetroundjoin%
\definecolor{currentfill}{rgb}{0.972013,0.975460,0.980285}%
\pgfsetfillcolor{currentfill}%
\pgfsetlinewidth{0.000000pt}%
\definecolor{currentstroke}{rgb}{0.000000,0.000000,0.000000}%
\pgfsetstrokecolor{currentstroke}%
\pgfsetdash{}{0pt}%
\pgfpathmoveto{\pgfqpoint{0.816540in}{1.099033in}}%
\pgfpathlineto{\pgfqpoint{0.854060in}{1.143466in}}%
\pgfpathlineto{\pgfqpoint{0.816613in}{1.175362in}}%
\pgfpathlineto{\pgfqpoint{0.779137in}{1.130940in}}%
\pgfpathclose%
\pgfusepath{fill}%
\end{pgfscope}%
\begin{pgfscope}%
\pgfpathrectangle{\pgfqpoint{0.150000in}{0.150000in}}{\pgfqpoint{2.700000in}{1.950000in}}%
\pgfusepath{clip}%
\pgfsetbuttcap%
\pgfsetroundjoin%
\definecolor{currentfill}{rgb}{0.891728,0.803539,0.810524}%
\pgfsetfillcolor{currentfill}%
\pgfsetlinewidth{0.000000pt}%
\definecolor{currentstroke}{rgb}{0.000000,0.000000,0.000000}%
\pgfsetstrokecolor{currentstroke}%
\pgfsetdash{}{0pt}%
\pgfpathmoveto{\pgfqpoint{1.840635in}{0.862270in}}%
\pgfpathlineto{\pgfqpoint{1.877212in}{0.857758in}}%
\pgfpathlineto{\pgfqpoint{1.838800in}{0.877594in}}%
\pgfpathlineto{\pgfqpoint{1.801811in}{0.869955in}}%
\pgfpathclose%
\pgfusepath{fill}%
\end{pgfscope}%
\begin{pgfscope}%
\pgfpathrectangle{\pgfqpoint{0.150000in}{0.150000in}}{\pgfqpoint{2.700000in}{1.950000in}}%
\pgfusepath{clip}%
\pgfsetbuttcap%
\pgfsetroundjoin%
\definecolor{currentfill}{rgb}{0.929718,0.872472,0.877007}%
\pgfsetfillcolor{currentfill}%
\pgfsetlinewidth{0.000000pt}%
\definecolor{currentstroke}{rgb}{0.000000,0.000000,0.000000}%
\pgfsetstrokecolor{currentstroke}%
\pgfsetdash{}{0pt}%
\pgfpathmoveto{\pgfqpoint{1.612516in}{0.938878in}}%
\pgfpathlineto{\pgfqpoint{1.650050in}{0.934043in}}%
\pgfpathlineto{\pgfqpoint{1.612057in}{0.953738in}}%
\pgfpathlineto{\pgfqpoint{1.574431in}{0.958646in}}%
\pgfpathclose%
\pgfusepath{fill}%
\end{pgfscope}%
\begin{pgfscope}%
\pgfpathrectangle{\pgfqpoint{0.150000in}{0.150000in}}{\pgfqpoint{2.700000in}{1.950000in}}%
\pgfusepath{clip}%
\pgfsetbuttcap%
\pgfsetroundjoin%
\definecolor{currentfill}{rgb}{0.940916,0.948192,0.958379}%
\pgfsetfillcolor{currentfill}%
\pgfsetlinewidth{0.000000pt}%
\definecolor{currentstroke}{rgb}{0.000000,0.000000,0.000000}%
\pgfsetstrokecolor{currentstroke}%
\pgfsetdash{}{0pt}%
\pgfpathmoveto{\pgfqpoint{1.231735in}{1.193867in}}%
\pgfpathlineto{\pgfqpoint{1.272213in}{1.137785in}}%
\pgfpathlineto{\pgfqpoint{1.235011in}{1.157055in}}%
\pgfpathlineto{\pgfqpoint{1.193863in}{1.225883in}}%
\pgfpathclose%
\pgfusepath{fill}%
\end{pgfscope}%
\begin{pgfscope}%
\pgfpathrectangle{\pgfqpoint{0.150000in}{0.150000in}}{\pgfqpoint{2.700000in}{1.950000in}}%
\pgfusepath{clip}%
\pgfsetbuttcap%
\pgfsetroundjoin%
\definecolor{currentfill}{rgb}{0.934697,0.942739,0.953998}%
\pgfsetfillcolor{currentfill}%
\pgfsetlinewidth{0.000000pt}%
\definecolor{currentstroke}{rgb}{0.000000,0.000000,0.000000}%
\pgfsetstrokecolor{currentstroke}%
\pgfsetdash{}{0pt}%
\pgfpathmoveto{\pgfqpoint{0.854060in}{1.143466in}}%
\pgfpathlineto{\pgfqpoint{0.891622in}{1.187947in}}%
\pgfpathlineto{\pgfqpoint{0.854130in}{1.219831in}}%
\pgfpathlineto{\pgfqpoint{0.816613in}{1.175362in}}%
\pgfpathclose%
\pgfusepath{fill}%
\end{pgfscope}%
\begin{pgfscope}%
\pgfpathrectangle{\pgfqpoint{0.150000in}{0.150000in}}{\pgfqpoint{2.700000in}{1.950000in}}%
\pgfusepath{clip}%
\pgfsetbuttcap%
\pgfsetroundjoin%
\definecolor{currentfill}{rgb}{0.872503,0.888205,0.910187}%
\pgfsetfillcolor{currentfill}%
\pgfsetlinewidth{0.000000pt}%
\definecolor{currentstroke}{rgb}{0.000000,0.000000,0.000000}%
\pgfsetstrokecolor{currentstroke}%
\pgfsetdash{}{0pt}%
\pgfpathmoveto{\pgfqpoint{0.854130in}{1.219831in}}%
\pgfpathlineto{\pgfqpoint{0.890919in}{1.277054in}}%
\pgfpathlineto{\pgfqpoint{0.853386in}{1.308916in}}%
\pgfpathlineto{\pgfqpoint{0.817542in}{1.239030in}}%
\pgfpathclose%
\pgfusepath{fill}%
\end{pgfscope}%
\begin{pgfscope}%
\pgfpathrectangle{\pgfqpoint{0.150000in}{0.150000in}}{\pgfqpoint{2.700000in}{1.950000in}}%
\pgfusepath{clip}%
\pgfsetbuttcap%
\pgfsetroundjoin%
\definecolor{currentfill}{rgb}{0.884130,0.789752,0.797227}%
\pgfsetfillcolor{currentfill}%
\pgfsetlinewidth{0.000000pt}%
\definecolor{currentstroke}{rgb}{0.000000,0.000000,0.000000}%
\pgfsetstrokecolor{currentstroke}%
\pgfsetdash{}{0pt}%
\pgfpathmoveto{\pgfqpoint{2.031049in}{0.830061in}}%
\pgfpathlineto{\pgfqpoint{2.067479in}{0.837849in}}%
\pgfpathlineto{\pgfqpoint{2.028062in}{0.845590in}}%
\pgfpathlineto{\pgfqpoint{1.991623in}{0.837849in}}%
\pgfpathclose%
\pgfusepath{fill}%
\end{pgfscope}%
\begin{pgfscope}%
\pgfpathrectangle{\pgfqpoint{0.150000in}{0.150000in}}{\pgfqpoint{2.700000in}{1.950000in}}%
\pgfusepath{clip}%
\pgfsetbuttcap%
\pgfsetroundjoin%
\definecolor{currentfill}{rgb}{0.956311,0.920726,0.923545}%
\pgfsetfillcolor{currentfill}%
\pgfsetlinewidth{0.000000pt}%
\definecolor{currentstroke}{rgb}{0.000000,0.000000,0.000000}%
\pgfsetstrokecolor{currentstroke}%
\pgfsetdash{}{0pt}%
\pgfpathmoveto{\pgfqpoint{1.498451in}{0.983367in}}%
\pgfpathlineto{\pgfqpoint{1.536486in}{0.978342in}}%
\pgfpathlineto{\pgfqpoint{1.498681in}{0.997966in}}%
\pgfpathlineto{\pgfqpoint{1.460465in}{1.015470in}}%
\pgfpathclose%
\pgfusepath{fill}%
\end{pgfscope}%
\begin{pgfscope}%
\pgfpathrectangle{\pgfqpoint{0.150000in}{0.150000in}}{\pgfqpoint{2.700000in}{1.950000in}}%
\pgfusepath{clip}%
\pgfsetbuttcap%
\pgfsetroundjoin%
\definecolor{currentfill}{rgb}{0.978232,0.980913,0.984666}%
\pgfsetfillcolor{currentfill}%
\pgfsetlinewidth{0.000000pt}%
\definecolor{currentstroke}{rgb}{0.000000,0.000000,0.000000}%
\pgfsetstrokecolor{currentstroke}%
\pgfsetdash{}{0pt}%
\pgfpathmoveto{\pgfqpoint{0.853177in}{1.079551in}}%
\pgfpathlineto{\pgfqpoint{0.890787in}{1.124054in}}%
\pgfpathlineto{\pgfqpoint{0.854060in}{1.143466in}}%
\pgfpathlineto{\pgfqpoint{0.816540in}{1.099033in}}%
\pgfpathclose%
\pgfusepath{fill}%
\end{pgfscope}%
\begin{pgfscope}%
\pgfpathrectangle{\pgfqpoint{0.150000in}{0.150000in}}{\pgfqpoint{2.700000in}{1.950000in}}%
\pgfusepath{clip}%
\pgfsetbuttcap%
\pgfsetroundjoin%
\definecolor{currentfill}{rgb}{0.994301,0.989660,0.990028}%
\pgfsetfillcolor{currentfill}%
\pgfsetlinewidth{0.000000pt}%
\definecolor{currentstroke}{rgb}{0.000000,0.000000,0.000000}%
\pgfsetstrokecolor{currentstroke}%
\pgfsetdash{}{0pt}%
\pgfpathmoveto{\pgfqpoint{0.814749in}{1.047531in}}%
\pgfpathlineto{\pgfqpoint{0.853177in}{1.079551in}}%
\pgfpathlineto{\pgfqpoint{0.816540in}{1.099033in}}%
\pgfpathlineto{\pgfqpoint{0.778158in}{1.067085in}}%
\pgfpathclose%
\pgfusepath{fill}%
\end{pgfscope}%
\begin{pgfscope}%
\pgfpathrectangle{\pgfqpoint{0.150000in}{0.150000in}}{\pgfqpoint{2.700000in}{1.950000in}}%
\pgfusepath{clip}%
\pgfsetbuttcap%
\pgfsetroundjoin%
\definecolor{currentfill}{rgb}{0.947135,0.953646,0.962760}%
\pgfsetfillcolor{currentfill}%
\pgfsetlinewidth{0.000000pt}%
\definecolor{currentstroke}{rgb}{0.000000,0.000000,0.000000}%
\pgfsetstrokecolor{currentstroke}%
\pgfsetdash{}{0pt}%
\pgfpathmoveto{\pgfqpoint{0.890787in}{1.124054in}}%
\pgfpathlineto{\pgfqpoint{0.929162in}{1.156021in}}%
\pgfpathlineto{\pgfqpoint{0.891622in}{1.187947in}}%
\pgfpathlineto{\pgfqpoint{0.854060in}{1.143466in}}%
\pgfpathclose%
\pgfusepath{fill}%
\end{pgfscope}%
\begin{pgfscope}%
\pgfpathrectangle{\pgfqpoint{0.150000in}{0.150000in}}{\pgfqpoint{2.700000in}{1.950000in}}%
\pgfusepath{clip}%
\pgfsetbuttcap%
\pgfsetroundjoin%
\definecolor{currentfill}{rgb}{0.947135,0.953646,0.962760}%
\pgfsetfillcolor{currentfill}%
\pgfsetlinewidth{0.000000pt}%
\definecolor{currentstroke}{rgb}{0.000000,0.000000,0.000000}%
\pgfsetstrokecolor{currentstroke}%
\pgfsetdash{}{0pt}%
\pgfpathmoveto{\pgfqpoint{1.269336in}{1.174456in}}%
\pgfpathlineto{\pgfqpoint{1.309821in}{1.105979in}}%
\pgfpathlineto{\pgfqpoint{1.272213in}{1.137785in}}%
\pgfpathlineto{\pgfqpoint{1.231735in}{1.193867in}}%
\pgfpathclose%
\pgfusepath{fill}%
\end{pgfscope}%
\begin{pgfscope}%
\pgfpathrectangle{\pgfqpoint{0.150000in}{0.150000in}}{\pgfqpoint{2.700000in}{1.950000in}}%
\pgfusepath{clip}%
\pgfsetbuttcap%
\pgfsetroundjoin%
\definecolor{currentfill}{rgb}{0.906924,0.831112,0.837117}%
\pgfsetfillcolor{currentfill}%
\pgfsetlinewidth{0.000000pt}%
\definecolor{currentstroke}{rgb}{0.000000,0.000000,0.000000}%
\pgfsetstrokecolor{currentstroke}%
\pgfsetdash{}{0pt}%
\pgfpathmoveto{\pgfqpoint{1.764871in}{0.874524in}}%
\pgfpathlineto{\pgfqpoint{1.801811in}{0.869955in}}%
\pgfpathlineto{\pgfqpoint{1.763491in}{0.889792in}}%
\pgfpathlineto{\pgfqpoint{1.726456in}{0.894437in}}%
\pgfpathclose%
\pgfusepath{fill}%
\end{pgfscope}%
\begin{pgfscope}%
\pgfpathrectangle{\pgfqpoint{0.150000in}{0.150000in}}{\pgfqpoint{2.700000in}{1.950000in}}%
\pgfusepath{clip}%
\pgfsetbuttcap%
\pgfsetroundjoin%
\definecolor{currentfill}{rgb}{0.925919,0.865579,0.870358}%
\pgfsetfillcolor{currentfill}%
\pgfsetlinewidth{0.000000pt}%
\definecolor{currentstroke}{rgb}{0.000000,0.000000,0.000000}%
\pgfsetstrokecolor{currentstroke}%
\pgfsetdash{}{0pt}%
\pgfpathmoveto{\pgfqpoint{1.650605in}{0.906722in}}%
\pgfpathlineto{\pgfqpoint{1.688183in}{0.914277in}}%
\pgfpathlineto{\pgfqpoint{1.650050in}{0.934043in}}%
\pgfpathlineto{\pgfqpoint{1.612516in}{0.938878in}}%
\pgfpathclose%
\pgfusepath{fill}%
\end{pgfscope}%
\begin{pgfscope}%
\pgfpathrectangle{\pgfqpoint{0.150000in}{0.150000in}}{\pgfqpoint{2.700000in}{1.950000in}}%
\pgfusepath{clip}%
\pgfsetbuttcap%
\pgfsetroundjoin%
\definecolor{currentfill}{rgb}{0.891728,0.803539,0.810524}%
\pgfsetfillcolor{currentfill}%
\pgfsetlinewidth{0.000000pt}%
\definecolor{currentstroke}{rgb}{0.000000,0.000000,0.000000}%
\pgfsetstrokecolor{currentstroke}%
\pgfsetdash{}{0pt}%
\pgfpathmoveto{\pgfqpoint{1.955463in}{0.842285in}}%
\pgfpathlineto{\pgfqpoint{1.991623in}{0.837849in}}%
\pgfpathlineto{\pgfqpoint{1.952929in}{0.857758in}}%
\pgfpathlineto{\pgfqpoint{1.916219in}{0.850045in}}%
\pgfpathclose%
\pgfusepath{fill}%
\end{pgfscope}%
\begin{pgfscope}%
\pgfpathrectangle{\pgfqpoint{0.150000in}{0.150000in}}{\pgfqpoint{2.700000in}{1.950000in}}%
\pgfusepath{clip}%
\pgfsetbuttcap%
\pgfsetroundjoin%
\definecolor{currentfill}{rgb}{0.878722,0.893658,0.914568}%
\pgfsetfillcolor{currentfill}%
\pgfsetlinewidth{0.000000pt}%
\definecolor{currentstroke}{rgb}{0.000000,0.000000,0.000000}%
\pgfsetstrokecolor{currentstroke}%
\pgfsetdash{}{0pt}%
\pgfpathmoveto{\pgfqpoint{0.891622in}{1.187947in}}%
\pgfpathlineto{\pgfqpoint{0.927773in}{1.257857in}}%
\pgfpathlineto{\pgfqpoint{0.890919in}{1.277054in}}%
\pgfpathlineto{\pgfqpoint{0.854130in}{1.219831in}}%
\pgfpathclose%
\pgfusepath{fill}%
\end{pgfscope}%
\begin{pgfscope}%
\pgfpathrectangle{\pgfqpoint{0.150000in}{0.150000in}}{\pgfqpoint{2.700000in}{1.950000in}}%
\pgfusepath{clip}%
\pgfsetbuttcap%
\pgfsetroundjoin%
\definecolor{currentfill}{rgb}{0.899326,0.817325,0.823820}%
\pgfsetfillcolor{currentfill}%
\pgfsetlinewidth{0.000000pt}%
\definecolor{currentstroke}{rgb}{0.000000,0.000000,0.000000}%
\pgfsetstrokecolor{currentstroke}%
\pgfsetdash{}{0pt}%
\pgfpathmoveto{\pgfqpoint{1.879285in}{0.842285in}}%
\pgfpathlineto{\pgfqpoint{1.916219in}{0.850045in}}%
\pgfpathlineto{\pgfqpoint{1.877212in}{0.857758in}}%
\pgfpathlineto{\pgfqpoint{1.840635in}{0.862270in}}%
\pgfpathclose%
\pgfusepath{fill}%
\end{pgfscope}%
\begin{pgfscope}%
\pgfpathrectangle{\pgfqpoint{0.150000in}{0.150000in}}{\pgfqpoint{2.700000in}{1.950000in}}%
\pgfusepath{clip}%
\pgfsetbuttcap%
\pgfsetroundjoin%
\definecolor{currentfill}{rgb}{0.952512,0.913833,0.916896}%
\pgfsetfillcolor{currentfill}%
\pgfsetlinewidth{0.000000pt}%
\definecolor{currentstroke}{rgb}{0.000000,0.000000,0.000000}%
\pgfsetstrokecolor{currentstroke}%
\pgfsetdash{}{0pt}%
\pgfpathmoveto{\pgfqpoint{1.536486in}{0.963597in}}%
\pgfpathlineto{\pgfqpoint{1.574431in}{0.958646in}}%
\pgfpathlineto{\pgfqpoint{1.536486in}{0.978342in}}%
\pgfpathlineto{\pgfqpoint{1.498451in}{0.983367in}}%
\pgfpathclose%
\pgfusepath{fill}%
\end{pgfscope}%
\begin{pgfscope}%
\pgfpathrectangle{\pgfqpoint{0.150000in}{0.150000in}}{\pgfqpoint{2.700000in}{1.950000in}}%
\pgfusepath{clip}%
\pgfsetbuttcap%
\pgfsetroundjoin%
\definecolor{currentfill}{rgb}{0.959574,0.964553,0.971523}%
\pgfsetfillcolor{currentfill}%
\pgfsetlinewidth{0.000000pt}%
\definecolor{currentstroke}{rgb}{0.000000,0.000000,0.000000}%
\pgfsetstrokecolor{currentstroke}%
\pgfsetdash{}{0pt}%
\pgfpathmoveto{\pgfqpoint{0.928376in}{1.092046in}}%
\pgfpathlineto{\pgfqpoint{0.966071in}{1.136610in}}%
\pgfpathlineto{\pgfqpoint{0.929162in}{1.156021in}}%
\pgfpathlineto{\pgfqpoint{0.890787in}{1.124054in}}%
\pgfpathclose%
\pgfusepath{fill}%
\end{pgfscope}%
\begin{pgfscope}%
\pgfpathrectangle{\pgfqpoint{0.150000in}{0.150000in}}{\pgfqpoint{2.700000in}{1.950000in}}%
\pgfusepath{clip}%
\pgfsetbuttcap%
\pgfsetroundjoin%
\definecolor{currentfill}{rgb}{0.959574,0.964553,0.971523}%
\pgfsetfillcolor{currentfill}%
\pgfsetlinewidth{0.000000pt}%
\definecolor{currentstroke}{rgb}{0.000000,0.000000,0.000000}%
\pgfsetstrokecolor{currentstroke}%
\pgfsetdash{}{0pt}%
\pgfpathmoveto{\pgfqpoint{1.307351in}{1.142327in}}%
\pgfpathlineto{\pgfqpoint{1.347252in}{1.086568in}}%
\pgfpathlineto{\pgfqpoint{1.309821in}{1.105979in}}%
\pgfpathlineto{\pgfqpoint{1.269336in}{1.174456in}}%
\pgfpathclose%
\pgfusepath{fill}%
\end{pgfscope}%
\begin{pgfscope}%
\pgfpathrectangle{\pgfqpoint{0.150000in}{0.150000in}}{\pgfqpoint{2.700000in}{1.950000in}}%
\pgfusepath{clip}%
\pgfsetbuttcap%
\pgfsetroundjoin%
\definecolor{currentfill}{rgb}{0.984452,0.986366,0.989047}%
\pgfsetfillcolor{currentfill}%
\pgfsetlinewidth{0.000000pt}%
\definecolor{currentstroke}{rgb}{0.000000,0.000000,0.000000}%
\pgfsetstrokecolor{currentstroke}%
\pgfsetdash{}{0pt}%
\pgfpathmoveto{\pgfqpoint{0.889950in}{1.059996in}}%
\pgfpathlineto{\pgfqpoint{0.928376in}{1.092046in}}%
\pgfpathlineto{\pgfqpoint{0.890787in}{1.124054in}}%
\pgfpathlineto{\pgfqpoint{0.853177in}{1.079551in}}%
\pgfpathclose%
\pgfusepath{fill}%
\end{pgfscope}%
\begin{pgfscope}%
\pgfpathrectangle{\pgfqpoint{0.150000in}{0.150000in}}{\pgfqpoint{2.700000in}{1.950000in}}%
\pgfusepath{clip}%
\pgfsetbuttcap%
\pgfsetroundjoin%
\definecolor{currentfill}{rgb}{0.990502,0.982767,0.983379}%
\pgfsetfillcolor{currentfill}%
\pgfsetlinewidth{0.000000pt}%
\definecolor{currentstroke}{rgb}{0.000000,0.000000,0.000000}%
\pgfsetstrokecolor{currentstroke}%
\pgfsetdash{}{0pt}%
\pgfpathmoveto{\pgfqpoint{0.852292in}{1.015470in}}%
\pgfpathlineto{\pgfqpoint{0.889950in}{1.059996in}}%
\pgfpathlineto{\pgfqpoint{0.853177in}{1.079551in}}%
\pgfpathlineto{\pgfqpoint{0.814749in}{1.047531in}}%
\pgfpathclose%
\pgfusepath{fill}%
\end{pgfscope}%
\begin{pgfscope}%
\pgfpathrectangle{\pgfqpoint{0.150000in}{0.150000in}}{\pgfqpoint{2.700000in}{1.950000in}}%
\pgfusepath{clip}%
\pgfsetbuttcap%
\pgfsetroundjoin%
\definecolor{currentfill}{rgb}{0.897381,0.910018,0.927711}%
\pgfsetfillcolor{currentfill}%
\pgfsetlinewidth{0.000000pt}%
\definecolor{currentstroke}{rgb}{0.000000,0.000000,0.000000}%
\pgfsetstrokecolor{currentstroke}%
\pgfsetdash{}{0pt}%
\pgfpathmoveto{\pgfqpoint{0.929162in}{1.156021in}}%
\pgfpathlineto{\pgfqpoint{0.965448in}{1.225883in}}%
\pgfpathlineto{\pgfqpoint{0.927773in}{1.257857in}}%
\pgfpathlineto{\pgfqpoint{0.891622in}{1.187947in}}%
\pgfpathclose%
\pgfusepath{fill}%
\end{pgfscope}%
\begin{pgfscope}%
\pgfpathrectangle{\pgfqpoint{0.150000in}{0.150000in}}{\pgfqpoint{2.700000in}{1.950000in}}%
\pgfusepath{clip}%
\pgfsetbuttcap%
\pgfsetroundjoin%
\definecolor{currentfill}{rgb}{0.925919,0.865579,0.870358}%
\pgfsetfillcolor{currentfill}%
\pgfsetlinewidth{0.000000pt}%
\definecolor{currentstroke}{rgb}{0.000000,0.000000,0.000000}%
\pgfsetstrokecolor{currentstroke}%
\pgfsetdash{}{0pt}%
\pgfpathmoveto{\pgfqpoint{1.688925in}{0.886808in}}%
\pgfpathlineto{\pgfqpoint{1.726456in}{0.894437in}}%
\pgfpathlineto{\pgfqpoint{1.688183in}{0.914277in}}%
\pgfpathlineto{\pgfqpoint{1.650605in}{0.906722in}}%
\pgfpathclose%
\pgfusepath{fill}%
\end{pgfscope}%
\begin{pgfscope}%
\pgfpathrectangle{\pgfqpoint{0.150000in}{0.150000in}}{\pgfqpoint{2.700000in}{1.950000in}}%
\pgfusepath{clip}%
\pgfsetbuttcap%
\pgfsetroundjoin%
\definecolor{currentfill}{rgb}{0.910723,0.838006,0.843765}%
\pgfsetfillcolor{currentfill}%
\pgfsetlinewidth{0.000000pt}%
\definecolor{currentstroke}{rgb}{0.000000,0.000000,0.000000}%
\pgfsetstrokecolor{currentstroke}%
\pgfsetdash{}{0pt}%
\pgfpathmoveto{\pgfqpoint{1.803427in}{0.854538in}}%
\pgfpathlineto{\pgfqpoint{1.840635in}{0.862270in}}%
\pgfpathlineto{\pgfqpoint{1.801811in}{0.869955in}}%
\pgfpathlineto{\pgfqpoint{1.764871in}{0.874524in}}%
\pgfpathclose%
\pgfusepath{fill}%
\end{pgfscope}%
\begin{pgfscope}%
\pgfpathrectangle{\pgfqpoint{0.150000in}{0.150000in}}{\pgfqpoint{2.700000in}{1.950000in}}%
\pgfusepath{clip}%
\pgfsetbuttcap%
\pgfsetroundjoin%
\definecolor{currentfill}{rgb}{0.909819,0.920925,0.936474}%
\pgfsetfillcolor{currentfill}%
\pgfsetlinewidth{0.000000pt}%
\definecolor{currentstroke}{rgb}{0.000000,0.000000,0.000000}%
\pgfsetstrokecolor{currentstroke}%
\pgfsetdash{}{0pt}%
\pgfpathmoveto{\pgfqpoint{0.966071in}{1.136610in}}%
\pgfpathlineto{\pgfqpoint{1.003171in}{1.193867in}}%
\pgfpathlineto{\pgfqpoint{0.965448in}{1.225883in}}%
\pgfpathlineto{\pgfqpoint{0.929162in}{1.156021in}}%
\pgfpathclose%
\pgfusepath{fill}%
\end{pgfscope}%
\begin{pgfscope}%
\pgfpathrectangle{\pgfqpoint{0.150000in}{0.150000in}}{\pgfqpoint{2.700000in}{1.950000in}}%
\pgfusepath{clip}%
\pgfsetbuttcap%
\pgfsetroundjoin%
\definecolor{currentfill}{rgb}{0.965794,0.970006,0.975904}%
\pgfsetfillcolor{currentfill}%
\pgfsetlinewidth{0.000000pt}%
\definecolor{currentstroke}{rgb}{0.000000,0.000000,0.000000}%
\pgfsetstrokecolor{currentstroke}%
\pgfsetdash{}{0pt}%
\pgfpathmoveto{\pgfqpoint{0.965331in}{1.072491in}}%
\pgfpathlineto{\pgfqpoint{1.003754in}{1.104571in}}%
\pgfpathlineto{\pgfqpoint{0.966071in}{1.136610in}}%
\pgfpathlineto{\pgfqpoint{0.928376in}{1.092046in}}%
\pgfpathclose%
\pgfusepath{fill}%
\end{pgfscope}%
\begin{pgfscope}%
\pgfpathrectangle{\pgfqpoint{0.150000in}{0.150000in}}{\pgfqpoint{2.700000in}{1.950000in}}%
\pgfusepath{clip}%
\pgfsetbuttcap%
\pgfsetroundjoin%
\definecolor{currentfill}{rgb}{0.965794,0.970006,0.975904}%
\pgfsetfillcolor{currentfill}%
\pgfsetlinewidth{0.000000pt}%
\definecolor{currentstroke}{rgb}{0.000000,0.000000,0.000000}%
\pgfsetstrokecolor{currentstroke}%
\pgfsetdash{}{0pt}%
\pgfpathmoveto{\pgfqpoint{1.345416in}{1.110156in}}%
\pgfpathlineto{\pgfqpoint{1.384821in}{1.067085in}}%
\pgfpathlineto{\pgfqpoint{1.347252in}{1.086568in}}%
\pgfpathlineto{\pgfqpoint{1.307351in}{1.142327in}}%
\pgfpathclose%
\pgfusepath{fill}%
\end{pgfscope}%
\begin{pgfscope}%
\pgfpathrectangle{\pgfqpoint{0.150000in}{0.150000in}}{\pgfqpoint{2.700000in}{1.950000in}}%
\pgfusepath{clip}%
\pgfsetbuttcap%
\pgfsetroundjoin%
\definecolor{currentfill}{rgb}{0.996890,0.997273,0.997809}%
\pgfsetfillcolor{currentfill}%
\pgfsetlinewidth{0.000000pt}%
\definecolor{currentstroke}{rgb}{0.000000,0.000000,0.000000}%
\pgfsetstrokecolor{currentstroke}%
\pgfsetdash{}{0pt}%
\pgfpathmoveto{\pgfqpoint{0.927587in}{1.027905in}}%
\pgfpathlineto{\pgfqpoint{0.965331in}{1.072491in}}%
\pgfpathlineto{\pgfqpoint{0.928376in}{1.092046in}}%
\pgfpathlineto{\pgfqpoint{0.889950in}{1.059996in}}%
\pgfpathclose%
\pgfusepath{fill}%
\end{pgfscope}%
\begin{pgfscope}%
\pgfpathrectangle{\pgfqpoint{0.150000in}{0.150000in}}{\pgfqpoint{2.700000in}{1.950000in}}%
\pgfusepath{clip}%
\pgfsetbuttcap%
\pgfsetroundjoin%
\definecolor{currentfill}{rgb}{0.986703,0.975873,0.976731}%
\pgfsetfillcolor{currentfill}%
\pgfsetlinewidth{0.000000pt}%
\definecolor{currentstroke}{rgb}{0.000000,0.000000,0.000000}%
\pgfsetstrokecolor{currentstroke}%
\pgfsetdash{}{0pt}%
\pgfpathmoveto{\pgfqpoint{0.889110in}{0.995771in}}%
\pgfpathlineto{\pgfqpoint{0.927587in}{1.027905in}}%
\pgfpathlineto{\pgfqpoint{0.889950in}{1.059996in}}%
\pgfpathlineto{\pgfqpoint{0.852292in}{1.015470in}}%
\pgfpathclose%
\pgfusepath{fill}%
\end{pgfscope}%
\begin{pgfscope}%
\pgfpathrectangle{\pgfqpoint{0.150000in}{0.150000in}}{\pgfqpoint{2.700000in}{1.950000in}}%
\pgfusepath{clip}%
\pgfsetbuttcap%
\pgfsetroundjoin%
\definecolor{currentfill}{rgb}{0.952512,0.913833,0.916896}%
\pgfsetfillcolor{currentfill}%
\pgfsetlinewidth{0.000000pt}%
\definecolor{currentstroke}{rgb}{0.000000,0.000000,0.000000}%
\pgfsetstrokecolor{currentstroke}%
\pgfsetdash{}{0pt}%
\pgfpathmoveto{\pgfqpoint{1.574617in}{0.931380in}}%
\pgfpathlineto{\pgfqpoint{1.612516in}{0.938878in}}%
\pgfpathlineto{\pgfqpoint{1.574431in}{0.958646in}}%
\pgfpathlineto{\pgfqpoint{1.536486in}{0.963597in}}%
\pgfpathclose%
\pgfusepath{fill}%
\end{pgfscope}%
\begin{pgfscope}%
\pgfpathrectangle{\pgfqpoint{0.150000in}{0.150000in}}{\pgfqpoint{2.700000in}{1.950000in}}%
\pgfusepath{clip}%
\pgfsetbuttcap%
\pgfsetroundjoin%
\definecolor{currentfill}{rgb}{0.899326,0.817325,0.823820}%
\pgfsetfillcolor{currentfill}%
\pgfsetlinewidth{0.000000pt}%
\definecolor{currentstroke}{rgb}{0.000000,0.000000,0.000000}%
\pgfsetstrokecolor{currentstroke}%
\pgfsetdash{}{0pt}%
\pgfpathmoveto{\pgfqpoint{1.994396in}{0.822226in}}%
\pgfpathlineto{\pgfqpoint{2.031049in}{0.830061in}}%
\pgfpathlineto{\pgfqpoint{1.991623in}{0.837849in}}%
\pgfpathlineto{\pgfqpoint{1.955463in}{0.842285in}}%
\pgfpathclose%
\pgfusepath{fill}%
\end{pgfscope}%
\begin{pgfscope}%
\pgfpathrectangle{\pgfqpoint{0.150000in}{0.150000in}}{\pgfqpoint{2.700000in}{1.950000in}}%
\pgfusepath{clip}%
\pgfsetbuttcap%
\pgfsetroundjoin%
\definecolor{currentfill}{rgb}{0.922258,0.931832,0.945236}%
\pgfsetfillcolor{currentfill}%
\pgfsetlinewidth{0.000000pt}%
\definecolor{currentstroke}{rgb}{0.000000,0.000000,0.000000}%
\pgfsetstrokecolor{currentstroke}%
\pgfsetdash{}{0pt}%
\pgfpathmoveto{\pgfqpoint{1.003754in}{1.104571in}}%
\pgfpathlineto{\pgfqpoint{1.040943in}{1.161810in}}%
\pgfpathlineto{\pgfqpoint{1.003171in}{1.193867in}}%
\pgfpathlineto{\pgfqpoint{0.966071in}{1.136610in}}%
\pgfpathclose%
\pgfusepath{fill}%
\end{pgfscope}%
\begin{pgfscope}%
\pgfpathrectangle{\pgfqpoint{0.150000in}{0.150000in}}{\pgfqpoint{2.700000in}{1.950000in}}%
\pgfusepath{clip}%
\pgfsetbuttcap%
\pgfsetroundjoin%
\definecolor{currentfill}{rgb}{0.978232,0.980913,0.984666}%
\pgfsetfillcolor{currentfill}%
\pgfsetlinewidth{0.000000pt}%
\definecolor{currentstroke}{rgb}{0.000000,0.000000,0.000000}%
\pgfsetstrokecolor{currentstroke}%
\pgfsetdash{}{0pt}%
\pgfpathmoveto{\pgfqpoint{1.003062in}{1.040369in}}%
\pgfpathlineto{\pgfqpoint{1.041485in}{1.072491in}}%
\pgfpathlineto{\pgfqpoint{1.003754in}{1.104571in}}%
\pgfpathlineto{\pgfqpoint{0.965331in}{1.072491in}}%
\pgfpathclose%
\pgfusepath{fill}%
\end{pgfscope}%
\begin{pgfscope}%
\pgfpathrectangle{\pgfqpoint{0.150000in}{0.150000in}}{\pgfqpoint{2.700000in}{1.950000in}}%
\pgfusepath{clip}%
\pgfsetbuttcap%
\pgfsetroundjoin%
\definecolor{currentfill}{rgb}{0.978232,0.980913,0.984666}%
\pgfsetfillcolor{currentfill}%
\pgfsetlinewidth{0.000000pt}%
\definecolor{currentstroke}{rgb}{0.000000,0.000000,0.000000}%
\pgfsetstrokecolor{currentstroke}%
\pgfsetdash{}{0pt}%
\pgfpathmoveto{\pgfqpoint{1.383530in}{1.077943in}}%
\pgfpathlineto{\pgfqpoint{1.422664in}{1.035095in}}%
\pgfpathlineto{\pgfqpoint{1.384821in}{1.067085in}}%
\pgfpathlineto{\pgfqpoint{1.345416in}{1.110156in}}%
\pgfpathclose%
\pgfusepath{fill}%
\end{pgfscope}%
\begin{pgfscope}%
\pgfpathrectangle{\pgfqpoint{0.150000in}{0.150000in}}{\pgfqpoint{2.700000in}{1.950000in}}%
\pgfusepath{clip}%
\pgfsetbuttcap%
\pgfsetroundjoin%
\definecolor{currentfill}{rgb}{0.925919,0.865579,0.870358}%
\pgfsetfillcolor{currentfill}%
\pgfsetlinewidth{0.000000pt}%
\definecolor{currentstroke}{rgb}{0.000000,0.000000,0.000000}%
\pgfsetstrokecolor{currentstroke}%
\pgfsetdash{}{0pt}%
\pgfpathmoveto{\pgfqpoint{1.727387in}{0.866821in}}%
\pgfpathlineto{\pgfqpoint{1.764871in}{0.874524in}}%
\pgfpathlineto{\pgfqpoint{1.726456in}{0.894437in}}%
\pgfpathlineto{\pgfqpoint{1.688925in}{0.886808in}}%
\pgfpathclose%
\pgfusepath{fill}%
\end{pgfscope}%
\begin{pgfscope}%
\pgfpathrectangle{\pgfqpoint{0.150000in}{0.150000in}}{\pgfqpoint{2.700000in}{1.950000in}}%
\pgfusepath{clip}%
\pgfsetbuttcap%
\pgfsetroundjoin%
\definecolor{currentfill}{rgb}{0.906924,0.831112,0.837117}%
\pgfsetfillcolor{currentfill}%
\pgfsetlinewidth{0.000000pt}%
\definecolor{currentstroke}{rgb}{0.000000,0.000000,0.000000}%
\pgfsetstrokecolor{currentstroke}%
\pgfsetdash{}{0pt}%
\pgfpathmoveto{\pgfqpoint{1.918536in}{0.834478in}}%
\pgfpathlineto{\pgfqpoint{1.955463in}{0.842285in}}%
\pgfpathlineto{\pgfqpoint{1.916219in}{0.850045in}}%
\pgfpathlineto{\pgfqpoint{1.879285in}{0.842285in}}%
\pgfpathclose%
\pgfusepath{fill}%
\end{pgfscope}%
\begin{pgfscope}%
\pgfpathrectangle{\pgfqpoint{0.150000in}{0.150000in}}{\pgfqpoint{2.700000in}{1.950000in}}%
\pgfusepath{clip}%
\pgfsetbuttcap%
\pgfsetroundjoin%
\definecolor{currentfill}{rgb}{0.998100,0.996553,0.996676}%
\pgfsetfillcolor{currentfill}%
\pgfsetlinewidth{0.000000pt}%
\definecolor{currentstroke}{rgb}{0.000000,0.000000,0.000000}%
\pgfsetstrokecolor{currentstroke}%
\pgfsetdash{}{0pt}%
\pgfpathmoveto{\pgfqpoint{0.964588in}{1.008206in}}%
\pgfpathlineto{\pgfqpoint{1.003062in}{1.040369in}}%
\pgfpathlineto{\pgfqpoint{0.965331in}{1.072491in}}%
\pgfpathlineto{\pgfqpoint{0.927587in}{1.027905in}}%
\pgfpathclose%
\pgfusepath{fill}%
\end{pgfscope}%
\begin{pgfscope}%
\pgfpathrectangle{\pgfqpoint{0.150000in}{0.150000in}}{\pgfqpoint{2.700000in}{1.950000in}}%
\pgfusepath{clip}%
\pgfsetbuttcap%
\pgfsetroundjoin%
\definecolor{currentfill}{rgb}{0.944914,0.900046,0.903600}%
\pgfsetfillcolor{currentfill}%
\pgfsetlinewidth{0.000000pt}%
\definecolor{currentstroke}{rgb}{0.000000,0.000000,0.000000}%
\pgfsetstrokecolor{currentstroke}%
\pgfsetdash{}{0pt}%
\pgfpathmoveto{\pgfqpoint{1.612889in}{0.911464in}}%
\pgfpathlineto{\pgfqpoint{1.650605in}{0.906722in}}%
\pgfpathlineto{\pgfqpoint{1.612516in}{0.938878in}}%
\pgfpathlineto{\pgfqpoint{1.574617in}{0.931380in}}%
\pgfpathclose%
\pgfusepath{fill}%
\end{pgfscope}%
\begin{pgfscope}%
\pgfpathrectangle{\pgfqpoint{0.150000in}{0.150000in}}{\pgfqpoint{2.700000in}{1.950000in}}%
\pgfusepath{clip}%
\pgfsetbuttcap%
\pgfsetroundjoin%
\definecolor{currentfill}{rgb}{0.940916,0.948192,0.958379}%
\pgfsetfillcolor{currentfill}%
\pgfsetlinewidth{0.000000pt}%
\definecolor{currentstroke}{rgb}{0.000000,0.000000,0.000000}%
\pgfsetstrokecolor{currentstroke}%
\pgfsetdash{}{0pt}%
\pgfpathmoveto{\pgfqpoint{1.041485in}{1.072491in}}%
\pgfpathlineto{\pgfqpoint{1.078764in}{1.129711in}}%
\pgfpathlineto{\pgfqpoint{1.040943in}{1.161810in}}%
\pgfpathlineto{\pgfqpoint{1.003754in}{1.104571in}}%
\pgfpathclose%
\pgfusepath{fill}%
\end{pgfscope}%
\begin{pgfscope}%
\pgfpathrectangle{\pgfqpoint{0.150000in}{0.150000in}}{\pgfqpoint{2.700000in}{1.950000in}}%
\pgfusepath{clip}%
\pgfsetbuttcap%
\pgfsetroundjoin%
\definecolor{currentfill}{rgb}{0.982904,0.968980,0.970083}%
\pgfsetfillcolor{currentfill}%
\pgfsetlinewidth{0.000000pt}%
\definecolor{currentstroke}{rgb}{0.000000,0.000000,0.000000}%
\pgfsetstrokecolor{currentstroke}%
\pgfsetdash{}{0pt}%
\pgfpathmoveto{\pgfqpoint{0.926065in}{0.976000in}}%
\pgfpathlineto{\pgfqpoint{0.964588in}{1.008206in}}%
\pgfpathlineto{\pgfqpoint{0.927587in}{1.027905in}}%
\pgfpathlineto{\pgfqpoint{0.889110in}{0.995771in}}%
\pgfpathclose%
\pgfusepath{fill}%
\end{pgfscope}%
\begin{pgfscope}%
\pgfpathrectangle{\pgfqpoint{0.150000in}{0.150000in}}{\pgfqpoint{2.700000in}{1.950000in}}%
\pgfusepath{clip}%
\pgfsetbuttcap%
\pgfsetroundjoin%
\definecolor{currentfill}{rgb}{0.914522,0.844899,0.850414}%
\pgfsetfillcolor{currentfill}%
\pgfsetlinewidth{0.000000pt}%
\definecolor{currentstroke}{rgb}{0.000000,0.000000,0.000000}%
\pgfsetstrokecolor{currentstroke}%
\pgfsetdash{}{0pt}%
\pgfpathmoveto{\pgfqpoint{1.842126in}{0.834478in}}%
\pgfpathlineto{\pgfqpoint{1.879285in}{0.842285in}}%
\pgfpathlineto{\pgfqpoint{1.840635in}{0.862270in}}%
\pgfpathlineto{\pgfqpoint{1.803427in}{0.854538in}}%
\pgfpathclose%
\pgfusepath{fill}%
\end{pgfscope}%
\begin{pgfscope}%
\pgfpathrectangle{\pgfqpoint{0.150000in}{0.150000in}}{\pgfqpoint{2.700000in}{1.950000in}}%
\pgfusepath{clip}%
\pgfsetbuttcap%
\pgfsetroundjoin%
\definecolor{currentfill}{rgb}{0.996890,0.997273,0.997809}%
\pgfsetfillcolor{currentfill}%
\pgfsetlinewidth{0.000000pt}%
\definecolor{currentstroke}{rgb}{0.000000,0.000000,0.000000}%
\pgfsetstrokecolor{currentstroke}%
\pgfsetdash{}{0pt}%
\pgfpathmoveto{\pgfqpoint{1.421695in}{1.045688in}}%
\pgfpathlineto{\pgfqpoint{1.460465in}{1.015470in}}%
\pgfpathlineto{\pgfqpoint{1.422664in}{1.035095in}}%
\pgfpathlineto{\pgfqpoint{1.383530in}{1.077943in}}%
\pgfpathclose%
\pgfusepath{fill}%
\end{pgfscope}%
\begin{pgfscope}%
\pgfpathrectangle{\pgfqpoint{0.150000in}{0.150000in}}{\pgfqpoint{2.700000in}{1.950000in}}%
\pgfusepath{clip}%
\pgfsetbuttcap%
\pgfsetroundjoin%
\definecolor{currentfill}{rgb}{0.990671,0.991820,0.993428}%
\pgfsetfillcolor{currentfill}%
\pgfsetlinewidth{0.000000pt}%
\definecolor{currentstroke}{rgb}{0.000000,0.000000,0.000000}%
\pgfsetstrokecolor{currentstroke}%
\pgfsetdash{}{0pt}%
\pgfpathmoveto{\pgfqpoint{1.040247in}{1.020670in}}%
\pgfpathlineto{\pgfqpoint{1.078717in}{1.052864in}}%
\pgfpathlineto{\pgfqpoint{1.041485in}{1.072491in}}%
\pgfpathlineto{\pgfqpoint{1.003062in}{1.040369in}}%
\pgfpathclose%
\pgfusepath{fill}%
\end{pgfscope}%
\begin{pgfscope}%
\pgfpathrectangle{\pgfqpoint{0.150000in}{0.150000in}}{\pgfqpoint{2.700000in}{1.950000in}}%
\pgfusepath{clip}%
\pgfsetbuttcap%
\pgfsetroundjoin%
\definecolor{currentfill}{rgb}{0.959574,0.964553,0.971523}%
\pgfsetfillcolor{currentfill}%
\pgfsetlinewidth{0.000000pt}%
\definecolor{currentstroke}{rgb}{0.000000,0.000000,0.000000}%
\pgfsetstrokecolor{currentstroke}%
\pgfsetdash{}{0pt}%
\pgfpathmoveto{\pgfqpoint{1.078717in}{1.052864in}}%
\pgfpathlineto{\pgfqpoint{1.116635in}{1.097571in}}%
\pgfpathlineto{\pgfqpoint{1.078764in}{1.129711in}}%
\pgfpathlineto{\pgfqpoint{1.041485in}{1.072491in}}%
\pgfpathclose%
\pgfusepath{fill}%
\end{pgfscope}%
\begin{pgfscope}%
\pgfpathrectangle{\pgfqpoint{0.150000in}{0.150000in}}{\pgfqpoint{2.700000in}{1.950000in}}%
\pgfusepath{clip}%
\pgfsetbuttcap%
\pgfsetroundjoin%
\definecolor{currentfill}{rgb}{0.994301,0.989660,0.990028}%
\pgfsetfillcolor{currentfill}%
\pgfsetlinewidth{0.000000pt}%
\definecolor{currentstroke}{rgb}{0.000000,0.000000,0.000000}%
\pgfsetstrokecolor{currentstroke}%
\pgfsetdash{}{0pt}%
\pgfpathmoveto{\pgfqpoint{1.001727in}{0.988434in}}%
\pgfpathlineto{\pgfqpoint{1.040247in}{1.020670in}}%
\pgfpathlineto{\pgfqpoint{1.003062in}{1.040369in}}%
\pgfpathlineto{\pgfqpoint{0.964588in}{1.008206in}}%
\pgfpathclose%
\pgfusepath{fill}%
\end{pgfscope}%
\begin{pgfscope}%
\pgfpathrectangle{\pgfqpoint{0.150000in}{0.150000in}}{\pgfqpoint{2.700000in}{1.950000in}}%
\pgfusepath{clip}%
\pgfsetbuttcap%
\pgfsetroundjoin%
\definecolor{currentfill}{rgb}{0.994301,0.989660,0.990028}%
\pgfsetfillcolor{currentfill}%
\pgfsetlinewidth{0.000000pt}%
\definecolor{currentstroke}{rgb}{0.000000,0.000000,0.000000}%
\pgfsetstrokecolor{currentstroke}%
\pgfsetdash{}{0pt}%
\pgfpathmoveto{\pgfqpoint{1.459909in}{1.013391in}}%
\pgfpathlineto{\pgfqpoint{1.498451in}{0.983367in}}%
\pgfpathlineto{\pgfqpoint{1.460465in}{1.015470in}}%
\pgfpathlineto{\pgfqpoint{1.421695in}{1.045688in}}%
\pgfpathclose%
\pgfusepath{fill}%
\end{pgfscope}%
\begin{pgfscope}%
\pgfpathrectangle{\pgfqpoint{0.150000in}{0.150000in}}{\pgfqpoint{2.700000in}{1.950000in}}%
\pgfusepath{clip}%
\pgfsetbuttcap%
\pgfsetroundjoin%
\definecolor{currentfill}{rgb}{0.941115,0.893153,0.896952}%
\pgfsetfillcolor{currentfill}%
\pgfsetlinewidth{0.000000pt}%
\definecolor{currentstroke}{rgb}{0.000000,0.000000,0.000000}%
\pgfsetstrokecolor{currentstroke}%
\pgfsetdash{}{0pt}%
\pgfpathmoveto{\pgfqpoint{1.651302in}{0.891474in}}%
\pgfpathlineto{\pgfqpoint{1.688925in}{0.886808in}}%
\pgfpathlineto{\pgfqpoint{1.650605in}{0.906722in}}%
\pgfpathlineto{\pgfqpoint{1.612889in}{0.911464in}}%
\pgfpathclose%
\pgfusepath{fill}%
\end{pgfscope}%
\begin{pgfscope}%
\pgfpathrectangle{\pgfqpoint{0.150000in}{0.150000in}}{\pgfqpoint{2.700000in}{1.950000in}}%
\pgfusepath{clip}%
\pgfsetbuttcap%
\pgfsetroundjoin%
\definecolor{currentfill}{rgb}{0.654825,0.697335,0.756847}%
\pgfsetfillcolor{currentfill}%
\pgfsetlinewidth{0.000000pt}%
\definecolor{currentstroke}{rgb}{0.000000,0.000000,0.000000}%
\pgfsetstrokecolor{currentstroke}%
\pgfsetdash{}{0pt}%
\pgfpathmoveto{\pgfqpoint{0.778467in}{1.372515in}}%
\pgfpathlineto{\pgfqpoint{0.806412in}{1.560548in}}%
\pgfpathlineto{\pgfqpoint{0.767563in}{1.605861in}}%
\pgfpathlineto{\pgfqpoint{0.741080in}{1.404253in}}%
\pgfpathclose%
\pgfusepath{fill}%
\end{pgfscope}%
\begin{pgfscope}%
\pgfpathrectangle{\pgfqpoint{0.150000in}{0.150000in}}{\pgfqpoint{2.700000in}{1.950000in}}%
\pgfusepath{clip}%
\pgfsetbuttcap%
\pgfsetroundjoin%
\definecolor{currentfill}{rgb}{0.925919,0.865579,0.870358}%
\pgfsetfillcolor{currentfill}%
\pgfsetlinewidth{0.000000pt}%
\definecolor{currentstroke}{rgb}{0.000000,0.000000,0.000000}%
\pgfsetstrokecolor{currentstroke}%
\pgfsetdash{}{0pt}%
\pgfpathmoveto{\pgfqpoint{1.766268in}{0.859070in}}%
\pgfpathlineto{\pgfqpoint{1.803427in}{0.854538in}}%
\pgfpathlineto{\pgfqpoint{1.764871in}{0.874524in}}%
\pgfpathlineto{\pgfqpoint{1.727387in}{0.866821in}}%
\pgfpathclose%
\pgfusepath{fill}%
\end{pgfscope}%
\begin{pgfscope}%
\pgfpathrectangle{\pgfqpoint{0.150000in}{0.150000in}}{\pgfqpoint{2.700000in}{1.950000in}}%
\pgfusepath{clip}%
\pgfsetbuttcap%
\pgfsetroundjoin%
\definecolor{currentfill}{rgb}{0.819547,0.672564,0.684206}%
\pgfsetfillcolor{currentfill}%
\pgfsetlinewidth{0.000000pt}%
\definecolor{currentstroke}{rgb}{0.000000,0.000000,0.000000}%
\pgfsetstrokecolor{currentstroke}%
\pgfsetdash{}{0pt}%
\pgfpathmoveto{\pgfqpoint{1.307735in}{0.579497in}}%
\pgfpathlineto{\pgfqpoint{1.345527in}{0.635929in}}%
\pgfpathlineto{\pgfqpoint{1.307210in}{0.680459in}}%
\pgfpathlineto{\pgfqpoint{1.269464in}{0.623919in}}%
\pgfpathclose%
\pgfusepath{fill}%
\end{pgfscope}%
\begin{pgfscope}%
\pgfpathrectangle{\pgfqpoint{0.150000in}{0.150000in}}{\pgfqpoint{2.700000in}{1.950000in}}%
\pgfusepath{clip}%
\pgfsetbuttcap%
\pgfsetroundjoin%
\definecolor{currentfill}{rgb}{0.592632,0.642800,0.713036}%
\pgfsetfillcolor{currentfill}%
\pgfsetlinewidth{0.000000pt}%
\definecolor{currentstroke}{rgb}{0.000000,0.000000,0.000000}%
\pgfsetstrokecolor{currentstroke}%
\pgfsetdash{}{0pt}%
\pgfpathmoveto{\pgfqpoint{0.843618in}{1.619152in}}%
\pgfpathlineto{\pgfqpoint{0.892651in}{1.480519in}}%
\pgfpathlineto{\pgfqpoint{0.855219in}{1.512164in}}%
\pgfpathlineto{\pgfqpoint{0.805605in}{1.651223in}}%
\pgfpathclose%
\pgfusepath{fill}%
\end{pgfscope}%
\begin{pgfscope}%
\pgfpathrectangle{\pgfqpoint{0.150000in}{0.150000in}}{\pgfqpoint{2.700000in}{1.950000in}}%
\pgfusepath{clip}%
\pgfsetbuttcap%
\pgfsetroundjoin%
\definecolor{currentfill}{rgb}{0.972013,0.975460,0.980285}%
\pgfsetfillcolor{currentfill}%
\pgfsetlinewidth{0.000000pt}%
\definecolor{currentstroke}{rgb}{0.000000,0.000000,0.000000}%
\pgfsetstrokecolor{currentstroke}%
\pgfsetdash{}{0pt}%
\pgfpathmoveto{\pgfqpoint{1.116592in}{1.020670in}}%
\pgfpathlineto{\pgfqpoint{1.154555in}{1.065388in}}%
\pgfpathlineto{\pgfqpoint{1.116635in}{1.097571in}}%
\pgfpathlineto{\pgfqpoint{1.078717in}{1.052864in}}%
\pgfpathclose%
\pgfusepath{fill}%
\end{pgfscope}%
\begin{pgfscope}%
\pgfpathrectangle{\pgfqpoint{0.150000in}{0.150000in}}{\pgfqpoint{2.700000in}{1.950000in}}%
\pgfusepath{clip}%
\pgfsetbuttcap%
\pgfsetroundjoin%
\definecolor{currentfill}{rgb}{0.996890,0.997273,0.997809}%
\pgfsetfillcolor{currentfill}%
\pgfsetlinewidth{0.000000pt}%
\definecolor{currentstroke}{rgb}{0.000000,0.000000,0.000000}%
\pgfsetstrokecolor{currentstroke}%
\pgfsetdash{}{0pt}%
\pgfpathmoveto{\pgfqpoint{1.078121in}{0.988434in}}%
\pgfpathlineto{\pgfqpoint{1.116592in}{1.020670in}}%
\pgfpathlineto{\pgfqpoint{1.078717in}{1.052864in}}%
\pgfpathlineto{\pgfqpoint{1.040247in}{1.020670in}}%
\pgfpathclose%
\pgfusepath{fill}%
\end{pgfscope}%
\begin{pgfscope}%
\pgfpathrectangle{\pgfqpoint{0.150000in}{0.150000in}}{\pgfqpoint{2.700000in}{1.950000in}}%
\pgfusepath{clip}%
\pgfsetbuttcap%
\pgfsetroundjoin%
\definecolor{currentfill}{rgb}{0.982904,0.968980,0.970083}%
\pgfsetfillcolor{currentfill}%
\pgfsetlinewidth{0.000000pt}%
\definecolor{currentstroke}{rgb}{0.000000,0.000000,0.000000}%
\pgfsetstrokecolor{currentstroke}%
\pgfsetdash{}{0pt}%
\pgfpathmoveto{\pgfqpoint{0.963157in}{0.956156in}}%
\pgfpathlineto{\pgfqpoint{1.001727in}{0.988434in}}%
\pgfpathlineto{\pgfqpoint{0.964588in}{1.008206in}}%
\pgfpathlineto{\pgfqpoint{0.926065in}{0.976000in}}%
\pgfpathclose%
\pgfusepath{fill}%
\end{pgfscope}%
\begin{pgfscope}%
\pgfpathrectangle{\pgfqpoint{0.150000in}{0.150000in}}{\pgfqpoint{2.700000in}{1.950000in}}%
\pgfusepath{clip}%
\pgfsetbuttcap%
\pgfsetroundjoin%
\definecolor{currentfill}{rgb}{0.679703,0.719148,0.774372}%
\pgfsetfillcolor{currentfill}%
\pgfsetlinewidth{0.000000pt}%
\definecolor{currentstroke}{rgb}{0.000000,0.000000,0.000000}%
\pgfsetstrokecolor{currentstroke}%
\pgfsetdash{}{0pt}%
\pgfpathmoveto{\pgfqpoint{0.815902in}{1.340736in}}%
\pgfpathlineto{\pgfqpoint{0.844384in}{1.528452in}}%
\pgfpathlineto{\pgfqpoint{0.806412in}{1.560548in}}%
\pgfpathlineto{\pgfqpoint{0.778467in}{1.372515in}}%
\pgfpathclose%
\pgfusepath{fill}%
\end{pgfscope}%
\begin{pgfscope}%
\pgfpathrectangle{\pgfqpoint{0.150000in}{0.150000in}}{\pgfqpoint{2.700000in}{1.950000in}}%
\pgfusepath{clip}%
\pgfsetbuttcap%
\pgfsetroundjoin%
\definecolor{currentfill}{rgb}{0.914522,0.844899,0.850414}%
\pgfsetfillcolor{currentfill}%
\pgfsetlinewidth{0.000000pt}%
\definecolor{currentstroke}{rgb}{0.000000,0.000000,0.000000}%
\pgfsetstrokecolor{currentstroke}%
\pgfsetdash{}{0pt}%
\pgfpathmoveto{\pgfqpoint{1.957520in}{0.814343in}}%
\pgfpathlineto{\pgfqpoint{1.994396in}{0.822226in}}%
\pgfpathlineto{\pgfqpoint{1.955463in}{0.842285in}}%
\pgfpathlineto{\pgfqpoint{1.918536in}{0.834478in}}%
\pgfpathclose%
\pgfusepath{fill}%
\end{pgfscope}%
\begin{pgfscope}%
\pgfpathrectangle{\pgfqpoint{0.150000in}{0.150000in}}{\pgfqpoint{2.700000in}{1.950000in}}%
\pgfusepath{clip}%
\pgfsetbuttcap%
\pgfsetroundjoin%
\definecolor{currentfill}{rgb}{0.617509,0.664614,0.730561}%
\pgfsetfillcolor{currentfill}%
\pgfsetlinewidth{0.000000pt}%
\definecolor{currentstroke}{rgb}{0.000000,0.000000,0.000000}%
\pgfsetstrokecolor{currentstroke}%
\pgfsetdash{}{0pt}%
\pgfpathmoveto{\pgfqpoint{0.882473in}{1.573777in}}%
\pgfpathlineto{\pgfqpoint{0.930133in}{1.448834in}}%
\pgfpathlineto{\pgfqpoint{0.892651in}{1.480519in}}%
\pgfpathlineto{\pgfqpoint{0.843618in}{1.619152in}}%
\pgfpathclose%
\pgfusepath{fill}%
\end{pgfscope}%
\begin{pgfscope}%
\pgfpathrectangle{\pgfqpoint{0.150000in}{0.150000in}}{\pgfqpoint{2.700000in}{1.950000in}}%
\pgfusepath{clip}%
\pgfsetbuttcap%
\pgfsetroundjoin%
\definecolor{currentfill}{rgb}{0.986703,0.975873,0.976731}%
\pgfsetfillcolor{currentfill}%
\pgfsetlinewidth{0.000000pt}%
\definecolor{currentstroke}{rgb}{0.000000,0.000000,0.000000}%
\pgfsetstrokecolor{currentstroke}%
\pgfsetdash{}{0pt}%
\pgfpathmoveto{\pgfqpoint{1.498173in}{0.981052in}}%
\pgfpathlineto{\pgfqpoint{1.536486in}{0.963597in}}%
\pgfpathlineto{\pgfqpoint{1.498451in}{0.983367in}}%
\pgfpathlineto{\pgfqpoint{1.459909in}{1.013391in}}%
\pgfpathclose%
\pgfusepath{fill}%
\end{pgfscope}%
\begin{pgfscope}%
\pgfpathrectangle{\pgfqpoint{0.150000in}{0.150000in}}{\pgfqpoint{2.700000in}{1.950000in}}%
\pgfusepath{clip}%
\pgfsetbuttcap%
\pgfsetroundjoin%
\definecolor{currentfill}{rgb}{0.698361,0.735509,0.787515}%
\pgfsetfillcolor{currentfill}%
\pgfsetlinewidth{0.000000pt}%
\definecolor{currentstroke}{rgb}{0.000000,0.000000,0.000000}%
\pgfsetstrokecolor{currentstroke}%
\pgfsetdash{}{0pt}%
\pgfpathmoveto{\pgfqpoint{0.853386in}{1.308916in}}%
\pgfpathlineto{\pgfqpoint{0.883194in}{1.483178in}}%
\pgfpathlineto{\pgfqpoint{0.844384in}{1.528452in}}%
\pgfpathlineto{\pgfqpoint{0.815902in}{1.340736in}}%
\pgfpathclose%
\pgfusepath{fill}%
\end{pgfscope}%
\begin{pgfscope}%
\pgfpathrectangle{\pgfqpoint{0.150000in}{0.150000in}}{\pgfqpoint{2.700000in}{1.950000in}}%
\pgfusepath{clip}%
\pgfsetbuttcap%
\pgfsetroundjoin%
\definecolor{currentfill}{rgb}{0.808150,0.651884,0.664262}%
\pgfsetfillcolor{currentfill}%
\pgfsetlinewidth{0.000000pt}%
\definecolor{currentstroke}{rgb}{0.000000,0.000000,0.000000}%
\pgfsetstrokecolor{currentstroke}%
\pgfsetdash{}{0pt}%
\pgfpathmoveto{\pgfqpoint{1.345508in}{0.558933in}}%
\pgfpathlineto{\pgfqpoint{1.383436in}{0.615433in}}%
\pgfpathlineto{\pgfqpoint{1.345527in}{0.635929in}}%
\pgfpathlineto{\pgfqpoint{1.307735in}{0.579497in}}%
\pgfpathclose%
\pgfusepath{fill}%
\end{pgfscope}%
\begin{pgfscope}%
\pgfpathrectangle{\pgfqpoint{0.150000in}{0.150000in}}{\pgfqpoint{2.700000in}{1.950000in}}%
\pgfusepath{clip}%
\pgfsetbuttcap%
\pgfsetroundjoin%
\definecolor{currentfill}{rgb}{0.990671,0.991820,0.993428}%
\pgfsetfillcolor{currentfill}%
\pgfsetlinewidth{0.000000pt}%
\definecolor{currentstroke}{rgb}{0.000000,0.000000,0.000000}%
\pgfsetstrokecolor{currentstroke}%
\pgfsetdash{}{0pt}%
\pgfpathmoveto{\pgfqpoint{1.154515in}{0.988434in}}%
\pgfpathlineto{\pgfqpoint{1.192524in}{1.033164in}}%
\pgfpathlineto{\pgfqpoint{1.154555in}{1.065388in}}%
\pgfpathlineto{\pgfqpoint{1.116592in}{1.020670in}}%
\pgfpathclose%
\pgfusepath{fill}%
\end{pgfscope}%
\begin{pgfscope}%
\pgfpathrectangle{\pgfqpoint{0.150000in}{0.150000in}}{\pgfqpoint{2.700000in}{1.950000in}}%
\pgfusepath{clip}%
\pgfsetbuttcap%
\pgfsetroundjoin%
\definecolor{currentfill}{rgb}{0.642387,0.686428,0.748085}%
\pgfsetfillcolor{currentfill}%
\pgfsetlinewidth{0.000000pt}%
\definecolor{currentstroke}{rgb}{0.000000,0.000000,0.000000}%
\pgfsetstrokecolor{currentstroke}%
\pgfsetdash{}{0pt}%
\pgfpathmoveto{\pgfqpoint{0.920540in}{1.541651in}}%
\pgfpathlineto{\pgfqpoint{0.967662in}{1.417107in}}%
\pgfpathlineto{\pgfqpoint{0.930133in}{1.448834in}}%
\pgfpathlineto{\pgfqpoint{0.882473in}{1.573777in}}%
\pgfpathclose%
\pgfusepath{fill}%
\end{pgfscope}%
\begin{pgfscope}%
\pgfpathrectangle{\pgfqpoint{0.150000in}{0.150000in}}{\pgfqpoint{2.700000in}{1.950000in}}%
\pgfusepath{clip}%
\pgfsetbuttcap%
\pgfsetroundjoin%
\definecolor{currentfill}{rgb}{0.986703,0.975873,0.976731}%
\pgfsetfillcolor{currentfill}%
\pgfsetlinewidth{0.000000pt}%
\definecolor{currentstroke}{rgb}{0.000000,0.000000,0.000000}%
\pgfsetstrokecolor{currentstroke}%
\pgfsetdash{}{0pt}%
\pgfpathmoveto{\pgfqpoint{1.116045in}{0.956156in}}%
\pgfpathlineto{\pgfqpoint{1.154515in}{0.988434in}}%
\pgfpathlineto{\pgfqpoint{1.116592in}{1.020670in}}%
\pgfpathlineto{\pgfqpoint{1.078121in}{0.988434in}}%
\pgfpathclose%
\pgfusepath{fill}%
\end{pgfscope}%
\begin{pgfscope}%
\pgfpathrectangle{\pgfqpoint{0.150000in}{0.150000in}}{\pgfqpoint{2.700000in}{1.950000in}}%
\pgfusepath{clip}%
\pgfsetbuttcap%
\pgfsetroundjoin%
\definecolor{currentfill}{rgb}{0.986703,0.975873,0.976731}%
\pgfsetfillcolor{currentfill}%
\pgfsetlinewidth{0.000000pt}%
\definecolor{currentstroke}{rgb}{0.000000,0.000000,0.000000}%
\pgfsetstrokecolor{currentstroke}%
\pgfsetdash{}{0pt}%
\pgfpathmoveto{\pgfqpoint{1.039601in}{0.956156in}}%
\pgfpathlineto{\pgfqpoint{1.078121in}{0.988434in}}%
\pgfpathlineto{\pgfqpoint{1.040247in}{1.020670in}}%
\pgfpathlineto{\pgfqpoint{1.001727in}{0.988434in}}%
\pgfpathclose%
\pgfusepath{fill}%
\end{pgfscope}%
\begin{pgfscope}%
\pgfpathrectangle{\pgfqpoint{0.150000in}{0.150000in}}{\pgfqpoint{2.700000in}{1.950000in}}%
\pgfusepath{clip}%
\pgfsetbuttcap%
\pgfsetroundjoin%
\definecolor{currentfill}{rgb}{0.846140,0.720818,0.730744}%
\pgfsetfillcolor{currentfill}%
\pgfsetlinewidth{0.000000pt}%
\definecolor{currentstroke}{rgb}{0.000000,0.000000,0.000000}%
\pgfsetstrokecolor{currentstroke}%
\pgfsetdash{}{0pt}%
\pgfpathmoveto{\pgfqpoint{1.269464in}{0.623919in}}%
\pgfpathlineto{\pgfqpoint{1.307210in}{0.680459in}}%
\pgfpathlineto{\pgfqpoint{1.268850in}{0.725038in}}%
\pgfpathlineto{\pgfqpoint{1.230784in}{0.680459in}}%
\pgfpathclose%
\pgfusepath{fill}%
\end{pgfscope}%
\begin{pgfscope}%
\pgfpathrectangle{\pgfqpoint{0.150000in}{0.150000in}}{\pgfqpoint{2.700000in}{1.950000in}}%
\pgfusepath{clip}%
\pgfsetbuttcap%
\pgfsetroundjoin%
\definecolor{currentfill}{rgb}{0.723238,0.757322,0.805040}%
\pgfsetfillcolor{currentfill}%
\pgfsetlinewidth{0.000000pt}%
\definecolor{currentstroke}{rgb}{0.000000,0.000000,0.000000}%
\pgfsetstrokecolor{currentstroke}%
\pgfsetdash{}{0pt}%
\pgfpathmoveto{\pgfqpoint{0.890919in}{1.277054in}}%
\pgfpathlineto{\pgfqpoint{0.921220in}{1.451028in}}%
\pgfpathlineto{\pgfqpoint{0.883194in}{1.483178in}}%
\pgfpathlineto{\pgfqpoint{0.853386in}{1.308916in}}%
\pgfpathclose%
\pgfusepath{fill}%
\end{pgfscope}%
\begin{pgfscope}%
\pgfpathrectangle{\pgfqpoint{0.150000in}{0.150000in}}{\pgfqpoint{2.700000in}{1.950000in}}%
\pgfusepath{clip}%
\pgfsetbuttcap%
\pgfsetroundjoin%
\definecolor{currentfill}{rgb}{0.918321,0.851792,0.857062}%
\pgfsetfillcolor{currentfill}%
\pgfsetlinewidth{0.000000pt}%
\definecolor{currentstroke}{rgb}{0.000000,0.000000,0.000000}%
\pgfsetstrokecolor{currentstroke}%
\pgfsetdash{}{0pt}%
\pgfpathmoveto{\pgfqpoint{1.881383in}{0.826623in}}%
\pgfpathlineto{\pgfqpoint{1.918536in}{0.834478in}}%
\pgfpathlineto{\pgfqpoint{1.879285in}{0.842285in}}%
\pgfpathlineto{\pgfqpoint{1.842126in}{0.834478in}}%
\pgfpathclose%
\pgfusepath{fill}%
\end{pgfscope}%
\begin{pgfscope}%
\pgfpathrectangle{\pgfqpoint{0.150000in}{0.150000in}}{\pgfqpoint{2.700000in}{1.950000in}}%
\pgfusepath{clip}%
\pgfsetbuttcap%
\pgfsetroundjoin%
\definecolor{currentfill}{rgb}{0.661045,0.702788,0.761229}%
\pgfsetfillcolor{currentfill}%
\pgfsetlinewidth{0.000000pt}%
\definecolor{currentstroke}{rgb}{0.000000,0.000000,0.000000}%
\pgfsetstrokecolor{currentstroke}%
\pgfsetdash{}{0pt}%
\pgfpathmoveto{\pgfqpoint{0.959356in}{1.496315in}}%
\pgfpathlineto{\pgfqpoint{1.005240in}{1.385340in}}%
\pgfpathlineto{\pgfqpoint{0.967662in}{1.417107in}}%
\pgfpathlineto{\pgfqpoint{0.920540in}{1.541651in}}%
\pgfpathclose%
\pgfusepath{fill}%
\end{pgfscope}%
\begin{pgfscope}%
\pgfpathrectangle{\pgfqpoint{0.150000in}{0.150000in}}{\pgfqpoint{2.700000in}{1.950000in}}%
\pgfusepath{clip}%
\pgfsetbuttcap%
\pgfsetroundjoin%
\definecolor{currentfill}{rgb}{0.994301,0.989660,0.990028}%
\pgfsetfillcolor{currentfill}%
\pgfsetlinewidth{0.000000pt}%
\definecolor{currentstroke}{rgb}{0.000000,0.000000,0.000000}%
\pgfsetstrokecolor{currentstroke}%
\pgfsetdash{}{0pt}%
\pgfpathmoveto{\pgfqpoint{1.192489in}{0.956156in}}%
\pgfpathlineto{\pgfqpoint{1.230543in}{1.000897in}}%
\pgfpathlineto{\pgfqpoint{1.192524in}{1.033164in}}%
\pgfpathlineto{\pgfqpoint{1.154515in}{0.988434in}}%
\pgfpathclose%
\pgfusepath{fill}%
\end{pgfscope}%
\begin{pgfscope}%
\pgfpathrectangle{\pgfqpoint{0.150000in}{0.150000in}}{\pgfqpoint{2.700000in}{1.950000in}}%
\pgfusepath{clip}%
\pgfsetbuttcap%
\pgfsetroundjoin%
\definecolor{currentfill}{rgb}{0.975306,0.955193,0.956786}%
\pgfsetfillcolor{currentfill}%
\pgfsetlinewidth{0.000000pt}%
\definecolor{currentstroke}{rgb}{0.000000,0.000000,0.000000}%
\pgfsetstrokecolor{currentstroke}%
\pgfsetdash{}{0pt}%
\pgfpathmoveto{\pgfqpoint{1.001030in}{0.923836in}}%
\pgfpathlineto{\pgfqpoint{1.039601in}{0.956156in}}%
\pgfpathlineto{\pgfqpoint{1.001727in}{0.988434in}}%
\pgfpathlineto{\pgfqpoint{0.963157in}{0.956156in}}%
\pgfpathclose%
\pgfusepath{fill}%
\end{pgfscope}%
\begin{pgfscope}%
\pgfpathrectangle{\pgfqpoint{0.150000in}{0.150000in}}{\pgfqpoint{2.700000in}{1.950000in}}%
\pgfusepath{clip}%
\pgfsetbuttcap%
\pgfsetroundjoin%
\definecolor{currentfill}{rgb}{0.975306,0.955193,0.956786}%
\pgfsetfillcolor{currentfill}%
\pgfsetlinewidth{0.000000pt}%
\definecolor{currentstroke}{rgb}{0.000000,0.000000,0.000000}%
\pgfsetstrokecolor{currentstroke}%
\pgfsetdash{}{0pt}%
\pgfpathmoveto{\pgfqpoint{1.536486in}{0.948670in}}%
\pgfpathlineto{\pgfqpoint{1.574617in}{0.931380in}}%
\pgfpathlineto{\pgfqpoint{1.536486in}{0.963597in}}%
\pgfpathlineto{\pgfqpoint{1.498173in}{0.981052in}}%
\pgfpathclose%
\pgfusepath{fill}%
\end{pgfscope}%
\begin{pgfscope}%
\pgfpathrectangle{\pgfqpoint{0.150000in}{0.150000in}}{\pgfqpoint{2.700000in}{1.950000in}}%
\pgfusepath{clip}%
\pgfsetbuttcap%
\pgfsetroundjoin%
\definecolor{currentfill}{rgb}{0.685922,0.724602,0.778753}%
\pgfsetfillcolor{currentfill}%
\pgfsetlinewidth{0.000000pt}%
\definecolor{currentstroke}{rgb}{0.000000,0.000000,0.000000}%
\pgfsetstrokecolor{currentstroke}%
\pgfsetdash{}{0pt}%
\pgfpathmoveto{\pgfqpoint{0.997478in}{1.464135in}}%
\pgfpathlineto{\pgfqpoint{1.042867in}{1.353531in}}%
\pgfpathlineto{\pgfqpoint{1.005240in}{1.385340in}}%
\pgfpathlineto{\pgfqpoint{0.959356in}{1.496315in}}%
\pgfpathclose%
\pgfusepath{fill}%
\end{pgfscope}%
\begin{pgfscope}%
\pgfpathrectangle{\pgfqpoint{0.150000in}{0.150000in}}{\pgfqpoint{2.700000in}{1.950000in}}%
\pgfusepath{clip}%
\pgfsetbuttcap%
\pgfsetroundjoin%
\definecolor{currentfill}{rgb}{0.979105,0.962086,0.963434}%
\pgfsetfillcolor{currentfill}%
\pgfsetlinewidth{0.000000pt}%
\definecolor{currentstroke}{rgb}{0.000000,0.000000,0.000000}%
\pgfsetstrokecolor{currentstroke}%
\pgfsetdash{}{0pt}%
\pgfpathmoveto{\pgfqpoint{1.153557in}{0.936238in}}%
\pgfpathlineto{\pgfqpoint{1.192489in}{0.956156in}}%
\pgfpathlineto{\pgfqpoint{1.154515in}{0.988434in}}%
\pgfpathlineto{\pgfqpoint{1.116045in}{0.956156in}}%
\pgfpathclose%
\pgfusepath{fill}%
\end{pgfscope}%
\begin{pgfscope}%
\pgfpathrectangle{\pgfqpoint{0.150000in}{0.150000in}}{\pgfqpoint{2.700000in}{1.950000in}}%
\pgfusepath{clip}%
\pgfsetbuttcap%
\pgfsetroundjoin%
\definecolor{currentfill}{rgb}{0.979105,0.962086,0.963434}%
\pgfsetfillcolor{currentfill}%
\pgfsetlinewidth{0.000000pt}%
\definecolor{currentstroke}{rgb}{0.000000,0.000000,0.000000}%
\pgfsetstrokecolor{currentstroke}%
\pgfsetdash{}{0pt}%
\pgfpathmoveto{\pgfqpoint{1.076972in}{0.936238in}}%
\pgfpathlineto{\pgfqpoint{1.116045in}{0.956156in}}%
\pgfpathlineto{\pgfqpoint{1.078121in}{0.988434in}}%
\pgfpathlineto{\pgfqpoint{1.039601in}{0.956156in}}%
\pgfpathclose%
\pgfusepath{fill}%
\end{pgfscope}%
\begin{pgfscope}%
\pgfpathrectangle{\pgfqpoint{0.150000in}{0.150000in}}{\pgfqpoint{2.700000in}{1.950000in}}%
\pgfusepath{clip}%
\pgfsetbuttcap%
\pgfsetroundjoin%
\definecolor{currentfill}{rgb}{0.872733,0.769072,0.777282}%
\pgfsetfillcolor{currentfill}%
\pgfsetlinewidth{0.000000pt}%
\definecolor{currentstroke}{rgb}{0.000000,0.000000,0.000000}%
\pgfsetstrokecolor{currentstroke}%
\pgfsetdash{}{0pt}%
\pgfpathmoveto{\pgfqpoint{1.230784in}{0.680459in}}%
\pgfpathlineto{\pgfqpoint{1.268850in}{0.725038in}}%
\pgfpathlineto{\pgfqpoint{1.230448in}{0.769666in}}%
\pgfpathlineto{\pgfqpoint{1.192383in}{0.725038in}}%
\pgfpathclose%
\pgfusepath{fill}%
\end{pgfscope}%
\begin{pgfscope}%
\pgfpathrectangle{\pgfqpoint{0.150000in}{0.150000in}}{\pgfqpoint{2.700000in}{1.950000in}}%
\pgfusepath{clip}%
\pgfsetbuttcap%
\pgfsetroundjoin%
\definecolor{currentfill}{rgb}{0.710800,0.746415,0.796278}%
\pgfsetfillcolor{currentfill}%
\pgfsetlinewidth{0.000000pt}%
\definecolor{currentstroke}{rgb}{0.000000,0.000000,0.000000}%
\pgfsetstrokecolor{currentstroke}%
\pgfsetdash{}{0pt}%
\pgfpathmoveto{\pgfqpoint{1.036255in}{1.418836in}}%
\pgfpathlineto{\pgfqpoint{1.080543in}{1.321681in}}%
\pgfpathlineto{\pgfqpoint{1.042867in}{1.353531in}}%
\pgfpathlineto{\pgfqpoint{0.997478in}{1.464135in}}%
\pgfpathclose%
\pgfusepath{fill}%
\end{pgfscope}%
\begin{pgfscope}%
\pgfpathrectangle{\pgfqpoint{0.150000in}{0.150000in}}{\pgfqpoint{2.700000in}{1.950000in}}%
\pgfusepath{clip}%
\pgfsetbuttcap%
\pgfsetroundjoin%
\definecolor{currentfill}{rgb}{0.944914,0.900046,0.903600}%
\pgfsetfillcolor{currentfill}%
\pgfsetlinewidth{0.000000pt}%
\definecolor{currentstroke}{rgb}{0.000000,0.000000,0.000000}%
\pgfsetstrokecolor{currentstroke}%
\pgfsetdash{}{0pt}%
\pgfpathmoveto{\pgfqpoint{1.690043in}{0.883780in}}%
\pgfpathlineto{\pgfqpoint{1.727387in}{0.866821in}}%
\pgfpathlineto{\pgfqpoint{1.688925in}{0.886808in}}%
\pgfpathlineto{\pgfqpoint{1.651302in}{0.891474in}}%
\pgfpathclose%
\pgfusepath{fill}%
\end{pgfscope}%
\begin{pgfscope}%
\pgfpathrectangle{\pgfqpoint{0.150000in}{0.150000in}}{\pgfqpoint{2.700000in}{1.950000in}}%
\pgfusepath{clip}%
\pgfsetbuttcap%
\pgfsetroundjoin%
\definecolor{currentfill}{rgb}{0.741896,0.773683,0.818183}%
\pgfsetfillcolor{currentfill}%
\pgfsetlinewidth{0.000000pt}%
\definecolor{currentstroke}{rgb}{0.000000,0.000000,0.000000}%
\pgfsetstrokecolor{currentstroke}%
\pgfsetdash{}{0pt}%
\pgfpathmoveto{\pgfqpoint{0.927773in}{1.257857in}}%
\pgfpathlineto{\pgfqpoint{0.959296in}{1.418836in}}%
\pgfpathlineto{\pgfqpoint{0.921220in}{1.451028in}}%
\pgfpathlineto{\pgfqpoint{0.890919in}{1.277054in}}%
\pgfpathclose%
\pgfusepath{fill}%
\end{pgfscope}%
\begin{pgfscope}%
\pgfpathrectangle{\pgfqpoint{0.150000in}{0.150000in}}{\pgfqpoint{2.700000in}{1.950000in}}%
\pgfusepath{clip}%
\pgfsetbuttcap%
\pgfsetroundjoin%
\definecolor{currentfill}{rgb}{0.857537,0.741498,0.750689}%
\pgfsetfillcolor{currentfill}%
\pgfsetlinewidth{0.000000pt}%
\definecolor{currentstroke}{rgb}{0.000000,0.000000,0.000000}%
\pgfsetstrokecolor{currentstroke}%
\pgfsetdash{}{0pt}%
\pgfpathmoveto{\pgfqpoint{1.345527in}{0.635929in}}%
\pgfpathlineto{\pgfqpoint{1.383268in}{0.704685in}}%
\pgfpathlineto{\pgfqpoint{1.345088in}{0.737197in}}%
\pgfpathlineto{\pgfqpoint{1.307210in}{0.680459in}}%
\pgfpathclose%
\pgfusepath{fill}%
\end{pgfscope}%
\begin{pgfscope}%
\pgfpathrectangle{\pgfqpoint{0.150000in}{0.150000in}}{\pgfqpoint{2.700000in}{1.950000in}}%
\pgfusepath{clip}%
\pgfsetbuttcap%
\pgfsetroundjoin%
\definecolor{currentfill}{rgb}{0.735677,0.768229,0.813802}%
\pgfsetfillcolor{currentfill}%
\pgfsetlinewidth{0.000000pt}%
\definecolor{currentstroke}{rgb}{0.000000,0.000000,0.000000}%
\pgfsetstrokecolor{currentstroke}%
\pgfsetdash{}{0pt}%
\pgfpathmoveto{\pgfqpoint{1.074431in}{1.386602in}}%
\pgfpathlineto{\pgfqpoint{1.118267in}{1.289790in}}%
\pgfpathlineto{\pgfqpoint{1.080543in}{1.321681in}}%
\pgfpathlineto{\pgfqpoint{1.036255in}{1.418836in}}%
\pgfpathclose%
\pgfusepath{fill}%
\end{pgfscope}%
\begin{pgfscope}%
\pgfpathrectangle{\pgfqpoint{0.150000in}{0.150000in}}{\pgfqpoint{2.700000in}{1.950000in}}%
\pgfusepath{clip}%
\pgfsetbuttcap%
\pgfsetroundjoin%
\definecolor{currentfill}{rgb}{0.929718,0.872472,0.877007}%
\pgfsetfillcolor{currentfill}%
\pgfsetlinewidth{0.000000pt}%
\definecolor{currentstroke}{rgb}{0.000000,0.000000,0.000000}%
\pgfsetstrokecolor{currentstroke}%
\pgfsetdash{}{0pt}%
\pgfpathmoveto{\pgfqpoint{1.805063in}{0.838932in}}%
\pgfpathlineto{\pgfqpoint{1.842126in}{0.834478in}}%
\pgfpathlineto{\pgfqpoint{1.803427in}{0.854538in}}%
\pgfpathlineto{\pgfqpoint{1.766268in}{0.859070in}}%
\pgfpathclose%
\pgfusepath{fill}%
\end{pgfscope}%
\begin{pgfscope}%
\pgfpathrectangle{\pgfqpoint{0.150000in}{0.150000in}}{\pgfqpoint{2.700000in}{1.950000in}}%
\pgfusepath{clip}%
\pgfsetbuttcap%
\pgfsetroundjoin%
\definecolor{currentfill}{rgb}{0.986703,0.975873,0.976731}%
\pgfsetfillcolor{currentfill}%
\pgfsetlinewidth{0.000000pt}%
\definecolor{currentstroke}{rgb}{0.000000,0.000000,0.000000}%
\pgfsetstrokecolor{currentstroke}%
\pgfsetdash{}{0pt}%
\pgfpathmoveto{\pgfqpoint{1.230511in}{0.923836in}}%
\pgfpathlineto{\pgfqpoint{1.268289in}{0.981052in}}%
\pgfpathlineto{\pgfqpoint{1.230543in}{1.000897in}}%
\pgfpathlineto{\pgfqpoint{1.192489in}{0.956156in}}%
\pgfpathclose%
\pgfusepath{fill}%
\end{pgfscope}%
\begin{pgfscope}%
\pgfpathrectangle{\pgfqpoint{0.150000in}{0.150000in}}{\pgfqpoint{2.700000in}{1.950000in}}%
\pgfusepath{clip}%
\pgfsetbuttcap%
\pgfsetroundjoin%
\definecolor{currentfill}{rgb}{0.880331,0.782858,0.790579}%
\pgfsetfillcolor{currentfill}%
\pgfsetlinewidth{0.000000pt}%
\definecolor{currentstroke}{rgb}{0.000000,0.000000,0.000000}%
\pgfsetstrokecolor{currentstroke}%
\pgfsetdash{}{0pt}%
\pgfpathmoveto{\pgfqpoint{1.307210in}{0.680459in}}%
\pgfpathlineto{\pgfqpoint{1.345088in}{0.737197in}}%
\pgfpathlineto{\pgfqpoint{1.306682in}{0.781885in}}%
\pgfpathlineto{\pgfqpoint{1.268850in}{0.725038in}}%
\pgfpathclose%
\pgfusepath{fill}%
\end{pgfscope}%
\begin{pgfscope}%
\pgfpathrectangle{\pgfqpoint{0.150000in}{0.150000in}}{\pgfqpoint{2.700000in}{1.950000in}}%
\pgfusepath{clip}%
\pgfsetbuttcap%
\pgfsetroundjoin%
\definecolor{currentfill}{rgb}{0.967708,0.941406,0.943490}%
\pgfsetfillcolor{currentfill}%
\pgfsetlinewidth{0.000000pt}%
\definecolor{currentstroke}{rgb}{0.000000,0.000000,0.000000}%
\pgfsetstrokecolor{currentstroke}%
\pgfsetdash{}{0pt}%
\pgfpathmoveto{\pgfqpoint{1.192040in}{0.891474in}}%
\pgfpathlineto{\pgfqpoint{1.230511in}{0.923836in}}%
\pgfpathlineto{\pgfqpoint{1.192489in}{0.956156in}}%
\pgfpathlineto{\pgfqpoint{1.153557in}{0.936238in}}%
\pgfpathclose%
\pgfusepath{fill}%
\end{pgfscope}%
\begin{pgfscope}%
\pgfpathrectangle{\pgfqpoint{0.150000in}{0.150000in}}{\pgfqpoint{2.700000in}{1.950000in}}%
\pgfusepath{clip}%
\pgfsetbuttcap%
\pgfsetroundjoin%
\definecolor{currentfill}{rgb}{0.967708,0.941406,0.943490}%
\pgfsetfillcolor{currentfill}%
\pgfsetlinewidth{0.000000pt}%
\definecolor{currentstroke}{rgb}{0.000000,0.000000,0.000000}%
\pgfsetstrokecolor{currentstroke}%
\pgfsetdash{}{0pt}%
\pgfpathmoveto{\pgfqpoint{1.574851in}{0.916246in}}%
\pgfpathlineto{\pgfqpoint{1.612889in}{0.911464in}}%
\pgfpathlineto{\pgfqpoint{1.574617in}{0.931380in}}%
\pgfpathlineto{\pgfqpoint{1.536486in}{0.948670in}}%
\pgfpathclose%
\pgfusepath{fill}%
\end{pgfscope}%
\begin{pgfscope}%
\pgfpathrectangle{\pgfqpoint{0.150000in}{0.150000in}}{\pgfqpoint{2.700000in}{1.950000in}}%
\pgfusepath{clip}%
\pgfsetbuttcap%
\pgfsetroundjoin%
\definecolor{currentfill}{rgb}{0.760555,0.790043,0.831327}%
\pgfsetfillcolor{currentfill}%
\pgfsetlinewidth{0.000000pt}%
\definecolor{currentstroke}{rgb}{0.000000,0.000000,0.000000}%
\pgfsetstrokecolor{currentstroke}%
\pgfsetdash{}{0pt}%
\pgfpathmoveto{\pgfqpoint{0.965448in}{1.225883in}}%
\pgfpathlineto{\pgfqpoint{0.998073in}{1.373588in}}%
\pgfpathlineto{\pgfqpoint{0.959296in}{1.418836in}}%
\pgfpathlineto{\pgfqpoint{0.927773in}{1.257857in}}%
\pgfpathclose%
\pgfusepath{fill}%
\end{pgfscope}%
\begin{pgfscope}%
\pgfpathrectangle{\pgfqpoint{0.150000in}{0.150000in}}{\pgfqpoint{2.700000in}{1.950000in}}%
\pgfusepath{clip}%
\pgfsetbuttcap%
\pgfsetroundjoin%
\definecolor{currentfill}{rgb}{0.760555,0.790043,0.831327}%
\pgfsetfillcolor{currentfill}%
\pgfsetlinewidth{0.000000pt}%
\definecolor{currentstroke}{rgb}{0.000000,0.000000,0.000000}%
\pgfsetstrokecolor{currentstroke}%
\pgfsetdash{}{0pt}%
\pgfpathmoveto{\pgfqpoint{1.113170in}{1.341342in}}%
\pgfpathlineto{\pgfqpoint{1.156041in}{1.257857in}}%
\pgfpathlineto{\pgfqpoint{1.118267in}{1.289790in}}%
\pgfpathlineto{\pgfqpoint{1.074431in}{1.386602in}}%
\pgfpathclose%
\pgfusepath{fill}%
\end{pgfscope}%
\begin{pgfscope}%
\pgfpathrectangle{\pgfqpoint{0.150000in}{0.150000in}}{\pgfqpoint{2.700000in}{1.950000in}}%
\pgfusepath{clip}%
\pgfsetbuttcap%
\pgfsetroundjoin%
\definecolor{currentfill}{rgb}{0.899326,0.817325,0.823820}%
\pgfsetfillcolor{currentfill}%
\pgfsetlinewidth{0.000000pt}%
\definecolor{currentstroke}{rgb}{0.000000,0.000000,0.000000}%
\pgfsetstrokecolor{currentstroke}%
\pgfsetdash{}{0pt}%
\pgfpathmoveto{\pgfqpoint{1.268850in}{0.725038in}}%
\pgfpathlineto{\pgfqpoint{1.306682in}{0.781885in}}%
\pgfpathlineto{\pgfqpoint{1.268556in}{0.814343in}}%
\pgfpathlineto{\pgfqpoint{1.230448in}{0.769666in}}%
\pgfpathclose%
\pgfusepath{fill}%
\end{pgfscope}%
\begin{pgfscope}%
\pgfpathrectangle{\pgfqpoint{0.150000in}{0.150000in}}{\pgfqpoint{2.700000in}{1.950000in}}%
\pgfusepath{clip}%
\pgfsetbuttcap%
\pgfsetroundjoin%
\definecolor{currentfill}{rgb}{0.899326,0.817325,0.823820}%
\pgfsetfillcolor{currentfill}%
\pgfsetlinewidth{0.000000pt}%
\definecolor{currentstroke}{rgb}{0.000000,0.000000,0.000000}%
\pgfsetstrokecolor{currentstroke}%
\pgfsetdash{}{0pt}%
\pgfpathmoveto{\pgfqpoint{1.192383in}{0.725038in}}%
\pgfpathlineto{\pgfqpoint{1.230448in}{0.769666in}}%
\pgfpathlineto{\pgfqpoint{1.192004in}{0.814343in}}%
\pgfpathlineto{\pgfqpoint{1.153479in}{0.781885in}}%
\pgfpathclose%
\pgfusepath{fill}%
\end{pgfscope}%
\begin{pgfscope}%
\pgfpathrectangle{\pgfqpoint{0.150000in}{0.150000in}}{\pgfqpoint{2.700000in}{1.950000in}}%
\pgfusepath{clip}%
\pgfsetbuttcap%
\pgfsetroundjoin%
\definecolor{currentfill}{rgb}{0.971507,0.948300,0.950138}%
\pgfsetfillcolor{currentfill}%
\pgfsetlinewidth{0.000000pt}%
\definecolor{currentstroke}{rgb}{0.000000,0.000000,0.000000}%
\pgfsetstrokecolor{currentstroke}%
\pgfsetdash{}{0pt}%
\pgfpathmoveto{\pgfqpoint{1.038354in}{0.903845in}}%
\pgfpathlineto{\pgfqpoint{1.076972in}{0.936238in}}%
\pgfpathlineto{\pgfqpoint{1.039601in}{0.956156in}}%
\pgfpathlineto{\pgfqpoint{1.001030in}{0.923836in}}%
\pgfpathclose%
\pgfusepath{fill}%
\end{pgfscope}%
\begin{pgfscope}%
\pgfpathrectangle{\pgfqpoint{0.150000in}{0.150000in}}{\pgfqpoint{2.700000in}{1.950000in}}%
\pgfusepath{clip}%
\pgfsetbuttcap%
\pgfsetroundjoin%
\definecolor{currentfill}{rgb}{0.952512,0.913833,0.916896}%
\pgfsetfillcolor{currentfill}%
\pgfsetlinewidth{0.000000pt}%
\definecolor{currentstroke}{rgb}{0.000000,0.000000,0.000000}%
\pgfsetstrokecolor{currentstroke}%
\pgfsetdash{}{0pt}%
\pgfpathmoveto{\pgfqpoint{1.230112in}{0.859070in}}%
\pgfpathlineto{\pgfqpoint{1.268583in}{0.891474in}}%
\pgfpathlineto{\pgfqpoint{1.230511in}{0.923836in}}%
\pgfpathlineto{\pgfqpoint{1.192040in}{0.891474in}}%
\pgfpathclose%
\pgfusepath{fill}%
\end{pgfscope}%
\begin{pgfscope}%
\pgfpathrectangle{\pgfqpoint{0.150000in}{0.150000in}}{\pgfqpoint{2.700000in}{1.950000in}}%
\pgfusepath{clip}%
\pgfsetbuttcap%
\pgfsetroundjoin%
\definecolor{currentfill}{rgb}{0.918321,0.851792,0.857062}%
\pgfsetfillcolor{currentfill}%
\pgfsetlinewidth{0.000000pt}%
\definecolor{currentstroke}{rgb}{0.000000,0.000000,0.000000}%
\pgfsetstrokecolor{currentstroke}%
\pgfsetdash{}{0pt}%
\pgfpathmoveto{\pgfqpoint{1.230448in}{0.769666in}}%
\pgfpathlineto{\pgfqpoint{1.268556in}{0.814343in}}%
\pgfpathlineto{\pgfqpoint{1.230112in}{0.859070in}}%
\pgfpathlineto{\pgfqpoint{1.192004in}{0.814343in}}%
\pgfpathclose%
\pgfusepath{fill}%
\end{pgfscope}%
\begin{pgfscope}%
\pgfpathrectangle{\pgfqpoint{0.150000in}{0.150000in}}{\pgfqpoint{2.700000in}{1.950000in}}%
\pgfusepath{clip}%
\pgfsetbuttcap%
\pgfsetroundjoin%
\definecolor{currentfill}{rgb}{0.779213,0.806403,0.844470}%
\pgfsetfillcolor{currentfill}%
\pgfsetlinewidth{0.000000pt}%
\definecolor{currentstroke}{rgb}{0.000000,0.000000,0.000000}%
\pgfsetstrokecolor{currentstroke}%
\pgfsetdash{}{0pt}%
\pgfpathmoveto{\pgfqpoint{1.003171in}{1.193867in}}%
\pgfpathlineto{\pgfqpoint{1.036203in}{1.341342in}}%
\pgfpathlineto{\pgfqpoint{0.998073in}{1.373588in}}%
\pgfpathlineto{\pgfqpoint{0.965448in}{1.225883in}}%
\pgfpathclose%
\pgfusepath{fill}%
\end{pgfscope}%
\begin{pgfscope}%
\pgfpathrectangle{\pgfqpoint{0.150000in}{0.150000in}}{\pgfqpoint{2.700000in}{1.950000in}}%
\pgfusepath{clip}%
\pgfsetbuttcap%
\pgfsetroundjoin%
\definecolor{currentfill}{rgb}{0.779213,0.806403,0.844470}%
\pgfsetfillcolor{currentfill}%
\pgfsetlinewidth{0.000000pt}%
\definecolor{currentstroke}{rgb}{0.000000,0.000000,0.000000}%
\pgfsetstrokecolor{currentstroke}%
\pgfsetdash{}{0pt}%
\pgfpathmoveto{\pgfqpoint{1.151401in}{1.309054in}}%
\pgfpathlineto{\pgfqpoint{1.193863in}{1.225883in}}%
\pgfpathlineto{\pgfqpoint{1.156041in}{1.257857in}}%
\pgfpathlineto{\pgfqpoint{1.113170in}{1.341342in}}%
\pgfpathclose%
\pgfusepath{fill}%
\end{pgfscope}%
\begin{pgfscope}%
\pgfpathrectangle{\pgfqpoint{0.150000in}{0.150000in}}{\pgfqpoint{2.700000in}{1.950000in}}%
\pgfusepath{clip}%
\pgfsetbuttcap%
\pgfsetroundjoin%
\definecolor{currentfill}{rgb}{0.975306,0.955193,0.956786}%
\pgfsetfillcolor{currentfill}%
\pgfsetlinewidth{0.000000pt}%
\definecolor{currentstroke}{rgb}{0.000000,0.000000,0.000000}%
\pgfsetstrokecolor{currentstroke}%
\pgfsetdash{}{0pt}%
\pgfpathmoveto{\pgfqpoint{1.114989in}{0.903845in}}%
\pgfpathlineto{\pgfqpoint{1.153557in}{0.936238in}}%
\pgfpathlineto{\pgfqpoint{1.116045in}{0.956156in}}%
\pgfpathlineto{\pgfqpoint{1.076972in}{0.936238in}}%
\pgfpathclose%
\pgfusepath{fill}%
\end{pgfscope}%
\begin{pgfscope}%
\pgfpathrectangle{\pgfqpoint{0.150000in}{0.150000in}}{\pgfqpoint{2.700000in}{1.950000in}}%
\pgfusepath{clip}%
\pgfsetbuttcap%
\pgfsetroundjoin%
\definecolor{currentfill}{rgb}{0.849939,0.727711,0.737393}%
\pgfsetfillcolor{currentfill}%
\pgfsetlinewidth{0.000000pt}%
\definecolor{currentstroke}{rgb}{0.000000,0.000000,0.000000}%
\pgfsetstrokecolor{currentstroke}%
\pgfsetdash{}{0pt}%
\pgfpathmoveto{\pgfqpoint{1.383436in}{0.615433in}}%
\pgfpathlineto{\pgfqpoint{1.421359in}{0.684256in}}%
\pgfpathlineto{\pgfqpoint{1.383268in}{0.704685in}}%
\pgfpathlineto{\pgfqpoint{1.345527in}{0.635929in}}%
\pgfpathclose%
\pgfusepath{fill}%
\end{pgfscope}%
\begin{pgfscope}%
\pgfpathrectangle{\pgfqpoint{0.150000in}{0.150000in}}{\pgfqpoint{2.700000in}{1.950000in}}%
\pgfusepath{clip}%
\pgfsetbuttcap%
\pgfsetroundjoin%
\definecolor{currentfill}{rgb}{0.941115,0.893153,0.896952}%
\pgfsetfillcolor{currentfill}%
\pgfsetlinewidth{0.000000pt}%
\definecolor{currentstroke}{rgb}{0.000000,0.000000,0.000000}%
\pgfsetstrokecolor{currentstroke}%
\pgfsetdash{}{0pt}%
\pgfpathmoveto{\pgfqpoint{1.268556in}{0.814343in}}%
\pgfpathlineto{\pgfqpoint{1.306705in}{0.859070in}}%
\pgfpathlineto{\pgfqpoint{1.268583in}{0.891474in}}%
\pgfpathlineto{\pgfqpoint{1.230112in}{0.859070in}}%
\pgfpathclose%
\pgfusepath{fill}%
\end{pgfscope}%
\begin{pgfscope}%
\pgfpathrectangle{\pgfqpoint{0.150000in}{0.150000in}}{\pgfqpoint{2.700000in}{1.950000in}}%
\pgfusepath{clip}%
\pgfsetbuttcap%
\pgfsetroundjoin%
\definecolor{currentfill}{rgb}{0.941115,0.893153,0.896952}%
\pgfsetfillcolor{currentfill}%
\pgfsetlinewidth{0.000000pt}%
\definecolor{currentstroke}{rgb}{0.000000,0.000000,0.000000}%
\pgfsetstrokecolor{currentstroke}%
\pgfsetdash{}{0pt}%
\pgfpathmoveto{\pgfqpoint{1.192004in}{0.814343in}}%
\pgfpathlineto{\pgfqpoint{1.230112in}{0.859070in}}%
\pgfpathlineto{\pgfqpoint{1.192040in}{0.891474in}}%
\pgfpathlineto{\pgfqpoint{1.153518in}{0.859070in}}%
\pgfpathclose%
\pgfusepath{fill}%
\end{pgfscope}%
\begin{pgfscope}%
\pgfpathrectangle{\pgfqpoint{0.150000in}{0.150000in}}{\pgfqpoint{2.700000in}{1.950000in}}%
\pgfusepath{clip}%
\pgfsetbuttcap%
\pgfsetroundjoin%
\definecolor{currentfill}{rgb}{0.815748,0.665671,0.677558}%
\pgfsetfillcolor{currentfill}%
\pgfsetlinewidth{0.000000pt}%
\definecolor{currentstroke}{rgb}{0.000000,0.000000,0.000000}%
\pgfsetstrokecolor{currentstroke}%
\pgfsetdash{}{0pt}%
\pgfpathmoveto{\pgfqpoint{1.383236in}{0.550237in}}%
\pgfpathlineto{\pgfqpoint{1.421347in}{0.606895in}}%
\pgfpathlineto{\pgfqpoint{1.383436in}{0.615433in}}%
\pgfpathlineto{\pgfqpoint{1.345508in}{0.558933in}}%
\pgfpathclose%
\pgfusepath{fill}%
\end{pgfscope}%
\begin{pgfscope}%
\pgfpathrectangle{\pgfqpoint{0.150000in}{0.150000in}}{\pgfqpoint{2.700000in}{1.950000in}}%
\pgfusepath{clip}%
\pgfsetbuttcap%
\pgfsetroundjoin%
\definecolor{currentfill}{rgb}{0.804090,0.828217,0.861994}%
\pgfsetfillcolor{currentfill}%
\pgfsetlinewidth{0.000000pt}%
\definecolor{currentstroke}{rgb}{0.000000,0.000000,0.000000}%
\pgfsetstrokecolor{currentstroke}%
\pgfsetdash{}{0pt}%
\pgfpathmoveto{\pgfqpoint{1.040943in}{1.161810in}}%
\pgfpathlineto{\pgfqpoint{1.074941in}{1.296132in}}%
\pgfpathlineto{\pgfqpoint{1.036203in}{1.341342in}}%
\pgfpathlineto{\pgfqpoint{1.003171in}{1.193867in}}%
\pgfpathclose%
\pgfusepath{fill}%
\end{pgfscope}%
\begin{pgfscope}%
\pgfpathrectangle{\pgfqpoint{0.150000in}{0.150000in}}{\pgfqpoint{2.700000in}{1.950000in}}%
\pgfusepath{clip}%
\pgfsetbuttcap%
\pgfsetroundjoin%
\definecolor{currentfill}{rgb}{0.804090,0.828217,0.861994}%
\pgfsetfillcolor{currentfill}%
\pgfsetlinewidth{0.000000pt}%
\definecolor{currentstroke}{rgb}{0.000000,0.000000,0.000000}%
\pgfsetstrokecolor{currentstroke}%
\pgfsetdash{}{0pt}%
\pgfpathmoveto{\pgfqpoint{1.190101in}{1.263831in}}%
\pgfpathlineto{\pgfqpoint{1.231735in}{1.193867in}}%
\pgfpathlineto{\pgfqpoint{1.193863in}{1.225883in}}%
\pgfpathlineto{\pgfqpoint{1.151401in}{1.309054in}}%
\pgfpathclose%
\pgfusepath{fill}%
\end{pgfscope}%
\begin{pgfscope}%
\pgfpathrectangle{\pgfqpoint{0.150000in}{0.150000in}}{\pgfqpoint{2.700000in}{1.950000in}}%
\pgfusepath{clip}%
\pgfsetbuttcap%
\pgfsetroundjoin%
\definecolor{currentfill}{rgb}{0.975306,0.955193,0.956786}%
\pgfsetfillcolor{currentfill}%
\pgfsetlinewidth{0.000000pt}%
\definecolor{currentstroke}{rgb}{0.000000,0.000000,0.000000}%
\pgfsetstrokecolor{currentstroke}%
\pgfsetdash{}{0pt}%
\pgfpathmoveto{\pgfqpoint{1.268583in}{0.891474in}}%
\pgfpathlineto{\pgfqpoint{1.306452in}{0.948670in}}%
\pgfpathlineto{\pgfqpoint{1.268289in}{0.981052in}}%
\pgfpathlineto{\pgfqpoint{1.230511in}{0.923836in}}%
\pgfpathclose%
\pgfusepath{fill}%
\end{pgfscope}%
\begin{pgfscope}%
\pgfpathrectangle{\pgfqpoint{0.150000in}{0.150000in}}{\pgfqpoint{2.700000in}{1.950000in}}%
\pgfusepath{clip}%
\pgfsetbuttcap%
\pgfsetroundjoin%
\definecolor{currentfill}{rgb}{0.960110,0.927619,0.930193}%
\pgfsetfillcolor{currentfill}%
\pgfsetlinewidth{0.000000pt}%
\definecolor{currentstroke}{rgb}{0.000000,0.000000,0.000000}%
\pgfsetstrokecolor{currentstroke}%
\pgfsetdash{}{0pt}%
\pgfpathmoveto{\pgfqpoint{1.153518in}{0.859070in}}%
\pgfpathlineto{\pgfqpoint{1.192040in}{0.891474in}}%
\pgfpathlineto{\pgfqpoint{1.153557in}{0.936238in}}%
\pgfpathlineto{\pgfqpoint{1.114989in}{0.903845in}}%
\pgfpathclose%
\pgfusepath{fill}%
\end{pgfscope}%
\begin{pgfscope}%
\pgfpathrectangle{\pgfqpoint{0.150000in}{0.150000in}}{\pgfqpoint{2.700000in}{1.950000in}}%
\pgfusepath{clip}%
\pgfsetbuttcap%
\pgfsetroundjoin%
\definecolor{currentfill}{rgb}{0.925919,0.865579,0.870358}%
\pgfsetfillcolor{currentfill}%
\pgfsetlinewidth{0.000000pt}%
\definecolor{currentstroke}{rgb}{0.000000,0.000000,0.000000}%
\pgfsetstrokecolor{currentstroke}%
\pgfsetdash{}{0pt}%
\pgfpathmoveto{\pgfqpoint{1.306682in}{0.781885in}}%
\pgfpathlineto{\pgfqpoint{1.344877in}{0.826623in}}%
\pgfpathlineto{\pgfqpoint{1.306705in}{0.859070in}}%
\pgfpathlineto{\pgfqpoint{1.268556in}{0.814343in}}%
\pgfpathclose%
\pgfusepath{fill}%
\end{pgfscope}%
\begin{pgfscope}%
\pgfpathrectangle{\pgfqpoint{0.150000in}{0.150000in}}{\pgfqpoint{2.700000in}{1.950000in}}%
\pgfusepath{clip}%
\pgfsetbuttcap%
\pgfsetroundjoin%
\definecolor{currentfill}{rgb}{0.925919,0.865579,0.870358}%
\pgfsetfillcolor{currentfill}%
\pgfsetlinewidth{0.000000pt}%
\definecolor{currentstroke}{rgb}{0.000000,0.000000,0.000000}%
\pgfsetstrokecolor{currentstroke}%
\pgfsetdash{}{0pt}%
\pgfpathmoveto{\pgfqpoint{1.153479in}{0.781885in}}%
\pgfpathlineto{\pgfqpoint{1.192004in}{0.814343in}}%
\pgfpathlineto{\pgfqpoint{1.153518in}{0.859070in}}%
\pgfpathlineto{\pgfqpoint{1.114946in}{0.826623in}}%
\pgfpathclose%
\pgfusepath{fill}%
\end{pgfscope}%
\begin{pgfscope}%
\pgfpathrectangle{\pgfqpoint{0.150000in}{0.150000in}}{\pgfqpoint{2.700000in}{1.950000in}}%
\pgfusepath{clip}%
\pgfsetbuttcap%
\pgfsetroundjoin%
\definecolor{currentfill}{rgb}{0.925919,0.865579,0.870358}%
\pgfsetfillcolor{currentfill}%
\pgfsetlinewidth{0.000000pt}%
\definecolor{currentstroke}{rgb}{0.000000,0.000000,0.000000}%
\pgfsetstrokecolor{currentstroke}%
\pgfsetdash{}{0pt}%
\pgfpathmoveto{\pgfqpoint{1.920418in}{0.806412in}}%
\pgfpathlineto{\pgfqpoint{1.957520in}{0.814343in}}%
\pgfpathlineto{\pgfqpoint{1.918536in}{0.834478in}}%
\pgfpathlineto{\pgfqpoint{1.881383in}{0.826623in}}%
\pgfpathclose%
\pgfusepath{fill}%
\end{pgfscope}%
\begin{pgfscope}%
\pgfpathrectangle{\pgfqpoint{0.150000in}{0.150000in}}{\pgfqpoint{2.700000in}{1.950000in}}%
\pgfusepath{clip}%
\pgfsetbuttcap%
\pgfsetroundjoin%
\definecolor{currentfill}{rgb}{0.828968,0.850031,0.879519}%
\pgfsetfillcolor{currentfill}%
\pgfsetlinewidth{0.000000pt}%
\definecolor{currentstroke}{rgb}{0.000000,0.000000,0.000000}%
\pgfsetstrokecolor{currentstroke}%
\pgfsetdash{}{0pt}%
\pgfpathmoveto{\pgfqpoint{1.078764in}{1.129711in}}%
\pgfpathlineto{\pgfqpoint{1.113126in}{1.263831in}}%
\pgfpathlineto{\pgfqpoint{1.074941in}{1.296132in}}%
\pgfpathlineto{\pgfqpoint{1.040943in}{1.161810in}}%
\pgfpathclose%
\pgfusepath{fill}%
\end{pgfscope}%
\begin{pgfscope}%
\pgfpathrectangle{\pgfqpoint{0.150000in}{0.150000in}}{\pgfqpoint{2.700000in}{1.950000in}}%
\pgfusepath{clip}%
\pgfsetbuttcap%
\pgfsetroundjoin%
\definecolor{currentfill}{rgb}{0.910723,0.838006,0.843765}%
\pgfsetfillcolor{currentfill}%
\pgfsetlinewidth{0.000000pt}%
\definecolor{currentstroke}{rgb}{0.000000,0.000000,0.000000}%
\pgfsetstrokecolor{currentstroke}%
\pgfsetdash{}{0pt}%
\pgfpathmoveto{\pgfqpoint{1.345088in}{0.737197in}}%
\pgfpathlineto{\pgfqpoint{1.383099in}{0.794134in}}%
\pgfpathlineto{\pgfqpoint{1.344877in}{0.826623in}}%
\pgfpathlineto{\pgfqpoint{1.306682in}{0.781885in}}%
\pgfpathclose%
\pgfusepath{fill}%
\end{pgfscope}%
\begin{pgfscope}%
\pgfpathrectangle{\pgfqpoint{0.150000in}{0.150000in}}{\pgfqpoint{2.700000in}{1.950000in}}%
\pgfusepath{clip}%
\pgfsetbuttcap%
\pgfsetroundjoin%
\definecolor{currentfill}{rgb}{0.822748,0.844577,0.875138}%
\pgfsetfillcolor{currentfill}%
\pgfsetlinewidth{0.000000pt}%
\definecolor{currentstroke}{rgb}{0.000000,0.000000,0.000000}%
\pgfsetstrokecolor{currentstroke}%
\pgfsetdash{}{0pt}%
\pgfpathmoveto{\pgfqpoint{1.228386in}{1.231489in}}%
\pgfpathlineto{\pgfqpoint{1.269336in}{1.174456in}}%
\pgfpathlineto{\pgfqpoint{1.231735in}{1.193867in}}%
\pgfpathlineto{\pgfqpoint{1.190101in}{1.263831in}}%
\pgfpathclose%
\pgfusepath{fill}%
\end{pgfscope}%
\begin{pgfscope}%
\pgfpathrectangle{\pgfqpoint{0.150000in}{0.150000in}}{\pgfqpoint{2.700000in}{1.950000in}}%
\pgfusepath{clip}%
\pgfsetbuttcap%
\pgfsetroundjoin%
\definecolor{currentfill}{rgb}{0.963909,0.934513,0.936841}%
\pgfsetfillcolor{currentfill}%
\pgfsetlinewidth{0.000000pt}%
\definecolor{currentstroke}{rgb}{0.000000,0.000000,0.000000}%
\pgfsetstrokecolor{currentstroke}%
\pgfsetdash{}{0pt}%
\pgfpathmoveto{\pgfqpoint{1.306705in}{0.859070in}}%
\pgfpathlineto{\pgfqpoint{1.344666in}{0.916246in}}%
\pgfpathlineto{\pgfqpoint{1.306452in}{0.948670in}}%
\pgfpathlineto{\pgfqpoint{1.268583in}{0.891474in}}%
\pgfpathclose%
\pgfusepath{fill}%
\end{pgfscope}%
\begin{pgfscope}%
\pgfpathrectangle{\pgfqpoint{0.150000in}{0.150000in}}{\pgfqpoint{2.700000in}{1.950000in}}%
\pgfusepath{clip}%
\pgfsetbuttcap%
\pgfsetroundjoin%
\definecolor{currentfill}{rgb}{0.963909,0.934513,0.936841}%
\pgfsetfillcolor{currentfill}%
\pgfsetlinewidth{0.000000pt}%
\definecolor{currentstroke}{rgb}{0.000000,0.000000,0.000000}%
\pgfsetstrokecolor{currentstroke}%
\pgfsetdash{}{0pt}%
\pgfpathmoveto{\pgfqpoint{1.613358in}{0.896180in}}%
\pgfpathlineto{\pgfqpoint{1.651302in}{0.891474in}}%
\pgfpathlineto{\pgfqpoint{1.612889in}{0.911464in}}%
\pgfpathlineto{\pgfqpoint{1.574851in}{0.916246in}}%
\pgfpathclose%
\pgfusepath{fill}%
\end{pgfscope}%
\begin{pgfscope}%
\pgfpathrectangle{\pgfqpoint{0.150000in}{0.150000in}}{\pgfqpoint{2.700000in}{1.950000in}}%
\pgfusepath{clip}%
\pgfsetbuttcap%
\pgfsetroundjoin%
\definecolor{currentfill}{rgb}{0.895527,0.810432,0.817172}%
\pgfsetfillcolor{currentfill}%
\pgfsetlinewidth{0.000000pt}%
\definecolor{currentstroke}{rgb}{0.000000,0.000000,0.000000}%
\pgfsetstrokecolor{currentstroke}%
\pgfsetdash{}{0pt}%
\pgfpathmoveto{\pgfqpoint{1.383268in}{0.704685in}}%
\pgfpathlineto{\pgfqpoint{1.421371in}{0.761602in}}%
\pgfpathlineto{\pgfqpoint{1.383099in}{0.794134in}}%
\pgfpathlineto{\pgfqpoint{1.345088in}{0.737197in}}%
\pgfpathclose%
\pgfusepath{fill}%
\end{pgfscope}%
\begin{pgfscope}%
\pgfpathrectangle{\pgfqpoint{0.150000in}{0.150000in}}{\pgfqpoint{2.700000in}{1.950000in}}%
\pgfusepath{clip}%
\pgfsetbuttcap%
\pgfsetroundjoin%
\definecolor{currentfill}{rgb}{0.847626,0.866391,0.892662}%
\pgfsetfillcolor{currentfill}%
\pgfsetlinewidth{0.000000pt}%
\definecolor{currentstroke}{rgb}{0.000000,0.000000,0.000000}%
\pgfsetstrokecolor{currentstroke}%
\pgfsetdash{}{0pt}%
\pgfpathmoveto{\pgfqpoint{1.116635in}{1.097571in}}%
\pgfpathlineto{\pgfqpoint{1.151361in}{1.231489in}}%
\pgfpathlineto{\pgfqpoint{1.113126in}{1.263831in}}%
\pgfpathlineto{\pgfqpoint{1.078764in}{1.129711in}}%
\pgfpathclose%
\pgfusepath{fill}%
\end{pgfscope}%
\begin{pgfscope}%
\pgfpathrectangle{\pgfqpoint{0.150000in}{0.150000in}}{\pgfqpoint{2.700000in}{1.950000in}}%
\pgfusepath{clip}%
\pgfsetbuttcap%
\pgfsetroundjoin%
\definecolor{currentfill}{rgb}{0.967708,0.941406,0.943490}%
\pgfsetfillcolor{currentfill}%
\pgfsetlinewidth{0.000000pt}%
\definecolor{currentstroke}{rgb}{0.000000,0.000000,0.000000}%
\pgfsetstrokecolor{currentstroke}%
\pgfsetdash{}{0pt}%
\pgfpathmoveto{\pgfqpoint{1.075816in}{0.883780in}}%
\pgfpathlineto{\pgfqpoint{1.114989in}{0.903845in}}%
\pgfpathlineto{\pgfqpoint{1.076972in}{0.936238in}}%
\pgfpathlineto{\pgfqpoint{1.038354in}{0.903845in}}%
\pgfpathclose%
\pgfusepath{fill}%
\end{pgfscope}%
\begin{pgfscope}%
\pgfpathrectangle{\pgfqpoint{0.150000in}{0.150000in}}{\pgfqpoint{2.700000in}{1.950000in}}%
\pgfusepath{clip}%
\pgfsetbuttcap%
\pgfsetroundjoin%
\definecolor{currentfill}{rgb}{0.952512,0.913833,0.916896}%
\pgfsetfillcolor{currentfill}%
\pgfsetlinewidth{0.000000pt}%
\definecolor{currentstroke}{rgb}{0.000000,0.000000,0.000000}%
\pgfsetstrokecolor{currentstroke}%
\pgfsetdash{}{0pt}%
\pgfpathmoveto{\pgfqpoint{1.114946in}{0.826623in}}%
\pgfpathlineto{\pgfqpoint{1.153518in}{0.859070in}}%
\pgfpathlineto{\pgfqpoint{1.114989in}{0.903845in}}%
\pgfpathlineto{\pgfqpoint{1.075816in}{0.883780in}}%
\pgfpathclose%
\pgfusepath{fill}%
\end{pgfscope}%
\begin{pgfscope}%
\pgfpathrectangle{\pgfqpoint{0.150000in}{0.150000in}}{\pgfqpoint{2.700000in}{1.950000in}}%
\pgfusepath{clip}%
\pgfsetbuttcap%
\pgfsetroundjoin%
\definecolor{currentfill}{rgb}{0.952512,0.913833,0.916896}%
\pgfsetfillcolor{currentfill}%
\pgfsetlinewidth{0.000000pt}%
\definecolor{currentstroke}{rgb}{0.000000,0.000000,0.000000}%
\pgfsetstrokecolor{currentstroke}%
\pgfsetdash{}{0pt}%
\pgfpathmoveto{\pgfqpoint{1.728790in}{0.863640in}}%
\pgfpathlineto{\pgfqpoint{1.766268in}{0.859070in}}%
\pgfpathlineto{\pgfqpoint{1.727387in}{0.866821in}}%
\pgfpathlineto{\pgfqpoint{1.690043in}{0.883780in}}%
\pgfpathclose%
\pgfusepath{fill}%
\end{pgfscope}%
\begin{pgfscope}%
\pgfpathrectangle{\pgfqpoint{0.150000in}{0.150000in}}{\pgfqpoint{2.700000in}{1.950000in}}%
\pgfusepath{clip}%
\pgfsetbuttcap%
\pgfsetroundjoin%
\definecolor{currentfill}{rgb}{0.933517,0.879366,0.883655}%
\pgfsetfillcolor{currentfill}%
\pgfsetlinewidth{0.000000pt}%
\definecolor{currentstroke}{rgb}{0.000000,0.000000,0.000000}%
\pgfsetstrokecolor{currentstroke}%
\pgfsetdash{}{0pt}%
\pgfpathmoveto{\pgfqpoint{1.844002in}{0.818720in}}%
\pgfpathlineto{\pgfqpoint{1.881383in}{0.826623in}}%
\pgfpathlineto{\pgfqpoint{1.842126in}{0.834478in}}%
\pgfpathlineto{\pgfqpoint{1.805063in}{0.838932in}}%
\pgfpathclose%
\pgfusepath{fill}%
\end{pgfscope}%
\begin{pgfscope}%
\pgfpathrectangle{\pgfqpoint{0.150000in}{0.150000in}}{\pgfqpoint{2.700000in}{1.950000in}}%
\pgfusepath{clip}%
\pgfsetbuttcap%
\pgfsetroundjoin%
\definecolor{currentfill}{rgb}{0.841406,0.860938,0.888281}%
\pgfsetfillcolor{currentfill}%
\pgfsetlinewidth{0.000000pt}%
\definecolor{currentstroke}{rgb}{0.000000,0.000000,0.000000}%
\pgfsetstrokecolor{currentstroke}%
\pgfsetdash{}{0pt}%
\pgfpathmoveto{\pgfqpoint{1.267047in}{1.186305in}}%
\pgfpathlineto{\pgfqpoint{1.307351in}{1.142327in}}%
\pgfpathlineto{\pgfqpoint{1.269336in}{1.174456in}}%
\pgfpathlineto{\pgfqpoint{1.228386in}{1.231489in}}%
\pgfpathclose%
\pgfusepath{fill}%
\end{pgfscope}%
\begin{pgfscope}%
\pgfpathrectangle{\pgfqpoint{0.150000in}{0.150000in}}{\pgfqpoint{2.700000in}{1.950000in}}%
\pgfusepath{clip}%
\pgfsetbuttcap%
\pgfsetroundjoin%
\definecolor{currentfill}{rgb}{0.872503,0.888205,0.910187}%
\pgfsetfillcolor{currentfill}%
\pgfsetlinewidth{0.000000pt}%
\definecolor{currentstroke}{rgb}{0.000000,0.000000,0.000000}%
\pgfsetstrokecolor{currentstroke}%
\pgfsetdash{}{0pt}%
\pgfpathmoveto{\pgfqpoint{1.154555in}{1.065388in}}%
\pgfpathlineto{\pgfqpoint{1.190065in}{1.186305in}}%
\pgfpathlineto{\pgfqpoint{1.151361in}{1.231489in}}%
\pgfpathlineto{\pgfqpoint{1.116635in}{1.097571in}}%
\pgfpathclose%
\pgfusepath{fill}%
\end{pgfscope}%
\begin{pgfscope}%
\pgfpathrectangle{\pgfqpoint{0.150000in}{0.150000in}}{\pgfqpoint{2.700000in}{1.950000in}}%
\pgfusepath{clip}%
\pgfsetbuttcap%
\pgfsetroundjoin%
\definecolor{currentfill}{rgb}{0.952512,0.913833,0.916896}%
\pgfsetfillcolor{currentfill}%
\pgfsetlinewidth{0.000000pt}%
\definecolor{currentstroke}{rgb}{0.000000,0.000000,0.000000}%
\pgfsetstrokecolor{currentstroke}%
\pgfsetdash{}{0pt}%
\pgfpathmoveto{\pgfqpoint{1.344877in}{0.826623in}}%
\pgfpathlineto{\pgfqpoint{1.382930in}{0.883780in}}%
\pgfpathlineto{\pgfqpoint{1.344666in}{0.916246in}}%
\pgfpathlineto{\pgfqpoint{1.306705in}{0.859070in}}%
\pgfpathclose%
\pgfusepath{fill}%
\end{pgfscope}%
\begin{pgfscope}%
\pgfpathrectangle{\pgfqpoint{0.150000in}{0.150000in}}{\pgfqpoint{2.700000in}{1.950000in}}%
\pgfusepath{clip}%
\pgfsetbuttcap%
\pgfsetroundjoin%
\definecolor{currentfill}{rgb}{0.866284,0.882751,0.905806}%
\pgfsetfillcolor{currentfill}%
\pgfsetlinewidth{0.000000pt}%
\definecolor{currentstroke}{rgb}{0.000000,0.000000,0.000000}%
\pgfsetstrokecolor{currentstroke}%
\pgfsetdash{}{0pt}%
\pgfpathmoveto{\pgfqpoint{1.305387in}{1.153908in}}%
\pgfpathlineto{\pgfqpoint{1.345416in}{1.110156in}}%
\pgfpathlineto{\pgfqpoint{1.307351in}{1.142327in}}%
\pgfpathlineto{\pgfqpoint{1.267047in}{1.186305in}}%
\pgfpathclose%
\pgfusepath{fill}%
\end{pgfscope}%
\begin{pgfscope}%
\pgfpathrectangle{\pgfqpoint{0.150000in}{0.150000in}}{\pgfqpoint{2.700000in}{1.950000in}}%
\pgfusepath{clip}%
\pgfsetbuttcap%
\pgfsetroundjoin%
\definecolor{currentfill}{rgb}{0.897381,0.910018,0.927711}%
\pgfsetfillcolor{currentfill}%
\pgfsetlinewidth{0.000000pt}%
\definecolor{currentstroke}{rgb}{0.000000,0.000000,0.000000}%
\pgfsetstrokecolor{currentstroke}%
\pgfsetdash{}{0pt}%
\pgfpathmoveto{\pgfqpoint{1.192524in}{1.033164in}}%
\pgfpathlineto{\pgfqpoint{1.228354in}{1.153908in}}%
\pgfpathlineto{\pgfqpoint{1.190065in}{1.186305in}}%
\pgfpathlineto{\pgfqpoint{1.154555in}{1.065388in}}%
\pgfpathclose%
\pgfusepath{fill}%
\end{pgfscope}%
\begin{pgfscope}%
\pgfpathrectangle{\pgfqpoint{0.150000in}{0.150000in}}{\pgfqpoint{2.700000in}{1.950000in}}%
\pgfusepath{clip}%
\pgfsetbuttcap%
\pgfsetroundjoin%
\definecolor{currentfill}{rgb}{0.891161,0.904565,0.923330}%
\pgfsetfillcolor{currentfill}%
\pgfsetlinewidth{0.000000pt}%
\definecolor{currentstroke}{rgb}{0.000000,0.000000,0.000000}%
\pgfsetstrokecolor{currentstroke}%
\pgfsetdash{}{0pt}%
\pgfpathmoveto{\pgfqpoint{1.344010in}{1.108763in}}%
\pgfpathlineto{\pgfqpoint{1.383530in}{1.077943in}}%
\pgfpathlineto{\pgfqpoint{1.345416in}{1.110156in}}%
\pgfpathlineto{\pgfqpoint{1.305387in}{1.153908in}}%
\pgfpathclose%
\pgfusepath{fill}%
\end{pgfscope}%
\begin{pgfscope}%
\pgfpathrectangle{\pgfqpoint{0.150000in}{0.150000in}}{\pgfqpoint{2.700000in}{1.950000in}}%
\pgfusepath{clip}%
\pgfsetbuttcap%
\pgfsetroundjoin%
\definecolor{currentfill}{rgb}{0.941115,0.893153,0.896952}%
\pgfsetfillcolor{currentfill}%
\pgfsetlinewidth{0.000000pt}%
\definecolor{currentstroke}{rgb}{0.000000,0.000000,0.000000}%
\pgfsetstrokecolor{currentstroke}%
\pgfsetdash{}{0pt}%
\pgfpathmoveto{\pgfqpoint{1.383099in}{0.794134in}}%
\pgfpathlineto{\pgfqpoint{1.421244in}{0.851271in}}%
\pgfpathlineto{\pgfqpoint{1.382930in}{0.883780in}}%
\pgfpathlineto{\pgfqpoint{1.344877in}{0.826623in}}%
\pgfpathclose%
\pgfusepath{fill}%
\end{pgfscope}%
\begin{pgfscope}%
\pgfpathrectangle{\pgfqpoint{0.150000in}{0.150000in}}{\pgfqpoint{2.700000in}{1.950000in}}%
\pgfusepath{clip}%
\pgfsetbuttcap%
\pgfsetroundjoin%
\definecolor{currentfill}{rgb}{0.922258,0.931832,0.945236}%
\pgfsetfillcolor{currentfill}%
\pgfsetlinewidth{0.000000pt}%
\definecolor{currentstroke}{rgb}{0.000000,0.000000,0.000000}%
\pgfsetstrokecolor{currentstroke}%
\pgfsetdash{}{0pt}%
\pgfpathmoveto{\pgfqpoint{1.230543in}{1.000897in}}%
\pgfpathlineto{\pgfqpoint{1.267020in}{1.108763in}}%
\pgfpathlineto{\pgfqpoint{1.228354in}{1.153908in}}%
\pgfpathlineto{\pgfqpoint{1.192524in}{1.033164in}}%
\pgfpathclose%
\pgfusepath{fill}%
\end{pgfscope}%
\begin{pgfscope}%
\pgfpathrectangle{\pgfqpoint{0.150000in}{0.150000in}}{\pgfqpoint{2.700000in}{1.950000in}}%
\pgfusepath{clip}%
\pgfsetbuttcap%
\pgfsetroundjoin%
\definecolor{currentfill}{rgb}{0.891728,0.803539,0.810524}%
\pgfsetfillcolor{currentfill}%
\pgfsetlinewidth{0.000000pt}%
\definecolor{currentstroke}{rgb}{0.000000,0.000000,0.000000}%
\pgfsetstrokecolor{currentstroke}%
\pgfsetdash{}{0pt}%
\pgfpathmoveto{\pgfqpoint{1.421359in}{0.684256in}}%
\pgfpathlineto{\pgfqpoint{1.459600in}{0.741244in}}%
\pgfpathlineto{\pgfqpoint{1.421371in}{0.761602in}}%
\pgfpathlineto{\pgfqpoint{1.383268in}{0.704685in}}%
\pgfpathclose%
\pgfusepath{fill}%
\end{pgfscope}%
\begin{pgfscope}%
\pgfpathrectangle{\pgfqpoint{0.150000in}{0.150000in}}{\pgfqpoint{2.700000in}{1.950000in}}%
\pgfusepath{clip}%
\pgfsetbuttcap%
\pgfsetroundjoin%
\definecolor{currentfill}{rgb}{0.853738,0.734605,0.744041}%
\pgfsetfillcolor{currentfill}%
\pgfsetlinewidth{0.000000pt}%
\definecolor{currentstroke}{rgb}{0.000000,0.000000,0.000000}%
\pgfsetstrokecolor{currentstroke}%
\pgfsetdash{}{0pt}%
\pgfpathmoveto{\pgfqpoint{1.421347in}{0.606895in}}%
\pgfpathlineto{\pgfqpoint{1.459499in}{0.675904in}}%
\pgfpathlineto{\pgfqpoint{1.421359in}{0.684256in}}%
\pgfpathlineto{\pgfqpoint{1.383436in}{0.615433in}}%
\pgfpathclose%
\pgfusepath{fill}%
\end{pgfscope}%
\begin{pgfscope}%
\pgfpathrectangle{\pgfqpoint{0.150000in}{0.150000in}}{\pgfqpoint{2.700000in}{1.950000in}}%
\pgfusepath{clip}%
\pgfsetbuttcap%
\pgfsetroundjoin%
\definecolor{currentfill}{rgb}{0.916039,0.926379,0.940855}%
\pgfsetfillcolor{currentfill}%
\pgfsetlinewidth{0.000000pt}%
\definecolor{currentstroke}{rgb}{0.000000,0.000000,0.000000}%
\pgfsetstrokecolor{currentstroke}%
\pgfsetdash{}{0pt}%
\pgfpathmoveto{\pgfqpoint{1.382404in}{1.076312in}}%
\pgfpathlineto{\pgfqpoint{1.421695in}{1.045688in}}%
\pgfpathlineto{\pgfqpoint{1.383530in}{1.077943in}}%
\pgfpathlineto{\pgfqpoint{1.344010in}{1.108763in}}%
\pgfpathclose%
\pgfusepath{fill}%
\end{pgfscope}%
\begin{pgfscope}%
\pgfpathrectangle{\pgfqpoint{0.150000in}{0.150000in}}{\pgfqpoint{2.700000in}{1.950000in}}%
\pgfusepath{clip}%
\pgfsetbuttcap%
\pgfsetroundjoin%
\definecolor{currentfill}{rgb}{0.929718,0.872472,0.877007}%
\pgfsetfillcolor{currentfill}%
\pgfsetlinewidth{0.000000pt}%
\definecolor{currentstroke}{rgb}{0.000000,0.000000,0.000000}%
\pgfsetstrokecolor{currentstroke}%
\pgfsetdash{}{0pt}%
\pgfpathmoveto{\pgfqpoint{1.421371in}{0.761602in}}%
\pgfpathlineto{\pgfqpoint{1.459608in}{0.818720in}}%
\pgfpathlineto{\pgfqpoint{1.421244in}{0.851271in}}%
\pgfpathlineto{\pgfqpoint{1.383099in}{0.794134in}}%
\pgfpathclose%
\pgfusepath{fill}%
\end{pgfscope}%
\begin{pgfscope}%
\pgfpathrectangle{\pgfqpoint{0.150000in}{0.150000in}}{\pgfqpoint{2.700000in}{1.950000in}}%
\pgfusepath{clip}%
\pgfsetbuttcap%
\pgfsetroundjoin%
\definecolor{currentfill}{rgb}{0.940916,0.948192,0.958379}%
\pgfsetfillcolor{currentfill}%
\pgfsetlinewidth{0.000000pt}%
\definecolor{currentstroke}{rgb}{0.000000,0.000000,0.000000}%
\pgfsetstrokecolor{currentstroke}%
\pgfsetdash{}{0pt}%
\pgfpathmoveto{\pgfqpoint{1.420989in}{1.031204in}}%
\pgfpathlineto{\pgfqpoint{1.459909in}{1.013391in}}%
\pgfpathlineto{\pgfqpoint{1.421695in}{1.045688in}}%
\pgfpathlineto{\pgfqpoint{1.382404in}{1.076312in}}%
\pgfpathclose%
\pgfusepath{fill}%
\end{pgfscope}%
\begin{pgfscope}%
\pgfpathrectangle{\pgfqpoint{0.150000in}{0.150000in}}{\pgfqpoint{2.700000in}{1.950000in}}%
\pgfusepath{clip}%
\pgfsetbuttcap%
\pgfsetroundjoin%
\definecolor{currentfill}{rgb}{0.940916,0.948192,0.958379}%
\pgfsetfillcolor{currentfill}%
\pgfsetlinewidth{0.000000pt}%
\definecolor{currentstroke}{rgb}{0.000000,0.000000,0.000000}%
\pgfsetstrokecolor{currentstroke}%
\pgfsetdash{}{0pt}%
\pgfpathmoveto{\pgfqpoint{1.268289in}{0.981052in}}%
\pgfpathlineto{\pgfqpoint{1.305363in}{1.076312in}}%
\pgfpathlineto{\pgfqpoint{1.267020in}{1.108763in}}%
\pgfpathlineto{\pgfqpoint{1.230543in}{1.000897in}}%
\pgfpathclose%
\pgfusepath{fill}%
\end{pgfscope}%
\begin{pgfscope}%
\pgfpathrectangle{\pgfqpoint{0.150000in}{0.150000in}}{\pgfqpoint{2.700000in}{1.950000in}}%
\pgfusepath{clip}%
\pgfsetbuttcap%
\pgfsetroundjoin%
\definecolor{currentfill}{rgb}{0.959574,0.964553,0.971523}%
\pgfsetfillcolor{currentfill}%
\pgfsetlinewidth{0.000000pt}%
\definecolor{currentstroke}{rgb}{0.000000,0.000000,0.000000}%
\pgfsetstrokecolor{currentstroke}%
\pgfsetdash{}{0pt}%
\pgfpathmoveto{\pgfqpoint{1.459437in}{0.998699in}}%
\pgfpathlineto{\pgfqpoint{1.498173in}{0.981052in}}%
\pgfpathlineto{\pgfqpoint{1.459909in}{1.013391in}}%
\pgfpathlineto{\pgfqpoint{1.420989in}{1.031204in}}%
\pgfpathclose%
\pgfusepath{fill}%
\end{pgfscope}%
\begin{pgfscope}%
\pgfpathrectangle{\pgfqpoint{0.150000in}{0.150000in}}{\pgfqpoint{2.700000in}{1.950000in}}%
\pgfusepath{clip}%
\pgfsetbuttcap%
\pgfsetroundjoin%
\definecolor{currentfill}{rgb}{0.971507,0.948300,0.950138}%
\pgfsetfillcolor{currentfill}%
\pgfsetlinewidth{0.000000pt}%
\definecolor{currentstroke}{rgb}{0.000000,0.000000,0.000000}%
\pgfsetstrokecolor{currentstroke}%
\pgfsetdash{}{0pt}%
\pgfpathmoveto{\pgfqpoint{1.652148in}{0.888467in}}%
\pgfpathlineto{\pgfqpoint{1.690043in}{0.883780in}}%
\pgfpathlineto{\pgfqpoint{1.651302in}{0.891474in}}%
\pgfpathlineto{\pgfqpoint{1.613358in}{0.896180in}}%
\pgfpathclose%
\pgfusepath{fill}%
\end{pgfscope}%
\begin{pgfscope}%
\pgfpathrectangle{\pgfqpoint{0.150000in}{0.150000in}}{\pgfqpoint{2.700000in}{1.950000in}}%
\pgfusepath{clip}%
\pgfsetbuttcap%
\pgfsetroundjoin%
\definecolor{currentfill}{rgb}{0.937316,0.886259,0.890303}%
\pgfsetfillcolor{currentfill}%
\pgfsetlinewidth{0.000000pt}%
\definecolor{currentstroke}{rgb}{0.000000,0.000000,0.000000}%
\pgfsetstrokecolor{currentstroke}%
\pgfsetdash{}{0pt}%
\pgfpathmoveto{\pgfqpoint{1.883087in}{0.798432in}}%
\pgfpathlineto{\pgfqpoint{1.920418in}{0.806412in}}%
\pgfpathlineto{\pgfqpoint{1.881383in}{0.826623in}}%
\pgfpathlineto{\pgfqpoint{1.844002in}{0.818720in}}%
\pgfpathclose%
\pgfusepath{fill}%
\end{pgfscope}%
\begin{pgfscope}%
\pgfpathrectangle{\pgfqpoint{0.150000in}{0.150000in}}{\pgfqpoint{2.700000in}{1.950000in}}%
\pgfusepath{clip}%
\pgfsetbuttcap%
\pgfsetroundjoin%
\definecolor{currentfill}{rgb}{0.830944,0.693244,0.704151}%
\pgfsetfillcolor{currentfill}%
\pgfsetlinewidth{0.000000pt}%
\definecolor{currentstroke}{rgb}{0.000000,0.000000,0.000000}%
\pgfsetstrokecolor{currentstroke}%
\pgfsetdash{}{0pt}%
\pgfpathmoveto{\pgfqpoint{1.421196in}{0.541487in}}%
\pgfpathlineto{\pgfqpoint{1.459398in}{0.610393in}}%
\pgfpathlineto{\pgfqpoint{1.421347in}{0.606895in}}%
\pgfpathlineto{\pgfqpoint{1.383236in}{0.550237in}}%
\pgfpathclose%
\pgfusepath{fill}%
\end{pgfscope}%
\begin{pgfscope}%
\pgfpathrectangle{\pgfqpoint{0.150000in}{0.150000in}}{\pgfqpoint{2.700000in}{1.950000in}}%
\pgfusepath{clip}%
\pgfsetbuttcap%
\pgfsetroundjoin%
\definecolor{currentfill}{rgb}{0.959574,0.964553,0.971523}%
\pgfsetfillcolor{currentfill}%
\pgfsetlinewidth{0.000000pt}%
\definecolor{currentstroke}{rgb}{0.000000,0.000000,0.000000}%
\pgfsetstrokecolor{currentstroke}%
\pgfsetdash{}{0pt}%
\pgfpathmoveto{\pgfqpoint{1.306452in}{0.948670in}}%
\pgfpathlineto{\pgfqpoint{1.343990in}{1.031204in}}%
\pgfpathlineto{\pgfqpoint{1.305363in}{1.076312in}}%
\pgfpathlineto{\pgfqpoint{1.268289in}{0.981052in}}%
\pgfpathclose%
\pgfusepath{fill}%
\end{pgfscope}%
\begin{pgfscope}%
\pgfpathrectangle{\pgfqpoint{0.150000in}{0.150000in}}{\pgfqpoint{2.700000in}{1.950000in}}%
\pgfusepath{clip}%
\pgfsetbuttcap%
\pgfsetroundjoin%
\definecolor{currentfill}{rgb}{0.984452,0.986366,0.989047}%
\pgfsetfillcolor{currentfill}%
\pgfsetlinewidth{0.000000pt}%
\definecolor{currentstroke}{rgb}{0.000000,0.000000,0.000000}%
\pgfsetstrokecolor{currentstroke}%
\pgfsetdash{}{0pt}%
\pgfpathmoveto{\pgfqpoint{1.497983in}{0.953629in}}%
\pgfpathlineto{\pgfqpoint{1.536486in}{0.948670in}}%
\pgfpathlineto{\pgfqpoint{1.498173in}{0.981052in}}%
\pgfpathlineto{\pgfqpoint{1.459437in}{0.998699in}}%
\pgfpathclose%
\pgfusepath{fill}%
\end{pgfscope}%
\begin{pgfscope}%
\pgfpathrectangle{\pgfqpoint{0.150000in}{0.150000in}}{\pgfqpoint{2.700000in}{1.950000in}}%
\pgfusepath{clip}%
\pgfsetbuttcap%
\pgfsetroundjoin%
\definecolor{currentfill}{rgb}{0.956311,0.920726,0.923545}%
\pgfsetfillcolor{currentfill}%
\pgfsetlinewidth{0.000000pt}%
\definecolor{currentstroke}{rgb}{0.000000,0.000000,0.000000}%
\pgfsetstrokecolor{currentstroke}%
\pgfsetdash{}{0pt}%
\pgfpathmoveto{\pgfqpoint{1.767962in}{0.855822in}}%
\pgfpathlineto{\pgfqpoint{1.805063in}{0.838932in}}%
\pgfpathlineto{\pgfqpoint{1.766268in}{0.859070in}}%
\pgfpathlineto{\pgfqpoint{1.728790in}{0.863640in}}%
\pgfpathclose%
\pgfusepath{fill}%
\end{pgfscope}%
\begin{pgfscope}%
\pgfpathrectangle{\pgfqpoint{0.150000in}{0.150000in}}{\pgfqpoint{2.700000in}{1.950000in}}%
\pgfusepath{clip}%
\pgfsetbuttcap%
\pgfsetroundjoin%
\definecolor{currentfill}{rgb}{0.978232,0.980913,0.984666}%
\pgfsetfillcolor{currentfill}%
\pgfsetlinewidth{0.000000pt}%
\definecolor{currentstroke}{rgb}{0.000000,0.000000,0.000000}%
\pgfsetstrokecolor{currentstroke}%
\pgfsetdash{}{0pt}%
\pgfpathmoveto{\pgfqpoint{1.344666in}{0.916246in}}%
\pgfpathlineto{\pgfqpoint{1.382388in}{0.998699in}}%
\pgfpathlineto{\pgfqpoint{1.343990in}{1.031204in}}%
\pgfpathlineto{\pgfqpoint{1.306452in}{0.948670in}}%
\pgfpathclose%
\pgfusepath{fill}%
\end{pgfscope}%
\begin{pgfscope}%
\pgfpathrectangle{\pgfqpoint{0.150000in}{0.150000in}}{\pgfqpoint{2.700000in}{1.950000in}}%
\pgfusepath{clip}%
\pgfsetbuttcap%
\pgfsetroundjoin%
\definecolor{currentfill}{rgb}{0.994301,0.989660,0.990028}%
\pgfsetfillcolor{currentfill}%
\pgfsetlinewidth{0.000000pt}%
\definecolor{currentstroke}{rgb}{0.000000,0.000000,0.000000}%
\pgfsetstrokecolor{currentstroke}%
\pgfsetdash{}{0pt}%
\pgfpathmoveto{\pgfqpoint{1.536486in}{0.921070in}}%
\pgfpathlineto{\pgfqpoint{1.574851in}{0.916246in}}%
\pgfpathlineto{\pgfqpoint{1.536486in}{0.948670in}}%
\pgfpathlineto{\pgfqpoint{1.497983in}{0.953629in}}%
\pgfpathclose%
\pgfusepath{fill}%
\end{pgfscope}%
\begin{pgfscope}%
\pgfpathrectangle{\pgfqpoint{0.150000in}{0.150000in}}{\pgfqpoint{2.700000in}{1.950000in}}%
\pgfusepath{clip}%
\pgfsetbuttcap%
\pgfsetroundjoin%
\definecolor{currentfill}{rgb}{0.925919,0.865579,0.870358}%
\pgfsetfillcolor{currentfill}%
\pgfsetlinewidth{0.000000pt}%
\definecolor{currentstroke}{rgb}{0.000000,0.000000,0.000000}%
\pgfsetstrokecolor{currentstroke}%
\pgfsetdash{}{0pt}%
\pgfpathmoveto{\pgfqpoint{1.459600in}{0.741244in}}%
\pgfpathlineto{\pgfqpoint{1.497929in}{0.810768in}}%
\pgfpathlineto{\pgfqpoint{1.459608in}{0.818720in}}%
\pgfpathlineto{\pgfqpoint{1.421371in}{0.761602in}}%
\pgfpathclose%
\pgfusepath{fill}%
\end{pgfscope}%
\begin{pgfscope}%
\pgfpathrectangle{\pgfqpoint{0.150000in}{0.150000in}}{\pgfqpoint{2.700000in}{1.950000in}}%
\pgfusepath{clip}%
\pgfsetbuttcap%
\pgfsetroundjoin%
\definecolor{currentfill}{rgb}{0.996890,0.997273,0.997809}%
\pgfsetfillcolor{currentfill}%
\pgfsetlinewidth{0.000000pt}%
\definecolor{currentstroke}{rgb}{0.000000,0.000000,0.000000}%
\pgfsetstrokecolor{currentstroke}%
\pgfsetdash{}{0pt}%
\pgfpathmoveto{\pgfqpoint{1.382930in}{0.883780in}}%
\pgfpathlineto{\pgfqpoint{1.420837in}{0.966151in}}%
\pgfpathlineto{\pgfqpoint{1.382388in}{0.998699in}}%
\pgfpathlineto{\pgfqpoint{1.344666in}{0.916246in}}%
\pgfpathclose%
\pgfusepath{fill}%
\end{pgfscope}%
\begin{pgfscope}%
\pgfpathrectangle{\pgfqpoint{0.150000in}{0.150000in}}{\pgfqpoint{2.700000in}{1.950000in}}%
\pgfusepath{clip}%
\pgfsetbuttcap%
\pgfsetroundjoin%
\definecolor{currentfill}{rgb}{0.499341,0.560999,0.647319}%
\pgfsetfillcolor{currentfill}%
\pgfsetlinewidth{0.000000pt}%
\definecolor{currentstroke}{rgb}{0.000000,0.000000,0.000000}%
\pgfsetstrokecolor{currentstroke}%
\pgfsetdash{}{0pt}%
\pgfpathmoveto{\pgfqpoint{0.806412in}{1.560548in}}%
\pgfpathlineto{\pgfqpoint{0.843618in}{1.619152in}}%
\pgfpathlineto{\pgfqpoint{0.805605in}{1.651223in}}%
\pgfpathlineto{\pgfqpoint{0.767563in}{1.605861in}}%
\pgfpathclose%
\pgfusepath{fill}%
\end{pgfscope}%
\begin{pgfscope}%
\pgfpathrectangle{\pgfqpoint{0.150000in}{0.150000in}}{\pgfqpoint{2.700000in}{1.950000in}}%
\pgfusepath{clip}%
\pgfsetbuttcap%
\pgfsetroundjoin%
\definecolor{currentfill}{rgb}{0.530438,0.588266,0.669225}%
\pgfsetfillcolor{currentfill}%
\pgfsetlinewidth{0.000000pt}%
\definecolor{currentstroke}{rgb}{0.000000,0.000000,0.000000}%
\pgfsetstrokecolor{currentstroke}%
\pgfsetdash{}{0pt}%
\pgfpathmoveto{\pgfqpoint{0.844384in}{1.528452in}}%
\pgfpathlineto{\pgfqpoint{0.882473in}{1.573777in}}%
\pgfpathlineto{\pgfqpoint{0.843618in}{1.619152in}}%
\pgfpathlineto{\pgfqpoint{0.806412in}{1.560548in}}%
\pgfpathclose%
\pgfusepath{fill}%
\end{pgfscope}%
\begin{pgfscope}%
\pgfpathrectangle{\pgfqpoint{0.150000in}{0.150000in}}{\pgfqpoint{2.700000in}{1.950000in}}%
\pgfusepath{clip}%
\pgfsetbuttcap%
\pgfsetroundjoin%
\definecolor{currentfill}{rgb}{0.561535,0.615533,0.691131}%
\pgfsetfillcolor{currentfill}%
\pgfsetlinewidth{0.000000pt}%
\definecolor{currentstroke}{rgb}{0.000000,0.000000,0.000000}%
\pgfsetstrokecolor{currentstroke}%
\pgfsetdash{}{0pt}%
\pgfpathmoveto{\pgfqpoint{0.883194in}{1.483178in}}%
\pgfpathlineto{\pgfqpoint{0.920540in}{1.541651in}}%
\pgfpathlineto{\pgfqpoint{0.882473in}{1.573777in}}%
\pgfpathlineto{\pgfqpoint{0.844384in}{1.528452in}}%
\pgfpathclose%
\pgfusepath{fill}%
\end{pgfscope}%
\begin{pgfscope}%
\pgfpathrectangle{\pgfqpoint{0.150000in}{0.150000in}}{\pgfqpoint{2.700000in}{1.950000in}}%
\pgfusepath{clip}%
\pgfsetbuttcap%
\pgfsetroundjoin%
\definecolor{currentfill}{rgb}{0.586412,0.637347,0.708655}%
\pgfsetfillcolor{currentfill}%
\pgfsetlinewidth{0.000000pt}%
\definecolor{currentstroke}{rgb}{0.000000,0.000000,0.000000}%
\pgfsetstrokecolor{currentstroke}%
\pgfsetdash{}{0pt}%
\pgfpathmoveto{\pgfqpoint{0.921220in}{1.451028in}}%
\pgfpathlineto{\pgfqpoint{0.959356in}{1.496315in}}%
\pgfpathlineto{\pgfqpoint{0.920540in}{1.541651in}}%
\pgfpathlineto{\pgfqpoint{0.883194in}{1.483178in}}%
\pgfpathclose%
\pgfusepath{fill}%
\end{pgfscope}%
\begin{pgfscope}%
\pgfpathrectangle{\pgfqpoint{0.150000in}{0.150000in}}{\pgfqpoint{2.700000in}{1.950000in}}%
\pgfusepath{clip}%
\pgfsetbuttcap%
\pgfsetroundjoin%
\definecolor{currentfill}{rgb}{0.986703,0.975873,0.976731}%
\pgfsetfillcolor{currentfill}%
\pgfsetlinewidth{0.000000pt}%
\definecolor{currentstroke}{rgb}{0.000000,0.000000,0.000000}%
\pgfsetstrokecolor{currentstroke}%
\pgfsetdash{}{0pt}%
\pgfpathmoveto{\pgfqpoint{1.421244in}{0.851271in}}%
\pgfpathlineto{\pgfqpoint{1.459429in}{0.921070in}}%
\pgfpathlineto{\pgfqpoint{1.420837in}{0.966151in}}%
\pgfpathlineto{\pgfqpoint{1.382930in}{0.883780in}}%
\pgfpathclose%
\pgfusepath{fill}%
\end{pgfscope}%
\begin{pgfscope}%
\pgfpathrectangle{\pgfqpoint{0.150000in}{0.150000in}}{\pgfqpoint{2.700000in}{1.950000in}}%
\pgfusepath{clip}%
\pgfsetbuttcap%
\pgfsetroundjoin%
\definecolor{currentfill}{rgb}{0.611290,0.659161,0.726180}%
\pgfsetfillcolor{currentfill}%
\pgfsetlinewidth{0.000000pt}%
\definecolor{currentstroke}{rgb}{0.000000,0.000000,0.000000}%
\pgfsetstrokecolor{currentstroke}%
\pgfsetdash{}{0pt}%
\pgfpathmoveto{\pgfqpoint{0.959296in}{1.418836in}}%
\pgfpathlineto{\pgfqpoint{0.997478in}{1.464135in}}%
\pgfpathlineto{\pgfqpoint{0.959356in}{1.496315in}}%
\pgfpathlineto{\pgfqpoint{0.921220in}{1.451028in}}%
\pgfpathclose%
\pgfusepath{fill}%
\end{pgfscope}%
\begin{pgfscope}%
\pgfpathrectangle{\pgfqpoint{0.150000in}{0.150000in}}{\pgfqpoint{2.700000in}{1.950000in}}%
\pgfusepath{clip}%
\pgfsetbuttcap%
\pgfsetroundjoin%
\definecolor{currentfill}{rgb}{0.642387,0.686428,0.748085}%
\pgfsetfillcolor{currentfill}%
\pgfsetlinewidth{0.000000pt}%
\definecolor{currentstroke}{rgb}{0.000000,0.000000,0.000000}%
\pgfsetstrokecolor{currentstroke}%
\pgfsetdash{}{0pt}%
\pgfpathmoveto{\pgfqpoint{0.998073in}{1.373588in}}%
\pgfpathlineto{\pgfqpoint{1.036255in}{1.418836in}}%
\pgfpathlineto{\pgfqpoint{0.997478in}{1.464135in}}%
\pgfpathlineto{\pgfqpoint{0.959296in}{1.418836in}}%
\pgfpathclose%
\pgfusepath{fill}%
\end{pgfscope}%
\begin{pgfscope}%
\pgfpathrectangle{\pgfqpoint{0.150000in}{0.150000in}}{\pgfqpoint{2.700000in}{1.950000in}}%
\pgfusepath{clip}%
\pgfsetbuttcap%
\pgfsetroundjoin%
\definecolor{currentfill}{rgb}{0.899326,0.817325,0.823820}%
\pgfsetfillcolor{currentfill}%
\pgfsetlinewidth{0.000000pt}%
\definecolor{currentstroke}{rgb}{0.000000,0.000000,0.000000}%
\pgfsetstrokecolor{currentstroke}%
\pgfsetdash{}{0pt}%
\pgfpathmoveto{\pgfqpoint{1.459499in}{0.675904in}}%
\pgfpathlineto{\pgfqpoint{1.497878in}{0.745326in}}%
\pgfpathlineto{\pgfqpoint{1.459600in}{0.741244in}}%
\pgfpathlineto{\pgfqpoint{1.421359in}{0.684256in}}%
\pgfpathclose%
\pgfusepath{fill}%
\end{pgfscope}%
\begin{pgfscope}%
\pgfpathrectangle{\pgfqpoint{0.150000in}{0.150000in}}{\pgfqpoint{2.700000in}{1.950000in}}%
\pgfusepath{clip}%
\pgfsetbuttcap%
\pgfsetroundjoin%
\definecolor{currentfill}{rgb}{0.667264,0.708241,0.765610}%
\pgfsetfillcolor{currentfill}%
\pgfsetlinewidth{0.000000pt}%
\definecolor{currentstroke}{rgb}{0.000000,0.000000,0.000000}%
\pgfsetstrokecolor{currentstroke}%
\pgfsetdash{}{0pt}%
\pgfpathmoveto{\pgfqpoint{1.036203in}{1.341342in}}%
\pgfpathlineto{\pgfqpoint{1.074431in}{1.386602in}}%
\pgfpathlineto{\pgfqpoint{1.036255in}{1.418836in}}%
\pgfpathlineto{\pgfqpoint{0.998073in}{1.373588in}}%
\pgfpathclose%
\pgfusepath{fill}%
\end{pgfscope}%
\begin{pgfscope}%
\pgfpathrectangle{\pgfqpoint{0.150000in}{0.150000in}}{\pgfqpoint{2.700000in}{1.950000in}}%
\pgfusepath{clip}%
\pgfsetbuttcap%
\pgfsetroundjoin%
\definecolor{currentfill}{rgb}{0.986703,0.975873,0.976731}%
\pgfsetfillcolor{currentfill}%
\pgfsetlinewidth{0.000000pt}%
\definecolor{currentstroke}{rgb}{0.000000,0.000000,0.000000}%
\pgfsetstrokecolor{currentstroke}%
\pgfsetdash{}{0pt}%
\pgfpathmoveto{\pgfqpoint{1.575087in}{0.900926in}}%
\pgfpathlineto{\pgfqpoint{1.613358in}{0.896180in}}%
\pgfpathlineto{\pgfqpoint{1.574851in}{0.916246in}}%
\pgfpathlineto{\pgfqpoint{1.536486in}{0.921070in}}%
\pgfpathclose%
\pgfusepath{fill}%
\end{pgfscope}%
\begin{pgfscope}%
\pgfpathrectangle{\pgfqpoint{0.150000in}{0.150000in}}{\pgfqpoint{2.700000in}{1.950000in}}%
\pgfusepath{clip}%
\pgfsetbuttcap%
\pgfsetroundjoin%
\definecolor{currentfill}{rgb}{0.698361,0.735509,0.787515}%
\pgfsetfillcolor{currentfill}%
\pgfsetlinewidth{0.000000pt}%
\definecolor{currentstroke}{rgb}{0.000000,0.000000,0.000000}%
\pgfsetstrokecolor{currentstroke}%
\pgfsetdash{}{0pt}%
\pgfpathmoveto{\pgfqpoint{1.074941in}{1.296132in}}%
\pgfpathlineto{\pgfqpoint{1.113170in}{1.341342in}}%
\pgfpathlineto{\pgfqpoint{1.074431in}{1.386602in}}%
\pgfpathlineto{\pgfqpoint{1.036203in}{1.341342in}}%
\pgfpathclose%
\pgfusepath{fill}%
\end{pgfscope}%
\begin{pgfscope}%
\pgfpathrectangle{\pgfqpoint{0.150000in}{0.150000in}}{\pgfqpoint{2.700000in}{1.950000in}}%
\pgfusepath{clip}%
\pgfsetbuttcap%
\pgfsetroundjoin%
\definecolor{currentfill}{rgb}{0.971507,0.948300,0.950138}%
\pgfsetfillcolor{currentfill}%
\pgfsetlinewidth{0.000000pt}%
\definecolor{currentstroke}{rgb}{0.000000,0.000000,0.000000}%
\pgfsetstrokecolor{currentstroke}%
\pgfsetdash{}{0pt}%
\pgfpathmoveto{\pgfqpoint{1.459608in}{0.818720in}}%
\pgfpathlineto{\pgfqpoint{1.497933in}{0.888467in}}%
\pgfpathlineto{\pgfqpoint{1.459429in}{0.921070in}}%
\pgfpathlineto{\pgfqpoint{1.421244in}{0.851271in}}%
\pgfpathclose%
\pgfusepath{fill}%
\end{pgfscope}%
\begin{pgfscope}%
\pgfpathrectangle{\pgfqpoint{0.150000in}{0.150000in}}{\pgfqpoint{2.700000in}{1.950000in}}%
\pgfusepath{clip}%
\pgfsetbuttcap%
\pgfsetroundjoin%
\definecolor{currentfill}{rgb}{0.723238,0.757322,0.805040}%
\pgfsetfillcolor{currentfill}%
\pgfsetlinewidth{0.000000pt}%
\definecolor{currentstroke}{rgb}{0.000000,0.000000,0.000000}%
\pgfsetstrokecolor{currentstroke}%
\pgfsetdash{}{0pt}%
\pgfpathmoveto{\pgfqpoint{1.113126in}{1.263831in}}%
\pgfpathlineto{\pgfqpoint{1.151401in}{1.309054in}}%
\pgfpathlineto{\pgfqpoint{1.113170in}{1.341342in}}%
\pgfpathlineto{\pgfqpoint{1.074941in}{1.296132in}}%
\pgfpathclose%
\pgfusepath{fill}%
\end{pgfscope}%
\begin{pgfscope}%
\pgfpathrectangle{\pgfqpoint{0.150000in}{0.150000in}}{\pgfqpoint{2.700000in}{1.950000in}}%
\pgfusepath{clip}%
\pgfsetbuttcap%
\pgfsetroundjoin%
\definecolor{currentfill}{rgb}{0.748116,0.779136,0.822564}%
\pgfsetfillcolor{currentfill}%
\pgfsetlinewidth{0.000000pt}%
\definecolor{currentstroke}{rgb}{0.000000,0.000000,0.000000}%
\pgfsetstrokecolor{currentstroke}%
\pgfsetdash{}{0pt}%
\pgfpathmoveto{\pgfqpoint{1.151361in}{1.231489in}}%
\pgfpathlineto{\pgfqpoint{1.190101in}{1.263831in}}%
\pgfpathlineto{\pgfqpoint{1.151401in}{1.309054in}}%
\pgfpathlineto{\pgfqpoint{1.113126in}{1.263831in}}%
\pgfpathclose%
\pgfusepath{fill}%
\end{pgfscope}%
\begin{pgfscope}%
\pgfpathrectangle{\pgfqpoint{0.150000in}{0.150000in}}{\pgfqpoint{2.700000in}{1.950000in}}%
\pgfusepath{clip}%
\pgfsetbuttcap%
\pgfsetroundjoin%
\definecolor{currentfill}{rgb}{0.779213,0.806403,0.844470}%
\pgfsetfillcolor{currentfill}%
\pgfsetlinewidth{0.000000pt}%
\definecolor{currentstroke}{rgb}{0.000000,0.000000,0.000000}%
\pgfsetstrokecolor{currentstroke}%
\pgfsetdash{}{0pt}%
\pgfpathmoveto{\pgfqpoint{1.190065in}{1.186305in}}%
\pgfpathlineto{\pgfqpoint{1.228386in}{1.231489in}}%
\pgfpathlineto{\pgfqpoint{1.190101in}{1.263831in}}%
\pgfpathlineto{\pgfqpoint{1.151361in}{1.231489in}}%
\pgfpathclose%
\pgfusepath{fill}%
\end{pgfscope}%
\begin{pgfscope}%
\pgfpathrectangle{\pgfqpoint{0.150000in}{0.150000in}}{\pgfqpoint{2.700000in}{1.950000in}}%
\pgfusepath{clip}%
\pgfsetbuttcap%
\pgfsetroundjoin%
\definecolor{currentfill}{rgb}{0.804090,0.828217,0.861994}%
\pgfsetfillcolor{currentfill}%
\pgfsetlinewidth{0.000000pt}%
\definecolor{currentstroke}{rgb}{0.000000,0.000000,0.000000}%
\pgfsetstrokecolor{currentstroke}%
\pgfsetdash{}{0pt}%
\pgfpathmoveto{\pgfqpoint{1.228354in}{1.153908in}}%
\pgfpathlineto{\pgfqpoint{1.267047in}{1.186305in}}%
\pgfpathlineto{\pgfqpoint{1.228386in}{1.231489in}}%
\pgfpathlineto{\pgfqpoint{1.190065in}{1.186305in}}%
\pgfpathclose%
\pgfusepath{fill}%
\end{pgfscope}%
\begin{pgfscope}%
\pgfpathrectangle{\pgfqpoint{0.150000in}{0.150000in}}{\pgfqpoint{2.700000in}{1.950000in}}%
\pgfusepath{clip}%
\pgfsetbuttcap%
\pgfsetroundjoin%
\definecolor{currentfill}{rgb}{0.835187,0.855484,0.883900}%
\pgfsetfillcolor{currentfill}%
\pgfsetlinewidth{0.000000pt}%
\definecolor{currentstroke}{rgb}{0.000000,0.000000,0.000000}%
\pgfsetstrokecolor{currentstroke}%
\pgfsetdash{}{0pt}%
\pgfpathmoveto{\pgfqpoint{1.267020in}{1.108763in}}%
\pgfpathlineto{\pgfqpoint{1.305387in}{1.153908in}}%
\pgfpathlineto{\pgfqpoint{1.267047in}{1.186305in}}%
\pgfpathlineto{\pgfqpoint{1.228354in}{1.153908in}}%
\pgfpathclose%
\pgfusepath{fill}%
\end{pgfscope}%
\begin{pgfscope}%
\pgfpathrectangle{\pgfqpoint{0.150000in}{0.150000in}}{\pgfqpoint{2.700000in}{1.950000in}}%
\pgfusepath{clip}%
\pgfsetbuttcap%
\pgfsetroundjoin%
\definecolor{currentfill}{rgb}{0.860064,0.877298,0.901425}%
\pgfsetfillcolor{currentfill}%
\pgfsetlinewidth{0.000000pt}%
\definecolor{currentstroke}{rgb}{0.000000,0.000000,0.000000}%
\pgfsetstrokecolor{currentstroke}%
\pgfsetdash{}{0pt}%
\pgfpathmoveto{\pgfqpoint{1.305363in}{1.076312in}}%
\pgfpathlineto{\pgfqpoint{1.344010in}{1.108763in}}%
\pgfpathlineto{\pgfqpoint{1.305387in}{1.153908in}}%
\pgfpathlineto{\pgfqpoint{1.267020in}{1.108763in}}%
\pgfpathclose%
\pgfusepath{fill}%
\end{pgfscope}%
\begin{pgfscope}%
\pgfpathrectangle{\pgfqpoint{0.150000in}{0.150000in}}{\pgfqpoint{2.700000in}{1.950000in}}%
\pgfusepath{clip}%
\pgfsetbuttcap%
\pgfsetroundjoin%
\definecolor{currentfill}{rgb}{0.868934,0.762178,0.770634}%
\pgfsetfillcolor{currentfill}%
\pgfsetlinewidth{0.000000pt}%
\definecolor{currentstroke}{rgb}{0.000000,0.000000,0.000000}%
\pgfsetstrokecolor{currentstroke}%
\pgfsetdash{}{0pt}%
\pgfpathmoveto{\pgfqpoint{1.459398in}{0.610393in}}%
\pgfpathlineto{\pgfqpoint{1.497874in}{0.667501in}}%
\pgfpathlineto{\pgfqpoint{1.459499in}{0.675904in}}%
\pgfpathlineto{\pgfqpoint{1.421347in}{0.606895in}}%
\pgfpathclose%
\pgfusepath{fill}%
\end{pgfscope}%
\begin{pgfscope}%
\pgfpathrectangle{\pgfqpoint{0.150000in}{0.150000in}}{\pgfqpoint{2.700000in}{1.950000in}}%
\pgfusepath{clip}%
\pgfsetbuttcap%
\pgfsetroundjoin%
\definecolor{currentfill}{rgb}{0.891161,0.904565,0.923330}%
\pgfsetfillcolor{currentfill}%
\pgfsetlinewidth{0.000000pt}%
\definecolor{currentstroke}{rgb}{0.000000,0.000000,0.000000}%
\pgfsetstrokecolor{currentstroke}%
\pgfsetdash{}{0pt}%
\pgfpathmoveto{\pgfqpoint{1.343990in}{1.031204in}}%
\pgfpathlineto{\pgfqpoint{1.382404in}{1.076312in}}%
\pgfpathlineto{\pgfqpoint{1.344010in}{1.108763in}}%
\pgfpathlineto{\pgfqpoint{1.305363in}{1.076312in}}%
\pgfpathclose%
\pgfusepath{fill}%
\end{pgfscope}%
\begin{pgfscope}%
\pgfpathrectangle{\pgfqpoint{0.150000in}{0.150000in}}{\pgfqpoint{2.700000in}{1.950000in}}%
\pgfusepath{clip}%
\pgfsetbuttcap%
\pgfsetroundjoin%
\definecolor{currentfill}{rgb}{0.922258,0.931832,0.945236}%
\pgfsetfillcolor{currentfill}%
\pgfsetlinewidth{0.000000pt}%
\definecolor{currentstroke}{rgb}{0.000000,0.000000,0.000000}%
\pgfsetstrokecolor{currentstroke}%
\pgfsetdash{}{0pt}%
\pgfpathmoveto{\pgfqpoint{1.382388in}{0.998699in}}%
\pgfpathlineto{\pgfqpoint{1.420989in}{1.031204in}}%
\pgfpathlineto{\pgfqpoint{1.382404in}{1.076312in}}%
\pgfpathlineto{\pgfqpoint{1.343990in}{1.031204in}}%
\pgfpathclose%
\pgfusepath{fill}%
\end{pgfscope}%
\begin{pgfscope}%
\pgfpathrectangle{\pgfqpoint{0.150000in}{0.150000in}}{\pgfqpoint{2.700000in}{1.950000in}}%
\pgfusepath{clip}%
\pgfsetbuttcap%
\pgfsetroundjoin%
\definecolor{currentfill}{rgb}{0.979105,0.962086,0.963434}%
\pgfsetfillcolor{currentfill}%
\pgfsetlinewidth{0.000000pt}%
\definecolor{currentstroke}{rgb}{0.000000,0.000000,0.000000}%
\pgfsetstrokecolor{currentstroke}%
\pgfsetdash{}{0pt}%
\pgfpathmoveto{\pgfqpoint{1.691178in}{0.880707in}}%
\pgfpathlineto{\pgfqpoint{1.728790in}{0.863640in}}%
\pgfpathlineto{\pgfqpoint{1.690043in}{0.883780in}}%
\pgfpathlineto{\pgfqpoint{1.652148in}{0.888467in}}%
\pgfpathclose%
\pgfusepath{fill}%
\end{pgfscope}%
\begin{pgfscope}%
\pgfpathrectangle{\pgfqpoint{0.150000in}{0.150000in}}{\pgfqpoint{2.700000in}{1.950000in}}%
\pgfusepath{clip}%
\pgfsetbuttcap%
\pgfsetroundjoin%
\definecolor{currentfill}{rgb}{0.963909,0.934513,0.936841}%
\pgfsetfillcolor{currentfill}%
\pgfsetlinewidth{0.000000pt}%
\definecolor{currentstroke}{rgb}{0.000000,0.000000,0.000000}%
\pgfsetstrokecolor{currentstroke}%
\pgfsetdash{}{0pt}%
\pgfpathmoveto{\pgfqpoint{1.807375in}{0.847956in}}%
\pgfpathlineto{\pgfqpoint{1.844002in}{0.818720in}}%
\pgfpathlineto{\pgfqpoint{1.805063in}{0.838932in}}%
\pgfpathlineto{\pgfqpoint{1.767962in}{0.855822in}}%
\pgfpathclose%
\pgfusepath{fill}%
\end{pgfscope}%
\begin{pgfscope}%
\pgfpathrectangle{\pgfqpoint{0.150000in}{0.150000in}}{\pgfqpoint{2.700000in}{1.950000in}}%
\pgfusepath{clip}%
\pgfsetbuttcap%
\pgfsetroundjoin%
\definecolor{currentfill}{rgb}{0.940916,0.948192,0.958379}%
\pgfsetfillcolor{currentfill}%
\pgfsetlinewidth{0.000000pt}%
\definecolor{currentstroke}{rgb}{0.000000,0.000000,0.000000}%
\pgfsetstrokecolor{currentstroke}%
\pgfsetdash{}{0pt}%
\pgfpathmoveto{\pgfqpoint{1.420837in}{0.966151in}}%
\pgfpathlineto{\pgfqpoint{1.459437in}{0.998699in}}%
\pgfpathlineto{\pgfqpoint{1.420989in}{1.031204in}}%
\pgfpathlineto{\pgfqpoint{1.382388in}{0.998699in}}%
\pgfpathclose%
\pgfusepath{fill}%
\end{pgfscope}%
\begin{pgfscope}%
\pgfpathrectangle{\pgfqpoint{0.150000in}{0.150000in}}{\pgfqpoint{2.700000in}{1.950000in}}%
\pgfusepath{clip}%
\pgfsetbuttcap%
\pgfsetroundjoin%
\definecolor{currentfill}{rgb}{0.972013,0.975460,0.980285}%
\pgfsetfillcolor{currentfill}%
\pgfsetlinewidth{0.000000pt}%
\definecolor{currentstroke}{rgb}{0.000000,0.000000,0.000000}%
\pgfsetstrokecolor{currentstroke}%
\pgfsetdash{}{0pt}%
\pgfpathmoveto{\pgfqpoint{1.459429in}{0.921070in}}%
\pgfpathlineto{\pgfqpoint{1.497983in}{0.953629in}}%
\pgfpathlineto{\pgfqpoint{1.459437in}{0.998699in}}%
\pgfpathlineto{\pgfqpoint{1.420837in}{0.966151in}}%
\pgfpathclose%
\pgfusepath{fill}%
\end{pgfscope}%
\begin{pgfscope}%
\pgfpathrectangle{\pgfqpoint{0.150000in}{0.150000in}}{\pgfqpoint{2.700000in}{1.950000in}}%
\pgfusepath{clip}%
\pgfsetbuttcap%
\pgfsetroundjoin%
\definecolor{currentfill}{rgb}{0.996890,0.997273,0.997809}%
\pgfsetfillcolor{currentfill}%
\pgfsetlinewidth{0.000000pt}%
\definecolor{currentstroke}{rgb}{0.000000,0.000000,0.000000}%
\pgfsetstrokecolor{currentstroke}%
\pgfsetdash{}{0pt}%
\pgfpathmoveto{\pgfqpoint{1.497933in}{0.888467in}}%
\pgfpathlineto{\pgfqpoint{1.536486in}{0.921070in}}%
\pgfpathlineto{\pgfqpoint{1.497983in}{0.953629in}}%
\pgfpathlineto{\pgfqpoint{1.459429in}{0.921070in}}%
\pgfpathclose%
\pgfusepath{fill}%
\end{pgfscope}%
\begin{pgfscope}%
\pgfpathrectangle{\pgfqpoint{0.150000in}{0.150000in}}{\pgfqpoint{2.700000in}{1.950000in}}%
\pgfusepath{clip}%
\pgfsetbuttcap%
\pgfsetroundjoin%
\definecolor{currentfill}{rgb}{0.967708,0.941406,0.943490}%
\pgfsetfillcolor{currentfill}%
\pgfsetlinewidth{0.000000pt}%
\definecolor{currentstroke}{rgb}{0.000000,0.000000,0.000000}%
\pgfsetstrokecolor{currentstroke}%
\pgfsetdash{}{0pt}%
\pgfpathmoveto{\pgfqpoint{1.497929in}{0.810768in}}%
\pgfpathlineto{\pgfqpoint{1.536486in}{0.868250in}}%
\pgfpathlineto{\pgfqpoint{1.497933in}{0.888467in}}%
\pgfpathlineto{\pgfqpoint{1.459608in}{0.818720in}}%
\pgfpathclose%
\pgfusepath{fill}%
\end{pgfscope}%
\begin{pgfscope}%
\pgfpathrectangle{\pgfqpoint{0.150000in}{0.150000in}}{\pgfqpoint{2.700000in}{1.950000in}}%
\pgfusepath{clip}%
\pgfsetbuttcap%
\pgfsetroundjoin%
\definecolor{currentfill}{rgb}{0.842341,0.713925,0.724096}%
\pgfsetfillcolor{currentfill}%
\pgfsetlinewidth{0.000000pt}%
\definecolor{currentstroke}{rgb}{0.000000,0.000000,0.000000}%
\pgfsetstrokecolor{currentstroke}%
\pgfsetdash{}{0pt}%
\pgfpathmoveto{\pgfqpoint{1.459390in}{0.532684in}}%
\pgfpathlineto{\pgfqpoint{1.497823in}{0.601777in}}%
\pgfpathlineto{\pgfqpoint{1.459398in}{0.610393in}}%
\pgfpathlineto{\pgfqpoint{1.421196in}{0.541487in}}%
\pgfpathclose%
\pgfusepath{fill}%
\end{pgfscope}%
\begin{pgfscope}%
\pgfpathrectangle{\pgfqpoint{0.150000in}{0.150000in}}{\pgfqpoint{2.700000in}{1.950000in}}%
\pgfusepath{clip}%
\pgfsetbuttcap%
\pgfsetroundjoin%
\definecolor{currentfill}{rgb}{0.941115,0.893153,0.896952}%
\pgfsetfillcolor{currentfill}%
\pgfsetlinewidth{0.000000pt}%
\definecolor{currentstroke}{rgb}{0.000000,0.000000,0.000000}%
\pgfsetstrokecolor{currentstroke}%
\pgfsetdash{}{0pt}%
\pgfpathmoveto{\pgfqpoint{1.497878in}{0.745326in}}%
\pgfpathlineto{\pgfqpoint{1.536486in}{0.802767in}}%
\pgfpathlineto{\pgfqpoint{1.497929in}{0.810768in}}%
\pgfpathlineto{\pgfqpoint{1.459600in}{0.741244in}}%
\pgfpathclose%
\pgfusepath{fill}%
\end{pgfscope}%
\begin{pgfscope}%
\pgfpathrectangle{\pgfqpoint{0.150000in}{0.150000in}}{\pgfqpoint{2.700000in}{1.950000in}}%
\pgfusepath{clip}%
\pgfsetbuttcap%
\pgfsetroundjoin%
\definecolor{currentfill}{rgb}{0.994301,0.989660,0.990028}%
\pgfsetfillcolor{currentfill}%
\pgfsetlinewidth{0.000000pt}%
\definecolor{currentstroke}{rgb}{0.000000,0.000000,0.000000}%
\pgfsetstrokecolor{currentstroke}%
\pgfsetdash{}{0pt}%
\pgfpathmoveto{\pgfqpoint{1.536486in}{0.868250in}}%
\pgfpathlineto{\pgfqpoint{1.575087in}{0.900926in}}%
\pgfpathlineto{\pgfqpoint{1.536486in}{0.921070in}}%
\pgfpathlineto{\pgfqpoint{1.497933in}{0.888467in}}%
\pgfpathclose%
\pgfusepath{fill}%
\end{pgfscope}%
\begin{pgfscope}%
\pgfpathrectangle{\pgfqpoint{0.150000in}{0.150000in}}{\pgfqpoint{2.700000in}{1.950000in}}%
\pgfusepath{clip}%
\pgfsetbuttcap%
\pgfsetroundjoin%
\definecolor{currentfill}{rgb}{0.998100,0.996553,0.996676}%
\pgfsetfillcolor{currentfill}%
\pgfsetlinewidth{0.000000pt}%
\definecolor{currentstroke}{rgb}{0.000000,0.000000,0.000000}%
\pgfsetstrokecolor{currentstroke}%
\pgfsetdash{}{0pt}%
\pgfpathmoveto{\pgfqpoint{1.614021in}{0.905713in}}%
\pgfpathlineto{\pgfqpoint{1.652148in}{0.888467in}}%
\pgfpathlineto{\pgfqpoint{1.613358in}{0.896180in}}%
\pgfpathlineto{\pgfqpoint{1.575087in}{0.900926in}}%
\pgfpathclose%
\pgfusepath{fill}%
\end{pgfscope}%
\begin{pgfscope}%
\pgfpathrectangle{\pgfqpoint{0.150000in}{0.150000in}}{\pgfqpoint{2.700000in}{1.950000in}}%
\pgfusepath{clip}%
\pgfsetbuttcap%
\pgfsetroundjoin%
\definecolor{currentfill}{rgb}{0.963909,0.934513,0.936841}%
\pgfsetfillcolor{currentfill}%
\pgfsetlinewidth{0.000000pt}%
\definecolor{currentstroke}{rgb}{0.000000,0.000000,0.000000}%
\pgfsetstrokecolor{currentstroke}%
\pgfsetdash{}{0pt}%
\pgfpathmoveto{\pgfqpoint{1.846655in}{0.827586in}}%
\pgfpathlineto{\pgfqpoint{1.883087in}{0.798432in}}%
\pgfpathlineto{\pgfqpoint{1.844002in}{0.818720in}}%
\pgfpathlineto{\pgfqpoint{1.807375in}{0.847956in}}%
\pgfpathclose%
\pgfusepath{fill}%
\end{pgfscope}%
\begin{pgfscope}%
\pgfpathrectangle{\pgfqpoint{0.150000in}{0.150000in}}{\pgfqpoint{2.700000in}{1.950000in}}%
\pgfusepath{clip}%
\pgfsetbuttcap%
\pgfsetroundjoin%
\definecolor{currentfill}{rgb}{0.914522,0.844899,0.850414}%
\pgfsetfillcolor{currentfill}%
\pgfsetlinewidth{0.000000pt}%
\definecolor{currentstroke}{rgb}{0.000000,0.000000,0.000000}%
\pgfsetstrokecolor{currentstroke}%
\pgfsetdash{}{0pt}%
\pgfpathmoveto{\pgfqpoint{1.497874in}{0.667501in}}%
\pgfpathlineto{\pgfqpoint{1.536486in}{0.737112in}}%
\pgfpathlineto{\pgfqpoint{1.497878in}{0.745326in}}%
\pgfpathlineto{\pgfqpoint{1.459499in}{0.675904in}}%
\pgfpathclose%
\pgfusepath{fill}%
\end{pgfscope}%
\begin{pgfscope}%
\pgfpathrectangle{\pgfqpoint{0.150000in}{0.150000in}}{\pgfqpoint{2.700000in}{1.950000in}}%
\pgfusepath{clip}%
\pgfsetbuttcap%
\pgfsetroundjoin%
\definecolor{currentfill}{rgb}{0.986703,0.975873,0.976731}%
\pgfsetfillcolor{currentfill}%
\pgfsetlinewidth{0.000000pt}%
\definecolor{currentstroke}{rgb}{0.000000,0.000000,0.000000}%
\pgfsetstrokecolor{currentstroke}%
\pgfsetdash{}{0pt}%
\pgfpathmoveto{\pgfqpoint{1.730450in}{0.872899in}}%
\pgfpathlineto{\pgfqpoint{1.767962in}{0.855822in}}%
\pgfpathlineto{\pgfqpoint{1.728790in}{0.863640in}}%
\pgfpathlineto{\pgfqpoint{1.691178in}{0.880707in}}%
\pgfpathclose%
\pgfusepath{fill}%
\end{pgfscope}%
\begin{pgfscope}%
\pgfpathrectangle{\pgfqpoint{0.150000in}{0.150000in}}{\pgfqpoint{2.700000in}{1.950000in}}%
\pgfusepath{clip}%
\pgfsetbuttcap%
\pgfsetroundjoin%
\definecolor{currentfill}{rgb}{0.887929,0.796645,0.803876}%
\pgfsetfillcolor{currentfill}%
\pgfsetlinewidth{0.000000pt}%
\definecolor{currentstroke}{rgb}{0.000000,0.000000,0.000000}%
\pgfsetstrokecolor{currentstroke}%
\pgfsetdash{}{0pt}%
\pgfpathmoveto{\pgfqpoint{1.497823in}{0.601777in}}%
\pgfpathlineto{\pgfqpoint{1.536486in}{0.671283in}}%
\pgfpathlineto{\pgfqpoint{1.497874in}{0.667501in}}%
\pgfpathlineto{\pgfqpoint{1.459398in}{0.610393in}}%
\pgfpathclose%
\pgfusepath{fill}%
\end{pgfscope}%
\begin{pgfscope}%
\pgfpathrectangle{\pgfqpoint{0.150000in}{0.150000in}}{\pgfqpoint{2.700000in}{1.950000in}}%
\pgfusepath{clip}%
\pgfsetbuttcap%
\pgfsetroundjoin%
\definecolor{currentfill}{rgb}{0.979105,0.962086,0.963434}%
\pgfsetfillcolor{currentfill}%
\pgfsetlinewidth{0.000000pt}%
\definecolor{currentstroke}{rgb}{0.000000,0.000000,0.000000}%
\pgfsetstrokecolor{currentstroke}%
\pgfsetdash{}{0pt}%
\pgfpathmoveto{\pgfqpoint{1.536486in}{0.802767in}}%
\pgfpathlineto{\pgfqpoint{1.575279in}{0.872899in}}%
\pgfpathlineto{\pgfqpoint{1.536486in}{0.868250in}}%
\pgfpathlineto{\pgfqpoint{1.497929in}{0.810768in}}%
\pgfpathclose%
\pgfusepath{fill}%
\end{pgfscope}%
\begin{pgfscope}%
\pgfpathrectangle{\pgfqpoint{0.150000in}{0.150000in}}{\pgfqpoint{2.700000in}{1.950000in}}%
\pgfusepath{clip}%
\pgfsetbuttcap%
\pgfsetroundjoin%
\definecolor{currentfill}{rgb}{0.857537,0.741498,0.750689}%
\pgfsetfillcolor{currentfill}%
\pgfsetlinewidth{0.000000pt}%
\definecolor{currentstroke}{rgb}{0.000000,0.000000,0.000000}%
\pgfsetstrokecolor{currentstroke}%
\pgfsetdash{}{0pt}%
\pgfpathmoveto{\pgfqpoint{1.497772in}{0.535879in}}%
\pgfpathlineto{\pgfqpoint{1.536486in}{0.593106in}}%
\pgfpathlineto{\pgfqpoint{1.497823in}{0.601777in}}%
\pgfpathlineto{\pgfqpoint{1.459390in}{0.532684in}}%
\pgfpathclose%
\pgfusepath{fill}%
\end{pgfscope}%
\begin{pgfscope}%
\pgfpathrectangle{\pgfqpoint{0.150000in}{0.150000in}}{\pgfqpoint{2.700000in}{1.950000in}}%
\pgfusepath{clip}%
\pgfsetbuttcap%
\pgfsetroundjoin%
\definecolor{currentfill}{rgb}{0.990671,0.991820,0.993428}%
\pgfsetfillcolor{currentfill}%
\pgfsetlinewidth{0.000000pt}%
\definecolor{currentstroke}{rgb}{0.000000,0.000000,0.000000}%
\pgfsetstrokecolor{currentstroke}%
\pgfsetdash{}{0pt}%
\pgfpathmoveto{\pgfqpoint{1.575279in}{0.872899in}}%
\pgfpathlineto{\pgfqpoint{1.614021in}{0.905713in}}%
\pgfpathlineto{\pgfqpoint{1.575087in}{0.900926in}}%
\pgfpathlineto{\pgfqpoint{1.536486in}{0.868250in}}%
\pgfpathclose%
\pgfusepath{fill}%
\end{pgfscope}%
\begin{pgfscope}%
\pgfpathrectangle{\pgfqpoint{0.150000in}{0.150000in}}{\pgfqpoint{2.700000in}{1.950000in}}%
\pgfusepath{clip}%
\pgfsetbuttcap%
\pgfsetroundjoin%
\definecolor{currentfill}{rgb}{0.956311,0.920726,0.923545}%
\pgfsetfillcolor{currentfill}%
\pgfsetlinewidth{0.000000pt}%
\definecolor{currentstroke}{rgb}{0.000000,0.000000,0.000000}%
\pgfsetstrokecolor{currentstroke}%
\pgfsetdash{}{0pt}%
\pgfpathmoveto{\pgfqpoint{1.536486in}{0.737112in}}%
\pgfpathlineto{\pgfqpoint{1.575331in}{0.807140in}}%
\pgfpathlineto{\pgfqpoint{1.536486in}{0.802767in}}%
\pgfpathlineto{\pgfqpoint{1.497878in}{0.745326in}}%
\pgfpathclose%
\pgfusepath{fill}%
\end{pgfscope}%
\begin{pgfscope}%
\pgfpathrectangle{\pgfqpoint{0.150000in}{0.150000in}}{\pgfqpoint{2.700000in}{1.950000in}}%
\pgfusepath{clip}%
\pgfsetbuttcap%
\pgfsetroundjoin%
\definecolor{currentfill}{rgb}{0.978232,0.980913,0.984666}%
\pgfsetfillcolor{currentfill}%
\pgfsetlinewidth{0.000000pt}%
\definecolor{currentstroke}{rgb}{0.000000,0.000000,0.000000}%
\pgfsetstrokecolor{currentstroke}%
\pgfsetdash{}{0pt}%
\pgfpathmoveto{\pgfqpoint{1.653149in}{0.897964in}}%
\pgfpathlineto{\pgfqpoint{1.691178in}{0.880707in}}%
\pgfpathlineto{\pgfqpoint{1.652148in}{0.888467in}}%
\pgfpathlineto{\pgfqpoint{1.614021in}{0.905713in}}%
\pgfpathclose%
\pgfusepath{fill}%
\end{pgfscope}%
\begin{pgfscope}%
\pgfpathrectangle{\pgfqpoint{0.150000in}{0.150000in}}{\pgfqpoint{2.700000in}{1.950000in}}%
\pgfusepath{clip}%
\pgfsetbuttcap%
\pgfsetroundjoin%
\definecolor{currentfill}{rgb}{0.996890,0.997273,0.997809}%
\pgfsetfillcolor{currentfill}%
\pgfsetlinewidth{0.000000pt}%
\definecolor{currentstroke}{rgb}{0.000000,0.000000,0.000000}%
\pgfsetstrokecolor{currentstroke}%
\pgfsetdash{}{0pt}%
\pgfpathmoveto{\pgfqpoint{1.769966in}{0.865042in}}%
\pgfpathlineto{\pgfqpoint{1.807375in}{0.847956in}}%
\pgfpathlineto{\pgfqpoint{1.767962in}{0.855822in}}%
\pgfpathlineto{\pgfqpoint{1.730450in}{0.872899in}}%
\pgfpathclose%
\pgfusepath{fill}%
\end{pgfscope}%
\begin{pgfscope}%
\pgfpathrectangle{\pgfqpoint{0.150000in}{0.150000in}}{\pgfqpoint{2.700000in}{1.950000in}}%
\pgfusepath{clip}%
\pgfsetbuttcap%
\pgfsetroundjoin%
\definecolor{currentfill}{rgb}{0.929718,0.872472,0.877007}%
\pgfsetfillcolor{currentfill}%
\pgfsetlinewidth{0.000000pt}%
\definecolor{currentstroke}{rgb}{0.000000,0.000000,0.000000}%
\pgfsetstrokecolor{currentstroke}%
\pgfsetdash{}{0pt}%
\pgfpathmoveto{\pgfqpoint{1.536486in}{0.671283in}}%
\pgfpathlineto{\pgfqpoint{1.575382in}{0.741206in}}%
\pgfpathlineto{\pgfqpoint{1.536486in}{0.737112in}}%
\pgfpathlineto{\pgfqpoint{1.497874in}{0.667501in}}%
\pgfpathclose%
\pgfusepath{fill}%
\end{pgfscope}%
\begin{pgfscope}%
\pgfpathrectangle{\pgfqpoint{0.150000in}{0.150000in}}{\pgfqpoint{2.700000in}{1.950000in}}%
\pgfusepath{clip}%
\pgfsetbuttcap%
\pgfsetroundjoin%
\definecolor{currentfill}{rgb}{0.903125,0.824219,0.830469}%
\pgfsetfillcolor{currentfill}%
\pgfsetlinewidth{0.000000pt}%
\definecolor{currentstroke}{rgb}{0.000000,0.000000,0.000000}%
\pgfsetstrokecolor{currentstroke}%
\pgfsetdash{}{0pt}%
\pgfpathmoveto{\pgfqpoint{1.536486in}{0.593106in}}%
\pgfpathlineto{\pgfqpoint{1.575386in}{0.662802in}}%
\pgfpathlineto{\pgfqpoint{1.536486in}{0.671283in}}%
\pgfpathlineto{\pgfqpoint{1.497823in}{0.601777in}}%
\pgfpathclose%
\pgfusepath{fill}%
\end{pgfscope}%
\begin{pgfscope}%
\pgfpathrectangle{\pgfqpoint{0.150000in}{0.150000in}}{\pgfqpoint{2.700000in}{1.950000in}}%
\pgfusepath{clip}%
\pgfsetbuttcap%
\pgfsetroundjoin%
\definecolor{currentfill}{rgb}{0.998100,0.996553,0.996676}%
\pgfsetfillcolor{currentfill}%
\pgfsetlinewidth{0.000000pt}%
\definecolor{currentstroke}{rgb}{0.000000,0.000000,0.000000}%
\pgfsetstrokecolor{currentstroke}%
\pgfsetdash{}{0pt}%
\pgfpathmoveto{\pgfqpoint{1.575331in}{0.807140in}}%
\pgfpathlineto{\pgfqpoint{1.614313in}{0.865042in}}%
\pgfpathlineto{\pgfqpoint{1.575279in}{0.872899in}}%
\pgfpathlineto{\pgfqpoint{1.536486in}{0.802767in}}%
\pgfpathclose%
\pgfusepath{fill}%
\end{pgfscope}%
\begin{pgfscope}%
\pgfpathrectangle{\pgfqpoint{0.150000in}{0.150000in}}{\pgfqpoint{2.700000in}{1.950000in}}%
\pgfusepath{clip}%
\pgfsetbuttcap%
\pgfsetroundjoin%
\definecolor{currentfill}{rgb}{0.876532,0.775965,0.783931}%
\pgfsetfillcolor{currentfill}%
\pgfsetlinewidth{0.000000pt}%
\definecolor{currentstroke}{rgb}{0.000000,0.000000,0.000000}%
\pgfsetstrokecolor{currentstroke}%
\pgfsetdash{}{0pt}%
\pgfpathmoveto{\pgfqpoint{1.536486in}{0.526993in}}%
\pgfpathlineto{\pgfqpoint{1.575438in}{0.596582in}}%
\pgfpathlineto{\pgfqpoint{1.536486in}{0.593106in}}%
\pgfpathlineto{\pgfqpoint{1.497772in}{0.535879in}}%
\pgfpathclose%
\pgfusepath{fill}%
\end{pgfscope}%
\begin{pgfscope}%
\pgfpathrectangle{\pgfqpoint{0.150000in}{0.150000in}}{\pgfqpoint{2.700000in}{1.950000in}}%
\pgfusepath{clip}%
\pgfsetbuttcap%
\pgfsetroundjoin%
\definecolor{currentfill}{rgb}{0.959574,0.964553,0.971523}%
\pgfsetfillcolor{currentfill}%
\pgfsetlinewidth{0.000000pt}%
\definecolor{currentstroke}{rgb}{0.000000,0.000000,0.000000}%
\pgfsetstrokecolor{currentstroke}%
\pgfsetdash{}{0pt}%
\pgfpathmoveto{\pgfqpoint{1.614313in}{0.865042in}}%
\pgfpathlineto{\pgfqpoint{1.653149in}{0.897964in}}%
\pgfpathlineto{\pgfqpoint{1.614021in}{0.905713in}}%
\pgfpathlineto{\pgfqpoint{1.575279in}{0.872899in}}%
\pgfpathclose%
\pgfusepath{fill}%
\end{pgfscope}%
\begin{pgfscope}%
\pgfpathrectangle{\pgfqpoint{0.150000in}{0.150000in}}{\pgfqpoint{2.700000in}{1.950000in}}%
\pgfusepath{clip}%
\pgfsetbuttcap%
\pgfsetroundjoin%
\definecolor{currentfill}{rgb}{0.959574,0.964553,0.971523}%
\pgfsetfillcolor{currentfill}%
\pgfsetlinewidth{0.000000pt}%
\definecolor{currentstroke}{rgb}{0.000000,0.000000,0.000000}%
\pgfsetstrokecolor{currentstroke}%
\pgfsetdash{}{0pt}%
\pgfpathmoveto{\pgfqpoint{1.692712in}{0.902774in}}%
\pgfpathlineto{\pgfqpoint{1.730450in}{0.872899in}}%
\pgfpathlineto{\pgfqpoint{1.691178in}{0.880707in}}%
\pgfpathlineto{\pgfqpoint{1.653149in}{0.897964in}}%
\pgfpathclose%
\pgfusepath{fill}%
\end{pgfscope}%
\begin{pgfscope}%
\pgfpathrectangle{\pgfqpoint{0.150000in}{0.150000in}}{\pgfqpoint{2.700000in}{1.950000in}}%
\pgfusepath{clip}%
\pgfsetbuttcap%
\pgfsetroundjoin%
\definecolor{currentfill}{rgb}{0.975306,0.955193,0.956786}%
\pgfsetfillcolor{currentfill}%
\pgfsetlinewidth{0.000000pt}%
\definecolor{currentstroke}{rgb}{0.000000,0.000000,0.000000}%
\pgfsetstrokecolor{currentstroke}%
\pgfsetdash{}{0pt}%
\pgfpathmoveto{\pgfqpoint{1.575382in}{0.741206in}}%
\pgfpathlineto{\pgfqpoint{1.614416in}{0.799068in}}%
\pgfpathlineto{\pgfqpoint{1.575331in}{0.807140in}}%
\pgfpathlineto{\pgfqpoint{1.536486in}{0.737112in}}%
\pgfpathclose%
\pgfusepath{fill}%
\end{pgfscope}%
\begin{pgfscope}%
\pgfpathrectangle{\pgfqpoint{0.150000in}{0.150000in}}{\pgfqpoint{2.700000in}{1.950000in}}%
\pgfusepath{clip}%
\pgfsetbuttcap%
\pgfsetroundjoin%
\definecolor{currentfill}{rgb}{0.978232,0.980913,0.984666}%
\pgfsetfillcolor{currentfill}%
\pgfsetlinewidth{0.000000pt}%
\definecolor{currentstroke}{rgb}{0.000000,0.000000,0.000000}%
\pgfsetstrokecolor{currentstroke}%
\pgfsetdash{}{0pt}%
\pgfpathmoveto{\pgfqpoint{1.810064in}{0.869712in}}%
\pgfpathlineto{\pgfqpoint{1.846655in}{0.827586in}}%
\pgfpathlineto{\pgfqpoint{1.807375in}{0.847956in}}%
\pgfpathlineto{\pgfqpoint{1.769966in}{0.865042in}}%
\pgfpathclose%
\pgfusepath{fill}%
\end{pgfscope}%
\begin{pgfscope}%
\pgfpathrectangle{\pgfqpoint{0.150000in}{0.150000in}}{\pgfqpoint{2.700000in}{1.950000in}}%
\pgfusepath{clip}%
\pgfsetbuttcap%
\pgfsetroundjoin%
\definecolor{currentfill}{rgb}{0.948713,0.906939,0.910248}%
\pgfsetfillcolor{currentfill}%
\pgfsetlinewidth{0.000000pt}%
\definecolor{currentstroke}{rgb}{0.000000,0.000000,0.000000}%
\pgfsetstrokecolor{currentstroke}%
\pgfsetdash{}{0pt}%
\pgfpathmoveto{\pgfqpoint{1.575386in}{0.662802in}}%
\pgfpathlineto{\pgfqpoint{1.614520in}{0.732918in}}%
\pgfpathlineto{\pgfqpoint{1.575382in}{0.741206in}}%
\pgfpathlineto{\pgfqpoint{1.536486in}{0.671283in}}%
\pgfpathclose%
\pgfusepath{fill}%
\end{pgfscope}%
\begin{pgfscope}%
\pgfpathrectangle{\pgfqpoint{0.150000in}{0.150000in}}{\pgfqpoint{2.700000in}{1.950000in}}%
\pgfusepath{clip}%
\pgfsetbuttcap%
\pgfsetroundjoin%
\definecolor{currentfill}{rgb}{0.918321,0.851792,0.857062}%
\pgfsetfillcolor{currentfill}%
\pgfsetlinewidth{0.000000pt}%
\definecolor{currentstroke}{rgb}{0.000000,0.000000,0.000000}%
\pgfsetstrokecolor{currentstroke}%
\pgfsetdash{}{0pt}%
\pgfpathmoveto{\pgfqpoint{1.575438in}{0.596582in}}%
\pgfpathlineto{\pgfqpoint{1.614624in}{0.666592in}}%
\pgfpathlineto{\pgfqpoint{1.575386in}{0.662802in}}%
\pgfpathlineto{\pgfqpoint{1.536486in}{0.593106in}}%
\pgfpathclose%
\pgfusepath{fill}%
\end{pgfscope}%
\begin{pgfscope}%
\pgfpathrectangle{\pgfqpoint{0.150000in}{0.150000in}}{\pgfqpoint{2.700000in}{1.950000in}}%
\pgfusepath{clip}%
\pgfsetbuttcap%
\pgfsetroundjoin%
\definecolor{currentfill}{rgb}{0.978232,0.980913,0.984666}%
\pgfsetfillcolor{currentfill}%
\pgfsetlinewidth{0.000000pt}%
\definecolor{currentstroke}{rgb}{0.000000,0.000000,0.000000}%
\pgfsetstrokecolor{currentstroke}%
\pgfsetdash{}{0pt}%
\pgfpathmoveto{\pgfqpoint{1.614416in}{0.799068in}}%
\pgfpathlineto{\pgfqpoint{1.653734in}{0.869712in}}%
\pgfpathlineto{\pgfqpoint{1.614313in}{0.865042in}}%
\pgfpathlineto{\pgfqpoint{1.575331in}{0.807140in}}%
\pgfpathclose%
\pgfusepath{fill}%
\end{pgfscope}%
\begin{pgfscope}%
\pgfpathrectangle{\pgfqpoint{0.150000in}{0.150000in}}{\pgfqpoint{2.700000in}{1.950000in}}%
\pgfusepath{clip}%
\pgfsetbuttcap%
\pgfsetroundjoin%
\definecolor{currentfill}{rgb}{0.934697,0.942739,0.953998}%
\pgfsetfillcolor{currentfill}%
\pgfsetlinewidth{0.000000pt}%
\definecolor{currentstroke}{rgb}{0.000000,0.000000,0.000000}%
\pgfsetstrokecolor{currentstroke}%
\pgfsetdash{}{0pt}%
\pgfpathmoveto{\pgfqpoint{1.653734in}{0.869712in}}%
\pgfpathlineto{\pgfqpoint{1.692712in}{0.902774in}}%
\pgfpathlineto{\pgfqpoint{1.653149in}{0.897964in}}%
\pgfpathlineto{\pgfqpoint{1.614313in}{0.865042in}}%
\pgfpathclose%
\pgfusepath{fill}%
\end{pgfscope}%
\begin{pgfscope}%
\pgfpathrectangle{\pgfqpoint{0.150000in}{0.150000in}}{\pgfqpoint{2.700000in}{1.950000in}}%
\pgfusepath{clip}%
\pgfsetbuttcap%
\pgfsetroundjoin%
\definecolor{currentfill}{rgb}{0.934697,0.942739,0.953998}%
\pgfsetfillcolor{currentfill}%
\pgfsetlinewidth{0.000000pt}%
\definecolor{currentstroke}{rgb}{0.000000,0.000000,0.000000}%
\pgfsetstrokecolor{currentstroke}%
\pgfsetdash{}{0pt}%
\pgfpathmoveto{\pgfqpoint{1.732380in}{0.894957in}}%
\pgfpathlineto{\pgfqpoint{1.769966in}{0.865042in}}%
\pgfpathlineto{\pgfqpoint{1.730450in}{0.872899in}}%
\pgfpathlineto{\pgfqpoint{1.692712in}{0.902774in}}%
\pgfpathclose%
\pgfusepath{fill}%
\end{pgfscope}%
\begin{pgfscope}%
\pgfpathrectangle{\pgfqpoint{0.150000in}{0.150000in}}{\pgfqpoint{2.700000in}{1.950000in}}%
\pgfusepath{clip}%
\pgfsetbuttcap%
\pgfsetroundjoin%
\definecolor{currentfill}{rgb}{0.990502,0.982767,0.983379}%
\pgfsetfillcolor{currentfill}%
\pgfsetlinewidth{0.000000pt}%
\definecolor{currentstroke}{rgb}{0.000000,0.000000,0.000000}%
\pgfsetstrokecolor{currentstroke}%
\pgfsetdash{}{0pt}%
\pgfpathmoveto{\pgfqpoint{1.614520in}{0.732918in}}%
\pgfpathlineto{\pgfqpoint{1.653890in}{0.803457in}}%
\pgfpathlineto{\pgfqpoint{1.614416in}{0.799068in}}%
\pgfpathlineto{\pgfqpoint{1.575382in}{0.741206in}}%
\pgfpathclose%
\pgfusepath{fill}%
\end{pgfscope}%
\begin{pgfscope}%
\pgfpathrectangle{\pgfqpoint{0.150000in}{0.150000in}}{\pgfqpoint{2.700000in}{1.950000in}}%
\pgfusepath{clip}%
\pgfsetbuttcap%
\pgfsetroundjoin%
\definecolor{currentfill}{rgb}{0.963909,0.934513,0.936841}%
\pgfsetfillcolor{currentfill}%
\pgfsetlinewidth{0.000000pt}%
\definecolor{currentstroke}{rgb}{0.000000,0.000000,0.000000}%
\pgfsetstrokecolor{currentstroke}%
\pgfsetdash{}{0pt}%
\pgfpathmoveto{\pgfqpoint{1.614624in}{0.666592in}}%
\pgfpathlineto{\pgfqpoint{1.654047in}{0.737024in}}%
\pgfpathlineto{\pgfqpoint{1.614520in}{0.732918in}}%
\pgfpathlineto{\pgfqpoint{1.575386in}{0.662802in}}%
\pgfpathclose%
\pgfusepath{fill}%
\end{pgfscope}%
\begin{pgfscope}%
\pgfpathrectangle{\pgfqpoint{0.150000in}{0.150000in}}{\pgfqpoint{2.700000in}{1.950000in}}%
\pgfusepath{clip}%
\pgfsetbuttcap%
\pgfsetroundjoin%
\definecolor{currentfill}{rgb}{0.940916,0.948192,0.958379}%
\pgfsetfillcolor{currentfill}%
\pgfsetlinewidth{0.000000pt}%
\definecolor{currentstroke}{rgb}{0.000000,0.000000,0.000000}%
\pgfsetstrokecolor{currentstroke}%
\pgfsetdash{}{0pt}%
\pgfpathmoveto{\pgfqpoint{1.653890in}{0.803457in}}%
\pgfpathlineto{\pgfqpoint{1.693499in}{0.874423in}}%
\pgfpathlineto{\pgfqpoint{1.653734in}{0.869712in}}%
\pgfpathlineto{\pgfqpoint{1.614416in}{0.799068in}}%
\pgfpathclose%
\pgfusepath{fill}%
\end{pgfscope}%
\begin{pgfscope}%
\pgfpathrectangle{\pgfqpoint{0.150000in}{0.150000in}}{\pgfqpoint{2.700000in}{1.950000in}}%
\pgfusepath{clip}%
\pgfsetbuttcap%
\pgfsetroundjoin%
\definecolor{currentfill}{rgb}{0.903600,0.915472,0.932093}%
\pgfsetfillcolor{currentfill}%
\pgfsetlinewidth{0.000000pt}%
\definecolor{currentstroke}{rgb}{0.000000,0.000000,0.000000}%
\pgfsetstrokecolor{currentstroke}%
\pgfsetdash{}{0pt}%
\pgfpathmoveto{\pgfqpoint{1.772587in}{0.899790in}}%
\pgfpathlineto{\pgfqpoint{1.810064in}{0.869712in}}%
\pgfpathlineto{\pgfqpoint{1.769966in}{0.865042in}}%
\pgfpathlineto{\pgfqpoint{1.732380in}{0.894957in}}%
\pgfpathclose%
\pgfusepath{fill}%
\end{pgfscope}%
\begin{pgfscope}%
\pgfpathrectangle{\pgfqpoint{0.150000in}{0.150000in}}{\pgfqpoint{2.700000in}{1.950000in}}%
\pgfusepath{clip}%
\pgfsetbuttcap%
\pgfsetroundjoin%
\definecolor{currentfill}{rgb}{0.897381,0.910018,0.927711}%
\pgfsetfillcolor{currentfill}%
\pgfsetlinewidth{0.000000pt}%
\definecolor{currentstroke}{rgb}{0.000000,0.000000,0.000000}%
\pgfsetstrokecolor{currentstroke}%
\pgfsetdash{}{0pt}%
\pgfpathmoveto{\pgfqpoint{1.693499in}{0.874423in}}%
\pgfpathlineto{\pgfqpoint{1.732380in}{0.894957in}}%
\pgfpathlineto{\pgfqpoint{1.692712in}{0.902774in}}%
\pgfpathlineto{\pgfqpoint{1.653734in}{0.869712in}}%
\pgfpathclose%
\pgfusepath{fill}%
\end{pgfscope}%
\begin{pgfscope}%
\pgfpathrectangle{\pgfqpoint{0.150000in}{0.150000in}}{\pgfqpoint{2.700000in}{1.950000in}}%
\pgfusepath{clip}%
\pgfsetbuttcap%
\pgfsetroundjoin%
\definecolor{currentfill}{rgb}{0.984452,0.986366,0.989047}%
\pgfsetfillcolor{currentfill}%
\pgfsetlinewidth{0.000000pt}%
\definecolor{currentstroke}{rgb}{0.000000,0.000000,0.000000}%
\pgfsetstrokecolor{currentstroke}%
\pgfsetdash{}{0pt}%
\pgfpathmoveto{\pgfqpoint{1.654047in}{0.737024in}}%
\pgfpathlineto{\pgfqpoint{1.693516in}{0.795313in}}%
\pgfpathlineto{\pgfqpoint{1.653890in}{0.803457in}}%
\pgfpathlineto{\pgfqpoint{1.614520in}{0.732918in}}%
\pgfpathclose%
\pgfusepath{fill}%
\end{pgfscope}%
\begin{pgfscope}%
\pgfpathrectangle{\pgfqpoint{0.150000in}{0.150000in}}{\pgfqpoint{2.700000in}{1.950000in}}%
\pgfusepath{clip}%
\pgfsetbuttcap%
\pgfsetroundjoin%
\definecolor{currentfill}{rgb}{0.916039,0.926379,0.940855}%
\pgfsetfillcolor{currentfill}%
\pgfsetlinewidth{0.000000pt}%
\definecolor{currentstroke}{rgb}{0.000000,0.000000,0.000000}%
\pgfsetstrokecolor{currentstroke}%
\pgfsetdash{}{0pt}%
\pgfpathmoveto{\pgfqpoint{1.693516in}{0.795313in}}%
\pgfpathlineto{\pgfqpoint{1.733369in}{0.866478in}}%
\pgfpathlineto{\pgfqpoint{1.693499in}{0.874423in}}%
\pgfpathlineto{\pgfqpoint{1.653890in}{0.803457in}}%
\pgfpathclose%
\pgfusepath{fill}%
\end{pgfscope}%
\begin{pgfscope}%
\pgfpathrectangle{\pgfqpoint{0.150000in}{0.150000in}}{\pgfqpoint{2.700000in}{1.950000in}}%
\pgfusepath{clip}%
\pgfsetbuttcap%
\pgfsetroundjoin%
\definecolor{currentfill}{rgb}{0.872503,0.888205,0.910187}%
\pgfsetfillcolor{currentfill}%
\pgfsetlinewidth{0.000000pt}%
\definecolor{currentstroke}{rgb}{0.000000,0.000000,0.000000}%
\pgfsetstrokecolor{currentstroke}%
\pgfsetdash{}{0pt}%
\pgfpathmoveto{\pgfqpoint{1.733369in}{0.866478in}}%
\pgfpathlineto{\pgfqpoint{1.772587in}{0.899790in}}%
\pgfpathlineto{\pgfqpoint{1.732380in}{0.894957in}}%
\pgfpathlineto{\pgfqpoint{1.693499in}{0.874423in}}%
\pgfpathclose%
\pgfusepath{fill}%
\end{pgfscope}%
\end{pgfpicture}%
\makeatother%
\endgroup%
}
            \hfill
            \subbottom[\label{fig:parameterised-incompetent-games-f}]%
                {%% Creator: Matplotlib, PGF backend
%%
%% To include the figure in your LaTeX document, write
%%   \input{<filename>.pgf}
%%
%% Make sure the required packages are loaded in your preamble
%%   \usepackage{pgf}
%%
%% Figures using additional raster images can only be included by \input if
%% they are in the same directory as the main LaTeX file. For loading figures
%% from other directories you can use the `import` package
%%   \usepackage{import}
%% and then include the figures with
%%   \import{<path to file>}{<filename>.pgf}
%%
%% Matplotlib used the following preamble
%%   \usepackage{fontspec}
%%   \setmainfont{DejaVuSerif.ttf}[Path=C:/Users/Thomas/anaconda3/lib/site-packages/matplotlib/mpl-data/fonts/ttf/]
%%   \setsansfont{DejaVuSans.ttf}[Path=C:/Users/Thomas/anaconda3/lib/site-packages/matplotlib/mpl-data/fonts/ttf/]
%%   \setmonofont{DejaVuSansMono.ttf}[Path=C:/Users/Thomas/anaconda3/lib/site-packages/matplotlib/mpl-data/fonts/ttf/]
%%
\begingroup%
\makeatletter%
\begin{pgfpicture}%
\pgfpathrectangle{\pgfpointorigin}{\pgfqpoint{3.000000in}{2.250000in}}%
\pgfusepath{use as bounding box, clip}%
\begin{pgfscope}%
\pgfsetbuttcap%
\pgfsetmiterjoin%
\definecolor{currentfill}{rgb}{1.000000,1.000000,1.000000}%
\pgfsetfillcolor{currentfill}%
\pgfsetlinewidth{0.000000pt}%
\definecolor{currentstroke}{rgb}{1.000000,1.000000,1.000000}%
\pgfsetstrokecolor{currentstroke}%
\pgfsetdash{}{0pt}%
\pgfpathmoveto{\pgfqpoint{0.000000in}{0.000000in}}%
\pgfpathlineto{\pgfqpoint{3.000000in}{0.000000in}}%
\pgfpathlineto{\pgfqpoint{3.000000in}{2.250000in}}%
\pgfpathlineto{\pgfqpoint{0.000000in}{2.250000in}}%
\pgfpathclose%
\pgfusepath{fill}%
\end{pgfscope}%
\begin{pgfscope}%
\pgfsetbuttcap%
\pgfsetmiterjoin%
\definecolor{currentfill}{rgb}{1.000000,1.000000,1.000000}%
\pgfsetfillcolor{currentfill}%
\pgfsetlinewidth{0.000000pt}%
\definecolor{currentstroke}{rgb}{0.000000,0.000000,0.000000}%
\pgfsetstrokecolor{currentstroke}%
\pgfsetstrokeopacity{0.000000}%
\pgfsetdash{}{0pt}%
\pgfpathmoveto{\pgfqpoint{0.150000in}{0.150000in}}%
\pgfpathlineto{\pgfqpoint{2.850000in}{0.150000in}}%
\pgfpathlineto{\pgfqpoint{2.850000in}{2.100000in}}%
\pgfpathlineto{\pgfqpoint{0.150000in}{2.100000in}}%
\pgfpathclose%
\pgfusepath{fill}%
\end{pgfscope}%
\begin{pgfscope}%
\pgfsetbuttcap%
\pgfsetmiterjoin%
\definecolor{currentfill}{rgb}{0.950000,0.950000,0.950000}%
\pgfsetfillcolor{currentfill}%
\pgfsetfillopacity{0.500000}%
\pgfsetlinewidth{1.003750pt}%
\definecolor{currentstroke}{rgb}{0.950000,0.950000,0.950000}%
\pgfsetstrokecolor{currentstroke}%
\pgfsetstrokeopacity{0.500000}%
\pgfsetdash{}{0pt}%
\pgfpathmoveto{\pgfqpoint{2.573296in}{0.776948in}}%
\pgfpathlineto{\pgfqpoint{1.536486in}{1.299017in}}%
\pgfpathlineto{\pgfqpoint{1.536486in}{2.074448in}}%
\pgfpathlineto{\pgfqpoint{2.652584in}{1.554387in}}%
\pgfusepath{stroke,fill}%
\end{pgfscope}%
\begin{pgfscope}%
\pgfsetbuttcap%
\pgfsetmiterjoin%
\definecolor{currentfill}{rgb}{0.900000,0.900000,0.900000}%
\pgfsetfillcolor{currentfill}%
\pgfsetfillopacity{0.500000}%
\pgfsetlinewidth{1.003750pt}%
\definecolor{currentstroke}{rgb}{0.900000,0.900000,0.900000}%
\pgfsetstrokecolor{currentstroke}%
\pgfsetstrokeopacity{0.500000}%
\pgfsetdash{}{0pt}%
\pgfpathmoveto{\pgfqpoint{0.499677in}{0.776948in}}%
\pgfpathlineto{\pgfqpoint{1.536486in}{1.299017in}}%
\pgfpathlineto{\pgfqpoint{1.536486in}{2.074448in}}%
\pgfpathlineto{\pgfqpoint{0.420389in}{1.554387in}}%
\pgfusepath{stroke,fill}%
\end{pgfscope}%
\begin{pgfscope}%
\pgfsetbuttcap%
\pgfsetmiterjoin%
\definecolor{currentfill}{rgb}{0.925000,0.925000,0.925000}%
\pgfsetfillcolor{currentfill}%
\pgfsetfillopacity{0.500000}%
\pgfsetlinewidth{1.003750pt}%
\definecolor{currentstroke}{rgb}{0.925000,0.925000,0.925000}%
\pgfsetstrokecolor{currentstroke}%
\pgfsetstrokeopacity{0.500000}%
\pgfsetdash{}{0pt}%
\pgfpathmoveto{\pgfqpoint{1.536486in}{0.199655in}}%
\pgfpathlineto{\pgfqpoint{2.573296in}{0.776948in}}%
\pgfpathlineto{\pgfqpoint{1.536486in}{1.299017in}}%
\pgfpathlineto{\pgfqpoint{0.499677in}{0.776948in}}%
\pgfusepath{stroke,fill}%
\end{pgfscope}%
\begin{pgfscope}%
\pgfsetrectcap%
\pgfsetroundjoin%
\pgfsetlinewidth{0.803000pt}%
\definecolor{currentstroke}{rgb}{0.000000,0.000000,0.000000}%
\pgfsetstrokecolor{currentstroke}%
\pgfsetdash{}{0pt}%
\pgfpathmoveto{\pgfqpoint{2.573296in}{0.776948in}}%
\pgfpathlineto{\pgfqpoint{1.536486in}{0.199655in}}%
\pgfusepath{stroke}%
\end{pgfscope}%
\begin{pgfscope}%
\definecolor{textcolor}{rgb}{0.000000,0.000000,0.000000}%
\pgfsetstrokecolor{textcolor}%
\pgfsetfillcolor{textcolor}%
\pgftext[x=2.017747in,y=0.045475in,left,base,rotate=29.108966]{\color{textcolor}\sffamily\fontsize{8.000000}{9.600000}\selectfont Player 2 (\(\displaystyle \mu\))}%
\end{pgfscope}%
\begin{pgfscope}%
\pgfsetbuttcap%
\pgfsetroundjoin%
\pgfsetlinewidth{0.803000pt}%
\definecolor{currentstroke}{rgb}{0.690196,0.690196,0.690196}%
\pgfsetstrokecolor{currentstroke}%
\pgfsetdash{}{0pt}%
\pgfpathmoveto{\pgfqpoint{1.605722in}{0.238205in}}%
\pgfpathlineto{\pgfqpoint{0.568749in}{0.811728in}}%
\pgfpathlineto{\pgfqpoint{0.494997in}{1.589151in}}%
\pgfusepath{stroke}%
\end{pgfscope}%
\begin{pgfscope}%
\pgfsetbuttcap%
\pgfsetroundjoin%
\pgfsetlinewidth{0.803000pt}%
\definecolor{currentstroke}{rgb}{0.690196,0.690196,0.690196}%
\pgfsetstrokecolor{currentstroke}%
\pgfsetdash{}{0pt}%
\pgfpathmoveto{\pgfqpoint{1.793262in}{0.342627in}}%
\pgfpathlineto{\pgfqpoint{0.755965in}{0.905998in}}%
\pgfpathlineto{\pgfqpoint{0.697035in}{1.683294in}}%
\pgfusepath{stroke}%
\end{pgfscope}%
\begin{pgfscope}%
\pgfsetbuttcap%
\pgfsetroundjoin%
\pgfsetlinewidth{0.803000pt}%
\definecolor{currentstroke}{rgb}{0.690196,0.690196,0.690196}%
\pgfsetstrokecolor{currentstroke}%
\pgfsetdash{}{0pt}%
\pgfpathmoveto{\pgfqpoint{1.977414in}{0.445162in}}%
\pgfpathlineto{\pgfqpoint{0.939964in}{0.998647in}}%
\pgfpathlineto{\pgfqpoint{0.895342in}{1.775698in}}%
\pgfusepath{stroke}%
\end{pgfscope}%
\begin{pgfscope}%
\pgfsetbuttcap%
\pgfsetroundjoin%
\pgfsetlinewidth{0.803000pt}%
\definecolor{currentstroke}{rgb}{0.690196,0.690196,0.690196}%
\pgfsetstrokecolor{currentstroke}%
\pgfsetdash{}{0pt}%
\pgfpathmoveto{\pgfqpoint{2.158267in}{0.545861in}}%
\pgfpathlineto{\pgfqpoint{1.120829in}{1.089719in}}%
\pgfpathlineto{\pgfqpoint{1.090021in}{1.866411in}}%
\pgfusepath{stroke}%
\end{pgfscope}%
\begin{pgfscope}%
\pgfsetbuttcap%
\pgfsetroundjoin%
\pgfsetlinewidth{0.803000pt}%
\definecolor{currentstroke}{rgb}{0.690196,0.690196,0.690196}%
\pgfsetstrokecolor{currentstroke}%
\pgfsetdash{}{0pt}%
\pgfpathmoveto{\pgfqpoint{2.335912in}{0.644773in}}%
\pgfpathlineto{\pgfqpoint{1.298639in}{1.179253in}}%
\pgfpathlineto{\pgfqpoint{1.281170in}{1.955480in}}%
\pgfusepath{stroke}%
\end{pgfscope}%
\begin{pgfscope}%
\pgfsetbuttcap%
\pgfsetroundjoin%
\pgfsetlinewidth{0.803000pt}%
\definecolor{currentstroke}{rgb}{0.690196,0.690196,0.690196}%
\pgfsetstrokecolor{currentstroke}%
\pgfsetdash{}{0pt}%
\pgfpathmoveto{\pgfqpoint{2.510430in}{0.741945in}}%
\pgfpathlineto{\pgfqpoint{1.473472in}{1.267287in}}%
\pgfpathlineto{\pgfqpoint{1.468885in}{2.042948in}}%
\pgfusepath{stroke}%
\end{pgfscope}%
\begin{pgfscope}%
\pgfsetrectcap%
\pgfsetroundjoin%
\pgfsetlinewidth{0.803000pt}%
\definecolor{currentstroke}{rgb}{0.000000,0.000000,0.000000}%
\pgfsetstrokecolor{currentstroke}%
\pgfsetdash{}{0pt}%
\pgfpathmoveto{\pgfqpoint{1.596992in}{0.243033in}}%
\pgfpathlineto{\pgfqpoint{1.623203in}{0.228537in}}%
\pgfusepath{stroke}%
\end{pgfscope}%
\begin{pgfscope}%
\definecolor{textcolor}{rgb}{0.000000,0.000000,0.000000}%
\pgfsetstrokecolor{textcolor}%
\pgfsetfillcolor{textcolor}%
\pgftext[x=1.680378in,y=0.147403in,,top]{\color{textcolor}\sffamily\fontsize{6.000000}{7.200000}\selectfont \(\displaystyle 0.0\)}%
\end{pgfscope}%
\begin{pgfscope}%
\pgfsetrectcap%
\pgfsetroundjoin%
\pgfsetlinewidth{0.803000pt}%
\definecolor{currentstroke}{rgb}{0.000000,0.000000,0.000000}%
\pgfsetstrokecolor{currentstroke}%
\pgfsetdash{}{0pt}%
\pgfpathmoveto{\pgfqpoint{1.784534in}{0.347367in}}%
\pgfpathlineto{\pgfqpoint{1.810740in}{0.333134in}}%
\pgfusepath{stroke}%
\end{pgfscope}%
\begin{pgfscope}%
\definecolor{textcolor}{rgb}{0.000000,0.000000,0.000000}%
\pgfsetstrokecolor{textcolor}%
\pgfsetfillcolor{textcolor}%
\pgftext[x=1.866959in,y=0.252496in,,top]{\color{textcolor}\sffamily\fontsize{6.000000}{7.200000}\selectfont \(\displaystyle 0.2\)}%
\end{pgfscope}%
\begin{pgfscope}%
\pgfsetrectcap%
\pgfsetroundjoin%
\pgfsetlinewidth{0.803000pt}%
\definecolor{currentstroke}{rgb}{0.000000,0.000000,0.000000}%
\pgfsetstrokecolor{currentstroke}%
\pgfsetdash{}{0pt}%
\pgfpathmoveto{\pgfqpoint{1.968688in}{0.449817in}}%
\pgfpathlineto{\pgfqpoint{1.994886in}{0.435840in}}%
\pgfusepath{stroke}%
\end{pgfscope}%
\begin{pgfscope}%
\definecolor{textcolor}{rgb}{0.000000,0.000000,0.000000}%
\pgfsetstrokecolor{textcolor}%
\pgfsetfillcolor{textcolor}%
\pgftext[x=2.050175in,y=0.355693in,,top]{\color{textcolor}\sffamily\fontsize{6.000000}{7.200000}\selectfont \(\displaystyle 0.4\)}%
\end{pgfscope}%
\begin{pgfscope}%
\pgfsetrectcap%
\pgfsetroundjoin%
\pgfsetlinewidth{0.803000pt}%
\definecolor{currentstroke}{rgb}{0.000000,0.000000,0.000000}%
\pgfsetstrokecolor{currentstroke}%
\pgfsetdash{}{0pt}%
\pgfpathmoveto{\pgfqpoint{2.149546in}{0.550433in}}%
\pgfpathlineto{\pgfqpoint{2.175732in}{0.536706in}}%
\pgfusepath{stroke}%
\end{pgfscope}%
\begin{pgfscope}%
\definecolor{textcolor}{rgb}{0.000000,0.000000,0.000000}%
\pgfsetstrokecolor{textcolor}%
\pgfsetfillcolor{textcolor}%
\pgftext[x=2.230114in,y=0.457045in,,top]{\color{textcolor}\sffamily\fontsize{6.000000}{7.200000}\selectfont \(\displaystyle 0.6\)}%
\end{pgfscope}%
\begin{pgfscope}%
\pgfsetrectcap%
\pgfsetroundjoin%
\pgfsetlinewidth{0.803000pt}%
\definecolor{currentstroke}{rgb}{0.000000,0.000000,0.000000}%
\pgfsetstrokecolor{currentstroke}%
\pgfsetdash{}{0pt}%
\pgfpathmoveto{\pgfqpoint{2.327195in}{0.649264in}}%
\pgfpathlineto{\pgfqpoint{2.353366in}{0.635779in}}%
\pgfusepath{stroke}%
\end{pgfscope}%
\begin{pgfscope}%
\definecolor{textcolor}{rgb}{0.000000,0.000000,0.000000}%
\pgfsetstrokecolor{textcolor}%
\pgfsetfillcolor{textcolor}%
\pgftext[x=2.406864in,y=0.556601in,,top]{\color{textcolor}\sffamily\fontsize{6.000000}{7.200000}\selectfont \(\displaystyle 0.8\)}%
\end{pgfscope}%
\begin{pgfscope}%
\pgfsetrectcap%
\pgfsetroundjoin%
\pgfsetlinewidth{0.803000pt}%
\definecolor{currentstroke}{rgb}{0.000000,0.000000,0.000000}%
\pgfsetstrokecolor{currentstroke}%
\pgfsetdash{}{0pt}%
\pgfpathmoveto{\pgfqpoint{2.501720in}{0.746357in}}%
\pgfpathlineto{\pgfqpoint{2.527872in}{0.733108in}}%
\pgfusepath{stroke}%
\end{pgfscope}%
\begin{pgfscope}%
\definecolor{textcolor}{rgb}{0.000000,0.000000,0.000000}%
\pgfsetstrokecolor{textcolor}%
\pgfsetfillcolor{textcolor}%
\pgftext[x=2.580510in,y=0.654408in,,top]{\color{textcolor}\sffamily\fontsize{6.000000}{7.200000}\selectfont \(\displaystyle 1.0\)}%
\end{pgfscope}%
\begin{pgfscope}%
\pgfsetrectcap%
\pgfsetroundjoin%
\pgfsetlinewidth{0.803000pt}%
\definecolor{currentstroke}{rgb}{0.000000,0.000000,0.000000}%
\pgfsetstrokecolor{currentstroke}%
\pgfsetdash{}{0pt}%
\pgfpathmoveto{\pgfqpoint{0.499677in}{0.776948in}}%
\pgfpathlineto{\pgfqpoint{1.536486in}{0.199655in}}%
\pgfusepath{stroke}%
\end{pgfscope}%
\begin{pgfscope}%
\definecolor{textcolor}{rgb}{0.000000,0.000000,0.000000}%
\pgfsetstrokecolor{textcolor}%
\pgfsetfillcolor{textcolor}%
\pgftext[x=0.492803in,y=0.358631in,left,base,rotate=330.891034]{\color{textcolor}\sffamily\fontsize{8.000000}{9.600000}\selectfont Player 1 (\(\displaystyle \lambda\))}%
\end{pgfscope}%
\begin{pgfscope}%
\pgfsetbuttcap%
\pgfsetroundjoin%
\pgfsetlinewidth{0.803000pt}%
\definecolor{currentstroke}{rgb}{0.690196,0.690196,0.690196}%
\pgfsetstrokecolor{currentstroke}%
\pgfsetdash{}{0pt}%
\pgfpathmoveto{\pgfqpoint{2.577976in}{1.589151in}}%
\pgfpathlineto{\pgfqpoint{2.504223in}{0.811728in}}%
\pgfpathlineto{\pgfqpoint{1.467251in}{0.238205in}}%
\pgfusepath{stroke}%
\end{pgfscope}%
\begin{pgfscope}%
\pgfsetbuttcap%
\pgfsetroundjoin%
\pgfsetlinewidth{0.803000pt}%
\definecolor{currentstroke}{rgb}{0.690196,0.690196,0.690196}%
\pgfsetstrokecolor{currentstroke}%
\pgfsetdash{}{0pt}%
\pgfpathmoveto{\pgfqpoint{2.375938in}{1.683294in}}%
\pgfpathlineto{\pgfqpoint{2.317008in}{0.905998in}}%
\pgfpathlineto{\pgfqpoint{1.279711in}{0.342627in}}%
\pgfusepath{stroke}%
\end{pgfscope}%
\begin{pgfscope}%
\pgfsetbuttcap%
\pgfsetroundjoin%
\pgfsetlinewidth{0.803000pt}%
\definecolor{currentstroke}{rgb}{0.690196,0.690196,0.690196}%
\pgfsetstrokecolor{currentstroke}%
\pgfsetdash{}{0pt}%
\pgfpathmoveto{\pgfqpoint{2.177631in}{1.775698in}}%
\pgfpathlineto{\pgfqpoint{2.133009in}{0.998647in}}%
\pgfpathlineto{\pgfqpoint{1.095559in}{0.445162in}}%
\pgfusepath{stroke}%
\end{pgfscope}%
\begin{pgfscope}%
\pgfsetbuttcap%
\pgfsetroundjoin%
\pgfsetlinewidth{0.803000pt}%
\definecolor{currentstroke}{rgb}{0.690196,0.690196,0.690196}%
\pgfsetstrokecolor{currentstroke}%
\pgfsetdash{}{0pt}%
\pgfpathmoveto{\pgfqpoint{1.982952in}{1.866411in}}%
\pgfpathlineto{\pgfqpoint{1.952144in}{1.089719in}}%
\pgfpathlineto{\pgfqpoint{0.914705in}{0.545861in}}%
\pgfusepath{stroke}%
\end{pgfscope}%
\begin{pgfscope}%
\pgfsetbuttcap%
\pgfsetroundjoin%
\pgfsetlinewidth{0.803000pt}%
\definecolor{currentstroke}{rgb}{0.690196,0.690196,0.690196}%
\pgfsetstrokecolor{currentstroke}%
\pgfsetdash{}{0pt}%
\pgfpathmoveto{\pgfqpoint{1.791803in}{1.955480in}}%
\pgfpathlineto{\pgfqpoint{1.774334in}{1.179253in}}%
\pgfpathlineto{\pgfqpoint{0.737061in}{0.644773in}}%
\pgfusepath{stroke}%
\end{pgfscope}%
\begin{pgfscope}%
\pgfsetbuttcap%
\pgfsetroundjoin%
\pgfsetlinewidth{0.803000pt}%
\definecolor{currentstroke}{rgb}{0.690196,0.690196,0.690196}%
\pgfsetstrokecolor{currentstroke}%
\pgfsetdash{}{0pt}%
\pgfpathmoveto{\pgfqpoint{1.604088in}{2.042948in}}%
\pgfpathlineto{\pgfqpoint{1.599501in}{1.267287in}}%
\pgfpathlineto{\pgfqpoint{0.562543in}{0.741945in}}%
\pgfusepath{stroke}%
\end{pgfscope}%
\begin{pgfscope}%
\pgfsetrectcap%
\pgfsetroundjoin%
\pgfsetlinewidth{0.803000pt}%
\definecolor{currentstroke}{rgb}{0.000000,0.000000,0.000000}%
\pgfsetstrokecolor{currentstroke}%
\pgfsetdash{}{0pt}%
\pgfpathmoveto{\pgfqpoint{1.475981in}{0.243033in}}%
\pgfpathlineto{\pgfqpoint{1.449770in}{0.228537in}}%
\pgfusepath{stroke}%
\end{pgfscope}%
\begin{pgfscope}%
\definecolor{textcolor}{rgb}{0.000000,0.000000,0.000000}%
\pgfsetstrokecolor{textcolor}%
\pgfsetfillcolor{textcolor}%
\pgftext[x=1.392595in,y=0.147403in,,top]{\color{textcolor}\sffamily\fontsize{6.000000}{7.200000}\selectfont \(\displaystyle 0.0\)}%
\end{pgfscope}%
\begin{pgfscope}%
\pgfsetrectcap%
\pgfsetroundjoin%
\pgfsetlinewidth{0.803000pt}%
\definecolor{currentstroke}{rgb}{0.000000,0.000000,0.000000}%
\pgfsetstrokecolor{currentstroke}%
\pgfsetdash{}{0pt}%
\pgfpathmoveto{\pgfqpoint{1.288439in}{0.347367in}}%
\pgfpathlineto{\pgfqpoint{1.262233in}{0.333134in}}%
\pgfusepath{stroke}%
\end{pgfscope}%
\begin{pgfscope}%
\definecolor{textcolor}{rgb}{0.000000,0.000000,0.000000}%
\pgfsetstrokecolor{textcolor}%
\pgfsetfillcolor{textcolor}%
\pgftext[x=1.206013in,y=0.252496in,,top]{\color{textcolor}\sffamily\fontsize{6.000000}{7.200000}\selectfont \(\displaystyle 0.2\)}%
\end{pgfscope}%
\begin{pgfscope}%
\pgfsetrectcap%
\pgfsetroundjoin%
\pgfsetlinewidth{0.803000pt}%
\definecolor{currentstroke}{rgb}{0.000000,0.000000,0.000000}%
\pgfsetstrokecolor{currentstroke}%
\pgfsetdash{}{0pt}%
\pgfpathmoveto{\pgfqpoint{1.104285in}{0.449817in}}%
\pgfpathlineto{\pgfqpoint{1.078087in}{0.435840in}}%
\pgfusepath{stroke}%
\end{pgfscope}%
\begin{pgfscope}%
\definecolor{textcolor}{rgb}{0.000000,0.000000,0.000000}%
\pgfsetstrokecolor{textcolor}%
\pgfsetfillcolor{textcolor}%
\pgftext[x=1.022798in,y=0.355693in,,top]{\color{textcolor}\sffamily\fontsize{6.000000}{7.200000}\selectfont \(\displaystyle 0.4\)}%
\end{pgfscope}%
\begin{pgfscope}%
\pgfsetrectcap%
\pgfsetroundjoin%
\pgfsetlinewidth{0.803000pt}%
\definecolor{currentstroke}{rgb}{0.000000,0.000000,0.000000}%
\pgfsetstrokecolor{currentstroke}%
\pgfsetdash{}{0pt}%
\pgfpathmoveto{\pgfqpoint{0.923427in}{0.550433in}}%
\pgfpathlineto{\pgfqpoint{0.897241in}{0.536706in}}%
\pgfusepath{stroke}%
\end{pgfscope}%
\begin{pgfscope}%
\definecolor{textcolor}{rgb}{0.000000,0.000000,0.000000}%
\pgfsetstrokecolor{textcolor}%
\pgfsetfillcolor{textcolor}%
\pgftext[x=0.842859in,y=0.457045in,,top]{\color{textcolor}\sffamily\fontsize{6.000000}{7.200000}\selectfont \(\displaystyle 0.6\)}%
\end{pgfscope}%
\begin{pgfscope}%
\pgfsetrectcap%
\pgfsetroundjoin%
\pgfsetlinewidth{0.803000pt}%
\definecolor{currentstroke}{rgb}{0.000000,0.000000,0.000000}%
\pgfsetstrokecolor{currentstroke}%
\pgfsetdash{}{0pt}%
\pgfpathmoveto{\pgfqpoint{0.745778in}{0.649264in}}%
\pgfpathlineto{\pgfqpoint{0.719607in}{0.635779in}}%
\pgfusepath{stroke}%
\end{pgfscope}%
\begin{pgfscope}%
\definecolor{textcolor}{rgb}{0.000000,0.000000,0.000000}%
\pgfsetstrokecolor{textcolor}%
\pgfsetfillcolor{textcolor}%
\pgftext[x=0.666109in,y=0.556601in,,top]{\color{textcolor}\sffamily\fontsize{6.000000}{7.200000}\selectfont \(\displaystyle 0.8\)}%
\end{pgfscope}%
\begin{pgfscope}%
\pgfsetrectcap%
\pgfsetroundjoin%
\pgfsetlinewidth{0.803000pt}%
\definecolor{currentstroke}{rgb}{0.000000,0.000000,0.000000}%
\pgfsetstrokecolor{currentstroke}%
\pgfsetdash{}{0pt}%
\pgfpathmoveto{\pgfqpoint{0.571253in}{0.746357in}}%
\pgfpathlineto{\pgfqpoint{0.545101in}{0.733108in}}%
\pgfusepath{stroke}%
\end{pgfscope}%
\begin{pgfscope}%
\definecolor{textcolor}{rgb}{0.000000,0.000000,0.000000}%
\pgfsetstrokecolor{textcolor}%
\pgfsetfillcolor{textcolor}%
\pgftext[x=0.492463in,y=0.654408in,,top]{\color{textcolor}\sffamily\fontsize{6.000000}{7.200000}\selectfont \(\displaystyle 1.0\)}%
\end{pgfscope}%
\begin{pgfscope}%
\pgfsetrectcap%
\pgfsetroundjoin%
\pgfsetlinewidth{0.803000pt}%
\definecolor{currentstroke}{rgb}{0.000000,0.000000,0.000000}%
\pgfsetstrokecolor{currentstroke}%
\pgfsetdash{}{0pt}%
\pgfpathmoveto{\pgfqpoint{0.499677in}{0.776948in}}%
\pgfpathlineto{\pgfqpoint{0.420389in}{1.554387in}}%
\pgfusepath{stroke}%
\end{pgfscope}%
\begin{pgfscope}%
\definecolor{textcolor}{rgb}{0.000000,0.000000,0.000000}%
\pgfsetstrokecolor{textcolor}%
\pgfsetfillcolor{textcolor}%
\pgftext[x=0.041630in,y=1.401767in,left,base,rotate=275.823265]{\color{textcolor}\sffamily\fontsize{8.000000}{9.600000}\selectfont \(\displaystyle \mathsf{val}(G_{\lambda, \mu}\))}%
\end{pgfscope}%
\begin{pgfscope}%
\pgfsetbuttcap%
\pgfsetroundjoin%
\pgfsetlinewidth{0.803000pt}%
\definecolor{currentstroke}{rgb}{0.690196,0.690196,0.690196}%
\pgfsetstrokecolor{currentstroke}%
\pgfsetdash{}{0pt}%
\pgfpathmoveto{\pgfqpoint{0.493607in}{0.836469in}}%
\pgfpathlineto{\pgfqpoint{1.536486in}{1.358584in}}%
\pgfpathlineto{\pgfqpoint{2.579366in}{0.836469in}}%
\pgfusepath{stroke}%
\end{pgfscope}%
\begin{pgfscope}%
\pgfsetbuttcap%
\pgfsetroundjoin%
\pgfsetlinewidth{0.803000pt}%
\definecolor{currentstroke}{rgb}{0.690196,0.690196,0.690196}%
\pgfsetstrokecolor{currentstroke}%
\pgfsetdash{}{0pt}%
\pgfpathmoveto{\pgfqpoint{0.481939in}{0.950874in}}%
\pgfpathlineto{\pgfqpoint{1.536486in}{1.472986in}}%
\pgfpathlineto{\pgfqpoint{2.591034in}{0.950874in}}%
\pgfusepath{stroke}%
\end{pgfscope}%
\begin{pgfscope}%
\pgfsetbuttcap%
\pgfsetroundjoin%
\pgfsetlinewidth{0.803000pt}%
\definecolor{currentstroke}{rgb}{0.690196,0.690196,0.690196}%
\pgfsetstrokecolor{currentstroke}%
\pgfsetdash{}{0pt}%
\pgfpathmoveto{\pgfqpoint{0.470007in}{1.067868in}}%
\pgfpathlineto{\pgfqpoint{1.536486in}{1.589849in}}%
\pgfpathlineto{\pgfqpoint{2.602966in}{1.067868in}}%
\pgfusepath{stroke}%
\end{pgfscope}%
\begin{pgfscope}%
\pgfsetbuttcap%
\pgfsetroundjoin%
\pgfsetlinewidth{0.803000pt}%
\definecolor{currentstroke}{rgb}{0.690196,0.690196,0.690196}%
\pgfsetstrokecolor{currentstroke}%
\pgfsetdash{}{0pt}%
\pgfpathmoveto{\pgfqpoint{0.457802in}{1.187539in}}%
\pgfpathlineto{\pgfqpoint{1.536486in}{1.709254in}}%
\pgfpathlineto{\pgfqpoint{2.615171in}{1.187539in}}%
\pgfusepath{stroke}%
\end{pgfscope}%
\begin{pgfscope}%
\pgfsetbuttcap%
\pgfsetroundjoin%
\pgfsetlinewidth{0.803000pt}%
\definecolor{currentstroke}{rgb}{0.690196,0.690196,0.690196}%
\pgfsetstrokecolor{currentstroke}%
\pgfsetdash{}{0pt}%
\pgfpathmoveto{\pgfqpoint{0.445315in}{1.309982in}}%
\pgfpathlineto{\pgfqpoint{1.536486in}{1.831284in}}%
\pgfpathlineto{\pgfqpoint{2.627658in}{1.309982in}}%
\pgfusepath{stroke}%
\end{pgfscope}%
\begin{pgfscope}%
\pgfsetbuttcap%
\pgfsetroundjoin%
\pgfsetlinewidth{0.803000pt}%
\definecolor{currentstroke}{rgb}{0.690196,0.690196,0.690196}%
\pgfsetstrokecolor{currentstroke}%
\pgfsetdash{}{0pt}%
\pgfpathmoveto{\pgfqpoint{0.432535in}{1.435292in}}%
\pgfpathlineto{\pgfqpoint{1.536486in}{1.956028in}}%
\pgfpathlineto{\pgfqpoint{2.640438in}{1.435292in}}%
\pgfusepath{stroke}%
\end{pgfscope}%
\begin{pgfscope}%
\pgfsetrectcap%
\pgfsetroundjoin%
\pgfsetlinewidth{0.803000pt}%
\definecolor{currentstroke}{rgb}{0.000000,0.000000,0.000000}%
\pgfsetstrokecolor{currentstroke}%
\pgfsetdash{}{0pt}%
\pgfpathmoveto{\pgfqpoint{0.502368in}{0.840855in}}%
\pgfpathlineto{\pgfqpoint{0.476063in}{0.827686in}}%
\pgfusepath{stroke}%
\end{pgfscope}%
\begin{pgfscope}%
\definecolor{textcolor}{rgb}{0.000000,0.000000,0.000000}%
\pgfsetstrokecolor{textcolor}%
\pgfsetfillcolor{textcolor}%
\pgftext[x=0.350071in,y=0.836469in,,top]{\color{textcolor}\sffamily\fontsize{6.000000}{7.200000}\selectfont \(\displaystyle -0.6\)}%
\end{pgfscope}%
\begin{pgfscope}%
\pgfsetrectcap%
\pgfsetroundjoin%
\pgfsetlinewidth{0.803000pt}%
\definecolor{currentstroke}{rgb}{0.000000,0.000000,0.000000}%
\pgfsetstrokecolor{currentstroke}%
\pgfsetdash{}{0pt}%
\pgfpathmoveto{\pgfqpoint{0.490803in}{0.955262in}}%
\pgfpathlineto{\pgfqpoint{0.464190in}{0.942086in}}%
\pgfusepath{stroke}%
\end{pgfscope}%
\begin{pgfscope}%
\definecolor{textcolor}{rgb}{0.000000,0.000000,0.000000}%
\pgfsetstrokecolor{textcolor}%
\pgfsetfillcolor{textcolor}%
\pgftext[x=0.336797in,y=0.950874in,,top]{\color{textcolor}\sffamily\fontsize{6.000000}{7.200000}\selectfont \(\displaystyle -0.4\)}%
\end{pgfscope}%
\begin{pgfscope}%
\pgfsetrectcap%
\pgfsetroundjoin%
\pgfsetlinewidth{0.803000pt}%
\definecolor{currentstroke}{rgb}{0.000000,0.000000,0.000000}%
\pgfsetstrokecolor{currentstroke}%
\pgfsetdash{}{0pt}%
\pgfpathmoveto{\pgfqpoint{0.478976in}{1.072258in}}%
\pgfpathlineto{\pgfqpoint{0.452047in}{1.059077in}}%
\pgfusepath{stroke}%
\end{pgfscope}%
\begin{pgfscope}%
\definecolor{textcolor}{rgb}{0.000000,0.000000,0.000000}%
\pgfsetstrokecolor{textcolor}%
\pgfsetfillcolor{textcolor}%
\pgftext[x=0.323223in,y=1.067868in,,top]{\color{textcolor}\sffamily\fontsize{6.000000}{7.200000}\selectfont \(\displaystyle -0.2\)}%
\end{pgfscope}%
\begin{pgfscope}%
\pgfsetrectcap%
\pgfsetroundjoin%
\pgfsetlinewidth{0.803000pt}%
\definecolor{currentstroke}{rgb}{0.000000,0.000000,0.000000}%
\pgfsetstrokecolor{currentstroke}%
\pgfsetdash{}{0pt}%
\pgfpathmoveto{\pgfqpoint{0.466879in}{1.191929in}}%
\pgfpathlineto{\pgfqpoint{0.439626in}{1.178748in}}%
\pgfusepath{stroke}%
\end{pgfscope}%
\begin{pgfscope}%
\definecolor{textcolor}{rgb}{0.000000,0.000000,0.000000}%
\pgfsetstrokecolor{textcolor}%
\pgfsetfillcolor{textcolor}%
\pgftext[x=0.309338in,y=1.187539in,,top]{\color{textcolor}\sffamily\fontsize{6.000000}{7.200000}\selectfont \(\displaystyle 0.0\)}%
\end{pgfscope}%
\begin{pgfscope}%
\pgfsetrectcap%
\pgfsetroundjoin%
\pgfsetlinewidth{0.803000pt}%
\definecolor{currentstroke}{rgb}{0.000000,0.000000,0.000000}%
\pgfsetstrokecolor{currentstroke}%
\pgfsetdash{}{0pt}%
\pgfpathmoveto{\pgfqpoint{0.454502in}{1.314371in}}%
\pgfpathlineto{\pgfqpoint{0.426918in}{1.301193in}}%
\pgfusepath{stroke}%
\end{pgfscope}%
\begin{pgfscope}%
\definecolor{textcolor}{rgb}{0.000000,0.000000,0.000000}%
\pgfsetstrokecolor{textcolor}%
\pgfsetfillcolor{textcolor}%
\pgftext[x=0.295132in,y=1.309982in,,top]{\color{textcolor}\sffamily\fontsize{6.000000}{7.200000}\selectfont \(\displaystyle 0.2\)}%
\end{pgfscope}%
\begin{pgfscope}%
\pgfsetrectcap%
\pgfsetroundjoin%
\pgfsetlinewidth{0.803000pt}%
\definecolor{currentstroke}{rgb}{0.000000,0.000000,0.000000}%
\pgfsetstrokecolor{currentstroke}%
\pgfsetdash{}{0pt}%
\pgfpathmoveto{\pgfqpoint{0.441835in}{1.439679in}}%
\pgfpathlineto{\pgfqpoint{0.413911in}{1.426508in}}%
\pgfusepath{stroke}%
\end{pgfscope}%
\begin{pgfscope}%
\definecolor{textcolor}{rgb}{0.000000,0.000000,0.000000}%
\pgfsetstrokecolor{textcolor}%
\pgfsetfillcolor{textcolor}%
\pgftext[x=0.280593in,y=1.435292in,,top]{\color{textcolor}\sffamily\fontsize{6.000000}{7.200000}\selectfont \(\displaystyle 0.4\)}%
\end{pgfscope}%
\begin{pgfscope}%
\pgfpathrectangle{\pgfqpoint{0.150000in}{0.150000in}}{\pgfqpoint{2.700000in}{1.950000in}}%
\pgfusepath{clip}%
\pgfsetbuttcap%
\pgfsetroundjoin%
\definecolor{currentfill}{rgb}{0.614400,0.300322,0.325199}%
\pgfsetfillcolor{currentfill}%
\pgfsetlinewidth{0.000000pt}%
\definecolor{currentstroke}{rgb}{0.000000,0.000000,0.000000}%
\pgfsetstrokecolor{currentstroke}%
\pgfsetdash{}{0pt}%
\pgfpathmoveto{\pgfqpoint{2.335567in}{0.849928in}}%
\pgfpathlineto{\pgfqpoint{2.368130in}{0.840793in}}%
\pgfpathlineto{\pgfqpoint{2.331889in}{0.876661in}}%
\pgfpathlineto{\pgfqpoint{2.299165in}{0.885958in}}%
\pgfpathclose%
\pgfusepath{fill}%
\end{pgfscope}%
\begin{pgfscope}%
\pgfpathrectangle{\pgfqpoint{0.150000in}{0.150000in}}{\pgfqpoint{2.700000in}{1.950000in}}%
\pgfusepath{clip}%
\pgfsetbuttcap%
\pgfsetroundjoin%
\definecolor{currentfill}{rgb}{0.599203,0.272748,0.298606}%
\pgfsetfillcolor{currentfill}%
\pgfsetlinewidth{0.000000pt}%
\definecolor{currentstroke}{rgb}{0.000000,0.000000,0.000000}%
\pgfsetstrokecolor{currentstroke}%
\pgfsetdash{}{0pt}%
\pgfpathmoveto{\pgfqpoint{2.372437in}{0.819526in}}%
\pgfpathlineto{\pgfqpoint{2.404855in}{0.810521in}}%
\pgfpathlineto{\pgfqpoint{2.368130in}{0.840793in}}%
\pgfpathlineto{\pgfqpoint{2.335567in}{0.849928in}}%
\pgfpathclose%
\pgfusepath{fill}%
\end{pgfscope}%
\begin{pgfscope}%
\pgfpathrectangle{\pgfqpoint{0.150000in}{0.150000in}}{\pgfqpoint{2.700000in}{1.950000in}}%
\pgfusepath{clip}%
\pgfsetbuttcap%
\pgfsetroundjoin%
\definecolor{currentfill}{rgb}{0.656189,0.376149,0.398330}%
\pgfsetfillcolor{currentfill}%
\pgfsetlinewidth{0.000000pt}%
\definecolor{currentstroke}{rgb}{0.000000,0.000000,0.000000}%
\pgfsetstrokecolor{currentstroke}%
\pgfsetdash{}{0pt}%
\pgfpathmoveto{\pgfqpoint{2.299165in}{0.885958in}}%
\pgfpathlineto{\pgfqpoint{2.331889in}{0.876661in}}%
\pgfpathlineto{\pgfqpoint{2.299007in}{0.957995in}}%
\pgfpathlineto{\pgfqpoint{2.265588in}{0.961964in}}%
\pgfpathclose%
\pgfusepath{fill}%
\end{pgfscope}%
\begin{pgfscope}%
\pgfpathrectangle{\pgfqpoint{0.150000in}{0.150000in}}{\pgfqpoint{2.700000in}{1.950000in}}%
\pgfusepath{clip}%
\pgfsetbuttcap%
\pgfsetroundjoin%
\definecolor{currentfill}{rgb}{0.603002,0.279642,0.305254}%
\pgfsetfillcolor{currentfill}%
\pgfsetlinewidth{0.000000pt}%
\definecolor{currentstroke}{rgb}{0.000000,0.000000,0.000000}%
\pgfsetstrokecolor{currentstroke}%
\pgfsetdash{}{0pt}%
\pgfpathmoveto{\pgfqpoint{2.411774in}{0.817288in}}%
\pgfpathlineto{\pgfqpoint{2.444626in}{0.813891in}}%
\pgfpathlineto{\pgfqpoint{2.404855in}{0.810521in}}%
\pgfpathlineto{\pgfqpoint{2.372437in}{0.819526in}}%
\pgfpathclose%
\pgfusepath{fill}%
\end{pgfscope}%
\begin{pgfscope}%
\pgfpathrectangle{\pgfqpoint{0.150000in}{0.150000in}}{\pgfqpoint{2.700000in}{1.950000in}}%
\pgfusepath{clip}%
\pgfsetbuttcap%
\pgfsetroundjoin%
\definecolor{currentfill}{rgb}{0.866284,0.882751,0.905806}%
\pgfsetfillcolor{currentfill}%
\pgfsetlinewidth{0.000000pt}%
\definecolor{currentstroke}{rgb}{0.000000,0.000000,0.000000}%
\pgfsetstrokecolor{currentstroke}%
\pgfsetdash{}{0pt}%
\pgfpathmoveto{\pgfqpoint{1.536486in}{1.695794in}}%
\pgfpathlineto{\pgfqpoint{1.572802in}{1.701147in}}%
\pgfpathlineto{\pgfqpoint{1.536486in}{1.718695in}}%
\pgfpathlineto{\pgfqpoint{1.500131in}{1.713401in}}%
\pgfpathclose%
\pgfusepath{fill}%
\end{pgfscope}%
\begin{pgfscope}%
\pgfpathrectangle{\pgfqpoint{0.150000in}{0.150000in}}{\pgfqpoint{2.700000in}{1.950000in}}%
\pgfusepath{clip}%
\pgfsetbuttcap%
\pgfsetroundjoin%
\definecolor{currentfill}{rgb}{0.644792,0.355469,0.378385}%
\pgfsetfillcolor{currentfill}%
\pgfsetlinewidth{0.000000pt}%
\definecolor{currentstroke}{rgb}{0.000000,0.000000,0.000000}%
\pgfsetstrokecolor{currentstroke}%
\pgfsetdash{}{0pt}%
\pgfpathmoveto{\pgfqpoint{2.302708in}{0.859147in}}%
\pgfpathlineto{\pgfqpoint{2.335567in}{0.849928in}}%
\pgfpathlineto{\pgfqpoint{2.299165in}{0.885958in}}%
\pgfpathlineto{\pgfqpoint{2.265740in}{0.889641in}}%
\pgfpathclose%
\pgfusepath{fill}%
\end{pgfscope}%
\begin{pgfscope}%
\pgfpathrectangle{\pgfqpoint{0.150000in}{0.150000in}}{\pgfqpoint{2.700000in}{1.950000in}}%
\pgfusepath{clip}%
\pgfsetbuttcap%
\pgfsetroundjoin%
\definecolor{currentfill}{rgb}{0.866284,0.882751,0.905806}%
\pgfsetfillcolor{currentfill}%
\pgfsetlinewidth{0.000000pt}%
\definecolor{currentstroke}{rgb}{0.000000,0.000000,0.000000}%
\pgfsetstrokecolor{currentstroke}%
\pgfsetdash{}{0pt}%
\pgfpathmoveto{\pgfqpoint{1.572970in}{1.678124in}}%
\pgfpathlineto{\pgfqpoint{1.609245in}{1.683538in}}%
\pgfpathlineto{\pgfqpoint{1.572802in}{1.701147in}}%
\pgfpathlineto{\pgfqpoint{1.536486in}{1.695794in}}%
\pgfpathclose%
\pgfusepath{fill}%
\end{pgfscope}%
\begin{pgfscope}%
\pgfpathrectangle{\pgfqpoint{0.150000in}{0.150000in}}{\pgfqpoint{2.700000in}{1.950000in}}%
\pgfusepath{clip}%
\pgfsetbuttcap%
\pgfsetroundjoin%
\definecolor{currentfill}{rgb}{0.629596,0.327895,0.351792}%
\pgfsetfillcolor{currentfill}%
\pgfsetlinewidth{0.000000pt}%
\definecolor{currentstroke}{rgb}{0.000000,0.000000,0.000000}%
\pgfsetstrokecolor{currentstroke}%
\pgfsetdash{}{0pt}%
\pgfpathmoveto{\pgfqpoint{2.339724in}{0.828613in}}%
\pgfpathlineto{\pgfqpoint{2.372437in}{0.819526in}}%
\pgfpathlineto{\pgfqpoint{2.335567in}{0.849928in}}%
\pgfpathlineto{\pgfqpoint{2.302708in}{0.859147in}}%
\pgfpathclose%
\pgfusepath{fill}%
\end{pgfscope}%
\begin{pgfscope}%
\pgfpathrectangle{\pgfqpoint{0.150000in}{0.150000in}}{\pgfqpoint{2.700000in}{1.950000in}}%
\pgfusepath{clip}%
\pgfsetbuttcap%
\pgfsetroundjoin%
\definecolor{currentfill}{rgb}{0.847626,0.866391,0.892662}%
\pgfsetfillcolor{currentfill}%
\pgfsetlinewidth{0.000000pt}%
\definecolor{currentstroke}{rgb}{0.000000,0.000000,0.000000}%
\pgfsetstrokecolor{currentstroke}%
\pgfsetdash{}{0pt}%
\pgfpathmoveto{\pgfqpoint{1.499962in}{1.690409in}}%
\pgfpathlineto{\pgfqpoint{1.536486in}{1.695794in}}%
\pgfpathlineto{\pgfqpoint{1.500131in}{1.713401in}}%
\pgfpathlineto{\pgfqpoint{1.463567in}{1.708076in}}%
\pgfpathclose%
\pgfusepath{fill}%
\end{pgfscope}%
\begin{pgfscope}%
\pgfpathrectangle{\pgfqpoint{0.150000in}{0.150000in}}{\pgfqpoint{2.700000in}{1.950000in}}%
\pgfusepath{clip}%
\pgfsetbuttcap%
\pgfsetroundjoin%
\definecolor{currentfill}{rgb}{0.872503,0.888205,0.910187}%
\pgfsetfillcolor{currentfill}%
\pgfsetlinewidth{0.000000pt}%
\definecolor{currentstroke}{rgb}{0.000000,0.000000,0.000000}%
\pgfsetstrokecolor{currentstroke}%
\pgfsetdash{}{0pt}%
\pgfpathmoveto{\pgfqpoint{1.609543in}{1.654259in}}%
\pgfpathlineto{\pgfqpoint{1.645817in}{1.665866in}}%
\pgfpathlineto{\pgfqpoint{1.609245in}{1.683538in}}%
\pgfpathlineto{\pgfqpoint{1.572970in}{1.678124in}}%
\pgfpathclose%
\pgfusepath{fill}%
\end{pgfscope}%
\begin{pgfscope}%
\pgfpathrectangle{\pgfqpoint{0.150000in}{0.150000in}}{\pgfqpoint{2.700000in}{1.950000in}}%
\pgfusepath{clip}%
\pgfsetbuttcap%
\pgfsetroundjoin%
\definecolor{currentfill}{rgb}{0.682782,0.424403,0.444868}%
\pgfsetfillcolor{currentfill}%
\pgfsetlinewidth{0.000000pt}%
\definecolor{currentstroke}{rgb}{0.000000,0.000000,0.000000}%
\pgfsetstrokecolor{currentstroke}%
\pgfsetdash{}{0pt}%
\pgfpathmoveto{\pgfqpoint{2.265740in}{0.889641in}}%
\pgfpathlineto{\pgfqpoint{2.299165in}{0.885958in}}%
\pgfpathlineto{\pgfqpoint{2.265588in}{0.961964in}}%
\pgfpathlineto{\pgfqpoint{2.231902in}{0.965965in}}%
\pgfpathclose%
\pgfusepath{fill}%
\end{pgfscope}%
\begin{pgfscope}%
\pgfpathrectangle{\pgfqpoint{0.150000in}{0.150000in}}{\pgfqpoint{2.700000in}{1.950000in}}%
\pgfusepath{clip}%
\pgfsetbuttcap%
\pgfsetroundjoin%
\definecolor{currentfill}{rgb}{0.724571,0.500230,0.517999}%
\pgfsetfillcolor{currentfill}%
\pgfsetlinewidth{0.000000pt}%
\definecolor{currentstroke}{rgb}{0.000000,0.000000,0.000000}%
\pgfsetstrokecolor{currentstroke}%
\pgfsetdash{}{0pt}%
\pgfpathmoveto{\pgfqpoint{2.265588in}{0.961964in}}%
\pgfpathlineto{\pgfqpoint{2.299007in}{0.957995in}}%
\pgfpathlineto{\pgfqpoint{2.266244in}{1.045816in}}%
\pgfpathlineto{\pgfqpoint{2.232143in}{1.044347in}}%
\pgfpathclose%
\pgfusepath{fill}%
\end{pgfscope}%
\begin{pgfscope}%
\pgfpathrectangle{\pgfqpoint{0.150000in}{0.150000in}}{\pgfqpoint{2.700000in}{1.950000in}}%
\pgfusepath{clip}%
\pgfsetbuttcap%
\pgfsetroundjoin%
\definecolor{currentfill}{rgb}{0.629596,0.327895,0.351792}%
\pgfsetfillcolor{currentfill}%
\pgfsetlinewidth{0.000000pt}%
\definecolor{currentstroke}{rgb}{0.000000,0.000000,0.000000}%
\pgfsetstrokecolor{currentstroke}%
\pgfsetdash{}{0pt}%
\pgfpathmoveto{\pgfqpoint{2.378659in}{0.820711in}}%
\pgfpathlineto{\pgfqpoint{2.411774in}{0.817288in}}%
\pgfpathlineto{\pgfqpoint{2.372437in}{0.819526in}}%
\pgfpathlineto{\pgfqpoint{2.339724in}{0.828613in}}%
\pgfpathclose%
\pgfusepath{fill}%
\end{pgfscope}%
\begin{pgfscope}%
\pgfpathrectangle{\pgfqpoint{0.150000in}{0.150000in}}{\pgfqpoint{2.700000in}{1.950000in}}%
\pgfusepath{clip}%
\pgfsetbuttcap%
\pgfsetroundjoin%
\definecolor{currentfill}{rgb}{0.847626,0.866391,0.892662}%
\pgfsetfillcolor{currentfill}%
\pgfsetlinewidth{0.000000pt}%
\definecolor{currentstroke}{rgb}{0.000000,0.000000,0.000000}%
\pgfsetstrokecolor{currentstroke}%
\pgfsetdash{}{0pt}%
\pgfpathmoveto{\pgfqpoint{1.536486in}{1.666532in}}%
\pgfpathlineto{\pgfqpoint{1.572970in}{1.678124in}}%
\pgfpathlineto{\pgfqpoint{1.536486in}{1.695794in}}%
\pgfpathlineto{\pgfqpoint{1.499962in}{1.690409in}}%
\pgfpathclose%
\pgfusepath{fill}%
\end{pgfscope}%
\begin{pgfscope}%
\pgfpathrectangle{\pgfqpoint{0.150000in}{0.150000in}}{\pgfqpoint{2.700000in}{1.950000in}}%
\pgfusepath{clip}%
\pgfsetbuttcap%
\pgfsetroundjoin%
\definecolor{currentfill}{rgb}{0.872503,0.888205,0.910187}%
\pgfsetfillcolor{currentfill}%
\pgfsetlinewidth{0.000000pt}%
\definecolor{currentstroke}{rgb}{0.000000,0.000000,0.000000}%
\pgfsetstrokecolor{currentstroke}%
\pgfsetdash{}{0pt}%
\pgfpathmoveto{\pgfqpoint{1.646265in}{1.636463in}}%
\pgfpathlineto{\pgfqpoint{1.682518in}{1.648132in}}%
\pgfpathlineto{\pgfqpoint{1.645817in}{1.665866in}}%
\pgfpathlineto{\pgfqpoint{1.609543in}{1.654259in}}%
\pgfpathclose%
\pgfusepath{fill}%
\end{pgfscope}%
\begin{pgfscope}%
\pgfpathrectangle{\pgfqpoint{0.150000in}{0.150000in}}{\pgfqpoint{2.700000in}{1.950000in}}%
\pgfusepath{clip}%
\pgfsetbuttcap%
\pgfsetroundjoin%
\definecolor{currentfill}{rgb}{0.828968,0.850031,0.879519}%
\pgfsetfillcolor{currentfill}%
\pgfsetlinewidth{0.000000pt}%
\definecolor{currentstroke}{rgb}{0.000000,0.000000,0.000000}%
\pgfsetstrokecolor{currentstroke}%
\pgfsetdash{}{0pt}%
\pgfpathmoveto{\pgfqpoint{1.463227in}{1.684994in}}%
\pgfpathlineto{\pgfqpoint{1.499962in}{1.690409in}}%
\pgfpathlineto{\pgfqpoint{1.463567in}{1.708076in}}%
\pgfpathlineto{\pgfqpoint{1.426792in}{1.702721in}}%
\pgfpathclose%
\pgfusepath{fill}%
\end{pgfscope}%
\begin{pgfscope}%
\pgfpathrectangle{\pgfqpoint{0.150000in}{0.150000in}}{\pgfqpoint{2.700000in}{1.950000in}}%
\pgfusepath{clip}%
\pgfsetbuttcap%
\pgfsetroundjoin%
\definecolor{currentfill}{rgb}{0.671385,0.403722,0.424923}%
\pgfsetfillcolor{currentfill}%
\pgfsetlinewidth{0.000000pt}%
\definecolor{currentstroke}{rgb}{0.000000,0.000000,0.000000}%
\pgfsetstrokecolor{currentstroke}%
\pgfsetdash{}{0pt}%
\pgfpathmoveto{\pgfqpoint{2.269141in}{0.862740in}}%
\pgfpathlineto{\pgfqpoint{2.302708in}{0.859147in}}%
\pgfpathlineto{\pgfqpoint{2.265740in}{0.889641in}}%
\pgfpathlineto{\pgfqpoint{2.232435in}{0.899077in}}%
\pgfpathclose%
\pgfusepath{fill}%
\end{pgfscope}%
\begin{pgfscope}%
\pgfpathrectangle{\pgfqpoint{0.150000in}{0.150000in}}{\pgfqpoint{2.700000in}{1.950000in}}%
\pgfusepath{clip}%
\pgfsetbuttcap%
\pgfsetroundjoin%
\definecolor{currentfill}{rgb}{0.853845,0.871844,0.897044}%
\pgfsetfillcolor{currentfill}%
\pgfsetlinewidth{0.000000pt}%
\definecolor{currentstroke}{rgb}{0.000000,0.000000,0.000000}%
\pgfsetstrokecolor{currentstroke}%
\pgfsetdash{}{0pt}%
\pgfpathmoveto{\pgfqpoint{1.573120in}{1.648738in}}%
\pgfpathlineto{\pgfqpoint{1.609543in}{1.654259in}}%
\pgfpathlineto{\pgfqpoint{1.572970in}{1.678124in}}%
\pgfpathlineto{\pgfqpoint{1.536486in}{1.666532in}}%
\pgfpathclose%
\pgfusepath{fill}%
\end{pgfscope}%
\begin{pgfscope}%
\pgfpathrectangle{\pgfqpoint{0.150000in}{0.150000in}}{\pgfqpoint{2.700000in}{1.950000in}}%
\pgfusepath{clip}%
\pgfsetbuttcap%
\pgfsetroundjoin%
\definecolor{currentfill}{rgb}{0.659988,0.383042,0.404979}%
\pgfsetfillcolor{currentfill}%
\pgfsetlinewidth{0.000000pt}%
\definecolor{currentstroke}{rgb}{0.000000,0.000000,0.000000}%
\pgfsetstrokecolor{currentstroke}%
\pgfsetdash{}{0pt}%
\pgfpathmoveto{\pgfqpoint{2.306712in}{0.837782in}}%
\pgfpathlineto{\pgfqpoint{2.339724in}{0.828613in}}%
\pgfpathlineto{\pgfqpoint{2.302708in}{0.859147in}}%
\pgfpathlineto{\pgfqpoint{2.269141in}{0.862740in}}%
\pgfpathclose%
\pgfusepath{fill}%
\end{pgfscope}%
\begin{pgfscope}%
\pgfpathrectangle{\pgfqpoint{0.150000in}{0.150000in}}{\pgfqpoint{2.700000in}{1.950000in}}%
\pgfusepath{clip}%
\pgfsetbuttcap%
\pgfsetroundjoin%
\definecolor{currentfill}{rgb}{0.872503,0.888205,0.910187}%
\pgfsetfillcolor{currentfill}%
\pgfsetlinewidth{0.000000pt}%
\definecolor{currentstroke}{rgb}{0.000000,0.000000,0.000000}%
\pgfsetstrokecolor{currentstroke}%
\pgfsetdash{}{0pt}%
\pgfpathmoveto{\pgfqpoint{1.683118in}{1.618604in}}%
\pgfpathlineto{\pgfqpoint{1.719248in}{1.624215in}}%
\pgfpathlineto{\pgfqpoint{1.682518in}{1.648132in}}%
\pgfpathlineto{\pgfqpoint{1.646265in}{1.636463in}}%
\pgfpathclose%
\pgfusepath{fill}%
\end{pgfscope}%
\begin{pgfscope}%
\pgfpathrectangle{\pgfqpoint{0.150000in}{0.150000in}}{\pgfqpoint{2.700000in}{1.950000in}}%
\pgfusepath{clip}%
\pgfsetbuttcap%
\pgfsetroundjoin%
\definecolor{currentfill}{rgb}{0.828968,0.850031,0.879519}%
\pgfsetfillcolor{currentfill}%
\pgfsetlinewidth{0.000000pt}%
\definecolor{currentstroke}{rgb}{0.000000,0.000000,0.000000}%
\pgfsetstrokecolor{currentstroke}%
\pgfsetdash{}{0pt}%
\pgfpathmoveto{\pgfqpoint{1.499812in}{1.661041in}}%
\pgfpathlineto{\pgfqpoint{1.536486in}{1.666532in}}%
\pgfpathlineto{\pgfqpoint{1.499962in}{1.690409in}}%
\pgfpathlineto{\pgfqpoint{1.463227in}{1.684994in}}%
\pgfpathclose%
\pgfusepath{fill}%
\end{pgfscope}%
\begin{pgfscope}%
\pgfpathrectangle{\pgfqpoint{0.150000in}{0.150000in}}{\pgfqpoint{2.700000in}{1.950000in}}%
\pgfusepath{clip}%
\pgfsetbuttcap%
\pgfsetroundjoin%
\definecolor{currentfill}{rgb}{0.747365,0.541590,0.557889}%
\pgfsetfillcolor{currentfill}%
\pgfsetlinewidth{0.000000pt}%
\definecolor{currentstroke}{rgb}{0.000000,0.000000,0.000000}%
\pgfsetstrokecolor{currentstroke}%
\pgfsetdash{}{0pt}%
\pgfpathmoveto{\pgfqpoint{2.231902in}{0.965965in}}%
\pgfpathlineto{\pgfqpoint{2.265588in}{0.961964in}}%
\pgfpathlineto{\pgfqpoint{2.232143in}{1.044347in}}%
\pgfpathlineto{\pgfqpoint{2.197807in}{1.042869in}}%
\pgfpathclose%
\pgfusepath{fill}%
\end{pgfscope}%
\begin{pgfscope}%
\pgfpathrectangle{\pgfqpoint{0.150000in}{0.150000in}}{\pgfqpoint{2.700000in}{1.950000in}}%
\pgfusepath{clip}%
\pgfsetbuttcap%
\pgfsetroundjoin%
\definecolor{currentfill}{rgb}{0.709375,0.472656,0.491406}%
\pgfsetfillcolor{currentfill}%
\pgfsetlinewidth{0.000000pt}%
\definecolor{currentstroke}{rgb}{0.000000,0.000000,0.000000}%
\pgfsetstrokecolor{currentstroke}%
\pgfsetdash{}{0pt}%
\pgfpathmoveto{\pgfqpoint{2.232435in}{0.899077in}}%
\pgfpathlineto{\pgfqpoint{2.265740in}{0.889641in}}%
\pgfpathlineto{\pgfqpoint{2.231902in}{0.965965in}}%
\pgfpathlineto{\pgfqpoint{2.197946in}{0.969998in}}%
\pgfpathclose%
\pgfusepath{fill}%
\end{pgfscope}%
\begin{pgfscope}%
\pgfpathrectangle{\pgfqpoint{0.150000in}{0.150000in}}{\pgfqpoint{2.700000in}{1.950000in}}%
\pgfusepath{clip}%
\pgfsetbuttcap%
\pgfsetroundjoin%
\definecolor{currentfill}{rgb}{0.860064,0.877298,0.901425}%
\pgfsetfillcolor{currentfill}%
\pgfsetlinewidth{0.000000pt}%
\definecolor{currentstroke}{rgb}{0.000000,0.000000,0.000000}%
\pgfsetstrokecolor{currentstroke}%
\pgfsetdash{}{0pt}%
\pgfpathmoveto{\pgfqpoint{1.609884in}{1.630881in}}%
\pgfpathlineto{\pgfqpoint{1.646265in}{1.636463in}}%
\pgfpathlineto{\pgfqpoint{1.609543in}{1.654259in}}%
\pgfpathlineto{\pgfqpoint{1.573120in}{1.648738in}}%
\pgfpathclose%
\pgfusepath{fill}%
\end{pgfscope}%
\begin{pgfscope}%
\pgfpathrectangle{\pgfqpoint{0.150000in}{0.150000in}}{\pgfqpoint{2.700000in}{1.950000in}}%
\pgfusepath{clip}%
\pgfsetbuttcap%
\pgfsetroundjoin%
\definecolor{currentfill}{rgb}{0.878722,0.893658,0.914568}%
\pgfsetfillcolor{currentfill}%
\pgfsetlinewidth{0.000000pt}%
\definecolor{currentstroke}{rgb}{0.000000,0.000000,0.000000}%
\pgfsetstrokecolor{currentstroke}%
\pgfsetdash{}{0pt}%
\pgfpathmoveto{\pgfqpoint{1.719999in}{1.594552in}}%
\pgfpathlineto{\pgfqpoint{1.756189in}{1.606354in}}%
\pgfpathlineto{\pgfqpoint{1.719248in}{1.624215in}}%
\pgfpathlineto{\pgfqpoint{1.683118in}{1.618604in}}%
\pgfpathclose%
\pgfusepath{fill}%
\end{pgfscope}%
\begin{pgfscope}%
\pgfpathrectangle{\pgfqpoint{0.150000in}{0.150000in}}{\pgfqpoint{2.700000in}{1.950000in}}%
\pgfusepath{clip}%
\pgfsetbuttcap%
\pgfsetroundjoin%
\definecolor{currentfill}{rgb}{0.796752,0.631204,0.644317}%
\pgfsetfillcolor{currentfill}%
\pgfsetlinewidth{0.000000pt}%
\definecolor{currentstroke}{rgb}{0.000000,0.000000,0.000000}%
\pgfsetstrokecolor{currentstroke}%
\pgfsetdash{}{0pt}%
\pgfpathmoveto{\pgfqpoint{2.232143in}{1.044347in}}%
\pgfpathlineto{\pgfqpoint{2.266244in}{1.045816in}}%
\pgfpathlineto{\pgfqpoint{2.233160in}{1.134499in}}%
\pgfpathlineto{\pgfqpoint{2.198407in}{1.127450in}}%
\pgfpathclose%
\pgfusepath{fill}%
\end{pgfscope}%
\begin{pgfscope}%
\pgfpathrectangle{\pgfqpoint{0.150000in}{0.150000in}}{\pgfqpoint{2.700000in}{1.950000in}}%
\pgfusepath{clip}%
\pgfsetbuttcap%
\pgfsetroundjoin%
\definecolor{currentfill}{rgb}{0.659988,0.383042,0.404979}%
\pgfsetfillcolor{currentfill}%
\pgfsetlinewidth{0.000000pt}%
\definecolor{currentstroke}{rgb}{0.000000,0.000000,0.000000}%
\pgfsetstrokecolor{currentstroke}%
\pgfsetdash{}{0pt}%
\pgfpathmoveto{\pgfqpoint{2.345729in}{0.829871in}}%
\pgfpathlineto{\pgfqpoint{2.378659in}{0.820711in}}%
\pgfpathlineto{\pgfqpoint{2.339724in}{0.828613in}}%
\pgfpathlineto{\pgfqpoint{2.306712in}{0.837782in}}%
\pgfpathclose%
\pgfusepath{fill}%
\end{pgfscope}%
\begin{pgfscope}%
\pgfpathrectangle{\pgfqpoint{0.150000in}{0.150000in}}{\pgfqpoint{2.700000in}{1.950000in}}%
\pgfusepath{clip}%
\pgfsetbuttcap%
\pgfsetroundjoin%
\definecolor{currentfill}{rgb}{0.835187,0.855484,0.883900}%
\pgfsetfillcolor{currentfill}%
\pgfsetlinewidth{0.000000pt}%
\definecolor{currentstroke}{rgb}{0.000000,0.000000,0.000000}%
\pgfsetstrokecolor{currentstroke}%
\pgfsetdash{}{0pt}%
\pgfpathmoveto{\pgfqpoint{1.536486in}{1.643186in}}%
\pgfpathlineto{\pgfqpoint{1.573120in}{1.648738in}}%
\pgfpathlineto{\pgfqpoint{1.536486in}{1.666532in}}%
\pgfpathlineto{\pgfqpoint{1.499812in}{1.661041in}}%
\pgfpathclose%
\pgfusepath{fill}%
\end{pgfscope}%
\begin{pgfscope}%
\pgfpathrectangle{\pgfqpoint{0.150000in}{0.150000in}}{\pgfqpoint{2.700000in}{1.950000in}}%
\pgfusepath{clip}%
\pgfsetbuttcap%
\pgfsetroundjoin%
\definecolor{currentfill}{rgb}{0.804090,0.828217,0.861994}%
\pgfsetfillcolor{currentfill}%
\pgfsetlinewidth{0.000000pt}%
\definecolor{currentstroke}{rgb}{0.000000,0.000000,0.000000}%
\pgfsetstrokecolor{currentstroke}%
\pgfsetdash{}{0pt}%
\pgfpathmoveto{\pgfqpoint{1.426279in}{1.679547in}}%
\pgfpathlineto{\pgfqpoint{1.463227in}{1.684994in}}%
\pgfpathlineto{\pgfqpoint{1.426792in}{1.702721in}}%
\pgfpathlineto{\pgfqpoint{1.389722in}{1.703517in}}%
\pgfpathclose%
\pgfusepath{fill}%
\end{pgfscope}%
\begin{pgfscope}%
\pgfpathrectangle{\pgfqpoint{0.150000in}{0.150000in}}{\pgfqpoint{2.700000in}{1.950000in}}%
\pgfusepath{clip}%
\pgfsetbuttcap%
\pgfsetroundjoin%
\definecolor{currentfill}{rgb}{0.860064,0.877298,0.901425}%
\pgfsetfillcolor{currentfill}%
\pgfsetlinewidth{0.000000pt}%
\definecolor{currentstroke}{rgb}{0.000000,0.000000,0.000000}%
\pgfsetstrokecolor{currentstroke}%
\pgfsetdash{}{0pt}%
\pgfpathmoveto{\pgfqpoint{1.646717in}{1.606817in}}%
\pgfpathlineto{\pgfqpoint{1.683118in}{1.618604in}}%
\pgfpathlineto{\pgfqpoint{1.646265in}{1.636463in}}%
\pgfpathlineto{\pgfqpoint{1.609884in}{1.630881in}}%
\pgfpathclose%
\pgfusepath{fill}%
\end{pgfscope}%
\begin{pgfscope}%
\pgfpathrectangle{\pgfqpoint{0.150000in}{0.150000in}}{\pgfqpoint{2.700000in}{1.950000in}}%
\pgfusepath{clip}%
\pgfsetbuttcap%
\pgfsetroundjoin%
\definecolor{currentfill}{rgb}{0.884942,0.899112,0.918949}%
\pgfsetfillcolor{currentfill}%
\pgfsetlinewidth{0.000000pt}%
\definecolor{currentstroke}{rgb}{0.000000,0.000000,0.000000}%
\pgfsetstrokecolor{currentstroke}%
\pgfsetdash{}{0pt}%
\pgfpathmoveto{\pgfqpoint{1.757094in}{1.576564in}}%
\pgfpathlineto{\pgfqpoint{1.793261in}{1.588429in}}%
\pgfpathlineto{\pgfqpoint{1.756189in}{1.606354in}}%
\pgfpathlineto{\pgfqpoint{1.719999in}{1.594552in}}%
\pgfpathclose%
\pgfusepath{fill}%
\end{pgfscope}%
\begin{pgfscope}%
\pgfpathrectangle{\pgfqpoint{0.150000in}{0.150000in}}{\pgfqpoint{2.700000in}{1.950000in}}%
\pgfusepath{clip}%
\pgfsetbuttcap%
\pgfsetroundjoin%
\definecolor{currentfill}{rgb}{0.810309,0.833670,0.866376}%
\pgfsetfillcolor{currentfill}%
\pgfsetlinewidth{0.000000pt}%
\definecolor{currentstroke}{rgb}{0.000000,0.000000,0.000000}%
\pgfsetstrokecolor{currentstroke}%
\pgfsetdash{}{0pt}%
\pgfpathmoveto{\pgfqpoint{1.462925in}{1.655518in}}%
\pgfpathlineto{\pgfqpoint{1.499812in}{1.661041in}}%
\pgfpathlineto{\pgfqpoint{1.463227in}{1.684994in}}%
\pgfpathlineto{\pgfqpoint{1.426279in}{1.679547in}}%
\pgfpathclose%
\pgfusepath{fill}%
\end{pgfscope}%
\begin{pgfscope}%
\pgfpathrectangle{\pgfqpoint{0.150000in}{0.150000in}}{\pgfqpoint{2.700000in}{1.950000in}}%
\pgfusepath{clip}%
\pgfsetbuttcap%
\pgfsetroundjoin%
\definecolor{currentfill}{rgb}{0.697978,0.451976,0.471461}%
\pgfsetfillcolor{currentfill}%
\pgfsetlinewidth{0.000000pt}%
\definecolor{currentstroke}{rgb}{0.000000,0.000000,0.000000}%
\pgfsetstrokecolor{currentstroke}%
\pgfsetdash{}{0pt}%
\pgfpathmoveto{\pgfqpoint{2.235695in}{0.872096in}}%
\pgfpathlineto{\pgfqpoint{2.269141in}{0.862740in}}%
\pgfpathlineto{\pgfqpoint{2.232435in}{0.899077in}}%
\pgfpathlineto{\pgfqpoint{2.198455in}{0.902844in}}%
\pgfpathclose%
\pgfusepath{fill}%
\end{pgfscope}%
\begin{pgfscope}%
\pgfpathrectangle{\pgfqpoint{0.150000in}{0.150000in}}{\pgfqpoint{2.700000in}{1.950000in}}%
\pgfusepath{clip}%
\pgfsetbuttcap%
\pgfsetroundjoin%
\definecolor{currentfill}{rgb}{0.841406,0.860938,0.888281}%
\pgfsetfillcolor{currentfill}%
\pgfsetlinewidth{0.000000pt}%
\definecolor{currentstroke}{rgb}{0.000000,0.000000,0.000000}%
\pgfsetstrokecolor{currentstroke}%
\pgfsetdash{}{0pt}%
\pgfpathmoveto{\pgfqpoint{1.573271in}{1.619110in}}%
\pgfpathlineto{\pgfqpoint{1.609884in}{1.630881in}}%
\pgfpathlineto{\pgfqpoint{1.573120in}{1.648738in}}%
\pgfpathlineto{\pgfqpoint{1.536486in}{1.643186in}}%
\pgfpathclose%
\pgfusepath{fill}%
\end{pgfscope}%
\begin{pgfscope}%
\pgfpathrectangle{\pgfqpoint{0.150000in}{0.150000in}}{\pgfqpoint{2.700000in}{1.950000in}}%
\pgfusepath{clip}%
\pgfsetbuttcap%
\pgfsetroundjoin%
\definecolor{currentfill}{rgb}{0.866284,0.882751,0.905806}%
\pgfsetfillcolor{currentfill}%
\pgfsetlinewidth{0.000000pt}%
\definecolor{currentstroke}{rgb}{0.000000,0.000000,0.000000}%
\pgfsetstrokecolor{currentstroke}%
\pgfsetdash{}{0pt}%
\pgfpathmoveto{\pgfqpoint{1.683723in}{1.588832in}}%
\pgfpathlineto{\pgfqpoint{1.719999in}{1.594552in}}%
\pgfpathlineto{\pgfqpoint{1.683118in}{1.618604in}}%
\pgfpathlineto{\pgfqpoint{1.646717in}{1.606817in}}%
\pgfpathclose%
\pgfusepath{fill}%
\end{pgfscope}%
\begin{pgfscope}%
\pgfpathrectangle{\pgfqpoint{0.150000in}{0.150000in}}{\pgfqpoint{2.700000in}{1.950000in}}%
\pgfusepath{clip}%
\pgfsetbuttcap%
\pgfsetroundjoin%
\definecolor{currentfill}{rgb}{0.686581,0.431296,0.451517}%
\pgfsetfillcolor{currentfill}%
\pgfsetlinewidth{0.000000pt}%
\definecolor{currentstroke}{rgb}{0.000000,0.000000,0.000000}%
\pgfsetstrokecolor{currentstroke}%
\pgfsetdash{}{0pt}%
\pgfpathmoveto{\pgfqpoint{2.272985in}{0.841308in}}%
\pgfpathlineto{\pgfqpoint{2.306712in}{0.837782in}}%
\pgfpathlineto{\pgfqpoint{2.269141in}{0.862740in}}%
\pgfpathlineto{\pgfqpoint{2.235695in}{0.872096in}}%
\pgfpathclose%
\pgfusepath{fill}%
\end{pgfscope}%
\begin{pgfscope}%
\pgfpathrectangle{\pgfqpoint{0.150000in}{0.150000in}}{\pgfqpoint{2.700000in}{1.950000in}}%
\pgfusepath{clip}%
\pgfsetbuttcap%
\pgfsetroundjoin%
\definecolor{currentfill}{rgb}{0.766360,0.576057,0.591131}%
\pgfsetfillcolor{currentfill}%
\pgfsetlinewidth{0.000000pt}%
\definecolor{currentstroke}{rgb}{0.000000,0.000000,0.000000}%
\pgfsetstrokecolor{currentstroke}%
\pgfsetdash{}{0pt}%
\pgfpathmoveto{\pgfqpoint{2.197946in}{0.969998in}}%
\pgfpathlineto{\pgfqpoint{2.231902in}{0.965965in}}%
\pgfpathlineto{\pgfqpoint{2.197807in}{1.042869in}}%
\pgfpathlineto{\pgfqpoint{2.163232in}{1.041380in}}%
\pgfpathclose%
\pgfusepath{fill}%
\end{pgfscope}%
\begin{pgfscope}%
\pgfpathrectangle{\pgfqpoint{0.150000in}{0.150000in}}{\pgfqpoint{2.700000in}{1.950000in}}%
\pgfusepath{clip}%
\pgfsetbuttcap%
\pgfsetroundjoin%
\definecolor{currentfill}{rgb}{0.732169,0.514017,0.531296}%
\pgfsetfillcolor{currentfill}%
\pgfsetlinewidth{0.000000pt}%
\definecolor{currentstroke}{rgb}{0.000000,0.000000,0.000000}%
\pgfsetstrokecolor{currentstroke}%
\pgfsetdash{}{0pt}%
\pgfpathmoveto{\pgfqpoint{2.198455in}{0.902844in}}%
\pgfpathlineto{\pgfqpoint{2.232435in}{0.899077in}}%
\pgfpathlineto{\pgfqpoint{2.197946in}{0.969998in}}%
\pgfpathlineto{\pgfqpoint{2.163364in}{0.968256in}}%
\pgfpathclose%
\pgfusepath{fill}%
\end{pgfscope}%
\begin{pgfscope}%
\pgfpathrectangle{\pgfqpoint{0.150000in}{0.150000in}}{\pgfqpoint{2.700000in}{1.950000in}}%
\pgfusepath{clip}%
\pgfsetbuttcap%
\pgfsetroundjoin%
\definecolor{currentfill}{rgb}{0.811949,0.658778,0.670910}%
\pgfsetfillcolor{currentfill}%
\pgfsetlinewidth{0.000000pt}%
\definecolor{currentstroke}{rgb}{0.000000,0.000000,0.000000}%
\pgfsetstrokecolor{currentstroke}%
\pgfsetdash{}{0pt}%
\pgfpathmoveto{\pgfqpoint{2.197807in}{1.042869in}}%
\pgfpathlineto{\pgfqpoint{2.232143in}{1.044347in}}%
\pgfpathlineto{\pgfqpoint{2.198407in}{1.127450in}}%
\pgfpathlineto{\pgfqpoint{2.163803in}{1.126253in}}%
\pgfpathclose%
\pgfusepath{fill}%
\end{pgfscope}%
\begin{pgfscope}%
\pgfpathrectangle{\pgfqpoint{0.150000in}{0.150000in}}{\pgfqpoint{2.700000in}{1.950000in}}%
\pgfusepath{clip}%
\pgfsetbuttcap%
\pgfsetroundjoin%
\definecolor{currentfill}{rgb}{0.884942,0.899112,0.918949}%
\pgfsetfillcolor{currentfill}%
\pgfsetlinewidth{0.000000pt}%
\definecolor{currentstroke}{rgb}{0.000000,0.000000,0.000000}%
\pgfsetstrokecolor{currentstroke}%
\pgfsetdash{}{0pt}%
\pgfpathmoveto{\pgfqpoint{1.794322in}{1.558513in}}%
\pgfpathlineto{\pgfqpoint{1.830466in}{1.570441in}}%
\pgfpathlineto{\pgfqpoint{1.793261in}{1.588429in}}%
\pgfpathlineto{\pgfqpoint{1.757094in}{1.576564in}}%
\pgfpathclose%
\pgfusepath{fill}%
\end{pgfscope}%
\begin{pgfscope}%
\pgfpathrectangle{\pgfqpoint{0.150000in}{0.150000in}}{\pgfqpoint{2.700000in}{1.950000in}}%
\pgfusepath{clip}%
\pgfsetbuttcap%
\pgfsetroundjoin%
\definecolor{currentfill}{rgb}{0.816529,0.839124,0.870757}%
\pgfsetfillcolor{currentfill}%
\pgfsetlinewidth{0.000000pt}%
\definecolor{currentstroke}{rgb}{0.000000,0.000000,0.000000}%
\pgfsetstrokecolor{currentstroke}%
\pgfsetdash{}{0pt}%
\pgfpathmoveto{\pgfqpoint{1.499640in}{1.637601in}}%
\pgfpathlineto{\pgfqpoint{1.536486in}{1.643186in}}%
\pgfpathlineto{\pgfqpoint{1.499812in}{1.661041in}}%
\pgfpathlineto{\pgfqpoint{1.462925in}{1.655518in}}%
\pgfpathclose%
\pgfusepath{fill}%
\end{pgfscope}%
\begin{pgfscope}%
\pgfpathrectangle{\pgfqpoint{0.150000in}{0.150000in}}{\pgfqpoint{2.700000in}{1.950000in}}%
\pgfusepath{clip}%
\pgfsetbuttcap%
\pgfsetroundjoin%
\definecolor{currentfill}{rgb}{0.779213,0.806403,0.844470}%
\pgfsetfillcolor{currentfill}%
\pgfsetlinewidth{0.000000pt}%
\definecolor{currentstroke}{rgb}{0.000000,0.000000,0.000000}%
\pgfsetstrokecolor{currentstroke}%
\pgfsetdash{}{0pt}%
\pgfpathmoveto{\pgfqpoint{1.389116in}{1.674068in}}%
\pgfpathlineto{\pgfqpoint{1.426279in}{1.679547in}}%
\pgfpathlineto{\pgfqpoint{1.389722in}{1.703517in}}%
\pgfpathlineto{\pgfqpoint{1.352499in}{1.698114in}}%
\pgfpathclose%
\pgfusepath{fill}%
\end{pgfscope}%
\begin{pgfscope}%
\pgfpathrectangle{\pgfqpoint{0.150000in}{0.150000in}}{\pgfqpoint{2.700000in}{1.950000in}}%
\pgfusepath{clip}%
\pgfsetbuttcap%
\pgfsetroundjoin%
\definecolor{currentfill}{rgb}{0.841406,0.860938,0.888281}%
\pgfsetfillcolor{currentfill}%
\pgfsetlinewidth{0.000000pt}%
\definecolor{currentstroke}{rgb}{0.000000,0.000000,0.000000}%
\pgfsetstrokecolor{currentstroke}%
\pgfsetdash{}{0pt}%
\pgfpathmoveto{\pgfqpoint{1.610187in}{1.601126in}}%
\pgfpathlineto{\pgfqpoint{1.646717in}{1.606817in}}%
\pgfpathlineto{\pgfqpoint{1.609884in}{1.630881in}}%
\pgfpathlineto{\pgfqpoint{1.573271in}{1.619110in}}%
\pgfpathclose%
\pgfusepath{fill}%
\end{pgfscope}%
\begin{pgfscope}%
\pgfpathrectangle{\pgfqpoint{0.150000in}{0.150000in}}{\pgfqpoint{2.700000in}{1.950000in}}%
\pgfusepath{clip}%
\pgfsetbuttcap%
\pgfsetroundjoin%
\definecolor{currentfill}{rgb}{0.872503,0.888205,0.910187}%
\pgfsetfillcolor{currentfill}%
\pgfsetlinewidth{0.000000pt}%
\definecolor{currentstroke}{rgb}{0.000000,0.000000,0.000000}%
\pgfsetstrokecolor{currentstroke}%
\pgfsetdash{}{0pt}%
\pgfpathmoveto{\pgfqpoint{1.720758in}{1.564644in}}%
\pgfpathlineto{\pgfqpoint{1.757094in}{1.576564in}}%
\pgfpathlineto{\pgfqpoint{1.719999in}{1.594552in}}%
\pgfpathlineto{\pgfqpoint{1.683723in}{1.588832in}}%
\pgfpathclose%
\pgfusepath{fill}%
\end{pgfscope}%
\begin{pgfscope}%
\pgfpathrectangle{\pgfqpoint{0.150000in}{0.150000in}}{\pgfqpoint{2.700000in}{1.950000in}}%
\pgfusepath{clip}%
\pgfsetbuttcap%
\pgfsetroundjoin%
\definecolor{currentfill}{rgb}{0.686581,0.431296,0.451517}%
\pgfsetfillcolor{currentfill}%
\pgfsetlinewidth{0.000000pt}%
\definecolor{currentstroke}{rgb}{0.000000,0.000000,0.000000}%
\pgfsetstrokecolor{currentstroke}%
\pgfsetdash{}{0pt}%
\pgfpathmoveto{\pgfqpoint{2.312062in}{0.833375in}}%
\pgfpathlineto{\pgfqpoint{2.345729in}{0.829871in}}%
\pgfpathlineto{\pgfqpoint{2.306712in}{0.837782in}}%
\pgfpathlineto{\pgfqpoint{2.272985in}{0.841308in}}%
\pgfpathclose%
\pgfusepath{fill}%
\end{pgfscope}%
\begin{pgfscope}%
\pgfpathrectangle{\pgfqpoint{0.150000in}{0.150000in}}{\pgfqpoint{2.700000in}{1.950000in}}%
\pgfusepath{clip}%
\pgfsetbuttcap%
\pgfsetroundjoin%
\definecolor{currentfill}{rgb}{0.868934,0.762178,0.770634}%
\pgfsetfillcolor{currentfill}%
\pgfsetlinewidth{0.000000pt}%
\definecolor{currentstroke}{rgb}{0.000000,0.000000,0.000000}%
\pgfsetstrokecolor{currentstroke}%
\pgfsetdash{}{0pt}%
\pgfpathmoveto{\pgfqpoint{2.198407in}{1.127450in}}%
\pgfpathlineto{\pgfqpoint{2.233160in}{1.134499in}}%
\pgfpathlineto{\pgfqpoint{2.199749in}{1.224057in}}%
\pgfpathlineto{\pgfqpoint{2.164727in}{1.217234in}}%
\pgfpathclose%
\pgfusepath{fill}%
\end{pgfscope}%
\begin{pgfscope}%
\pgfpathrectangle{\pgfqpoint{0.150000in}{0.150000in}}{\pgfqpoint{2.700000in}{1.950000in}}%
\pgfusepath{clip}%
\pgfsetbuttcap%
\pgfsetroundjoin%
\definecolor{currentfill}{rgb}{0.791651,0.817310,0.853232}%
\pgfsetfillcolor{currentfill}%
\pgfsetlinewidth{0.000000pt}%
\definecolor{currentstroke}{rgb}{0.000000,0.000000,0.000000}%
\pgfsetstrokecolor{currentstroke}%
\pgfsetdash{}{0pt}%
\pgfpathmoveto{\pgfqpoint{1.425823in}{1.649963in}}%
\pgfpathlineto{\pgfqpoint{1.462925in}{1.655518in}}%
\pgfpathlineto{\pgfqpoint{1.426279in}{1.679547in}}%
\pgfpathlineto{\pgfqpoint{1.389116in}{1.674068in}}%
\pgfpathclose%
\pgfusepath{fill}%
\end{pgfscope}%
\begin{pgfscope}%
\pgfpathrectangle{\pgfqpoint{0.150000in}{0.150000in}}{\pgfqpoint{2.700000in}{1.950000in}}%
\pgfusepath{clip}%
\pgfsetbuttcap%
\pgfsetroundjoin%
\definecolor{currentfill}{rgb}{0.884942,0.899112,0.918949}%
\pgfsetfillcolor{currentfill}%
\pgfsetlinewidth{0.000000pt}%
\definecolor{currentstroke}{rgb}{0.000000,0.000000,0.000000}%
\pgfsetstrokecolor{currentstroke}%
\pgfsetdash{}{0pt}%
\pgfpathmoveto{\pgfqpoint{1.831517in}{1.534272in}}%
\pgfpathlineto{\pgfqpoint{1.867618in}{1.546271in}}%
\pgfpathlineto{\pgfqpoint{1.830466in}{1.570441in}}%
\pgfpathlineto{\pgfqpoint{1.794322in}{1.558513in}}%
\pgfpathclose%
\pgfusepath{fill}%
\end{pgfscope}%
\begin{pgfscope}%
\pgfpathrectangle{\pgfqpoint{0.150000in}{0.150000in}}{\pgfqpoint{2.700000in}{1.950000in}}%
\pgfusepath{clip}%
\pgfsetbuttcap%
\pgfsetroundjoin%
\definecolor{currentfill}{rgb}{0.822748,0.844577,0.875138}%
\pgfsetfillcolor{currentfill}%
\pgfsetlinewidth{0.000000pt}%
\definecolor{currentstroke}{rgb}{0.000000,0.000000,0.000000}%
\pgfsetstrokecolor{currentstroke}%
\pgfsetdash{}{0pt}%
\pgfpathmoveto{\pgfqpoint{1.536486in}{1.613448in}}%
\pgfpathlineto{\pgfqpoint{1.573271in}{1.619110in}}%
\pgfpathlineto{\pgfqpoint{1.536486in}{1.643186in}}%
\pgfpathlineto{\pgfqpoint{1.499640in}{1.637601in}}%
\pgfpathclose%
\pgfusepath{fill}%
\end{pgfscope}%
\begin{pgfscope}%
\pgfpathrectangle{\pgfqpoint{0.150000in}{0.150000in}}{\pgfqpoint{2.700000in}{1.950000in}}%
\pgfusepath{clip}%
\pgfsetbuttcap%
\pgfsetroundjoin%
\definecolor{currentfill}{rgb}{0.847626,0.866391,0.892662}%
\pgfsetfillcolor{currentfill}%
\pgfsetlinewidth{0.000000pt}%
\definecolor{currentstroke}{rgb}{0.000000,0.000000,0.000000}%
\pgfsetstrokecolor{currentstroke}%
\pgfsetdash{}{0pt}%
\pgfpathmoveto{\pgfqpoint{1.647173in}{1.576926in}}%
\pgfpathlineto{\pgfqpoint{1.683723in}{1.588832in}}%
\pgfpathlineto{\pgfqpoint{1.646717in}{1.606817in}}%
\pgfpathlineto{\pgfqpoint{1.610187in}{1.601126in}}%
\pgfpathclose%
\pgfusepath{fill}%
\end{pgfscope}%
\begin{pgfscope}%
\pgfpathrectangle{\pgfqpoint{0.150000in}{0.150000in}}{\pgfqpoint{2.700000in}{1.950000in}}%
\pgfusepath{clip}%
\pgfsetbuttcap%
\pgfsetroundjoin%
\definecolor{currentfill}{rgb}{0.872503,0.888205,0.910187}%
\pgfsetfillcolor{currentfill}%
\pgfsetlinewidth{0.000000pt}%
\definecolor{currentstroke}{rgb}{0.000000,0.000000,0.000000}%
\pgfsetstrokecolor{currentstroke}%
\pgfsetdash{}{0pt}%
\pgfpathmoveto{\pgfqpoint{1.758007in}{1.546528in}}%
\pgfpathlineto{\pgfqpoint{1.794322in}{1.558513in}}%
\pgfpathlineto{\pgfqpoint{1.757094in}{1.576564in}}%
\pgfpathlineto{\pgfqpoint{1.720758in}{1.564644in}}%
\pgfpathclose%
\pgfusepath{fill}%
\end{pgfscope}%
\begin{pgfscope}%
\pgfpathrectangle{\pgfqpoint{0.150000in}{0.150000in}}{\pgfqpoint{2.700000in}{1.950000in}}%
\pgfusepath{clip}%
\pgfsetbuttcap%
\pgfsetroundjoin%
\definecolor{currentfill}{rgb}{0.797871,0.822763,0.857613}%
\pgfsetfillcolor{currentfill}%
\pgfsetlinewidth{0.000000pt}%
\definecolor{currentstroke}{rgb}{0.000000,0.000000,0.000000}%
\pgfsetstrokecolor{currentstroke}%
\pgfsetdash{}{0pt}%
\pgfpathmoveto{\pgfqpoint{1.462620in}{1.625798in}}%
\pgfpathlineto{\pgfqpoint{1.499640in}{1.637601in}}%
\pgfpathlineto{\pgfqpoint{1.462925in}{1.655518in}}%
\pgfpathlineto{\pgfqpoint{1.425823in}{1.649963in}}%
\pgfpathclose%
\pgfusepath{fill}%
\end{pgfscope}%
\begin{pgfscope}%
\pgfpathrectangle{\pgfqpoint{0.150000in}{0.150000in}}{\pgfqpoint{2.700000in}{1.950000in}}%
\pgfusepath{clip}%
\pgfsetbuttcap%
\pgfsetroundjoin%
\definecolor{currentfill}{rgb}{0.828968,0.850031,0.879519}%
\pgfsetfillcolor{currentfill}%
\pgfsetlinewidth{0.000000pt}%
\definecolor{currentstroke}{rgb}{0.000000,0.000000,0.000000}%
\pgfsetstrokecolor{currentstroke}%
\pgfsetdash{}{0pt}%
\pgfpathmoveto{\pgfqpoint{1.573423in}{1.589236in}}%
\pgfpathlineto{\pgfqpoint{1.610187in}{1.601126in}}%
\pgfpathlineto{\pgfqpoint{1.573271in}{1.619110in}}%
\pgfpathlineto{\pgfqpoint{1.536486in}{1.613448in}}%
\pgfpathclose%
\pgfusepath{fill}%
\end{pgfscope}%
\begin{pgfscope}%
\pgfpathrectangle{\pgfqpoint{0.150000in}{0.150000in}}{\pgfqpoint{2.700000in}{1.950000in}}%
\pgfusepath{clip}%
\pgfsetbuttcap%
\pgfsetroundjoin%
\definecolor{currentfill}{rgb}{0.891161,0.904565,0.923330}%
\pgfsetfillcolor{currentfill}%
\pgfsetlinewidth{0.000000pt}%
\definecolor{currentstroke}{rgb}{0.000000,0.000000,0.000000}%
\pgfsetstrokecolor{currentstroke}%
\pgfsetdash{}{0pt}%
\pgfpathmoveto{\pgfqpoint{1.868990in}{1.516090in}}%
\pgfpathlineto{\pgfqpoint{1.905068in}{1.528154in}}%
\pgfpathlineto{\pgfqpoint{1.867618in}{1.546271in}}%
\pgfpathlineto{\pgfqpoint{1.831517in}{1.534272in}}%
\pgfpathclose%
\pgfusepath{fill}%
\end{pgfscope}%
\begin{pgfscope}%
\pgfpathrectangle{\pgfqpoint{0.150000in}{0.150000in}}{\pgfqpoint{2.700000in}{1.950000in}}%
\pgfusepath{clip}%
\pgfsetbuttcap%
\pgfsetroundjoin%
\definecolor{currentfill}{rgb}{0.789154,0.617417,0.631020}%
\pgfsetfillcolor{currentfill}%
\pgfsetlinewidth{0.000000pt}%
\definecolor{currentstroke}{rgb}{0.000000,0.000000,0.000000}%
\pgfsetstrokecolor{currentstroke}%
\pgfsetdash{}{0pt}%
\pgfpathmoveto{\pgfqpoint{2.163364in}{0.968256in}}%
\pgfpathlineto{\pgfqpoint{2.197946in}{0.969998in}}%
\pgfpathlineto{\pgfqpoint{2.163232in}{1.041380in}}%
\pgfpathlineto{\pgfqpoint{2.128417in}{1.039881in}}%
\pgfpathclose%
\pgfusepath{fill}%
\end{pgfscope}%
\begin{pgfscope}%
\pgfpathrectangle{\pgfqpoint{0.150000in}{0.150000in}}{\pgfqpoint{2.700000in}{1.950000in}}%
\pgfusepath{clip}%
\pgfsetbuttcap%
\pgfsetroundjoin%
\definecolor{currentfill}{rgb}{0.728370,0.507123,0.524648}%
\pgfsetfillcolor{currentfill}%
\pgfsetlinewidth{0.000000pt}%
\definecolor{currentstroke}{rgb}{0.000000,0.000000,0.000000}%
\pgfsetstrokecolor{currentstroke}%
\pgfsetdash{}{0pt}%
\pgfpathmoveto{\pgfqpoint{2.201942in}{0.881539in}}%
\pgfpathlineto{\pgfqpoint{2.235695in}{0.872096in}}%
\pgfpathlineto{\pgfqpoint{2.198455in}{0.902844in}}%
\pgfpathlineto{\pgfqpoint{2.164553in}{0.912421in}}%
\pgfpathclose%
\pgfusepath{fill}%
\end{pgfscope}%
\begin{pgfscope}%
\pgfpathrectangle{\pgfqpoint{0.150000in}{0.150000in}}{\pgfqpoint{2.700000in}{1.950000in}}%
\pgfusepath{clip}%
\pgfsetbuttcap%
\pgfsetroundjoin%
\definecolor{currentfill}{rgb}{0.760555,0.790043,0.831327}%
\pgfsetfillcolor{currentfill}%
\pgfsetlinewidth{0.000000pt}%
\definecolor{currentstroke}{rgb}{0.000000,0.000000,0.000000}%
\pgfsetstrokecolor{currentstroke}%
\pgfsetdash{}{0pt}%
\pgfpathmoveto{\pgfqpoint{1.351737in}{1.668557in}}%
\pgfpathlineto{\pgfqpoint{1.389116in}{1.674068in}}%
\pgfpathlineto{\pgfqpoint{1.352499in}{1.698114in}}%
\pgfpathlineto{\pgfqpoint{1.315059in}{1.692680in}}%
\pgfpathclose%
\pgfusepath{fill}%
\end{pgfscope}%
\begin{pgfscope}%
\pgfpathrectangle{\pgfqpoint{0.150000in}{0.150000in}}{\pgfqpoint{2.700000in}{1.950000in}}%
\pgfusepath{clip}%
\pgfsetbuttcap%
\pgfsetroundjoin%
\definecolor{currentfill}{rgb}{0.830944,0.693244,0.704151}%
\pgfsetfillcolor{currentfill}%
\pgfsetlinewidth{0.000000pt}%
\definecolor{currentstroke}{rgb}{0.000000,0.000000,0.000000}%
\pgfsetstrokecolor{currentstroke}%
\pgfsetdash{}{0pt}%
\pgfpathmoveto{\pgfqpoint{2.163232in}{1.041380in}}%
\pgfpathlineto{\pgfqpoint{2.197807in}{1.042869in}}%
\pgfpathlineto{\pgfqpoint{2.163803in}{1.126253in}}%
\pgfpathlineto{\pgfqpoint{2.128624in}{1.119136in}}%
\pgfpathclose%
\pgfusepath{fill}%
\end{pgfscope}%
\begin{pgfscope}%
\pgfpathrectangle{\pgfqpoint{0.150000in}{0.150000in}}{\pgfqpoint{2.700000in}{1.950000in}}%
\pgfusepath{clip}%
\pgfsetbuttcap%
\pgfsetroundjoin%
\definecolor{currentfill}{rgb}{0.853845,0.871844,0.897044}%
\pgfsetfillcolor{currentfill}%
\pgfsetlinewidth{0.000000pt}%
\definecolor{currentstroke}{rgb}{0.000000,0.000000,0.000000}%
\pgfsetstrokecolor{currentstroke}%
\pgfsetdash{}{0pt}%
\pgfpathmoveto{\pgfqpoint{1.684333in}{1.558813in}}%
\pgfpathlineto{\pgfqpoint{1.720758in}{1.564644in}}%
\pgfpathlineto{\pgfqpoint{1.683723in}{1.588832in}}%
\pgfpathlineto{\pgfqpoint{1.647173in}{1.576926in}}%
\pgfpathclose%
\pgfusepath{fill}%
\end{pgfscope}%
\begin{pgfscope}%
\pgfpathrectangle{\pgfqpoint{0.150000in}{0.150000in}}{\pgfqpoint{2.700000in}{1.950000in}}%
\pgfusepath{clip}%
\pgfsetbuttcap%
\pgfsetroundjoin%
\definecolor{currentfill}{rgb}{0.716973,0.486443,0.504703}%
\pgfsetfillcolor{currentfill}%
\pgfsetlinewidth{0.000000pt}%
\definecolor{currentstroke}{rgb}{0.000000,0.000000,0.000000}%
\pgfsetstrokecolor{currentstroke}%
\pgfsetdash{}{0pt}%
\pgfpathmoveto{\pgfqpoint{2.239381in}{0.850615in}}%
\pgfpathlineto{\pgfqpoint{2.272985in}{0.841308in}}%
\pgfpathlineto{\pgfqpoint{2.235695in}{0.872096in}}%
\pgfpathlineto{\pgfqpoint{2.201942in}{0.881539in}}%
\pgfpathclose%
\pgfusepath{fill}%
\end{pgfscope}%
\begin{pgfscope}%
\pgfpathrectangle{\pgfqpoint{0.150000in}{0.150000in}}{\pgfqpoint{2.700000in}{1.950000in}}%
\pgfusepath{clip}%
\pgfsetbuttcap%
\pgfsetroundjoin%
\definecolor{currentfill}{rgb}{0.872503,0.888205,0.910187}%
\pgfsetfillcolor{currentfill}%
\pgfsetlinewidth{0.000000pt}%
\definecolor{currentstroke}{rgb}{0.000000,0.000000,0.000000}%
\pgfsetstrokecolor{currentstroke}%
\pgfsetdash{}{0pt}%
\pgfpathmoveto{\pgfqpoint{1.795391in}{1.528348in}}%
\pgfpathlineto{\pgfqpoint{1.831517in}{1.534272in}}%
\pgfpathlineto{\pgfqpoint{1.794322in}{1.558513in}}%
\pgfpathlineto{\pgfqpoint{1.758007in}{1.546528in}}%
\pgfpathclose%
\pgfusepath{fill}%
\end{pgfscope}%
\begin{pgfscope}%
\pgfpathrectangle{\pgfqpoint{0.150000in}{0.150000in}}{\pgfqpoint{2.700000in}{1.950000in}}%
\pgfusepath{clip}%
\pgfsetbuttcap%
\pgfsetroundjoin%
\definecolor{currentfill}{rgb}{0.758762,0.562270,0.577834}%
\pgfsetfillcolor{currentfill}%
\pgfsetlinewidth{0.000000pt}%
\definecolor{currentstroke}{rgb}{0.000000,0.000000,0.000000}%
\pgfsetstrokecolor{currentstroke}%
\pgfsetdash{}{0pt}%
\pgfpathmoveto{\pgfqpoint{2.164553in}{0.912421in}}%
\pgfpathlineto{\pgfqpoint{2.198455in}{0.902844in}}%
\pgfpathlineto{\pgfqpoint{2.163364in}{0.968256in}}%
\pgfpathlineto{\pgfqpoint{2.128875in}{0.972328in}}%
\pgfpathclose%
\pgfusepath{fill}%
\end{pgfscope}%
\begin{pgfscope}%
\pgfpathrectangle{\pgfqpoint{0.150000in}{0.150000in}}{\pgfqpoint{2.700000in}{1.950000in}}%
\pgfusepath{clip}%
\pgfsetbuttcap%
\pgfsetroundjoin%
\definecolor{currentfill}{rgb}{0.804090,0.828217,0.861994}%
\pgfsetfillcolor{currentfill}%
\pgfsetlinewidth{0.000000pt}%
\definecolor{currentstroke}{rgb}{0.000000,0.000000,0.000000}%
\pgfsetstrokecolor{currentstroke}%
\pgfsetdash{}{0pt}%
\pgfpathmoveto{\pgfqpoint{1.499487in}{1.607754in}}%
\pgfpathlineto{\pgfqpoint{1.536486in}{1.613448in}}%
\pgfpathlineto{\pgfqpoint{1.499640in}{1.637601in}}%
\pgfpathlineto{\pgfqpoint{1.462620in}{1.625798in}}%
\pgfpathclose%
\pgfusepath{fill}%
\end{pgfscope}%
\begin{pgfscope}%
\pgfpathrectangle{\pgfqpoint{0.150000in}{0.150000in}}{\pgfqpoint{2.700000in}{1.950000in}}%
\pgfusepath{clip}%
\pgfsetbuttcap%
\pgfsetroundjoin%
\definecolor{currentfill}{rgb}{0.880331,0.782858,0.790579}%
\pgfsetfillcolor{currentfill}%
\pgfsetlinewidth{0.000000pt}%
\definecolor{currentstroke}{rgb}{0.000000,0.000000,0.000000}%
\pgfsetstrokecolor{currentstroke}%
\pgfsetdash{}{0pt}%
\pgfpathmoveto{\pgfqpoint{2.163803in}{1.126253in}}%
\pgfpathlineto{\pgfqpoint{2.198407in}{1.127450in}}%
\pgfpathlineto{\pgfqpoint{2.164727in}{1.217234in}}%
\pgfpathlineto{\pgfqpoint{2.129165in}{1.204405in}}%
\pgfpathclose%
\pgfusepath{fill}%
\end{pgfscope}%
\begin{pgfscope}%
\pgfpathrectangle{\pgfqpoint{0.150000in}{0.150000in}}{\pgfqpoint{2.700000in}{1.950000in}}%
\pgfusepath{clip}%
\pgfsetbuttcap%
\pgfsetroundjoin%
\definecolor{currentfill}{rgb}{0.772993,0.800950,0.840089}%
\pgfsetfillcolor{currentfill}%
\pgfsetlinewidth{0.000000pt}%
\definecolor{currentstroke}{rgb}{0.000000,0.000000,0.000000}%
\pgfsetstrokecolor{currentstroke}%
\pgfsetdash{}{0pt}%
\pgfpathmoveto{\pgfqpoint{1.388505in}{1.644375in}}%
\pgfpathlineto{\pgfqpoint{1.425823in}{1.649963in}}%
\pgfpathlineto{\pgfqpoint{1.389116in}{1.674068in}}%
\pgfpathlineto{\pgfqpoint{1.351737in}{1.668557in}}%
\pgfpathclose%
\pgfusepath{fill}%
\end{pgfscope}%
\begin{pgfscope}%
\pgfpathrectangle{\pgfqpoint{0.150000in}{0.150000in}}{\pgfqpoint{2.700000in}{1.950000in}}%
\pgfusepath{clip}%
\pgfsetbuttcap%
\pgfsetroundjoin%
\definecolor{currentfill}{rgb}{0.835187,0.855484,0.883900}%
\pgfsetfillcolor{currentfill}%
\pgfsetlinewidth{0.000000pt}%
\definecolor{currentstroke}{rgb}{0.000000,0.000000,0.000000}%
\pgfsetstrokecolor{currentstroke}%
\pgfsetdash{}{0pt}%
\pgfpathmoveto{\pgfqpoint{1.610493in}{1.571124in}}%
\pgfpathlineto{\pgfqpoint{1.647173in}{1.576926in}}%
\pgfpathlineto{\pgfqpoint{1.610187in}{1.601126in}}%
\pgfpathlineto{\pgfqpoint{1.573423in}{1.589236in}}%
\pgfpathclose%
\pgfusepath{fill}%
\end{pgfscope}%
\begin{pgfscope}%
\pgfpathrectangle{\pgfqpoint{0.150000in}{0.150000in}}{\pgfqpoint{2.700000in}{1.950000in}}%
\pgfusepath{clip}%
\pgfsetbuttcap%
\pgfsetroundjoin%
\definecolor{currentfill}{rgb}{0.891161,0.904565,0.923330}%
\pgfsetfillcolor{currentfill}%
\pgfsetlinewidth{0.000000pt}%
\definecolor{currentstroke}{rgb}{0.000000,0.000000,0.000000}%
\pgfsetstrokecolor{currentstroke}%
\pgfsetdash{}{0pt}%
\pgfpathmoveto{\pgfqpoint{1.906597in}{1.497843in}}%
\pgfpathlineto{\pgfqpoint{1.942651in}{1.509971in}}%
\pgfpathlineto{\pgfqpoint{1.905068in}{1.528154in}}%
\pgfpathlineto{\pgfqpoint{1.868990in}{1.516090in}}%
\pgfpathclose%
\pgfusepath{fill}%
\end{pgfscope}%
\begin{pgfscope}%
\pgfpathrectangle{\pgfqpoint{0.150000in}{0.150000in}}{\pgfqpoint{2.700000in}{1.950000in}}%
\pgfusepath{clip}%
\pgfsetbuttcap%
\pgfsetroundjoin%
\definecolor{currentfill}{rgb}{0.860064,0.877298,0.901425}%
\pgfsetfillcolor{currentfill}%
\pgfsetlinewidth{0.000000pt}%
\definecolor{currentstroke}{rgb}{0.000000,0.000000,0.000000}%
\pgfsetstrokecolor{currentstroke}%
\pgfsetdash{}{0pt}%
\pgfpathmoveto{\pgfqpoint{1.721522in}{1.534487in}}%
\pgfpathlineto{\pgfqpoint{1.758007in}{1.546528in}}%
\pgfpathlineto{\pgfqpoint{1.720758in}{1.564644in}}%
\pgfpathlineto{\pgfqpoint{1.684333in}{1.558813in}}%
\pgfpathclose%
\pgfusepath{fill}%
\end{pgfscope}%
\begin{pgfscope}%
\pgfpathrectangle{\pgfqpoint{0.150000in}{0.150000in}}{\pgfqpoint{2.700000in}{1.950000in}}%
\pgfusepath{clip}%
\pgfsetbuttcap%
\pgfsetroundjoin%
\definecolor{currentfill}{rgb}{0.713174,0.479550,0.498055}%
\pgfsetfillcolor{currentfill}%
\pgfsetlinewidth{0.000000pt}%
\definecolor{currentstroke}{rgb}{0.000000,0.000000,0.000000}%
\pgfsetstrokecolor{currentstroke}%
\pgfsetdash{}{0pt}%
\pgfpathmoveto{\pgfqpoint{2.278122in}{0.836908in}}%
\pgfpathlineto{\pgfqpoint{2.312062in}{0.833375in}}%
\pgfpathlineto{\pgfqpoint{2.272985in}{0.841308in}}%
\pgfpathlineto{\pgfqpoint{2.239381in}{0.850615in}}%
\pgfpathclose%
\pgfusepath{fill}%
\end{pgfscope}%
\begin{pgfscope}%
\pgfpathrectangle{\pgfqpoint{0.150000in}{0.150000in}}{\pgfqpoint{2.700000in}{1.950000in}}%
\pgfusepath{clip}%
\pgfsetbuttcap%
\pgfsetroundjoin%
\definecolor{currentfill}{rgb}{0.878722,0.893658,0.914568}%
\pgfsetfillcolor{currentfill}%
\pgfsetlinewidth{0.000000pt}%
\definecolor{currentstroke}{rgb}{0.000000,0.000000,0.000000}%
\pgfsetstrokecolor{currentstroke}%
\pgfsetdash{}{0pt}%
\pgfpathmoveto{\pgfqpoint{1.832741in}{1.503969in}}%
\pgfpathlineto{\pgfqpoint{1.868990in}{1.516090in}}%
\pgfpathlineto{\pgfqpoint{1.831517in}{1.534272in}}%
\pgfpathlineto{\pgfqpoint{1.795391in}{1.528348in}}%
\pgfpathclose%
\pgfusepath{fill}%
\end{pgfscope}%
\begin{pgfscope}%
\pgfpathrectangle{\pgfqpoint{0.150000in}{0.150000in}}{\pgfqpoint{2.700000in}{1.950000in}}%
\pgfusepath{clip}%
\pgfsetbuttcap%
\pgfsetroundjoin%
\definecolor{currentfill}{rgb}{0.810309,0.833670,0.866376}%
\pgfsetfillcolor{currentfill}%
\pgfsetlinewidth{0.000000pt}%
\definecolor{currentstroke}{rgb}{0.000000,0.000000,0.000000}%
\pgfsetstrokecolor{currentstroke}%
\pgfsetdash{}{0pt}%
\pgfpathmoveto{\pgfqpoint{1.536486in}{1.583464in}}%
\pgfpathlineto{\pgfqpoint{1.573423in}{1.589236in}}%
\pgfpathlineto{\pgfqpoint{1.536486in}{1.613448in}}%
\pgfpathlineto{\pgfqpoint{1.499487in}{1.607754in}}%
\pgfpathclose%
\pgfusepath{fill}%
\end{pgfscope}%
\begin{pgfscope}%
\pgfpathrectangle{\pgfqpoint{0.150000in}{0.150000in}}{\pgfqpoint{2.700000in}{1.950000in}}%
\pgfusepath{clip}%
\pgfsetbuttcap%
\pgfsetroundjoin%
\definecolor{currentfill}{rgb}{0.779213,0.806403,0.844470}%
\pgfsetfillcolor{currentfill}%
\pgfsetlinewidth{0.000000pt}%
\definecolor{currentstroke}{rgb}{0.000000,0.000000,0.000000}%
\pgfsetstrokecolor{currentstroke}%
\pgfsetdash{}{0pt}%
\pgfpathmoveto{\pgfqpoint{1.425301in}{1.626333in}}%
\pgfpathlineto{\pgfqpoint{1.462620in}{1.625798in}}%
\pgfpathlineto{\pgfqpoint{1.425823in}{1.649963in}}%
\pgfpathlineto{\pgfqpoint{1.388505in}{1.644375in}}%
\pgfpathclose%
\pgfusepath{fill}%
\end{pgfscope}%
\begin{pgfscope}%
\pgfpathrectangle{\pgfqpoint{0.150000in}{0.150000in}}{\pgfqpoint{2.700000in}{1.950000in}}%
\pgfusepath{clip}%
\pgfsetbuttcap%
\pgfsetroundjoin%
\definecolor{currentfill}{rgb}{0.841406,0.860938,0.888281}%
\pgfsetfillcolor{currentfill}%
\pgfsetlinewidth{0.000000pt}%
\definecolor{currentstroke}{rgb}{0.000000,0.000000,0.000000}%
\pgfsetstrokecolor{currentstroke}%
\pgfsetdash{}{0pt}%
\pgfpathmoveto{\pgfqpoint{1.647633in}{1.546787in}}%
\pgfpathlineto{\pgfqpoint{1.684333in}{1.558813in}}%
\pgfpathlineto{\pgfqpoint{1.647173in}{1.576926in}}%
\pgfpathlineto{\pgfqpoint{1.610493in}{1.571124in}}%
\pgfpathclose%
\pgfusepath{fill}%
\end{pgfscope}%
\begin{pgfscope}%
\pgfpathrectangle{\pgfqpoint{0.150000in}{0.150000in}}{\pgfqpoint{2.700000in}{1.950000in}}%
\pgfusepath{clip}%
\pgfsetbuttcap%
\pgfsetroundjoin%
\definecolor{currentfill}{rgb}{0.941115,0.893153,0.896952}%
\pgfsetfillcolor{currentfill}%
\pgfsetlinewidth{0.000000pt}%
\definecolor{currentstroke}{rgb}{0.000000,0.000000,0.000000}%
\pgfsetstrokecolor{currentstroke}%
\pgfsetdash{}{0pt}%
\pgfpathmoveto{\pgfqpoint{2.164727in}{1.217234in}}%
\pgfpathlineto{\pgfqpoint{2.199749in}{1.224057in}}%
\pgfpathlineto{\pgfqpoint{2.166008in}{1.314503in}}%
\pgfpathlineto{\pgfqpoint{2.130377in}{1.301878in}}%
\pgfpathclose%
\pgfusepath{fill}%
\end{pgfscope}%
\begin{pgfscope}%
\pgfpathrectangle{\pgfqpoint{0.150000in}{0.150000in}}{\pgfqpoint{2.700000in}{1.950000in}}%
\pgfusepath{clip}%
\pgfsetbuttcap%
\pgfsetroundjoin%
\definecolor{currentfill}{rgb}{0.735677,0.768229,0.813802}%
\pgfsetfillcolor{currentfill}%
\pgfsetlinewidth{0.000000pt}%
\definecolor{currentstroke}{rgb}{0.000000,0.000000,0.000000}%
\pgfsetstrokecolor{currentstroke}%
\pgfsetdash{}{0pt}%
\pgfpathmoveto{\pgfqpoint{1.314014in}{1.669242in}}%
\pgfpathlineto{\pgfqpoint{1.351737in}{1.668557in}}%
\pgfpathlineto{\pgfqpoint{1.315059in}{1.692680in}}%
\pgfpathlineto{\pgfqpoint{1.277255in}{1.693448in}}%
\pgfpathclose%
\pgfusepath{fill}%
\end{pgfscope}%
\begin{pgfscope}%
\pgfpathrectangle{\pgfqpoint{0.150000in}{0.150000in}}{\pgfqpoint{2.700000in}{1.950000in}}%
\pgfusepath{clip}%
\pgfsetbuttcap%
\pgfsetroundjoin%
\definecolor{currentfill}{rgb}{0.860064,0.877298,0.901425}%
\pgfsetfillcolor{currentfill}%
\pgfsetlinewidth{0.000000pt}%
\definecolor{currentstroke}{rgb}{0.000000,0.000000,0.000000}%
\pgfsetstrokecolor{currentstroke}%
\pgfsetdash{}{0pt}%
\pgfpathmoveto{\pgfqpoint{1.758928in}{1.516243in}}%
\pgfpathlineto{\pgfqpoint{1.795391in}{1.528348in}}%
\pgfpathlineto{\pgfqpoint{1.758007in}{1.546528in}}%
\pgfpathlineto{\pgfqpoint{1.721522in}{1.534487in}}%
\pgfpathclose%
\pgfusepath{fill}%
\end{pgfscope}%
\begin{pgfscope}%
\pgfpathrectangle{\pgfqpoint{0.150000in}{0.150000in}}{\pgfqpoint{2.700000in}{1.950000in}}%
\pgfusepath{clip}%
\pgfsetbuttcap%
\pgfsetroundjoin%
\definecolor{currentfill}{rgb}{0.846140,0.720818,0.730744}%
\pgfsetfillcolor{currentfill}%
\pgfsetlinewidth{0.000000pt}%
\definecolor{currentstroke}{rgb}{0.000000,0.000000,0.000000}%
\pgfsetstrokecolor{currentstroke}%
\pgfsetdash{}{0pt}%
\pgfpathmoveto{\pgfqpoint{2.128417in}{1.039881in}}%
\pgfpathlineto{\pgfqpoint{2.163232in}{1.041380in}}%
\pgfpathlineto{\pgfqpoint{2.128624in}{1.119136in}}%
\pgfpathlineto{\pgfqpoint{2.093240in}{1.111976in}}%
\pgfpathclose%
\pgfusepath{fill}%
\end{pgfscope}%
\begin{pgfscope}%
\pgfpathrectangle{\pgfqpoint{0.150000in}{0.150000in}}{\pgfqpoint{2.700000in}{1.950000in}}%
\pgfusepath{clip}%
\pgfsetbuttcap%
\pgfsetroundjoin%
\definecolor{currentfill}{rgb}{0.891161,0.904565,0.923330}%
\pgfsetfillcolor{currentfill}%
\pgfsetlinewidth{0.000000pt}%
\definecolor{currentstroke}{rgb}{0.000000,0.000000,0.000000}%
\pgfsetstrokecolor{currentstroke}%
\pgfsetdash{}{0pt}%
\pgfpathmoveto{\pgfqpoint{1.944339in}{1.479530in}}%
\pgfpathlineto{\pgfqpoint{1.980369in}{1.491723in}}%
\pgfpathlineto{\pgfqpoint{1.942651in}{1.509971in}}%
\pgfpathlineto{\pgfqpoint{1.906597in}{1.497843in}}%
\pgfpathclose%
\pgfusepath{fill}%
\end{pgfscope}%
\begin{pgfscope}%
\pgfpathrectangle{\pgfqpoint{0.150000in}{0.150000in}}{\pgfqpoint{2.700000in}{1.950000in}}%
\pgfusepath{clip}%
\pgfsetbuttcap%
\pgfsetroundjoin%
\definecolor{currentfill}{rgb}{0.808150,0.651884,0.664262}%
\pgfsetfillcolor{currentfill}%
\pgfsetlinewidth{0.000000pt}%
\definecolor{currentstroke}{rgb}{0.000000,0.000000,0.000000}%
\pgfsetstrokecolor{currentstroke}%
\pgfsetdash{}{0pt}%
\pgfpathmoveto{\pgfqpoint{2.128875in}{0.972328in}}%
\pgfpathlineto{\pgfqpoint{2.163364in}{0.968256in}}%
\pgfpathlineto{\pgfqpoint{2.128417in}{1.039881in}}%
\pgfpathlineto{\pgfqpoint{2.093359in}{1.038372in}}%
\pgfpathclose%
\pgfusepath{fill}%
\end{pgfscope}%
\begin{pgfscope}%
\pgfpathrectangle{\pgfqpoint{0.150000in}{0.150000in}}{\pgfqpoint{2.700000in}{1.950000in}}%
\pgfusepath{clip}%
\pgfsetbuttcap%
\pgfsetroundjoin%
\definecolor{currentfill}{rgb}{0.816529,0.839124,0.870757}%
\pgfsetfillcolor{currentfill}%
\pgfsetlinewidth{0.000000pt}%
\definecolor{currentstroke}{rgb}{0.000000,0.000000,0.000000}%
\pgfsetstrokecolor{currentstroke}%
\pgfsetdash{}{0pt}%
\pgfpathmoveto{\pgfqpoint{1.573577in}{1.559114in}}%
\pgfpathlineto{\pgfqpoint{1.610493in}{1.571124in}}%
\pgfpathlineto{\pgfqpoint{1.573423in}{1.589236in}}%
\pgfpathlineto{\pgfqpoint{1.536486in}{1.583464in}}%
\pgfpathclose%
\pgfusepath{fill}%
\end{pgfscope}%
\begin{pgfscope}%
\pgfpathrectangle{\pgfqpoint{0.150000in}{0.150000in}}{\pgfqpoint{2.700000in}{1.950000in}}%
\pgfusepath{clip}%
\pgfsetbuttcap%
\pgfsetroundjoin%
\definecolor{currentfill}{rgb}{0.884942,0.899112,0.918949}%
\pgfsetfillcolor{currentfill}%
\pgfsetlinewidth{0.000000pt}%
\definecolor{currentstroke}{rgb}{0.000000,0.000000,0.000000}%
\pgfsetstrokecolor{currentstroke}%
\pgfsetdash{}{0pt}%
\pgfpathmoveto{\pgfqpoint{1.870372in}{1.485657in}}%
\pgfpathlineto{\pgfqpoint{1.906597in}{1.497843in}}%
\pgfpathlineto{\pgfqpoint{1.868990in}{1.516090in}}%
\pgfpathlineto{\pgfqpoint{1.832741in}{1.503969in}}%
\pgfpathclose%
\pgfusepath{fill}%
\end{pgfscope}%
\begin{pgfscope}%
\pgfpathrectangle{\pgfqpoint{0.150000in}{0.150000in}}{\pgfqpoint{2.700000in}{1.950000in}}%
\pgfusepath{clip}%
\pgfsetbuttcap%
\pgfsetroundjoin%
\definecolor{currentfill}{rgb}{0.785432,0.811857,0.848851}%
\pgfsetfillcolor{currentfill}%
\pgfsetlinewidth{0.000000pt}%
\definecolor{currentstroke}{rgb}{0.000000,0.000000,0.000000}%
\pgfsetstrokecolor{currentstroke}%
\pgfsetdash{}{0pt}%
\pgfpathmoveto{\pgfqpoint{1.462271in}{1.602025in}}%
\pgfpathlineto{\pgfqpoint{1.499487in}{1.607754in}}%
\pgfpathlineto{\pgfqpoint{1.462620in}{1.625798in}}%
\pgfpathlineto{\pgfqpoint{1.425301in}{1.626333in}}%
\pgfpathclose%
\pgfusepath{fill}%
\end{pgfscope}%
\begin{pgfscope}%
\pgfpathrectangle{\pgfqpoint{0.150000in}{0.150000in}}{\pgfqpoint{2.700000in}{1.950000in}}%
\pgfusepath{clip}%
\pgfsetbuttcap%
\pgfsetroundjoin%
\definecolor{currentfill}{rgb}{0.895527,0.810432,0.817172}%
\pgfsetfillcolor{currentfill}%
\pgfsetlinewidth{0.000000pt}%
\definecolor{currentstroke}{rgb}{0.000000,0.000000,0.000000}%
\pgfsetstrokecolor{currentstroke}%
\pgfsetdash{}{0pt}%
\pgfpathmoveto{\pgfqpoint{2.128624in}{1.119136in}}%
\pgfpathlineto{\pgfqpoint{2.163803in}{1.126253in}}%
\pgfpathlineto{\pgfqpoint{2.129165in}{1.204405in}}%
\pgfpathlineto{\pgfqpoint{2.093750in}{1.197489in}}%
\pgfpathclose%
\pgfusepath{fill}%
\end{pgfscope}%
\begin{pgfscope}%
\pgfpathrectangle{\pgfqpoint{0.150000in}{0.150000in}}{\pgfqpoint{2.700000in}{1.950000in}}%
\pgfusepath{clip}%
\pgfsetbuttcap%
\pgfsetroundjoin%
\definecolor{currentfill}{rgb}{0.847626,0.866391,0.892662}%
\pgfsetfillcolor{currentfill}%
\pgfsetlinewidth{0.000000pt}%
\definecolor{currentstroke}{rgb}{0.000000,0.000000,0.000000}%
\pgfsetstrokecolor{currentstroke}%
\pgfsetdash{}{0pt}%
\pgfpathmoveto{\pgfqpoint{1.684864in}{1.522390in}}%
\pgfpathlineto{\pgfqpoint{1.721522in}{1.534487in}}%
\pgfpathlineto{\pgfqpoint{1.684333in}{1.558813in}}%
\pgfpathlineto{\pgfqpoint{1.647633in}{1.546787in}}%
\pgfpathclose%
\pgfusepath{fill}%
\end{pgfscope}%
\begin{pgfscope}%
\pgfpathrectangle{\pgfqpoint{0.150000in}{0.150000in}}{\pgfqpoint{2.700000in}{1.950000in}}%
\pgfusepath{clip}%
\pgfsetbuttcap%
\pgfsetroundjoin%
\definecolor{currentfill}{rgb}{0.741896,0.773683,0.818183}%
\pgfsetfillcolor{currentfill}%
\pgfsetlinewidth{0.000000pt}%
\definecolor{currentstroke}{rgb}{0.000000,0.000000,0.000000}%
\pgfsetstrokecolor{currentstroke}%
\pgfsetdash{}{0pt}%
\pgfpathmoveto{\pgfqpoint{1.350864in}{1.644976in}}%
\pgfpathlineto{\pgfqpoint{1.388505in}{1.644375in}}%
\pgfpathlineto{\pgfqpoint{1.351737in}{1.668557in}}%
\pgfpathlineto{\pgfqpoint{1.314014in}{1.669242in}}%
\pgfpathclose%
\pgfusepath{fill}%
\end{pgfscope}%
\begin{pgfscope}%
\pgfpathrectangle{\pgfqpoint{0.150000in}{0.150000in}}{\pgfqpoint{2.700000in}{1.950000in}}%
\pgfusepath{clip}%
\pgfsetbuttcap%
\pgfsetroundjoin%
\definecolor{currentfill}{rgb}{0.758762,0.562270,0.577834}%
\pgfsetfillcolor{currentfill}%
\pgfsetlinewidth{0.000000pt}%
\definecolor{currentstroke}{rgb}{0.000000,0.000000,0.000000}%
\pgfsetstrokecolor{currentstroke}%
\pgfsetdash{}{0pt}%
\pgfpathmoveto{\pgfqpoint{2.167877in}{0.891069in}}%
\pgfpathlineto{\pgfqpoint{2.201942in}{0.881539in}}%
\pgfpathlineto{\pgfqpoint{2.164553in}{0.912421in}}%
\pgfpathlineto{\pgfqpoint{2.130003in}{0.916275in}}%
\pgfpathclose%
\pgfusepath{fill}%
\end{pgfscope}%
\begin{pgfscope}%
\pgfpathrectangle{\pgfqpoint{0.150000in}{0.150000in}}{\pgfqpoint{2.700000in}{1.950000in}}%
\pgfusepath{clip}%
\pgfsetbuttcap%
\pgfsetroundjoin%
\definecolor{currentfill}{rgb}{0.866284,0.882751,0.905806}%
\pgfsetfillcolor{currentfill}%
\pgfsetlinewidth{0.000000pt}%
\definecolor{currentstroke}{rgb}{0.000000,0.000000,0.000000}%
\pgfsetstrokecolor{currentstroke}%
\pgfsetdash{}{0pt}%
\pgfpathmoveto{\pgfqpoint{1.796322in}{1.491791in}}%
\pgfpathlineto{\pgfqpoint{1.832741in}{1.503969in}}%
\pgfpathlineto{\pgfqpoint{1.795391in}{1.528348in}}%
\pgfpathlineto{\pgfqpoint{1.758928in}{1.516243in}}%
\pgfpathclose%
\pgfusepath{fill}%
\end{pgfscope}%
\begin{pgfscope}%
\pgfpathrectangle{\pgfqpoint{0.150000in}{0.150000in}}{\pgfqpoint{2.700000in}{1.950000in}}%
\pgfusepath{clip}%
\pgfsetbuttcap%
\pgfsetroundjoin%
\definecolor{currentfill}{rgb}{0.785355,0.610524,0.624372}%
\pgfsetfillcolor{currentfill}%
\pgfsetlinewidth{0.000000pt}%
\definecolor{currentstroke}{rgb}{0.000000,0.000000,0.000000}%
\pgfsetstrokecolor{currentstroke}%
\pgfsetdash{}{0pt}%
\pgfpathmoveto{\pgfqpoint{2.130003in}{0.916275in}}%
\pgfpathlineto{\pgfqpoint{2.164553in}{0.912421in}}%
\pgfpathlineto{\pgfqpoint{2.128875in}{0.972328in}}%
\pgfpathlineto{\pgfqpoint{2.094106in}{0.976434in}}%
\pgfpathclose%
\pgfusepath{fill}%
\end{pgfscope}%
\begin{pgfscope}%
\pgfpathrectangle{\pgfqpoint{0.150000in}{0.150000in}}{\pgfqpoint{2.700000in}{1.950000in}}%
\pgfusepath{clip}%
\pgfsetbuttcap%
\pgfsetroundjoin%
\definecolor{currentfill}{rgb}{0.897381,0.910018,0.927711}%
\pgfsetfillcolor{currentfill}%
\pgfsetlinewidth{0.000000pt}%
\definecolor{currentstroke}{rgb}{0.000000,0.000000,0.000000}%
\pgfsetstrokecolor{currentstroke}%
\pgfsetdash{}{0pt}%
\pgfpathmoveto{\pgfqpoint{1.981966in}{1.455031in}}%
\pgfpathlineto{\pgfqpoint{2.018223in}{1.473410in}}%
\pgfpathlineto{\pgfqpoint{1.980369in}{1.491723in}}%
\pgfpathlineto{\pgfqpoint{1.944339in}{1.479530in}}%
\pgfpathclose%
\pgfusepath{fill}%
\end{pgfscope}%
\begin{pgfscope}%
\pgfpathrectangle{\pgfqpoint{0.150000in}{0.150000in}}{\pgfqpoint{2.700000in}{1.950000in}}%
\pgfusepath{clip}%
\pgfsetbuttcap%
\pgfsetroundjoin%
\definecolor{currentfill}{rgb}{0.747365,0.541590,0.557889}%
\pgfsetfillcolor{currentfill}%
\pgfsetlinewidth{0.000000pt}%
\definecolor{currentstroke}{rgb}{0.000000,0.000000,0.000000}%
\pgfsetstrokecolor{currentstroke}%
\pgfsetdash{}{0pt}%
\pgfpathmoveto{\pgfqpoint{2.205466in}{0.860009in}}%
\pgfpathlineto{\pgfqpoint{2.239381in}{0.850615in}}%
\pgfpathlineto{\pgfqpoint{2.201942in}{0.881539in}}%
\pgfpathlineto{\pgfqpoint{2.167877in}{0.891069in}}%
\pgfpathclose%
\pgfusepath{fill}%
\end{pgfscope}%
\begin{pgfscope}%
\pgfpathrectangle{\pgfqpoint{0.150000in}{0.150000in}}{\pgfqpoint{2.700000in}{1.950000in}}%
\pgfusepath{clip}%
\pgfsetbuttcap%
\pgfsetroundjoin%
\definecolor{currentfill}{rgb}{0.828968,0.850031,0.879519}%
\pgfsetfillcolor{currentfill}%
\pgfsetlinewidth{0.000000pt}%
\definecolor{currentstroke}{rgb}{0.000000,0.000000,0.000000}%
\pgfsetstrokecolor{currentstroke}%
\pgfsetdash{}{0pt}%
\pgfpathmoveto{\pgfqpoint{1.610759in}{1.534705in}}%
\pgfpathlineto{\pgfqpoint{1.647633in}{1.546787in}}%
\pgfpathlineto{\pgfqpoint{1.610493in}{1.571124in}}%
\pgfpathlineto{\pgfqpoint{1.573577in}{1.559114in}}%
\pgfpathclose%
\pgfusepath{fill}%
\end{pgfscope}%
\begin{pgfscope}%
\pgfpathrectangle{\pgfqpoint{0.150000in}{0.150000in}}{\pgfqpoint{2.700000in}{1.950000in}}%
\pgfusepath{clip}%
\pgfsetbuttcap%
\pgfsetroundjoin%
\definecolor{currentfill}{rgb}{0.791651,0.817310,0.853232}%
\pgfsetfillcolor{currentfill}%
\pgfsetlinewidth{0.000000pt}%
\definecolor{currentstroke}{rgb}{0.000000,0.000000,0.000000}%
\pgfsetstrokecolor{currentstroke}%
\pgfsetdash{}{0pt}%
\pgfpathmoveto{\pgfqpoint{1.499333in}{1.577658in}}%
\pgfpathlineto{\pgfqpoint{1.536486in}{1.583464in}}%
\pgfpathlineto{\pgfqpoint{1.499487in}{1.607754in}}%
\pgfpathlineto{\pgfqpoint{1.462271in}{1.602025in}}%
\pgfpathclose%
\pgfusepath{fill}%
\end{pgfscope}%
\begin{pgfscope}%
\pgfpathrectangle{\pgfqpoint{0.150000in}{0.150000in}}{\pgfqpoint{2.700000in}{1.950000in}}%
\pgfusepath{clip}%
\pgfsetbuttcap%
\pgfsetroundjoin%
\definecolor{currentfill}{rgb}{0.948713,0.906939,0.910248}%
\pgfsetfillcolor{currentfill}%
\pgfsetlinewidth{0.000000pt}%
\definecolor{currentstroke}{rgb}{0.000000,0.000000,0.000000}%
\pgfsetstrokecolor{currentstroke}%
\pgfsetdash{}{0pt}%
\pgfpathmoveto{\pgfqpoint{2.129165in}{1.204405in}}%
\pgfpathlineto{\pgfqpoint{2.164727in}{1.217234in}}%
\pgfpathlineto{\pgfqpoint{2.130377in}{1.301878in}}%
\pgfpathlineto{\pgfqpoint{2.094577in}{1.289193in}}%
\pgfpathclose%
\pgfusepath{fill}%
\end{pgfscope}%
\begin{pgfscope}%
\pgfpathrectangle{\pgfqpoint{0.150000in}{0.150000in}}{\pgfqpoint{2.700000in}{1.950000in}}%
\pgfusepath{clip}%
\pgfsetbuttcap%
\pgfsetroundjoin%
\definecolor{currentfill}{rgb}{0.884942,0.899112,0.918949}%
\pgfsetfillcolor{currentfill}%
\pgfsetlinewidth{0.000000pt}%
\definecolor{currentstroke}{rgb}{0.000000,0.000000,0.000000}%
\pgfsetstrokecolor{currentstroke}%
\pgfsetdash{}{0pt}%
\pgfpathmoveto{\pgfqpoint{1.907929in}{1.461152in}}%
\pgfpathlineto{\pgfqpoint{1.944339in}{1.479530in}}%
\pgfpathlineto{\pgfqpoint{1.906597in}{1.497843in}}%
\pgfpathlineto{\pgfqpoint{1.870372in}{1.485657in}}%
\pgfpathclose%
\pgfusepath{fill}%
\end{pgfscope}%
\begin{pgfscope}%
\pgfpathrectangle{\pgfqpoint{0.150000in}{0.150000in}}{\pgfqpoint{2.700000in}{1.950000in}}%
\pgfusepath{clip}%
\pgfsetbuttcap%
\pgfsetroundjoin%
\definecolor{currentfill}{rgb}{0.853845,0.871844,0.897044}%
\pgfsetfillcolor{currentfill}%
\pgfsetlinewidth{0.000000pt}%
\definecolor{currentstroke}{rgb}{0.000000,0.000000,0.000000}%
\pgfsetstrokecolor{currentstroke}%
\pgfsetdash{}{0pt}%
\pgfpathmoveto{\pgfqpoint{1.722293in}{1.504080in}}%
\pgfpathlineto{\pgfqpoint{1.758928in}{1.516243in}}%
\pgfpathlineto{\pgfqpoint{1.721522in}{1.534487in}}%
\pgfpathlineto{\pgfqpoint{1.684864in}{1.522390in}}%
\pgfpathclose%
\pgfusepath{fill}%
\end{pgfscope}%
\begin{pgfscope}%
\pgfpathrectangle{\pgfqpoint{0.150000in}{0.150000in}}{\pgfqpoint{2.700000in}{1.950000in}}%
\pgfusepath{clip}%
\pgfsetbuttcap%
\pgfsetroundjoin%
\definecolor{currentfill}{rgb}{0.754335,0.784589,0.826945}%
\pgfsetfillcolor{currentfill}%
\pgfsetlinewidth{0.000000pt}%
\definecolor{currentstroke}{rgb}{0.000000,0.000000,0.000000}%
\pgfsetstrokecolor{currentstroke}%
\pgfsetdash{}{0pt}%
\pgfpathmoveto{\pgfqpoint{1.387805in}{1.620650in}}%
\pgfpathlineto{\pgfqpoint{1.425301in}{1.626333in}}%
\pgfpathlineto{\pgfqpoint{1.388505in}{1.644375in}}%
\pgfpathlineto{\pgfqpoint{1.350864in}{1.644976in}}%
\pgfpathclose%
\pgfusepath{fill}%
\end{pgfscope}%
\begin{pgfscope}%
\pgfpathrectangle{\pgfqpoint{0.150000in}{0.150000in}}{\pgfqpoint{2.700000in}{1.950000in}}%
\pgfusepath{clip}%
\pgfsetbuttcap%
\pgfsetroundjoin%
\definecolor{currentfill}{rgb}{0.872503,0.888205,0.910187}%
\pgfsetfillcolor{currentfill}%
\pgfsetlinewidth{0.000000pt}%
\definecolor{currentstroke}{rgb}{0.000000,0.000000,0.000000}%
\pgfsetstrokecolor{currentstroke}%
\pgfsetdash{}{0pt}%
\pgfpathmoveto{\pgfqpoint{1.833976in}{1.473414in}}%
\pgfpathlineto{\pgfqpoint{1.870372in}{1.485657in}}%
\pgfpathlineto{\pgfqpoint{1.832741in}{1.503969in}}%
\pgfpathlineto{\pgfqpoint{1.796322in}{1.491791in}}%
\pgfpathclose%
\pgfusepath{fill}%
\end{pgfscope}%
\begin{pgfscope}%
\pgfpathrectangle{\pgfqpoint{0.150000in}{0.150000in}}{\pgfqpoint{2.700000in}{1.950000in}}%
\pgfusepath{clip}%
\pgfsetbuttcap%
\pgfsetroundjoin%
\definecolor{currentfill}{rgb}{0.897381,0.910018,0.927711}%
\pgfsetfillcolor{currentfill}%
\pgfsetlinewidth{0.000000pt}%
\definecolor{currentstroke}{rgb}{0.000000,0.000000,0.000000}%
\pgfsetstrokecolor{currentstroke}%
\pgfsetdash{}{0pt}%
\pgfpathmoveto{\pgfqpoint{2.019959in}{1.436585in}}%
\pgfpathlineto{\pgfqpoint{2.055920in}{1.448917in}}%
\pgfpathlineto{\pgfqpoint{2.018223in}{1.473410in}}%
\pgfpathlineto{\pgfqpoint{1.981966in}{1.455031in}}%
\pgfpathclose%
\pgfusepath{fill}%
\end{pgfscope}%
\begin{pgfscope}%
\pgfpathrectangle{\pgfqpoint{0.150000in}{0.150000in}}{\pgfqpoint{2.700000in}{1.950000in}}%
\pgfusepath{clip}%
\pgfsetbuttcap%
\pgfsetroundjoin%
\definecolor{currentfill}{rgb}{0.704580,0.740962,0.791896}%
\pgfsetfillcolor{currentfill}%
\pgfsetlinewidth{0.000000pt}%
\definecolor{currentstroke}{rgb}{0.000000,0.000000,0.000000}%
\pgfsetstrokecolor{currentstroke}%
\pgfsetdash{}{0pt}%
\pgfpathmoveto{\pgfqpoint{1.276027in}{1.669931in}}%
\pgfpathlineto{\pgfqpoint{1.314014in}{1.669242in}}%
\pgfpathlineto{\pgfqpoint{1.277255in}{1.693448in}}%
\pgfpathlineto{\pgfqpoint{1.239187in}{1.694221in}}%
\pgfpathclose%
\pgfusepath{fill}%
\end{pgfscope}%
\begin{pgfscope}%
\pgfpathrectangle{\pgfqpoint{0.150000in}{0.150000in}}{\pgfqpoint{2.700000in}{1.950000in}}%
\pgfusepath{clip}%
\pgfsetbuttcap%
\pgfsetroundjoin%
\definecolor{currentfill}{rgb}{0.835187,0.855484,0.883900}%
\pgfsetfillcolor{currentfill}%
\pgfsetlinewidth{0.000000pt}%
\definecolor{currentstroke}{rgb}{0.000000,0.000000,0.000000}%
\pgfsetstrokecolor{currentstroke}%
\pgfsetdash{}{0pt}%
\pgfpathmoveto{\pgfqpoint{1.648096in}{1.516396in}}%
\pgfpathlineto{\pgfqpoint{1.684864in}{1.522390in}}%
\pgfpathlineto{\pgfqpoint{1.647633in}{1.546787in}}%
\pgfpathlineto{\pgfqpoint{1.610759in}{1.534705in}}%
\pgfpathclose%
\pgfusepath{fill}%
\end{pgfscope}%
\begin{pgfscope}%
\pgfpathrectangle{\pgfqpoint{0.150000in}{0.150000in}}{\pgfqpoint{2.700000in}{1.950000in}}%
\pgfusepath{clip}%
\pgfsetbuttcap%
\pgfsetroundjoin%
\definecolor{currentfill}{rgb}{0.861336,0.748392,0.757338}%
\pgfsetfillcolor{currentfill}%
\pgfsetlinewidth{0.000000pt}%
\definecolor{currentstroke}{rgb}{0.000000,0.000000,0.000000}%
\pgfsetstrokecolor{currentstroke}%
\pgfsetdash{}{0pt}%
\pgfpathmoveto{\pgfqpoint{2.093359in}{1.038372in}}%
\pgfpathlineto{\pgfqpoint{2.128417in}{1.039881in}}%
\pgfpathlineto{\pgfqpoint{2.093240in}{1.111976in}}%
\pgfpathlineto{\pgfqpoint{2.057943in}{1.110714in}}%
\pgfpathclose%
\pgfusepath{fill}%
\end{pgfscope}%
\begin{pgfscope}%
\pgfpathrectangle{\pgfqpoint{0.150000in}{0.150000in}}{\pgfqpoint{2.700000in}{1.950000in}}%
\pgfusepath{clip}%
\pgfsetbuttcap%
\pgfsetroundjoin%
\definecolor{currentfill}{rgb}{0.797871,0.822763,0.857613}%
\pgfsetfillcolor{currentfill}%
\pgfsetlinewidth{0.000000pt}%
\definecolor{currentstroke}{rgb}{0.000000,0.000000,0.000000}%
\pgfsetstrokecolor{currentstroke}%
\pgfsetdash{}{0pt}%
\pgfpathmoveto{\pgfqpoint{1.536486in}{1.553229in}}%
\pgfpathlineto{\pgfqpoint{1.573577in}{1.559114in}}%
\pgfpathlineto{\pgfqpoint{1.536486in}{1.583464in}}%
\pgfpathlineto{\pgfqpoint{1.499333in}{1.577658in}}%
\pgfpathclose%
\pgfusepath{fill}%
\end{pgfscope}%
\begin{pgfscope}%
\pgfpathrectangle{\pgfqpoint{0.150000in}{0.150000in}}{\pgfqpoint{2.700000in}{1.950000in}}%
\pgfusepath{clip}%
\pgfsetbuttcap%
\pgfsetroundjoin%
\definecolor{currentfill}{rgb}{0.743566,0.534697,0.551241}%
\pgfsetfillcolor{currentfill}%
\pgfsetlinewidth{0.000000pt}%
\definecolor{currentstroke}{rgb}{0.000000,0.000000,0.000000}%
\pgfsetstrokecolor{currentstroke}%
\pgfsetdash{}{0pt}%
\pgfpathmoveto{\pgfqpoint{2.244306in}{0.846260in}}%
\pgfpathlineto{\pgfqpoint{2.278122in}{0.836908in}}%
\pgfpathlineto{\pgfqpoint{2.239381in}{0.850615in}}%
\pgfpathlineto{\pgfqpoint{2.205466in}{0.860009in}}%
\pgfpathclose%
\pgfusepath{fill}%
\end{pgfscope}%
\begin{pgfscope}%
\pgfpathrectangle{\pgfqpoint{0.150000in}{0.150000in}}{\pgfqpoint{2.700000in}{1.950000in}}%
\pgfusepath{clip}%
\pgfsetbuttcap%
\pgfsetroundjoin%
\definecolor{currentfill}{rgb}{0.903125,0.824219,0.830469}%
\pgfsetfillcolor{currentfill}%
\pgfsetlinewidth{0.000000pt}%
\definecolor{currentstroke}{rgb}{0.000000,0.000000,0.000000}%
\pgfsetstrokecolor{currentstroke}%
\pgfsetdash{}{0pt}%
\pgfpathmoveto{\pgfqpoint{2.093240in}{1.111976in}}%
\pgfpathlineto{\pgfqpoint{2.128624in}{1.119136in}}%
\pgfpathlineto{\pgfqpoint{2.093750in}{1.197489in}}%
\pgfpathlineto{\pgfqpoint{2.057832in}{1.184546in}}%
\pgfpathclose%
\pgfusepath{fill}%
\end{pgfscope}%
\begin{pgfscope}%
\pgfpathrectangle{\pgfqpoint{0.150000in}{0.150000in}}{\pgfqpoint{2.700000in}{1.950000in}}%
\pgfusepath{clip}%
\pgfsetbuttcap%
\pgfsetroundjoin%
\definecolor{currentfill}{rgb}{0.760555,0.790043,0.831327}%
\pgfsetfillcolor{currentfill}%
\pgfsetlinewidth{0.000000pt}%
\definecolor{currentstroke}{rgb}{0.000000,0.000000,0.000000}%
\pgfsetstrokecolor{currentstroke}%
\pgfsetdash{}{0pt}%
\pgfpathmoveto{\pgfqpoint{1.424837in}{1.596264in}}%
\pgfpathlineto{\pgfqpoint{1.462271in}{1.602025in}}%
\pgfpathlineto{\pgfqpoint{1.425301in}{1.626333in}}%
\pgfpathlineto{\pgfqpoint{1.387805in}{1.620650in}}%
\pgfpathclose%
\pgfusepath{fill}%
\end{pgfscope}%
\begin{pgfscope}%
\pgfpathrectangle{\pgfqpoint{0.150000in}{0.150000in}}{\pgfqpoint{2.700000in}{1.950000in}}%
\pgfusepath{clip}%
\pgfsetbuttcap%
\pgfsetroundjoin%
\definecolor{currentfill}{rgb}{0.891161,0.904565,0.923330}%
\pgfsetfillcolor{currentfill}%
\pgfsetlinewidth{0.000000pt}%
\definecolor{currentstroke}{rgb}{0.000000,0.000000,0.000000}%
\pgfsetstrokecolor{currentstroke}%
\pgfsetdash{}{0pt}%
\pgfpathmoveto{\pgfqpoint{1.945810in}{1.442707in}}%
\pgfpathlineto{\pgfqpoint{1.981966in}{1.455031in}}%
\pgfpathlineto{\pgfqpoint{1.944339in}{1.479530in}}%
\pgfpathlineto{\pgfqpoint{1.907929in}{1.461152in}}%
\pgfpathclose%
\pgfusepath{fill}%
\end{pgfscope}%
\begin{pgfscope}%
\pgfpathrectangle{\pgfqpoint{0.150000in}{0.150000in}}{\pgfqpoint{2.700000in}{1.950000in}}%
\pgfusepath{clip}%
\pgfsetbuttcap%
\pgfsetroundjoin%
\definecolor{currentfill}{rgb}{0.860064,0.877298,0.901425}%
\pgfsetfillcolor{currentfill}%
\pgfsetlinewidth{0.000000pt}%
\definecolor{currentstroke}{rgb}{0.000000,0.000000,0.000000}%
\pgfsetstrokecolor{currentstroke}%
\pgfsetdash{}{0pt}%
\pgfpathmoveto{\pgfqpoint{1.759730in}{1.479556in}}%
\pgfpathlineto{\pgfqpoint{1.796322in}{1.491791in}}%
\pgfpathlineto{\pgfqpoint{1.758928in}{1.516243in}}%
\pgfpathlineto{\pgfqpoint{1.722293in}{1.504080in}}%
\pgfpathclose%
\pgfusepath{fill}%
\end{pgfscope}%
\begin{pgfscope}%
\pgfpathrectangle{\pgfqpoint{0.150000in}{0.150000in}}{\pgfqpoint{2.700000in}{1.950000in}}%
\pgfusepath{clip}%
\pgfsetbuttcap%
\pgfsetroundjoin%
\definecolor{currentfill}{rgb}{0.830944,0.693244,0.704151}%
\pgfsetfillcolor{currentfill}%
\pgfsetlinewidth{0.000000pt}%
\definecolor{currentstroke}{rgb}{0.000000,0.000000,0.000000}%
\pgfsetstrokecolor{currentstroke}%
\pgfsetdash{}{0pt}%
\pgfpathmoveto{\pgfqpoint{2.094106in}{0.976434in}}%
\pgfpathlineto{\pgfqpoint{2.128875in}{0.972328in}}%
\pgfpathlineto{\pgfqpoint{2.093359in}{1.038372in}}%
\pgfpathlineto{\pgfqpoint{2.058350in}{1.042757in}}%
\pgfpathclose%
\pgfusepath{fill}%
\end{pgfscope}%
\begin{pgfscope}%
\pgfpathrectangle{\pgfqpoint{0.150000in}{0.150000in}}{\pgfqpoint{2.700000in}{1.950000in}}%
\pgfusepath{clip}%
\pgfsetbuttcap%
\pgfsetroundjoin%
\definecolor{currentfill}{rgb}{0.978232,0.980913,0.984666}%
\pgfsetfillcolor{currentfill}%
\pgfsetlinewidth{0.000000pt}%
\definecolor{currentstroke}{rgb}{0.000000,0.000000,0.000000}%
\pgfsetstrokecolor{currentstroke}%
\pgfsetdash{}{0pt}%
\pgfpathmoveto{\pgfqpoint{2.130377in}{1.301878in}}%
\pgfpathlineto{\pgfqpoint{2.166008in}{1.314503in}}%
\pgfpathlineto{\pgfqpoint{2.131929in}{1.405850in}}%
\pgfpathlineto{\pgfqpoint{2.095722in}{1.387270in}}%
\pgfpathclose%
\pgfusepath{fill}%
\end{pgfscope}%
\begin{pgfscope}%
\pgfpathrectangle{\pgfqpoint{0.150000in}{0.150000in}}{\pgfqpoint{2.700000in}{1.950000in}}%
\pgfusepath{clip}%
\pgfsetbuttcap%
\pgfsetroundjoin%
\definecolor{currentfill}{rgb}{0.878722,0.893658,0.914568}%
\pgfsetfillcolor{currentfill}%
\pgfsetlinewidth{0.000000pt}%
\definecolor{currentstroke}{rgb}{0.000000,0.000000,0.000000}%
\pgfsetstrokecolor{currentstroke}%
\pgfsetdash{}{0pt}%
\pgfpathmoveto{\pgfqpoint{1.871577in}{1.448835in}}%
\pgfpathlineto{\pgfqpoint{1.907929in}{1.461152in}}%
\pgfpathlineto{\pgfqpoint{1.870372in}{1.485657in}}%
\pgfpathlineto{\pgfqpoint{1.833976in}{1.473414in}}%
\pgfpathclose%
\pgfusepath{fill}%
\end{pgfscope}%
\begin{pgfscope}%
\pgfpathrectangle{\pgfqpoint{0.150000in}{0.150000in}}{\pgfqpoint{2.700000in}{1.950000in}}%
\pgfusepath{clip}%
\pgfsetbuttcap%
\pgfsetroundjoin%
\definecolor{currentfill}{rgb}{0.717019,0.751869,0.800659}%
\pgfsetfillcolor{currentfill}%
\pgfsetlinewidth{0.000000pt}%
\definecolor{currentstroke}{rgb}{0.000000,0.000000,0.000000}%
\pgfsetstrokecolor{currentstroke}%
\pgfsetdash{}{0pt}%
\pgfpathmoveto{\pgfqpoint{1.312959in}{1.645581in}}%
\pgfpathlineto{\pgfqpoint{1.350864in}{1.644976in}}%
\pgfpathlineto{\pgfqpoint{1.314014in}{1.669242in}}%
\pgfpathlineto{\pgfqpoint{1.276027in}{1.669931in}}%
\pgfpathclose%
\pgfusepath{fill}%
\end{pgfscope}%
\begin{pgfscope}%
\pgfpathrectangle{\pgfqpoint{0.150000in}{0.150000in}}{\pgfqpoint{2.700000in}{1.950000in}}%
\pgfusepath{clip}%
\pgfsetbuttcap%
\pgfsetroundjoin%
\definecolor{currentfill}{rgb}{0.810309,0.833670,0.866376}%
\pgfsetfillcolor{currentfill}%
\pgfsetlinewidth{0.000000pt}%
\definecolor{currentstroke}{rgb}{0.000000,0.000000,0.000000}%
\pgfsetstrokecolor{currentstroke}%
\pgfsetdash{}{0pt}%
\pgfpathmoveto{\pgfqpoint{1.573732in}{1.528741in}}%
\pgfpathlineto{\pgfqpoint{1.610759in}{1.534705in}}%
\pgfpathlineto{\pgfqpoint{1.573577in}{1.559114in}}%
\pgfpathlineto{\pgfqpoint{1.536486in}{1.553229in}}%
\pgfpathclose%
\pgfusepath{fill}%
\end{pgfscope}%
\begin{pgfscope}%
\pgfpathrectangle{\pgfqpoint{0.150000in}{0.150000in}}{\pgfqpoint{2.700000in}{1.950000in}}%
\pgfusepath{clip}%
\pgfsetbuttcap%
\pgfsetroundjoin%
\definecolor{currentfill}{rgb}{0.956311,0.920726,0.923545}%
\pgfsetfillcolor{currentfill}%
\pgfsetlinewidth{0.000000pt}%
\definecolor{currentstroke}{rgb}{0.000000,0.000000,0.000000}%
\pgfsetstrokecolor{currentstroke}%
\pgfsetdash{}{0pt}%
\pgfpathmoveto{\pgfqpoint{2.093750in}{1.197489in}}%
\pgfpathlineto{\pgfqpoint{2.129165in}{1.204405in}}%
\pgfpathlineto{\pgfqpoint{2.094577in}{1.289193in}}%
\pgfpathlineto{\pgfqpoint{2.058312in}{1.270407in}}%
\pgfpathclose%
\pgfusepath{fill}%
\end{pgfscope}%
\begin{pgfscope}%
\pgfpathrectangle{\pgfqpoint{0.150000in}{0.150000in}}{\pgfqpoint{2.700000in}{1.950000in}}%
\pgfusepath{clip}%
\pgfsetbuttcap%
\pgfsetroundjoin%
\definecolor{currentfill}{rgb}{0.841406,0.860938,0.888281}%
\pgfsetfillcolor{currentfill}%
\pgfsetlinewidth{0.000000pt}%
\definecolor{currentstroke}{rgb}{0.000000,0.000000,0.000000}%
\pgfsetstrokecolor{currentstroke}%
\pgfsetdash{}{0pt}%
\pgfpathmoveto{\pgfqpoint{1.685484in}{1.491860in}}%
\pgfpathlineto{\pgfqpoint{1.722293in}{1.504080in}}%
\pgfpathlineto{\pgfqpoint{1.684864in}{1.522390in}}%
\pgfpathlineto{\pgfqpoint{1.648096in}{1.516396in}}%
\pgfpathclose%
\pgfusepath{fill}%
\end{pgfscope}%
\begin{pgfscope}%
\pgfpathrectangle{\pgfqpoint{0.150000in}{0.150000in}}{\pgfqpoint{2.700000in}{1.950000in}}%
\pgfusepath{clip}%
\pgfsetbuttcap%
\pgfsetroundjoin%
\definecolor{currentfill}{rgb}{0.903600,0.915472,0.932093}%
\pgfsetfillcolor{currentfill}%
\pgfsetlinewidth{0.000000pt}%
\definecolor{currentstroke}{rgb}{0.000000,0.000000,0.000000}%
\pgfsetstrokecolor{currentstroke}%
\pgfsetdash{}{0pt}%
\pgfpathmoveto{\pgfqpoint{2.058089in}{1.418073in}}%
\pgfpathlineto{\pgfqpoint{2.094024in}{1.430471in}}%
\pgfpathlineto{\pgfqpoint{2.055920in}{1.448917in}}%
\pgfpathlineto{\pgfqpoint{2.019959in}{1.436585in}}%
\pgfpathclose%
\pgfusepath{fill}%
\end{pgfscope}%
\begin{pgfscope}%
\pgfpathrectangle{\pgfqpoint{0.150000in}{0.150000in}}{\pgfqpoint{2.700000in}{1.950000in}}%
\pgfusepath{clip}%
\pgfsetbuttcap%
\pgfsetroundjoin%
\definecolor{currentfill}{rgb}{0.772993,0.800950,0.840089}%
\pgfsetfillcolor{currentfill}%
\pgfsetlinewidth{0.000000pt}%
\definecolor{currentstroke}{rgb}{0.000000,0.000000,0.000000}%
\pgfsetstrokecolor{currentstroke}%
\pgfsetdash{}{0pt}%
\pgfpathmoveto{\pgfqpoint{1.461961in}{1.571817in}}%
\pgfpathlineto{\pgfqpoint{1.499333in}{1.577658in}}%
\pgfpathlineto{\pgfqpoint{1.462271in}{1.602025in}}%
\pgfpathlineto{\pgfqpoint{1.424837in}{1.596264in}}%
\pgfpathclose%
\pgfusepath{fill}%
\end{pgfscope}%
\begin{pgfscope}%
\pgfpathrectangle{\pgfqpoint{0.150000in}{0.150000in}}{\pgfqpoint{2.700000in}{1.950000in}}%
\pgfusepath{clip}%
\pgfsetbuttcap%
\pgfsetroundjoin%
\definecolor{currentfill}{rgb}{0.781556,0.603631,0.617724}%
\pgfsetfillcolor{currentfill}%
\pgfsetlinewidth{0.000000pt}%
\definecolor{currentstroke}{rgb}{0.000000,0.000000,0.000000}%
\pgfsetstrokecolor{currentstroke}%
\pgfsetdash{}{0pt}%
\pgfpathmoveto{\pgfqpoint{2.133157in}{0.894856in}}%
\pgfpathlineto{\pgfqpoint{2.167877in}{0.891069in}}%
\pgfpathlineto{\pgfqpoint{2.130003in}{0.916275in}}%
\pgfpathlineto{\pgfqpoint{2.094856in}{0.914329in}}%
\pgfpathclose%
\pgfusepath{fill}%
\end{pgfscope}%
\begin{pgfscope}%
\pgfpathrectangle{\pgfqpoint{0.150000in}{0.150000in}}{\pgfqpoint{2.700000in}{1.950000in}}%
\pgfusepath{clip}%
\pgfsetbuttcap%
\pgfsetroundjoin%
\definecolor{currentfill}{rgb}{0.860064,0.877298,0.901425}%
\pgfsetfillcolor{currentfill}%
\pgfsetlinewidth{0.000000pt}%
\definecolor{currentstroke}{rgb}{0.000000,0.000000,0.000000}%
\pgfsetstrokecolor{currentstroke}%
\pgfsetdash{}{0pt}%
\pgfpathmoveto{\pgfqpoint{1.797408in}{1.461113in}}%
\pgfpathlineto{\pgfqpoint{1.833976in}{1.473414in}}%
\pgfpathlineto{\pgfqpoint{1.796322in}{1.491791in}}%
\pgfpathlineto{\pgfqpoint{1.759730in}{1.479556in}}%
\pgfpathclose%
\pgfusepath{fill}%
\end{pgfscope}%
\begin{pgfscope}%
\pgfpathrectangle{\pgfqpoint{0.150000in}{0.150000in}}{\pgfqpoint{2.700000in}{1.950000in}}%
\pgfusepath{clip}%
\pgfsetbuttcap%
\pgfsetroundjoin%
\definecolor{currentfill}{rgb}{0.891161,0.904565,0.923330}%
\pgfsetfillcolor{currentfill}%
\pgfsetlinewidth{0.000000pt}%
\definecolor{currentstroke}{rgb}{0.000000,0.000000,0.000000}%
\pgfsetstrokecolor{currentstroke}%
\pgfsetdash{}{0pt}%
\pgfpathmoveto{\pgfqpoint{1.983827in}{1.424195in}}%
\pgfpathlineto{\pgfqpoint{2.019959in}{1.436585in}}%
\pgfpathlineto{\pgfqpoint{1.981966in}{1.455031in}}%
\pgfpathlineto{\pgfqpoint{1.945810in}{1.442707in}}%
\pgfpathclose%
\pgfusepath{fill}%
\end{pgfscope}%
\begin{pgfscope}%
\pgfpathrectangle{\pgfqpoint{0.150000in}{0.150000in}}{\pgfqpoint{2.700000in}{1.950000in}}%
\pgfusepath{clip}%
\pgfsetbuttcap%
\pgfsetroundjoin%
\definecolor{currentfill}{rgb}{0.808150,0.651884,0.664262}%
\pgfsetfillcolor{currentfill}%
\pgfsetlinewidth{0.000000pt}%
\definecolor{currentstroke}{rgb}{0.000000,0.000000,0.000000}%
\pgfsetstrokecolor{currentstroke}%
\pgfsetdash{}{0pt}%
\pgfpathmoveto{\pgfqpoint{2.094856in}{0.914329in}}%
\pgfpathlineto{\pgfqpoint{2.130003in}{0.916275in}}%
\pgfpathlineto{\pgfqpoint{2.094106in}{0.976434in}}%
\pgfpathlineto{\pgfqpoint{2.059054in}{0.980573in}}%
\pgfpathclose%
\pgfusepath{fill}%
\end{pgfscope}%
\begin{pgfscope}%
\pgfpathrectangle{\pgfqpoint{0.150000in}{0.150000in}}{\pgfqpoint{2.700000in}{1.950000in}}%
\pgfusepath{clip}%
\pgfsetbuttcap%
\pgfsetroundjoin%
\definecolor{currentfill}{rgb}{0.723238,0.757322,0.805040}%
\pgfsetfillcolor{currentfill}%
\pgfsetlinewidth{0.000000pt}%
\definecolor{currentstroke}{rgb}{0.000000,0.000000,0.000000}%
\pgfsetstrokecolor{currentstroke}%
\pgfsetdash{}{0pt}%
\pgfpathmoveto{\pgfqpoint{1.349983in}{1.621170in}}%
\pgfpathlineto{\pgfqpoint{1.387805in}{1.620650in}}%
\pgfpathlineto{\pgfqpoint{1.350864in}{1.644976in}}%
\pgfpathlineto{\pgfqpoint{1.312959in}{1.645581in}}%
\pgfpathclose%
\pgfusepath{fill}%
\end{pgfscope}%
\begin{pgfscope}%
\pgfpathrectangle{\pgfqpoint{0.150000in}{0.150000in}}{\pgfqpoint{2.700000in}{1.950000in}}%
\pgfusepath{clip}%
\pgfsetbuttcap%
\pgfsetroundjoin%
\definecolor{currentfill}{rgb}{0.816529,0.839124,0.870757}%
\pgfsetfillcolor{currentfill}%
\pgfsetlinewidth{0.000000pt}%
\definecolor{currentstroke}{rgb}{0.000000,0.000000,0.000000}%
\pgfsetstrokecolor{currentstroke}%
\pgfsetdash{}{0pt}%
\pgfpathmoveto{\pgfqpoint{1.611070in}{1.504191in}}%
\pgfpathlineto{\pgfqpoint{1.648096in}{1.516396in}}%
\pgfpathlineto{\pgfqpoint{1.610759in}{1.534705in}}%
\pgfpathlineto{\pgfqpoint{1.573732in}{1.528741in}}%
\pgfpathclose%
\pgfusepath{fill}%
\end{pgfscope}%
\begin{pgfscope}%
\pgfpathrectangle{\pgfqpoint{0.150000in}{0.150000in}}{\pgfqpoint{2.700000in}{1.950000in}}%
\pgfusepath{clip}%
\pgfsetbuttcap%
\pgfsetroundjoin%
\definecolor{currentfill}{rgb}{0.884942,0.899112,0.918949}%
\pgfsetfillcolor{currentfill}%
\pgfsetlinewidth{0.000000pt}%
\definecolor{currentstroke}{rgb}{0.000000,0.000000,0.000000}%
\pgfsetstrokecolor{currentstroke}%
\pgfsetdash{}{0pt}%
\pgfpathmoveto{\pgfqpoint{1.909482in}{1.430324in}}%
\pgfpathlineto{\pgfqpoint{1.945810in}{1.442707in}}%
\pgfpathlineto{\pgfqpoint{1.907929in}{1.461152in}}%
\pgfpathlineto{\pgfqpoint{1.871577in}{1.448835in}}%
\pgfpathclose%
\pgfusepath{fill}%
\end{pgfscope}%
\begin{pgfscope}%
\pgfpathrectangle{\pgfqpoint{0.150000in}{0.150000in}}{\pgfqpoint{2.700000in}{1.950000in}}%
\pgfusepath{clip}%
\pgfsetbuttcap%
\pgfsetroundjoin%
\definecolor{currentfill}{rgb}{0.847626,0.866391,0.892662}%
\pgfsetfillcolor{currentfill}%
\pgfsetlinewidth{0.000000pt}%
\definecolor{currentstroke}{rgb}{0.000000,0.000000,0.000000}%
\pgfsetstrokecolor{currentstroke}%
\pgfsetdash{}{0pt}%
\pgfpathmoveto{\pgfqpoint{1.722964in}{1.467262in}}%
\pgfpathlineto{\pgfqpoint{1.759730in}{1.479556in}}%
\pgfpathlineto{\pgfqpoint{1.722293in}{1.504080in}}%
\pgfpathlineto{\pgfqpoint{1.685484in}{1.491860in}}%
\pgfpathclose%
\pgfusepath{fill}%
\end{pgfscope}%
\begin{pgfscope}%
\pgfpathrectangle{\pgfqpoint{0.150000in}{0.150000in}}{\pgfqpoint{2.700000in}{1.950000in}}%
\pgfusepath{clip}%
\pgfsetbuttcap%
\pgfsetroundjoin%
\definecolor{currentfill}{rgb}{0.779213,0.806403,0.844470}%
\pgfsetfillcolor{currentfill}%
\pgfsetlinewidth{0.000000pt}%
\definecolor{currentstroke}{rgb}{0.000000,0.000000,0.000000}%
\pgfsetstrokecolor{currentstroke}%
\pgfsetdash{}{0pt}%
\pgfpathmoveto{\pgfqpoint{1.499178in}{1.547310in}}%
\pgfpathlineto{\pgfqpoint{1.536486in}{1.553229in}}%
\pgfpathlineto{\pgfqpoint{1.499333in}{1.577658in}}%
\pgfpathlineto{\pgfqpoint{1.461961in}{1.571817in}}%
\pgfpathclose%
\pgfusepath{fill}%
\end{pgfscope}%
\begin{pgfscope}%
\pgfpathrectangle{\pgfqpoint{0.150000in}{0.150000in}}{\pgfqpoint{2.700000in}{1.950000in}}%
\pgfusepath{clip}%
\pgfsetbuttcap%
\pgfsetroundjoin%
\definecolor{currentfill}{rgb}{0.909819,0.920925,0.936474}%
\pgfsetfillcolor{currentfill}%
\pgfsetlinewidth{0.000000pt}%
\definecolor{currentstroke}{rgb}{0.000000,0.000000,0.000000}%
\pgfsetstrokecolor{currentstroke}%
\pgfsetdash{}{0pt}%
\pgfpathmoveto{\pgfqpoint{2.095722in}{1.387270in}}%
\pgfpathlineto{\pgfqpoint{2.131929in}{1.405850in}}%
\pgfpathlineto{\pgfqpoint{2.094024in}{1.430471in}}%
\pgfpathlineto{\pgfqpoint{2.058089in}{1.418073in}}%
\pgfpathclose%
\pgfusepath{fill}%
\end{pgfscope}%
\begin{pgfscope}%
\pgfpathrectangle{\pgfqpoint{0.150000in}{0.150000in}}{\pgfqpoint{2.700000in}{1.950000in}}%
\pgfusepath{clip}%
\pgfsetbuttcap%
\pgfsetroundjoin%
\definecolor{currentfill}{rgb}{0.777757,0.596737,0.611075}%
\pgfsetfillcolor{currentfill}%
\pgfsetlinewidth{0.000000pt}%
\definecolor{currentstroke}{rgb}{0.000000,0.000000,0.000000}%
\pgfsetstrokecolor{currentstroke}%
\pgfsetdash{}{0pt}%
\pgfpathmoveto{\pgfqpoint{2.171236in}{0.869489in}}%
\pgfpathlineto{\pgfqpoint{2.205466in}{0.860009in}}%
\pgfpathlineto{\pgfqpoint{2.167877in}{0.891069in}}%
\pgfpathlineto{\pgfqpoint{2.133157in}{0.894856in}}%
\pgfpathclose%
\pgfusepath{fill}%
\end{pgfscope}%
\begin{pgfscope}%
\pgfpathrectangle{\pgfqpoint{0.150000in}{0.150000in}}{\pgfqpoint{2.700000in}{1.950000in}}%
\pgfusepath{clip}%
\pgfsetbuttcap%
\pgfsetroundjoin%
\definecolor{currentfill}{rgb}{0.918321,0.851792,0.857062}%
\pgfsetfillcolor{currentfill}%
\pgfsetlinewidth{0.000000pt}%
\definecolor{currentstroke}{rgb}{0.000000,0.000000,0.000000}%
\pgfsetstrokecolor{currentstroke}%
\pgfsetdash{}{0pt}%
\pgfpathmoveto{\pgfqpoint{2.057943in}{1.110714in}}%
\pgfpathlineto{\pgfqpoint{2.093240in}{1.111976in}}%
\pgfpathlineto{\pgfqpoint{2.057832in}{1.184546in}}%
\pgfpathlineto{\pgfqpoint{2.022019in}{1.177534in}}%
\pgfpathclose%
\pgfusepath{fill}%
\end{pgfscope}%
\begin{pgfscope}%
\pgfpathrectangle{\pgfqpoint{0.150000in}{0.150000in}}{\pgfqpoint{2.700000in}{1.950000in}}%
\pgfusepath{clip}%
\pgfsetbuttcap%
\pgfsetroundjoin%
\definecolor{currentfill}{rgb}{0.972013,0.975460,0.980285}%
\pgfsetfillcolor{currentfill}%
\pgfsetlinewidth{0.000000pt}%
\definecolor{currentstroke}{rgb}{0.000000,0.000000,0.000000}%
\pgfsetstrokecolor{currentstroke}%
\pgfsetdash{}{0pt}%
\pgfpathmoveto{\pgfqpoint{2.094577in}{1.289193in}}%
\pgfpathlineto{\pgfqpoint{2.130377in}{1.301878in}}%
\pgfpathlineto{\pgfqpoint{2.095722in}{1.387270in}}%
\pgfpathlineto{\pgfqpoint{2.059385in}{1.368623in}}%
\pgfpathclose%
\pgfusepath{fill}%
\end{pgfscope}%
\begin{pgfscope}%
\pgfpathrectangle{\pgfqpoint{0.150000in}{0.150000in}}{\pgfqpoint{2.700000in}{1.950000in}}%
\pgfusepath{clip}%
\pgfsetbuttcap%
\pgfsetroundjoin%
\definecolor{currentfill}{rgb}{0.880331,0.782858,0.790579}%
\pgfsetfillcolor{currentfill}%
\pgfsetlinewidth{0.000000pt}%
\definecolor{currentstroke}{rgb}{0.000000,0.000000,0.000000}%
\pgfsetstrokecolor{currentstroke}%
\pgfsetdash{}{0pt}%
\pgfpathmoveto{\pgfqpoint{2.058350in}{1.042757in}}%
\pgfpathlineto{\pgfqpoint{2.093359in}{1.038372in}}%
\pgfpathlineto{\pgfqpoint{2.057943in}{1.110714in}}%
\pgfpathlineto{\pgfqpoint{2.022123in}{1.103484in}}%
\pgfpathclose%
\pgfusepath{fill}%
\end{pgfscope}%
\begin{pgfscope}%
\pgfpathrectangle{\pgfqpoint{0.150000in}{0.150000in}}{\pgfqpoint{2.700000in}{1.950000in}}%
\pgfusepath{clip}%
\pgfsetbuttcap%
\pgfsetroundjoin%
\definecolor{currentfill}{rgb}{0.735677,0.768229,0.813802}%
\pgfsetfillcolor{currentfill}%
\pgfsetlinewidth{0.000000pt}%
\definecolor{currentstroke}{rgb}{0.000000,0.000000,0.000000}%
\pgfsetstrokecolor{currentstroke}%
\pgfsetdash{}{0pt}%
\pgfpathmoveto{\pgfqpoint{1.387183in}{1.590468in}}%
\pgfpathlineto{\pgfqpoint{1.424837in}{1.596264in}}%
\pgfpathlineto{\pgfqpoint{1.387805in}{1.620650in}}%
\pgfpathlineto{\pgfqpoint{1.349983in}{1.621170in}}%
\pgfpathclose%
\pgfusepath{fill}%
\end{pgfscope}%
\begin{pgfscope}%
\pgfpathrectangle{\pgfqpoint{0.150000in}{0.150000in}}{\pgfqpoint{2.700000in}{1.950000in}}%
\pgfusepath{clip}%
\pgfsetbuttcap%
\pgfsetroundjoin%
\definecolor{currentfill}{rgb}{0.866284,0.882751,0.905806}%
\pgfsetfillcolor{currentfill}%
\pgfsetlinewidth{0.000000pt}%
\definecolor{currentstroke}{rgb}{0.000000,0.000000,0.000000}%
\pgfsetstrokecolor{currentstroke}%
\pgfsetdash{}{0pt}%
\pgfpathmoveto{\pgfqpoint{1.835052in}{1.436460in}}%
\pgfpathlineto{\pgfqpoint{1.871577in}{1.448835in}}%
\pgfpathlineto{\pgfqpoint{1.833976in}{1.473414in}}%
\pgfpathlineto{\pgfqpoint{1.797408in}{1.461113in}}%
\pgfpathclose%
\pgfusepath{fill}%
\end{pgfscope}%
\begin{pgfscope}%
\pgfpathrectangle{\pgfqpoint{0.150000in}{0.150000in}}{\pgfqpoint{2.700000in}{1.950000in}}%
\pgfusepath{clip}%
\pgfsetbuttcap%
\pgfsetroundjoin%
\definecolor{currentfill}{rgb}{0.963909,0.934513,0.936841}%
\pgfsetfillcolor{currentfill}%
\pgfsetlinewidth{0.000000pt}%
\definecolor{currentstroke}{rgb}{0.000000,0.000000,0.000000}%
\pgfsetstrokecolor{currentstroke}%
\pgfsetdash{}{0pt}%
\pgfpathmoveto{\pgfqpoint{2.057832in}{1.184546in}}%
\pgfpathlineto{\pgfqpoint{2.093750in}{1.197489in}}%
\pgfpathlineto{\pgfqpoint{2.058312in}{1.270407in}}%
\pgfpathlineto{\pgfqpoint{2.022191in}{1.257594in}}%
\pgfpathclose%
\pgfusepath{fill}%
\end{pgfscope}%
\begin{pgfscope}%
\pgfpathrectangle{\pgfqpoint{0.150000in}{0.150000in}}{\pgfqpoint{2.700000in}{1.950000in}}%
\pgfusepath{clip}%
\pgfsetbuttcap%
\pgfsetroundjoin%
\definecolor{currentfill}{rgb}{0.667264,0.708241,0.765610}%
\pgfsetfillcolor{currentfill}%
\pgfsetlinewidth{0.000000pt}%
\definecolor{currentstroke}{rgb}{0.000000,0.000000,0.000000}%
\pgfsetstrokecolor{currentstroke}%
\pgfsetdash{}{0pt}%
\pgfpathmoveto{\pgfqpoint{1.237604in}{1.676904in}}%
\pgfpathlineto{\pgfqpoint{1.276027in}{1.669931in}}%
\pgfpathlineto{\pgfqpoint{1.239187in}{1.694221in}}%
\pgfpathlineto{\pgfqpoint{1.200661in}{1.701284in}}%
\pgfpathclose%
\pgfusepath{fill}%
\end{pgfscope}%
\begin{pgfscope}%
\pgfpathrectangle{\pgfqpoint{0.150000in}{0.150000in}}{\pgfqpoint{2.700000in}{1.950000in}}%
\pgfusepath{clip}%
\pgfsetbuttcap%
\pgfsetroundjoin%
\definecolor{currentfill}{rgb}{0.897381,0.910018,0.927711}%
\pgfsetfillcolor{currentfill}%
\pgfsetlinewidth{0.000000pt}%
\definecolor{currentstroke}{rgb}{0.000000,0.000000,0.000000}%
\pgfsetstrokecolor{currentstroke}%
\pgfsetdash{}{0pt}%
\pgfpathmoveto{\pgfqpoint{2.021707in}{1.399494in}}%
\pgfpathlineto{\pgfqpoint{2.058089in}{1.418073in}}%
\pgfpathlineto{\pgfqpoint{2.019959in}{1.436585in}}%
\pgfpathlineto{\pgfqpoint{1.983827in}{1.424195in}}%
\pgfpathclose%
\pgfusepath{fill}%
\end{pgfscope}%
\begin{pgfscope}%
\pgfpathrectangle{\pgfqpoint{0.150000in}{0.150000in}}{\pgfqpoint{2.700000in}{1.950000in}}%
\pgfusepath{clip}%
\pgfsetbuttcap%
\pgfsetroundjoin%
\definecolor{currentfill}{rgb}{0.828968,0.850031,0.879519}%
\pgfsetfillcolor{currentfill}%
\pgfsetlinewidth{0.000000pt}%
\definecolor{currentstroke}{rgb}{0.000000,0.000000,0.000000}%
\pgfsetstrokecolor{currentstroke}%
\pgfsetdash{}{0pt}%
\pgfpathmoveto{\pgfqpoint{1.648500in}{1.479581in}}%
\pgfpathlineto{\pgfqpoint{1.685484in}{1.491860in}}%
\pgfpathlineto{\pgfqpoint{1.648096in}{1.516396in}}%
\pgfpathlineto{\pgfqpoint{1.611070in}{1.504191in}}%
\pgfpathclose%
\pgfusepath{fill}%
\end{pgfscope}%
\begin{pgfscope}%
\pgfpathrectangle{\pgfqpoint{0.150000in}{0.150000in}}{\pgfqpoint{2.700000in}{1.950000in}}%
\pgfusepath{clip}%
\pgfsetbuttcap%
\pgfsetroundjoin%
\definecolor{currentfill}{rgb}{0.791651,0.817310,0.853232}%
\pgfsetfillcolor{currentfill}%
\pgfsetlinewidth{0.000000pt}%
\definecolor{currentstroke}{rgb}{0.000000,0.000000,0.000000}%
\pgfsetstrokecolor{currentstroke}%
\pgfsetdash{}{0pt}%
\pgfpathmoveto{\pgfqpoint{1.536486in}{1.522742in}}%
\pgfpathlineto{\pgfqpoint{1.573732in}{1.528741in}}%
\pgfpathlineto{\pgfqpoint{1.536486in}{1.553229in}}%
\pgfpathlineto{\pgfqpoint{1.499178in}{1.547310in}}%
\pgfpathclose%
\pgfusepath{fill}%
\end{pgfscope}%
\begin{pgfscope}%
\pgfpathrectangle{\pgfqpoint{0.150000in}{0.150000in}}{\pgfqpoint{2.700000in}{1.950000in}}%
\pgfusepath{clip}%
\pgfsetbuttcap%
\pgfsetroundjoin%
\definecolor{currentfill}{rgb}{0.884942,0.899112,0.918949}%
\pgfsetfillcolor{currentfill}%
\pgfsetlinewidth{0.000000pt}%
\definecolor{currentstroke}{rgb}{0.000000,0.000000,0.000000}%
\pgfsetstrokecolor{currentstroke}%
\pgfsetdash{}{0pt}%
\pgfpathmoveto{\pgfqpoint{1.947291in}{1.405617in}}%
\pgfpathlineto{\pgfqpoint{1.983827in}{1.424195in}}%
\pgfpathlineto{\pgfqpoint{1.945810in}{1.442707in}}%
\pgfpathlineto{\pgfqpoint{1.909482in}{1.430324in}}%
\pgfpathclose%
\pgfusepath{fill}%
\end{pgfscope}%
\begin{pgfscope}%
\pgfpathrectangle{\pgfqpoint{0.150000in}{0.150000in}}{\pgfqpoint{2.700000in}{1.950000in}}%
\pgfusepath{clip}%
\pgfsetbuttcap%
\pgfsetroundjoin%
\definecolor{currentfill}{rgb}{0.853845,0.871844,0.897044}%
\pgfsetfillcolor{currentfill}%
\pgfsetlinewidth{0.000000pt}%
\definecolor{currentstroke}{rgb}{0.000000,0.000000,0.000000}%
\pgfsetstrokecolor{currentstroke}%
\pgfsetdash{}{0pt}%
\pgfpathmoveto{\pgfqpoint{1.760665in}{1.448753in}}%
\pgfpathlineto{\pgfqpoint{1.797408in}{1.461113in}}%
\pgfpathlineto{\pgfqpoint{1.759730in}{1.479556in}}%
\pgfpathlineto{\pgfqpoint{1.722964in}{1.467262in}}%
\pgfpathclose%
\pgfusepath{fill}%
\end{pgfscope}%
\begin{pgfscope}%
\pgfpathrectangle{\pgfqpoint{0.150000in}{0.150000in}}{\pgfqpoint{2.700000in}{1.950000in}}%
\pgfusepath{clip}%
\pgfsetbuttcap%
\pgfsetroundjoin%
\definecolor{currentfill}{rgb}{0.773958,0.589844,0.604427}%
\pgfsetfillcolor{currentfill}%
\pgfsetlinewidth{0.000000pt}%
\definecolor{currentstroke}{rgb}{0.000000,0.000000,0.000000}%
\pgfsetstrokecolor{currentstroke}%
\pgfsetdash{}{0pt}%
\pgfpathmoveto{\pgfqpoint{2.209792in}{0.849876in}}%
\pgfpathlineto{\pgfqpoint{2.244306in}{0.846260in}}%
\pgfpathlineto{\pgfqpoint{2.205466in}{0.860009in}}%
\pgfpathlineto{\pgfqpoint{2.171236in}{0.869489in}}%
\pgfpathclose%
\pgfusepath{fill}%
\end{pgfscope}%
\begin{pgfscope}%
\pgfpathrectangle{\pgfqpoint{0.150000in}{0.150000in}}{\pgfqpoint{2.700000in}{1.950000in}}%
\pgfusepath{clip}%
\pgfsetbuttcap%
\pgfsetroundjoin%
\definecolor{currentfill}{rgb}{0.748116,0.779136,0.822564}%
\pgfsetfillcolor{currentfill}%
\pgfsetlinewidth{0.000000pt}%
\definecolor{currentstroke}{rgb}{0.000000,0.000000,0.000000}%
\pgfsetstrokecolor{currentstroke}%
\pgfsetdash{}{0pt}%
\pgfpathmoveto{\pgfqpoint{1.424370in}{1.565942in}}%
\pgfpathlineto{\pgfqpoint{1.461961in}{1.571817in}}%
\pgfpathlineto{\pgfqpoint{1.424837in}{1.596264in}}%
\pgfpathlineto{\pgfqpoint{1.387183in}{1.590468in}}%
\pgfpathclose%
\pgfusepath{fill}%
\end{pgfscope}%
\begin{pgfscope}%
\pgfpathrectangle{\pgfqpoint{0.150000in}{0.150000in}}{\pgfqpoint{2.700000in}{1.950000in}}%
\pgfusepath{clip}%
\pgfsetbuttcap%
\pgfsetroundjoin%
\definecolor{currentfill}{rgb}{0.853738,0.734605,0.744041}%
\pgfsetfillcolor{currentfill}%
\pgfsetlinewidth{0.000000pt}%
\definecolor{currentstroke}{rgb}{0.000000,0.000000,0.000000}%
\pgfsetstrokecolor{currentstroke}%
\pgfsetdash{}{0pt}%
\pgfpathmoveto{\pgfqpoint{2.059054in}{0.980573in}}%
\pgfpathlineto{\pgfqpoint{2.094106in}{0.976434in}}%
\pgfpathlineto{\pgfqpoint{2.058350in}{1.042757in}}%
\pgfpathlineto{\pgfqpoint{2.022779in}{1.041247in}}%
\pgfpathclose%
\pgfusepath{fill}%
\end{pgfscope}%
\begin{pgfscope}%
\pgfpathrectangle{\pgfqpoint{0.150000in}{0.150000in}}{\pgfqpoint{2.700000in}{1.950000in}}%
\pgfusepath{clip}%
\pgfsetbuttcap%
\pgfsetroundjoin%
\definecolor{currentfill}{rgb}{0.679703,0.719148,0.774372}%
\pgfsetfillcolor{currentfill}%
\pgfsetlinewidth{0.000000pt}%
\definecolor{currentstroke}{rgb}{0.000000,0.000000,0.000000}%
\pgfsetstrokecolor{currentstroke}%
\pgfsetdash{}{0pt}%
\pgfpathmoveto{\pgfqpoint{1.274788in}{1.646190in}}%
\pgfpathlineto{\pgfqpoint{1.312959in}{1.645581in}}%
\pgfpathlineto{\pgfqpoint{1.276027in}{1.669931in}}%
\pgfpathlineto{\pgfqpoint{1.237604in}{1.676904in}}%
\pgfpathclose%
\pgfusepath{fill}%
\end{pgfscope}%
\begin{pgfscope}%
\pgfpathrectangle{\pgfqpoint{0.150000in}{0.150000in}}{\pgfqpoint{2.700000in}{1.950000in}}%
\pgfusepath{clip}%
\pgfsetbuttcap%
\pgfsetroundjoin%
\definecolor{currentfill}{rgb}{0.909819,0.920925,0.936474}%
\pgfsetfillcolor{currentfill}%
\pgfsetlinewidth{0.000000pt}%
\definecolor{currentstroke}{rgb}{0.000000,0.000000,0.000000}%
\pgfsetstrokecolor{currentstroke}%
\pgfsetdash{}{0pt}%
\pgfpathmoveto{\pgfqpoint{2.059385in}{1.368623in}}%
\pgfpathlineto{\pgfqpoint{2.095722in}{1.387270in}}%
\pgfpathlineto{\pgfqpoint{2.058089in}{1.418073in}}%
\pgfpathlineto{\pgfqpoint{2.021707in}{1.399494in}}%
\pgfpathclose%
\pgfusepath{fill}%
\end{pgfscope}%
\begin{pgfscope}%
\pgfpathrectangle{\pgfqpoint{0.150000in}{0.150000in}}{\pgfqpoint{2.700000in}{1.950000in}}%
\pgfusepath{clip}%
\pgfsetbuttcap%
\pgfsetroundjoin%
\definecolor{currentfill}{rgb}{0.872503,0.888205,0.910187}%
\pgfsetfillcolor{currentfill}%
\pgfsetlinewidth{0.000000pt}%
\definecolor{currentstroke}{rgb}{0.000000,0.000000,0.000000}%
\pgfsetstrokecolor{currentstroke}%
\pgfsetdash{}{0pt}%
\pgfpathmoveto{\pgfqpoint{1.872981in}{1.417883in}}%
\pgfpathlineto{\pgfqpoint{1.909482in}{1.430324in}}%
\pgfpathlineto{\pgfqpoint{1.871577in}{1.448835in}}%
\pgfpathlineto{\pgfqpoint{1.835052in}{1.436460in}}%
\pgfpathclose%
\pgfusepath{fill}%
\end{pgfscope}%
\begin{pgfscope}%
\pgfpathrectangle{\pgfqpoint{0.150000in}{0.150000in}}{\pgfqpoint{2.700000in}{1.950000in}}%
\pgfusepath{clip}%
\pgfsetbuttcap%
\pgfsetroundjoin%
\definecolor{currentfill}{rgb}{0.972013,0.975460,0.980285}%
\pgfsetfillcolor{currentfill}%
\pgfsetlinewidth{0.000000pt}%
\definecolor{currentstroke}{rgb}{0.000000,0.000000,0.000000}%
\pgfsetstrokecolor{currentstroke}%
\pgfsetdash{}{0pt}%
\pgfpathmoveto{\pgfqpoint{2.058312in}{1.270407in}}%
\pgfpathlineto{\pgfqpoint{2.094577in}{1.289193in}}%
\pgfpathlineto{\pgfqpoint{2.059385in}{1.368623in}}%
\pgfpathlineto{\pgfqpoint{2.022915in}{1.349908in}}%
\pgfpathclose%
\pgfusepath{fill}%
\end{pgfscope}%
\begin{pgfscope}%
\pgfpathrectangle{\pgfqpoint{0.150000in}{0.150000in}}{\pgfqpoint{2.700000in}{1.950000in}}%
\pgfusepath{clip}%
\pgfsetbuttcap%
\pgfsetroundjoin%
\definecolor{currentfill}{rgb}{0.800551,0.638097,0.650965}%
\pgfsetfillcolor{currentfill}%
\pgfsetlinewidth{0.000000pt}%
\definecolor{currentstroke}{rgb}{0.000000,0.000000,0.000000}%
\pgfsetstrokecolor{currentstroke}%
\pgfsetdash{}{0pt}%
\pgfpathmoveto{\pgfqpoint{2.098473in}{0.904532in}}%
\pgfpathlineto{\pgfqpoint{2.133157in}{0.894856in}}%
\pgfpathlineto{\pgfqpoint{2.094856in}{0.914329in}}%
\pgfpathlineto{\pgfqpoint{2.058573in}{0.894856in}}%
\pgfpathclose%
\pgfusepath{fill}%
\end{pgfscope}%
\begin{pgfscope}%
\pgfpathrectangle{\pgfqpoint{0.150000in}{0.150000in}}{\pgfqpoint{2.700000in}{1.950000in}}%
\pgfusepath{clip}%
\pgfsetbuttcap%
\pgfsetroundjoin%
\definecolor{currentfill}{rgb}{0.823346,0.679458,0.690855}%
\pgfsetfillcolor{currentfill}%
\pgfsetlinewidth{0.000000pt}%
\definecolor{currentstroke}{rgb}{0.000000,0.000000,0.000000}%
\pgfsetstrokecolor{currentstroke}%
\pgfsetdash{}{0pt}%
\pgfpathmoveto{\pgfqpoint{2.058573in}{0.894856in}}%
\pgfpathlineto{\pgfqpoint{2.094856in}{0.914329in}}%
\pgfpathlineto{\pgfqpoint{2.059054in}{0.980573in}}%
\pgfpathlineto{\pgfqpoint{2.023160in}{0.972943in}}%
\pgfpathclose%
\pgfusepath{fill}%
\end{pgfscope}%
\begin{pgfscope}%
\pgfpathrectangle{\pgfqpoint{0.150000in}{0.150000in}}{\pgfqpoint{2.700000in}{1.950000in}}%
\pgfusepath{clip}%
\pgfsetbuttcap%
\pgfsetroundjoin%
\definecolor{currentfill}{rgb}{0.835187,0.855484,0.883900}%
\pgfsetfillcolor{currentfill}%
\pgfsetlinewidth{0.000000pt}%
\definecolor{currentstroke}{rgb}{0.000000,0.000000,0.000000}%
\pgfsetstrokecolor{currentstroke}%
\pgfsetdash{}{0pt}%
\pgfpathmoveto{\pgfqpoint{1.686109in}{1.461073in}}%
\pgfpathlineto{\pgfqpoint{1.722964in}{1.467262in}}%
\pgfpathlineto{\pgfqpoint{1.685484in}{1.491860in}}%
\pgfpathlineto{\pgfqpoint{1.648500in}{1.479581in}}%
\pgfpathclose%
\pgfusepath{fill}%
\end{pgfscope}%
\begin{pgfscope}%
\pgfpathrectangle{\pgfqpoint{0.150000in}{0.150000in}}{\pgfqpoint{2.700000in}{1.950000in}}%
\pgfusepath{clip}%
\pgfsetbuttcap%
\pgfsetroundjoin%
\definecolor{currentfill}{rgb}{0.797871,0.822763,0.857613}%
\pgfsetfillcolor{currentfill}%
\pgfsetlinewidth{0.000000pt}%
\definecolor{currentstroke}{rgb}{0.000000,0.000000,0.000000}%
\pgfsetstrokecolor{currentstroke}%
\pgfsetdash{}{0pt}%
\pgfpathmoveto{\pgfqpoint{1.573888in}{1.498113in}}%
\pgfpathlineto{\pgfqpoint{1.611070in}{1.504191in}}%
\pgfpathlineto{\pgfqpoint{1.573732in}{1.528741in}}%
\pgfpathlineto{\pgfqpoint{1.536486in}{1.522742in}}%
\pgfpathclose%
\pgfusepath{fill}%
\end{pgfscope}%
\begin{pgfscope}%
\pgfpathrectangle{\pgfqpoint{0.150000in}{0.150000in}}{\pgfqpoint{2.700000in}{1.950000in}}%
\pgfusepath{clip}%
\pgfsetbuttcap%
\pgfsetroundjoin%
\definecolor{currentfill}{rgb}{0.760555,0.790043,0.831327}%
\pgfsetfillcolor{currentfill}%
\pgfsetlinewidth{0.000000pt}%
\definecolor{currentstroke}{rgb}{0.000000,0.000000,0.000000}%
\pgfsetstrokecolor{currentstroke}%
\pgfsetdash{}{0pt}%
\pgfpathmoveto{\pgfqpoint{1.461649in}{1.541356in}}%
\pgfpathlineto{\pgfqpoint{1.499178in}{1.547310in}}%
\pgfpathlineto{\pgfqpoint{1.461961in}{1.571817in}}%
\pgfpathlineto{\pgfqpoint{1.424370in}{1.565942in}}%
\pgfpathclose%
\pgfusepath{fill}%
\end{pgfscope}%
\begin{pgfscope}%
\pgfpathrectangle{\pgfqpoint{0.150000in}{0.150000in}}{\pgfqpoint{2.700000in}{1.950000in}}%
\pgfusepath{clip}%
\pgfsetbuttcap%
\pgfsetroundjoin%
\definecolor{currentfill}{rgb}{0.891161,0.904565,0.923330}%
\pgfsetfillcolor{currentfill}%
\pgfsetlinewidth{0.000000pt}%
\definecolor{currentstroke}{rgb}{0.000000,0.000000,0.000000}%
\pgfsetstrokecolor{currentstroke}%
\pgfsetdash{}{0pt}%
\pgfpathmoveto{\pgfqpoint{1.985449in}{1.386971in}}%
\pgfpathlineto{\pgfqpoint{2.021707in}{1.399494in}}%
\pgfpathlineto{\pgfqpoint{1.983827in}{1.424195in}}%
\pgfpathlineto{\pgfqpoint{1.947291in}{1.405617in}}%
\pgfpathclose%
\pgfusepath{fill}%
\end{pgfscope}%
\begin{pgfscope}%
\pgfpathrectangle{\pgfqpoint{0.150000in}{0.150000in}}{\pgfqpoint{2.700000in}{1.950000in}}%
\pgfusepath{clip}%
\pgfsetbuttcap%
\pgfsetroundjoin%
\definecolor{currentfill}{rgb}{0.860064,0.877298,0.901425}%
\pgfsetfillcolor{currentfill}%
\pgfsetlinewidth{0.000000pt}%
\definecolor{currentstroke}{rgb}{0.000000,0.000000,0.000000}%
\pgfsetstrokecolor{currentstroke}%
\pgfsetdash{}{0pt}%
\pgfpathmoveto{\pgfqpoint{1.798353in}{1.424026in}}%
\pgfpathlineto{\pgfqpoint{1.835052in}{1.436460in}}%
\pgfpathlineto{\pgfqpoint{1.797408in}{1.461113in}}%
\pgfpathlineto{\pgfqpoint{1.760665in}{1.448753in}}%
\pgfpathclose%
\pgfusepath{fill}%
\end{pgfscope}%
\begin{pgfscope}%
\pgfpathrectangle{\pgfqpoint{0.150000in}{0.150000in}}{\pgfqpoint{2.700000in}{1.950000in}}%
\pgfusepath{clip}%
\pgfsetbuttcap%
\pgfsetroundjoin%
\definecolor{currentfill}{rgb}{0.692142,0.730055,0.783134}%
\pgfsetfillcolor{currentfill}%
\pgfsetlinewidth{0.000000pt}%
\definecolor{currentstroke}{rgb}{0.000000,0.000000,0.000000}%
\pgfsetstrokecolor{currentstroke}%
\pgfsetdash{}{0pt}%
\pgfpathmoveto{\pgfqpoint{1.311894in}{1.621695in}}%
\pgfpathlineto{\pgfqpoint{1.349983in}{1.621170in}}%
\pgfpathlineto{\pgfqpoint{1.312959in}{1.645581in}}%
\pgfpathlineto{\pgfqpoint{1.274788in}{1.646190in}}%
\pgfpathclose%
\pgfusepath{fill}%
\end{pgfscope}%
\begin{pgfscope}%
\pgfpathrectangle{\pgfqpoint{0.150000in}{0.150000in}}{\pgfqpoint{2.700000in}{1.950000in}}%
\pgfusepath{clip}%
\pgfsetbuttcap%
\pgfsetroundjoin%
\definecolor{currentfill}{rgb}{0.929718,0.872472,0.877007}%
\pgfsetfillcolor{currentfill}%
\pgfsetlinewidth{0.000000pt}%
\definecolor{currentstroke}{rgb}{0.000000,0.000000,0.000000}%
\pgfsetstrokecolor{currentstroke}%
\pgfsetdash{}{0pt}%
\pgfpathmoveto{\pgfqpoint{2.022123in}{1.103484in}}%
\pgfpathlineto{\pgfqpoint{2.057943in}{1.110714in}}%
\pgfpathlineto{\pgfqpoint{2.022019in}{1.177534in}}%
\pgfpathlineto{\pgfqpoint{1.985738in}{1.164474in}}%
\pgfpathclose%
\pgfusepath{fill}%
\end{pgfscope}%
\begin{pgfscope}%
\pgfpathrectangle{\pgfqpoint{0.150000in}{0.150000in}}{\pgfqpoint{2.700000in}{1.950000in}}%
\pgfusepath{clip}%
\pgfsetbuttcap%
\pgfsetroundjoin%
\definecolor{currentfill}{rgb}{0.971507,0.948300,0.950138}%
\pgfsetfillcolor{currentfill}%
\pgfsetlinewidth{0.000000pt}%
\definecolor{currentstroke}{rgb}{0.000000,0.000000,0.000000}%
\pgfsetstrokecolor{currentstroke}%
\pgfsetdash{}{0pt}%
\pgfpathmoveto{\pgfqpoint{2.022019in}{1.177534in}}%
\pgfpathlineto{\pgfqpoint{2.057832in}{1.184546in}}%
\pgfpathlineto{\pgfqpoint{2.022191in}{1.257594in}}%
\pgfpathlineto{\pgfqpoint{1.985897in}{1.244720in}}%
\pgfpathclose%
\pgfusepath{fill}%
\end{pgfscope}%
\begin{pgfscope}%
\pgfpathrectangle{\pgfqpoint{0.150000in}{0.150000in}}{\pgfqpoint{2.700000in}{1.950000in}}%
\pgfusepath{clip}%
\pgfsetbuttcap%
\pgfsetroundjoin%
\definecolor{currentfill}{rgb}{0.878722,0.893658,0.914568}%
\pgfsetfillcolor{currentfill}%
\pgfsetlinewidth{0.000000pt}%
\definecolor{currentstroke}{rgb}{0.000000,0.000000,0.000000}%
\pgfsetstrokecolor{currentstroke}%
\pgfsetdash{}{0pt}%
\pgfpathmoveto{\pgfqpoint{1.910835in}{1.393101in}}%
\pgfpathlineto{\pgfqpoint{1.947291in}{1.405617in}}%
\pgfpathlineto{\pgfqpoint{1.909482in}{1.430324in}}%
\pgfpathlineto{\pgfqpoint{1.872981in}{1.417883in}}%
\pgfpathclose%
\pgfusepath{fill}%
\end{pgfscope}%
\begin{pgfscope}%
\pgfpathrectangle{\pgfqpoint{0.150000in}{0.150000in}}{\pgfqpoint{2.700000in}{1.950000in}}%
\pgfusepath{clip}%
\pgfsetbuttcap%
\pgfsetroundjoin%
\definecolor{currentfill}{rgb}{0.810309,0.833670,0.866376}%
\pgfsetfillcolor{currentfill}%
\pgfsetlinewidth{0.000000pt}%
\definecolor{currentstroke}{rgb}{0.000000,0.000000,0.000000}%
\pgfsetstrokecolor{currentstroke}%
\pgfsetdash{}{0pt}%
\pgfpathmoveto{\pgfqpoint{1.611383in}{1.473422in}}%
\pgfpathlineto{\pgfqpoint{1.648500in}{1.479581in}}%
\pgfpathlineto{\pgfqpoint{1.611070in}{1.504191in}}%
\pgfpathlineto{\pgfqpoint{1.573888in}{1.498113in}}%
\pgfpathclose%
\pgfusepath{fill}%
\end{pgfscope}%
\begin{pgfscope}%
\pgfpathrectangle{\pgfqpoint{0.150000in}{0.150000in}}{\pgfqpoint{2.700000in}{1.950000in}}%
\pgfusepath{clip}%
\pgfsetbuttcap%
\pgfsetroundjoin%
\definecolor{currentfill}{rgb}{0.903600,0.915472,0.932093}%
\pgfsetfillcolor{currentfill}%
\pgfsetlinewidth{0.000000pt}%
\definecolor{currentstroke}{rgb}{0.000000,0.000000,0.000000}%
\pgfsetstrokecolor{currentstroke}%
\pgfsetdash{}{0pt}%
\pgfpathmoveto{\pgfqpoint{2.022915in}{1.349908in}}%
\pgfpathlineto{\pgfqpoint{2.059385in}{1.368623in}}%
\pgfpathlineto{\pgfqpoint{2.021707in}{1.399494in}}%
\pgfpathlineto{\pgfqpoint{1.985449in}{1.386971in}}%
\pgfpathclose%
\pgfusepath{fill}%
\end{pgfscope}%
\begin{pgfscope}%
\pgfpathrectangle{\pgfqpoint{0.150000in}{0.150000in}}{\pgfqpoint{2.700000in}{1.950000in}}%
\pgfusepath{clip}%
\pgfsetbuttcap%
\pgfsetroundjoin%
\definecolor{currentfill}{rgb}{0.899326,0.817325,0.823820}%
\pgfsetfillcolor{currentfill}%
\pgfsetlinewidth{0.000000pt}%
\definecolor{currentstroke}{rgb}{0.000000,0.000000,0.000000}%
\pgfsetstrokecolor{currentstroke}%
\pgfsetdash{}{0pt}%
\pgfpathmoveto{\pgfqpoint{2.022779in}{1.041247in}}%
\pgfpathlineto{\pgfqpoint{2.058350in}{1.042757in}}%
\pgfpathlineto{\pgfqpoint{2.022123in}{1.103484in}}%
\pgfpathlineto{\pgfqpoint{1.986090in}{1.096211in}}%
\pgfpathclose%
\pgfusepath{fill}%
\end{pgfscope}%
\begin{pgfscope}%
\pgfpathrectangle{\pgfqpoint{0.150000in}{0.150000in}}{\pgfqpoint{2.700000in}{1.950000in}}%
\pgfusepath{clip}%
\pgfsetbuttcap%
\pgfsetroundjoin%
\definecolor{currentfill}{rgb}{0.841406,0.860938,0.888281}%
\pgfsetfillcolor{currentfill}%
\pgfsetlinewidth{0.000000pt}%
\definecolor{currentstroke}{rgb}{0.000000,0.000000,0.000000}%
\pgfsetstrokecolor{currentstroke}%
\pgfsetdash{}{0pt}%
\pgfpathmoveto{\pgfqpoint{1.723747in}{1.436334in}}%
\pgfpathlineto{\pgfqpoint{1.760665in}{1.448753in}}%
\pgfpathlineto{\pgfqpoint{1.722964in}{1.467262in}}%
\pgfpathlineto{\pgfqpoint{1.686109in}{1.461073in}}%
\pgfpathclose%
\pgfusepath{fill}%
\end{pgfscope}%
\begin{pgfscope}%
\pgfpathrectangle{\pgfqpoint{0.150000in}{0.150000in}}{\pgfqpoint{2.700000in}{1.950000in}}%
\pgfusepath{clip}%
\pgfsetbuttcap%
\pgfsetroundjoin%
\definecolor{currentfill}{rgb}{0.704580,0.740962,0.791896}%
\pgfsetfillcolor{currentfill}%
\pgfsetlinewidth{0.000000pt}%
\definecolor{currentstroke}{rgb}{0.000000,0.000000,0.000000}%
\pgfsetstrokecolor{currentstroke}%
\pgfsetdash{}{0pt}%
\pgfpathmoveto{\pgfqpoint{1.349093in}{1.597138in}}%
\pgfpathlineto{\pgfqpoint{1.387183in}{1.590468in}}%
\pgfpathlineto{\pgfqpoint{1.349983in}{1.621170in}}%
\pgfpathlineto{\pgfqpoint{1.311894in}{1.621695in}}%
\pgfpathclose%
\pgfusepath{fill}%
\end{pgfscope}%
\begin{pgfscope}%
\pgfpathrectangle{\pgfqpoint{0.150000in}{0.150000in}}{\pgfqpoint{2.700000in}{1.950000in}}%
\pgfusepath{clip}%
\pgfsetbuttcap%
\pgfsetroundjoin%
\definecolor{currentfill}{rgb}{0.965794,0.970006,0.975904}%
\pgfsetfillcolor{currentfill}%
\pgfsetlinewidth{0.000000pt}%
\definecolor{currentstroke}{rgb}{0.000000,0.000000,0.000000}%
\pgfsetstrokecolor{currentstroke}%
\pgfsetdash{}{0pt}%
\pgfpathmoveto{\pgfqpoint{2.022191in}{1.257594in}}%
\pgfpathlineto{\pgfqpoint{2.058312in}{1.270407in}}%
\pgfpathlineto{\pgfqpoint{2.022915in}{1.349908in}}%
\pgfpathlineto{\pgfqpoint{1.986313in}{1.331125in}}%
\pgfpathclose%
\pgfusepath{fill}%
\end{pgfscope}%
\begin{pgfscope}%
\pgfpathrectangle{\pgfqpoint{0.150000in}{0.150000in}}{\pgfqpoint{2.700000in}{1.950000in}}%
\pgfusepath{clip}%
\pgfsetbuttcap%
\pgfsetroundjoin%
\definecolor{currentfill}{rgb}{0.772993,0.800950,0.840089}%
\pgfsetfillcolor{currentfill}%
\pgfsetlinewidth{0.000000pt}%
\definecolor{currentstroke}{rgb}{0.000000,0.000000,0.000000}%
\pgfsetstrokecolor{currentstroke}%
\pgfsetdash{}{0pt}%
\pgfpathmoveto{\pgfqpoint{1.499021in}{1.516707in}}%
\pgfpathlineto{\pgfqpoint{1.536486in}{1.522742in}}%
\pgfpathlineto{\pgfqpoint{1.499178in}{1.547310in}}%
\pgfpathlineto{\pgfqpoint{1.461649in}{1.541356in}}%
\pgfpathclose%
\pgfusepath{fill}%
\end{pgfscope}%
\begin{pgfscope}%
\pgfpathrectangle{\pgfqpoint{0.150000in}{0.150000in}}{\pgfqpoint{2.700000in}{1.950000in}}%
\pgfusepath{clip}%
\pgfsetbuttcap%
\pgfsetroundjoin%
\definecolor{currentfill}{rgb}{0.866284,0.882751,0.905806}%
\pgfsetfillcolor{currentfill}%
\pgfsetlinewidth{0.000000pt}%
\definecolor{currentstroke}{rgb}{0.000000,0.000000,0.000000}%
\pgfsetstrokecolor{currentstroke}%
\pgfsetdash{}{0pt}%
\pgfpathmoveto{\pgfqpoint{1.836136in}{1.399238in}}%
\pgfpathlineto{\pgfqpoint{1.872981in}{1.417883in}}%
\pgfpathlineto{\pgfqpoint{1.835052in}{1.436460in}}%
\pgfpathlineto{\pgfqpoint{1.798353in}{1.424026in}}%
\pgfpathclose%
\pgfusepath{fill}%
\end{pgfscope}%
\begin{pgfscope}%
\pgfpathrectangle{\pgfqpoint{0.150000in}{0.150000in}}{\pgfqpoint{2.700000in}{1.950000in}}%
\pgfusepath{clip}%
\pgfsetbuttcap%
\pgfsetroundjoin%
\definecolor{currentfill}{rgb}{0.827145,0.686351,0.697503}%
\pgfsetfillcolor{currentfill}%
\pgfsetlinewidth{0.000000pt}%
\definecolor{currentstroke}{rgb}{0.000000,0.000000,0.000000}%
\pgfsetstrokecolor{currentstroke}%
\pgfsetdash{}{0pt}%
\pgfpathmoveto{\pgfqpoint{2.022159in}{0.875313in}}%
\pgfpathlineto{\pgfqpoint{2.058573in}{0.894856in}}%
\pgfpathlineto{\pgfqpoint{2.023160in}{0.972943in}}%
\pgfpathlineto{\pgfqpoint{1.986539in}{0.953465in}}%
\pgfpathclose%
\pgfusepath{fill}%
\end{pgfscope}%
\begin{pgfscope}%
\pgfpathrectangle{\pgfqpoint{0.150000in}{0.150000in}}{\pgfqpoint{2.700000in}{1.950000in}}%
\pgfusepath{clip}%
\pgfsetbuttcap%
\pgfsetroundjoin%
\definecolor{currentfill}{rgb}{0.808150,0.651884,0.664262}%
\pgfsetfillcolor{currentfill}%
\pgfsetlinewidth{0.000000pt}%
\definecolor{currentstroke}{rgb}{0.000000,0.000000,0.000000}%
\pgfsetstrokecolor{currentstroke}%
\pgfsetdash{}{0pt}%
\pgfpathmoveto{\pgfqpoint{2.136686in}{0.879058in}}%
\pgfpathlineto{\pgfqpoint{2.171236in}{0.869489in}}%
\pgfpathlineto{\pgfqpoint{2.133157in}{0.894856in}}%
\pgfpathlineto{\pgfqpoint{2.098473in}{0.904532in}}%
\pgfpathclose%
\pgfusepath{fill}%
\end{pgfscope}%
\begin{pgfscope}%
\pgfpathrectangle{\pgfqpoint{0.150000in}{0.150000in}}{\pgfqpoint{2.700000in}{1.950000in}}%
\pgfusepath{clip}%
\pgfsetbuttcap%
\pgfsetroundjoin%
\definecolor{currentfill}{rgb}{0.822748,0.844577,0.875138}%
\pgfsetfillcolor{currentfill}%
\pgfsetlinewidth{0.000000pt}%
\definecolor{currentstroke}{rgb}{0.000000,0.000000,0.000000}%
\pgfsetstrokecolor{currentstroke}%
\pgfsetdash{}{0pt}%
\pgfpathmoveto{\pgfqpoint{0.975024in}{1.430177in}}%
\pgfpathlineto{\pgfqpoint{1.012213in}{1.473422in}}%
\pgfpathlineto{\pgfqpoint{0.975463in}{1.498113in}}%
\pgfpathlineto{\pgfqpoint{0.938338in}{1.454910in}}%
\pgfpathclose%
\pgfusepath{fill}%
\end{pgfscope}%
\begin{pgfscope}%
\pgfpathrectangle{\pgfqpoint{0.150000in}{0.150000in}}{\pgfqpoint{2.700000in}{1.950000in}}%
\pgfusepath{clip}%
\pgfsetbuttcap%
\pgfsetroundjoin%
\definecolor{currentfill}{rgb}{0.723238,0.757322,0.805040}%
\pgfsetfillcolor{currentfill}%
\pgfsetlinewidth{0.000000pt}%
\definecolor{currentstroke}{rgb}{0.000000,0.000000,0.000000}%
\pgfsetstrokecolor{currentstroke}%
\pgfsetdash{}{0pt}%
\pgfpathmoveto{\pgfqpoint{1.386470in}{1.566273in}}%
\pgfpathlineto{\pgfqpoint{1.424370in}{1.565942in}}%
\pgfpathlineto{\pgfqpoint{1.387183in}{1.590468in}}%
\pgfpathlineto{\pgfqpoint{1.349093in}{1.597138in}}%
\pgfpathclose%
\pgfusepath{fill}%
\end{pgfscope}%
\begin{pgfscope}%
\pgfpathrectangle{\pgfqpoint{0.150000in}{0.150000in}}{\pgfqpoint{2.700000in}{1.950000in}}%
\pgfusepath{clip}%
\pgfsetbuttcap%
\pgfsetroundjoin%
\definecolor{currentfill}{rgb}{0.816529,0.839124,0.870757}%
\pgfsetfillcolor{currentfill}%
\pgfsetlinewidth{0.000000pt}%
\definecolor{currentstroke}{rgb}{0.000000,0.000000,0.000000}%
\pgfsetstrokecolor{currentstroke}%
\pgfsetdash{}{0pt}%
\pgfpathmoveto{\pgfqpoint{1.648971in}{1.448670in}}%
\pgfpathlineto{\pgfqpoint{1.686109in}{1.461073in}}%
\pgfpathlineto{\pgfqpoint{1.648500in}{1.479581in}}%
\pgfpathlineto{\pgfqpoint{1.611383in}{1.473422in}}%
\pgfpathclose%
\pgfusepath{fill}%
\end{pgfscope}%
\begin{pgfscope}%
\pgfpathrectangle{\pgfqpoint{0.150000in}{0.150000in}}{\pgfqpoint{2.700000in}{1.950000in}}%
\pgfusepath{clip}%
\pgfsetbuttcap%
\pgfsetroundjoin%
\definecolor{currentfill}{rgb}{0.623729,0.670067,0.734942}%
\pgfsetfillcolor{currentfill}%
\pgfsetlinewidth{0.000000pt}%
\definecolor{currentstroke}{rgb}{0.000000,0.000000,0.000000}%
\pgfsetstrokecolor{currentstroke}%
\pgfsetdash{}{0pt}%
\pgfpathmoveto{\pgfqpoint{1.198673in}{1.690250in}}%
\pgfpathlineto{\pgfqpoint{1.237604in}{1.676904in}}%
\pgfpathlineto{\pgfqpoint{1.200661in}{1.701284in}}%
\pgfpathlineto{\pgfqpoint{1.161606in}{1.714727in}}%
\pgfpathclose%
\pgfusepath{fill}%
\end{pgfscope}%
\begin{pgfscope}%
\pgfpathrectangle{\pgfqpoint{0.150000in}{0.150000in}}{\pgfqpoint{2.700000in}{1.950000in}}%
\pgfusepath{clip}%
\pgfsetbuttcap%
\pgfsetroundjoin%
\definecolor{currentfill}{rgb}{0.884942,0.899112,0.918949}%
\pgfsetfillcolor{currentfill}%
\pgfsetlinewidth{0.000000pt}%
\definecolor{currentstroke}{rgb}{0.000000,0.000000,0.000000}%
\pgfsetstrokecolor{currentstroke}%
\pgfsetdash{}{0pt}%
\pgfpathmoveto{\pgfqpoint{1.949018in}{1.374388in}}%
\pgfpathlineto{\pgfqpoint{1.985449in}{1.386971in}}%
\pgfpathlineto{\pgfqpoint{1.947291in}{1.405617in}}%
\pgfpathlineto{\pgfqpoint{1.910835in}{1.393101in}}%
\pgfpathclose%
\pgfusepath{fill}%
\end{pgfscope}%
\begin{pgfscope}%
\pgfpathrectangle{\pgfqpoint{0.150000in}{0.150000in}}{\pgfqpoint{2.700000in}{1.950000in}}%
\pgfusepath{clip}%
\pgfsetbuttcap%
\pgfsetroundjoin%
\definecolor{currentfill}{rgb}{0.872733,0.769072,0.777282}%
\pgfsetfillcolor{currentfill}%
\pgfsetlinewidth{0.000000pt}%
\definecolor{currentstroke}{rgb}{0.000000,0.000000,0.000000}%
\pgfsetstrokecolor{currentstroke}%
\pgfsetdash{}{0pt}%
\pgfpathmoveto{\pgfqpoint{2.023160in}{0.972943in}}%
\pgfpathlineto{\pgfqpoint{2.059054in}{0.980573in}}%
\pgfpathlineto{\pgfqpoint{2.022779in}{1.041247in}}%
\pgfpathlineto{\pgfqpoint{1.986956in}{1.039725in}}%
\pgfpathclose%
\pgfusepath{fill}%
\end{pgfscope}%
\begin{pgfscope}%
\pgfpathrectangle{\pgfqpoint{0.150000in}{0.150000in}}{\pgfqpoint{2.700000in}{1.950000in}}%
\pgfusepath{clip}%
\pgfsetbuttcap%
\pgfsetroundjoin%
\definecolor{currentfill}{rgb}{0.815748,0.665671,0.677558}%
\pgfsetfillcolor{currentfill}%
\pgfsetlinewidth{0.000000pt}%
\definecolor{currentstroke}{rgb}{0.000000,0.000000,0.000000}%
\pgfsetstrokecolor{currentstroke}%
\pgfsetdash{}{0pt}%
\pgfpathmoveto{\pgfqpoint{2.062861in}{0.902524in}}%
\pgfpathlineto{\pgfqpoint{2.098473in}{0.904532in}}%
\pgfpathlineto{\pgfqpoint{2.058573in}{0.894856in}}%
\pgfpathlineto{\pgfqpoint{2.022159in}{0.875313in}}%
\pgfpathclose%
\pgfusepath{fill}%
\end{pgfscope}%
\begin{pgfscope}%
\pgfpathrectangle{\pgfqpoint{0.150000in}{0.150000in}}{\pgfqpoint{2.700000in}{1.950000in}}%
\pgfusepath{clip}%
\pgfsetbuttcap%
\pgfsetroundjoin%
\definecolor{currentfill}{rgb}{0.847626,0.866391,0.892662}%
\pgfsetfillcolor{currentfill}%
\pgfsetlinewidth{0.000000pt}%
\definecolor{currentstroke}{rgb}{0.000000,0.000000,0.000000}%
\pgfsetstrokecolor{currentstroke}%
\pgfsetdash{}{0pt}%
\pgfpathmoveto{\pgfqpoint{1.761480in}{1.411533in}}%
\pgfpathlineto{\pgfqpoint{1.798353in}{1.424026in}}%
\pgfpathlineto{\pgfqpoint{1.760665in}{1.448753in}}%
\pgfpathlineto{\pgfqpoint{1.723747in}{1.436334in}}%
\pgfpathclose%
\pgfusepath{fill}%
\end{pgfscope}%
\begin{pgfscope}%
\pgfpathrectangle{\pgfqpoint{0.150000in}{0.150000in}}{\pgfqpoint{2.700000in}{1.950000in}}%
\pgfusepath{clip}%
\pgfsetbuttcap%
\pgfsetroundjoin%
\definecolor{currentfill}{rgb}{0.779213,0.806403,0.844470}%
\pgfsetfillcolor{currentfill}%
\pgfsetlinewidth{0.000000pt}%
\definecolor{currentstroke}{rgb}{0.000000,0.000000,0.000000}%
\pgfsetstrokecolor{currentstroke}%
\pgfsetdash{}{0pt}%
\pgfpathmoveto{\pgfqpoint{1.536486in}{1.491998in}}%
\pgfpathlineto{\pgfqpoint{1.573888in}{1.498113in}}%
\pgfpathlineto{\pgfqpoint{1.536486in}{1.522742in}}%
\pgfpathlineto{\pgfqpoint{1.499021in}{1.516707in}}%
\pgfpathclose%
\pgfusepath{fill}%
\end{pgfscope}%
\begin{pgfscope}%
\pgfpathrectangle{\pgfqpoint{0.150000in}{0.150000in}}{\pgfqpoint{2.700000in}{1.950000in}}%
\pgfusepath{clip}%
\pgfsetbuttcap%
\pgfsetroundjoin%
\definecolor{currentfill}{rgb}{0.979105,0.962086,0.963434}%
\pgfsetfillcolor{currentfill}%
\pgfsetlinewidth{0.000000pt}%
\definecolor{currentstroke}{rgb}{0.000000,0.000000,0.000000}%
\pgfsetstrokecolor{currentstroke}%
\pgfsetdash{}{0pt}%
\pgfpathmoveto{\pgfqpoint{1.985738in}{1.164474in}}%
\pgfpathlineto{\pgfqpoint{2.022019in}{1.177534in}}%
\pgfpathlineto{\pgfqpoint{1.985897in}{1.244720in}}%
\pgfpathlineto{\pgfqpoint{1.949431in}{1.231784in}}%
\pgfpathclose%
\pgfusepath{fill}%
\end{pgfscope}%
\begin{pgfscope}%
\pgfpathrectangle{\pgfqpoint{0.150000in}{0.150000in}}{\pgfqpoint{2.700000in}{1.950000in}}%
\pgfusepath{clip}%
\pgfsetbuttcap%
\pgfsetroundjoin%
\definecolor{currentfill}{rgb}{0.872503,0.888205,0.910187}%
\pgfsetfillcolor{currentfill}%
\pgfsetlinewidth{0.000000pt}%
\definecolor{currentstroke}{rgb}{0.000000,0.000000,0.000000}%
\pgfsetstrokecolor{currentstroke}%
\pgfsetdash{}{0pt}%
\pgfpathmoveto{\pgfqpoint{1.874205in}{1.380526in}}%
\pgfpathlineto{\pgfqpoint{1.910835in}{1.393101in}}%
\pgfpathlineto{\pgfqpoint{1.872981in}{1.417883in}}%
\pgfpathlineto{\pgfqpoint{1.836136in}{1.399238in}}%
\pgfpathclose%
\pgfusepath{fill}%
\end{pgfscope}%
\begin{pgfscope}%
\pgfpathrectangle{\pgfqpoint{0.150000in}{0.150000in}}{\pgfqpoint{2.700000in}{1.950000in}}%
\pgfusepath{clip}%
\pgfsetbuttcap%
\pgfsetroundjoin%
\definecolor{currentfill}{rgb}{0.941115,0.893153,0.896952}%
\pgfsetfillcolor{currentfill}%
\pgfsetlinewidth{0.000000pt}%
\definecolor{currentstroke}{rgb}{0.000000,0.000000,0.000000}%
\pgfsetstrokecolor{currentstroke}%
\pgfsetdash{}{0pt}%
\pgfpathmoveto{\pgfqpoint{1.986090in}{1.096211in}}%
\pgfpathlineto{\pgfqpoint{2.022123in}{1.103484in}}%
\pgfpathlineto{\pgfqpoint{1.985738in}{1.164474in}}%
\pgfpathlineto{\pgfqpoint{1.949519in}{1.157365in}}%
\pgfpathclose%
\pgfusepath{fill}%
\end{pgfscope}%
\begin{pgfscope}%
\pgfpathrectangle{\pgfqpoint{0.150000in}{0.150000in}}{\pgfqpoint{2.700000in}{1.950000in}}%
\pgfusepath{clip}%
\pgfsetbuttcap%
\pgfsetroundjoin%
\definecolor{currentfill}{rgb}{0.897381,0.910018,0.927711}%
\pgfsetfillcolor{currentfill}%
\pgfsetlinewidth{0.000000pt}%
\definecolor{currentstroke}{rgb}{0.000000,0.000000,0.000000}%
\pgfsetstrokecolor{currentstroke}%
\pgfsetdash{}{0pt}%
\pgfpathmoveto{\pgfqpoint{1.986313in}{1.331125in}}%
\pgfpathlineto{\pgfqpoint{2.022915in}{1.349908in}}%
\pgfpathlineto{\pgfqpoint{1.985449in}{1.386971in}}%
\pgfpathlineto{\pgfqpoint{1.949018in}{1.374388in}}%
\pgfpathclose%
\pgfusepath{fill}%
\end{pgfscope}%
\begin{pgfscope}%
\pgfpathrectangle{\pgfqpoint{0.150000in}{0.150000in}}{\pgfqpoint{2.700000in}{1.950000in}}%
\pgfusepath{clip}%
\pgfsetbuttcap%
\pgfsetroundjoin%
\definecolor{currentfill}{rgb}{0.804350,0.644991,0.657613}%
\pgfsetfillcolor{currentfill}%
\pgfsetlinewidth{0.000000pt}%
\definecolor{currentstroke}{rgb}{0.000000,0.000000,0.000000}%
\pgfsetstrokecolor{currentstroke}%
\pgfsetdash{}{0pt}%
\pgfpathmoveto{\pgfqpoint{2.175360in}{0.859370in}}%
\pgfpathlineto{\pgfqpoint{2.209792in}{0.849876in}}%
\pgfpathlineto{\pgfqpoint{2.171236in}{0.869489in}}%
\pgfpathlineto{\pgfqpoint{2.136686in}{0.879058in}}%
\pgfpathclose%
\pgfusepath{fill}%
\end{pgfscope}%
\begin{pgfscope}%
\pgfpathrectangle{\pgfqpoint{0.150000in}{0.150000in}}{\pgfqpoint{2.700000in}{1.950000in}}%
\pgfusepath{clip}%
\pgfsetbuttcap%
\pgfsetroundjoin%
\definecolor{currentfill}{rgb}{0.636167,0.680974,0.743704}%
\pgfsetfillcolor{currentfill}%
\pgfsetlinewidth{0.000000pt}%
\definecolor{currentstroke}{rgb}{0.000000,0.000000,0.000000}%
\pgfsetstrokecolor{currentstroke}%
\pgfsetdash{}{0pt}%
\pgfpathmoveto{\pgfqpoint{1.236004in}{1.659401in}}%
\pgfpathlineto{\pgfqpoint{1.274788in}{1.646190in}}%
\pgfpathlineto{\pgfqpoint{1.237604in}{1.676904in}}%
\pgfpathlineto{\pgfqpoint{1.198673in}{1.690250in}}%
\pgfpathclose%
\pgfusepath{fill}%
\end{pgfscope}%
\begin{pgfscope}%
\pgfpathrectangle{\pgfqpoint{0.150000in}{0.150000in}}{\pgfqpoint{2.700000in}{1.950000in}}%
\pgfusepath{clip}%
\pgfsetbuttcap%
\pgfsetroundjoin%
\definecolor{currentfill}{rgb}{0.735677,0.768229,0.813802}%
\pgfsetfillcolor{currentfill}%
\pgfsetlinewidth{0.000000pt}%
\definecolor{currentstroke}{rgb}{0.000000,0.000000,0.000000}%
\pgfsetstrokecolor{currentstroke}%
\pgfsetdash{}{0pt}%
\pgfpathmoveto{\pgfqpoint{1.423834in}{1.541600in}}%
\pgfpathlineto{\pgfqpoint{1.461649in}{1.541356in}}%
\pgfpathlineto{\pgfqpoint{1.424370in}{1.565942in}}%
\pgfpathlineto{\pgfqpoint{1.386470in}{1.566273in}}%
\pgfpathclose%
\pgfusepath{fill}%
\end{pgfscope}%
\begin{pgfscope}%
\pgfpathrectangle{\pgfqpoint{0.150000in}{0.150000in}}{\pgfqpoint{2.700000in}{1.950000in}}%
\pgfusepath{clip}%
\pgfsetbuttcap%
\pgfsetroundjoin%
\definecolor{currentfill}{rgb}{0.959574,0.964553,0.971523}%
\pgfsetfillcolor{currentfill}%
\pgfsetlinewidth{0.000000pt}%
\definecolor{currentstroke}{rgb}{0.000000,0.000000,0.000000}%
\pgfsetstrokecolor{currentstroke}%
\pgfsetdash{}{0pt}%
\pgfpathmoveto{\pgfqpoint{1.985897in}{1.244720in}}%
\pgfpathlineto{\pgfqpoint{2.022191in}{1.257594in}}%
\pgfpathlineto{\pgfqpoint{1.986313in}{1.331125in}}%
\pgfpathlineto{\pgfqpoint{1.949578in}{1.312274in}}%
\pgfpathclose%
\pgfusepath{fill}%
\end{pgfscope}%
\begin{pgfscope}%
\pgfpathrectangle{\pgfqpoint{0.150000in}{0.150000in}}{\pgfqpoint{2.700000in}{1.950000in}}%
\pgfusepath{clip}%
\pgfsetbuttcap%
\pgfsetroundjoin%
\definecolor{currentfill}{rgb}{0.827145,0.686351,0.697503}%
\pgfsetfillcolor{currentfill}%
\pgfsetlinewidth{0.000000pt}%
\definecolor{currentstroke}{rgb}{0.000000,0.000000,0.000000}%
\pgfsetstrokecolor{currentstroke}%
\pgfsetdash{}{0pt}%
\pgfpathmoveto{\pgfqpoint{1.985868in}{0.861529in}}%
\pgfpathlineto{\pgfqpoint{2.022159in}{0.875313in}}%
\pgfpathlineto{\pgfqpoint{1.986539in}{0.953465in}}%
\pgfpathlineto{\pgfqpoint{1.949786in}{0.933917in}}%
\pgfpathclose%
\pgfusepath{fill}%
\end{pgfscope}%
\begin{pgfscope}%
\pgfpathrectangle{\pgfqpoint{0.150000in}{0.150000in}}{\pgfqpoint{2.700000in}{1.950000in}}%
\pgfusepath{clip}%
\pgfsetbuttcap%
\pgfsetroundjoin%
\definecolor{currentfill}{rgb}{0.828968,0.850031,0.879519}%
\pgfsetfillcolor{currentfill}%
\pgfsetlinewidth{0.000000pt}%
\definecolor{currentstroke}{rgb}{0.000000,0.000000,0.000000}%
\pgfsetstrokecolor{currentstroke}%
\pgfsetdash{}{0pt}%
\pgfpathmoveto{\pgfqpoint{1.011503in}{1.411533in}}%
\pgfpathlineto{\pgfqpoint{1.048776in}{1.454848in}}%
\pgfpathlineto{\pgfqpoint{1.012213in}{1.473422in}}%
\pgfpathlineto{\pgfqpoint{0.975024in}{1.430177in}}%
\pgfpathclose%
\pgfusepath{fill}%
\end{pgfscope}%
\begin{pgfscope}%
\pgfpathrectangle{\pgfqpoint{0.150000in}{0.150000in}}{\pgfqpoint{2.700000in}{1.950000in}}%
\pgfusepath{clip}%
\pgfsetbuttcap%
\pgfsetroundjoin%
\definecolor{currentfill}{rgb}{0.828968,0.850031,0.879519}%
\pgfsetfillcolor{currentfill}%
\pgfsetlinewidth{0.000000pt}%
\definecolor{currentstroke}{rgb}{0.000000,0.000000,0.000000}%
\pgfsetstrokecolor{currentstroke}%
\pgfsetdash{}{0pt}%
\pgfpathmoveto{\pgfqpoint{1.686653in}{1.423856in}}%
\pgfpathlineto{\pgfqpoint{1.723747in}{1.436334in}}%
\pgfpathlineto{\pgfqpoint{1.686109in}{1.461073in}}%
\pgfpathlineto{\pgfqpoint{1.648971in}{1.448670in}}%
\pgfpathclose%
\pgfusepath{fill}%
\end{pgfscope}%
\begin{pgfscope}%
\pgfpathrectangle{\pgfqpoint{0.150000in}{0.150000in}}{\pgfqpoint{2.700000in}{1.950000in}}%
\pgfusepath{clip}%
\pgfsetbuttcap%
\pgfsetroundjoin%
\definecolor{currentfill}{rgb}{0.860064,0.877298,0.901425}%
\pgfsetfillcolor{currentfill}%
\pgfsetlinewidth{0.000000pt}%
\definecolor{currentstroke}{rgb}{0.000000,0.000000,0.000000}%
\pgfsetstrokecolor{currentstroke}%
\pgfsetdash{}{0pt}%
\pgfpathmoveto{\pgfqpoint{1.799306in}{1.386670in}}%
\pgfpathlineto{\pgfqpoint{1.836136in}{1.399238in}}%
\pgfpathlineto{\pgfqpoint{1.798353in}{1.424026in}}%
\pgfpathlineto{\pgfqpoint{1.761480in}{1.411533in}}%
\pgfpathclose%
\pgfusepath{fill}%
\end{pgfscope}%
\begin{pgfscope}%
\pgfpathrectangle{\pgfqpoint{0.150000in}{0.150000in}}{\pgfqpoint{2.700000in}{1.950000in}}%
\pgfusepath{clip}%
\pgfsetbuttcap%
\pgfsetroundjoin%
\definecolor{currentfill}{rgb}{0.791651,0.817310,0.853232}%
\pgfsetfillcolor{currentfill}%
\pgfsetlinewidth{0.000000pt}%
\definecolor{currentstroke}{rgb}{0.000000,0.000000,0.000000}%
\pgfsetstrokecolor{currentstroke}%
\pgfsetdash{}{0pt}%
\pgfpathmoveto{\pgfqpoint{1.574046in}{1.467227in}}%
\pgfpathlineto{\pgfqpoint{1.611383in}{1.473422in}}%
\pgfpathlineto{\pgfqpoint{1.573888in}{1.498113in}}%
\pgfpathlineto{\pgfqpoint{1.536486in}{1.491998in}}%
\pgfpathclose%
\pgfusepath{fill}%
\end{pgfscope}%
\begin{pgfscope}%
\pgfpathrectangle{\pgfqpoint{0.150000in}{0.150000in}}{\pgfqpoint{2.700000in}{1.950000in}}%
\pgfusepath{clip}%
\pgfsetbuttcap%
\pgfsetroundjoin%
\definecolor{currentfill}{rgb}{0.654825,0.697335,0.756847}%
\pgfsetfillcolor{currentfill}%
\pgfsetlinewidth{0.000000pt}%
\definecolor{currentstroke}{rgb}{0.000000,0.000000,0.000000}%
\pgfsetstrokecolor{currentstroke}%
\pgfsetdash{}{0pt}%
\pgfpathmoveto{\pgfqpoint{1.273386in}{1.628511in}}%
\pgfpathlineto{\pgfqpoint{1.311894in}{1.621695in}}%
\pgfpathlineto{\pgfqpoint{1.274788in}{1.646190in}}%
\pgfpathlineto{\pgfqpoint{1.236004in}{1.659401in}}%
\pgfpathclose%
\pgfusepath{fill}%
\end{pgfscope}%
\begin{pgfscope}%
\pgfpathrectangle{\pgfqpoint{0.150000in}{0.150000in}}{\pgfqpoint{2.700000in}{1.950000in}}%
\pgfusepath{clip}%
\pgfsetbuttcap%
\pgfsetroundjoin%
\definecolor{currentfill}{rgb}{0.914522,0.844899,0.850414}%
\pgfsetfillcolor{currentfill}%
\pgfsetlinewidth{0.000000pt}%
\definecolor{currentstroke}{rgb}{0.000000,0.000000,0.000000}%
\pgfsetstrokecolor{currentstroke}%
\pgfsetdash{}{0pt}%
\pgfpathmoveto{\pgfqpoint{1.986956in}{1.039725in}}%
\pgfpathlineto{\pgfqpoint{2.022779in}{1.041247in}}%
\pgfpathlineto{\pgfqpoint{1.986090in}{1.096211in}}%
\pgfpathlineto{\pgfqpoint{1.950080in}{1.094881in}}%
\pgfpathclose%
\pgfusepath{fill}%
\end{pgfscope}%
\begin{pgfscope}%
\pgfpathrectangle{\pgfqpoint{0.150000in}{0.150000in}}{\pgfqpoint{2.700000in}{1.950000in}}%
\pgfusepath{clip}%
\pgfsetbuttcap%
\pgfsetroundjoin%
\definecolor{currentfill}{rgb}{0.810309,0.833670,0.866376}%
\pgfsetfillcolor{currentfill}%
\pgfsetlinewidth{0.000000pt}%
\definecolor{currentstroke}{rgb}{0.000000,0.000000,0.000000}%
\pgfsetstrokecolor{currentstroke}%
\pgfsetdash{}{0pt}%
\pgfpathmoveto{\pgfqpoint{0.934447in}{1.448586in}}%
\pgfpathlineto{\pgfqpoint{0.975024in}{1.430177in}}%
\pgfpathlineto{\pgfqpoint{0.938338in}{1.454910in}}%
\pgfpathlineto{\pgfqpoint{0.897619in}{1.473426in}}%
\pgfpathclose%
\pgfusepath{fill}%
\end{pgfscope}%
\begin{pgfscope}%
\pgfpathrectangle{\pgfqpoint{0.150000in}{0.150000in}}{\pgfqpoint{2.700000in}{1.950000in}}%
\pgfusepath{clip}%
\pgfsetbuttcap%
\pgfsetroundjoin%
\definecolor{currentfill}{rgb}{0.741896,0.773683,0.818183}%
\pgfsetfillcolor{currentfill}%
\pgfsetlinewidth{0.000000pt}%
\definecolor{currentstroke}{rgb}{0.000000,0.000000,0.000000}%
\pgfsetstrokecolor{currentstroke}%
\pgfsetdash{}{0pt}%
\pgfpathmoveto{\pgfqpoint{1.461291in}{1.516865in}}%
\pgfpathlineto{\pgfqpoint{1.499021in}{1.516707in}}%
\pgfpathlineto{\pgfqpoint{1.461649in}{1.541356in}}%
\pgfpathlineto{\pgfqpoint{1.423834in}{1.541600in}}%
\pgfpathclose%
\pgfusepath{fill}%
\end{pgfscope}%
\begin{pgfscope}%
\pgfpathrectangle{\pgfqpoint{0.150000in}{0.150000in}}{\pgfqpoint{2.700000in}{1.950000in}}%
\pgfusepath{clip}%
\pgfsetbuttcap%
\pgfsetroundjoin%
\definecolor{currentfill}{rgb}{0.872503,0.888205,0.910187}%
\pgfsetfillcolor{currentfill}%
\pgfsetlinewidth{0.000000pt}%
\definecolor{currentstroke}{rgb}{0.000000,0.000000,0.000000}%
\pgfsetstrokecolor{currentstroke}%
\pgfsetdash{}{0pt}%
\pgfpathmoveto{\pgfqpoint{1.912198in}{1.355607in}}%
\pgfpathlineto{\pgfqpoint{1.949018in}{1.374388in}}%
\pgfpathlineto{\pgfqpoint{1.910835in}{1.393101in}}%
\pgfpathlineto{\pgfqpoint{1.874205in}{1.380526in}}%
\pgfpathclose%
\pgfusepath{fill}%
\end{pgfscope}%
\begin{pgfscope}%
\pgfpathrectangle{\pgfqpoint{0.150000in}{0.150000in}}{\pgfqpoint{2.700000in}{1.950000in}}%
\pgfusepath{clip}%
\pgfsetbuttcap%
\pgfsetroundjoin%
\definecolor{currentfill}{rgb}{0.819547,0.672564,0.684206}%
\pgfsetfillcolor{currentfill}%
\pgfsetlinewidth{0.000000pt}%
\definecolor{currentstroke}{rgb}{0.000000,0.000000,0.000000}%
\pgfsetstrokecolor{currentstroke}%
\pgfsetdash{}{0pt}%
\pgfpathmoveto{\pgfqpoint{2.026155in}{0.882834in}}%
\pgfpathlineto{\pgfqpoint{2.062861in}{0.902524in}}%
\pgfpathlineto{\pgfqpoint{2.022159in}{0.875313in}}%
\pgfpathlineto{\pgfqpoint{1.985868in}{0.861529in}}%
\pgfpathclose%
\pgfusepath{fill}%
\end{pgfscope}%
\begin{pgfscope}%
\pgfpathrectangle{\pgfqpoint{0.150000in}{0.150000in}}{\pgfqpoint{2.700000in}{1.950000in}}%
\pgfusepath{clip}%
\pgfsetbuttcap%
\pgfsetroundjoin%
\definecolor{currentfill}{rgb}{0.741896,0.773683,0.818183}%
\pgfsetfillcolor{currentfill}%
\pgfsetlinewidth{0.000000pt}%
\definecolor{currentstroke}{rgb}{0.000000,0.000000,0.000000}%
\pgfsetstrokecolor{currentstroke}%
\pgfsetdash{}{0pt}%
\pgfpathmoveto{\pgfqpoint{1.012213in}{1.473422in}}%
\pgfpathlineto{\pgfqpoint{1.047208in}{1.566605in}}%
\pgfpathlineto{\pgfqpoint{1.010229in}{1.591304in}}%
\pgfpathlineto{\pgfqpoint{0.975463in}{1.498113in}}%
\pgfpathclose%
\pgfusepath{fill}%
\end{pgfscope}%
\begin{pgfscope}%
\pgfpathrectangle{\pgfqpoint{0.150000in}{0.150000in}}{\pgfqpoint{2.700000in}{1.950000in}}%
\pgfusepath{clip}%
\pgfsetbuttcap%
\pgfsetroundjoin%
\definecolor{currentfill}{rgb}{0.835187,0.855484,0.883900}%
\pgfsetfillcolor{currentfill}%
\pgfsetlinewidth{0.000000pt}%
\definecolor{currentstroke}{rgb}{0.000000,0.000000,0.000000}%
\pgfsetstrokecolor{currentstroke}%
\pgfsetdash{}{0pt}%
\pgfpathmoveto{\pgfqpoint{1.724429in}{1.398980in}}%
\pgfpathlineto{\pgfqpoint{1.761480in}{1.411533in}}%
\pgfpathlineto{\pgfqpoint{1.723747in}{1.436334in}}%
\pgfpathlineto{\pgfqpoint{1.686653in}{1.423856in}}%
\pgfpathclose%
\pgfusepath{fill}%
\end{pgfscope}%
\begin{pgfscope}%
\pgfpathrectangle{\pgfqpoint{0.150000in}{0.150000in}}{\pgfqpoint{2.700000in}{1.950000in}}%
\pgfusepath{clip}%
\pgfsetbuttcap%
\pgfsetroundjoin%
\definecolor{currentfill}{rgb}{0.797871,0.822763,0.857613}%
\pgfsetfillcolor{currentfill}%
\pgfsetlinewidth{0.000000pt}%
\definecolor{currentstroke}{rgb}{0.000000,0.000000,0.000000}%
\pgfsetstrokecolor{currentstroke}%
\pgfsetdash{}{0pt}%
\pgfpathmoveto{\pgfqpoint{1.611698in}{1.442393in}}%
\pgfpathlineto{\pgfqpoint{1.648971in}{1.448670in}}%
\pgfpathlineto{\pgfqpoint{1.611383in}{1.473422in}}%
\pgfpathlineto{\pgfqpoint{1.574046in}{1.467227in}}%
\pgfpathclose%
\pgfusepath{fill}%
\end{pgfscope}%
\begin{pgfscope}%
\pgfpathrectangle{\pgfqpoint{0.150000in}{0.150000in}}{\pgfqpoint{2.700000in}{1.950000in}}%
\pgfusepath{clip}%
\pgfsetbuttcap%
\pgfsetroundjoin%
\definecolor{currentfill}{rgb}{0.897381,0.910018,0.927711}%
\pgfsetfillcolor{currentfill}%
\pgfsetlinewidth{0.000000pt}%
\definecolor{currentstroke}{rgb}{0.000000,0.000000,0.000000}%
\pgfsetstrokecolor{currentstroke}%
\pgfsetdash{}{0pt}%
\pgfpathmoveto{\pgfqpoint{1.949578in}{1.312274in}}%
\pgfpathlineto{\pgfqpoint{1.986313in}{1.331125in}}%
\pgfpathlineto{\pgfqpoint{1.949018in}{1.374388in}}%
\pgfpathlineto{\pgfqpoint{1.912198in}{1.355607in}}%
\pgfpathclose%
\pgfusepath{fill}%
\end{pgfscope}%
\begin{pgfscope}%
\pgfpathrectangle{\pgfqpoint{0.150000in}{0.150000in}}{\pgfqpoint{2.700000in}{1.950000in}}%
\pgfusepath{clip}%
\pgfsetbuttcap%
\pgfsetroundjoin%
\definecolor{currentfill}{rgb}{0.667264,0.708241,0.765610}%
\pgfsetfillcolor{currentfill}%
\pgfsetlinewidth{0.000000pt}%
\definecolor{currentstroke}{rgb}{0.000000,0.000000,0.000000}%
\pgfsetstrokecolor{currentstroke}%
\pgfsetdash{}{0pt}%
\pgfpathmoveto{\pgfqpoint{1.310690in}{1.603862in}}%
\pgfpathlineto{\pgfqpoint{1.349093in}{1.597138in}}%
\pgfpathlineto{\pgfqpoint{1.311894in}{1.621695in}}%
\pgfpathlineto{\pgfqpoint{1.273386in}{1.628511in}}%
\pgfpathclose%
\pgfusepath{fill}%
\end{pgfscope}%
\begin{pgfscope}%
\pgfpathrectangle{\pgfqpoint{0.150000in}{0.150000in}}{\pgfqpoint{2.700000in}{1.950000in}}%
\pgfusepath{clip}%
\pgfsetbuttcap%
\pgfsetroundjoin%
\definecolor{currentfill}{rgb}{0.860064,0.877298,0.901425}%
\pgfsetfillcolor{currentfill}%
\pgfsetlinewidth{0.000000pt}%
\definecolor{currentstroke}{rgb}{0.000000,0.000000,0.000000}%
\pgfsetstrokecolor{currentstroke}%
\pgfsetdash{}{0pt}%
\pgfpathmoveto{\pgfqpoint{1.837399in}{1.367890in}}%
\pgfpathlineto{\pgfqpoint{1.874205in}{1.380526in}}%
\pgfpathlineto{\pgfqpoint{1.836136in}{1.399238in}}%
\pgfpathlineto{\pgfqpoint{1.799306in}{1.386670in}}%
\pgfpathclose%
\pgfusepath{fill}%
\end{pgfscope}%
\begin{pgfscope}%
\pgfpathrectangle{\pgfqpoint{0.150000in}{0.150000in}}{\pgfqpoint{2.700000in}{1.950000in}}%
\pgfusepath{clip}%
\pgfsetbuttcap%
\pgfsetroundjoin%
\definecolor{currentfill}{rgb}{0.828968,0.850031,0.879519}%
\pgfsetfillcolor{currentfill}%
\pgfsetlinewidth{0.000000pt}%
\definecolor{currentstroke}{rgb}{0.000000,0.000000,0.000000}%
\pgfsetstrokecolor{currentstroke}%
\pgfsetdash{}{0pt}%
\pgfpathmoveto{\pgfqpoint{1.048114in}{1.392822in}}%
\pgfpathlineto{\pgfqpoint{1.085730in}{1.430028in}}%
\pgfpathlineto{\pgfqpoint{1.048776in}{1.454848in}}%
\pgfpathlineto{\pgfqpoint{1.011503in}{1.411533in}}%
\pgfpathclose%
\pgfusepath{fill}%
\end{pgfscope}%
\begin{pgfscope}%
\pgfpathrectangle{\pgfqpoint{0.150000in}{0.150000in}}{\pgfqpoint{2.700000in}{1.950000in}}%
\pgfusepath{clip}%
\pgfsetbuttcap%
\pgfsetroundjoin%
\definecolor{currentfill}{rgb}{0.887929,0.796645,0.803876}%
\pgfsetfillcolor{currentfill}%
\pgfsetlinewidth{0.000000pt}%
\definecolor{currentstroke}{rgb}{0.000000,0.000000,0.000000}%
\pgfsetstrokecolor{currentstroke}%
\pgfsetdash{}{0pt}%
\pgfpathmoveto{\pgfqpoint{1.986539in}{0.953465in}}%
\pgfpathlineto{\pgfqpoint{2.023160in}{0.972943in}}%
\pgfpathlineto{\pgfqpoint{1.986956in}{1.039725in}}%
\pgfpathlineto{\pgfqpoint{1.950880in}{1.038193in}}%
\pgfpathclose%
\pgfusepath{fill}%
\end{pgfscope}%
\begin{pgfscope}%
\pgfpathrectangle{\pgfqpoint{0.150000in}{0.150000in}}{\pgfqpoint{2.700000in}{1.950000in}}%
\pgfusepath{clip}%
\pgfsetbuttcap%
\pgfsetroundjoin%
\definecolor{currentfill}{rgb}{0.986703,0.975873,0.976731}%
\pgfsetfillcolor{currentfill}%
\pgfsetlinewidth{0.000000pt}%
\definecolor{currentstroke}{rgb}{0.000000,0.000000,0.000000}%
\pgfsetstrokecolor{currentstroke}%
\pgfsetdash{}{0pt}%
\pgfpathmoveto{\pgfqpoint{1.949519in}{1.157365in}}%
\pgfpathlineto{\pgfqpoint{1.985738in}{1.164474in}}%
\pgfpathlineto{\pgfqpoint{1.949431in}{1.231784in}}%
\pgfpathlineto{\pgfqpoint{1.912789in}{1.218787in}}%
\pgfpathclose%
\pgfusepath{fill}%
\end{pgfscope}%
\begin{pgfscope}%
\pgfpathrectangle{\pgfqpoint{0.150000in}{0.150000in}}{\pgfqpoint{2.700000in}{1.950000in}}%
\pgfusepath{clip}%
\pgfsetbuttcap%
\pgfsetroundjoin%
\definecolor{currentfill}{rgb}{0.959574,0.964553,0.971523}%
\pgfsetfillcolor{currentfill}%
\pgfsetlinewidth{0.000000pt}%
\definecolor{currentstroke}{rgb}{0.000000,0.000000,0.000000}%
\pgfsetstrokecolor{currentstroke}%
\pgfsetdash{}{0pt}%
\pgfpathmoveto{\pgfqpoint{1.949431in}{1.231784in}}%
\pgfpathlineto{\pgfqpoint{1.985897in}{1.244720in}}%
\pgfpathlineto{\pgfqpoint{1.949578in}{1.312274in}}%
\pgfpathlineto{\pgfqpoint{1.912708in}{1.293354in}}%
\pgfpathclose%
\pgfusepath{fill}%
\end{pgfscope}%
\begin{pgfscope}%
\pgfpathrectangle{\pgfqpoint{0.150000in}{0.150000in}}{\pgfqpoint{2.700000in}{1.950000in}}%
\pgfusepath{clip}%
\pgfsetbuttcap%
\pgfsetroundjoin%
\definecolor{currentfill}{rgb}{0.760555,0.790043,0.831327}%
\pgfsetfillcolor{currentfill}%
\pgfsetlinewidth{0.000000pt}%
\definecolor{currentstroke}{rgb}{0.000000,0.000000,0.000000}%
\pgfsetstrokecolor{currentstroke}%
\pgfsetdash{}{0pt}%
\pgfpathmoveto{\pgfqpoint{1.498863in}{1.485847in}}%
\pgfpathlineto{\pgfqpoint{1.536486in}{1.491998in}}%
\pgfpathlineto{\pgfqpoint{1.499021in}{1.516707in}}%
\pgfpathlineto{\pgfqpoint{1.461291in}{1.516865in}}%
\pgfpathclose%
\pgfusepath{fill}%
\end{pgfscope}%
\begin{pgfscope}%
\pgfpathrectangle{\pgfqpoint{0.150000in}{0.150000in}}{\pgfqpoint{2.700000in}{1.950000in}}%
\pgfusepath{clip}%
\pgfsetbuttcap%
\pgfsetroundjoin%
\definecolor{currentfill}{rgb}{0.830944,0.693244,0.704151}%
\pgfsetfillcolor{currentfill}%
\pgfsetlinewidth{0.000000pt}%
\definecolor{currentstroke}{rgb}{0.000000,0.000000,0.000000}%
\pgfsetstrokecolor{currentstroke}%
\pgfsetdash{}{0pt}%
\pgfpathmoveto{\pgfqpoint{1.949168in}{0.841843in}}%
\pgfpathlineto{\pgfqpoint{1.985868in}{0.861529in}}%
\pgfpathlineto{\pgfqpoint{1.949786in}{0.933917in}}%
\pgfpathlineto{\pgfqpoint{1.913114in}{0.920195in}}%
\pgfpathclose%
\pgfusepath{fill}%
\end{pgfscope}%
\begin{pgfscope}%
\pgfpathrectangle{\pgfqpoint{0.150000in}{0.150000in}}{\pgfqpoint{2.700000in}{1.950000in}}%
\pgfusepath{clip}%
\pgfsetbuttcap%
\pgfsetroundjoin%
\definecolor{currentfill}{rgb}{0.816529,0.839124,0.870757}%
\pgfsetfillcolor{currentfill}%
\pgfsetlinewidth{0.000000pt}%
\definecolor{currentstroke}{rgb}{0.000000,0.000000,0.000000}%
\pgfsetstrokecolor{currentstroke}%
\pgfsetdash{}{0pt}%
\pgfpathmoveto{\pgfqpoint{0.971043in}{1.429878in}}%
\pgfpathlineto{\pgfqpoint{1.011503in}{1.411533in}}%
\pgfpathlineto{\pgfqpoint{0.975024in}{1.430177in}}%
\pgfpathlineto{\pgfqpoint{0.934447in}{1.448586in}}%
\pgfpathclose%
\pgfusepath{fill}%
\end{pgfscope}%
\begin{pgfscope}%
\pgfpathrectangle{\pgfqpoint{0.150000in}{0.150000in}}{\pgfqpoint{2.700000in}{1.950000in}}%
\pgfusepath{clip}%
\pgfsetbuttcap%
\pgfsetroundjoin%
\definecolor{currentfill}{rgb}{0.748116,0.779136,0.822564}%
\pgfsetfillcolor{currentfill}%
\pgfsetlinewidth{0.000000pt}%
\definecolor{currentstroke}{rgb}{0.000000,0.000000,0.000000}%
\pgfsetstrokecolor{currentstroke}%
\pgfsetdash{}{0pt}%
\pgfpathmoveto{\pgfqpoint{1.048776in}{1.454848in}}%
\pgfpathlineto{\pgfqpoint{1.084537in}{1.535589in}}%
\pgfpathlineto{\pgfqpoint{1.047208in}{1.566605in}}%
\pgfpathlineto{\pgfqpoint{1.012213in}{1.473422in}}%
\pgfpathclose%
\pgfusepath{fill}%
\end{pgfscope}%
\begin{pgfscope}%
\pgfpathrectangle{\pgfqpoint{0.150000in}{0.150000in}}{\pgfqpoint{2.700000in}{1.950000in}}%
\pgfusepath{clip}%
\pgfsetbuttcap%
\pgfsetroundjoin%
\definecolor{currentfill}{rgb}{0.834743,0.700138,0.710800}%
\pgfsetfillcolor{currentfill}%
\pgfsetlinewidth{0.000000pt}%
\definecolor{currentstroke}{rgb}{0.000000,0.000000,0.000000}%
\pgfsetstrokecolor{currentstroke}%
\pgfsetdash{}{0pt}%
\pgfpathmoveto{\pgfqpoint{2.101812in}{0.888717in}}%
\pgfpathlineto{\pgfqpoint{2.136686in}{0.879058in}}%
\pgfpathlineto{\pgfqpoint{2.098473in}{0.904532in}}%
\pgfpathlineto{\pgfqpoint{2.062861in}{0.902524in}}%
\pgfpathclose%
\pgfusepath{fill}%
\end{pgfscope}%
\begin{pgfscope}%
\pgfpathrectangle{\pgfqpoint{0.150000in}{0.150000in}}{\pgfqpoint{2.700000in}{1.950000in}}%
\pgfusepath{clip}%
\pgfsetbuttcap%
\pgfsetroundjoin%
\definecolor{currentfill}{rgb}{0.847626,0.866391,0.892662}%
\pgfsetfillcolor{currentfill}%
\pgfsetlinewidth{0.000000pt}%
\definecolor{currentstroke}{rgb}{0.000000,0.000000,0.000000}%
\pgfsetstrokecolor{currentstroke}%
\pgfsetdash{}{0pt}%
\pgfpathmoveto{\pgfqpoint{1.762300in}{1.374042in}}%
\pgfpathlineto{\pgfqpoint{1.799306in}{1.386670in}}%
\pgfpathlineto{\pgfqpoint{1.761480in}{1.411533in}}%
\pgfpathlineto{\pgfqpoint{1.724429in}{1.398980in}}%
\pgfpathclose%
\pgfusepath{fill}%
\end{pgfscope}%
\begin{pgfscope}%
\pgfpathrectangle{\pgfqpoint{0.150000in}{0.150000in}}{\pgfqpoint{2.700000in}{1.950000in}}%
\pgfusepath{clip}%
\pgfsetbuttcap%
\pgfsetroundjoin%
\definecolor{currentfill}{rgb}{0.685922,0.724602,0.778753}%
\pgfsetfillcolor{currentfill}%
\pgfsetlinewidth{0.000000pt}%
\definecolor{currentstroke}{rgb}{0.000000,0.000000,0.000000}%
\pgfsetstrokecolor{currentstroke}%
\pgfsetdash{}{0pt}%
\pgfpathmoveto{\pgfqpoint{1.348195in}{1.572875in}}%
\pgfpathlineto{\pgfqpoint{1.386470in}{1.566273in}}%
\pgfpathlineto{\pgfqpoint{1.349093in}{1.597138in}}%
\pgfpathlineto{\pgfqpoint{1.310690in}{1.603862in}}%
\pgfpathclose%
\pgfusepath{fill}%
\end{pgfscope}%
\begin{pgfscope}%
\pgfpathrectangle{\pgfqpoint{0.150000in}{0.150000in}}{\pgfqpoint{2.700000in}{1.950000in}}%
\pgfusepath{clip}%
\pgfsetbuttcap%
\pgfsetroundjoin%
\definecolor{currentfill}{rgb}{0.956311,0.920726,0.923545}%
\pgfsetfillcolor{currentfill}%
\pgfsetlinewidth{0.000000pt}%
\definecolor{currentstroke}{rgb}{0.000000,0.000000,0.000000}%
\pgfsetstrokecolor{currentstroke}%
\pgfsetdash{}{0pt}%
\pgfpathmoveto{\pgfqpoint{1.950080in}{1.094881in}}%
\pgfpathlineto{\pgfqpoint{1.986090in}{1.096211in}}%
\pgfpathlineto{\pgfqpoint{1.949519in}{1.157365in}}%
\pgfpathlineto{\pgfqpoint{1.913086in}{1.150215in}}%
\pgfpathclose%
\pgfusepath{fill}%
\end{pgfscope}%
\begin{pgfscope}%
\pgfpathrectangle{\pgfqpoint{0.150000in}{0.150000in}}{\pgfqpoint{2.700000in}{1.950000in}}%
\pgfusepath{clip}%
\pgfsetbuttcap%
\pgfsetroundjoin%
\definecolor{currentfill}{rgb}{0.810309,0.833670,0.866376}%
\pgfsetfillcolor{currentfill}%
\pgfsetlinewidth{0.000000pt}%
\definecolor{currentstroke}{rgb}{0.000000,0.000000,0.000000}%
\pgfsetstrokecolor{currentstroke}%
\pgfsetdash{}{0pt}%
\pgfpathmoveto{\pgfqpoint{1.649446in}{1.417498in}}%
\pgfpathlineto{\pgfqpoint{1.686653in}{1.423856in}}%
\pgfpathlineto{\pgfqpoint{1.648971in}{1.448670in}}%
\pgfpathlineto{\pgfqpoint{1.611698in}{1.442393in}}%
\pgfpathclose%
\pgfusepath{fill}%
\end{pgfscope}%
\begin{pgfscope}%
\pgfpathrectangle{\pgfqpoint{0.150000in}{0.150000in}}{\pgfqpoint{2.700000in}{1.950000in}}%
\pgfusepath{clip}%
\pgfsetbuttcap%
\pgfsetroundjoin%
\definecolor{currentfill}{rgb}{0.835187,0.855484,0.883900}%
\pgfsetfillcolor{currentfill}%
\pgfsetlinewidth{0.000000pt}%
\definecolor{currentstroke}{rgb}{0.000000,0.000000,0.000000}%
\pgfsetstrokecolor{currentstroke}%
\pgfsetdash{}{0pt}%
\pgfpathmoveto{\pgfqpoint{1.085117in}{1.367890in}}%
\pgfpathlineto{\pgfqpoint{1.122776in}{1.405146in}}%
\pgfpathlineto{\pgfqpoint{1.085730in}{1.430028in}}%
\pgfpathlineto{\pgfqpoint{1.048114in}{1.392822in}}%
\pgfpathclose%
\pgfusepath{fill}%
\end{pgfscope}%
\begin{pgfscope}%
\pgfpathrectangle{\pgfqpoint{0.150000in}{0.150000in}}{\pgfqpoint{2.700000in}{1.950000in}}%
\pgfusepath{clip}%
\pgfsetbuttcap%
\pgfsetroundjoin%
\definecolor{currentfill}{rgb}{0.772993,0.800950,0.840089}%
\pgfsetfillcolor{currentfill}%
\pgfsetlinewidth{0.000000pt}%
\definecolor{currentstroke}{rgb}{0.000000,0.000000,0.000000}%
\pgfsetstrokecolor{currentstroke}%
\pgfsetdash{}{0pt}%
\pgfpathmoveto{\pgfqpoint{1.536486in}{1.460994in}}%
\pgfpathlineto{\pgfqpoint{1.574046in}{1.467227in}}%
\pgfpathlineto{\pgfqpoint{1.536486in}{1.491998in}}%
\pgfpathlineto{\pgfqpoint{1.498863in}{1.485847in}}%
\pgfpathclose%
\pgfusepath{fill}%
\end{pgfscope}%
\begin{pgfscope}%
\pgfpathrectangle{\pgfqpoint{0.150000in}{0.150000in}}{\pgfqpoint{2.700000in}{1.950000in}}%
\pgfusepath{clip}%
\pgfsetbuttcap%
\pgfsetroundjoin%
\definecolor{currentfill}{rgb}{0.823346,0.679458,0.690855}%
\pgfsetfillcolor{currentfill}%
\pgfsetlinewidth{0.000000pt}%
\definecolor{currentstroke}{rgb}{0.000000,0.000000,0.000000}%
\pgfsetstrokecolor{currentstroke}%
\pgfsetdash{}{0pt}%
\pgfpathmoveto{\pgfqpoint{1.989315in}{0.863072in}}%
\pgfpathlineto{\pgfqpoint{2.026155in}{0.882834in}}%
\pgfpathlineto{\pgfqpoint{1.985868in}{0.861529in}}%
\pgfpathlineto{\pgfqpoint{1.949168in}{0.841843in}}%
\pgfpathclose%
\pgfusepath{fill}%
\end{pgfscope}%
\begin{pgfscope}%
\pgfpathrectangle{\pgfqpoint{0.150000in}{0.150000in}}{\pgfqpoint{2.700000in}{1.950000in}}%
\pgfusepath{clip}%
\pgfsetbuttcap%
\pgfsetroundjoin%
\definecolor{currentfill}{rgb}{0.866284,0.882751,0.905806}%
\pgfsetfillcolor{currentfill}%
\pgfsetlinewidth{0.000000pt}%
\definecolor{currentstroke}{rgb}{0.000000,0.000000,0.000000}%
\pgfsetstrokecolor{currentstroke}%
\pgfsetdash{}{0pt}%
\pgfpathmoveto{\pgfqpoint{1.875437in}{1.342896in}}%
\pgfpathlineto{\pgfqpoint{1.912198in}{1.355607in}}%
\pgfpathlineto{\pgfqpoint{1.874205in}{1.380526in}}%
\pgfpathlineto{\pgfqpoint{1.837399in}{1.367890in}}%
\pgfpathclose%
\pgfusepath{fill}%
\end{pgfscope}%
\begin{pgfscope}%
\pgfpathrectangle{\pgfqpoint{0.150000in}{0.150000in}}{\pgfqpoint{2.700000in}{1.950000in}}%
\pgfusepath{clip}%
\pgfsetbuttcap%
\pgfsetroundjoin%
\definecolor{currentfill}{rgb}{0.897381,0.910018,0.927711}%
\pgfsetfillcolor{currentfill}%
\pgfsetlinewidth{0.000000pt}%
\definecolor{currentstroke}{rgb}{0.000000,0.000000,0.000000}%
\pgfsetstrokecolor{currentstroke}%
\pgfsetdash{}{0pt}%
\pgfpathmoveto{\pgfqpoint{1.912708in}{1.293354in}}%
\pgfpathlineto{\pgfqpoint{1.949578in}{1.312274in}}%
\pgfpathlineto{\pgfqpoint{1.912198in}{1.355607in}}%
\pgfpathlineto{\pgfqpoint{1.875437in}{1.342896in}}%
\pgfpathclose%
\pgfusepath{fill}%
\end{pgfscope}%
\begin{pgfscope}%
\pgfpathrectangle{\pgfqpoint{0.150000in}{0.150000in}}{\pgfqpoint{2.700000in}{1.950000in}}%
\pgfusepath{clip}%
\pgfsetbuttcap%
\pgfsetroundjoin%
\definecolor{currentfill}{rgb}{0.698361,0.735509,0.787515}%
\pgfsetfillcolor{currentfill}%
\pgfsetlinewidth{0.000000pt}%
\definecolor{currentstroke}{rgb}{0.000000,0.000000,0.000000}%
\pgfsetstrokecolor{currentstroke}%
\pgfsetdash{}{0pt}%
\pgfpathmoveto{\pgfqpoint{1.385664in}{1.548108in}}%
\pgfpathlineto{\pgfqpoint{1.423834in}{1.541600in}}%
\pgfpathlineto{\pgfqpoint{1.386470in}{1.566273in}}%
\pgfpathlineto{\pgfqpoint{1.348195in}{1.572875in}}%
\pgfpathclose%
\pgfusepath{fill}%
\end{pgfscope}%
\begin{pgfscope}%
\pgfpathrectangle{\pgfqpoint{0.150000in}{0.150000in}}{\pgfqpoint{2.700000in}{1.950000in}}%
\pgfusepath{clip}%
\pgfsetbuttcap%
\pgfsetroundjoin%
\definecolor{currentfill}{rgb}{0.760555,0.790043,0.831327}%
\pgfsetfillcolor{currentfill}%
\pgfsetlinewidth{0.000000pt}%
\definecolor{currentstroke}{rgb}{0.000000,0.000000,0.000000}%
\pgfsetstrokecolor{currentstroke}%
\pgfsetdash{}{0pt}%
\pgfpathmoveto{\pgfqpoint{1.085730in}{1.430028in}}%
\pgfpathlineto{\pgfqpoint{1.121680in}{1.510773in}}%
\pgfpathlineto{\pgfqpoint{1.084537in}{1.535589in}}%
\pgfpathlineto{\pgfqpoint{1.048776in}{1.454848in}}%
\pgfpathclose%
\pgfusepath{fill}%
\end{pgfscope}%
\begin{pgfscope}%
\pgfpathrectangle{\pgfqpoint{0.150000in}{0.150000in}}{\pgfqpoint{2.700000in}{1.950000in}}%
\pgfusepath{clip}%
\pgfsetbuttcap%
\pgfsetroundjoin%
\definecolor{currentfill}{rgb}{0.822748,0.844577,0.875138}%
\pgfsetfillcolor{currentfill}%
\pgfsetlinewidth{0.000000pt}%
\definecolor{currentstroke}{rgb}{0.000000,0.000000,0.000000}%
\pgfsetstrokecolor{currentstroke}%
\pgfsetdash{}{0pt}%
\pgfpathmoveto{\pgfqpoint{1.687288in}{1.392540in}}%
\pgfpathlineto{\pgfqpoint{1.724429in}{1.398980in}}%
\pgfpathlineto{\pgfqpoint{1.686653in}{1.423856in}}%
\pgfpathlineto{\pgfqpoint{1.649446in}{1.417498in}}%
\pgfpathclose%
\pgfusepath{fill}%
\end{pgfscope}%
\begin{pgfscope}%
\pgfpathrectangle{\pgfqpoint{0.150000in}{0.150000in}}{\pgfqpoint{2.700000in}{1.950000in}}%
\pgfusepath{clip}%
\pgfsetbuttcap%
\pgfsetroundjoin%
\definecolor{currentfill}{rgb}{0.953355,0.959099,0.967142}%
\pgfsetfillcolor{currentfill}%
\pgfsetlinewidth{0.000000pt}%
\definecolor{currentstroke}{rgb}{0.000000,0.000000,0.000000}%
\pgfsetstrokecolor{currentstroke}%
\pgfsetdash{}{0pt}%
\pgfpathmoveto{\pgfqpoint{1.912789in}{1.218787in}}%
\pgfpathlineto{\pgfqpoint{1.949431in}{1.231784in}}%
\pgfpathlineto{\pgfqpoint{1.912708in}{1.293354in}}%
\pgfpathlineto{\pgfqpoint{1.875704in}{1.274365in}}%
\pgfpathclose%
\pgfusepath{fill}%
\end{pgfscope}%
\begin{pgfscope}%
\pgfpathrectangle{\pgfqpoint{0.150000in}{0.150000in}}{\pgfqpoint{2.700000in}{1.950000in}}%
\pgfusepath{clip}%
\pgfsetbuttcap%
\pgfsetroundjoin%
\definecolor{currentfill}{rgb}{0.830944,0.693244,0.704151}%
\pgfsetfillcolor{currentfill}%
\pgfsetlinewidth{0.000000pt}%
\definecolor{currentstroke}{rgb}{0.000000,0.000000,0.000000}%
\pgfsetstrokecolor{currentstroke}%
\pgfsetdash{}{0pt}%
\pgfpathmoveto{\pgfqpoint{1.912335in}{0.822085in}}%
\pgfpathlineto{\pgfqpoint{1.949168in}{0.841843in}}%
\pgfpathlineto{\pgfqpoint{1.913114in}{0.920195in}}%
\pgfpathlineto{\pgfqpoint{1.876071in}{0.900503in}}%
\pgfpathclose%
\pgfusepath{fill}%
\end{pgfscope}%
\begin{pgfscope}%
\pgfpathrectangle{\pgfqpoint{0.150000in}{0.150000in}}{\pgfqpoint{2.700000in}{1.950000in}}%
\pgfusepath{clip}%
\pgfsetbuttcap%
\pgfsetroundjoin%
\definecolor{currentfill}{rgb}{0.853845,0.871844,0.897044}%
\pgfsetfillcolor{currentfill}%
\pgfsetlinewidth{0.000000pt}%
\definecolor{currentstroke}{rgb}{0.000000,0.000000,0.000000}%
\pgfsetstrokecolor{currentstroke}%
\pgfsetdash{}{0pt}%
\pgfpathmoveto{\pgfqpoint{1.800417in}{1.355193in}}%
\pgfpathlineto{\pgfqpoint{1.837399in}{1.367890in}}%
\pgfpathlineto{\pgfqpoint{1.799306in}{1.386670in}}%
\pgfpathlineto{\pgfqpoint{1.762300in}{1.374042in}}%
\pgfpathclose%
\pgfusepath{fill}%
\end{pgfscope}%
\begin{pgfscope}%
\pgfpathrectangle{\pgfqpoint{0.150000in}{0.150000in}}{\pgfqpoint{2.700000in}{1.950000in}}%
\pgfusepath{clip}%
\pgfsetbuttcap%
\pgfsetroundjoin%
\definecolor{currentfill}{rgb}{0.816529,0.839124,0.870757}%
\pgfsetfillcolor{currentfill}%
\pgfsetlinewidth{0.000000pt}%
\definecolor{currentstroke}{rgb}{0.000000,0.000000,0.000000}%
\pgfsetstrokecolor{currentstroke}%
\pgfsetdash{}{0pt}%
\pgfpathmoveto{\pgfqpoint{1.008077in}{1.404907in}}%
\pgfpathlineto{\pgfqpoint{1.048114in}{1.392822in}}%
\pgfpathlineto{\pgfqpoint{1.011503in}{1.411533in}}%
\pgfpathlineto{\pgfqpoint{0.971043in}{1.429878in}}%
\pgfpathclose%
\pgfusepath{fill}%
\end{pgfscope}%
\begin{pgfscope}%
\pgfpathrectangle{\pgfqpoint{0.150000in}{0.150000in}}{\pgfqpoint{2.700000in}{1.950000in}}%
\pgfusepath{clip}%
\pgfsetbuttcap%
\pgfsetroundjoin%
\definecolor{currentfill}{rgb}{0.891728,0.803539,0.810524}%
\pgfsetfillcolor{currentfill}%
\pgfsetlinewidth{0.000000pt}%
\definecolor{currentstroke}{rgb}{0.000000,0.000000,0.000000}%
\pgfsetstrokecolor{currentstroke}%
\pgfsetdash{}{0pt}%
\pgfpathmoveto{\pgfqpoint{1.949786in}{0.933917in}}%
\pgfpathlineto{\pgfqpoint{1.986539in}{0.953465in}}%
\pgfpathlineto{\pgfqpoint{1.950880in}{1.038193in}}%
\pgfpathlineto{\pgfqpoint{1.914113in}{1.024684in}}%
\pgfpathclose%
\pgfusepath{fill}%
\end{pgfscope}%
\begin{pgfscope}%
\pgfpathrectangle{\pgfqpoint{0.150000in}{0.150000in}}{\pgfqpoint{2.700000in}{1.950000in}}%
\pgfusepath{clip}%
\pgfsetbuttcap%
\pgfsetroundjoin%
\definecolor{currentfill}{rgb}{0.933517,0.879366,0.883655}%
\pgfsetfillcolor{currentfill}%
\pgfsetlinewidth{0.000000pt}%
\definecolor{currentstroke}{rgb}{0.000000,0.000000,0.000000}%
\pgfsetstrokecolor{currentstroke}%
\pgfsetdash{}{0pt}%
\pgfpathmoveto{\pgfqpoint{1.950880in}{1.038193in}}%
\pgfpathlineto{\pgfqpoint{1.986956in}{1.039725in}}%
\pgfpathlineto{\pgfqpoint{1.950080in}{1.094881in}}%
\pgfpathlineto{\pgfqpoint{1.913599in}{1.087535in}}%
\pgfpathclose%
\pgfusepath{fill}%
\end{pgfscope}%
\begin{pgfscope}%
\pgfpathrectangle{\pgfqpoint{0.150000in}{0.150000in}}{\pgfqpoint{2.700000in}{1.950000in}}%
\pgfusepath{clip}%
\pgfsetbuttcap%
\pgfsetroundjoin%
\definecolor{currentfill}{rgb}{0.834743,0.700138,0.710800}%
\pgfsetfillcolor{currentfill}%
\pgfsetlinewidth{0.000000pt}%
\definecolor{currentstroke}{rgb}{0.000000,0.000000,0.000000}%
\pgfsetstrokecolor{currentstroke}%
\pgfsetdash{}{0pt}%
\pgfpathmoveto{\pgfqpoint{2.140604in}{0.868955in}}%
\pgfpathlineto{\pgfqpoint{2.175360in}{0.859370in}}%
\pgfpathlineto{\pgfqpoint{2.136686in}{0.879058in}}%
\pgfpathlineto{\pgfqpoint{2.101812in}{0.888717in}}%
\pgfpathclose%
\pgfusepath{fill}%
\end{pgfscope}%
\begin{pgfscope}%
\pgfpathrectangle{\pgfqpoint{0.150000in}{0.150000in}}{\pgfqpoint{2.700000in}{1.950000in}}%
\pgfusepath{clip}%
\pgfsetbuttcap%
\pgfsetroundjoin%
\definecolor{currentfill}{rgb}{0.779213,0.806403,0.844470}%
\pgfsetfillcolor{currentfill}%
\pgfsetlinewidth{0.000000pt}%
\definecolor{currentstroke}{rgb}{0.000000,0.000000,0.000000}%
\pgfsetstrokecolor{currentstroke}%
\pgfsetdash{}{0pt}%
\pgfpathmoveto{\pgfqpoint{1.574204in}{1.436079in}}%
\pgfpathlineto{\pgfqpoint{1.611698in}{1.442393in}}%
\pgfpathlineto{\pgfqpoint{1.574046in}{1.467227in}}%
\pgfpathlineto{\pgfqpoint{1.536486in}{1.460994in}}%
\pgfpathclose%
\pgfusepath{fill}%
\end{pgfscope}%
\begin{pgfscope}%
\pgfpathrectangle{\pgfqpoint{0.150000in}{0.150000in}}{\pgfqpoint{2.700000in}{1.950000in}}%
\pgfusepath{clip}%
\pgfsetbuttcap%
\pgfsetroundjoin%
\definecolor{currentfill}{rgb}{0.994301,0.989660,0.990028}%
\pgfsetfillcolor{currentfill}%
\pgfsetlinewidth{0.000000pt}%
\definecolor{currentstroke}{rgb}{0.000000,0.000000,0.000000}%
\pgfsetstrokecolor{currentstroke}%
\pgfsetdash{}{0pt}%
\pgfpathmoveto{\pgfqpoint{1.913086in}{1.150215in}}%
\pgfpathlineto{\pgfqpoint{1.949519in}{1.157365in}}%
\pgfpathlineto{\pgfqpoint{1.912789in}{1.218787in}}%
\pgfpathlineto{\pgfqpoint{1.875972in}{1.205727in}}%
\pgfpathclose%
\pgfusepath{fill}%
\end{pgfscope}%
\begin{pgfscope}%
\pgfpathrectangle{\pgfqpoint{0.150000in}{0.150000in}}{\pgfqpoint{2.700000in}{1.950000in}}%
\pgfusepath{clip}%
\pgfsetbuttcap%
\pgfsetroundjoin%
\definecolor{currentfill}{rgb}{0.841406,0.860938,0.888281}%
\pgfsetfillcolor{currentfill}%
\pgfsetlinewidth{0.000000pt}%
\definecolor{currentstroke}{rgb}{0.000000,0.000000,0.000000}%
\pgfsetstrokecolor{currentstroke}%
\pgfsetdash{}{0pt}%
\pgfpathmoveto{\pgfqpoint{1.121976in}{1.349041in}}%
\pgfpathlineto{\pgfqpoint{1.159700in}{1.386367in}}%
\pgfpathlineto{\pgfqpoint{1.122776in}{1.405146in}}%
\pgfpathlineto{\pgfqpoint{1.085117in}{1.367890in}}%
\pgfpathclose%
\pgfusepath{fill}%
\end{pgfscope}%
\begin{pgfscope}%
\pgfpathrectangle{\pgfqpoint{0.150000in}{0.150000in}}{\pgfqpoint{2.700000in}{1.950000in}}%
\pgfusepath{clip}%
\pgfsetbuttcap%
\pgfsetroundjoin%
\definecolor{currentfill}{rgb}{0.710800,0.746415,0.796278}%
\pgfsetfillcolor{currentfill}%
\pgfsetlinewidth{0.000000pt}%
\definecolor{currentstroke}{rgb}{0.000000,0.000000,0.000000}%
\pgfsetstrokecolor{currentstroke}%
\pgfsetdash{}{0pt}%
\pgfpathmoveto{\pgfqpoint{1.423293in}{1.517023in}}%
\pgfpathlineto{\pgfqpoint{1.461291in}{1.516865in}}%
\pgfpathlineto{\pgfqpoint{1.423834in}{1.541600in}}%
\pgfpathlineto{\pgfqpoint{1.385664in}{1.548108in}}%
\pgfpathclose%
\pgfusepath{fill}%
\end{pgfscope}%
\begin{pgfscope}%
\pgfpathrectangle{\pgfqpoint{0.150000in}{0.150000in}}{\pgfqpoint{2.700000in}{1.950000in}}%
\pgfusepath{clip}%
\pgfsetbuttcap%
\pgfsetroundjoin%
\definecolor{currentfill}{rgb}{0.772993,0.800950,0.840089}%
\pgfsetfillcolor{currentfill}%
\pgfsetlinewidth{0.000000pt}%
\definecolor{currentstroke}{rgb}{0.000000,0.000000,0.000000}%
\pgfsetstrokecolor{currentstroke}%
\pgfsetdash{}{0pt}%
\pgfpathmoveto{\pgfqpoint{1.122776in}{1.405146in}}%
\pgfpathlineto{\pgfqpoint{1.159133in}{1.479659in}}%
\pgfpathlineto{\pgfqpoint{1.121680in}{1.510773in}}%
\pgfpathlineto{\pgfqpoint{1.085730in}{1.430028in}}%
\pgfpathclose%
\pgfusepath{fill}%
\end{pgfscope}%
\begin{pgfscope}%
\pgfpathrectangle{\pgfqpoint{0.150000in}{0.150000in}}{\pgfqpoint{2.700000in}{1.950000in}}%
\pgfusepath{clip}%
\pgfsetbuttcap%
\pgfsetroundjoin%
\definecolor{currentfill}{rgb}{0.542877,0.599173,0.677987}%
\pgfsetfillcolor{currentfill}%
\pgfsetlinewidth{0.000000pt}%
\definecolor{currentstroke}{rgb}{0.000000,0.000000,0.000000}%
\pgfsetstrokecolor{currentstroke}%
\pgfsetdash{}{0pt}%
\pgfpathmoveto{\pgfqpoint{1.158294in}{1.735595in}}%
\pgfpathlineto{\pgfqpoint{1.198673in}{1.690250in}}%
\pgfpathlineto{\pgfqpoint{1.161606in}{1.714727in}}%
\pgfpathlineto{\pgfqpoint{1.120758in}{1.766604in}}%
\pgfpathclose%
\pgfusepath{fill}%
\end{pgfscope}%
\begin{pgfscope}%
\pgfpathrectangle{\pgfqpoint{0.150000in}{0.150000in}}{\pgfqpoint{2.700000in}{1.950000in}}%
\pgfusepath{clip}%
\pgfsetbuttcap%
\pgfsetroundjoin%
\definecolor{currentfill}{rgb}{0.823346,0.679458,0.690855}%
\pgfsetfillcolor{currentfill}%
\pgfsetlinewidth{0.000000pt}%
\definecolor{currentstroke}{rgb}{0.000000,0.000000,0.000000}%
\pgfsetstrokecolor{currentstroke}%
\pgfsetdash{}{0pt}%
\pgfpathmoveto{\pgfqpoint{1.952339in}{0.843238in}}%
\pgfpathlineto{\pgfqpoint{1.989315in}{0.863072in}}%
\pgfpathlineto{\pgfqpoint{1.949168in}{0.841843in}}%
\pgfpathlineto{\pgfqpoint{1.912335in}{0.822085in}}%
\pgfpathclose%
\pgfusepath{fill}%
\end{pgfscope}%
\begin{pgfscope}%
\pgfpathrectangle{\pgfqpoint{0.150000in}{0.150000in}}{\pgfqpoint{2.700000in}{1.950000in}}%
\pgfusepath{clip}%
\pgfsetbuttcap%
\pgfsetroundjoin%
\definecolor{currentfill}{rgb}{0.828968,0.850031,0.879519}%
\pgfsetfillcolor{currentfill}%
\pgfsetlinewidth{0.000000pt}%
\definecolor{currentstroke}{rgb}{0.000000,0.000000,0.000000}%
\pgfsetstrokecolor{currentstroke}%
\pgfsetdash{}{0pt}%
\pgfpathmoveto{\pgfqpoint{1.725224in}{1.367519in}}%
\pgfpathlineto{\pgfqpoint{1.762300in}{1.374042in}}%
\pgfpathlineto{\pgfqpoint{1.724429in}{1.398980in}}%
\pgfpathlineto{\pgfqpoint{1.687288in}{1.392540in}}%
\pgfpathclose%
\pgfusepath{fill}%
\end{pgfscope}%
\begin{pgfscope}%
\pgfpathrectangle{\pgfqpoint{0.150000in}{0.150000in}}{\pgfqpoint{2.700000in}{1.950000in}}%
\pgfusepath{clip}%
\pgfsetbuttcap%
\pgfsetroundjoin%
\definecolor{currentfill}{rgb}{0.828968,0.850031,0.879519}%
\pgfsetfillcolor{currentfill}%
\pgfsetlinewidth{0.000000pt}%
\definecolor{currentstroke}{rgb}{0.000000,0.000000,0.000000}%
\pgfsetstrokecolor{currentstroke}%
\pgfsetdash{}{0pt}%
\pgfpathmoveto{\pgfqpoint{1.045204in}{1.379874in}}%
\pgfpathlineto{\pgfqpoint{1.085117in}{1.367890in}}%
\pgfpathlineto{\pgfqpoint{1.048114in}{1.392822in}}%
\pgfpathlineto{\pgfqpoint{1.008077in}{1.404907in}}%
\pgfpathclose%
\pgfusepath{fill}%
\end{pgfscope}%
\begin{pgfscope}%
\pgfpathrectangle{\pgfqpoint{0.150000in}{0.150000in}}{\pgfqpoint{2.700000in}{1.950000in}}%
\pgfusepath{clip}%
\pgfsetbuttcap%
\pgfsetroundjoin%
\definecolor{currentfill}{rgb}{0.967708,0.941406,0.943490}%
\pgfsetfillcolor{currentfill}%
\pgfsetlinewidth{0.000000pt}%
\definecolor{currentstroke}{rgb}{0.000000,0.000000,0.000000}%
\pgfsetstrokecolor{currentstroke}%
\pgfsetdash{}{0pt}%
\pgfpathmoveto{\pgfqpoint{1.913599in}{1.087535in}}%
\pgfpathlineto{\pgfqpoint{1.950080in}{1.094881in}}%
\pgfpathlineto{\pgfqpoint{1.913086in}{1.150215in}}%
\pgfpathlineto{\pgfqpoint{1.876240in}{1.136980in}}%
\pgfpathclose%
\pgfusepath{fill}%
\end{pgfscope}%
\begin{pgfscope}%
\pgfpathrectangle{\pgfqpoint{0.150000in}{0.150000in}}{\pgfqpoint{2.700000in}{1.950000in}}%
\pgfusepath{clip}%
\pgfsetbuttcap%
\pgfsetroundjoin%
\definecolor{currentfill}{rgb}{0.891161,0.904565,0.923330}%
\pgfsetfillcolor{currentfill}%
\pgfsetlinewidth{0.000000pt}%
\definecolor{currentstroke}{rgb}{0.000000,0.000000,0.000000}%
\pgfsetstrokecolor{currentstroke}%
\pgfsetdash{}{0pt}%
\pgfpathmoveto{\pgfqpoint{1.875704in}{1.274365in}}%
\pgfpathlineto{\pgfqpoint{1.912708in}{1.293354in}}%
\pgfpathlineto{\pgfqpoint{1.875437in}{1.342896in}}%
\pgfpathlineto{\pgfqpoint{1.838500in}{1.330123in}}%
\pgfpathclose%
\pgfusepath{fill}%
\end{pgfscope}%
\begin{pgfscope}%
\pgfpathrectangle{\pgfqpoint{0.150000in}{0.150000in}}{\pgfqpoint{2.700000in}{1.950000in}}%
\pgfusepath{clip}%
\pgfsetbuttcap%
\pgfsetroundjoin%
\definecolor{currentfill}{rgb}{0.860064,0.877298,0.901425}%
\pgfsetfillcolor{currentfill}%
\pgfsetlinewidth{0.000000pt}%
\definecolor{currentstroke}{rgb}{0.000000,0.000000,0.000000}%
\pgfsetstrokecolor{currentstroke}%
\pgfsetdash{}{0pt}%
\pgfpathmoveto{\pgfqpoint{1.838500in}{1.330123in}}%
\pgfpathlineto{\pgfqpoint{1.875437in}{1.342896in}}%
\pgfpathlineto{\pgfqpoint{1.837399in}{1.367890in}}%
\pgfpathlineto{\pgfqpoint{1.800417in}{1.355193in}}%
\pgfpathclose%
\pgfusepath{fill}%
\end{pgfscope}%
\begin{pgfscope}%
\pgfpathrectangle{\pgfqpoint{0.150000in}{0.150000in}}{\pgfqpoint{2.700000in}{1.950000in}}%
\pgfusepath{clip}%
\pgfsetbuttcap%
\pgfsetroundjoin%
\definecolor{currentfill}{rgb}{0.729458,0.762776,0.809421}%
\pgfsetfillcolor{currentfill}%
\pgfsetlinewidth{0.000000pt}%
\definecolor{currentstroke}{rgb}{0.000000,0.000000,0.000000}%
\pgfsetstrokecolor{currentstroke}%
\pgfsetdash{}{0pt}%
\pgfpathmoveto{\pgfqpoint{1.460929in}{1.492138in}}%
\pgfpathlineto{\pgfqpoint{1.498863in}{1.485847in}}%
\pgfpathlineto{\pgfqpoint{1.461291in}{1.516865in}}%
\pgfpathlineto{\pgfqpoint{1.423293in}{1.517023in}}%
\pgfpathclose%
\pgfusepath{fill}%
\end{pgfscope}%
\begin{pgfscope}%
\pgfpathrectangle{\pgfqpoint{0.150000in}{0.150000in}}{\pgfqpoint{2.700000in}{1.950000in}}%
\pgfusepath{clip}%
\pgfsetbuttcap%
\pgfsetroundjoin%
\definecolor{currentfill}{rgb}{0.791651,0.817310,0.853232}%
\pgfsetfillcolor{currentfill}%
\pgfsetlinewidth{0.000000pt}%
\definecolor{currentstroke}{rgb}{0.000000,0.000000,0.000000}%
\pgfsetstrokecolor{currentstroke}%
\pgfsetdash{}{0pt}%
\pgfpathmoveto{\pgfqpoint{1.612017in}{1.411102in}}%
\pgfpathlineto{\pgfqpoint{1.649446in}{1.417498in}}%
\pgfpathlineto{\pgfqpoint{1.611698in}{1.442393in}}%
\pgfpathlineto{\pgfqpoint{1.574204in}{1.436079in}}%
\pgfpathclose%
\pgfusepath{fill}%
\end{pgfscope}%
\begin{pgfscope}%
\pgfpathrectangle{\pgfqpoint{0.150000in}{0.150000in}}{\pgfqpoint{2.700000in}{1.950000in}}%
\pgfusepath{clip}%
\pgfsetbuttcap%
\pgfsetroundjoin%
\definecolor{currentfill}{rgb}{0.561535,0.615533,0.691131}%
\pgfsetfillcolor{currentfill}%
\pgfsetlinewidth{0.000000pt}%
\definecolor{currentstroke}{rgb}{0.000000,0.000000,0.000000}%
\pgfsetstrokecolor{currentstroke}%
\pgfsetdash{}{0pt}%
\pgfpathmoveto{\pgfqpoint{1.195880in}{1.704544in}}%
\pgfpathlineto{\pgfqpoint{1.236004in}{1.659401in}}%
\pgfpathlineto{\pgfqpoint{1.198673in}{1.690250in}}%
\pgfpathlineto{\pgfqpoint{1.158294in}{1.735595in}}%
\pgfpathclose%
\pgfusepath{fill}%
\end{pgfscope}%
\begin{pgfscope}%
\pgfpathrectangle{\pgfqpoint{0.150000in}{0.150000in}}{\pgfqpoint{2.700000in}{1.950000in}}%
\pgfusepath{clip}%
\pgfsetbuttcap%
\pgfsetroundjoin%
\definecolor{currentfill}{rgb}{0.754335,0.784589,0.826945}%
\pgfsetfillcolor{currentfill}%
\pgfsetlinewidth{0.000000pt}%
\definecolor{currentstroke}{rgb}{0.000000,0.000000,0.000000}%
\pgfsetstrokecolor{currentstroke}%
\pgfsetdash{}{0pt}%
\pgfpathmoveto{\pgfqpoint{0.893810in}{1.460954in}}%
\pgfpathlineto{\pgfqpoint{0.934447in}{1.448586in}}%
\pgfpathlineto{\pgfqpoint{0.897619in}{1.473426in}}%
\pgfpathlineto{\pgfqpoint{0.856860in}{1.485895in}}%
\pgfpathclose%
\pgfusepath{fill}%
\end{pgfscope}%
\begin{pgfscope}%
\pgfpathrectangle{\pgfqpoint{0.150000in}{0.150000in}}{\pgfqpoint{2.700000in}{1.950000in}}%
\pgfusepath{clip}%
\pgfsetbuttcap%
\pgfsetroundjoin%
\definecolor{currentfill}{rgb}{0.834743,0.700138,0.710800}%
\pgfsetfillcolor{currentfill}%
\pgfsetlinewidth{0.000000pt}%
\definecolor{currentstroke}{rgb}{0.000000,0.000000,0.000000}%
\pgfsetstrokecolor{currentstroke}%
\pgfsetdash{}{0pt}%
\pgfpathmoveto{\pgfqpoint{1.875561in}{0.808090in}}%
\pgfpathlineto{\pgfqpoint{1.912335in}{0.822085in}}%
\pgfpathlineto{\pgfqpoint{1.876071in}{0.900503in}}%
\pgfpathlineto{\pgfqpoint{1.838892in}{0.880739in}}%
\pgfpathclose%
\pgfusepath{fill}%
\end{pgfscope}%
\begin{pgfscope}%
\pgfpathrectangle{\pgfqpoint{0.150000in}{0.150000in}}{\pgfqpoint{2.700000in}{1.950000in}}%
\pgfusepath{clip}%
\pgfsetbuttcap%
\pgfsetroundjoin%
\definecolor{currentfill}{rgb}{0.785432,0.811857,0.848851}%
\pgfsetfillcolor{currentfill}%
\pgfsetlinewidth{0.000000pt}%
\definecolor{currentstroke}{rgb}{0.000000,0.000000,0.000000}%
\pgfsetstrokecolor{currentstroke}%
\pgfsetdash{}{0pt}%
\pgfpathmoveto{\pgfqpoint{1.159700in}{1.386367in}}%
\pgfpathlineto{\pgfqpoint{1.196441in}{1.454725in}}%
\pgfpathlineto{\pgfqpoint{1.159133in}{1.479659in}}%
\pgfpathlineto{\pgfqpoint{1.122776in}{1.405146in}}%
\pgfpathclose%
\pgfusepath{fill}%
\end{pgfscope}%
\begin{pgfscope}%
\pgfpathrectangle{\pgfqpoint{0.150000in}{0.150000in}}{\pgfqpoint{2.700000in}{1.950000in}}%
\pgfusepath{clip}%
\pgfsetbuttcap%
\pgfsetroundjoin%
\definecolor{currentfill}{rgb}{0.617509,0.664614,0.730561}%
\pgfsetfillcolor{currentfill}%
\pgfsetlinewidth{0.000000pt}%
\definecolor{currentstroke}{rgb}{0.000000,0.000000,0.000000}%
\pgfsetstrokecolor{currentstroke}%
\pgfsetdash{}{0pt}%
\pgfpathmoveto{\pgfqpoint{1.047208in}{1.566605in}}%
\pgfpathlineto{\pgfqpoint{1.082818in}{1.654385in}}%
\pgfpathlineto{\pgfqpoint{1.045347in}{1.685438in}}%
\pgfpathlineto{\pgfqpoint{1.010229in}{1.591304in}}%
\pgfpathclose%
\pgfusepath{fill}%
\end{pgfscope}%
\begin{pgfscope}%
\pgfpathrectangle{\pgfqpoint{0.150000in}{0.150000in}}{\pgfqpoint{2.700000in}{1.950000in}}%
\pgfusepath{clip}%
\pgfsetbuttcap%
\pgfsetroundjoin%
\definecolor{currentfill}{rgb}{0.847626,0.866391,0.892662}%
\pgfsetfillcolor{currentfill}%
\pgfsetlinewidth{0.000000pt}%
\definecolor{currentstroke}{rgb}{0.000000,0.000000,0.000000}%
\pgfsetstrokecolor{currentstroke}%
\pgfsetdash{}{0pt}%
\pgfpathmoveto{\pgfqpoint{1.159186in}{1.323978in}}%
\pgfpathlineto{\pgfqpoint{1.196953in}{1.361353in}}%
\pgfpathlineto{\pgfqpoint{1.159700in}{1.386367in}}%
\pgfpathlineto{\pgfqpoint{1.121976in}{1.349041in}}%
\pgfpathclose%
\pgfusepath{fill}%
\end{pgfscope}%
\begin{pgfscope}%
\pgfpathrectangle{\pgfqpoint{0.150000in}{0.150000in}}{\pgfqpoint{2.700000in}{1.950000in}}%
\pgfusepath{clip}%
\pgfsetbuttcap%
\pgfsetroundjoin%
\definecolor{currentfill}{rgb}{0.853738,0.734605,0.744041}%
\pgfsetfillcolor{currentfill}%
\pgfsetlinewidth{0.000000pt}%
\definecolor{currentstroke}{rgb}{0.000000,0.000000,0.000000}%
\pgfsetstrokecolor{currentstroke}%
\pgfsetdash{}{0pt}%
\pgfpathmoveto{\pgfqpoint{2.066610in}{0.898467in}}%
\pgfpathlineto{\pgfqpoint{2.101812in}{0.888717in}}%
\pgfpathlineto{\pgfqpoint{2.062861in}{0.902524in}}%
\pgfpathlineto{\pgfqpoint{2.026155in}{0.882834in}}%
\pgfpathclose%
\pgfusepath{fill}%
\end{pgfscope}%
\begin{pgfscope}%
\pgfpathrectangle{\pgfqpoint{0.150000in}{0.150000in}}{\pgfqpoint{2.700000in}{1.950000in}}%
\pgfusepath{clip}%
\pgfsetbuttcap%
\pgfsetroundjoin%
\definecolor{currentfill}{rgb}{0.940916,0.948192,0.958379}%
\pgfsetfillcolor{currentfill}%
\pgfsetlinewidth{0.000000pt}%
\definecolor{currentstroke}{rgb}{0.000000,0.000000,0.000000}%
\pgfsetstrokecolor{currentstroke}%
\pgfsetdash{}{0pt}%
\pgfpathmoveto{\pgfqpoint{1.875972in}{1.205727in}}%
\pgfpathlineto{\pgfqpoint{1.912789in}{1.218787in}}%
\pgfpathlineto{\pgfqpoint{1.875704in}{1.274365in}}%
\pgfpathlineto{\pgfqpoint{1.838739in}{1.261418in}}%
\pgfpathclose%
\pgfusepath{fill}%
\end{pgfscope}%
\begin{pgfscope}%
\pgfpathrectangle{\pgfqpoint{0.150000in}{0.150000in}}{\pgfqpoint{2.700000in}{1.950000in}}%
\pgfusepath{clip}%
\pgfsetbuttcap%
\pgfsetroundjoin%
\definecolor{currentfill}{rgb}{0.895527,0.810432,0.817172}%
\pgfsetfillcolor{currentfill}%
\pgfsetlinewidth{0.000000pt}%
\definecolor{currentstroke}{rgb}{0.000000,0.000000,0.000000}%
\pgfsetstrokecolor{currentstroke}%
\pgfsetdash{}{0pt}%
\pgfpathmoveto{\pgfqpoint{1.913114in}{0.920195in}}%
\pgfpathlineto{\pgfqpoint{1.949786in}{0.933917in}}%
\pgfpathlineto{\pgfqpoint{1.914113in}{1.024684in}}%
\pgfpathlineto{\pgfqpoint{1.876973in}{1.005132in}}%
\pgfpathclose%
\pgfusepath{fill}%
\end{pgfscope}%
\begin{pgfscope}%
\pgfpathrectangle{\pgfqpoint{0.150000in}{0.150000in}}{\pgfqpoint{2.700000in}{1.950000in}}%
\pgfusepath{clip}%
\pgfsetbuttcap%
\pgfsetroundjoin%
\definecolor{currentfill}{rgb}{0.580193,0.631893,0.704274}%
\pgfsetfillcolor{currentfill}%
\pgfsetlinewidth{0.000000pt}%
\definecolor{currentstroke}{rgb}{0.000000,0.000000,0.000000}%
\pgfsetstrokecolor{currentstroke}%
\pgfsetdash{}{0pt}%
\pgfpathmoveto{\pgfqpoint{1.233518in}{1.673451in}}%
\pgfpathlineto{\pgfqpoint{1.273386in}{1.628511in}}%
\pgfpathlineto{\pgfqpoint{1.236004in}{1.659401in}}%
\pgfpathlineto{\pgfqpoint{1.195880in}{1.704544in}}%
\pgfpathclose%
\pgfusepath{fill}%
\end{pgfscope}%
\begin{pgfscope}%
\pgfpathrectangle{\pgfqpoint{0.150000in}{0.150000in}}{\pgfqpoint{2.700000in}{1.950000in}}%
\pgfusepath{clip}%
\pgfsetbuttcap%
\pgfsetroundjoin%
\definecolor{currentfill}{rgb}{0.741896,0.773683,0.818183}%
\pgfsetfillcolor{currentfill}%
\pgfsetlinewidth{0.000000pt}%
\definecolor{currentstroke}{rgb}{0.000000,0.000000,0.000000}%
\pgfsetstrokecolor{currentstroke}%
\pgfsetdash{}{0pt}%
\pgfpathmoveto{\pgfqpoint{1.498682in}{1.460954in}}%
\pgfpathlineto{\pgfqpoint{1.536486in}{1.460994in}}%
\pgfpathlineto{\pgfqpoint{1.498863in}{1.485847in}}%
\pgfpathlineto{\pgfqpoint{1.460929in}{1.492138in}}%
\pgfpathclose%
\pgfusepath{fill}%
\end{pgfscope}%
\begin{pgfscope}%
\pgfpathrectangle{\pgfqpoint{0.150000in}{0.150000in}}{\pgfqpoint{2.700000in}{1.950000in}}%
\pgfusepath{clip}%
\pgfsetbuttcap%
\pgfsetroundjoin%
\definecolor{currentfill}{rgb}{0.841406,0.860938,0.888281}%
\pgfsetfillcolor{currentfill}%
\pgfsetlinewidth{0.000000pt}%
\definecolor{currentstroke}{rgb}{0.000000,0.000000,0.000000}%
\pgfsetstrokecolor{currentstroke}%
\pgfsetdash{}{0pt}%
\pgfpathmoveto{\pgfqpoint{1.763257in}{1.342436in}}%
\pgfpathlineto{\pgfqpoint{1.800417in}{1.355193in}}%
\pgfpathlineto{\pgfqpoint{1.762300in}{1.374042in}}%
\pgfpathlineto{\pgfqpoint{1.725224in}{1.367519in}}%
\pgfpathclose%
\pgfusepath{fill}%
\end{pgfscope}%
\begin{pgfscope}%
\pgfpathrectangle{\pgfqpoint{0.150000in}{0.150000in}}{\pgfqpoint{2.700000in}{1.950000in}}%
\pgfusepath{clip}%
\pgfsetbuttcap%
\pgfsetroundjoin%
\definecolor{currentfill}{rgb}{0.804090,0.828217,0.861994}%
\pgfsetfillcolor{currentfill}%
\pgfsetlinewidth{0.000000pt}%
\definecolor{currentstroke}{rgb}{0.000000,0.000000,0.000000}%
\pgfsetstrokecolor{currentstroke}%
\pgfsetdash{}{0pt}%
\pgfpathmoveto{\pgfqpoint{1.649859in}{1.379874in}}%
\pgfpathlineto{\pgfqpoint{1.687288in}{1.392540in}}%
\pgfpathlineto{\pgfqpoint{1.649446in}{1.417498in}}%
\pgfpathlineto{\pgfqpoint{1.612017in}{1.411102in}}%
\pgfpathclose%
\pgfusepath{fill}%
\end{pgfscope}%
\begin{pgfscope}%
\pgfpathrectangle{\pgfqpoint{0.150000in}{0.150000in}}{\pgfqpoint{2.700000in}{1.950000in}}%
\pgfusepath{clip}%
\pgfsetbuttcap%
\pgfsetroundjoin%
\definecolor{currentfill}{rgb}{0.996890,0.997273,0.997809}%
\pgfsetfillcolor{currentfill}%
\pgfsetlinewidth{0.000000pt}%
\definecolor{currentstroke}{rgb}{0.000000,0.000000,0.000000}%
\pgfsetstrokecolor{currentstroke}%
\pgfsetdash{}{0pt}%
\pgfpathmoveto{\pgfqpoint{1.876240in}{1.136980in}}%
\pgfpathlineto{\pgfqpoint{1.913086in}{1.150215in}}%
\pgfpathlineto{\pgfqpoint{1.875972in}{1.205727in}}%
\pgfpathlineto{\pgfqpoint{1.838804in}{1.186528in}}%
\pgfpathclose%
\pgfusepath{fill}%
\end{pgfscope}%
\begin{pgfscope}%
\pgfpathrectangle{\pgfqpoint{0.150000in}{0.150000in}}{\pgfqpoint{2.700000in}{1.950000in}}%
\pgfusepath{clip}%
\pgfsetbuttcap%
\pgfsetroundjoin%
\definecolor{currentfill}{rgb}{0.835187,0.855484,0.883900}%
\pgfsetfillcolor{currentfill}%
\pgfsetlinewidth{0.000000pt}%
\definecolor{currentstroke}{rgb}{0.000000,0.000000,0.000000}%
\pgfsetstrokecolor{currentstroke}%
\pgfsetdash{}{0pt}%
\pgfpathmoveto{\pgfqpoint{1.082424in}{1.354777in}}%
\pgfpathlineto{\pgfqpoint{1.121976in}{1.349041in}}%
\pgfpathlineto{\pgfqpoint{1.085117in}{1.367890in}}%
\pgfpathlineto{\pgfqpoint{1.045204in}{1.379874in}}%
\pgfpathclose%
\pgfusepath{fill}%
\end{pgfscope}%
\begin{pgfscope}%
\pgfpathrectangle{\pgfqpoint{0.150000in}{0.150000in}}{\pgfqpoint{2.700000in}{1.950000in}}%
\pgfusepath{clip}%
\pgfsetbuttcap%
\pgfsetroundjoin%
\definecolor{currentfill}{rgb}{0.636167,0.680974,0.743704}%
\pgfsetfillcolor{currentfill}%
\pgfsetlinewidth{0.000000pt}%
\definecolor{currentstroke}{rgb}{0.000000,0.000000,0.000000}%
\pgfsetstrokecolor{currentstroke}%
\pgfsetdash{}{0pt}%
\pgfpathmoveto{\pgfqpoint{1.084537in}{1.535589in}}%
\pgfpathlineto{\pgfqpoint{1.120339in}{1.623289in}}%
\pgfpathlineto{\pgfqpoint{1.082818in}{1.654385in}}%
\pgfpathlineto{\pgfqpoint{1.047208in}{1.566605in}}%
\pgfpathclose%
\pgfusepath{fill}%
\end{pgfscope}%
\begin{pgfscope}%
\pgfpathrectangle{\pgfqpoint{0.150000in}{0.150000in}}{\pgfqpoint{2.700000in}{1.950000in}}%
\pgfusepath{clip}%
\pgfsetbuttcap%
\pgfsetroundjoin%
\definecolor{currentfill}{rgb}{0.823346,0.679458,0.690855}%
\pgfsetfillcolor{currentfill}%
\pgfsetlinewidth{0.000000pt}%
\definecolor{currentstroke}{rgb}{0.000000,0.000000,0.000000}%
\pgfsetstrokecolor{currentstroke}%
\pgfsetdash{}{0pt}%
\pgfpathmoveto{\pgfqpoint{1.915228in}{0.823331in}}%
\pgfpathlineto{\pgfqpoint{1.952339in}{0.843238in}}%
\pgfpathlineto{\pgfqpoint{1.912335in}{0.822085in}}%
\pgfpathlineto{\pgfqpoint{1.875561in}{0.808090in}}%
\pgfpathclose%
\pgfusepath{fill}%
\end{pgfscope}%
\begin{pgfscope}%
\pgfpathrectangle{\pgfqpoint{0.150000in}{0.150000in}}{\pgfqpoint{2.700000in}{1.950000in}}%
\pgfusepath{clip}%
\pgfsetbuttcap%
\pgfsetroundjoin%
\definecolor{currentfill}{rgb}{0.944914,0.900046,0.903600}%
\pgfsetfillcolor{currentfill}%
\pgfsetlinewidth{0.000000pt}%
\definecolor{currentstroke}{rgb}{0.000000,0.000000,0.000000}%
\pgfsetstrokecolor{currentstroke}%
\pgfsetdash{}{0pt}%
\pgfpathmoveto{\pgfqpoint{1.914113in}{1.024684in}}%
\pgfpathlineto{\pgfqpoint{1.950880in}{1.038193in}}%
\pgfpathlineto{\pgfqpoint{1.913599in}{1.087535in}}%
\pgfpathlineto{\pgfqpoint{1.876900in}{1.080145in}}%
\pgfpathclose%
\pgfusepath{fill}%
\end{pgfscope}%
\begin{pgfscope}%
\pgfpathrectangle{\pgfqpoint{0.150000in}{0.150000in}}{\pgfqpoint{2.700000in}{1.950000in}}%
\pgfusepath{clip}%
\pgfsetbuttcap%
\pgfsetroundjoin%
\definecolor{currentfill}{rgb}{0.760555,0.790043,0.831327}%
\pgfsetfillcolor{currentfill}%
\pgfsetlinewidth{0.000000pt}%
\definecolor{currentstroke}{rgb}{0.000000,0.000000,0.000000}%
\pgfsetstrokecolor{currentstroke}%
\pgfsetdash{}{0pt}%
\pgfpathmoveto{\pgfqpoint{0.930853in}{1.435950in}}%
\pgfpathlineto{\pgfqpoint{0.971043in}{1.429878in}}%
\pgfpathlineto{\pgfqpoint{0.934447in}{1.448586in}}%
\pgfpathlineto{\pgfqpoint{0.893810in}{1.460954in}}%
\pgfpathclose%
\pgfusepath{fill}%
\end{pgfscope}%
\begin{pgfscope}%
\pgfpathrectangle{\pgfqpoint{0.150000in}{0.150000in}}{\pgfqpoint{2.700000in}{1.950000in}}%
\pgfusepath{clip}%
\pgfsetbuttcap%
\pgfsetroundjoin%
\definecolor{currentfill}{rgb}{0.598851,0.648254,0.717417}%
\pgfsetfillcolor{currentfill}%
\pgfsetlinewidth{0.000000pt}%
\definecolor{currentstroke}{rgb}{0.000000,0.000000,0.000000}%
\pgfsetstrokecolor{currentstroke}%
\pgfsetdash{}{0pt}%
\pgfpathmoveto{\pgfqpoint{1.271208in}{1.642315in}}%
\pgfpathlineto{\pgfqpoint{1.310690in}{1.603862in}}%
\pgfpathlineto{\pgfqpoint{1.273386in}{1.628511in}}%
\pgfpathlineto{\pgfqpoint{1.233518in}{1.673451in}}%
\pgfpathclose%
\pgfusepath{fill}%
\end{pgfscope}%
\begin{pgfscope}%
\pgfpathrectangle{\pgfqpoint{0.150000in}{0.150000in}}{\pgfqpoint{2.700000in}{1.950000in}}%
\pgfusepath{clip}%
\pgfsetbuttcap%
\pgfsetroundjoin%
\definecolor{currentfill}{rgb}{0.791651,0.817310,0.853232}%
\pgfsetfillcolor{currentfill}%
\pgfsetlinewidth{0.000000pt}%
\definecolor{currentstroke}{rgb}{0.000000,0.000000,0.000000}%
\pgfsetstrokecolor{currentstroke}%
\pgfsetdash{}{0pt}%
\pgfpathmoveto{\pgfqpoint{1.196953in}{1.361353in}}%
\pgfpathlineto{\pgfqpoint{1.234018in}{1.423512in}}%
\pgfpathlineto{\pgfqpoint{1.196441in}{1.454725in}}%
\pgfpathlineto{\pgfqpoint{1.159700in}{1.386367in}}%
\pgfpathclose%
\pgfusepath{fill}%
\end{pgfscope}%
\begin{pgfscope}%
\pgfpathrectangle{\pgfqpoint{0.150000in}{0.150000in}}{\pgfqpoint{2.700000in}{1.950000in}}%
\pgfusepath{clip}%
\pgfsetbuttcap%
\pgfsetroundjoin%
\definecolor{currentfill}{rgb}{0.853845,0.871844,0.897044}%
\pgfsetfillcolor{currentfill}%
\pgfsetlinewidth{0.000000pt}%
\definecolor{currentstroke}{rgb}{0.000000,0.000000,0.000000}%
\pgfsetstrokecolor{currentstroke}%
\pgfsetdash{}{0pt}%
\pgfpathmoveto{\pgfqpoint{1.196294in}{1.304990in}}%
\pgfpathlineto{\pgfqpoint{1.234300in}{1.336276in}}%
\pgfpathlineto{\pgfqpoint{1.196953in}{1.361353in}}%
\pgfpathlineto{\pgfqpoint{1.159186in}{1.323978in}}%
\pgfpathclose%
\pgfusepath{fill}%
\end{pgfscope}%
\begin{pgfscope}%
\pgfpathrectangle{\pgfqpoint{0.150000in}{0.150000in}}{\pgfqpoint{2.700000in}{1.950000in}}%
\pgfusepath{clip}%
\pgfsetbuttcap%
\pgfsetroundjoin%
\definecolor{currentfill}{rgb}{0.661045,0.702788,0.761229}%
\pgfsetfillcolor{currentfill}%
\pgfsetlinewidth{0.000000pt}%
\definecolor{currentstroke}{rgb}{0.000000,0.000000,0.000000}%
\pgfsetstrokecolor{currentstroke}%
\pgfsetdash{}{0pt}%
\pgfpathmoveto{\pgfqpoint{1.121680in}{1.510773in}}%
\pgfpathlineto{\pgfqpoint{1.158130in}{1.585838in}}%
\pgfpathlineto{\pgfqpoint{1.120339in}{1.623289in}}%
\pgfpathlineto{\pgfqpoint{1.084537in}{1.535589in}}%
\pgfpathclose%
\pgfusepath{fill}%
\end{pgfscope}%
\begin{pgfscope}%
\pgfpathrectangle{\pgfqpoint{0.150000in}{0.150000in}}{\pgfqpoint{2.700000in}{1.950000in}}%
\pgfusepath{clip}%
\pgfsetbuttcap%
\pgfsetroundjoin%
\definecolor{currentfill}{rgb}{0.754335,0.784589,0.826945}%
\pgfsetfillcolor{currentfill}%
\pgfsetlinewidth{0.000000pt}%
\definecolor{currentstroke}{rgb}{0.000000,0.000000,0.000000}%
\pgfsetstrokecolor{currentstroke}%
\pgfsetdash{}{0pt}%
\pgfpathmoveto{\pgfqpoint{1.536486in}{1.435950in}}%
\pgfpathlineto{\pgfqpoint{1.574204in}{1.436079in}}%
\pgfpathlineto{\pgfqpoint{1.536486in}{1.460994in}}%
\pgfpathlineto{\pgfqpoint{1.498682in}{1.460954in}}%
\pgfpathclose%
\pgfusepath{fill}%
\end{pgfscope}%
\begin{pgfscope}%
\pgfpathrectangle{\pgfqpoint{0.150000in}{0.150000in}}{\pgfqpoint{2.700000in}{1.950000in}}%
\pgfusepath{clip}%
\pgfsetbuttcap%
\pgfsetroundjoin%
\definecolor{currentfill}{rgb}{0.884942,0.899112,0.918949}%
\pgfsetfillcolor{currentfill}%
\pgfsetlinewidth{0.000000pt}%
\definecolor{currentstroke}{rgb}{0.000000,0.000000,0.000000}%
\pgfsetstrokecolor{currentstroke}%
\pgfsetdash{}{0pt}%
\pgfpathmoveto{\pgfqpoint{1.838739in}{1.261418in}}%
\pgfpathlineto{\pgfqpoint{1.875704in}{1.274365in}}%
\pgfpathlineto{\pgfqpoint{1.838500in}{1.330123in}}%
\pgfpathlineto{\pgfqpoint{1.801385in}{1.317289in}}%
\pgfpathclose%
\pgfusepath{fill}%
\end{pgfscope}%
\begin{pgfscope}%
\pgfpathrectangle{\pgfqpoint{0.150000in}{0.150000in}}{\pgfqpoint{2.700000in}{1.950000in}}%
\pgfusepath{clip}%
\pgfsetbuttcap%
\pgfsetroundjoin%
\definecolor{currentfill}{rgb}{0.816529,0.839124,0.870757}%
\pgfsetfillcolor{currentfill}%
\pgfsetlinewidth{0.000000pt}%
\definecolor{currentstroke}{rgb}{0.000000,0.000000,0.000000}%
\pgfsetstrokecolor{currentstroke}%
\pgfsetdash{}{0pt}%
\pgfpathmoveto{\pgfqpoint{1.687841in}{1.354777in}}%
\pgfpathlineto{\pgfqpoint{1.725224in}{1.367519in}}%
\pgfpathlineto{\pgfqpoint{1.687288in}{1.392540in}}%
\pgfpathlineto{\pgfqpoint{1.649859in}{1.379874in}}%
\pgfpathclose%
\pgfusepath{fill}%
\end{pgfscope}%
\begin{pgfscope}%
\pgfpathrectangle{\pgfqpoint{0.150000in}{0.150000in}}{\pgfqpoint{2.700000in}{1.950000in}}%
\pgfusepath{clip}%
\pgfsetbuttcap%
\pgfsetroundjoin%
\definecolor{currentfill}{rgb}{0.617509,0.664614,0.730561}%
\pgfsetfillcolor{currentfill}%
\pgfsetlinewidth{0.000000pt}%
\definecolor{currentstroke}{rgb}{0.000000,0.000000,0.000000}%
\pgfsetstrokecolor{currentstroke}%
\pgfsetdash{}{0pt}%
\pgfpathmoveto{\pgfqpoint{1.309080in}{1.604800in}}%
\pgfpathlineto{\pgfqpoint{1.348195in}{1.572875in}}%
\pgfpathlineto{\pgfqpoint{1.310690in}{1.603862in}}%
\pgfpathlineto{\pgfqpoint{1.271208in}{1.642315in}}%
\pgfpathclose%
\pgfusepath{fill}%
\end{pgfscope}%
\begin{pgfscope}%
\pgfpathrectangle{\pgfqpoint{0.150000in}{0.150000in}}{\pgfqpoint{2.700000in}{1.950000in}}%
\pgfusepath{clip}%
\pgfsetbuttcap%
\pgfsetroundjoin%
\definecolor{currentfill}{rgb}{0.834743,0.700138,0.710800}%
\pgfsetfillcolor{currentfill}%
\pgfsetlinewidth{0.000000pt}%
\definecolor{currentstroke}{rgb}{0.000000,0.000000,0.000000}%
\pgfsetstrokecolor{currentstroke}%
\pgfsetdash{}{0pt}%
\pgfpathmoveto{\pgfqpoint{1.838437in}{0.788187in}}%
\pgfpathlineto{\pgfqpoint{1.875561in}{0.808090in}}%
\pgfpathlineto{\pgfqpoint{1.838892in}{0.880739in}}%
\pgfpathlineto{\pgfqpoint{1.801577in}{0.860903in}}%
\pgfpathclose%
\pgfusepath{fill}%
\end{pgfscope}%
\begin{pgfscope}%
\pgfpathrectangle{\pgfqpoint{0.150000in}{0.150000in}}{\pgfqpoint{2.700000in}{1.950000in}}%
\pgfusepath{clip}%
\pgfsetbuttcap%
\pgfsetroundjoin%
\definecolor{currentfill}{rgb}{0.847626,0.866391,0.892662}%
\pgfsetfillcolor{currentfill}%
\pgfsetlinewidth{0.000000pt}%
\definecolor{currentstroke}{rgb}{0.000000,0.000000,0.000000}%
\pgfsetstrokecolor{currentstroke}%
\pgfsetdash{}{0pt}%
\pgfpathmoveto{\pgfqpoint{1.801385in}{1.317289in}}%
\pgfpathlineto{\pgfqpoint{1.838500in}{1.330123in}}%
\pgfpathlineto{\pgfqpoint{1.800417in}{1.355193in}}%
\pgfpathlineto{\pgfqpoint{1.763257in}{1.342436in}}%
\pgfpathclose%
\pgfusepath{fill}%
\end{pgfscope}%
\begin{pgfscope}%
\pgfpathrectangle{\pgfqpoint{0.150000in}{0.150000in}}{\pgfqpoint{2.700000in}{1.950000in}}%
\pgfusepath{clip}%
\pgfsetbuttcap%
\pgfsetroundjoin%
\definecolor{currentfill}{rgb}{0.895527,0.810432,0.817172}%
\pgfsetfillcolor{currentfill}%
\pgfsetlinewidth{0.000000pt}%
\definecolor{currentstroke}{rgb}{0.000000,0.000000,0.000000}%
\pgfsetstrokecolor{currentstroke}%
\pgfsetdash{}{0pt}%
\pgfpathmoveto{\pgfqpoint{1.876071in}{0.900503in}}%
\pgfpathlineto{\pgfqpoint{1.913114in}{0.920195in}}%
\pgfpathlineto{\pgfqpoint{1.876973in}{1.005132in}}%
\pgfpathlineto{\pgfqpoint{1.839697in}{0.985508in}}%
\pgfpathclose%
\pgfusepath{fill}%
\end{pgfscope}%
\begin{pgfscope}%
\pgfpathrectangle{\pgfqpoint{0.150000in}{0.150000in}}{\pgfqpoint{2.700000in}{1.950000in}}%
\pgfusepath{clip}%
\pgfsetbuttcap%
\pgfsetroundjoin%
\definecolor{currentfill}{rgb}{0.679703,0.719148,0.774372}%
\pgfsetfillcolor{currentfill}%
\pgfsetlinewidth{0.000000pt}%
\definecolor{currentstroke}{rgb}{0.000000,0.000000,0.000000}%
\pgfsetstrokecolor{currentstroke}%
\pgfsetdash{}{0pt}%
\pgfpathmoveto{\pgfqpoint{1.159133in}{1.479659in}}%
\pgfpathlineto{\pgfqpoint{1.195733in}{1.554671in}}%
\pgfpathlineto{\pgfqpoint{1.158130in}{1.585838in}}%
\pgfpathlineto{\pgfqpoint{1.121680in}{1.510773in}}%
\pgfpathclose%
\pgfusepath{fill}%
\end{pgfscope}%
\begin{pgfscope}%
\pgfpathrectangle{\pgfqpoint{0.150000in}{0.150000in}}{\pgfqpoint{2.700000in}{1.950000in}}%
\pgfusepath{clip}%
\pgfsetbuttcap%
\pgfsetroundjoin%
\definecolor{currentfill}{rgb}{0.940916,0.948192,0.958379}%
\pgfsetfillcolor{currentfill}%
\pgfsetlinewidth{0.000000pt}%
\definecolor{currentstroke}{rgb}{0.000000,0.000000,0.000000}%
\pgfsetstrokecolor{currentstroke}%
\pgfsetdash{}{0pt}%
\pgfpathmoveto{\pgfqpoint{1.838804in}{1.186528in}}%
\pgfpathlineto{\pgfqpoint{1.875972in}{1.205727in}}%
\pgfpathlineto{\pgfqpoint{1.838739in}{1.261418in}}%
\pgfpathlineto{\pgfqpoint{1.801442in}{1.242290in}}%
\pgfpathclose%
\pgfusepath{fill}%
\end{pgfscope}%
\begin{pgfscope}%
\pgfpathrectangle{\pgfqpoint{0.150000in}{0.150000in}}{\pgfqpoint{2.700000in}{1.950000in}}%
\pgfusepath{clip}%
\pgfsetbuttcap%
\pgfsetroundjoin%
\definecolor{currentfill}{rgb}{0.979105,0.962086,0.963434}%
\pgfsetfillcolor{currentfill}%
\pgfsetlinewidth{0.000000pt}%
\definecolor{currentstroke}{rgb}{0.000000,0.000000,0.000000}%
\pgfsetstrokecolor{currentstroke}%
\pgfsetdash{}{0pt}%
\pgfpathmoveto{\pgfqpoint{1.876900in}{1.080145in}}%
\pgfpathlineto{\pgfqpoint{1.913599in}{1.087535in}}%
\pgfpathlineto{\pgfqpoint{1.876240in}{1.136980in}}%
\pgfpathlineto{\pgfqpoint{1.839391in}{1.129730in}}%
\pgfpathclose%
\pgfusepath{fill}%
\end{pgfscope}%
\begin{pgfscope}%
\pgfpathrectangle{\pgfqpoint{0.150000in}{0.150000in}}{\pgfqpoint{2.700000in}{1.950000in}}%
\pgfusepath{clip}%
\pgfsetbuttcap%
\pgfsetroundjoin%
\definecolor{currentfill}{rgb}{0.841406,0.860938,0.888281}%
\pgfsetfillcolor{currentfill}%
\pgfsetlinewidth{0.000000pt}%
\definecolor{currentstroke}{rgb}{0.000000,0.000000,0.000000}%
\pgfsetstrokecolor{currentstroke}%
\pgfsetdash{}{0pt}%
\pgfpathmoveto{\pgfqpoint{1.119498in}{1.335791in}}%
\pgfpathlineto{\pgfqpoint{1.159186in}{1.323978in}}%
\pgfpathlineto{\pgfqpoint{1.121976in}{1.349041in}}%
\pgfpathlineto{\pgfqpoint{1.082424in}{1.354777in}}%
\pgfpathclose%
\pgfusepath{fill}%
\end{pgfscope}%
\begin{pgfscope}%
\pgfpathrectangle{\pgfqpoint{0.150000in}{0.150000in}}{\pgfqpoint{2.700000in}{1.950000in}}%
\pgfusepath{clip}%
\pgfsetbuttcap%
\pgfsetroundjoin%
\definecolor{currentfill}{rgb}{0.804090,0.828217,0.861994}%
\pgfsetfillcolor{currentfill}%
\pgfsetlinewidth{0.000000pt}%
\definecolor{currentstroke}{rgb}{0.000000,0.000000,0.000000}%
\pgfsetstrokecolor{currentstroke}%
\pgfsetdash{}{0pt}%
\pgfpathmoveto{\pgfqpoint{1.234300in}{1.336276in}}%
\pgfpathlineto{\pgfqpoint{1.271493in}{1.398458in}}%
\pgfpathlineto{\pgfqpoint{1.234018in}{1.423512in}}%
\pgfpathlineto{\pgfqpoint{1.196953in}{1.361353in}}%
\pgfpathclose%
\pgfusepath{fill}%
\end{pgfscope}%
\begin{pgfscope}%
\pgfpathrectangle{\pgfqpoint{0.150000in}{0.150000in}}{\pgfqpoint{2.700000in}{1.950000in}}%
\pgfusepath{clip}%
\pgfsetbuttcap%
\pgfsetroundjoin%
\definecolor{currentfill}{rgb}{0.772993,0.800950,0.840089}%
\pgfsetfillcolor{currentfill}%
\pgfsetlinewidth{0.000000pt}%
\definecolor{currentstroke}{rgb}{0.000000,0.000000,0.000000}%
\pgfsetstrokecolor{currentstroke}%
\pgfsetdash{}{0pt}%
\pgfpathmoveto{\pgfqpoint{0.967990in}{1.410884in}}%
\pgfpathlineto{\pgfqpoint{1.008077in}{1.404907in}}%
\pgfpathlineto{\pgfqpoint{0.971043in}{1.429878in}}%
\pgfpathlineto{\pgfqpoint{0.930853in}{1.435950in}}%
\pgfpathclose%
\pgfusepath{fill}%
\end{pgfscope}%
\begin{pgfscope}%
\pgfpathrectangle{\pgfqpoint{0.150000in}{0.150000in}}{\pgfqpoint{2.700000in}{1.950000in}}%
\pgfusepath{clip}%
\pgfsetbuttcap%
\pgfsetroundjoin%
\definecolor{currentfill}{rgb}{0.642387,0.686428,0.748085}%
\pgfsetfillcolor{currentfill}%
\pgfsetlinewidth{0.000000pt}%
\definecolor{currentstroke}{rgb}{0.000000,0.000000,0.000000}%
\pgfsetstrokecolor{currentstroke}%
\pgfsetdash{}{0pt}%
\pgfpathmoveto{\pgfqpoint{1.346851in}{1.573593in}}%
\pgfpathlineto{\pgfqpoint{1.385664in}{1.548108in}}%
\pgfpathlineto{\pgfqpoint{1.348195in}{1.572875in}}%
\pgfpathlineto{\pgfqpoint{1.309080in}{1.604800in}}%
\pgfpathclose%
\pgfusepath{fill}%
\end{pgfscope}%
\begin{pgfscope}%
\pgfpathrectangle{\pgfqpoint{0.150000in}{0.150000in}}{\pgfqpoint{2.700000in}{1.950000in}}%
\pgfusepath{clip}%
\pgfsetbuttcap%
\pgfsetroundjoin%
\definecolor{currentfill}{rgb}{0.766774,0.795496,0.835708}%
\pgfsetfillcolor{currentfill}%
\pgfsetlinewidth{0.000000pt}%
\definecolor{currentstroke}{rgb}{0.000000,0.000000,0.000000}%
\pgfsetstrokecolor{currentstroke}%
\pgfsetdash{}{0pt}%
\pgfpathmoveto{\pgfqpoint{1.574364in}{1.404667in}}%
\pgfpathlineto{\pgfqpoint{1.612017in}{1.411102in}}%
\pgfpathlineto{\pgfqpoint{1.574204in}{1.436079in}}%
\pgfpathlineto{\pgfqpoint{1.536486in}{1.435950in}}%
\pgfpathclose%
\pgfusepath{fill}%
\end{pgfscope}%
\begin{pgfscope}%
\pgfpathrectangle{\pgfqpoint{0.150000in}{0.150000in}}{\pgfqpoint{2.700000in}{1.950000in}}%
\pgfusepath{clip}%
\pgfsetbuttcap%
\pgfsetroundjoin%
\definecolor{currentfill}{rgb}{0.865135,0.755285,0.763986}%
\pgfsetfillcolor{currentfill}%
\pgfsetlinewidth{0.000000pt}%
\definecolor{currentstroke}{rgb}{0.000000,0.000000,0.000000}%
\pgfsetstrokecolor{currentstroke}%
\pgfsetdash{}{0pt}%
\pgfpathmoveto{\pgfqpoint{2.105192in}{0.872713in}}%
\pgfpathlineto{\pgfqpoint{2.140604in}{0.868955in}}%
\pgfpathlineto{\pgfqpoint{2.101812in}{0.888717in}}%
\pgfpathlineto{\pgfqpoint{2.066610in}{0.898467in}}%
\pgfpathclose%
\pgfusepath{fill}%
\end{pgfscope}%
\begin{pgfscope}%
\pgfpathrectangle{\pgfqpoint{0.150000in}{0.150000in}}{\pgfqpoint{2.700000in}{1.950000in}}%
\pgfusepath{clip}%
\pgfsetbuttcap%
\pgfsetroundjoin%
\definecolor{currentfill}{rgb}{0.865135,0.755285,0.763986}%
\pgfsetfillcolor{currentfill}%
\pgfsetlinewidth{0.000000pt}%
\definecolor{currentstroke}{rgb}{0.000000,0.000000,0.000000}%
\pgfsetstrokecolor{currentstroke}%
\pgfsetdash{}{0pt}%
\pgfpathmoveto{\pgfqpoint{2.029933in}{0.884551in}}%
\pgfpathlineto{\pgfqpoint{2.066610in}{0.898467in}}%
\pgfpathlineto{\pgfqpoint{2.026155in}{0.882834in}}%
\pgfpathlineto{\pgfqpoint{1.989315in}{0.863072in}}%
\pgfpathclose%
\pgfusepath{fill}%
\end{pgfscope}%
\begin{pgfscope}%
\pgfpathrectangle{\pgfqpoint{0.150000in}{0.150000in}}{\pgfqpoint{2.700000in}{1.950000in}}%
\pgfusepath{clip}%
\pgfsetbuttcap%
\pgfsetroundjoin%
\definecolor{currentfill}{rgb}{0.827145,0.686351,0.697503}%
\pgfsetfillcolor{currentfill}%
\pgfsetlinewidth{0.000000pt}%
\definecolor{currentstroke}{rgb}{0.000000,0.000000,0.000000}%
\pgfsetstrokecolor{currentstroke}%
\pgfsetdash{}{0pt}%
\pgfpathmoveto{\pgfqpoint{1.877981in}{0.803350in}}%
\pgfpathlineto{\pgfqpoint{1.915228in}{0.823331in}}%
\pgfpathlineto{\pgfqpoint{1.875561in}{0.808090in}}%
\pgfpathlineto{\pgfqpoint{1.838437in}{0.788187in}}%
\pgfpathclose%
\pgfusepath{fill}%
\end{pgfscope}%
\begin{pgfscope}%
\pgfpathrectangle{\pgfqpoint{0.150000in}{0.150000in}}{\pgfqpoint{2.700000in}{1.950000in}}%
\pgfusepath{clip}%
\pgfsetbuttcap%
\pgfsetroundjoin%
\definecolor{currentfill}{rgb}{0.860064,0.877298,0.901425}%
\pgfsetfillcolor{currentfill}%
\pgfsetlinewidth{0.000000pt}%
\definecolor{currentstroke}{rgb}{0.000000,0.000000,0.000000}%
\pgfsetstrokecolor{currentstroke}%
\pgfsetdash{}{0pt}%
\pgfpathmoveto{\pgfqpoint{1.233713in}{1.279794in}}%
\pgfpathlineto{\pgfqpoint{1.271588in}{1.317289in}}%
\pgfpathlineto{\pgfqpoint{1.234300in}{1.336276in}}%
\pgfpathlineto{\pgfqpoint{1.196294in}{1.304990in}}%
\pgfpathclose%
\pgfusepath{fill}%
\end{pgfscope}%
\begin{pgfscope}%
\pgfpathrectangle{\pgfqpoint{0.150000in}{0.150000in}}{\pgfqpoint{2.700000in}{1.950000in}}%
\pgfusepath{clip}%
\pgfsetbuttcap%
\pgfsetroundjoin%
\definecolor{currentfill}{rgb}{0.698361,0.735509,0.787515}%
\pgfsetfillcolor{currentfill}%
\pgfsetlinewidth{0.000000pt}%
\definecolor{currentstroke}{rgb}{0.000000,0.000000,0.000000}%
\pgfsetstrokecolor{currentstroke}%
\pgfsetdash{}{0pt}%
\pgfpathmoveto{\pgfqpoint{1.196441in}{1.454725in}}%
\pgfpathlineto{\pgfqpoint{1.233387in}{1.523461in}}%
\pgfpathlineto{\pgfqpoint{1.195733in}{1.554671in}}%
\pgfpathlineto{\pgfqpoint{1.159133in}{1.479659in}}%
\pgfpathclose%
\pgfusepath{fill}%
\end{pgfscope}%
\begin{pgfscope}%
\pgfpathrectangle{\pgfqpoint{0.150000in}{0.150000in}}{\pgfqpoint{2.700000in}{1.950000in}}%
\pgfusepath{clip}%
\pgfsetbuttcap%
\pgfsetroundjoin%
\definecolor{currentfill}{rgb}{0.828968,0.850031,0.879519}%
\pgfsetfillcolor{currentfill}%
\pgfsetlinewidth{0.000000pt}%
\definecolor{currentstroke}{rgb}{0.000000,0.000000,0.000000}%
\pgfsetstrokecolor{currentstroke}%
\pgfsetdash{}{0pt}%
\pgfpathmoveto{\pgfqpoint{1.725917in}{1.329617in}}%
\pgfpathlineto{\pgfqpoint{1.763257in}{1.342436in}}%
\pgfpathlineto{\pgfqpoint{1.725224in}{1.367519in}}%
\pgfpathlineto{\pgfqpoint{1.687841in}{1.354777in}}%
\pgfpathclose%
\pgfusepath{fill}%
\end{pgfscope}%
\begin{pgfscope}%
\pgfpathrectangle{\pgfqpoint{0.150000in}{0.150000in}}{\pgfqpoint{2.700000in}{1.950000in}}%
\pgfusepath{clip}%
\pgfsetbuttcap%
\pgfsetroundjoin%
\definecolor{currentfill}{rgb}{0.984452,0.986366,0.989047}%
\pgfsetfillcolor{currentfill}%
\pgfsetlinewidth{0.000000pt}%
\definecolor{currentstroke}{rgb}{0.000000,0.000000,0.000000}%
\pgfsetstrokecolor{currentstroke}%
\pgfsetdash{}{0pt}%
\pgfpathmoveto{\pgfqpoint{1.839391in}{1.129730in}}%
\pgfpathlineto{\pgfqpoint{1.876240in}{1.136980in}}%
\pgfpathlineto{\pgfqpoint{1.838804in}{1.186528in}}%
\pgfpathlineto{\pgfqpoint{1.801652in}{1.173335in}}%
\pgfpathclose%
\pgfusepath{fill}%
\end{pgfscope}%
\begin{pgfscope}%
\pgfpathrectangle{\pgfqpoint{0.150000in}{0.150000in}}{\pgfqpoint{2.700000in}{1.950000in}}%
\pgfusepath{clip}%
\pgfsetbuttcap%
\pgfsetroundjoin%
\definecolor{currentfill}{rgb}{0.661045,0.702788,0.761229}%
\pgfsetfillcolor{currentfill}%
\pgfsetlinewidth{0.000000pt}%
\definecolor{currentstroke}{rgb}{0.000000,0.000000,0.000000}%
\pgfsetstrokecolor{currentstroke}%
\pgfsetdash{}{0pt}%
\pgfpathmoveto{\pgfqpoint{1.384675in}{1.542342in}}%
\pgfpathlineto{\pgfqpoint{1.423293in}{1.517023in}}%
\pgfpathlineto{\pgfqpoint{1.385664in}{1.548108in}}%
\pgfpathlineto{\pgfqpoint{1.346851in}{1.573593in}}%
\pgfpathclose%
\pgfusepath{fill}%
\end{pgfscope}%
\begin{pgfscope}%
\pgfpathrectangle{\pgfqpoint{0.150000in}{0.150000in}}{\pgfqpoint{2.700000in}{1.950000in}}%
\pgfusepath{clip}%
\pgfsetbuttcap%
\pgfsetroundjoin%
\definecolor{currentfill}{rgb}{0.816529,0.839124,0.870757}%
\pgfsetfillcolor{currentfill}%
\pgfsetlinewidth{0.000000pt}%
\definecolor{currentstroke}{rgb}{0.000000,0.000000,0.000000}%
\pgfsetstrokecolor{currentstroke}%
\pgfsetdash{}{0pt}%
\pgfpathmoveto{\pgfqpoint{1.271588in}{1.317289in}}%
\pgfpathlineto{\pgfqpoint{1.309194in}{1.367146in}}%
\pgfpathlineto{\pgfqpoint{1.271493in}{1.398458in}}%
\pgfpathlineto{\pgfqpoint{1.234300in}{1.336276in}}%
\pgfpathclose%
\pgfusepath{fill}%
\end{pgfscope}%
\begin{pgfscope}%
\pgfpathrectangle{\pgfqpoint{0.150000in}{0.150000in}}{\pgfqpoint{2.700000in}{1.950000in}}%
\pgfusepath{clip}%
\pgfsetbuttcap%
\pgfsetroundjoin%
\definecolor{currentfill}{rgb}{0.785432,0.811857,0.848851}%
\pgfsetfillcolor{currentfill}%
\pgfsetlinewidth{0.000000pt}%
\definecolor{currentstroke}{rgb}{0.000000,0.000000,0.000000}%
\pgfsetstrokecolor{currentstroke}%
\pgfsetdash{}{0pt}%
\pgfpathmoveto{\pgfqpoint{1.612338in}{1.379544in}}%
\pgfpathlineto{\pgfqpoint{1.649859in}{1.379874in}}%
\pgfpathlineto{\pgfqpoint{1.612017in}{1.411102in}}%
\pgfpathlineto{\pgfqpoint{1.574364in}{1.404667in}}%
\pgfpathclose%
\pgfusepath{fill}%
\end{pgfscope}%
\begin{pgfscope}%
\pgfpathrectangle{\pgfqpoint{0.150000in}{0.150000in}}{\pgfqpoint{2.700000in}{1.950000in}}%
\pgfusepath{clip}%
\pgfsetbuttcap%
\pgfsetroundjoin%
\definecolor{currentfill}{rgb}{0.878722,0.893658,0.914568}%
\pgfsetfillcolor{currentfill}%
\pgfsetlinewidth{0.000000pt}%
\definecolor{currentstroke}{rgb}{0.000000,0.000000,0.000000}%
\pgfsetstrokecolor{currentstroke}%
\pgfsetdash{}{0pt}%
\pgfpathmoveto{\pgfqpoint{1.801442in}{1.242290in}}%
\pgfpathlineto{\pgfqpoint{1.838739in}{1.261418in}}%
\pgfpathlineto{\pgfqpoint{1.801385in}{1.317289in}}%
\pgfpathlineto{\pgfqpoint{1.763828in}{1.292079in}}%
\pgfpathclose%
\pgfusepath{fill}%
\end{pgfscope}%
\begin{pgfscope}%
\pgfpathrectangle{\pgfqpoint{0.150000in}{0.150000in}}{\pgfqpoint{2.700000in}{1.950000in}}%
\pgfusepath{clip}%
\pgfsetbuttcap%
\pgfsetroundjoin%
\definecolor{currentfill}{rgb}{0.956311,0.920726,0.923545}%
\pgfsetfillcolor{currentfill}%
\pgfsetlinewidth{0.000000pt}%
\definecolor{currentstroke}{rgb}{0.000000,0.000000,0.000000}%
\pgfsetstrokecolor{currentstroke}%
\pgfsetdash{}{0pt}%
\pgfpathmoveto{\pgfqpoint{1.876973in}{1.005132in}}%
\pgfpathlineto{\pgfqpoint{1.914113in}{1.024684in}}%
\pgfpathlineto{\pgfqpoint{1.876900in}{1.080145in}}%
\pgfpathlineto{\pgfqpoint{1.840156in}{1.078745in}}%
\pgfpathclose%
\pgfusepath{fill}%
\end{pgfscope}%
\begin{pgfscope}%
\pgfpathrectangle{\pgfqpoint{0.150000in}{0.150000in}}{\pgfqpoint{2.700000in}{1.950000in}}%
\pgfusepath{clip}%
\pgfsetbuttcap%
\pgfsetroundjoin%
\definecolor{currentfill}{rgb}{0.779213,0.806403,0.844470}%
\pgfsetfillcolor{currentfill}%
\pgfsetlinewidth{0.000000pt}%
\definecolor{currentstroke}{rgb}{0.000000,0.000000,0.000000}%
\pgfsetstrokecolor{currentstroke}%
\pgfsetdash{}{0pt}%
\pgfpathmoveto{\pgfqpoint{1.005220in}{1.385754in}}%
\pgfpathlineto{\pgfqpoint{1.045204in}{1.379874in}}%
\pgfpathlineto{\pgfqpoint{1.008077in}{1.404907in}}%
\pgfpathlineto{\pgfqpoint{0.967990in}{1.410884in}}%
\pgfpathclose%
\pgfusepath{fill}%
\end{pgfscope}%
\begin{pgfscope}%
\pgfpathrectangle{\pgfqpoint{0.150000in}{0.150000in}}{\pgfqpoint{2.700000in}{1.950000in}}%
\pgfusepath{clip}%
\pgfsetbuttcap%
\pgfsetroundjoin%
\definecolor{currentfill}{rgb}{0.834743,0.700138,0.710800}%
\pgfsetfillcolor{currentfill}%
\pgfsetlinewidth{0.000000pt}%
\definecolor{currentstroke}{rgb}{0.000000,0.000000,0.000000}%
\pgfsetstrokecolor{currentstroke}%
\pgfsetdash{}{0pt}%
\pgfpathmoveto{\pgfqpoint{1.801330in}{0.774050in}}%
\pgfpathlineto{\pgfqpoint{1.838437in}{0.788187in}}%
\pgfpathlineto{\pgfqpoint{1.801577in}{0.860903in}}%
\pgfpathlineto{\pgfqpoint{1.764125in}{0.840994in}}%
\pgfpathclose%
\pgfusepath{fill}%
\end{pgfscope}%
\begin{pgfscope}%
\pgfpathrectangle{\pgfqpoint{0.150000in}{0.150000in}}{\pgfqpoint{2.700000in}{1.950000in}}%
\pgfusepath{clip}%
\pgfsetbuttcap%
\pgfsetroundjoin%
\definecolor{currentfill}{rgb}{0.717019,0.751869,0.800659}%
\pgfsetfillcolor{currentfill}%
\pgfsetlinewidth{0.000000pt}%
\definecolor{currentstroke}{rgb}{0.000000,0.000000,0.000000}%
\pgfsetstrokecolor{currentstroke}%
\pgfsetdash{}{0pt}%
\pgfpathmoveto{\pgfqpoint{1.234018in}{1.423512in}}%
\pgfpathlineto{\pgfqpoint{1.271093in}{1.492209in}}%
\pgfpathlineto{\pgfqpoint{1.233387in}{1.523461in}}%
\pgfpathlineto{\pgfqpoint{1.196441in}{1.454725in}}%
\pgfpathclose%
\pgfusepath{fill}%
\end{pgfscope}%
\begin{pgfscope}%
\pgfpathrectangle{\pgfqpoint{0.150000in}{0.150000in}}{\pgfqpoint{2.700000in}{1.950000in}}%
\pgfusepath{clip}%
\pgfsetbuttcap%
\pgfsetroundjoin%
\definecolor{currentfill}{rgb}{0.717019,0.751869,0.800659}%
\pgfsetfillcolor{currentfill}%
\pgfsetlinewidth{0.000000pt}%
\definecolor{currentstroke}{rgb}{0.000000,0.000000,0.000000}%
\pgfsetstrokecolor{currentstroke}%
\pgfsetdash{}{0pt}%
\pgfpathmoveto{\pgfqpoint{0.853578in}{1.460914in}}%
\pgfpathlineto{\pgfqpoint{0.893810in}{1.460954in}}%
\pgfpathlineto{\pgfqpoint{0.856860in}{1.485895in}}%
\pgfpathlineto{\pgfqpoint{0.816547in}{1.485944in}}%
\pgfpathclose%
\pgfusepath{fill}%
\end{pgfscope}%
\begin{pgfscope}%
\pgfpathrectangle{\pgfqpoint{0.150000in}{0.150000in}}{\pgfqpoint{2.700000in}{1.950000in}}%
\pgfusepath{clip}%
\pgfsetbuttcap%
\pgfsetroundjoin%
\definecolor{currentfill}{rgb}{0.841406,0.860938,0.888281}%
\pgfsetfillcolor{currentfill}%
\pgfsetlinewidth{0.000000pt}%
\definecolor{currentstroke}{rgb}{0.000000,0.000000,0.000000}%
\pgfsetstrokecolor{currentstroke}%
\pgfsetdash{}{0pt}%
\pgfpathmoveto{\pgfqpoint{1.156928in}{1.310561in}}%
\pgfpathlineto{\pgfqpoint{1.196294in}{1.304990in}}%
\pgfpathlineto{\pgfqpoint{1.159186in}{1.323978in}}%
\pgfpathlineto{\pgfqpoint{1.119498in}{1.335791in}}%
\pgfpathclose%
\pgfusepath{fill}%
\end{pgfscope}%
\begin{pgfscope}%
\pgfpathrectangle{\pgfqpoint{0.150000in}{0.150000in}}{\pgfqpoint{2.700000in}{1.950000in}}%
\pgfusepath{clip}%
\pgfsetbuttcap%
\pgfsetroundjoin%
\definecolor{currentfill}{rgb}{0.895527,0.810432,0.817172}%
\pgfsetfillcolor{currentfill}%
\pgfsetlinewidth{0.000000pt}%
\definecolor{currentstroke}{rgb}{0.000000,0.000000,0.000000}%
\pgfsetstrokecolor{currentstroke}%
\pgfsetdash{}{0pt}%
\pgfpathmoveto{\pgfqpoint{1.838892in}{0.880739in}}%
\pgfpathlineto{\pgfqpoint{1.876071in}{0.900503in}}%
\pgfpathlineto{\pgfqpoint{1.839697in}{0.985508in}}%
\pgfpathlineto{\pgfqpoint{1.802284in}{0.965811in}}%
\pgfpathclose%
\pgfusepath{fill}%
\end{pgfscope}%
\begin{pgfscope}%
\pgfpathrectangle{\pgfqpoint{0.150000in}{0.150000in}}{\pgfqpoint{2.700000in}{1.950000in}}%
\pgfusepath{clip}%
\pgfsetbuttcap%
\pgfsetroundjoin%
\definecolor{currentfill}{rgb}{0.841406,0.860938,0.888281}%
\pgfsetfillcolor{currentfill}%
\pgfsetlinewidth{0.000000pt}%
\definecolor{currentstroke}{rgb}{0.000000,0.000000,0.000000}%
\pgfsetstrokecolor{currentstroke}%
\pgfsetdash{}{0pt}%
\pgfpathmoveto{\pgfqpoint{1.763828in}{1.292079in}}%
\pgfpathlineto{\pgfqpoint{1.801385in}{1.317289in}}%
\pgfpathlineto{\pgfqpoint{1.763257in}{1.342436in}}%
\pgfpathlineto{\pgfqpoint{1.725917in}{1.329617in}}%
\pgfpathclose%
\pgfusepath{fill}%
\end{pgfscope}%
\begin{pgfscope}%
\pgfpathrectangle{\pgfqpoint{0.150000in}{0.150000in}}{\pgfqpoint{2.700000in}{1.950000in}}%
\pgfusepath{clip}%
\pgfsetbuttcap%
\pgfsetroundjoin%
\definecolor{currentfill}{rgb}{0.934697,0.942739,0.953998}%
\pgfsetfillcolor{currentfill}%
\pgfsetlinewidth{0.000000pt}%
\definecolor{currentstroke}{rgb}{0.000000,0.000000,0.000000}%
\pgfsetstrokecolor{currentstroke}%
\pgfsetdash{}{0pt}%
\pgfpathmoveto{\pgfqpoint{1.801652in}{1.173335in}}%
\pgfpathlineto{\pgfqpoint{1.838804in}{1.186528in}}%
\pgfpathlineto{\pgfqpoint{1.801442in}{1.242290in}}%
\pgfpathlineto{\pgfqpoint{1.764009in}{1.223092in}}%
\pgfpathclose%
\pgfusepath{fill}%
\end{pgfscope}%
\begin{pgfscope}%
\pgfpathrectangle{\pgfqpoint{0.150000in}{0.150000in}}{\pgfqpoint{2.700000in}{1.950000in}}%
\pgfusepath{clip}%
\pgfsetbuttcap%
\pgfsetroundjoin%
\definecolor{currentfill}{rgb}{0.673483,0.713695,0.769991}%
\pgfsetfillcolor{currentfill}%
\pgfsetlinewidth{0.000000pt}%
\definecolor{currentstroke}{rgb}{0.000000,0.000000,0.000000}%
\pgfsetstrokecolor{currentstroke}%
\pgfsetdash{}{0pt}%
\pgfpathmoveto{\pgfqpoint{1.422549in}{1.511048in}}%
\pgfpathlineto{\pgfqpoint{1.460929in}{1.492138in}}%
\pgfpathlineto{\pgfqpoint{1.423293in}{1.517023in}}%
\pgfpathlineto{\pgfqpoint{1.384675in}{1.542342in}}%
\pgfpathclose%
\pgfusepath{fill}%
\end{pgfscope}%
\begin{pgfscope}%
\pgfpathrectangle{\pgfqpoint{0.150000in}{0.150000in}}{\pgfqpoint{2.700000in}{1.950000in}}%
\pgfusepath{clip}%
\pgfsetbuttcap%
\pgfsetroundjoin%
\definecolor{currentfill}{rgb}{0.866284,0.882751,0.905806}%
\pgfsetfillcolor{currentfill}%
\pgfsetlinewidth{0.000000pt}%
\definecolor{currentstroke}{rgb}{0.000000,0.000000,0.000000}%
\pgfsetstrokecolor{currentstroke}%
\pgfsetdash{}{0pt}%
\pgfpathmoveto{\pgfqpoint{1.271073in}{1.260666in}}%
\pgfpathlineto{\pgfqpoint{1.309145in}{1.292079in}}%
\pgfpathlineto{\pgfqpoint{1.271588in}{1.317289in}}%
\pgfpathlineto{\pgfqpoint{1.233713in}{1.279794in}}%
\pgfpathclose%
\pgfusepath{fill}%
\end{pgfscope}%
\begin{pgfscope}%
\pgfpathrectangle{\pgfqpoint{0.150000in}{0.150000in}}{\pgfqpoint{2.700000in}{1.950000in}}%
\pgfusepath{clip}%
\pgfsetbuttcap%
\pgfsetroundjoin%
\definecolor{currentfill}{rgb}{0.729458,0.762776,0.809421}%
\pgfsetfillcolor{currentfill}%
\pgfsetlinewidth{0.000000pt}%
\definecolor{currentstroke}{rgb}{0.000000,0.000000,0.000000}%
\pgfsetstrokecolor{currentstroke}%
\pgfsetdash{}{0pt}%
\pgfpathmoveto{\pgfqpoint{1.271493in}{1.398458in}}%
\pgfpathlineto{\pgfqpoint{1.308850in}{1.460914in}}%
\pgfpathlineto{\pgfqpoint{1.271093in}{1.492209in}}%
\pgfpathlineto{\pgfqpoint{1.234018in}{1.423512in}}%
\pgfpathclose%
\pgfusepath{fill}%
\end{pgfscope}%
\begin{pgfscope}%
\pgfpathrectangle{\pgfqpoint{0.150000in}{0.150000in}}{\pgfqpoint{2.700000in}{1.950000in}}%
\pgfusepath{clip}%
\pgfsetbuttcap%
\pgfsetroundjoin%
\definecolor{currentfill}{rgb}{0.865135,0.755285,0.763986}%
\pgfsetfillcolor{currentfill}%
\pgfsetlinewidth{0.000000pt}%
\definecolor{currentstroke}{rgb}{0.000000,0.000000,0.000000}%
\pgfsetstrokecolor{currentstroke}%
\pgfsetdash{}{0pt}%
\pgfpathmoveto{\pgfqpoint{1.992815in}{0.864640in}}%
\pgfpathlineto{\pgfqpoint{2.029933in}{0.884551in}}%
\pgfpathlineto{\pgfqpoint{1.989315in}{0.863072in}}%
\pgfpathlineto{\pgfqpoint{1.952339in}{0.843238in}}%
\pgfpathclose%
\pgfusepath{fill}%
\end{pgfscope}%
\begin{pgfscope}%
\pgfpathrectangle{\pgfqpoint{0.150000in}{0.150000in}}{\pgfqpoint{2.700000in}{1.950000in}}%
\pgfusepath{clip}%
\pgfsetbuttcap%
\pgfsetroundjoin%
\definecolor{currentfill}{rgb}{0.797871,0.822763,0.857613}%
\pgfsetfillcolor{currentfill}%
\pgfsetlinewidth{0.000000pt}%
\definecolor{currentstroke}{rgb}{0.000000,0.000000,0.000000}%
\pgfsetstrokecolor{currentstroke}%
\pgfsetdash{}{0pt}%
\pgfpathmoveto{\pgfqpoint{1.650407in}{1.354357in}}%
\pgfpathlineto{\pgfqpoint{1.687841in}{1.354777in}}%
\pgfpathlineto{\pgfqpoint{1.649859in}{1.379874in}}%
\pgfpathlineto{\pgfqpoint{1.612338in}{1.379544in}}%
\pgfpathclose%
\pgfusepath{fill}%
\end{pgfscope}%
\begin{pgfscope}%
\pgfpathrectangle{\pgfqpoint{0.150000in}{0.150000in}}{\pgfqpoint{2.700000in}{1.950000in}}%
\pgfusepath{clip}%
\pgfsetbuttcap%
\pgfsetroundjoin%
\definecolor{currentfill}{rgb}{0.499341,0.560999,0.647319}%
\pgfsetfillcolor{currentfill}%
\pgfsetlinewidth{0.000000pt}%
\definecolor{currentstroke}{rgb}{0.000000,0.000000,0.000000}%
\pgfsetstrokecolor{currentstroke}%
\pgfsetdash{}{0pt}%
\pgfpathmoveto{\pgfqpoint{1.082818in}{1.654385in}}%
\pgfpathlineto{\pgfqpoint{1.118749in}{1.742956in}}%
\pgfpathlineto{\pgfqpoint{1.080822in}{1.780529in}}%
\pgfpathlineto{\pgfqpoint{1.045347in}{1.685438in}}%
\pgfpathclose%
\pgfusepath{fill}%
\end{pgfscope}%
\begin{pgfscope}%
\pgfpathrectangle{\pgfqpoint{0.150000in}{0.150000in}}{\pgfqpoint{2.700000in}{1.950000in}}%
\pgfusepath{clip}%
\pgfsetbuttcap%
\pgfsetroundjoin%
\definecolor{currentfill}{rgb}{0.827145,0.686351,0.697503}%
\pgfsetfillcolor{currentfill}%
\pgfsetlinewidth{0.000000pt}%
\definecolor{currentstroke}{rgb}{0.000000,0.000000,0.000000}%
\pgfsetstrokecolor{currentstroke}%
\pgfsetdash{}{0pt}%
\pgfpathmoveto{\pgfqpoint{1.840596in}{0.783296in}}%
\pgfpathlineto{\pgfqpoint{1.877981in}{0.803350in}}%
\pgfpathlineto{\pgfqpoint{1.838437in}{0.788187in}}%
\pgfpathlineto{\pgfqpoint{1.801330in}{0.774050in}}%
\pgfpathclose%
\pgfusepath{fill}%
\end{pgfscope}%
\begin{pgfscope}%
\pgfpathrectangle{\pgfqpoint{0.150000in}{0.150000in}}{\pgfqpoint{2.700000in}{1.950000in}}%
\pgfusepath{clip}%
\pgfsetbuttcap%
\pgfsetroundjoin%
\definecolor{currentfill}{rgb}{0.828968,0.850031,0.879519}%
\pgfsetfillcolor{currentfill}%
\pgfsetlinewidth{0.000000pt}%
\definecolor{currentstroke}{rgb}{0.000000,0.000000,0.000000}%
\pgfsetstrokecolor{currentstroke}%
\pgfsetdash{}{0pt}%
\pgfpathmoveto{\pgfqpoint{1.309145in}{1.292079in}}%
\pgfpathlineto{\pgfqpoint{1.346837in}{1.341973in}}%
\pgfpathlineto{\pgfqpoint{1.309194in}{1.367146in}}%
\pgfpathlineto{\pgfqpoint{1.271588in}{1.317289in}}%
\pgfpathclose%
\pgfusepath{fill}%
\end{pgfscope}%
\begin{pgfscope}%
\pgfpathrectangle{\pgfqpoint{0.150000in}{0.150000in}}{\pgfqpoint{2.700000in}{1.950000in}}%
\pgfusepath{clip}%
\pgfsetbuttcap%
\pgfsetroundjoin%
\definecolor{currentfill}{rgb}{0.468244,0.533732,0.625414}%
\pgfsetfillcolor{currentfill}%
\pgfsetlinewidth{0.000000pt}%
\definecolor{currentstroke}{rgb}{0.000000,0.000000,0.000000}%
\pgfsetstrokecolor{currentstroke}%
\pgfsetdash{}{0pt}%
\pgfpathmoveto{\pgfqpoint{1.118749in}{1.742956in}}%
\pgfpathlineto{\pgfqpoint{1.158294in}{1.735595in}}%
\pgfpathlineto{\pgfqpoint{1.120758in}{1.766604in}}%
\pgfpathlineto{\pgfqpoint{1.080822in}{1.780529in}}%
\pgfpathclose%
\pgfusepath{fill}%
\end{pgfscope}%
\begin{pgfscope}%
\pgfpathrectangle{\pgfqpoint{0.150000in}{0.150000in}}{\pgfqpoint{2.700000in}{1.950000in}}%
\pgfusepath{clip}%
\pgfsetbuttcap%
\pgfsetroundjoin%
\definecolor{currentfill}{rgb}{0.692142,0.730055,0.783134}%
\pgfsetfillcolor{currentfill}%
\pgfsetlinewidth{0.000000pt}%
\definecolor{currentstroke}{rgb}{0.000000,0.000000,0.000000}%
\pgfsetstrokecolor{currentstroke}%
\pgfsetdash{}{0pt}%
\pgfpathmoveto{\pgfqpoint{1.460476in}{1.479712in}}%
\pgfpathlineto{\pgfqpoint{1.498682in}{1.460954in}}%
\pgfpathlineto{\pgfqpoint{1.460929in}{1.492138in}}%
\pgfpathlineto{\pgfqpoint{1.422549in}{1.511048in}}%
\pgfpathclose%
\pgfusepath{fill}%
\end{pgfscope}%
\begin{pgfscope}%
\pgfpathrectangle{\pgfqpoint{0.150000in}{0.150000in}}{\pgfqpoint{2.700000in}{1.950000in}}%
\pgfusepath{clip}%
\pgfsetbuttcap%
\pgfsetroundjoin%
\definecolor{currentfill}{rgb}{0.791651,0.817310,0.853232}%
\pgfsetfillcolor{currentfill}%
\pgfsetlinewidth{0.000000pt}%
\definecolor{currentstroke}{rgb}{0.000000,0.000000,0.000000}%
\pgfsetstrokecolor{currentstroke}%
\pgfsetdash{}{0pt}%
\pgfpathmoveto{\pgfqpoint{1.042544in}{1.360560in}}%
\pgfpathlineto{\pgfqpoint{1.082424in}{1.354777in}}%
\pgfpathlineto{\pgfqpoint{1.045204in}{1.379874in}}%
\pgfpathlineto{\pgfqpoint{1.005220in}{1.385754in}}%
\pgfpathclose%
\pgfusepath{fill}%
\end{pgfscope}%
\begin{pgfscope}%
\pgfpathrectangle{\pgfqpoint{0.150000in}{0.150000in}}{\pgfqpoint{2.700000in}{1.950000in}}%
\pgfusepath{clip}%
\pgfsetbuttcap%
\pgfsetroundjoin%
\definecolor{currentfill}{rgb}{0.723238,0.757322,0.805040}%
\pgfsetfillcolor{currentfill}%
\pgfsetlinewidth{0.000000pt}%
\definecolor{currentstroke}{rgb}{0.000000,0.000000,0.000000}%
\pgfsetstrokecolor{currentstroke}%
\pgfsetdash{}{0pt}%
\pgfpathmoveto{\pgfqpoint{0.890703in}{1.435821in}}%
\pgfpathlineto{\pgfqpoint{0.930853in}{1.435950in}}%
\pgfpathlineto{\pgfqpoint{0.893810in}{1.460954in}}%
\pgfpathlineto{\pgfqpoint{0.853578in}{1.460914in}}%
\pgfpathclose%
\pgfusepath{fill}%
\end{pgfscope}%
\begin{pgfscope}%
\pgfpathrectangle{\pgfqpoint{0.150000in}{0.150000in}}{\pgfqpoint{2.700000in}{1.950000in}}%
\pgfusepath{clip}%
\pgfsetbuttcap%
\pgfsetroundjoin%
\definecolor{currentfill}{rgb}{0.530438,0.588266,0.669225}%
\pgfsetfillcolor{currentfill}%
\pgfsetlinewidth{0.000000pt}%
\definecolor{currentstroke}{rgb}{0.000000,0.000000,0.000000}%
\pgfsetstrokecolor{currentstroke}%
\pgfsetdash{}{0pt}%
\pgfpathmoveto{\pgfqpoint{1.120339in}{1.623289in}}%
\pgfpathlineto{\pgfqpoint{1.156684in}{1.705374in}}%
\pgfpathlineto{\pgfqpoint{1.118749in}{1.742956in}}%
\pgfpathlineto{\pgfqpoint{1.082818in}{1.654385in}}%
\pgfpathclose%
\pgfusepath{fill}%
\end{pgfscope}%
\begin{pgfscope}%
\pgfpathrectangle{\pgfqpoint{0.150000in}{0.150000in}}{\pgfqpoint{2.700000in}{1.950000in}}%
\pgfusepath{clip}%
\pgfsetbuttcap%
\pgfsetroundjoin%
\definecolor{currentfill}{rgb}{0.884942,0.899112,0.918949}%
\pgfsetfillcolor{currentfill}%
\pgfsetlinewidth{0.000000pt}%
\definecolor{currentstroke}{rgb}{0.000000,0.000000,0.000000}%
\pgfsetstrokecolor{currentstroke}%
\pgfsetdash{}{0pt}%
\pgfpathmoveto{\pgfqpoint{1.764009in}{1.223092in}}%
\pgfpathlineto{\pgfqpoint{1.801442in}{1.242290in}}%
\pgfpathlineto{\pgfqpoint{1.763828in}{1.292079in}}%
\pgfpathlineto{\pgfqpoint{1.726177in}{1.266806in}}%
\pgfpathclose%
\pgfusepath{fill}%
\end{pgfscope}%
\begin{pgfscope}%
\pgfpathrectangle{\pgfqpoint{0.150000in}{0.150000in}}{\pgfqpoint{2.700000in}{1.950000in}}%
\pgfusepath{clip}%
\pgfsetbuttcap%
\pgfsetroundjoin%
\definecolor{currentfill}{rgb}{0.990502,0.982767,0.983379}%
\pgfsetfillcolor{currentfill}%
\pgfsetlinewidth{0.000000pt}%
\definecolor{currentstroke}{rgb}{0.000000,0.000000,0.000000}%
\pgfsetstrokecolor{currentstroke}%
\pgfsetdash{}{0pt}%
\pgfpathmoveto{\pgfqpoint{1.840156in}{1.078745in}}%
\pgfpathlineto{\pgfqpoint{1.876900in}{1.080145in}}%
\pgfpathlineto{\pgfqpoint{1.839391in}{1.129730in}}%
\pgfpathlineto{\pgfqpoint{1.802169in}{1.116374in}}%
\pgfpathclose%
\pgfusepath{fill}%
\end{pgfscope}%
\begin{pgfscope}%
\pgfpathrectangle{\pgfqpoint{0.150000in}{0.150000in}}{\pgfqpoint{2.700000in}{1.950000in}}%
\pgfusepath{clip}%
\pgfsetbuttcap%
\pgfsetroundjoin%
\definecolor{currentfill}{rgb}{0.853845,0.871844,0.897044}%
\pgfsetfillcolor{currentfill}%
\pgfsetlinewidth{0.000000pt}%
\definecolor{currentstroke}{rgb}{0.000000,0.000000,0.000000}%
\pgfsetstrokecolor{currentstroke}%
\pgfsetdash{}{0pt}%
\pgfpathmoveto{\pgfqpoint{1.194452in}{1.285267in}}%
\pgfpathlineto{\pgfqpoint{1.233713in}{1.279794in}}%
\pgfpathlineto{\pgfqpoint{1.196294in}{1.304990in}}%
\pgfpathlineto{\pgfqpoint{1.156928in}{1.310561in}}%
\pgfpathclose%
\pgfusepath{fill}%
\end{pgfscope}%
\begin{pgfscope}%
\pgfpathrectangle{\pgfqpoint{0.150000in}{0.150000in}}{\pgfqpoint{2.700000in}{1.950000in}}%
\pgfusepath{clip}%
\pgfsetbuttcap%
\pgfsetroundjoin%
\definecolor{currentfill}{rgb}{0.493122,0.555545,0.642938}%
\pgfsetfillcolor{currentfill}%
\pgfsetlinewidth{0.000000pt}%
\definecolor{currentstroke}{rgb}{0.000000,0.000000,0.000000}%
\pgfsetstrokecolor{currentstroke}%
\pgfsetdash{}{0pt}%
\pgfpathmoveto{\pgfqpoint{1.156684in}{1.705374in}}%
\pgfpathlineto{\pgfqpoint{1.195880in}{1.704544in}}%
\pgfpathlineto{\pgfqpoint{1.158294in}{1.735595in}}%
\pgfpathlineto{\pgfqpoint{1.118749in}{1.742956in}}%
\pgfpathclose%
\pgfusepath{fill}%
\end{pgfscope}%
\begin{pgfscope}%
\pgfpathrectangle{\pgfqpoint{0.150000in}{0.150000in}}{\pgfqpoint{2.700000in}{1.950000in}}%
\pgfusepath{clip}%
\pgfsetbuttcap%
\pgfsetroundjoin%
\definecolor{currentfill}{rgb}{0.754335,0.784589,0.826945}%
\pgfsetfillcolor{currentfill}%
\pgfsetlinewidth{0.000000pt}%
\definecolor{currentstroke}{rgb}{0.000000,0.000000,0.000000}%
\pgfsetstrokecolor{currentstroke}%
\pgfsetdash{}{0pt}%
\pgfpathmoveto{\pgfqpoint{1.309194in}{1.367146in}}%
\pgfpathlineto{\pgfqpoint{1.346769in}{1.423338in}}%
\pgfpathlineto{\pgfqpoint{1.308850in}{1.460914in}}%
\pgfpathlineto{\pgfqpoint{1.271493in}{1.398458in}}%
\pgfpathclose%
\pgfusepath{fill}%
\end{pgfscope}%
\begin{pgfscope}%
\pgfpathrectangle{\pgfqpoint{0.150000in}{0.150000in}}{\pgfqpoint{2.700000in}{1.950000in}}%
\pgfusepath{clip}%
\pgfsetbuttcap%
\pgfsetroundjoin%
\definecolor{currentfill}{rgb}{0.555316,0.610080,0.686749}%
\pgfsetfillcolor{currentfill}%
\pgfsetlinewidth{0.000000pt}%
\definecolor{currentstroke}{rgb}{0.000000,0.000000,0.000000}%
\pgfsetstrokecolor{currentstroke}%
\pgfsetdash{}{0pt}%
\pgfpathmoveto{\pgfqpoint{1.158130in}{1.585838in}}%
\pgfpathlineto{\pgfqpoint{1.194627in}{1.667784in}}%
\pgfpathlineto{\pgfqpoint{1.156684in}{1.705374in}}%
\pgfpathlineto{\pgfqpoint{1.120339in}{1.623289in}}%
\pgfpathclose%
\pgfusepath{fill}%
\end{pgfscope}%
\begin{pgfscope}%
\pgfpathrectangle{\pgfqpoint{0.150000in}{0.150000in}}{\pgfqpoint{2.700000in}{1.950000in}}%
\pgfusepath{clip}%
\pgfsetbuttcap%
\pgfsetroundjoin%
\definecolor{currentfill}{rgb}{0.586412,0.637347,0.708655}%
\pgfsetfillcolor{currentfill}%
\pgfsetlinewidth{0.000000pt}%
\definecolor{currentstroke}{rgb}{0.000000,0.000000,0.000000}%
\pgfsetstrokecolor{currentstroke}%
\pgfsetdash{}{0pt}%
\pgfpathmoveto{\pgfqpoint{1.195733in}{1.554671in}}%
\pgfpathlineto{\pgfqpoint{1.232578in}{1.630186in}}%
\pgfpathlineto{\pgfqpoint{1.194627in}{1.667784in}}%
\pgfpathlineto{\pgfqpoint{1.158130in}{1.585838in}}%
\pgfpathclose%
\pgfusepath{fill}%
\end{pgfscope}%
\begin{pgfscope}%
\pgfpathrectangle{\pgfqpoint{0.150000in}{0.150000in}}{\pgfqpoint{2.700000in}{1.950000in}}%
\pgfusepath{clip}%
\pgfsetbuttcap%
\pgfsetroundjoin%
\definecolor{currentfill}{rgb}{0.972013,0.975460,0.980285}%
\pgfsetfillcolor{currentfill}%
\pgfsetlinewidth{0.000000pt}%
\definecolor{currentstroke}{rgb}{0.000000,0.000000,0.000000}%
\pgfsetstrokecolor{currentstroke}%
\pgfsetdash{}{0pt}%
\pgfpathmoveto{\pgfqpoint{1.802169in}{1.116374in}}%
\pgfpathlineto{\pgfqpoint{1.839391in}{1.129730in}}%
\pgfpathlineto{\pgfqpoint{1.801652in}{1.173335in}}%
\pgfpathlineto{\pgfqpoint{1.764321in}{1.160078in}}%
\pgfpathclose%
\pgfusepath{fill}%
\end{pgfscope}%
\begin{pgfscope}%
\pgfpathrectangle{\pgfqpoint{0.150000in}{0.150000in}}{\pgfqpoint{2.700000in}{1.950000in}}%
\pgfusepath{clip}%
\pgfsetbuttcap%
\pgfsetroundjoin%
\definecolor{currentfill}{rgb}{0.810309,0.833670,0.866376}%
\pgfsetfillcolor{currentfill}%
\pgfsetlinewidth{0.000000pt}%
\definecolor{currentstroke}{rgb}{0.000000,0.000000,0.000000}%
\pgfsetstrokecolor{currentstroke}%
\pgfsetdash{}{0pt}%
\pgfpathmoveto{\pgfqpoint{1.688398in}{1.316736in}}%
\pgfpathlineto{\pgfqpoint{1.725917in}{1.329617in}}%
\pgfpathlineto{\pgfqpoint{1.687841in}{1.354777in}}%
\pgfpathlineto{\pgfqpoint{1.650407in}{1.354357in}}%
\pgfpathclose%
\pgfusepath{fill}%
\end{pgfscope}%
\begin{pgfscope}%
\pgfpathrectangle{\pgfqpoint{0.150000in}{0.150000in}}{\pgfqpoint{2.700000in}{1.950000in}}%
\pgfusepath{clip}%
\pgfsetbuttcap%
\pgfsetroundjoin%
\definecolor{currentfill}{rgb}{0.517999,0.577359,0.660463}%
\pgfsetfillcolor{currentfill}%
\pgfsetlinewidth{0.000000pt}%
\definecolor{currentstroke}{rgb}{0.000000,0.000000,0.000000}%
\pgfsetstrokecolor{currentstroke}%
\pgfsetdash{}{0pt}%
\pgfpathmoveto{\pgfqpoint{1.194627in}{1.667784in}}%
\pgfpathlineto{\pgfqpoint{1.233518in}{1.673451in}}%
\pgfpathlineto{\pgfqpoint{1.195880in}{1.704544in}}%
\pgfpathlineto{\pgfqpoint{1.156684in}{1.705374in}}%
\pgfpathclose%
\pgfusepath{fill}%
\end{pgfscope}%
\begin{pgfscope}%
\pgfpathrectangle{\pgfqpoint{0.150000in}{0.150000in}}{\pgfqpoint{2.700000in}{1.950000in}}%
\pgfusepath{clip}%
\pgfsetbuttcap%
\pgfsetroundjoin%
\definecolor{currentfill}{rgb}{0.895527,0.810432,0.817172}%
\pgfsetfillcolor{currentfill}%
\pgfsetlinewidth{0.000000pt}%
\definecolor{currentstroke}{rgb}{0.000000,0.000000,0.000000}%
\pgfsetstrokecolor{currentstroke}%
\pgfsetdash{}{0pt}%
\pgfpathmoveto{\pgfqpoint{1.801577in}{0.860903in}}%
\pgfpathlineto{\pgfqpoint{1.838892in}{0.880739in}}%
\pgfpathlineto{\pgfqpoint{1.802284in}{0.965811in}}%
\pgfpathlineto{\pgfqpoint{1.764733in}{0.946042in}}%
\pgfpathclose%
\pgfusepath{fill}%
\end{pgfscope}%
\begin{pgfscope}%
\pgfpathrectangle{\pgfqpoint{0.150000in}{0.150000in}}{\pgfqpoint{2.700000in}{1.950000in}}%
\pgfusepath{clip}%
\pgfsetbuttcap%
\pgfsetroundjoin%
\definecolor{currentfill}{rgb}{0.838542,0.707031,0.717448}%
\pgfsetfillcolor{currentfill}%
\pgfsetlinewidth{0.000000pt}%
\definecolor{currentstroke}{rgb}{0.000000,0.000000,0.000000}%
\pgfsetstrokecolor{currentstroke}%
\pgfsetdash{}{0pt}%
\pgfpathmoveto{\pgfqpoint{1.763912in}{0.754000in}}%
\pgfpathlineto{\pgfqpoint{1.801330in}{0.774050in}}%
\pgfpathlineto{\pgfqpoint{1.764125in}{0.840994in}}%
\pgfpathlineto{\pgfqpoint{1.726534in}{0.821011in}}%
\pgfpathclose%
\pgfusepath{fill}%
\end{pgfscope}%
\begin{pgfscope}%
\pgfpathrectangle{\pgfqpoint{0.150000in}{0.150000in}}{\pgfqpoint{2.700000in}{1.950000in}}%
\pgfusepath{clip}%
\pgfsetbuttcap%
\pgfsetroundjoin%
\definecolor{currentfill}{rgb}{0.841406,0.860938,0.888281}%
\pgfsetfillcolor{currentfill}%
\pgfsetlinewidth{0.000000pt}%
\definecolor{currentstroke}{rgb}{0.000000,0.000000,0.000000}%
\pgfsetstrokecolor{currentstroke}%
\pgfsetdash{}{0pt}%
\pgfpathmoveto{\pgfqpoint{1.726177in}{1.266806in}}%
\pgfpathlineto{\pgfqpoint{1.763828in}{1.292079in}}%
\pgfpathlineto{\pgfqpoint{1.725917in}{1.329617in}}%
\pgfpathlineto{\pgfqpoint{1.688398in}{1.316736in}}%
\pgfpathclose%
\pgfusepath{fill}%
\end{pgfscope}%
\begin{pgfscope}%
\pgfpathrectangle{\pgfqpoint{0.150000in}{0.150000in}}{\pgfqpoint{2.700000in}{1.950000in}}%
\pgfusepath{clip}%
\pgfsetbuttcap%
\pgfsetroundjoin%
\definecolor{currentfill}{rgb}{0.710800,0.746415,0.796278}%
\pgfsetfillcolor{currentfill}%
\pgfsetlinewidth{0.000000pt}%
\definecolor{currentstroke}{rgb}{0.000000,0.000000,0.000000}%
\pgfsetstrokecolor{currentstroke}%
\pgfsetdash{}{0pt}%
\pgfpathmoveto{\pgfqpoint{1.498455in}{1.448332in}}%
\pgfpathlineto{\pgfqpoint{1.536486in}{1.435950in}}%
\pgfpathlineto{\pgfqpoint{1.498682in}{1.460954in}}%
\pgfpathlineto{\pgfqpoint{1.460476in}{1.479712in}}%
\pgfpathclose%
\pgfusepath{fill}%
\end{pgfscope}%
\begin{pgfscope}%
\pgfpathrectangle{\pgfqpoint{0.150000in}{0.150000in}}{\pgfqpoint{2.700000in}{1.950000in}}%
\pgfusepath{clip}%
\pgfsetbuttcap%
\pgfsetroundjoin%
\definecolor{currentfill}{rgb}{0.772993,0.800950,0.840089}%
\pgfsetfillcolor{currentfill}%
\pgfsetlinewidth{0.000000pt}%
\definecolor{currentstroke}{rgb}{0.000000,0.000000,0.000000}%
\pgfsetstrokecolor{currentstroke}%
\pgfsetdash{}{0pt}%
\pgfpathmoveto{\pgfqpoint{1.346837in}{1.341973in}}%
\pgfpathlineto{\pgfqpoint{1.384608in}{1.391971in}}%
\pgfpathlineto{\pgfqpoint{1.346769in}{1.423338in}}%
\pgfpathlineto{\pgfqpoint{1.309194in}{1.367146in}}%
\pgfpathclose%
\pgfusepath{fill}%
\end{pgfscope}%
\begin{pgfscope}%
\pgfpathrectangle{\pgfqpoint{0.150000in}{0.150000in}}{\pgfqpoint{2.700000in}{1.950000in}}%
\pgfusepath{clip}%
\pgfsetbuttcap%
\pgfsetroundjoin%
\definecolor{currentfill}{rgb}{0.611290,0.659161,0.726180}%
\pgfsetfillcolor{currentfill}%
\pgfsetlinewidth{0.000000pt}%
\definecolor{currentstroke}{rgb}{0.000000,0.000000,0.000000}%
\pgfsetstrokecolor{currentstroke}%
\pgfsetdash{}{0pt}%
\pgfpathmoveto{\pgfqpoint{1.233387in}{1.523461in}}%
\pgfpathlineto{\pgfqpoint{1.270538in}{1.592580in}}%
\pgfpathlineto{\pgfqpoint{1.232578in}{1.630186in}}%
\pgfpathlineto{\pgfqpoint{1.195733in}{1.554671in}}%
\pgfpathclose%
\pgfusepath{fill}%
\end{pgfscope}%
\begin{pgfscope}%
\pgfpathrectangle{\pgfqpoint{0.150000in}{0.150000in}}{\pgfqpoint{2.700000in}{1.950000in}}%
\pgfusepath{clip}%
\pgfsetbuttcap%
\pgfsetroundjoin%
\definecolor{currentfill}{rgb}{0.872503,0.888205,0.910187}%
\pgfsetfillcolor{currentfill}%
\pgfsetlinewidth{0.000000pt}%
\definecolor{currentstroke}{rgb}{0.000000,0.000000,0.000000}%
\pgfsetstrokecolor{currentstroke}%
\pgfsetdash{}{0pt}%
\pgfpathmoveto{\pgfqpoint{1.308570in}{1.241468in}}%
\pgfpathlineto{\pgfqpoint{1.346796in}{1.266806in}}%
\pgfpathlineto{\pgfqpoint{1.309145in}{1.292079in}}%
\pgfpathlineto{\pgfqpoint{1.271073in}{1.260666in}}%
\pgfpathclose%
\pgfusepath{fill}%
\end{pgfscope}%
\begin{pgfscope}%
\pgfpathrectangle{\pgfqpoint{0.150000in}{0.150000in}}{\pgfqpoint{2.700000in}{1.950000in}}%
\pgfusepath{clip}%
\pgfsetbuttcap%
\pgfsetroundjoin%
\definecolor{currentfill}{rgb}{0.835187,0.855484,0.883900}%
\pgfsetfillcolor{currentfill}%
\pgfsetlinewidth{0.000000pt}%
\definecolor{currentstroke}{rgb}{0.000000,0.000000,0.000000}%
\pgfsetstrokecolor{currentstroke}%
\pgfsetdash{}{0pt}%
\pgfpathmoveto{\pgfqpoint{1.346796in}{1.266806in}}%
\pgfpathlineto{\pgfqpoint{1.384575in}{1.316736in}}%
\pgfpathlineto{\pgfqpoint{1.346837in}{1.341973in}}%
\pgfpathlineto{\pgfqpoint{1.309145in}{1.292079in}}%
\pgfpathclose%
\pgfusepath{fill}%
\end{pgfscope}%
\begin{pgfscope}%
\pgfpathrectangle{\pgfqpoint{0.150000in}{0.150000in}}{\pgfqpoint{2.700000in}{1.950000in}}%
\pgfusepath{clip}%
\pgfsetbuttcap%
\pgfsetroundjoin%
\definecolor{currentfill}{rgb}{0.542877,0.599173,0.677987}%
\pgfsetfillcolor{currentfill}%
\pgfsetlinewidth{0.000000pt}%
\definecolor{currentstroke}{rgb}{0.000000,0.000000,0.000000}%
\pgfsetstrokecolor{currentstroke}%
\pgfsetdash{}{0pt}%
\pgfpathmoveto{\pgfqpoint{1.232578in}{1.630186in}}%
\pgfpathlineto{\pgfqpoint{1.271208in}{1.642315in}}%
\pgfpathlineto{\pgfqpoint{1.233518in}{1.673451in}}%
\pgfpathlineto{\pgfqpoint{1.194627in}{1.667784in}}%
\pgfpathclose%
\pgfusepath{fill}%
\end{pgfscope}%
\begin{pgfscope}%
\pgfpathrectangle{\pgfqpoint{0.150000in}{0.150000in}}{\pgfqpoint{2.700000in}{1.950000in}}%
\pgfusepath{clip}%
\pgfsetbuttcap%
\pgfsetroundjoin%
\definecolor{currentfill}{rgb}{0.735677,0.768229,0.813802}%
\pgfsetfillcolor{currentfill}%
\pgfsetlinewidth{0.000000pt}%
\definecolor{currentstroke}{rgb}{0.000000,0.000000,0.000000}%
\pgfsetstrokecolor{currentstroke}%
\pgfsetdash{}{0pt}%
\pgfpathmoveto{\pgfqpoint{0.928273in}{1.404426in}}%
\pgfpathlineto{\pgfqpoint{0.967990in}{1.410884in}}%
\pgfpathlineto{\pgfqpoint{0.930853in}{1.435950in}}%
\pgfpathlineto{\pgfqpoint{0.890703in}{1.435821in}}%
\pgfpathclose%
\pgfusepath{fill}%
\end{pgfscope}%
\begin{pgfscope}%
\pgfpathrectangle{\pgfqpoint{0.150000in}{0.150000in}}{\pgfqpoint{2.700000in}{1.950000in}}%
\pgfusepath{clip}%
\pgfsetbuttcap%
\pgfsetroundjoin%
\definecolor{currentfill}{rgb}{0.928477,0.937286,0.949617}%
\pgfsetfillcolor{currentfill}%
\pgfsetlinewidth{0.000000pt}%
\definecolor{currentstroke}{rgb}{0.000000,0.000000,0.000000}%
\pgfsetstrokecolor{currentstroke}%
\pgfsetdash{}{0pt}%
\pgfpathmoveto{\pgfqpoint{1.764321in}{1.160078in}}%
\pgfpathlineto{\pgfqpoint{1.801652in}{1.173335in}}%
\pgfpathlineto{\pgfqpoint{1.764009in}{1.223092in}}%
\pgfpathlineto{\pgfqpoint{1.726437in}{1.203822in}}%
\pgfpathclose%
\pgfusepath{fill}%
\end{pgfscope}%
\begin{pgfscope}%
\pgfpathrectangle{\pgfqpoint{0.150000in}{0.150000in}}{\pgfqpoint{2.700000in}{1.950000in}}%
\pgfusepath{clip}%
\pgfsetbuttcap%
\pgfsetroundjoin%
\definecolor{currentfill}{rgb}{0.963909,0.934513,0.936841}%
\pgfsetfillcolor{currentfill}%
\pgfsetlinewidth{0.000000pt}%
\definecolor{currentstroke}{rgb}{0.000000,0.000000,0.000000}%
\pgfsetstrokecolor{currentstroke}%
\pgfsetdash{}{0pt}%
\pgfpathmoveto{\pgfqpoint{1.839697in}{0.985508in}}%
\pgfpathlineto{\pgfqpoint{1.876973in}{1.005132in}}%
\pgfpathlineto{\pgfqpoint{1.840156in}{1.078745in}}%
\pgfpathlineto{\pgfqpoint{1.802995in}{1.071280in}}%
\pgfpathclose%
\pgfusepath{fill}%
\end{pgfscope}%
\begin{pgfscope}%
\pgfpathrectangle{\pgfqpoint{0.150000in}{0.150000in}}{\pgfqpoint{2.700000in}{1.950000in}}%
\pgfusepath{clip}%
\pgfsetbuttcap%
\pgfsetroundjoin%
\definecolor{currentfill}{rgb}{0.573974,0.626440,0.699893}%
\pgfsetfillcolor{currentfill}%
\pgfsetlinewidth{0.000000pt}%
\definecolor{currentstroke}{rgb}{0.000000,0.000000,0.000000}%
\pgfsetstrokecolor{currentstroke}%
\pgfsetdash{}{0pt}%
\pgfpathmoveto{\pgfqpoint{1.270538in}{1.592580in}}%
\pgfpathlineto{\pgfqpoint{1.309080in}{1.604800in}}%
\pgfpathlineto{\pgfqpoint{1.271208in}{1.642315in}}%
\pgfpathlineto{\pgfqpoint{1.232578in}{1.630186in}}%
\pgfpathclose%
\pgfusepath{fill}%
\end{pgfscope}%
\begin{pgfscope}%
\pgfpathrectangle{\pgfqpoint{0.150000in}{0.150000in}}{\pgfqpoint{2.700000in}{1.950000in}}%
\pgfusepath{clip}%
\pgfsetbuttcap%
\pgfsetroundjoin%
\definecolor{currentfill}{rgb}{0.865135,0.755285,0.763986}%
\pgfsetfillcolor{currentfill}%
\pgfsetlinewidth{0.000000pt}%
\definecolor{currentstroke}{rgb}{0.000000,0.000000,0.000000}%
\pgfsetstrokecolor{currentstroke}%
\pgfsetdash{}{0pt}%
\pgfpathmoveto{\pgfqpoint{1.955317in}{0.838733in}}%
\pgfpathlineto{\pgfqpoint{1.992815in}{0.864640in}}%
\pgfpathlineto{\pgfqpoint{1.952339in}{0.843238in}}%
\pgfpathlineto{\pgfqpoint{1.915228in}{0.823331in}}%
\pgfpathclose%
\pgfusepath{fill}%
\end{pgfscope}%
\begin{pgfscope}%
\pgfpathrectangle{\pgfqpoint{0.150000in}{0.150000in}}{\pgfqpoint{2.700000in}{1.950000in}}%
\pgfusepath{clip}%
\pgfsetbuttcap%
\pgfsetroundjoin%
\definecolor{currentfill}{rgb}{0.797871,0.822763,0.857613}%
\pgfsetfillcolor{currentfill}%
\pgfsetlinewidth{0.000000pt}%
\definecolor{currentstroke}{rgb}{0.000000,0.000000,0.000000}%
\pgfsetstrokecolor{currentstroke}%
\pgfsetdash{}{0pt}%
\pgfpathmoveto{\pgfqpoint{1.079963in}{1.335303in}}%
\pgfpathlineto{\pgfqpoint{1.119498in}{1.335791in}}%
\pgfpathlineto{\pgfqpoint{1.082424in}{1.354777in}}%
\pgfpathlineto{\pgfqpoint{1.042544in}{1.360560in}}%
\pgfpathclose%
\pgfusepath{fill}%
\end{pgfscope}%
\begin{pgfscope}%
\pgfpathrectangle{\pgfqpoint{0.150000in}{0.150000in}}{\pgfqpoint{2.700000in}{1.950000in}}%
\pgfusepath{clip}%
\pgfsetbuttcap%
\pgfsetroundjoin%
\definecolor{currentfill}{rgb}{0.636167,0.680974,0.743704}%
\pgfsetfillcolor{currentfill}%
\pgfsetlinewidth{0.000000pt}%
\definecolor{currentstroke}{rgb}{0.000000,0.000000,0.000000}%
\pgfsetstrokecolor{currentstroke}%
\pgfsetdash{}{0pt}%
\pgfpathmoveto{\pgfqpoint{1.271093in}{1.492209in}}%
\pgfpathlineto{\pgfqpoint{1.308506in}{1.554965in}}%
\pgfpathlineto{\pgfqpoint{1.270538in}{1.592580in}}%
\pgfpathlineto{\pgfqpoint{1.233387in}{1.523461in}}%
\pgfpathclose%
\pgfusepath{fill}%
\end{pgfscope}%
\begin{pgfscope}%
\pgfpathrectangle{\pgfqpoint{0.150000in}{0.150000in}}{\pgfqpoint{2.700000in}{1.950000in}}%
\pgfusepath{clip}%
\pgfsetbuttcap%
\pgfsetroundjoin%
\definecolor{currentfill}{rgb}{0.729458,0.762776,0.809421}%
\pgfsetfillcolor{currentfill}%
\pgfsetlinewidth{0.000000pt}%
\definecolor{currentstroke}{rgb}{0.000000,0.000000,0.000000}%
\pgfsetstrokecolor{currentstroke}%
\pgfsetdash{}{0pt}%
\pgfpathmoveto{\pgfqpoint{1.536486in}{1.410664in}}%
\pgfpathlineto{\pgfqpoint{1.574364in}{1.404667in}}%
\pgfpathlineto{\pgfqpoint{1.536486in}{1.435950in}}%
\pgfpathlineto{\pgfqpoint{1.498455in}{1.448332in}}%
\pgfpathclose%
\pgfusepath{fill}%
\end{pgfscope}%
\begin{pgfscope}%
\pgfpathrectangle{\pgfqpoint{0.150000in}{0.150000in}}{\pgfqpoint{2.700000in}{1.950000in}}%
\pgfusepath{clip}%
\pgfsetbuttcap%
\pgfsetroundjoin%
\definecolor{currentfill}{rgb}{0.598851,0.648254,0.717417}%
\pgfsetfillcolor{currentfill}%
\pgfsetlinewidth{0.000000pt}%
\definecolor{currentstroke}{rgb}{0.000000,0.000000,0.000000}%
\pgfsetstrokecolor{currentstroke}%
\pgfsetdash{}{0pt}%
\pgfpathmoveto{\pgfqpoint{1.308506in}{1.554965in}}%
\pgfpathlineto{\pgfqpoint{1.346851in}{1.573593in}}%
\pgfpathlineto{\pgfqpoint{1.309080in}{1.604800in}}%
\pgfpathlineto{\pgfqpoint{1.270538in}{1.592580in}}%
\pgfpathclose%
\pgfusepath{fill}%
\end{pgfscope}%
\begin{pgfscope}%
\pgfpathrectangle{\pgfqpoint{0.150000in}{0.150000in}}{\pgfqpoint{2.700000in}{1.950000in}}%
\pgfusepath{clip}%
\pgfsetbuttcap%
\pgfsetroundjoin%
\definecolor{currentfill}{rgb}{0.860064,0.877298,0.901425}%
\pgfsetfillcolor{currentfill}%
\pgfsetlinewidth{0.000000pt}%
\definecolor{currentstroke}{rgb}{0.000000,0.000000,0.000000}%
\pgfsetstrokecolor{currentstroke}%
\pgfsetdash{}{0pt}%
\pgfpathmoveto{\pgfqpoint{1.232071in}{1.259909in}}%
\pgfpathlineto{\pgfqpoint{1.271073in}{1.260666in}}%
\pgfpathlineto{\pgfqpoint{1.233713in}{1.279794in}}%
\pgfpathlineto{\pgfqpoint{1.194452in}{1.285267in}}%
\pgfpathclose%
\pgfusepath{fill}%
\end{pgfscope}%
\begin{pgfscope}%
\pgfpathrectangle{\pgfqpoint{0.150000in}{0.150000in}}{\pgfqpoint{2.700000in}{1.950000in}}%
\pgfusepath{clip}%
\pgfsetbuttcap%
\pgfsetroundjoin%
\definecolor{currentfill}{rgb}{0.887929,0.796645,0.803876}%
\pgfsetfillcolor{currentfill}%
\pgfsetlinewidth{0.000000pt}%
\definecolor{currentstroke}{rgb}{0.000000,0.000000,0.000000}%
\pgfsetstrokecolor{currentstroke}%
\pgfsetdash{}{0pt}%
\pgfpathmoveto{\pgfqpoint{2.069794in}{0.882446in}}%
\pgfpathlineto{\pgfqpoint{2.105192in}{0.872713in}}%
\pgfpathlineto{\pgfqpoint{2.066610in}{0.898467in}}%
\pgfpathlineto{\pgfqpoint{2.029933in}{0.884551in}}%
\pgfpathclose%
\pgfusepath{fill}%
\end{pgfscope}%
\begin{pgfscope}%
\pgfpathrectangle{\pgfqpoint{0.150000in}{0.150000in}}{\pgfqpoint{2.700000in}{1.950000in}}%
\pgfusepath{clip}%
\pgfsetbuttcap%
\pgfsetroundjoin%
\definecolor{currentfill}{rgb}{0.891161,0.904565,0.923330}%
\pgfsetfillcolor{currentfill}%
\pgfsetlinewidth{0.000000pt}%
\definecolor{currentstroke}{rgb}{0.000000,0.000000,0.000000}%
\pgfsetstrokecolor{currentstroke}%
\pgfsetdash{}{0pt}%
\pgfpathmoveto{\pgfqpoint{1.726437in}{1.203822in}}%
\pgfpathlineto{\pgfqpoint{1.764009in}{1.223092in}}%
\pgfpathlineto{\pgfqpoint{1.726177in}{1.266806in}}%
\pgfpathlineto{\pgfqpoint{1.688431in}{1.241468in}}%
\pgfpathclose%
\pgfusepath{fill}%
\end{pgfscope}%
\begin{pgfscope}%
\pgfpathrectangle{\pgfqpoint{0.150000in}{0.150000in}}{\pgfqpoint{2.700000in}{1.950000in}}%
\pgfusepath{clip}%
\pgfsetbuttcap%
\pgfsetroundjoin%
\definecolor{currentfill}{rgb}{0.827145,0.686351,0.697503}%
\pgfsetfillcolor{currentfill}%
\pgfsetlinewidth{0.000000pt}%
\definecolor{currentstroke}{rgb}{0.000000,0.000000,0.000000}%
\pgfsetstrokecolor{currentstroke}%
\pgfsetdash{}{0pt}%
\pgfpathmoveto{\pgfqpoint{1.803073in}{0.763168in}}%
\pgfpathlineto{\pgfqpoint{1.840596in}{0.783296in}}%
\pgfpathlineto{\pgfqpoint{1.801330in}{0.774050in}}%
\pgfpathlineto{\pgfqpoint{1.763912in}{0.754000in}}%
\pgfpathclose%
\pgfusepath{fill}%
\end{pgfscope}%
\begin{pgfscope}%
\pgfpathrectangle{\pgfqpoint{0.150000in}{0.150000in}}{\pgfqpoint{2.700000in}{1.950000in}}%
\pgfusepath{clip}%
\pgfsetbuttcap%
\pgfsetroundjoin%
\definecolor{currentfill}{rgb}{0.661045,0.702788,0.761229}%
\pgfsetfillcolor{currentfill}%
\pgfsetlinewidth{0.000000pt}%
\definecolor{currentstroke}{rgb}{0.000000,0.000000,0.000000}%
\pgfsetstrokecolor{currentstroke}%
\pgfsetdash{}{0pt}%
\pgfpathmoveto{\pgfqpoint{1.308850in}{1.460914in}}%
\pgfpathlineto{\pgfqpoint{1.346482in}{1.517343in}}%
\pgfpathlineto{\pgfqpoint{1.308506in}{1.554965in}}%
\pgfpathlineto{\pgfqpoint{1.271093in}{1.492209in}}%
\pgfpathclose%
\pgfusepath{fill}%
\end{pgfscope}%
\begin{pgfscope}%
\pgfpathrectangle{\pgfqpoint{0.150000in}{0.150000in}}{\pgfqpoint{2.700000in}{1.950000in}}%
\pgfusepath{clip}%
\pgfsetbuttcap%
\pgfsetroundjoin%
\definecolor{currentfill}{rgb}{0.785432,0.811857,0.848851}%
\pgfsetfillcolor{currentfill}%
\pgfsetlinewidth{0.000000pt}%
\definecolor{currentstroke}{rgb}{0.000000,0.000000,0.000000}%
\pgfsetstrokecolor{currentstroke}%
\pgfsetdash{}{0pt}%
\pgfpathmoveto{\pgfqpoint{1.384575in}{1.316736in}}%
\pgfpathlineto{\pgfqpoint{1.422500in}{1.360560in}}%
\pgfpathlineto{\pgfqpoint{1.384608in}{1.391971in}}%
\pgfpathlineto{\pgfqpoint{1.346837in}{1.341973in}}%
\pgfpathclose%
\pgfusepath{fill}%
\end{pgfscope}%
\begin{pgfscope}%
\pgfpathrectangle{\pgfqpoint{0.150000in}{0.150000in}}{\pgfqpoint{2.700000in}{1.950000in}}%
\pgfusepath{clip}%
\pgfsetbuttcap%
\pgfsetroundjoin%
\definecolor{currentfill}{rgb}{0.623729,0.670067,0.734942}%
\pgfsetfillcolor{currentfill}%
\pgfsetlinewidth{0.000000pt}%
\definecolor{currentstroke}{rgb}{0.000000,0.000000,0.000000}%
\pgfsetstrokecolor{currentstroke}%
\pgfsetdash{}{0pt}%
\pgfpathmoveto{\pgfqpoint{1.346482in}{1.517343in}}%
\pgfpathlineto{\pgfqpoint{1.384675in}{1.542342in}}%
\pgfpathlineto{\pgfqpoint{1.346851in}{1.573593in}}%
\pgfpathlineto{\pgfqpoint{1.308506in}{1.554965in}}%
\pgfpathclose%
\pgfusepath{fill}%
\end{pgfscope}%
\begin{pgfscope}%
\pgfpathrectangle{\pgfqpoint{0.150000in}{0.150000in}}{\pgfqpoint{2.700000in}{1.950000in}}%
\pgfusepath{clip}%
\pgfsetbuttcap%
\pgfsetroundjoin%
\definecolor{currentfill}{rgb}{0.754335,0.784589,0.826945}%
\pgfsetfillcolor{currentfill}%
\pgfsetlinewidth{0.000000pt}%
\definecolor{currentstroke}{rgb}{0.000000,0.000000,0.000000}%
\pgfsetstrokecolor{currentstroke}%
\pgfsetdash{}{0pt}%
\pgfpathmoveto{\pgfqpoint{1.574526in}{1.372988in}}%
\pgfpathlineto{\pgfqpoint{1.612338in}{1.379544in}}%
\pgfpathlineto{\pgfqpoint{1.574364in}{1.404667in}}%
\pgfpathlineto{\pgfqpoint{1.536486in}{1.410664in}}%
\pgfpathclose%
\pgfusepath{fill}%
\end{pgfscope}%
\begin{pgfscope}%
\pgfpathrectangle{\pgfqpoint{0.150000in}{0.150000in}}{\pgfqpoint{2.700000in}{1.950000in}}%
\pgfusepath{clip}%
\pgfsetbuttcap%
\pgfsetroundjoin%
\definecolor{currentfill}{rgb}{0.685922,0.724602,0.778753}%
\pgfsetfillcolor{currentfill}%
\pgfsetlinewidth{0.000000pt}%
\definecolor{currentstroke}{rgb}{0.000000,0.000000,0.000000}%
\pgfsetstrokecolor{currentstroke}%
\pgfsetdash{}{0pt}%
\pgfpathmoveto{\pgfqpoint{1.346769in}{1.423338in}}%
\pgfpathlineto{\pgfqpoint{1.384466in}{1.479712in}}%
\pgfpathlineto{\pgfqpoint{1.346482in}{1.517343in}}%
\pgfpathlineto{\pgfqpoint{1.308850in}{1.460914in}}%
\pgfpathclose%
\pgfusepath{fill}%
\end{pgfscope}%
\begin{pgfscope}%
\pgfpathrectangle{\pgfqpoint{0.150000in}{0.150000in}}{\pgfqpoint{2.700000in}{1.950000in}}%
\pgfusepath{clip}%
\pgfsetbuttcap%
\pgfsetroundjoin%
\definecolor{currentfill}{rgb}{0.685922,0.724602,0.778753}%
\pgfsetfillcolor{currentfill}%
\pgfsetlinewidth{0.000000pt}%
\definecolor{currentstroke}{rgb}{0.000000,0.000000,0.000000}%
\pgfsetstrokecolor{currentstroke}%
\pgfsetdash{}{0pt}%
\pgfpathmoveto{\pgfqpoint{0.813476in}{1.454599in}}%
\pgfpathlineto{\pgfqpoint{0.853578in}{1.460914in}}%
\pgfpathlineto{\pgfqpoint{0.816547in}{1.485944in}}%
\pgfpathlineto{\pgfqpoint{0.776385in}{1.479712in}}%
\pgfpathclose%
\pgfusepath{fill}%
\end{pgfscope}%
\begin{pgfscope}%
\pgfpathrectangle{\pgfqpoint{0.150000in}{0.150000in}}{\pgfqpoint{2.700000in}{1.950000in}}%
\pgfusepath{clip}%
\pgfsetbuttcap%
\pgfsetroundjoin%
\definecolor{currentfill}{rgb}{0.847626,0.866391,0.892662}%
\pgfsetfillcolor{currentfill}%
\pgfsetlinewidth{0.000000pt}%
\definecolor{currentstroke}{rgb}{0.000000,0.000000,0.000000}%
\pgfsetstrokecolor{currentstroke}%
\pgfsetdash{}{0pt}%
\pgfpathmoveto{\pgfqpoint{1.384542in}{1.241468in}}%
\pgfpathlineto{\pgfqpoint{1.422475in}{1.285267in}}%
\pgfpathlineto{\pgfqpoint{1.384575in}{1.316736in}}%
\pgfpathlineto{\pgfqpoint{1.346796in}{1.266806in}}%
\pgfpathclose%
\pgfusepath{fill}%
\end{pgfscope}%
\begin{pgfscope}%
\pgfpathrectangle{\pgfqpoint{0.150000in}{0.150000in}}{\pgfqpoint{2.700000in}{1.950000in}}%
\pgfusepath{clip}%
\pgfsetbuttcap%
\pgfsetroundjoin%
\definecolor{currentfill}{rgb}{0.847626,0.866391,0.892662}%
\pgfsetfillcolor{currentfill}%
\pgfsetlinewidth{0.000000pt}%
\definecolor{currentstroke}{rgb}{0.000000,0.000000,0.000000}%
\pgfsetstrokecolor{currentstroke}%
\pgfsetdash{}{0pt}%
\pgfpathmoveto{\pgfqpoint{1.688431in}{1.241468in}}%
\pgfpathlineto{\pgfqpoint{1.726177in}{1.266806in}}%
\pgfpathlineto{\pgfqpoint{1.688398in}{1.316736in}}%
\pgfpathlineto{\pgfqpoint{1.650498in}{1.285267in}}%
\pgfpathclose%
\pgfusepath{fill}%
\end{pgfscope}%
\begin{pgfscope}%
\pgfpathrectangle{\pgfqpoint{0.150000in}{0.150000in}}{\pgfqpoint{2.700000in}{1.950000in}}%
\pgfusepath{clip}%
\pgfsetbuttcap%
\pgfsetroundjoin%
\definecolor{currentfill}{rgb}{0.748116,0.779136,0.822564}%
\pgfsetfillcolor{currentfill}%
\pgfsetlinewidth{0.000000pt}%
\definecolor{currentstroke}{rgb}{0.000000,0.000000,0.000000}%
\pgfsetstrokecolor{currentstroke}%
\pgfsetdash{}{0pt}%
\pgfpathmoveto{\pgfqpoint{0.965564in}{1.379212in}}%
\pgfpathlineto{\pgfqpoint{1.005220in}{1.385754in}}%
\pgfpathlineto{\pgfqpoint{0.967990in}{1.410884in}}%
\pgfpathlineto{\pgfqpoint{0.928273in}{1.404426in}}%
\pgfpathclose%
\pgfusepath{fill}%
\end{pgfscope}%
\begin{pgfscope}%
\pgfpathrectangle{\pgfqpoint{0.150000in}{0.150000in}}{\pgfqpoint{2.700000in}{1.950000in}}%
\pgfusepath{clip}%
\pgfsetbuttcap%
\pgfsetroundjoin%
\definecolor{currentfill}{rgb}{0.717019,0.751869,0.800659}%
\pgfsetfillcolor{currentfill}%
\pgfsetlinewidth{0.000000pt}%
\definecolor{currentstroke}{rgb}{0.000000,0.000000,0.000000}%
\pgfsetstrokecolor{currentstroke}%
\pgfsetdash{}{0pt}%
\pgfpathmoveto{\pgfqpoint{1.384608in}{1.391971in}}%
\pgfpathlineto{\pgfqpoint{1.422459in}{1.442073in}}%
\pgfpathlineto{\pgfqpoint{1.384466in}{1.479712in}}%
\pgfpathlineto{\pgfqpoint{1.346769in}{1.423338in}}%
\pgfpathclose%
\pgfusepath{fill}%
\end{pgfscope}%
\begin{pgfscope}%
\pgfpathrectangle{\pgfqpoint{0.150000in}{0.150000in}}{\pgfqpoint{2.700000in}{1.950000in}}%
\pgfusepath{clip}%
\pgfsetbuttcap%
\pgfsetroundjoin%
\definecolor{currentfill}{rgb}{0.648606,0.691881,0.752466}%
\pgfsetfillcolor{currentfill}%
\pgfsetlinewidth{0.000000pt}%
\definecolor{currentstroke}{rgb}{0.000000,0.000000,0.000000}%
\pgfsetstrokecolor{currentstroke}%
\pgfsetdash{}{0pt}%
\pgfpathmoveto{\pgfqpoint{1.384466in}{1.479712in}}%
\pgfpathlineto{\pgfqpoint{1.422549in}{1.511048in}}%
\pgfpathlineto{\pgfqpoint{1.384675in}{1.542342in}}%
\pgfpathlineto{\pgfqpoint{1.346482in}{1.517343in}}%
\pgfpathclose%
\pgfusepath{fill}%
\end{pgfscope}%
\begin{pgfscope}%
\pgfpathrectangle{\pgfqpoint{0.150000in}{0.150000in}}{\pgfqpoint{2.700000in}{1.950000in}}%
\pgfusepath{clip}%
\pgfsetbuttcap%
\pgfsetroundjoin%
\definecolor{currentfill}{rgb}{0.895527,0.810432,0.817172}%
\pgfsetfillcolor{currentfill}%
\pgfsetlinewidth{0.000000pt}%
\definecolor{currentstroke}{rgb}{0.000000,0.000000,0.000000}%
\pgfsetstrokecolor{currentstroke}%
\pgfsetdash{}{0pt}%
\pgfpathmoveto{\pgfqpoint{1.764125in}{0.840994in}}%
\pgfpathlineto{\pgfqpoint{1.801577in}{0.860903in}}%
\pgfpathlineto{\pgfqpoint{1.764733in}{0.946042in}}%
\pgfpathlineto{\pgfqpoint{1.726933in}{0.920229in}}%
\pgfpathclose%
\pgfusepath{fill}%
\end{pgfscope}%
\begin{pgfscope}%
\pgfpathrectangle{\pgfqpoint{0.150000in}{0.150000in}}{\pgfqpoint{2.700000in}{1.950000in}}%
\pgfusepath{clip}%
\pgfsetbuttcap%
\pgfsetroundjoin%
\definecolor{currentfill}{rgb}{0.878722,0.893658,0.914568}%
\pgfsetfillcolor{currentfill}%
\pgfsetlinewidth{0.000000pt}%
\definecolor{currentstroke}{rgb}{0.000000,0.000000,0.000000}%
\pgfsetstrokecolor{currentstroke}%
\pgfsetdash{}{0pt}%
\pgfpathmoveto{\pgfqpoint{1.346316in}{1.216067in}}%
\pgfpathlineto{\pgfqpoint{1.384542in}{1.241468in}}%
\pgfpathlineto{\pgfqpoint{1.346796in}{1.266806in}}%
\pgfpathlineto{\pgfqpoint{1.308570in}{1.241468in}}%
\pgfpathclose%
\pgfusepath{fill}%
\end{pgfscope}%
\begin{pgfscope}%
\pgfpathrectangle{\pgfqpoint{0.150000in}{0.150000in}}{\pgfqpoint{2.700000in}{1.950000in}}%
\pgfusepath{clip}%
\pgfsetbuttcap%
\pgfsetroundjoin%
\definecolor{currentfill}{rgb}{0.779213,0.806403,0.844470}%
\pgfsetfillcolor{currentfill}%
\pgfsetlinewidth{0.000000pt}%
\definecolor{currentstroke}{rgb}{0.000000,0.000000,0.000000}%
\pgfsetstrokecolor{currentstroke}%
\pgfsetdash{}{0pt}%
\pgfpathmoveto{\pgfqpoint{1.612530in}{1.329107in}}%
\pgfpathlineto{\pgfqpoint{1.650407in}{1.354357in}}%
\pgfpathlineto{\pgfqpoint{1.612338in}{1.379544in}}%
\pgfpathlineto{\pgfqpoint{1.574526in}{1.372988in}}%
\pgfpathclose%
\pgfusepath{fill}%
\end{pgfscope}%
\begin{pgfscope}%
\pgfpathrectangle{\pgfqpoint{0.150000in}{0.150000in}}{\pgfqpoint{2.700000in}{1.950000in}}%
\pgfusepath{clip}%
\pgfsetbuttcap%
\pgfsetroundjoin%
\definecolor{currentfill}{rgb}{0.838542,0.707031,0.717448}%
\pgfsetfillcolor{currentfill}%
\pgfsetlinewidth{0.000000pt}%
\definecolor{currentstroke}{rgb}{0.000000,0.000000,0.000000}%
\pgfsetstrokecolor{currentstroke}%
\pgfsetdash{}{0pt}%
\pgfpathmoveto{\pgfqpoint{1.726356in}{0.733876in}}%
\pgfpathlineto{\pgfqpoint{1.763912in}{0.754000in}}%
\pgfpathlineto{\pgfqpoint{1.726534in}{0.821011in}}%
\pgfpathlineto{\pgfqpoint{1.688805in}{0.800955in}}%
\pgfpathclose%
\pgfusepath{fill}%
\end{pgfscope}%
\begin{pgfscope}%
\pgfpathrectangle{\pgfqpoint{0.150000in}{0.150000in}}{\pgfqpoint{2.700000in}{1.950000in}}%
\pgfusepath{clip}%
\pgfsetbuttcap%
\pgfsetroundjoin%
\definecolor{currentfill}{rgb}{0.810309,0.833670,0.866376}%
\pgfsetfillcolor{currentfill}%
\pgfsetlinewidth{0.000000pt}%
\definecolor{currentstroke}{rgb}{0.000000,0.000000,0.000000}%
\pgfsetstrokecolor{currentstroke}%
\pgfsetdash{}{0pt}%
\pgfpathmoveto{\pgfqpoint{1.650498in}{1.285267in}}%
\pgfpathlineto{\pgfqpoint{1.688398in}{1.316736in}}%
\pgfpathlineto{\pgfqpoint{1.650407in}{1.354357in}}%
\pgfpathlineto{\pgfqpoint{1.612530in}{1.329107in}}%
\pgfpathclose%
\pgfusepath{fill}%
\end{pgfscope}%
\begin{pgfscope}%
\pgfpathrectangle{\pgfqpoint{0.150000in}{0.150000in}}{\pgfqpoint{2.700000in}{1.950000in}}%
\pgfusepath{clip}%
\pgfsetbuttcap%
\pgfsetroundjoin%
\definecolor{currentfill}{rgb}{0.673483,0.713695,0.769991}%
\pgfsetfillcolor{currentfill}%
\pgfsetlinewidth{0.000000pt}%
\definecolor{currentstroke}{rgb}{0.000000,0.000000,0.000000}%
\pgfsetstrokecolor{currentstroke}%
\pgfsetdash{}{0pt}%
\pgfpathmoveto{\pgfqpoint{1.422459in}{1.442073in}}%
\pgfpathlineto{\pgfqpoint{1.460476in}{1.479712in}}%
\pgfpathlineto{\pgfqpoint{1.422549in}{1.511048in}}%
\pgfpathlineto{\pgfqpoint{1.384466in}{1.479712in}}%
\pgfpathclose%
\pgfusepath{fill}%
\end{pgfscope}%
\begin{pgfscope}%
\pgfpathrectangle{\pgfqpoint{0.150000in}{0.150000in}}{\pgfqpoint{2.700000in}{1.950000in}}%
\pgfusepath{clip}%
\pgfsetbuttcap%
\pgfsetroundjoin%
\definecolor{currentfill}{rgb}{0.804090,0.828217,0.861994}%
\pgfsetfillcolor{currentfill}%
\pgfsetlinewidth{0.000000pt}%
\definecolor{currentstroke}{rgb}{0.000000,0.000000,0.000000}%
\pgfsetstrokecolor{currentstroke}%
\pgfsetdash{}{0pt}%
\pgfpathmoveto{\pgfqpoint{1.422475in}{1.285267in}}%
\pgfpathlineto{\pgfqpoint{1.460443in}{1.329107in}}%
\pgfpathlineto{\pgfqpoint{1.422500in}{1.360560in}}%
\pgfpathlineto{\pgfqpoint{1.384575in}{1.316736in}}%
\pgfpathclose%
\pgfusepath{fill}%
\end{pgfscope}%
\begin{pgfscope}%
\pgfpathrectangle{\pgfqpoint{0.150000in}{0.150000in}}{\pgfqpoint{2.700000in}{1.950000in}}%
\pgfusepath{clip}%
\pgfsetbuttcap%
\pgfsetroundjoin%
\definecolor{currentfill}{rgb}{0.804090,0.828217,0.861994}%
\pgfsetfillcolor{currentfill}%
\pgfsetlinewidth{0.000000pt}%
\definecolor{currentstroke}{rgb}{0.000000,0.000000,0.000000}%
\pgfsetstrokecolor{currentstroke}%
\pgfsetdash{}{0pt}%
\pgfpathmoveto{\pgfqpoint{1.117477in}{1.309982in}}%
\pgfpathlineto{\pgfqpoint{1.156928in}{1.310561in}}%
\pgfpathlineto{\pgfqpoint{1.119498in}{1.335791in}}%
\pgfpathlineto{\pgfqpoint{1.079963in}{1.335303in}}%
\pgfpathclose%
\pgfusepath{fill}%
\end{pgfscope}%
\begin{pgfscope}%
\pgfpathrectangle{\pgfqpoint{0.150000in}{0.150000in}}{\pgfqpoint{2.700000in}{1.950000in}}%
\pgfusepath{clip}%
\pgfsetbuttcap%
\pgfsetroundjoin%
\definecolor{currentfill}{rgb}{0.741896,0.773683,0.818183}%
\pgfsetfillcolor{currentfill}%
\pgfsetlinewidth{0.000000pt}%
\definecolor{currentstroke}{rgb}{0.000000,0.000000,0.000000}%
\pgfsetstrokecolor{currentstroke}%
\pgfsetdash{}{0pt}%
\pgfpathmoveto{\pgfqpoint{1.422500in}{1.360560in}}%
\pgfpathlineto{\pgfqpoint{1.460416in}{1.410664in}}%
\pgfpathlineto{\pgfqpoint{1.422459in}{1.442073in}}%
\pgfpathlineto{\pgfqpoint{1.384608in}{1.391971in}}%
\pgfpathclose%
\pgfusepath{fill}%
\end{pgfscope}%
\begin{pgfscope}%
\pgfpathrectangle{\pgfqpoint{0.150000in}{0.150000in}}{\pgfqpoint{2.700000in}{1.950000in}}%
\pgfusepath{clip}%
\pgfsetbuttcap%
\pgfsetroundjoin%
\definecolor{currentfill}{rgb}{0.861336,0.748392,0.757338}%
\pgfsetfillcolor{currentfill}%
\pgfsetlinewidth{0.000000pt}%
\definecolor{currentstroke}{rgb}{0.000000,0.000000,0.000000}%
\pgfsetstrokecolor{currentstroke}%
\pgfsetdash{}{0pt}%
\pgfpathmoveto{\pgfqpoint{1.917945in}{0.818675in}}%
\pgfpathlineto{\pgfqpoint{1.955317in}{0.838733in}}%
\pgfpathlineto{\pgfqpoint{1.915228in}{0.823331in}}%
\pgfpathlineto{\pgfqpoint{1.877981in}{0.803350in}}%
\pgfpathclose%
\pgfusepath{fill}%
\end{pgfscope}%
\begin{pgfscope}%
\pgfpathrectangle{\pgfqpoint{0.150000in}{0.150000in}}{\pgfqpoint{2.700000in}{1.950000in}}%
\pgfusepath{clip}%
\pgfsetbuttcap%
\pgfsetroundjoin%
\definecolor{currentfill}{rgb}{0.996890,0.997273,0.997809}%
\pgfsetfillcolor{currentfill}%
\pgfsetlinewidth{0.000000pt}%
\definecolor{currentstroke}{rgb}{0.000000,0.000000,0.000000}%
\pgfsetstrokecolor{currentstroke}%
\pgfsetdash{}{0pt}%
\pgfpathmoveto{\pgfqpoint{1.802995in}{1.071280in}}%
\pgfpathlineto{\pgfqpoint{1.840156in}{1.078745in}}%
\pgfpathlineto{\pgfqpoint{1.802169in}{1.116374in}}%
\pgfpathlineto{\pgfqpoint{1.764897in}{1.109023in}}%
\pgfpathclose%
\pgfusepath{fill}%
\end{pgfscope}%
\begin{pgfscope}%
\pgfpathrectangle{\pgfqpoint{0.150000in}{0.150000in}}{\pgfqpoint{2.700000in}{1.950000in}}%
\pgfusepath{clip}%
\pgfsetbuttcap%
\pgfsetroundjoin%
\definecolor{currentfill}{rgb}{0.698361,0.735509,0.787515}%
\pgfsetfillcolor{currentfill}%
\pgfsetlinewidth{0.000000pt}%
\definecolor{currentstroke}{rgb}{0.000000,0.000000,0.000000}%
\pgfsetstrokecolor{currentstroke}%
\pgfsetdash{}{0pt}%
\pgfpathmoveto{\pgfqpoint{0.851058in}{1.423163in}}%
\pgfpathlineto{\pgfqpoint{0.890703in}{1.435821in}}%
\pgfpathlineto{\pgfqpoint{0.853578in}{1.460914in}}%
\pgfpathlineto{\pgfqpoint{0.813476in}{1.454599in}}%
\pgfpathclose%
\pgfusepath{fill}%
\end{pgfscope}%
\begin{pgfscope}%
\pgfpathrectangle{\pgfqpoint{0.150000in}{0.150000in}}{\pgfqpoint{2.700000in}{1.950000in}}%
\pgfusepath{clip}%
\pgfsetbuttcap%
\pgfsetroundjoin%
\definecolor{currentfill}{rgb}{0.959574,0.964553,0.971523}%
\pgfsetfillcolor{currentfill}%
\pgfsetlinewidth{0.000000pt}%
\definecolor{currentstroke}{rgb}{0.000000,0.000000,0.000000}%
\pgfsetstrokecolor{currentstroke}%
\pgfsetdash{}{0pt}%
\pgfpathmoveto{\pgfqpoint{1.764897in}{1.109023in}}%
\pgfpathlineto{\pgfqpoint{1.802169in}{1.116374in}}%
\pgfpathlineto{\pgfqpoint{1.764321in}{1.160078in}}%
\pgfpathlineto{\pgfqpoint{1.726808in}{1.146757in}}%
\pgfpathclose%
\pgfusepath{fill}%
\end{pgfscope}%
\begin{pgfscope}%
\pgfpathrectangle{\pgfqpoint{0.150000in}{0.150000in}}{\pgfqpoint{2.700000in}{1.950000in}}%
\pgfusepath{clip}%
\pgfsetbuttcap%
\pgfsetroundjoin%
\definecolor{currentfill}{rgb}{0.860064,0.877298,0.901425}%
\pgfsetfillcolor{currentfill}%
\pgfsetlinewidth{0.000000pt}%
\definecolor{currentstroke}{rgb}{0.000000,0.000000,0.000000}%
\pgfsetstrokecolor{currentstroke}%
\pgfsetdash{}{0pt}%
\pgfpathmoveto{\pgfqpoint{1.269631in}{1.240641in}}%
\pgfpathlineto{\pgfqpoint{1.308570in}{1.241468in}}%
\pgfpathlineto{\pgfqpoint{1.271073in}{1.260666in}}%
\pgfpathlineto{\pgfqpoint{1.232071in}{1.259909in}}%
\pgfpathclose%
\pgfusepath{fill}%
\end{pgfscope}%
\begin{pgfscope}%
\pgfpathrectangle{\pgfqpoint{0.150000in}{0.150000in}}{\pgfqpoint{2.700000in}{1.950000in}}%
\pgfusepath{clip}%
\pgfsetbuttcap%
\pgfsetroundjoin%
\definecolor{currentfill}{rgb}{0.698361,0.735509,0.787515}%
\pgfsetfillcolor{currentfill}%
\pgfsetlinewidth{0.000000pt}%
\definecolor{currentstroke}{rgb}{0.000000,0.000000,0.000000}%
\pgfsetstrokecolor{currentstroke}%
\pgfsetdash{}{0pt}%
\pgfpathmoveto{\pgfqpoint{1.460416in}{1.410664in}}%
\pgfpathlineto{\pgfqpoint{1.498455in}{1.448332in}}%
\pgfpathlineto{\pgfqpoint{1.460476in}{1.479712in}}%
\pgfpathlineto{\pgfqpoint{1.422459in}{1.442073in}}%
\pgfpathclose%
\pgfusepath{fill}%
\end{pgfscope}%
\begin{pgfscope}%
\pgfpathrectangle{\pgfqpoint{0.150000in}{0.150000in}}{\pgfqpoint{2.700000in}{1.950000in}}%
\pgfusepath{clip}%
\pgfsetbuttcap%
\pgfsetroundjoin%
\definecolor{currentfill}{rgb}{0.928477,0.937286,0.949617}%
\pgfsetfillcolor{currentfill}%
\pgfsetlinewidth{0.000000pt}%
\definecolor{currentstroke}{rgb}{0.000000,0.000000,0.000000}%
\pgfsetstrokecolor{currentstroke}%
\pgfsetdash{}{0pt}%
\pgfpathmoveto{\pgfqpoint{1.726808in}{1.146757in}}%
\pgfpathlineto{\pgfqpoint{1.764321in}{1.160078in}}%
\pgfpathlineto{\pgfqpoint{1.726437in}{1.203822in}}%
\pgfpathlineto{\pgfqpoint{1.688727in}{1.184482in}}%
\pgfpathclose%
\pgfusepath{fill}%
\end{pgfscope}%
\begin{pgfscope}%
\pgfpathrectangle{\pgfqpoint{0.150000in}{0.150000in}}{\pgfqpoint{2.700000in}{1.950000in}}%
\pgfusepath{clip}%
\pgfsetbuttcap%
\pgfsetroundjoin%
\definecolor{currentfill}{rgb}{0.963909,0.934513,0.936841}%
\pgfsetfillcolor{currentfill}%
\pgfsetlinewidth{0.000000pt}%
\definecolor{currentstroke}{rgb}{0.000000,0.000000,0.000000}%
\pgfsetstrokecolor{currentstroke}%
\pgfsetdash{}{0pt}%
\pgfpathmoveto{\pgfqpoint{1.802284in}{0.965811in}}%
\pgfpathlineto{\pgfqpoint{1.839697in}{0.985508in}}%
\pgfpathlineto{\pgfqpoint{1.802995in}{1.071280in}}%
\pgfpathlineto{\pgfqpoint{1.765345in}{1.051653in}}%
\pgfpathclose%
\pgfusepath{fill}%
\end{pgfscope}%
\begin{pgfscope}%
\pgfpathrectangle{\pgfqpoint{0.150000in}{0.150000in}}{\pgfqpoint{2.700000in}{1.950000in}}%
\pgfusepath{clip}%
\pgfsetbuttcap%
\pgfsetroundjoin%
\definecolor{currentfill}{rgb}{0.760555,0.790043,0.831327}%
\pgfsetfillcolor{currentfill}%
\pgfsetlinewidth{0.000000pt}%
\definecolor{currentstroke}{rgb}{0.000000,0.000000,0.000000}%
\pgfsetstrokecolor{currentstroke}%
\pgfsetdash{}{0pt}%
\pgfpathmoveto{\pgfqpoint{1.460443in}{1.329107in}}%
\pgfpathlineto{\pgfqpoint{1.498447in}{1.372988in}}%
\pgfpathlineto{\pgfqpoint{1.460416in}{1.410664in}}%
\pgfpathlineto{\pgfqpoint{1.422500in}{1.360560in}}%
\pgfpathclose%
\pgfusepath{fill}%
\end{pgfscope}%
\begin{pgfscope}%
\pgfpathrectangle{\pgfqpoint{0.150000in}{0.150000in}}{\pgfqpoint{2.700000in}{1.950000in}}%
\pgfusepath{clip}%
\pgfsetbuttcap%
\pgfsetroundjoin%
\definecolor{currentfill}{rgb}{0.891161,0.904565,0.923330}%
\pgfsetfillcolor{currentfill}%
\pgfsetlinewidth{0.000000pt}%
\definecolor{currentstroke}{rgb}{0.000000,0.000000,0.000000}%
\pgfsetstrokecolor{currentstroke}%
\pgfsetdash{}{0pt}%
\pgfpathmoveto{\pgfqpoint{1.688727in}{1.184482in}}%
\pgfpathlineto{\pgfqpoint{1.726437in}{1.203822in}}%
\pgfpathlineto{\pgfqpoint{1.688431in}{1.241468in}}%
\pgfpathlineto{\pgfqpoint{1.650655in}{1.222200in}}%
\pgfpathclose%
\pgfusepath{fill}%
\end{pgfscope}%
\begin{pgfscope}%
\pgfpathrectangle{\pgfqpoint{0.150000in}{0.150000in}}{\pgfqpoint{2.700000in}{1.950000in}}%
\pgfusepath{clip}%
\pgfsetbuttcap%
\pgfsetroundjoin%
\definecolor{currentfill}{rgb}{0.860064,0.877298,0.901425}%
\pgfsetfillcolor{currentfill}%
\pgfsetlinewidth{0.000000pt}%
\definecolor{currentstroke}{rgb}{0.000000,0.000000,0.000000}%
\pgfsetstrokecolor{currentstroke}%
\pgfsetdash{}{0pt}%
\pgfpathmoveto{\pgfqpoint{1.650655in}{1.222200in}}%
\pgfpathlineto{\pgfqpoint{1.688431in}{1.241468in}}%
\pgfpathlineto{\pgfqpoint{1.650498in}{1.285267in}}%
\pgfpathlineto{\pgfqpoint{1.612590in}{1.259909in}}%
\pgfpathclose%
\pgfusepath{fill}%
\end{pgfscope}%
\begin{pgfscope}%
\pgfpathrectangle{\pgfqpoint{0.150000in}{0.150000in}}{\pgfqpoint{2.700000in}{1.950000in}}%
\pgfusepath{clip}%
\pgfsetbuttcap%
\pgfsetroundjoin%
\definecolor{currentfill}{rgb}{0.860064,0.877298,0.901425}%
\pgfsetfillcolor{currentfill}%
\pgfsetlinewidth{0.000000pt}%
\definecolor{currentstroke}{rgb}{0.000000,0.000000,0.000000}%
\pgfsetstrokecolor{currentstroke}%
\pgfsetdash{}{0pt}%
\pgfpathmoveto{\pgfqpoint{1.422318in}{1.222200in}}%
\pgfpathlineto{\pgfqpoint{1.460383in}{1.259909in}}%
\pgfpathlineto{\pgfqpoint{1.422475in}{1.285267in}}%
\pgfpathlineto{\pgfqpoint{1.384542in}{1.241468in}}%
\pgfpathclose%
\pgfusepath{fill}%
\end{pgfscope}%
\begin{pgfscope}%
\pgfpathrectangle{\pgfqpoint{0.150000in}{0.150000in}}{\pgfqpoint{2.700000in}{1.950000in}}%
\pgfusepath{clip}%
\pgfsetbuttcap%
\pgfsetroundjoin%
\definecolor{currentfill}{rgb}{0.760555,0.790043,0.831327}%
\pgfsetfillcolor{currentfill}%
\pgfsetlinewidth{0.000000pt}%
\definecolor{currentstroke}{rgb}{0.000000,0.000000,0.000000}%
\pgfsetstrokecolor{currentstroke}%
\pgfsetdash{}{0pt}%
\pgfpathmoveto{\pgfqpoint{1.002951in}{1.353935in}}%
\pgfpathlineto{\pgfqpoint{1.042544in}{1.360560in}}%
\pgfpathlineto{\pgfqpoint{1.005220in}{1.385754in}}%
\pgfpathlineto{\pgfqpoint{0.965564in}{1.379212in}}%
\pgfpathclose%
\pgfusepath{fill}%
\end{pgfscope}%
\begin{pgfscope}%
\pgfpathrectangle{\pgfqpoint{0.150000in}{0.150000in}}{\pgfqpoint{2.700000in}{1.950000in}}%
\pgfusepath{clip}%
\pgfsetbuttcap%
\pgfsetroundjoin%
\definecolor{currentfill}{rgb}{0.827145,0.686351,0.697503}%
\pgfsetfillcolor{currentfill}%
\pgfsetlinewidth{0.000000pt}%
\definecolor{currentstroke}{rgb}{0.000000,0.000000,0.000000}%
\pgfsetstrokecolor{currentstroke}%
\pgfsetdash{}{0pt}%
\pgfpathmoveto{\pgfqpoint{1.765412in}{0.742965in}}%
\pgfpathlineto{\pgfqpoint{1.803073in}{0.763168in}}%
\pgfpathlineto{\pgfqpoint{1.763912in}{0.754000in}}%
\pgfpathlineto{\pgfqpoint{1.726356in}{0.733876in}}%
\pgfpathclose%
\pgfusepath{fill}%
\end{pgfscope}%
\begin{pgfscope}%
\pgfpathrectangle{\pgfqpoint{0.150000in}{0.150000in}}{\pgfqpoint{2.700000in}{1.950000in}}%
\pgfusepath{clip}%
\pgfsetbuttcap%
\pgfsetroundjoin%
\definecolor{currentfill}{rgb}{0.822748,0.844577,0.875138}%
\pgfsetfillcolor{currentfill}%
\pgfsetlinewidth{0.000000pt}%
\definecolor{currentstroke}{rgb}{0.000000,0.000000,0.000000}%
\pgfsetstrokecolor{currentstroke}%
\pgfsetdash{}{0pt}%
\pgfpathmoveto{\pgfqpoint{1.612590in}{1.259909in}}%
\pgfpathlineto{\pgfqpoint{1.650498in}{1.285267in}}%
\pgfpathlineto{\pgfqpoint{1.612530in}{1.329107in}}%
\pgfpathlineto{\pgfqpoint{1.574534in}{1.297610in}}%
\pgfpathclose%
\pgfusepath{fill}%
\end{pgfscope}%
\begin{pgfscope}%
\pgfpathrectangle{\pgfqpoint{0.150000in}{0.150000in}}{\pgfqpoint{2.700000in}{1.950000in}}%
\pgfusepath{clip}%
\pgfsetbuttcap%
\pgfsetroundjoin%
\definecolor{currentfill}{rgb}{0.822748,0.844577,0.875138}%
\pgfsetfillcolor{currentfill}%
\pgfsetlinewidth{0.000000pt}%
\definecolor{currentstroke}{rgb}{0.000000,0.000000,0.000000}%
\pgfsetstrokecolor{currentstroke}%
\pgfsetdash{}{0pt}%
\pgfpathmoveto{\pgfqpoint{1.460383in}{1.259909in}}%
\pgfpathlineto{\pgfqpoint{1.498439in}{1.297610in}}%
\pgfpathlineto{\pgfqpoint{1.460443in}{1.329107in}}%
\pgfpathlineto{\pgfqpoint{1.422475in}{1.285267in}}%
\pgfpathclose%
\pgfusepath{fill}%
\end{pgfscope}%
\begin{pgfscope}%
\pgfpathrectangle{\pgfqpoint{0.150000in}{0.150000in}}{\pgfqpoint{2.700000in}{1.950000in}}%
\pgfusepath{clip}%
\pgfsetbuttcap%
\pgfsetroundjoin%
\definecolor{currentfill}{rgb}{0.723238,0.757322,0.805040}%
\pgfsetfillcolor{currentfill}%
\pgfsetlinewidth{0.000000pt}%
\definecolor{currentstroke}{rgb}{0.000000,0.000000,0.000000}%
\pgfsetstrokecolor{currentstroke}%
\pgfsetdash{}{0pt}%
\pgfpathmoveto{\pgfqpoint{1.498447in}{1.372988in}}%
\pgfpathlineto{\pgfqpoint{1.536486in}{1.410664in}}%
\pgfpathlineto{\pgfqpoint{1.498455in}{1.448332in}}%
\pgfpathlineto{\pgfqpoint{1.460416in}{1.410664in}}%
\pgfpathclose%
\pgfusepath{fill}%
\end{pgfscope}%
\begin{pgfscope}%
\pgfpathrectangle{\pgfqpoint{0.150000in}{0.150000in}}{\pgfqpoint{2.700000in}{1.950000in}}%
\pgfusepath{clip}%
\pgfsetbuttcap%
\pgfsetroundjoin%
\definecolor{currentfill}{rgb}{0.785432,0.811857,0.848851}%
\pgfsetfillcolor{currentfill}%
\pgfsetlinewidth{0.000000pt}%
\definecolor{currentstroke}{rgb}{0.000000,0.000000,0.000000}%
\pgfsetstrokecolor{currentstroke}%
\pgfsetdash{}{0pt}%
\pgfpathmoveto{\pgfqpoint{1.574534in}{1.297610in}}%
\pgfpathlineto{\pgfqpoint{1.612530in}{1.329107in}}%
\pgfpathlineto{\pgfqpoint{1.574526in}{1.372988in}}%
\pgfpathlineto{\pgfqpoint{1.536486in}{1.335303in}}%
\pgfpathclose%
\pgfusepath{fill}%
\end{pgfscope}%
\begin{pgfscope}%
\pgfpathrectangle{\pgfqpoint{0.150000in}{0.150000in}}{\pgfqpoint{2.700000in}{1.950000in}}%
\pgfusepath{clip}%
\pgfsetbuttcap%
\pgfsetroundjoin%
\definecolor{currentfill}{rgb}{0.785432,0.811857,0.848851}%
\pgfsetfillcolor{currentfill}%
\pgfsetlinewidth{0.000000pt}%
\definecolor{currentstroke}{rgb}{0.000000,0.000000,0.000000}%
\pgfsetstrokecolor{currentstroke}%
\pgfsetdash{}{0pt}%
\pgfpathmoveto{\pgfqpoint{1.498439in}{1.297610in}}%
\pgfpathlineto{\pgfqpoint{1.536486in}{1.335303in}}%
\pgfpathlineto{\pgfqpoint{1.498447in}{1.372988in}}%
\pgfpathlineto{\pgfqpoint{1.460443in}{1.329107in}}%
\pgfpathclose%
\pgfusepath{fill}%
\end{pgfscope}%
\begin{pgfscope}%
\pgfpathrectangle{\pgfqpoint{0.150000in}{0.150000in}}{\pgfqpoint{2.700000in}{1.950000in}}%
\pgfusepath{clip}%
\pgfsetbuttcap%
\pgfsetroundjoin%
\definecolor{currentfill}{rgb}{0.754335,0.784589,0.826945}%
\pgfsetfillcolor{currentfill}%
\pgfsetlinewidth{0.000000pt}%
\definecolor{currentstroke}{rgb}{0.000000,0.000000,0.000000}%
\pgfsetstrokecolor{currentstroke}%
\pgfsetdash{}{0pt}%
\pgfpathmoveto{\pgfqpoint{1.536486in}{1.335303in}}%
\pgfpathlineto{\pgfqpoint{1.574526in}{1.372988in}}%
\pgfpathlineto{\pgfqpoint{1.536486in}{1.410664in}}%
\pgfpathlineto{\pgfqpoint{1.498447in}{1.372988in}}%
\pgfpathclose%
\pgfusepath{fill}%
\end{pgfscope}%
\begin{pgfscope}%
\pgfpathrectangle{\pgfqpoint{0.150000in}{0.150000in}}{\pgfqpoint{2.700000in}{1.950000in}}%
\pgfusepath{clip}%
\pgfsetbuttcap%
\pgfsetroundjoin%
\definecolor{currentfill}{rgb}{0.884942,0.899112,0.918949}%
\pgfsetfillcolor{currentfill}%
\pgfsetlinewidth{0.000000pt}%
\definecolor{currentstroke}{rgb}{0.000000,0.000000,0.000000}%
\pgfsetstrokecolor{currentstroke}%
\pgfsetdash{}{0pt}%
\pgfpathmoveto{\pgfqpoint{1.384069in}{1.196727in}}%
\pgfpathlineto{\pgfqpoint{1.422318in}{1.222200in}}%
\pgfpathlineto{\pgfqpoint{1.384542in}{1.241468in}}%
\pgfpathlineto{\pgfqpoint{1.346316in}{1.216067in}}%
\pgfpathclose%
\pgfusepath{fill}%
\end{pgfscope}%
\begin{pgfscope}%
\pgfpathrectangle{\pgfqpoint{0.150000in}{0.150000in}}{\pgfqpoint{2.700000in}{1.950000in}}%
\pgfusepath{clip}%
\pgfsetbuttcap%
\pgfsetroundjoin%
\definecolor{currentfill}{rgb}{0.816529,0.839124,0.870757}%
\pgfsetfillcolor{currentfill}%
\pgfsetlinewidth{0.000000pt}%
\definecolor{currentstroke}{rgb}{0.000000,0.000000,0.000000}%
\pgfsetstrokecolor{currentstroke}%
\pgfsetdash{}{0pt}%
\pgfpathmoveto{\pgfqpoint{1.536486in}{1.259909in}}%
\pgfpathlineto{\pgfqpoint{1.574534in}{1.297610in}}%
\pgfpathlineto{\pgfqpoint{1.536486in}{1.335303in}}%
\pgfpathlineto{\pgfqpoint{1.498439in}{1.297610in}}%
\pgfpathclose%
\pgfusepath{fill}%
\end{pgfscope}%
\begin{pgfscope}%
\pgfpathrectangle{\pgfqpoint{0.150000in}{0.150000in}}{\pgfqpoint{2.700000in}{1.950000in}}%
\pgfusepath{clip}%
\pgfsetbuttcap%
\pgfsetroundjoin%
\definecolor{currentfill}{rgb}{0.816529,0.839124,0.870757}%
\pgfsetfillcolor{currentfill}%
\pgfsetlinewidth{0.000000pt}%
\definecolor{currentstroke}{rgb}{0.000000,0.000000,0.000000}%
\pgfsetstrokecolor{currentstroke}%
\pgfsetdash{}{0pt}%
\pgfpathmoveto{\pgfqpoint{1.155085in}{1.284596in}}%
\pgfpathlineto{\pgfqpoint{1.194452in}{1.285267in}}%
\pgfpathlineto{\pgfqpoint{1.156928in}{1.310561in}}%
\pgfpathlineto{\pgfqpoint{1.117477in}{1.309982in}}%
\pgfpathclose%
\pgfusepath{fill}%
\end{pgfscope}%
\begin{pgfscope}%
\pgfpathrectangle{\pgfqpoint{0.150000in}{0.150000in}}{\pgfqpoint{2.700000in}{1.950000in}}%
\pgfusepath{clip}%
\pgfsetbuttcap%
\pgfsetroundjoin%
\definecolor{currentfill}{rgb}{0.895527,0.810432,0.817172}%
\pgfsetfillcolor{currentfill}%
\pgfsetlinewidth{0.000000pt}%
\definecolor{currentstroke}{rgb}{0.000000,0.000000,0.000000}%
\pgfsetstrokecolor{currentstroke}%
\pgfsetdash{}{0pt}%
\pgfpathmoveto{\pgfqpoint{1.726534in}{0.821011in}}%
\pgfpathlineto{\pgfqpoint{1.764125in}{0.840994in}}%
\pgfpathlineto{\pgfqpoint{1.726933in}{0.920229in}}%
\pgfpathlineto{\pgfqpoint{1.689125in}{0.900314in}}%
\pgfpathclose%
\pgfusepath{fill}%
\end{pgfscope}%
\begin{pgfscope}%
\pgfpathrectangle{\pgfqpoint{0.150000in}{0.150000in}}{\pgfqpoint{2.700000in}{1.950000in}}%
\pgfusepath{clip}%
\pgfsetbuttcap%
\pgfsetroundjoin%
\definecolor{currentfill}{rgb}{0.717019,0.751869,0.800659}%
\pgfsetfillcolor{currentfill}%
\pgfsetlinewidth{0.000000pt}%
\definecolor{currentstroke}{rgb}{0.000000,0.000000,0.000000}%
\pgfsetstrokecolor{currentstroke}%
\pgfsetdash{}{0pt}%
\pgfpathmoveto{\pgfqpoint{0.888317in}{1.397929in}}%
\pgfpathlineto{\pgfqpoint{0.928273in}{1.404426in}}%
\pgfpathlineto{\pgfqpoint{0.890703in}{1.435821in}}%
\pgfpathlineto{\pgfqpoint{0.851058in}{1.423163in}}%
\pgfpathclose%
\pgfusepath{fill}%
\end{pgfscope}%
\begin{pgfscope}%
\pgfpathrectangle{\pgfqpoint{0.150000in}{0.150000in}}{\pgfqpoint{2.700000in}{1.950000in}}%
\pgfusepath{clip}%
\pgfsetbuttcap%
\pgfsetroundjoin%
\definecolor{currentfill}{rgb}{0.841406,0.860938,0.888281}%
\pgfsetfillcolor{currentfill}%
\pgfsetlinewidth{0.000000pt}%
\definecolor{currentstroke}{rgb}{0.000000,0.000000,0.000000}%
\pgfsetstrokecolor{currentstroke}%
\pgfsetdash{}{0pt}%
\pgfpathmoveto{\pgfqpoint{1.574565in}{1.228340in}}%
\pgfpathlineto{\pgfqpoint{1.612590in}{1.259909in}}%
\pgfpathlineto{\pgfqpoint{1.574534in}{1.297610in}}%
\pgfpathlineto{\pgfqpoint{1.536486in}{1.259909in}}%
\pgfpathclose%
\pgfusepath{fill}%
\end{pgfscope}%
\begin{pgfscope}%
\pgfpathrectangle{\pgfqpoint{0.150000in}{0.150000in}}{\pgfqpoint{2.700000in}{1.950000in}}%
\pgfusepath{clip}%
\pgfsetbuttcap%
\pgfsetroundjoin%
\definecolor{currentfill}{rgb}{0.841406,0.860938,0.888281}%
\pgfsetfillcolor{currentfill}%
\pgfsetlinewidth{0.000000pt}%
\definecolor{currentstroke}{rgb}{0.000000,0.000000,0.000000}%
\pgfsetstrokecolor{currentstroke}%
\pgfsetdash{}{0pt}%
\pgfpathmoveto{\pgfqpoint{1.498408in}{1.228340in}}%
\pgfpathlineto{\pgfqpoint{1.536486in}{1.259909in}}%
\pgfpathlineto{\pgfqpoint{1.498439in}{1.297610in}}%
\pgfpathlineto{\pgfqpoint{1.460383in}{1.259909in}}%
\pgfpathclose%
\pgfusepath{fill}%
\end{pgfscope}%
\begin{pgfscope}%
\pgfpathrectangle{\pgfqpoint{0.150000in}{0.150000in}}{\pgfqpoint{2.700000in}{1.950000in}}%
\pgfusepath{clip}%
\pgfsetbuttcap%
\pgfsetroundjoin%
\definecolor{currentfill}{rgb}{0.679703,0.719148,0.774372}%
\pgfsetfillcolor{currentfill}%
\pgfsetlinewidth{0.000000pt}%
\definecolor{currentstroke}{rgb}{0.000000,0.000000,0.000000}%
\pgfsetstrokecolor{currentstroke}%
\pgfsetdash{}{0pt}%
\pgfpathmoveto{\pgfqpoint{0.774017in}{1.435690in}}%
\pgfpathlineto{\pgfqpoint{0.813476in}{1.454599in}}%
\pgfpathlineto{\pgfqpoint{0.776385in}{1.479712in}}%
\pgfpathlineto{\pgfqpoint{0.736907in}{1.460873in}}%
\pgfpathclose%
\pgfusepath{fill}%
\end{pgfscope}%
\begin{pgfscope}%
\pgfpathrectangle{\pgfqpoint{0.150000in}{0.150000in}}{\pgfqpoint{2.700000in}{1.950000in}}%
\pgfusepath{clip}%
\pgfsetbuttcap%
\pgfsetroundjoin%
\definecolor{currentfill}{rgb}{0.838542,0.707031,0.717448}%
\pgfsetfillcolor{currentfill}%
\pgfsetlinewidth{0.000000pt}%
\definecolor{currentstroke}{rgb}{0.000000,0.000000,0.000000}%
\pgfsetstrokecolor{currentstroke}%
\pgfsetdash{}{0pt}%
\pgfpathmoveto{\pgfqpoint{1.688750in}{0.719522in}}%
\pgfpathlineto{\pgfqpoint{1.726356in}{0.733876in}}%
\pgfpathlineto{\pgfqpoint{1.688805in}{0.800955in}}%
\pgfpathlineto{\pgfqpoint{1.650937in}{0.780824in}}%
\pgfpathclose%
\pgfusepath{fill}%
\end{pgfscope}%
\begin{pgfscope}%
\pgfpathrectangle{\pgfqpoint{0.150000in}{0.150000in}}{\pgfqpoint{2.700000in}{1.950000in}}%
\pgfusepath{clip}%
\pgfsetbuttcap%
\pgfsetroundjoin%
\definecolor{currentfill}{rgb}{0.872503,0.888205,0.910187}%
\pgfsetfillcolor{currentfill}%
\pgfsetlinewidth{0.000000pt}%
\definecolor{currentstroke}{rgb}{0.000000,0.000000,0.000000}%
\pgfsetstrokecolor{currentstroke}%
\pgfsetdash{}{0pt}%
\pgfpathmoveto{\pgfqpoint{1.612695in}{1.196727in}}%
\pgfpathlineto{\pgfqpoint{1.650655in}{1.222200in}}%
\pgfpathlineto{\pgfqpoint{1.612590in}{1.259909in}}%
\pgfpathlineto{\pgfqpoint{1.574565in}{1.228340in}}%
\pgfpathclose%
\pgfusepath{fill}%
\end{pgfscope}%
\begin{pgfscope}%
\pgfpathrectangle{\pgfqpoint{0.150000in}{0.150000in}}{\pgfqpoint{2.700000in}{1.950000in}}%
\pgfusepath{clip}%
\pgfsetbuttcap%
\pgfsetroundjoin%
\definecolor{currentfill}{rgb}{0.872503,0.888205,0.910187}%
\pgfsetfillcolor{currentfill}%
\pgfsetlinewidth{0.000000pt}%
\definecolor{currentstroke}{rgb}{0.000000,0.000000,0.000000}%
\pgfsetstrokecolor{currentstroke}%
\pgfsetdash{}{0pt}%
\pgfpathmoveto{\pgfqpoint{1.460278in}{1.196727in}}%
\pgfpathlineto{\pgfqpoint{1.498408in}{1.228340in}}%
\pgfpathlineto{\pgfqpoint{1.460383in}{1.259909in}}%
\pgfpathlineto{\pgfqpoint{1.422318in}{1.222200in}}%
\pgfpathclose%
\pgfusepath{fill}%
\end{pgfscope}%
\begin{pgfscope}%
\pgfpathrectangle{\pgfqpoint{0.150000in}{0.150000in}}{\pgfqpoint{2.700000in}{1.950000in}}%
\pgfusepath{clip}%
\pgfsetbuttcap%
\pgfsetroundjoin%
\definecolor{currentfill}{rgb}{0.861336,0.748392,0.757338}%
\pgfsetfillcolor{currentfill}%
\pgfsetlinewidth{0.000000pt}%
\definecolor{currentstroke}{rgb}{0.000000,0.000000,0.000000}%
\pgfsetstrokecolor{currentstroke}%
\pgfsetdash{}{0pt}%
\pgfpathmoveto{\pgfqpoint{1.880236in}{0.792629in}}%
\pgfpathlineto{\pgfqpoint{1.917945in}{0.818675in}}%
\pgfpathlineto{\pgfqpoint{1.877981in}{0.803350in}}%
\pgfpathlineto{\pgfqpoint{1.840596in}{0.783296in}}%
\pgfpathclose%
\pgfusepath{fill}%
\end{pgfscope}%
\begin{pgfscope}%
\pgfpathrectangle{\pgfqpoint{0.150000in}{0.150000in}}{\pgfqpoint{2.700000in}{1.950000in}}%
\pgfusepath{clip}%
\pgfsetbuttcap%
\pgfsetroundjoin%
\definecolor{currentfill}{rgb}{0.772993,0.800950,0.840089}%
\pgfsetfillcolor{currentfill}%
\pgfsetlinewidth{0.000000pt}%
\definecolor{currentstroke}{rgb}{0.000000,0.000000,0.000000}%
\pgfsetstrokecolor{currentstroke}%
\pgfsetdash{}{0pt}%
\pgfpathmoveto{\pgfqpoint{1.040432in}{1.328593in}}%
\pgfpathlineto{\pgfqpoint{1.079963in}{1.335303in}}%
\pgfpathlineto{\pgfqpoint{1.042544in}{1.360560in}}%
\pgfpathlineto{\pgfqpoint{1.002951in}{1.353935in}}%
\pgfpathclose%
\pgfusepath{fill}%
\end{pgfscope}%
\begin{pgfscope}%
\pgfpathrectangle{\pgfqpoint{0.150000in}{0.150000in}}{\pgfqpoint{2.700000in}{1.950000in}}%
\pgfusepath{clip}%
\pgfsetbuttcap%
\pgfsetroundjoin%
\definecolor{currentfill}{rgb}{0.903125,0.824219,0.830469}%
\pgfsetfillcolor{currentfill}%
\pgfsetlinewidth{0.000000pt}%
\definecolor{currentstroke}{rgb}{0.000000,0.000000,0.000000}%
\pgfsetstrokecolor{currentstroke}%
\pgfsetdash{}{0pt}%
\pgfpathmoveto{\pgfqpoint{2.033481in}{0.880326in}}%
\pgfpathlineto{\pgfqpoint{2.069794in}{0.882446in}}%
\pgfpathlineto{\pgfqpoint{2.029933in}{0.884551in}}%
\pgfpathlineto{\pgfqpoint{1.992815in}{0.864640in}}%
\pgfpathclose%
\pgfusepath{fill}%
\end{pgfscope}%
\begin{pgfscope}%
\pgfpathrectangle{\pgfqpoint{0.150000in}{0.150000in}}{\pgfqpoint{2.700000in}{1.950000in}}%
\pgfusepath{clip}%
\pgfsetbuttcap%
\pgfsetroundjoin%
\definecolor{currentfill}{rgb}{0.866284,0.882751,0.905806}%
\pgfsetfillcolor{currentfill}%
\pgfsetlinewidth{0.000000pt}%
\definecolor{currentstroke}{rgb}{0.000000,0.000000,0.000000}%
\pgfsetstrokecolor{currentstroke}%
\pgfsetdash{}{0pt}%
\pgfpathmoveto{\pgfqpoint{1.307463in}{1.215147in}}%
\pgfpathlineto{\pgfqpoint{1.346316in}{1.216067in}}%
\pgfpathlineto{\pgfqpoint{1.308570in}{1.241468in}}%
\pgfpathlineto{\pgfqpoint{1.269631in}{1.240641in}}%
\pgfpathclose%
\pgfusepath{fill}%
\end{pgfscope}%
\begin{pgfscope}%
\pgfpathrectangle{\pgfqpoint{0.150000in}{0.150000in}}{\pgfqpoint{2.700000in}{1.950000in}}%
\pgfusepath{clip}%
\pgfsetbuttcap%
\pgfsetroundjoin%
\definecolor{currentfill}{rgb}{0.866284,0.882751,0.905806}%
\pgfsetfillcolor{currentfill}%
\pgfsetlinewidth{0.000000pt}%
\definecolor{currentstroke}{rgb}{0.000000,0.000000,0.000000}%
\pgfsetstrokecolor{currentstroke}%
\pgfsetdash{}{0pt}%
\pgfpathmoveto{\pgfqpoint{1.536486in}{1.202860in}}%
\pgfpathlineto{\pgfqpoint{1.574565in}{1.228340in}}%
\pgfpathlineto{\pgfqpoint{1.536486in}{1.259909in}}%
\pgfpathlineto{\pgfqpoint{1.498408in}{1.228340in}}%
\pgfpathclose%
\pgfusepath{fill}%
\end{pgfscope}%
\begin{pgfscope}%
\pgfpathrectangle{\pgfqpoint{0.150000in}{0.150000in}}{\pgfqpoint{2.700000in}{1.950000in}}%
\pgfusepath{clip}%
\pgfsetbuttcap%
\pgfsetroundjoin%
\definecolor{currentfill}{rgb}{0.963909,0.934513,0.936841}%
\pgfsetfillcolor{currentfill}%
\pgfsetlinewidth{0.000000pt}%
\definecolor{currentstroke}{rgb}{0.000000,0.000000,0.000000}%
\pgfsetstrokecolor{currentstroke}%
\pgfsetdash{}{0pt}%
\pgfpathmoveto{\pgfqpoint{1.764733in}{0.946042in}}%
\pgfpathlineto{\pgfqpoint{1.802284in}{0.965811in}}%
\pgfpathlineto{\pgfqpoint{1.765345in}{1.051653in}}%
\pgfpathlineto{\pgfqpoint{1.727444in}{1.025906in}}%
\pgfpathclose%
\pgfusepath{fill}%
\end{pgfscope}%
\begin{pgfscope}%
\pgfpathrectangle{\pgfqpoint{0.150000in}{0.150000in}}{\pgfqpoint{2.700000in}{1.950000in}}%
\pgfusepath{clip}%
\pgfsetbuttcap%
\pgfsetroundjoin%
\definecolor{currentfill}{rgb}{0.897381,0.910018,0.927711}%
\pgfsetfillcolor{currentfill}%
\pgfsetlinewidth{0.000000pt}%
\definecolor{currentstroke}{rgb}{0.000000,0.000000,0.000000}%
\pgfsetstrokecolor{currentstroke}%
\pgfsetdash{}{0pt}%
\pgfpathmoveto{\pgfqpoint{1.650878in}{1.165071in}}%
\pgfpathlineto{\pgfqpoint{1.688727in}{1.184482in}}%
\pgfpathlineto{\pgfqpoint{1.650655in}{1.222200in}}%
\pgfpathlineto{\pgfqpoint{1.612695in}{1.196727in}}%
\pgfpathclose%
\pgfusepath{fill}%
\end{pgfscope}%
\begin{pgfscope}%
\pgfpathrectangle{\pgfqpoint{0.150000in}{0.150000in}}{\pgfqpoint{2.700000in}{1.950000in}}%
\pgfusepath{clip}%
\pgfsetbuttcap%
\pgfsetroundjoin%
\definecolor{currentfill}{rgb}{0.729458,0.762776,0.809421}%
\pgfsetfillcolor{currentfill}%
\pgfsetlinewidth{0.000000pt}%
\definecolor{currentstroke}{rgb}{0.000000,0.000000,0.000000}%
\pgfsetstrokecolor{currentstroke}%
\pgfsetdash{}{0pt}%
\pgfpathmoveto{\pgfqpoint{0.925670in}{1.372631in}}%
\pgfpathlineto{\pgfqpoint{0.965564in}{1.379212in}}%
\pgfpathlineto{\pgfqpoint{0.928273in}{1.404426in}}%
\pgfpathlineto{\pgfqpoint{0.888317in}{1.397929in}}%
\pgfpathclose%
\pgfusepath{fill}%
\end{pgfscope}%
\begin{pgfscope}%
\pgfpathrectangle{\pgfqpoint{0.150000in}{0.150000in}}{\pgfqpoint{2.700000in}{1.950000in}}%
\pgfusepath{clip}%
\pgfsetbuttcap%
\pgfsetroundjoin%
\definecolor{currentfill}{rgb}{0.828968,0.850031,0.879519}%
\pgfsetfillcolor{currentfill}%
\pgfsetlinewidth{0.000000pt}%
\definecolor{currentstroke}{rgb}{0.000000,0.000000,0.000000}%
\pgfsetstrokecolor{currentstroke}%
\pgfsetdash{}{0pt}%
\pgfpathmoveto{\pgfqpoint{1.192789in}{1.259147in}}%
\pgfpathlineto{\pgfqpoint{1.232071in}{1.259909in}}%
\pgfpathlineto{\pgfqpoint{1.194452in}{1.285267in}}%
\pgfpathlineto{\pgfqpoint{1.155085in}{1.284596in}}%
\pgfpathclose%
\pgfusepath{fill}%
\end{pgfscope}%
\begin{pgfscope}%
\pgfpathrectangle{\pgfqpoint{0.150000in}{0.150000in}}{\pgfqpoint{2.700000in}{1.950000in}}%
\pgfusepath{clip}%
\pgfsetbuttcap%
\pgfsetroundjoin%
\definecolor{currentfill}{rgb}{0.922258,0.931832,0.945236}%
\pgfsetfillcolor{currentfill}%
\pgfsetlinewidth{0.000000pt}%
\definecolor{currentstroke}{rgb}{0.000000,0.000000,0.000000}%
\pgfsetstrokecolor{currentstroke}%
\pgfsetdash{}{0pt}%
\pgfpathmoveto{\pgfqpoint{1.689114in}{1.133371in}}%
\pgfpathlineto{\pgfqpoint{1.726808in}{1.146757in}}%
\pgfpathlineto{\pgfqpoint{1.688727in}{1.184482in}}%
\pgfpathlineto{\pgfqpoint{1.650878in}{1.165071in}}%
\pgfpathclose%
\pgfusepath{fill}%
\end{pgfscope}%
\begin{pgfscope}%
\pgfpathrectangle{\pgfqpoint{0.150000in}{0.150000in}}{\pgfqpoint{2.700000in}{1.950000in}}%
\pgfusepath{clip}%
\pgfsetbuttcap%
\pgfsetroundjoin%
\definecolor{currentfill}{rgb}{0.884942,0.899112,0.918949}%
\pgfsetfillcolor{currentfill}%
\pgfsetlinewidth{0.000000pt}%
\definecolor{currentstroke}{rgb}{0.000000,0.000000,0.000000}%
\pgfsetstrokecolor{currentstroke}%
\pgfsetdash{}{0pt}%
\pgfpathmoveto{\pgfqpoint{1.422029in}{1.171190in}}%
\pgfpathlineto{\pgfqpoint{1.460278in}{1.196727in}}%
\pgfpathlineto{\pgfqpoint{1.422318in}{1.222200in}}%
\pgfpathlineto{\pgfqpoint{1.384069in}{1.196727in}}%
\pgfpathclose%
\pgfusepath{fill}%
\end{pgfscope}%
\begin{pgfscope}%
\pgfpathrectangle{\pgfqpoint{0.150000in}{0.150000in}}{\pgfqpoint{2.700000in}{1.950000in}}%
\pgfusepath{clip}%
\pgfsetbuttcap%
\pgfsetroundjoin%
\definecolor{currentfill}{rgb}{0.830944,0.693244,0.704151}%
\pgfsetfillcolor{currentfill}%
\pgfsetlinewidth{0.000000pt}%
\definecolor{currentstroke}{rgb}{0.000000,0.000000,0.000000}%
\pgfsetstrokecolor{currentstroke}%
\pgfsetdash{}{0pt}%
\pgfpathmoveto{\pgfqpoint{1.727611in}{0.722688in}}%
\pgfpathlineto{\pgfqpoint{1.765412in}{0.742965in}}%
\pgfpathlineto{\pgfqpoint{1.726356in}{0.733876in}}%
\pgfpathlineto{\pgfqpoint{1.688750in}{0.719522in}}%
\pgfpathclose%
\pgfusepath{fill}%
\end{pgfscope}%
\begin{pgfscope}%
\pgfpathrectangle{\pgfqpoint{0.150000in}{0.150000in}}{\pgfqpoint{2.700000in}{1.950000in}}%
\pgfusepath{clip}%
\pgfsetbuttcap%
\pgfsetroundjoin%
\definecolor{currentfill}{rgb}{0.692142,0.730055,0.783134}%
\pgfsetfillcolor{currentfill}%
\pgfsetlinewidth{0.000000pt}%
\definecolor{currentstroke}{rgb}{0.000000,0.000000,0.000000}%
\pgfsetstrokecolor{currentstroke}%
\pgfsetdash{}{0pt}%
\pgfpathmoveto{\pgfqpoint{0.811221in}{1.410443in}}%
\pgfpathlineto{\pgfqpoint{0.851058in}{1.423163in}}%
\pgfpathlineto{\pgfqpoint{0.813476in}{1.454599in}}%
\pgfpathlineto{\pgfqpoint{0.774017in}{1.435690in}}%
\pgfpathclose%
\pgfusepath{fill}%
\end{pgfscope}%
\begin{pgfscope}%
\pgfpathrectangle{\pgfqpoint{0.150000in}{0.150000in}}{\pgfqpoint{2.700000in}{1.950000in}}%
\pgfusepath{clip}%
\pgfsetbuttcap%
\pgfsetroundjoin%
\definecolor{currentfill}{rgb}{0.884942,0.899112,0.918949}%
\pgfsetfillcolor{currentfill}%
\pgfsetlinewidth{0.000000pt}%
\definecolor{currentstroke}{rgb}{0.000000,0.000000,0.000000}%
\pgfsetstrokecolor{currentstroke}%
\pgfsetdash{}{0pt}%
\pgfpathmoveto{\pgfqpoint{1.574639in}{1.171190in}}%
\pgfpathlineto{\pgfqpoint{1.612695in}{1.196727in}}%
\pgfpathlineto{\pgfqpoint{1.574565in}{1.228340in}}%
\pgfpathlineto{\pgfqpoint{1.536486in}{1.202860in}}%
\pgfpathclose%
\pgfusepath{fill}%
\end{pgfscope}%
\begin{pgfscope}%
\pgfpathrectangle{\pgfqpoint{0.150000in}{0.150000in}}{\pgfqpoint{2.700000in}{1.950000in}}%
\pgfusepath{clip}%
\pgfsetbuttcap%
\pgfsetroundjoin%
\definecolor{currentfill}{rgb}{0.884942,0.899112,0.918949}%
\pgfsetfillcolor{currentfill}%
\pgfsetlinewidth{0.000000pt}%
\definecolor{currentstroke}{rgb}{0.000000,0.000000,0.000000}%
\pgfsetstrokecolor{currentstroke}%
\pgfsetdash{}{0pt}%
\pgfpathmoveto{\pgfqpoint{1.498334in}{1.171190in}}%
\pgfpathlineto{\pgfqpoint{1.536486in}{1.202860in}}%
\pgfpathlineto{\pgfqpoint{1.498408in}{1.228340in}}%
\pgfpathlineto{\pgfqpoint{1.460278in}{1.196727in}}%
\pgfpathclose%
\pgfusepath{fill}%
\end{pgfscope}%
\begin{pgfscope}%
\pgfpathrectangle{\pgfqpoint{0.150000in}{0.150000in}}{\pgfqpoint{2.700000in}{1.950000in}}%
\pgfusepath{clip}%
\pgfsetbuttcap%
\pgfsetroundjoin%
\definecolor{currentfill}{rgb}{0.978232,0.980913,0.984666}%
\pgfsetfillcolor{currentfill}%
\pgfsetlinewidth{0.000000pt}%
\definecolor{currentstroke}{rgb}{0.000000,0.000000,0.000000}%
\pgfsetstrokecolor{currentstroke}%
\pgfsetdash{}{0pt}%
\pgfpathmoveto{\pgfqpoint{1.765345in}{1.051653in}}%
\pgfpathlineto{\pgfqpoint{1.802995in}{1.071280in}}%
\pgfpathlineto{\pgfqpoint{1.764897in}{1.109023in}}%
\pgfpathlineto{\pgfqpoint{1.727402in}{1.101627in}}%
\pgfpathclose%
\pgfusepath{fill}%
\end{pgfscope}%
\begin{pgfscope}%
\pgfpathrectangle{\pgfqpoint{0.150000in}{0.150000in}}{\pgfqpoint{2.700000in}{1.950000in}}%
\pgfusepath{clip}%
\pgfsetbuttcap%
\pgfsetroundjoin%
\definecolor{currentfill}{rgb}{0.947135,0.953646,0.962760}%
\pgfsetfillcolor{currentfill}%
\pgfsetlinewidth{0.000000pt}%
\definecolor{currentstroke}{rgb}{0.000000,0.000000,0.000000}%
\pgfsetstrokecolor{currentstroke}%
\pgfsetdash{}{0pt}%
\pgfpathmoveto{\pgfqpoint{1.727402in}{1.101627in}}%
\pgfpathlineto{\pgfqpoint{1.764897in}{1.109023in}}%
\pgfpathlineto{\pgfqpoint{1.726808in}{1.146757in}}%
\pgfpathlineto{\pgfqpoint{1.689114in}{1.133371in}}%
\pgfpathclose%
\pgfusepath{fill}%
\end{pgfscope}%
\begin{pgfscope}%
\pgfpathrectangle{\pgfqpoint{0.150000in}{0.150000in}}{\pgfqpoint{2.700000in}{1.950000in}}%
\pgfusepath{clip}%
\pgfsetbuttcap%
\pgfsetroundjoin%
\definecolor{currentfill}{rgb}{0.895527,0.810432,0.817172}%
\pgfsetfillcolor{currentfill}%
\pgfsetlinewidth{0.000000pt}%
\definecolor{currentstroke}{rgb}{0.000000,0.000000,0.000000}%
\pgfsetstrokecolor{currentstroke}%
\pgfsetdash{}{0pt}%
\pgfpathmoveto{\pgfqpoint{1.688805in}{0.800955in}}%
\pgfpathlineto{\pgfqpoint{1.726534in}{0.821011in}}%
\pgfpathlineto{\pgfqpoint{1.689125in}{0.900314in}}%
\pgfpathlineto{\pgfqpoint{1.651178in}{0.880326in}}%
\pgfpathclose%
\pgfusepath{fill}%
\end{pgfscope}%
\begin{pgfscope}%
\pgfpathrectangle{\pgfqpoint{0.150000in}{0.150000in}}{\pgfqpoint{2.700000in}{1.950000in}}%
\pgfusepath{clip}%
\pgfsetbuttcap%
\pgfsetroundjoin%
\definecolor{currentfill}{rgb}{0.779213,0.806403,0.844470}%
\pgfsetfillcolor{currentfill}%
\pgfsetlinewidth{0.000000pt}%
\definecolor{currentstroke}{rgb}{0.000000,0.000000,0.000000}%
\pgfsetstrokecolor{currentstroke}%
\pgfsetdash{}{0pt}%
\pgfpathmoveto{\pgfqpoint{1.078008in}{1.303187in}}%
\pgfpathlineto{\pgfqpoint{1.117477in}{1.309982in}}%
\pgfpathlineto{\pgfqpoint{1.079963in}{1.335303in}}%
\pgfpathlineto{\pgfqpoint{1.040432in}{1.328593in}}%
\pgfpathclose%
\pgfusepath{fill}%
\end{pgfscope}%
\begin{pgfscope}%
\pgfpathrectangle{\pgfqpoint{0.150000in}{0.150000in}}{\pgfqpoint{2.700000in}{1.950000in}}%
\pgfusepath{clip}%
\pgfsetbuttcap%
\pgfsetroundjoin%
\definecolor{currentfill}{rgb}{0.872503,0.888205,0.910187}%
\pgfsetfillcolor{currentfill}%
\pgfsetlinewidth{0.000000pt}%
\definecolor{currentstroke}{rgb}{0.000000,0.000000,0.000000}%
\pgfsetstrokecolor{currentstroke}%
\pgfsetdash{}{0pt}%
\pgfpathmoveto{\pgfqpoint{1.345391in}{1.189589in}}%
\pgfpathlineto{\pgfqpoint{1.384069in}{1.196727in}}%
\pgfpathlineto{\pgfqpoint{1.346316in}{1.216067in}}%
\pgfpathlineto{\pgfqpoint{1.307463in}{1.215147in}}%
\pgfpathclose%
\pgfusepath{fill}%
\end{pgfscope}%
\begin{pgfscope}%
\pgfpathrectangle{\pgfqpoint{0.150000in}{0.150000in}}{\pgfqpoint{2.700000in}{1.950000in}}%
\pgfusepath{clip}%
\pgfsetbuttcap%
\pgfsetroundjoin%
\definecolor{currentfill}{rgb}{0.679703,0.719148,0.774372}%
\pgfsetfillcolor{currentfill}%
\pgfsetlinewidth{0.000000pt}%
\definecolor{currentstroke}{rgb}{0.000000,0.000000,0.000000}%
\pgfsetstrokecolor{currentstroke}%
\pgfsetdash{}{0pt}%
\pgfpathmoveto{\pgfqpoint{0.734412in}{1.416711in}}%
\pgfpathlineto{\pgfqpoint{0.774017in}{1.435690in}}%
\pgfpathlineto{\pgfqpoint{0.736907in}{1.460873in}}%
\pgfpathlineto{\pgfqpoint{0.697284in}{1.441965in}}%
\pgfpathclose%
\pgfusepath{fill}%
\end{pgfscope}%
\begin{pgfscope}%
\pgfpathrectangle{\pgfqpoint{0.150000in}{0.150000in}}{\pgfqpoint{2.700000in}{1.950000in}}%
\pgfusepath{clip}%
\pgfsetbuttcap%
\pgfsetroundjoin%
\definecolor{currentfill}{rgb}{0.861336,0.748392,0.757338}%
\pgfsetfillcolor{currentfill}%
\pgfsetlinewidth{0.000000pt}%
\definecolor{currentstroke}{rgb}{0.000000,0.000000,0.000000}%
\pgfsetstrokecolor{currentstroke}%
\pgfsetdash{}{0pt}%
\pgfpathmoveto{\pgfqpoint{1.842608in}{0.772423in}}%
\pgfpathlineto{\pgfqpoint{1.880236in}{0.792629in}}%
\pgfpathlineto{\pgfqpoint{1.840596in}{0.783296in}}%
\pgfpathlineto{\pgfqpoint{1.803073in}{0.763168in}}%
\pgfpathclose%
\pgfusepath{fill}%
\end{pgfscope}%
\begin{pgfscope}%
\pgfpathrectangle{\pgfqpoint{0.150000in}{0.150000in}}{\pgfqpoint{2.700000in}{1.950000in}}%
\pgfusepath{clip}%
\pgfsetbuttcap%
\pgfsetroundjoin%
\definecolor{currentfill}{rgb}{0.741896,0.773683,0.818183}%
\pgfsetfillcolor{currentfill}%
\pgfsetlinewidth{0.000000pt}%
\definecolor{currentstroke}{rgb}{0.000000,0.000000,0.000000}%
\pgfsetstrokecolor{currentstroke}%
\pgfsetdash{}{0pt}%
\pgfpathmoveto{\pgfqpoint{0.963118in}{1.347269in}}%
\pgfpathlineto{\pgfqpoint{1.002951in}{1.353935in}}%
\pgfpathlineto{\pgfqpoint{0.965564in}{1.379212in}}%
\pgfpathlineto{\pgfqpoint{0.925670in}{1.372631in}}%
\pgfpathclose%
\pgfusepath{fill}%
\end{pgfscope}%
\begin{pgfscope}%
\pgfpathrectangle{\pgfqpoint{0.150000in}{0.150000in}}{\pgfqpoint{2.700000in}{1.950000in}}%
\pgfusepath{clip}%
\pgfsetbuttcap%
\pgfsetroundjoin%
\definecolor{currentfill}{rgb}{0.704580,0.740962,0.791896}%
\pgfsetfillcolor{currentfill}%
\pgfsetlinewidth{0.000000pt}%
\definecolor{currentstroke}{rgb}{0.000000,0.000000,0.000000}%
\pgfsetstrokecolor{currentstroke}%
\pgfsetdash{}{0pt}%
\pgfpathmoveto{\pgfqpoint{0.848919in}{1.378878in}}%
\pgfpathlineto{\pgfqpoint{0.888317in}{1.397929in}}%
\pgfpathlineto{\pgfqpoint{0.851058in}{1.423163in}}%
\pgfpathlineto{\pgfqpoint{0.811221in}{1.410443in}}%
\pgfpathclose%
\pgfusepath{fill}%
\end{pgfscope}%
\begin{pgfscope}%
\pgfpathrectangle{\pgfqpoint{0.150000in}{0.150000in}}{\pgfqpoint{2.700000in}{1.950000in}}%
\pgfusepath{clip}%
\pgfsetbuttcap%
\pgfsetroundjoin%
\definecolor{currentfill}{rgb}{0.960110,0.927619,0.930193}%
\pgfsetfillcolor{currentfill}%
\pgfsetlinewidth{0.000000pt}%
\definecolor{currentstroke}{rgb}{0.000000,0.000000,0.000000}%
\pgfsetstrokecolor{currentstroke}%
\pgfsetdash{}{0pt}%
\pgfpathmoveto{\pgfqpoint{1.726933in}{0.920229in}}%
\pgfpathlineto{\pgfqpoint{1.764733in}{0.946042in}}%
\pgfpathlineto{\pgfqpoint{1.727444in}{1.025906in}}%
\pgfpathlineto{\pgfqpoint{1.689535in}{1.006133in}}%
\pgfpathclose%
\pgfusepath{fill}%
\end{pgfscope}%
\begin{pgfscope}%
\pgfpathrectangle{\pgfqpoint{0.150000in}{0.150000in}}{\pgfqpoint{2.700000in}{1.950000in}}%
\pgfusepath{clip}%
\pgfsetbuttcap%
\pgfsetroundjoin%
\definecolor{currentfill}{rgb}{0.897381,0.910018,0.927711}%
\pgfsetfillcolor{currentfill}%
\pgfsetlinewidth{0.000000pt}%
\definecolor{currentstroke}{rgb}{0.000000,0.000000,0.000000}%
\pgfsetstrokecolor{currentstroke}%
\pgfsetdash{}{0pt}%
\pgfpathmoveto{\pgfqpoint{1.612889in}{1.145587in}}%
\pgfpathlineto{\pgfqpoint{1.650878in}{1.165071in}}%
\pgfpathlineto{\pgfqpoint{1.612695in}{1.196727in}}%
\pgfpathlineto{\pgfqpoint{1.574639in}{1.171190in}}%
\pgfpathclose%
\pgfusepath{fill}%
\end{pgfscope}%
\begin{pgfscope}%
\pgfpathrectangle{\pgfqpoint{0.150000in}{0.150000in}}{\pgfqpoint{2.700000in}{1.950000in}}%
\pgfusepath{clip}%
\pgfsetbuttcap%
\pgfsetroundjoin%
\definecolor{currentfill}{rgb}{0.842341,0.713925,0.724096}%
\pgfsetfillcolor{currentfill}%
\pgfsetlinewidth{0.000000pt}%
\definecolor{currentstroke}{rgb}{0.000000,0.000000,0.000000}%
\pgfsetstrokecolor{currentstroke}%
\pgfsetdash{}{0pt}%
\pgfpathmoveto{\pgfqpoint{1.650895in}{0.699249in}}%
\pgfpathlineto{\pgfqpoint{1.688750in}{0.719522in}}%
\pgfpathlineto{\pgfqpoint{1.650937in}{0.780824in}}%
\pgfpathlineto{\pgfqpoint{1.612972in}{0.766518in}}%
\pgfpathclose%
\pgfusepath{fill}%
\end{pgfscope}%
\begin{pgfscope}%
\pgfpathrectangle{\pgfqpoint{0.150000in}{0.150000in}}{\pgfqpoint{2.700000in}{1.950000in}}%
\pgfusepath{clip}%
\pgfsetbuttcap%
\pgfsetroundjoin%
\definecolor{currentfill}{rgb}{0.835187,0.855484,0.883900}%
\pgfsetfillcolor{currentfill}%
\pgfsetlinewidth{0.000000pt}%
\definecolor{currentstroke}{rgb}{0.000000,0.000000,0.000000}%
\pgfsetstrokecolor{currentstroke}%
\pgfsetdash{}{0pt}%
\pgfpathmoveto{\pgfqpoint{1.230590in}{1.233632in}}%
\pgfpathlineto{\pgfqpoint{1.269631in}{1.240641in}}%
\pgfpathlineto{\pgfqpoint{1.232071in}{1.259909in}}%
\pgfpathlineto{\pgfqpoint{1.192789in}{1.259147in}}%
\pgfpathclose%
\pgfusepath{fill}%
\end{pgfscope}%
\begin{pgfscope}%
\pgfpathrectangle{\pgfqpoint{0.150000in}{0.150000in}}{\pgfqpoint{2.700000in}{1.950000in}}%
\pgfusepath{clip}%
\pgfsetbuttcap%
\pgfsetroundjoin%
\definecolor{currentfill}{rgb}{0.705576,0.465763,0.484758}%
\pgfsetfillcolor{currentfill}%
\pgfsetlinewidth{0.000000pt}%
\definecolor{currentstroke}{rgb}{0.000000,0.000000,0.000000}%
\pgfsetstrokecolor{currentstroke}%
\pgfsetdash{}{0pt}%
\pgfpathmoveto{\pgfqpoint{1.460067in}{0.445503in}}%
\pgfpathlineto{\pgfqpoint{1.498259in}{0.489185in}}%
\pgfpathlineto{\pgfqpoint{1.460305in}{0.492706in}}%
\pgfpathlineto{\pgfqpoint{1.422268in}{0.449179in}}%
\pgfpathclose%
\pgfusepath{fill}%
\end{pgfscope}%
\begin{pgfscope}%
\pgfpathrectangle{\pgfqpoint{0.150000in}{0.150000in}}{\pgfqpoint{2.700000in}{1.950000in}}%
\pgfusepath{clip}%
\pgfsetbuttcap%
\pgfsetroundjoin%
\definecolor{currentfill}{rgb}{0.903125,0.824219,0.830469}%
\pgfsetfillcolor{currentfill}%
\pgfsetlinewidth{0.000000pt}%
\definecolor{currentstroke}{rgb}{0.000000,0.000000,0.000000}%
\pgfsetstrokecolor{currentstroke}%
\pgfsetdash{}{0pt}%
\pgfpathmoveto{\pgfqpoint{1.995835in}{0.854301in}}%
\pgfpathlineto{\pgfqpoint{2.033481in}{0.880326in}}%
\pgfpathlineto{\pgfqpoint{1.992815in}{0.864640in}}%
\pgfpathlineto{\pgfqpoint{1.955317in}{0.838733in}}%
\pgfpathclose%
\pgfusepath{fill}%
\end{pgfscope}%
\begin{pgfscope}%
\pgfpathrectangle{\pgfqpoint{0.150000in}{0.150000in}}{\pgfqpoint{2.700000in}{1.950000in}}%
\pgfusepath{clip}%
\pgfsetbuttcap%
\pgfsetroundjoin%
\definecolor{currentfill}{rgb}{0.897381,0.910018,0.927711}%
\pgfsetfillcolor{currentfill}%
\pgfsetlinewidth{0.000000pt}%
\definecolor{currentstroke}{rgb}{0.000000,0.000000,0.000000}%
\pgfsetstrokecolor{currentstroke}%
\pgfsetdash{}{0pt}%
\pgfpathmoveto{\pgfqpoint{1.460040in}{1.151706in}}%
\pgfpathlineto{\pgfqpoint{1.498334in}{1.171190in}}%
\pgfpathlineto{\pgfqpoint{1.460278in}{1.196727in}}%
\pgfpathlineto{\pgfqpoint{1.422029in}{1.171190in}}%
\pgfpathclose%
\pgfusepath{fill}%
\end{pgfscope}%
\begin{pgfscope}%
\pgfpathrectangle{\pgfqpoint{0.150000in}{0.150000in}}{\pgfqpoint{2.700000in}{1.950000in}}%
\pgfusepath{clip}%
\pgfsetbuttcap%
\pgfsetroundjoin%
\definecolor{currentfill}{rgb}{0.728370,0.507123,0.524648}%
\pgfsetfillcolor{currentfill}%
\pgfsetlinewidth{0.000000pt}%
\definecolor{currentstroke}{rgb}{0.000000,0.000000,0.000000}%
\pgfsetstrokecolor{currentstroke}%
\pgfsetdash{}{0pt}%
\pgfpathmoveto{\pgfqpoint{1.422268in}{0.449179in}}%
\pgfpathlineto{\pgfqpoint{1.460305in}{0.492706in}}%
\pgfpathlineto{\pgfqpoint{1.421762in}{0.570880in}}%
\pgfpathlineto{\pgfqpoint{1.383503in}{0.532907in}}%
\pgfpathclose%
\pgfusepath{fill}%
\end{pgfscope}%
\begin{pgfscope}%
\pgfpathrectangle{\pgfqpoint{0.150000in}{0.150000in}}{\pgfqpoint{2.700000in}{1.950000in}}%
\pgfusepath{clip}%
\pgfsetbuttcap%
\pgfsetroundjoin%
\definecolor{currentfill}{rgb}{0.891161,0.904565,0.923330}%
\pgfsetfillcolor{currentfill}%
\pgfsetlinewidth{0.000000pt}%
\definecolor{currentstroke}{rgb}{0.000000,0.000000,0.000000}%
\pgfsetstrokecolor{currentstroke}%
\pgfsetdash{}{0pt}%
\pgfpathmoveto{\pgfqpoint{1.536486in}{1.151706in}}%
\pgfpathlineto{\pgfqpoint{1.574639in}{1.171190in}}%
\pgfpathlineto{\pgfqpoint{1.536486in}{1.202860in}}%
\pgfpathlineto{\pgfqpoint{1.498334in}{1.171190in}}%
\pgfpathclose%
\pgfusepath{fill}%
\end{pgfscope}%
\begin{pgfscope}%
\pgfpathrectangle{\pgfqpoint{0.150000in}{0.150000in}}{\pgfqpoint{2.700000in}{1.950000in}}%
\pgfusepath{clip}%
\pgfsetbuttcap%
\pgfsetroundjoin%
\definecolor{currentfill}{rgb}{0.728370,0.507123,0.524648}%
\pgfsetfillcolor{currentfill}%
\pgfsetlinewidth{0.000000pt}%
\definecolor{currentstroke}{rgb}{0.000000,0.000000,0.000000}%
\pgfsetstrokecolor{currentstroke}%
\pgfsetdash{}{0pt}%
\pgfpathmoveto{\pgfqpoint{1.498259in}{0.489185in}}%
\pgfpathlineto{\pgfqpoint{1.536486in}{0.532907in}}%
\pgfpathlineto{\pgfqpoint{1.498378in}{0.536274in}}%
\pgfpathlineto{\pgfqpoint{1.460305in}{0.492706in}}%
\pgfpathclose%
\pgfusepath{fill}%
\end{pgfscope}%
\begin{pgfscope}%
\pgfpathrectangle{\pgfqpoint{0.150000in}{0.150000in}}{\pgfqpoint{2.700000in}{1.950000in}}%
\pgfusepath{clip}%
\pgfsetbuttcap%
\pgfsetroundjoin%
\definecolor{currentfill}{rgb}{0.791651,0.817310,0.853232}%
\pgfsetfillcolor{currentfill}%
\pgfsetlinewidth{0.000000pt}%
\definecolor{currentstroke}{rgb}{0.000000,0.000000,0.000000}%
\pgfsetstrokecolor{currentstroke}%
\pgfsetdash{}{0pt}%
\pgfpathmoveto{\pgfqpoint{1.115680in}{1.277717in}}%
\pgfpathlineto{\pgfqpoint{1.155085in}{1.284596in}}%
\pgfpathlineto{\pgfqpoint{1.117477in}{1.309982in}}%
\pgfpathlineto{\pgfqpoint{1.078008in}{1.303187in}}%
\pgfpathclose%
\pgfusepath{fill}%
\end{pgfscope}%
\begin{pgfscope}%
\pgfpathrectangle{\pgfqpoint{0.150000in}{0.150000in}}{\pgfqpoint{2.700000in}{1.950000in}}%
\pgfusepath{clip}%
\pgfsetbuttcap%
\pgfsetroundjoin%
\definecolor{currentfill}{rgb}{0.751164,0.548483,0.564537}%
\pgfsetfillcolor{currentfill}%
\pgfsetlinewidth{0.000000pt}%
\definecolor{currentstroke}{rgb}{0.000000,0.000000,0.000000}%
\pgfsetstrokecolor{currentstroke}%
\pgfsetdash{}{0pt}%
\pgfpathmoveto{\pgfqpoint{1.460305in}{0.492706in}}%
\pgfpathlineto{\pgfqpoint{1.498378in}{0.536274in}}%
\pgfpathlineto{\pgfqpoint{1.459967in}{0.614657in}}%
\pgfpathlineto{\pgfqpoint{1.421762in}{0.570880in}}%
\pgfpathclose%
\pgfusepath{fill}%
\end{pgfscope}%
\begin{pgfscope}%
\pgfpathrectangle{\pgfqpoint{0.150000in}{0.150000in}}{\pgfqpoint{2.700000in}{1.950000in}}%
\pgfusepath{clip}%
\pgfsetbuttcap%
\pgfsetroundjoin%
\definecolor{currentfill}{rgb}{0.692142,0.730055,0.783134}%
\pgfsetfillcolor{currentfill}%
\pgfsetlinewidth{0.000000pt}%
\definecolor{currentstroke}{rgb}{0.000000,0.000000,0.000000}%
\pgfsetstrokecolor{currentstroke}%
\pgfsetdash{}{0pt}%
\pgfpathmoveto{\pgfqpoint{0.772079in}{1.385132in}}%
\pgfpathlineto{\pgfqpoint{0.811221in}{1.410443in}}%
\pgfpathlineto{\pgfqpoint{0.774017in}{1.435690in}}%
\pgfpathlineto{\pgfqpoint{0.734412in}{1.416711in}}%
\pgfpathclose%
\pgfusepath{fill}%
\end{pgfscope}%
\begin{pgfscope}%
\pgfpathrectangle{\pgfqpoint{0.150000in}{0.150000in}}{\pgfqpoint{2.700000in}{1.950000in}}%
\pgfusepath{clip}%
\pgfsetbuttcap%
\pgfsetroundjoin%
\definecolor{currentfill}{rgb}{0.830944,0.693244,0.704151}%
\pgfsetfillcolor{currentfill}%
\pgfsetlinewidth{0.000000pt}%
\definecolor{currentstroke}{rgb}{0.000000,0.000000,0.000000}%
\pgfsetstrokecolor{currentstroke}%
\pgfsetdash{}{0pt}%
\pgfpathmoveto{\pgfqpoint{1.689670in}{0.702336in}}%
\pgfpathlineto{\pgfqpoint{1.727611in}{0.722688in}}%
\pgfpathlineto{\pgfqpoint{1.688750in}{0.719522in}}%
\pgfpathlineto{\pgfqpoint{1.650895in}{0.699249in}}%
\pgfpathclose%
\pgfusepath{fill}%
\end{pgfscope}%
\begin{pgfscope}%
\pgfpathrectangle{\pgfqpoint{0.150000in}{0.150000in}}{\pgfqpoint{2.700000in}{1.950000in}}%
\pgfusepath{clip}%
\pgfsetbuttcap%
\pgfsetroundjoin%
\definecolor{currentfill}{rgb}{0.916039,0.926379,0.940855}%
\pgfsetfillcolor{currentfill}%
\pgfsetlinewidth{0.000000pt}%
\definecolor{currentstroke}{rgb}{0.000000,0.000000,0.000000}%
\pgfsetstrokecolor{currentstroke}%
\pgfsetdash{}{0pt}%
\pgfpathmoveto{\pgfqpoint{1.651236in}{1.119920in}}%
\pgfpathlineto{\pgfqpoint{1.689114in}{1.133371in}}%
\pgfpathlineto{\pgfqpoint{1.650878in}{1.165071in}}%
\pgfpathlineto{\pgfqpoint{1.612889in}{1.145587in}}%
\pgfpathclose%
\pgfusepath{fill}%
\end{pgfscope}%
\begin{pgfscope}%
\pgfpathrectangle{\pgfqpoint{0.150000in}{0.150000in}}{\pgfqpoint{2.700000in}{1.950000in}}%
\pgfusepath{clip}%
\pgfsetbuttcap%
\pgfsetroundjoin%
\definecolor{currentfill}{rgb}{0.754335,0.784589,0.826945}%
\pgfsetfillcolor{currentfill}%
\pgfsetlinewidth{0.000000pt}%
\definecolor{currentstroke}{rgb}{0.000000,0.000000,0.000000}%
\pgfsetstrokecolor{currentstroke}%
\pgfsetdash{}{0pt}%
\pgfpathmoveto{\pgfqpoint{1.000973in}{1.315617in}}%
\pgfpathlineto{\pgfqpoint{1.040432in}{1.328593in}}%
\pgfpathlineto{\pgfqpoint{1.002951in}{1.353935in}}%
\pgfpathlineto{\pgfqpoint{0.963118in}{1.347269in}}%
\pgfpathclose%
\pgfusepath{fill}%
\end{pgfscope}%
\begin{pgfscope}%
\pgfpathrectangle{\pgfqpoint{0.150000in}{0.150000in}}{\pgfqpoint{2.700000in}{1.950000in}}%
\pgfusepath{clip}%
\pgfsetbuttcap%
\pgfsetroundjoin%
\definecolor{currentfill}{rgb}{0.754963,0.555377,0.571186}%
\pgfsetfillcolor{currentfill}%
\pgfsetlinewidth{0.000000pt}%
\definecolor{currentstroke}{rgb}{0.000000,0.000000,0.000000}%
\pgfsetstrokecolor{currentstroke}%
\pgfsetdash{}{0pt}%
\pgfpathmoveto{\pgfqpoint{1.536486in}{0.532907in}}%
\pgfpathlineto{\pgfqpoint{1.574750in}{0.576671in}}%
\pgfpathlineto{\pgfqpoint{1.536486in}{0.579882in}}%
\pgfpathlineto{\pgfqpoint{1.498378in}{0.536274in}}%
\pgfpathclose%
\pgfusepath{fill}%
\end{pgfscope}%
\begin{pgfscope}%
\pgfpathrectangle{\pgfqpoint{0.150000in}{0.150000in}}{\pgfqpoint{2.700000in}{1.950000in}}%
\pgfusepath{clip}%
\pgfsetbuttcap%
\pgfsetroundjoin%
\definecolor{currentfill}{rgb}{0.717019,0.751869,0.800659}%
\pgfsetfillcolor{currentfill}%
\pgfsetlinewidth{0.000000pt}%
\definecolor{currentstroke}{rgb}{0.000000,0.000000,0.000000}%
\pgfsetstrokecolor{currentstroke}%
\pgfsetdash{}{0pt}%
\pgfpathmoveto{\pgfqpoint{0.886291in}{1.353509in}}%
\pgfpathlineto{\pgfqpoint{0.925670in}{1.372631in}}%
\pgfpathlineto{\pgfqpoint{0.888317in}{1.397929in}}%
\pgfpathlineto{\pgfqpoint{0.848919in}{1.378878in}}%
\pgfpathclose%
\pgfusepath{fill}%
\end{pgfscope}%
\begin{pgfscope}%
\pgfpathrectangle{\pgfqpoint{0.150000in}{0.150000in}}{\pgfqpoint{2.700000in}{1.950000in}}%
\pgfusepath{clip}%
\pgfsetbuttcap%
\pgfsetroundjoin%
\definecolor{currentfill}{rgb}{0.773958,0.589844,0.604427}%
\pgfsetfillcolor{currentfill}%
\pgfsetlinewidth{0.000000pt}%
\definecolor{currentstroke}{rgb}{0.000000,0.000000,0.000000}%
\pgfsetstrokecolor{currentstroke}%
\pgfsetdash{}{0pt}%
\pgfpathmoveto{\pgfqpoint{1.498378in}{0.536274in}}%
\pgfpathlineto{\pgfqpoint{1.536486in}{0.579882in}}%
\pgfpathlineto{\pgfqpoint{1.498231in}{0.652635in}}%
\pgfpathlineto{\pgfqpoint{1.459967in}{0.614657in}}%
\pgfpathclose%
\pgfusepath{fill}%
\end{pgfscope}%
\begin{pgfscope}%
\pgfpathrectangle{\pgfqpoint{0.150000in}{0.150000in}}{\pgfqpoint{2.700000in}{1.950000in}}%
\pgfusepath{clip}%
\pgfsetbuttcap%
\pgfsetroundjoin%
\definecolor{currentfill}{rgb}{0.685922,0.724602,0.778753}%
\pgfsetfillcolor{currentfill}%
\pgfsetlinewidth{0.000000pt}%
\definecolor{currentstroke}{rgb}{0.000000,0.000000,0.000000}%
\pgfsetstrokecolor{currentstroke}%
\pgfsetdash{}{0pt}%
\pgfpathmoveto{\pgfqpoint{0.695638in}{1.385132in}}%
\pgfpathlineto{\pgfqpoint{0.734412in}{1.416711in}}%
\pgfpathlineto{\pgfqpoint{0.697284in}{1.441965in}}%
\pgfpathlineto{\pgfqpoint{0.658024in}{1.416711in}}%
\pgfpathclose%
\pgfusepath{fill}%
\end{pgfscope}%
\begin{pgfscope}%
\pgfpathrectangle{\pgfqpoint{0.150000in}{0.150000in}}{\pgfqpoint{2.700000in}{1.950000in}}%
\pgfusepath{clip}%
\pgfsetbuttcap%
\pgfsetroundjoin%
\definecolor{currentfill}{rgb}{0.895527,0.810432,0.817172}%
\pgfsetfillcolor{currentfill}%
\pgfsetlinewidth{0.000000pt}%
\definecolor{currentstroke}{rgb}{0.000000,0.000000,0.000000}%
\pgfsetstrokecolor{currentstroke}%
\pgfsetdash{}{0pt}%
\pgfpathmoveto{\pgfqpoint{1.650937in}{0.780824in}}%
\pgfpathlineto{\pgfqpoint{1.688805in}{0.800955in}}%
\pgfpathlineto{\pgfqpoint{1.651178in}{0.880326in}}%
\pgfpathlineto{\pgfqpoint{1.613089in}{0.860263in}}%
\pgfpathclose%
\pgfusepath{fill}%
\end{pgfscope}%
\begin{pgfscope}%
\pgfpathrectangle{\pgfqpoint{0.150000in}{0.150000in}}{\pgfqpoint{2.700000in}{1.950000in}}%
\pgfusepath{clip}%
\pgfsetbuttcap%
\pgfsetroundjoin%
\definecolor{currentfill}{rgb}{0.878722,0.893658,0.914568}%
\pgfsetfillcolor{currentfill}%
\pgfsetlinewidth{0.000000pt}%
\definecolor{currentstroke}{rgb}{0.000000,0.000000,0.000000}%
\pgfsetstrokecolor{currentstroke}%
\pgfsetdash{}{0pt}%
\pgfpathmoveto{\pgfqpoint{1.383327in}{1.170106in}}%
\pgfpathlineto{\pgfqpoint{1.422029in}{1.171190in}}%
\pgfpathlineto{\pgfqpoint{1.384069in}{1.196727in}}%
\pgfpathlineto{\pgfqpoint{1.345391in}{1.189589in}}%
\pgfpathclose%
\pgfusepath{fill}%
\end{pgfscope}%
\begin{pgfscope}%
\pgfpathrectangle{\pgfqpoint{0.150000in}{0.150000in}}{\pgfqpoint{2.700000in}{1.950000in}}%
\pgfusepath{clip}%
\pgfsetbuttcap%
\pgfsetroundjoin%
\definecolor{currentfill}{rgb}{0.796752,0.631204,0.644317}%
\pgfsetfillcolor{currentfill}%
\pgfsetlinewidth{0.000000pt}%
\definecolor{currentstroke}{rgb}{0.000000,0.000000,0.000000}%
\pgfsetstrokecolor{currentstroke}%
\pgfsetdash{}{0pt}%
\pgfpathmoveto{\pgfqpoint{1.536486in}{0.579882in}}%
\pgfpathlineto{\pgfqpoint{1.574631in}{0.623532in}}%
\pgfpathlineto{\pgfqpoint{1.536486in}{0.690604in}}%
\pgfpathlineto{\pgfqpoint{1.498231in}{0.652635in}}%
\pgfpathclose%
\pgfusepath{fill}%
\end{pgfscope}%
\begin{pgfscope}%
\pgfpathrectangle{\pgfqpoint{0.150000in}{0.150000in}}{\pgfqpoint{2.700000in}{1.950000in}}%
\pgfusepath{clip}%
\pgfsetbuttcap%
\pgfsetroundjoin%
\definecolor{currentfill}{rgb}{0.972013,0.975460,0.980285}%
\pgfsetfillcolor{currentfill}%
\pgfsetlinewidth{0.000000pt}%
\definecolor{currentstroke}{rgb}{0.000000,0.000000,0.000000}%
\pgfsetstrokecolor{currentstroke}%
\pgfsetdash{}{0pt}%
\pgfpathmoveto{\pgfqpoint{1.727444in}{1.025906in}}%
\pgfpathlineto{\pgfqpoint{1.765345in}{1.051653in}}%
\pgfpathlineto{\pgfqpoint{1.727402in}{1.101627in}}%
\pgfpathlineto{\pgfqpoint{1.689591in}{1.088090in}}%
\pgfpathclose%
\pgfusepath{fill}%
\end{pgfscope}%
\begin{pgfscope}%
\pgfpathrectangle{\pgfqpoint{0.150000in}{0.150000in}}{\pgfqpoint{2.700000in}{1.950000in}}%
\pgfusepath{clip}%
\pgfsetbuttcap%
\pgfsetroundjoin%
\definecolor{currentfill}{rgb}{0.838542,0.707031,0.717448}%
\pgfsetfillcolor{currentfill}%
\pgfsetlinewidth{0.000000pt}%
\definecolor{currentstroke}{rgb}{0.000000,0.000000,0.000000}%
\pgfsetstrokecolor{currentstroke}%
\pgfsetdash{}{0pt}%
\pgfpathmoveto{\pgfqpoint{1.612812in}{0.667223in}}%
\pgfpathlineto{\pgfqpoint{1.650895in}{0.699249in}}%
\pgfpathlineto{\pgfqpoint{1.612972in}{0.766518in}}%
\pgfpathlineto{\pgfqpoint{1.574756in}{0.734449in}}%
\pgfpathclose%
\pgfusepath{fill}%
\end{pgfscope}%
\begin{pgfscope}%
\pgfpathrectangle{\pgfqpoint{0.150000in}{0.150000in}}{\pgfqpoint{2.700000in}{1.950000in}}%
\pgfusepath{clip}%
\pgfsetbuttcap%
\pgfsetroundjoin%
\definecolor{currentfill}{rgb}{0.841406,0.860938,0.888281}%
\pgfsetfillcolor{currentfill}%
\pgfsetlinewidth{0.000000pt}%
\definecolor{currentstroke}{rgb}{0.000000,0.000000,0.000000}%
\pgfsetstrokecolor{currentstroke}%
\pgfsetdash{}{0pt}%
\pgfpathmoveto{\pgfqpoint{1.268486in}{1.208052in}}%
\pgfpathlineto{\pgfqpoint{1.307463in}{1.215147in}}%
\pgfpathlineto{\pgfqpoint{1.269631in}{1.240641in}}%
\pgfpathlineto{\pgfqpoint{1.230590in}{1.233632in}}%
\pgfpathclose%
\pgfusepath{fill}%
\end{pgfscope}%
\begin{pgfscope}%
\pgfpathrectangle{\pgfqpoint{0.150000in}{0.150000in}}{\pgfqpoint{2.700000in}{1.950000in}}%
\pgfusepath{clip}%
\pgfsetbuttcap%
\pgfsetroundjoin%
\definecolor{currentfill}{rgb}{0.777757,0.596737,0.611075}%
\pgfsetfillcolor{currentfill}%
\pgfsetlinewidth{0.000000pt}%
\definecolor{currentstroke}{rgb}{0.000000,0.000000,0.000000}%
\pgfsetstrokecolor{currentstroke}%
\pgfsetdash{}{0pt}%
\pgfpathmoveto{\pgfqpoint{1.574750in}{0.576671in}}%
\pgfpathlineto{\pgfqpoint{1.613050in}{0.620477in}}%
\pgfpathlineto{\pgfqpoint{1.574631in}{0.623532in}}%
\pgfpathlineto{\pgfqpoint{1.536486in}{0.579882in}}%
\pgfpathclose%
\pgfusepath{fill}%
\end{pgfscope}%
\begin{pgfscope}%
\pgfpathrectangle{\pgfqpoint{0.150000in}{0.150000in}}{\pgfqpoint{2.700000in}{1.950000in}}%
\pgfusepath{clip}%
\pgfsetbuttcap%
\pgfsetroundjoin%
\definecolor{currentfill}{rgb}{0.857537,0.741498,0.750689}%
\pgfsetfillcolor{currentfill}%
\pgfsetlinewidth{0.000000pt}%
\definecolor{currentstroke}{rgb}{0.000000,0.000000,0.000000}%
\pgfsetstrokecolor{currentstroke}%
\pgfsetdash{}{0pt}%
\pgfpathmoveto{\pgfqpoint{1.804684in}{0.746237in}}%
\pgfpathlineto{\pgfqpoint{1.842608in}{0.772423in}}%
\pgfpathlineto{\pgfqpoint{1.803073in}{0.763168in}}%
\pgfpathlineto{\pgfqpoint{1.765412in}{0.742965in}}%
\pgfpathclose%
\pgfusepath{fill}%
\end{pgfscope}%
\begin{pgfscope}%
\pgfpathrectangle{\pgfqpoint{0.150000in}{0.150000in}}{\pgfqpoint{2.700000in}{1.950000in}}%
\pgfusepath{clip}%
\pgfsetbuttcap%
\pgfsetroundjoin%
\definecolor{currentfill}{rgb}{0.956311,0.920726,0.923545}%
\pgfsetfillcolor{currentfill}%
\pgfsetlinewidth{0.000000pt}%
\definecolor{currentstroke}{rgb}{0.000000,0.000000,0.000000}%
\pgfsetstrokecolor{currentstroke}%
\pgfsetdash{}{0pt}%
\pgfpathmoveto{\pgfqpoint{1.689125in}{0.900314in}}%
\pgfpathlineto{\pgfqpoint{1.726933in}{0.920229in}}%
\pgfpathlineto{\pgfqpoint{1.689535in}{1.006133in}}%
\pgfpathlineto{\pgfqpoint{1.651419in}{0.980247in}}%
\pgfpathclose%
\pgfusepath{fill}%
\end{pgfscope}%
\begin{pgfscope}%
\pgfpathrectangle{\pgfqpoint{0.150000in}{0.150000in}}{\pgfqpoint{2.700000in}{1.950000in}}%
\pgfusepath{clip}%
\pgfsetbuttcap%
\pgfsetroundjoin%
\definecolor{currentfill}{rgb}{0.819547,0.672564,0.684206}%
\pgfsetfillcolor{currentfill}%
\pgfsetlinewidth{0.000000pt}%
\definecolor{currentstroke}{rgb}{0.000000,0.000000,0.000000}%
\pgfsetstrokecolor{currentstroke}%
\pgfsetdash{}{0pt}%
\pgfpathmoveto{\pgfqpoint{1.574631in}{0.623532in}}%
\pgfpathlineto{\pgfqpoint{1.612812in}{0.667223in}}%
\pgfpathlineto{\pgfqpoint{1.574756in}{0.734449in}}%
\pgfpathlineto{\pgfqpoint{1.536486in}{0.690604in}}%
\pgfpathclose%
\pgfusepath{fill}%
\end{pgfscope}%
\begin{pgfscope}%
\pgfpathrectangle{\pgfqpoint{0.150000in}{0.150000in}}{\pgfqpoint{2.700000in}{1.950000in}}%
\pgfusepath{clip}%
\pgfsetbuttcap%
\pgfsetroundjoin%
\definecolor{currentfill}{rgb}{0.899326,0.817325,0.823820}%
\pgfsetfillcolor{currentfill}%
\pgfsetlinewidth{0.000000pt}%
\definecolor{currentstroke}{rgb}{0.000000,0.000000,0.000000}%
\pgfsetstrokecolor{currentstroke}%
\pgfsetdash{}{0pt}%
\pgfpathmoveto{\pgfqpoint{1.958092in}{0.828210in}}%
\pgfpathlineto{\pgfqpoint{1.995835in}{0.854301in}}%
\pgfpathlineto{\pgfqpoint{1.955317in}{0.838733in}}%
\pgfpathlineto{\pgfqpoint{1.917945in}{0.818675in}}%
\pgfpathclose%
\pgfusepath{fill}%
\end{pgfscope}%
\begin{pgfscope}%
\pgfpathrectangle{\pgfqpoint{0.150000in}{0.150000in}}{\pgfqpoint{2.700000in}{1.950000in}}%
\pgfusepath{clip}%
\pgfsetbuttcap%
\pgfsetroundjoin%
\definecolor{currentfill}{rgb}{0.903600,0.915472,0.932093}%
\pgfsetfillcolor{currentfill}%
\pgfsetlinewidth{0.000000pt}%
\definecolor{currentstroke}{rgb}{0.000000,0.000000,0.000000}%
\pgfsetstrokecolor{currentstroke}%
\pgfsetdash{}{0pt}%
\pgfpathmoveto{\pgfqpoint{1.574758in}{1.126032in}}%
\pgfpathlineto{\pgfqpoint{1.612889in}{1.145587in}}%
\pgfpathlineto{\pgfqpoint{1.574639in}{1.171190in}}%
\pgfpathlineto{\pgfqpoint{1.536486in}{1.151706in}}%
\pgfpathclose%
\pgfusepath{fill}%
\end{pgfscope}%
\begin{pgfscope}%
\pgfpathrectangle{\pgfqpoint{0.150000in}{0.150000in}}{\pgfqpoint{2.700000in}{1.950000in}}%
\pgfusepath{clip}%
\pgfsetbuttcap%
\pgfsetroundjoin%
\definecolor{currentfill}{rgb}{0.704580,0.740962,0.791896}%
\pgfsetfillcolor{currentfill}%
\pgfsetlinewidth{0.000000pt}%
\definecolor{currentstroke}{rgb}{0.000000,0.000000,0.000000}%
\pgfsetstrokecolor{currentstroke}%
\pgfsetdash{}{0pt}%
\pgfpathmoveto{\pgfqpoint{0.809375in}{1.359756in}}%
\pgfpathlineto{\pgfqpoint{0.848919in}{1.378878in}}%
\pgfpathlineto{\pgfqpoint{0.811221in}{1.410443in}}%
\pgfpathlineto{\pgfqpoint{0.772079in}{1.385132in}}%
\pgfpathclose%
\pgfusepath{fill}%
\end{pgfscope}%
\begin{pgfscope}%
\pgfpathrectangle{\pgfqpoint{0.150000in}{0.150000in}}{\pgfqpoint{2.700000in}{1.950000in}}%
\pgfusepath{clip}%
\pgfsetbuttcap%
\pgfsetroundjoin%
\definecolor{currentfill}{rgb}{0.934697,0.942739,0.953998}%
\pgfsetfillcolor{currentfill}%
\pgfsetlinewidth{0.000000pt}%
\definecolor{currentstroke}{rgb}{0.000000,0.000000,0.000000}%
\pgfsetstrokecolor{currentstroke}%
\pgfsetdash{}{0pt}%
\pgfpathmoveto{\pgfqpoint{1.689591in}{1.088090in}}%
\pgfpathlineto{\pgfqpoint{1.727402in}{1.101627in}}%
\pgfpathlineto{\pgfqpoint{1.689114in}{1.133371in}}%
\pgfpathlineto{\pgfqpoint{1.651236in}{1.119920in}}%
\pgfpathclose%
\pgfusepath{fill}%
\end{pgfscope}%
\begin{pgfscope}%
\pgfpathrectangle{\pgfqpoint{0.150000in}{0.150000in}}{\pgfqpoint{2.700000in}{1.950000in}}%
\pgfusepath{clip}%
\pgfsetbuttcap%
\pgfsetroundjoin%
\definecolor{currentfill}{rgb}{0.797871,0.822763,0.857613}%
\pgfsetfillcolor{currentfill}%
\pgfsetlinewidth{0.000000pt}%
\definecolor{currentstroke}{rgb}{0.000000,0.000000,0.000000}%
\pgfsetstrokecolor{currentstroke}%
\pgfsetdash{}{0pt}%
\pgfpathmoveto{\pgfqpoint{1.153448in}{1.252181in}}%
\pgfpathlineto{\pgfqpoint{1.192789in}{1.259147in}}%
\pgfpathlineto{\pgfqpoint{1.155085in}{1.284596in}}%
\pgfpathlineto{\pgfqpoint{1.115680in}{1.277717in}}%
\pgfpathclose%
\pgfusepath{fill}%
\end{pgfscope}%
\begin{pgfscope}%
\pgfpathrectangle{\pgfqpoint{0.150000in}{0.150000in}}{\pgfqpoint{2.700000in}{1.950000in}}%
\pgfusepath{clip}%
\pgfsetbuttcap%
\pgfsetroundjoin%
\definecolor{currentfill}{rgb}{0.804350,0.644991,0.657613}%
\pgfsetfillcolor{currentfill}%
\pgfsetlinewidth{0.000000pt}%
\definecolor{currentstroke}{rgb}{0.000000,0.000000,0.000000}%
\pgfsetstrokecolor{currentstroke}%
\pgfsetdash{}{0pt}%
\pgfpathmoveto{\pgfqpoint{1.613050in}{0.620477in}}%
\pgfpathlineto{\pgfqpoint{1.651387in}{0.664323in}}%
\pgfpathlineto{\pgfqpoint{1.612812in}{0.667223in}}%
\pgfpathlineto{\pgfqpoint{1.574631in}{0.623532in}}%
\pgfpathclose%
\pgfusepath{fill}%
\end{pgfscope}%
\begin{pgfscope}%
\pgfpathrectangle{\pgfqpoint{0.150000in}{0.150000in}}{\pgfqpoint{2.700000in}{1.950000in}}%
\pgfusepath{clip}%
\pgfsetbuttcap%
\pgfsetroundjoin%
\definecolor{currentfill}{rgb}{0.897381,0.910018,0.927711}%
\pgfsetfillcolor{currentfill}%
\pgfsetlinewidth{0.000000pt}%
\definecolor{currentstroke}{rgb}{0.000000,0.000000,0.000000}%
\pgfsetstrokecolor{currentstroke}%
\pgfsetdash{}{0pt}%
\pgfpathmoveto{\pgfqpoint{1.498192in}{1.132150in}}%
\pgfpathlineto{\pgfqpoint{1.536486in}{1.151706in}}%
\pgfpathlineto{\pgfqpoint{1.498334in}{1.171190in}}%
\pgfpathlineto{\pgfqpoint{1.460040in}{1.151706in}}%
\pgfpathclose%
\pgfusepath{fill}%
\end{pgfscope}%
\begin{pgfscope}%
\pgfpathrectangle{\pgfqpoint{0.150000in}{0.150000in}}{\pgfqpoint{2.700000in}{1.950000in}}%
\pgfusepath{clip}%
\pgfsetbuttcap%
\pgfsetroundjoin%
\definecolor{currentfill}{rgb}{0.823346,0.679458,0.690855}%
\pgfsetfillcolor{currentfill}%
\pgfsetlinewidth{0.000000pt}%
\definecolor{currentstroke}{rgb}{0.000000,0.000000,0.000000}%
\pgfsetstrokecolor{currentstroke}%
\pgfsetdash{}{0pt}%
\pgfpathmoveto{\pgfqpoint{1.651387in}{0.664323in}}%
\pgfpathlineto{\pgfqpoint{1.689670in}{0.702336in}}%
\pgfpathlineto{\pgfqpoint{1.650895in}{0.699249in}}%
\pgfpathlineto{\pgfqpoint{1.612812in}{0.667223in}}%
\pgfpathclose%
\pgfusepath{fill}%
\end{pgfscope}%
\begin{pgfscope}%
\pgfpathrectangle{\pgfqpoint{0.150000in}{0.150000in}}{\pgfqpoint{2.700000in}{1.950000in}}%
\pgfusepath{clip}%
\pgfsetbuttcap%
\pgfsetroundjoin%
\definecolor{currentfill}{rgb}{0.698361,0.735509,0.787515}%
\pgfsetfillcolor{currentfill}%
\pgfsetlinewidth{0.000000pt}%
\definecolor{currentstroke}{rgb}{0.000000,0.000000,0.000000}%
\pgfsetstrokecolor{currentstroke}%
\pgfsetdash{}{0pt}%
\pgfpathmoveto{\pgfqpoint{0.732837in}{1.359756in}}%
\pgfpathlineto{\pgfqpoint{0.772079in}{1.385132in}}%
\pgfpathlineto{\pgfqpoint{0.734412in}{1.416711in}}%
\pgfpathlineto{\pgfqpoint{0.695638in}{1.385132in}}%
\pgfpathclose%
\pgfusepath{fill}%
\end{pgfscope}%
\begin{pgfscope}%
\pgfpathrectangle{\pgfqpoint{0.150000in}{0.150000in}}{\pgfqpoint{2.700000in}{1.950000in}}%
\pgfusepath{clip}%
\pgfsetbuttcap%
\pgfsetroundjoin%
\definecolor{currentfill}{rgb}{0.698361,0.735509,0.787515}%
\pgfsetfillcolor{currentfill}%
\pgfsetlinewidth{0.000000pt}%
\definecolor{currentstroke}{rgb}{0.000000,0.000000,0.000000}%
\pgfsetstrokecolor{currentstroke}%
\pgfsetdash{}{0pt}%
\pgfpathmoveto{\pgfqpoint{0.656299in}{1.359756in}}%
\pgfpathlineto{\pgfqpoint{0.695638in}{1.385132in}}%
\pgfpathlineto{\pgfqpoint{0.658024in}{1.416711in}}%
\pgfpathlineto{\pgfqpoint{0.619197in}{1.385132in}}%
\pgfpathclose%
\pgfusepath{fill}%
\end{pgfscope}%
\begin{pgfscope}%
\pgfpathrectangle{\pgfqpoint{0.150000in}{0.150000in}}{\pgfqpoint{2.700000in}{1.950000in}}%
\pgfusepath{clip}%
\pgfsetbuttcap%
\pgfsetroundjoin%
\definecolor{currentfill}{rgb}{0.766774,0.795496,0.835708}%
\pgfsetfillcolor{currentfill}%
\pgfsetlinewidth{0.000000pt}%
\definecolor{currentstroke}{rgb}{0.000000,0.000000,0.000000}%
\pgfsetstrokecolor{currentstroke}%
\pgfsetdash{}{0pt}%
\pgfpathmoveto{\pgfqpoint{1.038591in}{1.290133in}}%
\pgfpathlineto{\pgfqpoint{1.078008in}{1.303187in}}%
\pgfpathlineto{\pgfqpoint{1.040432in}{1.328593in}}%
\pgfpathlineto{\pgfqpoint{1.000973in}{1.315617in}}%
\pgfpathclose%
\pgfusepath{fill}%
\end{pgfscope}%
\begin{pgfscope}%
\pgfpathrectangle{\pgfqpoint{0.150000in}{0.150000in}}{\pgfqpoint{2.700000in}{1.950000in}}%
\pgfusepath{clip}%
\pgfsetbuttcap%
\pgfsetroundjoin%
\definecolor{currentfill}{rgb}{0.729458,0.762776,0.809421}%
\pgfsetfillcolor{currentfill}%
\pgfsetlinewidth{0.000000pt}%
\definecolor{currentstroke}{rgb}{0.000000,0.000000,0.000000}%
\pgfsetstrokecolor{currentstroke}%
\pgfsetdash{}{0pt}%
\pgfpathmoveto{\pgfqpoint{0.923759in}{1.328076in}}%
\pgfpathlineto{\pgfqpoint{0.963118in}{1.347269in}}%
\pgfpathlineto{\pgfqpoint{0.925670in}{1.372631in}}%
\pgfpathlineto{\pgfqpoint{0.886291in}{1.353509in}}%
\pgfpathclose%
\pgfusepath{fill}%
\end{pgfscope}%
\begin{pgfscope}%
\pgfpathrectangle{\pgfqpoint{0.150000in}{0.150000in}}{\pgfqpoint{2.700000in}{1.950000in}}%
\pgfusepath{clip}%
\pgfsetbuttcap%
\pgfsetroundjoin%
\definecolor{currentfill}{rgb}{0.723238,0.757322,0.805040}%
\pgfsetfillcolor{currentfill}%
\pgfsetlinewidth{0.000000pt}%
\definecolor{currentstroke}{rgb}{0.000000,0.000000,0.000000}%
\pgfsetstrokecolor{currentstroke}%
\pgfsetdash{}{0pt}%
\pgfpathmoveto{\pgfqpoint{0.617929in}{1.321843in}}%
\pgfpathlineto{\pgfqpoint{0.656299in}{1.359756in}}%
\pgfpathlineto{\pgfqpoint{0.619197in}{1.385132in}}%
\pgfpathlineto{\pgfqpoint{0.580872in}{1.347269in}}%
\pgfpathclose%
\pgfusepath{fill}%
\end{pgfscope}%
\begin{pgfscope}%
\pgfpathrectangle{\pgfqpoint{0.150000in}{0.150000in}}{\pgfqpoint{2.700000in}{1.950000in}}%
\pgfusepath{clip}%
\pgfsetbuttcap%
\pgfsetroundjoin%
\definecolor{currentfill}{rgb}{0.884942,0.899112,0.918949}%
\pgfsetfillcolor{currentfill}%
\pgfsetlinewidth{0.000000pt}%
\definecolor{currentstroke}{rgb}{0.000000,0.000000,0.000000}%
\pgfsetstrokecolor{currentstroke}%
\pgfsetdash{}{0pt}%
\pgfpathmoveto{\pgfqpoint{1.421470in}{1.144410in}}%
\pgfpathlineto{\pgfqpoint{1.460040in}{1.151706in}}%
\pgfpathlineto{\pgfqpoint{1.422029in}{1.171190in}}%
\pgfpathlineto{\pgfqpoint{1.383327in}{1.170106in}}%
\pgfpathclose%
\pgfusepath{fill}%
\end{pgfscope}%
\begin{pgfscope}%
\pgfpathrectangle{\pgfqpoint{0.150000in}{0.150000in}}{\pgfqpoint{2.700000in}{1.950000in}}%
\pgfusepath{clip}%
\pgfsetbuttcap%
\pgfsetroundjoin%
\definecolor{currentfill}{rgb}{0.853845,0.871844,0.897044}%
\pgfsetfillcolor{currentfill}%
\pgfsetlinewidth{0.000000pt}%
\definecolor{currentstroke}{rgb}{0.000000,0.000000,0.000000}%
\pgfsetstrokecolor{currentstroke}%
\pgfsetdash{}{0pt}%
\pgfpathmoveto{\pgfqpoint{1.306479in}{1.182408in}}%
\pgfpathlineto{\pgfqpoint{1.345391in}{1.189589in}}%
\pgfpathlineto{\pgfqpoint{1.307463in}{1.215147in}}%
\pgfpathlineto{\pgfqpoint{1.268486in}{1.208052in}}%
\pgfpathclose%
\pgfusepath{fill}%
\end{pgfscope}%
\begin{pgfscope}%
\pgfpathrectangle{\pgfqpoint{0.150000in}{0.150000in}}{\pgfqpoint{2.700000in}{1.950000in}}%
\pgfusepath{clip}%
\pgfsetbuttcap%
\pgfsetroundjoin%
\definecolor{currentfill}{rgb}{0.717019,0.751869,0.800659}%
\pgfsetfillcolor{currentfill}%
\pgfsetlinewidth{0.000000pt}%
\definecolor{currentstroke}{rgb}{0.000000,0.000000,0.000000}%
\pgfsetstrokecolor{currentstroke}%
\pgfsetdash{}{0pt}%
\pgfpathmoveto{\pgfqpoint{0.846766in}{1.334316in}}%
\pgfpathlineto{\pgfqpoint{0.886291in}{1.353509in}}%
\pgfpathlineto{\pgfqpoint{0.848919in}{1.378878in}}%
\pgfpathlineto{\pgfqpoint{0.809375in}{1.359756in}}%
\pgfpathclose%
\pgfusepath{fill}%
\end{pgfscope}%
\begin{pgfscope}%
\pgfpathrectangle{\pgfqpoint{0.150000in}{0.150000in}}{\pgfqpoint{2.700000in}{1.950000in}}%
\pgfusepath{clip}%
\pgfsetbuttcap%
\pgfsetroundjoin%
\definecolor{currentfill}{rgb}{0.916039,0.926379,0.940855}%
\pgfsetfillcolor{currentfill}%
\pgfsetlinewidth{0.000000pt}%
\definecolor{currentstroke}{rgb}{0.000000,0.000000,0.000000}%
\pgfsetstrokecolor{currentstroke}%
\pgfsetdash{}{0pt}%
\pgfpathmoveto{\pgfqpoint{1.613128in}{1.100292in}}%
\pgfpathlineto{\pgfqpoint{1.651236in}{1.119920in}}%
\pgfpathlineto{\pgfqpoint{1.612889in}{1.145587in}}%
\pgfpathlineto{\pgfqpoint{1.574758in}{1.126032in}}%
\pgfpathclose%
\pgfusepath{fill}%
\end{pgfscope}%
\begin{pgfscope}%
\pgfpathrectangle{\pgfqpoint{0.150000in}{0.150000in}}{\pgfqpoint{2.700000in}{1.950000in}}%
\pgfusepath{clip}%
\pgfsetbuttcap%
\pgfsetroundjoin%
\definecolor{currentfill}{rgb}{0.895527,0.810432,0.817172}%
\pgfsetfillcolor{currentfill}%
\pgfsetlinewidth{0.000000pt}%
\definecolor{currentstroke}{rgb}{0.000000,0.000000,0.000000}%
\pgfsetstrokecolor{currentstroke}%
\pgfsetdash{}{0pt}%
\pgfpathmoveto{\pgfqpoint{1.920254in}{0.802052in}}%
\pgfpathlineto{\pgfqpoint{1.958092in}{0.828210in}}%
\pgfpathlineto{\pgfqpoint{1.917945in}{0.818675in}}%
\pgfpathlineto{\pgfqpoint{1.880236in}{0.792629in}}%
\pgfpathclose%
\pgfusepath{fill}%
\end{pgfscope}%
\begin{pgfscope}%
\pgfpathrectangle{\pgfqpoint{0.150000in}{0.150000in}}{\pgfqpoint{2.700000in}{1.950000in}}%
\pgfusepath{clip}%
\pgfsetbuttcap%
\pgfsetroundjoin%
\definecolor{currentfill}{rgb}{0.717019,0.751869,0.800659}%
\pgfsetfillcolor{currentfill}%
\pgfsetlinewidth{0.000000pt}%
\definecolor{currentstroke}{rgb}{0.000000,0.000000,0.000000}%
\pgfsetstrokecolor{currentstroke}%
\pgfsetdash{}{0pt}%
\pgfpathmoveto{\pgfqpoint{0.693986in}{1.328076in}}%
\pgfpathlineto{\pgfqpoint{0.732837in}{1.359756in}}%
\pgfpathlineto{\pgfqpoint{0.695638in}{1.385132in}}%
\pgfpathlineto{\pgfqpoint{0.656299in}{1.359756in}}%
\pgfpathclose%
\pgfusepath{fill}%
\end{pgfscope}%
\begin{pgfscope}%
\pgfpathrectangle{\pgfqpoint{0.150000in}{0.150000in}}{\pgfqpoint{2.700000in}{1.950000in}}%
\pgfusepath{clip}%
\pgfsetbuttcap%
\pgfsetroundjoin%
\definecolor{currentfill}{rgb}{0.853738,0.734605,0.744041}%
\pgfsetfillcolor{currentfill}%
\pgfsetlinewidth{0.000000pt}%
\definecolor{currentstroke}{rgb}{0.000000,0.000000,0.000000}%
\pgfsetstrokecolor{currentstroke}%
\pgfsetdash{}{0pt}%
\pgfpathmoveto{\pgfqpoint{1.766663in}{0.719984in}}%
\pgfpathlineto{\pgfqpoint{1.804684in}{0.746237in}}%
\pgfpathlineto{\pgfqpoint{1.765412in}{0.742965in}}%
\pgfpathlineto{\pgfqpoint{1.727611in}{0.722688in}}%
\pgfpathclose%
\pgfusepath{fill}%
\end{pgfscope}%
\begin{pgfscope}%
\pgfpathrectangle{\pgfqpoint{0.150000in}{0.150000in}}{\pgfqpoint{2.700000in}{1.950000in}}%
\pgfusepath{clip}%
\pgfsetbuttcap%
\pgfsetroundjoin%
\definecolor{currentfill}{rgb}{0.816529,0.839124,0.870757}%
\pgfsetfillcolor{currentfill}%
\pgfsetlinewidth{0.000000pt}%
\definecolor{currentstroke}{rgb}{0.000000,0.000000,0.000000}%
\pgfsetstrokecolor{currentstroke}%
\pgfsetdash{}{0pt}%
\pgfpathmoveto{\pgfqpoint{1.191513in}{1.220397in}}%
\pgfpathlineto{\pgfqpoint{1.230590in}{1.233632in}}%
\pgfpathlineto{\pgfqpoint{1.192789in}{1.259147in}}%
\pgfpathlineto{\pgfqpoint{1.153448in}{1.252181in}}%
\pgfpathclose%
\pgfusepath{fill}%
\end{pgfscope}%
\begin{pgfscope}%
\pgfpathrectangle{\pgfqpoint{0.150000in}{0.150000in}}{\pgfqpoint{2.700000in}{1.950000in}}%
\pgfusepath{clip}%
\pgfsetbuttcap%
\pgfsetroundjoin%
\definecolor{currentfill}{rgb}{0.895527,0.810432,0.817172}%
\pgfsetfillcolor{currentfill}%
\pgfsetlinewidth{0.000000pt}%
\definecolor{currentstroke}{rgb}{0.000000,0.000000,0.000000}%
\pgfsetstrokecolor{currentstroke}%
\pgfsetdash{}{0pt}%
\pgfpathmoveto{\pgfqpoint{1.612972in}{0.766518in}}%
\pgfpathlineto{\pgfqpoint{1.650937in}{0.780824in}}%
\pgfpathlineto{\pgfqpoint{1.613089in}{0.860263in}}%
\pgfpathlineto{\pgfqpoint{1.574859in}{0.840125in}}%
\pgfpathclose%
\pgfusepath{fill}%
\end{pgfscope}%
\begin{pgfscope}%
\pgfpathrectangle{\pgfqpoint{0.150000in}{0.150000in}}{\pgfqpoint{2.700000in}{1.950000in}}%
\pgfusepath{clip}%
\pgfsetbuttcap%
\pgfsetroundjoin%
\definecolor{currentfill}{rgb}{0.717019,0.751869,0.800659}%
\pgfsetfillcolor{currentfill}%
\pgfsetlinewidth{0.000000pt}%
\definecolor{currentstroke}{rgb}{0.000000,0.000000,0.000000}%
\pgfsetstrokecolor{currentstroke}%
\pgfsetdash{}{0pt}%
\pgfpathmoveto{\pgfqpoint{0.770131in}{1.334316in}}%
\pgfpathlineto{\pgfqpoint{0.809375in}{1.359756in}}%
\pgfpathlineto{\pgfqpoint{0.772079in}{1.385132in}}%
\pgfpathlineto{\pgfqpoint{0.732837in}{1.359756in}}%
\pgfpathclose%
\pgfusepath{fill}%
\end{pgfscope}%
\begin{pgfscope}%
\pgfpathrectangle{\pgfqpoint{0.150000in}{0.150000in}}{\pgfqpoint{2.700000in}{1.950000in}}%
\pgfusepath{clip}%
\pgfsetbuttcap%
\pgfsetroundjoin%
\definecolor{currentfill}{rgb}{0.956311,0.920726,0.923545}%
\pgfsetfillcolor{currentfill}%
\pgfsetlinewidth{0.000000pt}%
\definecolor{currentstroke}{rgb}{0.000000,0.000000,0.000000}%
\pgfsetstrokecolor{currentstroke}%
\pgfsetdash{}{0pt}%
\pgfpathmoveto{\pgfqpoint{1.651178in}{0.880326in}}%
\pgfpathlineto{\pgfqpoint{1.689125in}{0.900314in}}%
\pgfpathlineto{\pgfqpoint{1.651419in}{0.980247in}}%
\pgfpathlineto{\pgfqpoint{1.613251in}{0.960327in}}%
\pgfpathclose%
\pgfusepath{fill}%
\end{pgfscope}%
\begin{pgfscope}%
\pgfpathrectangle{\pgfqpoint{0.150000in}{0.150000in}}{\pgfqpoint{2.700000in}{1.950000in}}%
\pgfusepath{clip}%
\pgfsetbuttcap%
\pgfsetroundjoin%
\definecolor{currentfill}{rgb}{0.772993,0.800950,0.840089}%
\pgfsetfillcolor{currentfill}%
\pgfsetlinewidth{0.000000pt}%
\definecolor{currentstroke}{rgb}{0.000000,0.000000,0.000000}%
\pgfsetstrokecolor{currentstroke}%
\pgfsetdash{}{0pt}%
\pgfpathmoveto{\pgfqpoint{1.076304in}{1.264583in}}%
\pgfpathlineto{\pgfqpoint{1.115680in}{1.277717in}}%
\pgfpathlineto{\pgfqpoint{1.078008in}{1.303187in}}%
\pgfpathlineto{\pgfqpoint{1.038591in}{1.290133in}}%
\pgfpathclose%
\pgfusepath{fill}%
\end{pgfscope}%
\begin{pgfscope}%
\pgfpathrectangle{\pgfqpoint{0.150000in}{0.150000in}}{\pgfqpoint{2.700000in}{1.950000in}}%
\pgfusepath{clip}%
\pgfsetbuttcap%
\pgfsetroundjoin%
\definecolor{currentfill}{rgb}{0.741896,0.773683,0.818183}%
\pgfsetfillcolor{currentfill}%
\pgfsetlinewidth{0.000000pt}%
\definecolor{currentstroke}{rgb}{0.000000,0.000000,0.000000}%
\pgfsetstrokecolor{currentstroke}%
\pgfsetdash{}{0pt}%
\pgfpathmoveto{\pgfqpoint{0.961322in}{1.302578in}}%
\pgfpathlineto{\pgfqpoint{1.000973in}{1.315617in}}%
\pgfpathlineto{\pgfqpoint{0.963118in}{1.347269in}}%
\pgfpathlineto{\pgfqpoint{0.923759in}{1.328076in}}%
\pgfpathclose%
\pgfusepath{fill}%
\end{pgfscope}%
\begin{pgfscope}%
\pgfpathrectangle{\pgfqpoint{0.150000in}{0.150000in}}{\pgfqpoint{2.700000in}{1.950000in}}%
\pgfusepath{clip}%
\pgfsetbuttcap%
\pgfsetroundjoin%
\definecolor{currentfill}{rgb}{0.972013,0.975460,0.980285}%
\pgfsetfillcolor{currentfill}%
\pgfsetlinewidth{0.000000pt}%
\definecolor{currentstroke}{rgb}{0.000000,0.000000,0.000000}%
\pgfsetstrokecolor{currentstroke}%
\pgfsetdash{}{0pt}%
\pgfpathmoveto{\pgfqpoint{1.689535in}{1.006133in}}%
\pgfpathlineto{\pgfqpoint{1.727444in}{1.025906in}}%
\pgfpathlineto{\pgfqpoint{1.689591in}{1.088090in}}%
\pgfpathlineto{\pgfqpoint{1.651662in}{1.080590in}}%
\pgfpathclose%
\pgfusepath{fill}%
\end{pgfscope}%
\begin{pgfscope}%
\pgfpathrectangle{\pgfqpoint{0.150000in}{0.150000in}}{\pgfqpoint{2.700000in}{1.950000in}}%
\pgfusepath{clip}%
\pgfsetbuttcap%
\pgfsetroundjoin%
\definecolor{currentfill}{rgb}{0.897381,0.910018,0.927711}%
\pgfsetfillcolor{currentfill}%
\pgfsetlinewidth{0.000000pt}%
\definecolor{currentstroke}{rgb}{0.000000,0.000000,0.000000}%
\pgfsetstrokecolor{currentstroke}%
\pgfsetdash{}{0pt}%
\pgfpathmoveto{\pgfqpoint{1.536486in}{1.112522in}}%
\pgfpathlineto{\pgfqpoint{1.574758in}{1.126032in}}%
\pgfpathlineto{\pgfqpoint{1.536486in}{1.151706in}}%
\pgfpathlineto{\pgfqpoint{1.498192in}{1.132150in}}%
\pgfpathclose%
\pgfusepath{fill}%
\end{pgfscope}%
\begin{pgfscope}%
\pgfpathrectangle{\pgfqpoint{0.150000in}{0.150000in}}{\pgfqpoint{2.700000in}{1.950000in}}%
\pgfusepath{clip}%
\pgfsetbuttcap%
\pgfsetroundjoin%
\definecolor{currentfill}{rgb}{0.735677,0.768229,0.813802}%
\pgfsetfillcolor{currentfill}%
\pgfsetlinewidth{0.000000pt}%
\definecolor{currentstroke}{rgb}{0.000000,0.000000,0.000000}%
\pgfsetstrokecolor{currentstroke}%
\pgfsetdash{}{0pt}%
\pgfpathmoveto{\pgfqpoint{0.655081in}{1.296351in}}%
\pgfpathlineto{\pgfqpoint{0.693986in}{1.328076in}}%
\pgfpathlineto{\pgfqpoint{0.656299in}{1.359756in}}%
\pgfpathlineto{\pgfqpoint{0.617929in}{1.321843in}}%
\pgfpathclose%
\pgfusepath{fill}%
\end{pgfscope}%
\begin{pgfscope}%
\pgfpathrectangle{\pgfqpoint{0.150000in}{0.150000in}}{\pgfqpoint{2.700000in}{1.950000in}}%
\pgfusepath{clip}%
\pgfsetbuttcap%
\pgfsetroundjoin%
\definecolor{currentfill}{rgb}{0.724571,0.500230,0.517999}%
\pgfsetfillcolor{currentfill}%
\pgfsetlinewidth{0.000000pt}%
\definecolor{currentstroke}{rgb}{0.000000,0.000000,0.000000}%
\pgfsetstrokecolor{currentstroke}%
\pgfsetdash{}{0pt}%
\pgfpathmoveto{\pgfqpoint{1.498117in}{0.447533in}}%
\pgfpathlineto{\pgfqpoint{1.536486in}{0.491398in}}%
\pgfpathlineto{\pgfqpoint{1.498259in}{0.489185in}}%
\pgfpathlineto{\pgfqpoint{1.460067in}{0.445503in}}%
\pgfpathclose%
\pgfusepath{fill}%
\end{pgfscope}%
\begin{pgfscope}%
\pgfpathrectangle{\pgfqpoint{0.150000in}{0.150000in}}{\pgfqpoint{2.700000in}{1.950000in}}%
\pgfusepath{clip}%
\pgfsetbuttcap%
\pgfsetroundjoin%
\definecolor{currentfill}{rgb}{0.729458,0.762776,0.809421}%
\pgfsetfillcolor{currentfill}%
\pgfsetlinewidth{0.000000pt}%
\definecolor{currentstroke}{rgb}{0.000000,0.000000,0.000000}%
\pgfsetstrokecolor{currentstroke}%
\pgfsetdash{}{0pt}%
\pgfpathmoveto{\pgfqpoint{0.884633in}{1.302578in}}%
\pgfpathlineto{\pgfqpoint{0.923759in}{1.328076in}}%
\pgfpathlineto{\pgfqpoint{0.886291in}{1.353509in}}%
\pgfpathlineto{\pgfqpoint{0.846766in}{1.334316in}}%
\pgfpathclose%
\pgfusepath{fill}%
\end{pgfscope}%
\begin{pgfscope}%
\pgfpathrectangle{\pgfqpoint{0.150000in}{0.150000in}}{\pgfqpoint{2.700000in}{1.950000in}}%
\pgfusepath{clip}%
\pgfsetbuttcap%
\pgfsetroundjoin%
\definecolor{currentfill}{rgb}{0.729458,0.762776,0.809421}%
\pgfsetfillcolor{currentfill}%
\pgfsetlinewidth{0.000000pt}%
\definecolor{currentstroke}{rgb}{0.000000,0.000000,0.000000}%
\pgfsetstrokecolor{currentstroke}%
\pgfsetdash{}{0pt}%
\pgfpathmoveto{\pgfqpoint{0.731256in}{1.302578in}}%
\pgfpathlineto{\pgfqpoint{0.770131in}{1.334316in}}%
\pgfpathlineto{\pgfqpoint{0.732837in}{1.359756in}}%
\pgfpathlineto{\pgfqpoint{0.693986in}{1.328076in}}%
\pgfpathclose%
\pgfusepath{fill}%
\end{pgfscope}%
\begin{pgfscope}%
\pgfpathrectangle{\pgfqpoint{0.150000in}{0.150000in}}{\pgfqpoint{2.700000in}{1.950000in}}%
\pgfusepath{clip}%
\pgfsetbuttcap%
\pgfsetroundjoin%
\definecolor{currentfill}{rgb}{0.860064,0.877298,0.901425}%
\pgfsetfillcolor{currentfill}%
\pgfsetlinewidth{0.000000pt}%
\definecolor{currentstroke}{rgb}{0.000000,0.000000,0.000000}%
\pgfsetstrokecolor{currentstroke}%
\pgfsetdash{}{0pt}%
\pgfpathmoveto{\pgfqpoint{1.344569in}{1.156697in}}%
\pgfpathlineto{\pgfqpoint{1.383327in}{1.170106in}}%
\pgfpathlineto{\pgfqpoint{1.345391in}{1.189589in}}%
\pgfpathlineto{\pgfqpoint{1.306479in}{1.182408in}}%
\pgfpathclose%
\pgfusepath{fill}%
\end{pgfscope}%
\begin{pgfscope}%
\pgfpathrectangle{\pgfqpoint{0.150000in}{0.150000in}}{\pgfqpoint{2.700000in}{1.950000in}}%
\pgfusepath{clip}%
\pgfsetbuttcap%
\pgfsetroundjoin%
\definecolor{currentfill}{rgb}{0.828968,0.850031,0.879519}%
\pgfsetfillcolor{currentfill}%
\pgfsetlinewidth{0.000000pt}%
\definecolor{currentstroke}{rgb}{0.000000,0.000000,0.000000}%
\pgfsetstrokecolor{currentstroke}%
\pgfsetdash{}{0pt}%
\pgfpathmoveto{\pgfqpoint{1.229452in}{1.194738in}}%
\pgfpathlineto{\pgfqpoint{1.268486in}{1.208052in}}%
\pgfpathlineto{\pgfqpoint{1.230590in}{1.233632in}}%
\pgfpathlineto{\pgfqpoint{1.191513in}{1.220397in}}%
\pgfpathclose%
\pgfusepath{fill}%
\end{pgfscope}%
\begin{pgfscope}%
\pgfpathrectangle{\pgfqpoint{0.150000in}{0.150000in}}{\pgfqpoint{2.700000in}{1.950000in}}%
\pgfusepath{clip}%
\pgfsetbuttcap%
\pgfsetroundjoin%
\definecolor{currentfill}{rgb}{0.922258,0.931832,0.945236}%
\pgfsetfillcolor{currentfill}%
\pgfsetlinewidth{0.000000pt}%
\definecolor{currentstroke}{rgb}{0.000000,0.000000,0.000000}%
\pgfsetstrokecolor{currentstroke}%
\pgfsetdash{}{0pt}%
\pgfpathmoveto{\pgfqpoint{1.651662in}{1.080590in}}%
\pgfpathlineto{\pgfqpoint{1.689591in}{1.088090in}}%
\pgfpathlineto{\pgfqpoint{1.651236in}{1.119920in}}%
\pgfpathlineto{\pgfqpoint{1.613128in}{1.100292in}}%
\pgfpathclose%
\pgfusepath{fill}%
\end{pgfscope}%
\begin{pgfscope}%
\pgfpathrectangle{\pgfqpoint{0.150000in}{0.150000in}}{\pgfqpoint{2.700000in}{1.950000in}}%
\pgfusepath{clip}%
\pgfsetbuttcap%
\pgfsetroundjoin%
\definecolor{currentfill}{rgb}{0.891161,0.904565,0.923330}%
\pgfsetfillcolor{currentfill}%
\pgfsetlinewidth{0.000000pt}%
\definecolor{currentstroke}{rgb}{0.000000,0.000000,0.000000}%
\pgfsetstrokecolor{currentstroke}%
\pgfsetdash{}{0pt}%
\pgfpathmoveto{\pgfqpoint{1.459711in}{1.118648in}}%
\pgfpathlineto{\pgfqpoint{1.498192in}{1.132150in}}%
\pgfpathlineto{\pgfqpoint{1.460040in}{1.151706in}}%
\pgfpathlineto{\pgfqpoint{1.421470in}{1.144410in}}%
\pgfpathclose%
\pgfusepath{fill}%
\end{pgfscope}%
\begin{pgfscope}%
\pgfpathrectangle{\pgfqpoint{0.150000in}{0.150000in}}{\pgfqpoint{2.700000in}{1.950000in}}%
\pgfusepath{clip}%
\pgfsetbuttcap%
\pgfsetroundjoin%
\definecolor{currentfill}{rgb}{0.887929,0.796645,0.803876}%
\pgfsetfillcolor{currentfill}%
\pgfsetlinewidth{0.000000pt}%
\definecolor{currentstroke}{rgb}{0.000000,0.000000,0.000000}%
\pgfsetstrokecolor{currentstroke}%
\pgfsetdash{}{0pt}%
\pgfpathmoveto{\pgfqpoint{1.882318in}{0.775827in}}%
\pgfpathlineto{\pgfqpoint{1.920254in}{0.802052in}}%
\pgfpathlineto{\pgfqpoint{1.880236in}{0.792629in}}%
\pgfpathlineto{\pgfqpoint{1.842608in}{0.772423in}}%
\pgfpathclose%
\pgfusepath{fill}%
\end{pgfscope}%
\begin{pgfscope}%
\pgfpathrectangle{\pgfqpoint{0.150000in}{0.150000in}}{\pgfqpoint{2.700000in}{1.950000in}}%
\pgfusepath{clip}%
\pgfsetbuttcap%
\pgfsetroundjoin%
\definecolor{currentfill}{rgb}{0.760555,0.790043,0.831327}%
\pgfsetfillcolor{currentfill}%
\pgfsetlinewidth{0.000000pt}%
\definecolor{currentstroke}{rgb}{0.000000,0.000000,0.000000}%
\pgfsetstrokecolor{currentstroke}%
\pgfsetdash{}{0pt}%
\pgfpathmoveto{\pgfqpoint{0.999294in}{1.270795in}}%
\pgfpathlineto{\pgfqpoint{1.038591in}{1.290133in}}%
\pgfpathlineto{\pgfqpoint{1.000973in}{1.315617in}}%
\pgfpathlineto{\pgfqpoint{0.961322in}{1.302578in}}%
\pgfpathclose%
\pgfusepath{fill}%
\end{pgfscope}%
\begin{pgfscope}%
\pgfpathrectangle{\pgfqpoint{0.150000in}{0.150000in}}{\pgfqpoint{2.700000in}{1.950000in}}%
\pgfusepath{clip}%
\pgfsetbuttcap%
\pgfsetroundjoin%
\definecolor{currentfill}{rgb}{0.751164,0.548483,0.564537}%
\pgfsetfillcolor{currentfill}%
\pgfsetlinewidth{0.000000pt}%
\definecolor{currentstroke}{rgb}{0.000000,0.000000,0.000000}%
\pgfsetstrokecolor{currentstroke}%
\pgfsetdash{}{0pt}%
\pgfpathmoveto{\pgfqpoint{1.536486in}{0.491398in}}%
\pgfpathlineto{\pgfqpoint{1.574893in}{0.535305in}}%
\pgfpathlineto{\pgfqpoint{1.536486in}{0.532907in}}%
\pgfpathlineto{\pgfqpoint{1.498259in}{0.489185in}}%
\pgfpathclose%
\pgfusepath{fill}%
\end{pgfscope}%
\begin{pgfscope}%
\pgfpathrectangle{\pgfqpoint{0.150000in}{0.150000in}}{\pgfqpoint{2.700000in}{1.950000in}}%
\pgfusepath{clip}%
\pgfsetbuttcap%
\pgfsetroundjoin%
\definecolor{currentfill}{rgb}{0.723238,0.757322,0.805040}%
\pgfsetfillcolor{currentfill}%
\pgfsetlinewidth{0.000000pt}%
\definecolor{currentstroke}{rgb}{0.000000,0.000000,0.000000}%
\pgfsetstrokecolor{currentstroke}%
\pgfsetdash{}{0pt}%
\pgfpathmoveto{\pgfqpoint{0.807520in}{1.308811in}}%
\pgfpathlineto{\pgfqpoint{0.846766in}{1.334316in}}%
\pgfpathlineto{\pgfqpoint{0.809375in}{1.359756in}}%
\pgfpathlineto{\pgfqpoint{0.770131in}{1.334316in}}%
\pgfpathclose%
\pgfusepath{fill}%
\end{pgfscope}%
\begin{pgfscope}%
\pgfpathrectangle{\pgfqpoint{0.150000in}{0.150000in}}{\pgfqpoint{2.700000in}{1.950000in}}%
\pgfusepath{clip}%
\pgfsetbuttcap%
\pgfsetroundjoin%
\definecolor{currentfill}{rgb}{0.891728,0.803539,0.810524}%
\pgfsetfillcolor{currentfill}%
\pgfsetlinewidth{0.000000pt}%
\definecolor{currentstroke}{rgb}{0.000000,0.000000,0.000000}%
\pgfsetstrokecolor{currentstroke}%
\pgfsetdash{}{0pt}%
\pgfpathmoveto{\pgfqpoint{1.574756in}{0.734449in}}%
\pgfpathlineto{\pgfqpoint{1.612972in}{0.766518in}}%
\pgfpathlineto{\pgfqpoint{1.574859in}{0.840125in}}%
\pgfpathlineto{\pgfqpoint{1.536486in}{0.807999in}}%
\pgfpathclose%
\pgfusepath{fill}%
\end{pgfscope}%
\begin{pgfscope}%
\pgfpathrectangle{\pgfqpoint{0.150000in}{0.150000in}}{\pgfqpoint{2.700000in}{1.950000in}}%
\pgfusepath{clip}%
\pgfsetbuttcap%
\pgfsetroundjoin%
\definecolor{currentfill}{rgb}{0.785432,0.811857,0.848851}%
\pgfsetfillcolor{currentfill}%
\pgfsetlinewidth{0.000000pt}%
\definecolor{currentstroke}{rgb}{0.000000,0.000000,0.000000}%
\pgfsetstrokecolor{currentstroke}%
\pgfsetdash{}{0pt}%
\pgfpathmoveto{\pgfqpoint{1.114114in}{1.238968in}}%
\pgfpathlineto{\pgfqpoint{1.153448in}{1.252181in}}%
\pgfpathlineto{\pgfqpoint{1.115680in}{1.277717in}}%
\pgfpathlineto{\pgfqpoint{1.076304in}{1.264583in}}%
\pgfpathclose%
\pgfusepath{fill}%
\end{pgfscope}%
\begin{pgfscope}%
\pgfpathrectangle{\pgfqpoint{0.150000in}{0.150000in}}{\pgfqpoint{2.700000in}{1.950000in}}%
\pgfusepath{clip}%
\pgfsetbuttcap%
\pgfsetroundjoin%
\definecolor{currentfill}{rgb}{0.853738,0.734605,0.744041}%
\pgfsetfillcolor{currentfill}%
\pgfsetlinewidth{0.000000pt}%
\definecolor{currentstroke}{rgb}{0.000000,0.000000,0.000000}%
\pgfsetstrokecolor{currentstroke}%
\pgfsetdash{}{0pt}%
\pgfpathmoveto{\pgfqpoint{1.728658in}{0.699553in}}%
\pgfpathlineto{\pgfqpoint{1.766663in}{0.719984in}}%
\pgfpathlineto{\pgfqpoint{1.727611in}{0.722688in}}%
\pgfpathlineto{\pgfqpoint{1.689670in}{0.702336in}}%
\pgfpathclose%
\pgfusepath{fill}%
\end{pgfscope}%
\begin{pgfscope}%
\pgfpathrectangle{\pgfqpoint{0.150000in}{0.150000in}}{\pgfqpoint{2.700000in}{1.950000in}}%
\pgfusepath{clip}%
\pgfsetbuttcap%
\pgfsetroundjoin%
\definecolor{currentfill}{rgb}{0.748116,0.779136,0.822564}%
\pgfsetfillcolor{currentfill}%
\pgfsetlinewidth{0.000000pt}%
\definecolor{currentstroke}{rgb}{0.000000,0.000000,0.000000}%
\pgfsetstrokecolor{currentstroke}%
\pgfsetdash{}{0pt}%
\pgfpathmoveto{\pgfqpoint{0.692327in}{1.270795in}}%
\pgfpathlineto{\pgfqpoint{0.731256in}{1.302578in}}%
\pgfpathlineto{\pgfqpoint{0.693986in}{1.328076in}}%
\pgfpathlineto{\pgfqpoint{0.655081in}{1.296351in}}%
\pgfpathclose%
\pgfusepath{fill}%
\end{pgfscope}%
\begin{pgfscope}%
\pgfpathrectangle{\pgfqpoint{0.150000in}{0.150000in}}{\pgfqpoint{2.700000in}{1.950000in}}%
\pgfusepath{clip}%
\pgfsetbuttcap%
\pgfsetroundjoin%
\definecolor{currentfill}{rgb}{0.773958,0.589844,0.604427}%
\pgfsetfillcolor{currentfill}%
\pgfsetlinewidth{0.000000pt}%
\definecolor{currentstroke}{rgb}{0.000000,0.000000,0.000000}%
\pgfsetstrokecolor{currentstroke}%
\pgfsetdash{}{0pt}%
\pgfpathmoveto{\pgfqpoint{1.574893in}{0.535305in}}%
\pgfpathlineto{\pgfqpoint{1.613335in}{0.579254in}}%
\pgfpathlineto{\pgfqpoint{1.574750in}{0.576671in}}%
\pgfpathlineto{\pgfqpoint{1.536486in}{0.532907in}}%
\pgfpathclose%
\pgfusepath{fill}%
\end{pgfscope}%
\begin{pgfscope}%
\pgfpathrectangle{\pgfqpoint{0.150000in}{0.150000in}}{\pgfqpoint{2.700000in}{1.950000in}}%
\pgfusepath{clip}%
\pgfsetbuttcap%
\pgfsetroundjoin%
\definecolor{currentfill}{rgb}{0.956311,0.920726,0.923545}%
\pgfsetfillcolor{currentfill}%
\pgfsetlinewidth{0.000000pt}%
\definecolor{currentstroke}{rgb}{0.000000,0.000000,0.000000}%
\pgfsetstrokecolor{currentstroke}%
\pgfsetdash{}{0pt}%
\pgfpathmoveto{\pgfqpoint{1.613089in}{0.860263in}}%
\pgfpathlineto{\pgfqpoint{1.651178in}{0.880326in}}%
\pgfpathlineto{\pgfqpoint{1.613251in}{0.960327in}}%
\pgfpathlineto{\pgfqpoint{1.574918in}{0.934301in}}%
\pgfpathclose%
\pgfusepath{fill}%
\end{pgfscope}%
\begin{pgfscope}%
\pgfpathrectangle{\pgfqpoint{0.150000in}{0.150000in}}{\pgfqpoint{2.700000in}{1.950000in}}%
\pgfusepath{clip}%
\pgfsetbuttcap%
\pgfsetroundjoin%
\definecolor{currentfill}{rgb}{0.876532,0.775965,0.783931}%
\pgfsetfillcolor{currentfill}%
\pgfsetlinewidth{0.000000pt}%
\definecolor{currentstroke}{rgb}{0.000000,0.000000,0.000000}%
\pgfsetstrokecolor{currentstroke}%
\pgfsetdash{}{0pt}%
\pgfpathmoveto{\pgfqpoint{1.536486in}{0.690604in}}%
\pgfpathlineto{\pgfqpoint{1.574756in}{0.734449in}}%
\pgfpathlineto{\pgfqpoint{1.536486in}{0.807999in}}%
\pgfpathlineto{\pgfqpoint{1.498061in}{0.775827in}}%
\pgfpathclose%
\pgfusepath{fill}%
\end{pgfscope}%
\begin{pgfscope}%
\pgfpathrectangle{\pgfqpoint{0.150000in}{0.150000in}}{\pgfqpoint{2.700000in}{1.950000in}}%
\pgfusepath{clip}%
\pgfsetbuttcap%
\pgfsetroundjoin%
\definecolor{currentfill}{rgb}{0.857537,0.741498,0.750689}%
\pgfsetfillcolor{currentfill}%
\pgfsetlinewidth{0.000000pt}%
\definecolor{currentstroke}{rgb}{0.000000,0.000000,0.000000}%
\pgfsetstrokecolor{currentstroke}%
\pgfsetdash{}{0pt}%
\pgfpathmoveto{\pgfqpoint{1.498231in}{0.652635in}}%
\pgfpathlineto{\pgfqpoint{1.536486in}{0.690604in}}%
\pgfpathlineto{\pgfqpoint{1.498061in}{0.775827in}}%
\pgfpathlineto{\pgfqpoint{1.459627in}{0.737694in}}%
\pgfpathclose%
\pgfusepath{fill}%
\end{pgfscope}%
\begin{pgfscope}%
\pgfpathrectangle{\pgfqpoint{0.150000in}{0.150000in}}{\pgfqpoint{2.700000in}{1.950000in}}%
\pgfusepath{clip}%
\pgfsetbuttcap%
\pgfsetroundjoin%
\definecolor{currentfill}{rgb}{0.909819,0.920925,0.936474}%
\pgfsetfillcolor{currentfill}%
\pgfsetlinewidth{0.000000pt}%
\definecolor{currentstroke}{rgb}{0.000000,0.000000,0.000000}%
\pgfsetstrokecolor{currentstroke}%
\pgfsetdash{}{0pt}%
\pgfpathmoveto{\pgfqpoint{1.574901in}{1.086702in}}%
\pgfpathlineto{\pgfqpoint{1.613128in}{1.100292in}}%
\pgfpathlineto{\pgfqpoint{1.574758in}{1.126032in}}%
\pgfpathlineto{\pgfqpoint{1.536486in}{1.112522in}}%
\pgfpathclose%
\pgfusepath{fill}%
\end{pgfscope}%
\begin{pgfscope}%
\pgfpathrectangle{\pgfqpoint{0.150000in}{0.150000in}}{\pgfqpoint{2.700000in}{1.950000in}}%
\pgfusepath{clip}%
\pgfsetbuttcap%
\pgfsetroundjoin%
\definecolor{currentfill}{rgb}{0.741896,0.773683,0.818183}%
\pgfsetfillcolor{currentfill}%
\pgfsetlinewidth{0.000000pt}%
\definecolor{currentstroke}{rgb}{0.000000,0.000000,0.000000}%
\pgfsetstrokecolor{currentstroke}%
\pgfsetdash{}{0pt}%
\pgfpathmoveto{\pgfqpoint{0.922194in}{1.277014in}}%
\pgfpathlineto{\pgfqpoint{0.961322in}{1.302578in}}%
\pgfpathlineto{\pgfqpoint{0.923759in}{1.328076in}}%
\pgfpathlineto{\pgfqpoint{0.884633in}{1.302578in}}%
\pgfpathclose%
\pgfusepath{fill}%
\end{pgfscope}%
\begin{pgfscope}%
\pgfpathrectangle{\pgfqpoint{0.150000in}{0.150000in}}{\pgfqpoint{2.700000in}{1.950000in}}%
\pgfusepath{clip}%
\pgfsetbuttcap%
\pgfsetroundjoin%
\definecolor{currentfill}{rgb}{0.838542,0.707031,0.717448}%
\pgfsetfillcolor{currentfill}%
\pgfsetlinewidth{0.000000pt}%
\definecolor{currentstroke}{rgb}{0.000000,0.000000,0.000000}%
\pgfsetstrokecolor{currentstroke}%
\pgfsetdash{}{0pt}%
\pgfpathmoveto{\pgfqpoint{1.459967in}{0.614657in}}%
\pgfpathlineto{\pgfqpoint{1.498231in}{0.652635in}}%
\pgfpathlineto{\pgfqpoint{1.459627in}{0.737694in}}%
\pgfpathlineto{\pgfqpoint{1.421116in}{0.705448in}}%
\pgfpathclose%
\pgfusepath{fill}%
\end{pgfscope}%
\begin{pgfscope}%
\pgfpathrectangle{\pgfqpoint{0.150000in}{0.150000in}}{\pgfqpoint{2.700000in}{1.950000in}}%
\pgfusepath{clip}%
\pgfsetbuttcap%
\pgfsetroundjoin%
\definecolor{currentfill}{rgb}{0.800551,0.638097,0.650965}%
\pgfsetfillcolor{currentfill}%
\pgfsetlinewidth{0.000000pt}%
\definecolor{currentstroke}{rgb}{0.000000,0.000000,0.000000}%
\pgfsetstrokecolor{currentstroke}%
\pgfsetdash{}{0pt}%
\pgfpathmoveto{\pgfqpoint{1.613335in}{0.579254in}}%
\pgfpathlineto{\pgfqpoint{1.651815in}{0.623245in}}%
\pgfpathlineto{\pgfqpoint{1.613050in}{0.620477in}}%
\pgfpathlineto{\pgfqpoint{1.574750in}{0.576671in}}%
\pgfpathclose%
\pgfusepath{fill}%
\end{pgfscope}%
\begin{pgfscope}%
\pgfpathrectangle{\pgfqpoint{0.150000in}{0.150000in}}{\pgfqpoint{2.700000in}{1.950000in}}%
\pgfusepath{clip}%
\pgfsetbuttcap%
\pgfsetroundjoin%
\definecolor{currentfill}{rgb}{0.800551,0.638097,0.650965}%
\pgfsetfillcolor{currentfill}%
\pgfsetlinewidth{0.000000pt}%
\definecolor{currentstroke}{rgb}{0.000000,0.000000,0.000000}%
\pgfsetstrokecolor{currentstroke}%
\pgfsetdash{}{0pt}%
\pgfpathmoveto{\pgfqpoint{1.383503in}{0.532907in}}%
\pgfpathlineto{\pgfqpoint{1.421762in}{0.570880in}}%
\pgfpathlineto{\pgfqpoint{1.382643in}{0.667277in}}%
\pgfpathlineto{\pgfqpoint{1.344161in}{0.629097in}}%
\pgfpathclose%
\pgfusepath{fill}%
\end{pgfscope}%
\begin{pgfscope}%
\pgfpathrectangle{\pgfqpoint{0.150000in}{0.150000in}}{\pgfqpoint{2.700000in}{1.950000in}}%
\pgfusepath{clip}%
\pgfsetbuttcap%
\pgfsetroundjoin%
\definecolor{currentfill}{rgb}{0.819547,0.672564,0.684206}%
\pgfsetfillcolor{currentfill}%
\pgfsetlinewidth{0.000000pt}%
\definecolor{currentstroke}{rgb}{0.000000,0.000000,0.000000}%
\pgfsetstrokecolor{currentstroke}%
\pgfsetdash{}{0pt}%
\pgfpathmoveto{\pgfqpoint{1.421762in}{0.570880in}}%
\pgfpathlineto{\pgfqpoint{1.459967in}{0.614657in}}%
\pgfpathlineto{\pgfqpoint{1.421116in}{0.705448in}}%
\pgfpathlineto{\pgfqpoint{1.382643in}{0.667277in}}%
\pgfpathclose%
\pgfusepath{fill}%
\end{pgfscope}%
\begin{pgfscope}%
\pgfpathrectangle{\pgfqpoint{0.150000in}{0.150000in}}{\pgfqpoint{2.700000in}{1.950000in}}%
\pgfusepath{clip}%
\pgfsetbuttcap%
\pgfsetroundjoin%
\definecolor{currentfill}{rgb}{0.741896,0.773683,0.818183}%
\pgfsetfillcolor{currentfill}%
\pgfsetlinewidth{0.000000pt}%
\definecolor{currentstroke}{rgb}{0.000000,0.000000,0.000000}%
\pgfsetstrokecolor{currentstroke}%
\pgfsetdash{}{0pt}%
\pgfpathmoveto{\pgfqpoint{0.768621in}{1.277014in}}%
\pgfpathlineto{\pgfqpoint{0.807520in}{1.308811in}}%
\pgfpathlineto{\pgfqpoint{0.770131in}{1.334316in}}%
\pgfpathlineto{\pgfqpoint{0.731256in}{1.302578in}}%
\pgfpathclose%
\pgfusepath{fill}%
\end{pgfscope}%
\begin{pgfscope}%
\pgfpathrectangle{\pgfqpoint{0.150000in}{0.150000in}}{\pgfqpoint{2.700000in}{1.950000in}}%
\pgfusepath{clip}%
\pgfsetbuttcap%
\pgfsetroundjoin%
\definecolor{currentfill}{rgb}{0.835187,0.855484,0.883900}%
\pgfsetfillcolor{currentfill}%
\pgfsetlinewidth{0.000000pt}%
\definecolor{currentstroke}{rgb}{0.000000,0.000000,0.000000}%
\pgfsetstrokecolor{currentstroke}%
\pgfsetdash{}{0pt}%
\pgfpathmoveto{\pgfqpoint{1.267488in}{1.169014in}}%
\pgfpathlineto{\pgfqpoint{1.306479in}{1.182408in}}%
\pgfpathlineto{\pgfqpoint{1.268486in}{1.208052in}}%
\pgfpathlineto{\pgfqpoint{1.229452in}{1.194738in}}%
\pgfpathclose%
\pgfusepath{fill}%
\end{pgfscope}%
\begin{pgfscope}%
\pgfpathrectangle{\pgfqpoint{0.150000in}{0.150000in}}{\pgfqpoint{2.700000in}{1.950000in}}%
\pgfusepath{clip}%
\pgfsetbuttcap%
\pgfsetroundjoin%
\definecolor{currentfill}{rgb}{0.965794,0.970006,0.975904}%
\pgfsetfillcolor{currentfill}%
\pgfsetlinewidth{0.000000pt}%
\definecolor{currentstroke}{rgb}{0.000000,0.000000,0.000000}%
\pgfsetstrokecolor{currentstroke}%
\pgfsetdash{}{0pt}%
\pgfpathmoveto{\pgfqpoint{1.651419in}{0.980247in}}%
\pgfpathlineto{\pgfqpoint{1.689535in}{1.006133in}}%
\pgfpathlineto{\pgfqpoint{1.651662in}{1.080590in}}%
\pgfpathlineto{\pgfqpoint{1.613414in}{1.060816in}}%
\pgfpathclose%
\pgfusepath{fill}%
\end{pgfscope}%
\begin{pgfscope}%
\pgfpathrectangle{\pgfqpoint{0.150000in}{0.150000in}}{\pgfqpoint{2.700000in}{1.950000in}}%
\pgfusepath{clip}%
\pgfsetbuttcap%
\pgfsetroundjoin%
\definecolor{currentfill}{rgb}{0.772993,0.800950,0.840089}%
\pgfsetfillcolor{currentfill}%
\pgfsetlinewidth{0.000000pt}%
\definecolor{currentstroke}{rgb}{0.000000,0.000000,0.000000}%
\pgfsetstrokecolor{currentstroke}%
\pgfsetdash{}{0pt}%
\pgfpathmoveto{\pgfqpoint{1.037028in}{1.245173in}}%
\pgfpathlineto{\pgfqpoint{1.076304in}{1.264583in}}%
\pgfpathlineto{\pgfqpoint{1.038591in}{1.290133in}}%
\pgfpathlineto{\pgfqpoint{0.999294in}{1.270795in}}%
\pgfpathclose%
\pgfusepath{fill}%
\end{pgfscope}%
\begin{pgfscope}%
\pgfpathrectangle{\pgfqpoint{0.150000in}{0.150000in}}{\pgfqpoint{2.700000in}{1.950000in}}%
\pgfusepath{clip}%
\pgfsetbuttcap%
\pgfsetroundjoin%
\definecolor{currentfill}{rgb}{0.735677,0.768229,0.813802}%
\pgfsetfillcolor{currentfill}%
\pgfsetlinewidth{0.000000pt}%
\definecolor{currentstroke}{rgb}{0.000000,0.000000,0.000000}%
\pgfsetstrokecolor{currentstroke}%
\pgfsetdash{}{0pt}%
\pgfpathmoveto{\pgfqpoint{0.845004in}{1.283241in}}%
\pgfpathlineto{\pgfqpoint{0.884633in}{1.302578in}}%
\pgfpathlineto{\pgfqpoint{0.846766in}{1.334316in}}%
\pgfpathlineto{\pgfqpoint{0.807520in}{1.308811in}}%
\pgfpathclose%
\pgfusepath{fill}%
\end{pgfscope}%
\begin{pgfscope}%
\pgfpathrectangle{\pgfqpoint{0.150000in}{0.150000in}}{\pgfqpoint{2.700000in}{1.950000in}}%
\pgfusepath{clip}%
\pgfsetbuttcap%
\pgfsetroundjoin%
\definecolor{currentfill}{rgb}{0.866284,0.882751,0.905806}%
\pgfsetfillcolor{currentfill}%
\pgfsetlinewidth{0.000000pt}%
\definecolor{currentstroke}{rgb}{0.000000,0.000000,0.000000}%
\pgfsetstrokecolor{currentstroke}%
\pgfsetdash{}{0pt}%
\pgfpathmoveto{\pgfqpoint{1.382756in}{1.130921in}}%
\pgfpathlineto{\pgfqpoint{1.421470in}{1.144410in}}%
\pgfpathlineto{\pgfqpoint{1.383327in}{1.170106in}}%
\pgfpathlineto{\pgfqpoint{1.344569in}{1.156697in}}%
\pgfpathclose%
\pgfusepath{fill}%
\end{pgfscope}%
\begin{pgfscope}%
\pgfpathrectangle{\pgfqpoint{0.150000in}{0.150000in}}{\pgfqpoint{2.700000in}{1.950000in}}%
\pgfusepath{clip}%
\pgfsetbuttcap%
\pgfsetroundjoin%
\definecolor{currentfill}{rgb}{0.797871,0.822763,0.857613}%
\pgfsetfillcolor{currentfill}%
\pgfsetlinewidth{0.000000pt}%
\definecolor{currentstroke}{rgb}{0.000000,0.000000,0.000000}%
\pgfsetstrokecolor{currentstroke}%
\pgfsetdash{}{0pt}%
\pgfpathmoveto{\pgfqpoint{1.152245in}{1.207098in}}%
\pgfpathlineto{\pgfqpoint{1.191513in}{1.220397in}}%
\pgfpathlineto{\pgfqpoint{1.153448in}{1.252181in}}%
\pgfpathlineto{\pgfqpoint{1.114114in}{1.238968in}}%
\pgfpathclose%
\pgfusepath{fill}%
\end{pgfscope}%
\begin{pgfscope}%
\pgfpathrectangle{\pgfqpoint{0.150000in}{0.150000in}}{\pgfqpoint{2.700000in}{1.950000in}}%
\pgfusepath{clip}%
\pgfsetbuttcap%
\pgfsetroundjoin%
\definecolor{currentfill}{rgb}{0.884130,0.789752,0.797227}%
\pgfsetfillcolor{currentfill}%
\pgfsetlinewidth{0.000000pt}%
\definecolor{currentstroke}{rgb}{0.000000,0.000000,0.000000}%
\pgfsetstrokecolor{currentstroke}%
\pgfsetdash{}{0pt}%
\pgfpathmoveto{\pgfqpoint{1.844286in}{0.749536in}}%
\pgfpathlineto{\pgfqpoint{1.882318in}{0.775827in}}%
\pgfpathlineto{\pgfqpoint{1.842608in}{0.772423in}}%
\pgfpathlineto{\pgfqpoint{1.804684in}{0.746237in}}%
\pgfpathclose%
\pgfusepath{fill}%
\end{pgfscope}%
\begin{pgfscope}%
\pgfpathrectangle{\pgfqpoint{0.150000in}{0.150000in}}{\pgfqpoint{2.700000in}{1.950000in}}%
\pgfusepath{clip}%
\pgfsetbuttcap%
\pgfsetroundjoin%
\definecolor{currentfill}{rgb}{0.827145,0.686351,0.697503}%
\pgfsetfillcolor{currentfill}%
\pgfsetlinewidth{0.000000pt}%
\definecolor{currentstroke}{rgb}{0.000000,0.000000,0.000000}%
\pgfsetstrokecolor{currentstroke}%
\pgfsetdash{}{0pt}%
\pgfpathmoveto{\pgfqpoint{1.651815in}{0.623245in}}%
\pgfpathlineto{\pgfqpoint{1.690330in}{0.667277in}}%
\pgfpathlineto{\pgfqpoint{1.651387in}{0.664323in}}%
\pgfpathlineto{\pgfqpoint{1.613050in}{0.620477in}}%
\pgfpathclose%
\pgfusepath{fill}%
\end{pgfscope}%
\begin{pgfscope}%
\pgfpathrectangle{\pgfqpoint{0.150000in}{0.150000in}}{\pgfqpoint{2.700000in}{1.950000in}}%
\pgfusepath{clip}%
\pgfsetbuttcap%
\pgfsetroundjoin%
\definecolor{currentfill}{rgb}{0.846140,0.720818,0.730744}%
\pgfsetfillcolor{currentfill}%
\pgfsetlinewidth{0.000000pt}%
\definecolor{currentstroke}{rgb}{0.000000,0.000000,0.000000}%
\pgfsetstrokecolor{currentstroke}%
\pgfsetdash{}{0pt}%
\pgfpathmoveto{\pgfqpoint{1.690330in}{0.667277in}}%
\pgfpathlineto{\pgfqpoint{1.728658in}{0.699553in}}%
\pgfpathlineto{\pgfqpoint{1.689670in}{0.702336in}}%
\pgfpathlineto{\pgfqpoint{1.651387in}{0.664323in}}%
\pgfpathclose%
\pgfusepath{fill}%
\end{pgfscope}%
\begin{pgfscope}%
\pgfpathrectangle{\pgfqpoint{0.150000in}{0.150000in}}{\pgfqpoint{2.700000in}{1.950000in}}%
\pgfusepath{clip}%
\pgfsetbuttcap%
\pgfsetroundjoin%
\definecolor{currentfill}{rgb}{0.891161,0.904565,0.923330}%
\pgfsetfillcolor{currentfill}%
\pgfsetlinewidth{0.000000pt}%
\definecolor{currentstroke}{rgb}{0.000000,0.000000,0.000000}%
\pgfsetstrokecolor{currentstroke}%
\pgfsetdash{}{0pt}%
\pgfpathmoveto{\pgfqpoint{1.498027in}{1.098946in}}%
\pgfpathlineto{\pgfqpoint{1.536486in}{1.112522in}}%
\pgfpathlineto{\pgfqpoint{1.498192in}{1.132150in}}%
\pgfpathlineto{\pgfqpoint{1.459711in}{1.118648in}}%
\pgfpathclose%
\pgfusepath{fill}%
\end{pgfscope}%
\begin{pgfscope}%
\pgfpathrectangle{\pgfqpoint{0.150000in}{0.150000in}}{\pgfqpoint{2.700000in}{1.950000in}}%
\pgfusepath{clip}%
\pgfsetbuttcap%
\pgfsetroundjoin%
\definecolor{currentfill}{rgb}{0.760555,0.790043,0.831327}%
\pgfsetfillcolor{currentfill}%
\pgfsetlinewidth{0.000000pt}%
\definecolor{currentstroke}{rgb}{0.000000,0.000000,0.000000}%
\pgfsetstrokecolor{currentstroke}%
\pgfsetdash{}{0pt}%
\pgfpathmoveto{\pgfqpoint{0.729668in}{1.245173in}}%
\pgfpathlineto{\pgfqpoint{0.768621in}{1.277014in}}%
\pgfpathlineto{\pgfqpoint{0.731256in}{1.302578in}}%
\pgfpathlineto{\pgfqpoint{0.692327in}{1.270795in}}%
\pgfpathclose%
\pgfusepath{fill}%
\end{pgfscope}%
\begin{pgfscope}%
\pgfpathrectangle{\pgfqpoint{0.150000in}{0.150000in}}{\pgfqpoint{2.700000in}{1.950000in}}%
\pgfusepath{clip}%
\pgfsetbuttcap%
\pgfsetroundjoin%
\definecolor{currentfill}{rgb}{0.760555,0.790043,0.831327}%
\pgfsetfillcolor{currentfill}%
\pgfsetlinewidth{0.000000pt}%
\definecolor{currentstroke}{rgb}{0.000000,0.000000,0.000000}%
\pgfsetstrokecolor{currentstroke}%
\pgfsetdash{}{0pt}%
\pgfpathmoveto{\pgfqpoint{0.959851in}{1.251385in}}%
\pgfpathlineto{\pgfqpoint{0.999294in}{1.270795in}}%
\pgfpathlineto{\pgfqpoint{0.961322in}{1.302578in}}%
\pgfpathlineto{\pgfqpoint{0.922194in}{1.277014in}}%
\pgfpathclose%
\pgfusepath{fill}%
\end{pgfscope}%
\begin{pgfscope}%
\pgfpathrectangle{\pgfqpoint{0.150000in}{0.150000in}}{\pgfqpoint{2.700000in}{1.950000in}}%
\pgfusepath{clip}%
\pgfsetbuttcap%
\pgfsetroundjoin%
\definecolor{currentfill}{rgb}{0.916039,0.926379,0.940855}%
\pgfsetfillcolor{currentfill}%
\pgfsetlinewidth{0.000000pt}%
\definecolor{currentstroke}{rgb}{0.000000,0.000000,0.000000}%
\pgfsetstrokecolor{currentstroke}%
\pgfsetdash{}{0pt}%
\pgfpathmoveto{\pgfqpoint{1.613414in}{1.060816in}}%
\pgfpathlineto{\pgfqpoint{1.651662in}{1.080590in}}%
\pgfpathlineto{\pgfqpoint{1.613128in}{1.100292in}}%
\pgfpathlineto{\pgfqpoint{1.574901in}{1.086702in}}%
\pgfpathclose%
\pgfusepath{fill}%
\end{pgfscope}%
\begin{pgfscope}%
\pgfpathrectangle{\pgfqpoint{0.150000in}{0.150000in}}{\pgfqpoint{2.700000in}{1.950000in}}%
\pgfusepath{clip}%
\pgfsetbuttcap%
\pgfsetroundjoin%
\definecolor{currentfill}{rgb}{0.754335,0.784589,0.826945}%
\pgfsetfillcolor{currentfill}%
\pgfsetlinewidth{0.000000pt}%
\definecolor{currentstroke}{rgb}{0.000000,0.000000,0.000000}%
\pgfsetstrokecolor{currentstroke}%
\pgfsetdash{}{0pt}%
\pgfpathmoveto{\pgfqpoint{0.882967in}{1.251385in}}%
\pgfpathlineto{\pgfqpoint{0.922194in}{1.277014in}}%
\pgfpathlineto{\pgfqpoint{0.884633in}{1.302578in}}%
\pgfpathlineto{\pgfqpoint{0.845004in}{1.283241in}}%
\pgfpathclose%
\pgfusepath{fill}%
\end{pgfscope}%
\begin{pgfscope}%
\pgfpathrectangle{\pgfqpoint{0.150000in}{0.150000in}}{\pgfqpoint{2.700000in}{1.950000in}}%
\pgfusepath{clip}%
\pgfsetbuttcap%
\pgfsetroundjoin%
\definecolor{currentfill}{rgb}{0.816529,0.839124,0.870757}%
\pgfsetfillcolor{currentfill}%
\pgfsetlinewidth{0.000000pt}%
\definecolor{currentstroke}{rgb}{0.000000,0.000000,0.000000}%
\pgfsetstrokecolor{currentstroke}%
\pgfsetdash{}{0pt}%
\pgfpathmoveto{\pgfqpoint{1.190227in}{1.181359in}}%
\pgfpathlineto{\pgfqpoint{1.229452in}{1.194738in}}%
\pgfpathlineto{\pgfqpoint{1.191513in}{1.220397in}}%
\pgfpathlineto{\pgfqpoint{1.152245in}{1.207098in}}%
\pgfpathclose%
\pgfusepath{fill}%
\end{pgfscope}%
\begin{pgfscope}%
\pgfpathrectangle{\pgfqpoint{0.150000in}{0.150000in}}{\pgfqpoint{2.700000in}{1.950000in}}%
\pgfusepath{clip}%
\pgfsetbuttcap%
\pgfsetroundjoin%
\definecolor{currentfill}{rgb}{0.952512,0.913833,0.916896}%
\pgfsetfillcolor{currentfill}%
\pgfsetlinewidth{0.000000pt}%
\definecolor{currentstroke}{rgb}{0.000000,0.000000,0.000000}%
\pgfsetstrokecolor{currentstroke}%
\pgfsetdash{}{0pt}%
\pgfpathmoveto{\pgfqpoint{1.574859in}{0.840125in}}%
\pgfpathlineto{\pgfqpoint{1.613089in}{0.860263in}}%
\pgfpathlineto{\pgfqpoint{1.574918in}{0.934301in}}%
\pgfpathlineto{\pgfqpoint{1.536486in}{0.914233in}}%
\pgfpathclose%
\pgfusepath{fill}%
\end{pgfscope}%
\begin{pgfscope}%
\pgfpathrectangle{\pgfqpoint{0.150000in}{0.150000in}}{\pgfqpoint{2.700000in}{1.950000in}}%
\pgfusepath{clip}%
\pgfsetbuttcap%
\pgfsetroundjoin%
\definecolor{currentfill}{rgb}{0.748116,0.779136,0.822564}%
\pgfsetfillcolor{currentfill}%
\pgfsetlinewidth{0.000000pt}%
\definecolor{currentstroke}{rgb}{0.000000,0.000000,0.000000}%
\pgfsetstrokecolor{currentstroke}%
\pgfsetdash{}{0pt}%
\pgfpathmoveto{\pgfqpoint{0.806082in}{1.251385in}}%
\pgfpathlineto{\pgfqpoint{0.845004in}{1.283241in}}%
\pgfpathlineto{\pgfqpoint{0.807520in}{1.308811in}}%
\pgfpathlineto{\pgfqpoint{0.768621in}{1.277014in}}%
\pgfpathclose%
\pgfusepath{fill}%
\end{pgfscope}%
\begin{pgfscope}%
\pgfpathrectangle{\pgfqpoint{0.150000in}{0.150000in}}{\pgfqpoint{2.700000in}{1.950000in}}%
\pgfusepath{clip}%
\pgfsetbuttcap%
\pgfsetroundjoin%
\definecolor{currentfill}{rgb}{0.847626,0.866391,0.892662}%
\pgfsetfillcolor{currentfill}%
\pgfsetlinewidth{0.000000pt}%
\definecolor{currentstroke}{rgb}{0.000000,0.000000,0.000000}%
\pgfsetstrokecolor{currentstroke}%
\pgfsetdash{}{0pt}%
\pgfpathmoveto{\pgfqpoint{1.305621in}{1.143223in}}%
\pgfpathlineto{\pgfqpoint{1.344569in}{1.156697in}}%
\pgfpathlineto{\pgfqpoint{1.306479in}{1.182408in}}%
\pgfpathlineto{\pgfqpoint{1.267488in}{1.169014in}}%
\pgfpathclose%
\pgfusepath{fill}%
\end{pgfscope}%
\begin{pgfscope}%
\pgfpathrectangle{\pgfqpoint{0.150000in}{0.150000in}}{\pgfqpoint{2.700000in}{1.950000in}}%
\pgfusepath{clip}%
\pgfsetbuttcap%
\pgfsetroundjoin%
\definecolor{currentfill}{rgb}{0.779213,0.806403,0.844470}%
\pgfsetfillcolor{currentfill}%
\pgfsetlinewidth{0.000000pt}%
\definecolor{currentstroke}{rgb}{0.000000,0.000000,0.000000}%
\pgfsetstrokecolor{currentstroke}%
\pgfsetdash{}{0pt}%
\pgfpathmoveto{\pgfqpoint{1.074858in}{1.219486in}}%
\pgfpathlineto{\pgfqpoint{1.114114in}{1.238968in}}%
\pgfpathlineto{\pgfqpoint{1.076304in}{1.264583in}}%
\pgfpathlineto{\pgfqpoint{1.037028in}{1.245173in}}%
\pgfpathclose%
\pgfusepath{fill}%
\end{pgfscope}%
\begin{pgfscope}%
\pgfpathrectangle{\pgfqpoint{0.150000in}{0.150000in}}{\pgfqpoint{2.700000in}{1.950000in}}%
\pgfusepath{clip}%
\pgfsetbuttcap%
\pgfsetroundjoin%
\definecolor{currentfill}{rgb}{0.872503,0.888205,0.910187}%
\pgfsetfillcolor{currentfill}%
\pgfsetlinewidth{0.000000pt}%
\definecolor{currentstroke}{rgb}{0.000000,0.000000,0.000000}%
\pgfsetstrokecolor{currentstroke}%
\pgfsetdash{}{0pt}%
\pgfpathmoveto{\pgfqpoint{1.420974in}{1.111219in}}%
\pgfpathlineto{\pgfqpoint{1.459711in}{1.118648in}}%
\pgfpathlineto{\pgfqpoint{1.421470in}{1.144410in}}%
\pgfpathlineto{\pgfqpoint{1.382756in}{1.130921in}}%
\pgfpathclose%
\pgfusepath{fill}%
\end{pgfscope}%
\begin{pgfscope}%
\pgfpathrectangle{\pgfqpoint{0.150000in}{0.150000in}}{\pgfqpoint{2.700000in}{1.950000in}}%
\pgfusepath{clip}%
\pgfsetbuttcap%
\pgfsetroundjoin%
\definecolor{currentfill}{rgb}{0.880331,0.782858,0.790579}%
\pgfsetfillcolor{currentfill}%
\pgfsetlinewidth{0.000000pt}%
\definecolor{currentstroke}{rgb}{0.000000,0.000000,0.000000}%
\pgfsetstrokecolor{currentstroke}%
\pgfsetdash{}{0pt}%
\pgfpathmoveto{\pgfqpoint{1.806156in}{0.723177in}}%
\pgfpathlineto{\pgfqpoint{1.844286in}{0.749536in}}%
\pgfpathlineto{\pgfqpoint{1.804684in}{0.746237in}}%
\pgfpathlineto{\pgfqpoint{1.766663in}{0.719984in}}%
\pgfpathclose%
\pgfusepath{fill}%
\end{pgfscope}%
\begin{pgfscope}%
\pgfpathrectangle{\pgfqpoint{0.150000in}{0.150000in}}{\pgfqpoint{2.700000in}{1.950000in}}%
\pgfusepath{clip}%
\pgfsetbuttcap%
\pgfsetroundjoin%
\definecolor{currentfill}{rgb}{0.972013,0.975460,0.980285}%
\pgfsetfillcolor{currentfill}%
\pgfsetlinewidth{0.000000pt}%
\definecolor{currentstroke}{rgb}{0.000000,0.000000,0.000000}%
\pgfsetstrokecolor{currentstroke}%
\pgfsetdash{}{0pt}%
\pgfpathmoveto{\pgfqpoint{1.613251in}{0.960327in}}%
\pgfpathlineto{\pgfqpoint{1.651419in}{0.980247in}}%
\pgfpathlineto{\pgfqpoint{1.613414in}{1.060816in}}%
\pgfpathlineto{\pgfqpoint{1.575022in}{1.040967in}}%
\pgfpathclose%
\pgfusepath{fill}%
\end{pgfscope}%
\begin{pgfscope}%
\pgfpathrectangle{\pgfqpoint{0.150000in}{0.150000in}}{\pgfqpoint{2.700000in}{1.950000in}}%
\pgfusepath{clip}%
\pgfsetbuttcap%
\pgfsetroundjoin%
\definecolor{currentfill}{rgb}{0.772993,0.800950,0.840089}%
\pgfsetfillcolor{currentfill}%
\pgfsetlinewidth{0.000000pt}%
\definecolor{currentstroke}{rgb}{0.000000,0.000000,0.000000}%
\pgfsetstrokecolor{currentstroke}%
\pgfsetdash{}{0pt}%
\pgfpathmoveto{\pgfqpoint{0.767105in}{1.219486in}}%
\pgfpathlineto{\pgfqpoint{0.806082in}{1.251385in}}%
\pgfpathlineto{\pgfqpoint{0.768621in}{1.277014in}}%
\pgfpathlineto{\pgfqpoint{0.729668in}{1.245173in}}%
\pgfpathclose%
\pgfusepath{fill}%
\end{pgfscope}%
\begin{pgfscope}%
\pgfpathrectangle{\pgfqpoint{0.150000in}{0.150000in}}{\pgfqpoint{2.700000in}{1.950000in}}%
\pgfusepath{clip}%
\pgfsetbuttcap%
\pgfsetroundjoin%
\definecolor{currentfill}{rgb}{0.772993,0.800950,0.840089}%
\pgfsetfillcolor{currentfill}%
\pgfsetlinewidth{0.000000pt}%
\definecolor{currentstroke}{rgb}{0.000000,0.000000,0.000000}%
\pgfsetstrokecolor{currentstroke}%
\pgfsetdash{}{0pt}%
\pgfpathmoveto{\pgfqpoint{0.997605in}{1.225691in}}%
\pgfpathlineto{\pgfqpoint{1.037028in}{1.245173in}}%
\pgfpathlineto{\pgfqpoint{0.999294in}{1.270795in}}%
\pgfpathlineto{\pgfqpoint{0.959851in}{1.251385in}}%
\pgfpathclose%
\pgfusepath{fill}%
\end{pgfscope}%
\begin{pgfscope}%
\pgfpathrectangle{\pgfqpoint{0.150000in}{0.150000in}}{\pgfqpoint{2.700000in}{1.950000in}}%
\pgfusepath{clip}%
\pgfsetbuttcap%
\pgfsetroundjoin%
\definecolor{currentfill}{rgb}{0.766774,0.795496,0.835708}%
\pgfsetfillcolor{currentfill}%
\pgfsetlinewidth{0.000000pt}%
\definecolor{currentstroke}{rgb}{0.000000,0.000000,0.000000}%
\pgfsetstrokecolor{currentstroke}%
\pgfsetdash{}{0pt}%
\pgfpathmoveto{\pgfqpoint{0.920622in}{1.225691in}}%
\pgfpathlineto{\pgfqpoint{0.959851in}{1.251385in}}%
\pgfpathlineto{\pgfqpoint{0.922194in}{1.277014in}}%
\pgfpathlineto{\pgfqpoint{0.882967in}{1.251385in}}%
\pgfpathclose%
\pgfusepath{fill}%
\end{pgfscope}%
\begin{pgfscope}%
\pgfpathrectangle{\pgfqpoint{0.150000in}{0.150000in}}{\pgfqpoint{2.700000in}{1.950000in}}%
\pgfusepath{clip}%
\pgfsetbuttcap%
\pgfsetroundjoin%
\definecolor{currentfill}{rgb}{0.897381,0.910018,0.927711}%
\pgfsetfillcolor{currentfill}%
\pgfsetlinewidth{0.000000pt}%
\definecolor{currentstroke}{rgb}{0.000000,0.000000,0.000000}%
\pgfsetstrokecolor{currentstroke}%
\pgfsetdash{}{0pt}%
\pgfpathmoveto{\pgfqpoint{1.536486in}{1.079171in}}%
\pgfpathlineto{\pgfqpoint{1.574901in}{1.086702in}}%
\pgfpathlineto{\pgfqpoint{1.536486in}{1.112522in}}%
\pgfpathlineto{\pgfqpoint{1.498027in}{1.098946in}}%
\pgfpathclose%
\pgfusepath{fill}%
\end{pgfscope}%
\begin{pgfscope}%
\pgfpathrectangle{\pgfqpoint{0.150000in}{0.150000in}}{\pgfqpoint{2.700000in}{1.950000in}}%
\pgfusepath{clip}%
\pgfsetbuttcap%
\pgfsetroundjoin%
\definecolor{currentfill}{rgb}{0.760555,0.790043,0.831327}%
\pgfsetfillcolor{currentfill}%
\pgfsetlinewidth{0.000000pt}%
\definecolor{currentstroke}{rgb}{0.000000,0.000000,0.000000}%
\pgfsetstrokecolor{currentstroke}%
\pgfsetdash{}{0pt}%
\pgfpathmoveto{\pgfqpoint{0.843638in}{1.225691in}}%
\pgfpathlineto{\pgfqpoint{0.882967in}{1.251385in}}%
\pgfpathlineto{\pgfqpoint{0.845004in}{1.283241in}}%
\pgfpathlineto{\pgfqpoint{0.806082in}{1.251385in}}%
\pgfpathclose%
\pgfusepath{fill}%
\end{pgfscope}%
\begin{pgfscope}%
\pgfpathrectangle{\pgfqpoint{0.150000in}{0.150000in}}{\pgfqpoint{2.700000in}{1.950000in}}%
\pgfusepath{clip}%
\pgfsetbuttcap%
\pgfsetroundjoin%
\definecolor{currentfill}{rgb}{0.747365,0.541590,0.557889}%
\pgfsetfillcolor{currentfill}%
\pgfsetlinewidth{0.000000pt}%
\definecolor{currentstroke}{rgb}{0.000000,0.000000,0.000000}%
\pgfsetstrokecolor{currentstroke}%
\pgfsetdash{}{0pt}%
\pgfpathmoveto{\pgfqpoint{1.536486in}{0.443822in}}%
\pgfpathlineto{\pgfqpoint{1.575014in}{0.487845in}}%
\pgfpathlineto{\pgfqpoint{1.536486in}{0.491398in}}%
\pgfpathlineto{\pgfqpoint{1.498117in}{0.447533in}}%
\pgfpathclose%
\pgfusepath{fill}%
\end{pgfscope}%
\begin{pgfscope}%
\pgfpathrectangle{\pgfqpoint{0.150000in}{0.150000in}}{\pgfqpoint{2.700000in}{1.950000in}}%
\pgfusepath{clip}%
\pgfsetbuttcap%
\pgfsetroundjoin%
\definecolor{currentfill}{rgb}{0.828968,0.850031,0.879519}%
\pgfsetfillcolor{currentfill}%
\pgfsetlinewidth{0.000000pt}%
\definecolor{currentstroke}{rgb}{0.000000,0.000000,0.000000}%
\pgfsetstrokecolor{currentstroke}%
\pgfsetdash{}{0pt}%
\pgfpathmoveto{\pgfqpoint{1.228306in}{1.155554in}}%
\pgfpathlineto{\pgfqpoint{1.267488in}{1.169014in}}%
\pgfpathlineto{\pgfqpoint{1.229452in}{1.194738in}}%
\pgfpathlineto{\pgfqpoint{1.190227in}{1.181359in}}%
\pgfpathclose%
\pgfusepath{fill}%
\end{pgfscope}%
\begin{pgfscope}%
\pgfpathrectangle{\pgfqpoint{0.150000in}{0.150000in}}{\pgfqpoint{2.700000in}{1.950000in}}%
\pgfusepath{clip}%
\pgfsetbuttcap%
\pgfsetroundjoin%
\definecolor{currentfill}{rgb}{0.791651,0.817310,0.853232}%
\pgfsetfillcolor{currentfill}%
\pgfsetlinewidth{0.000000pt}%
\definecolor{currentstroke}{rgb}{0.000000,0.000000,0.000000}%
\pgfsetstrokecolor{currentstroke}%
\pgfsetdash{}{0pt}%
\pgfpathmoveto{\pgfqpoint{1.112785in}{1.193733in}}%
\pgfpathlineto{\pgfqpoint{1.152245in}{1.207098in}}%
\pgfpathlineto{\pgfqpoint{1.114114in}{1.238968in}}%
\pgfpathlineto{\pgfqpoint{1.074858in}{1.219486in}}%
\pgfpathclose%
\pgfusepath{fill}%
\end{pgfscope}%
\begin{pgfscope}%
\pgfpathrectangle{\pgfqpoint{0.150000in}{0.150000in}}{\pgfqpoint{2.700000in}{1.950000in}}%
\pgfusepath{clip}%
\pgfsetbuttcap%
\pgfsetroundjoin%
\definecolor{currentfill}{rgb}{0.860064,0.877298,0.901425}%
\pgfsetfillcolor{currentfill}%
\pgfsetlinewidth{0.000000pt}%
\definecolor{currentstroke}{rgb}{0.000000,0.000000,0.000000}%
\pgfsetstrokecolor{currentstroke}%
\pgfsetdash{}{0pt}%
\pgfpathmoveto{\pgfqpoint{1.343852in}{1.117367in}}%
\pgfpathlineto{\pgfqpoint{1.382756in}{1.130921in}}%
\pgfpathlineto{\pgfqpoint{1.344569in}{1.156697in}}%
\pgfpathlineto{\pgfqpoint{1.305621in}{1.143223in}}%
\pgfpathclose%
\pgfusepath{fill}%
\end{pgfscope}%
\begin{pgfscope}%
\pgfpathrectangle{\pgfqpoint{0.150000in}{0.150000in}}{\pgfqpoint{2.700000in}{1.950000in}}%
\pgfusepath{clip}%
\pgfsetbuttcap%
\pgfsetroundjoin%
\definecolor{currentfill}{rgb}{0.948713,0.906939,0.910248}%
\pgfsetfillcolor{currentfill}%
\pgfsetlinewidth{0.000000pt}%
\definecolor{currentstroke}{rgb}{0.000000,0.000000,0.000000}%
\pgfsetstrokecolor{currentstroke}%
\pgfsetdash{}{0pt}%
\pgfpathmoveto{\pgfqpoint{1.536486in}{0.807999in}}%
\pgfpathlineto{\pgfqpoint{1.574859in}{0.840125in}}%
\pgfpathlineto{\pgfqpoint{1.536486in}{0.914233in}}%
\pgfpathlineto{\pgfqpoint{1.497934in}{0.888066in}}%
\pgfpathclose%
\pgfusepath{fill}%
\end{pgfscope}%
\begin{pgfscope}%
\pgfpathrectangle{\pgfqpoint{0.150000in}{0.150000in}}{\pgfqpoint{2.700000in}{1.950000in}}%
\pgfusepath{clip}%
\pgfsetbuttcap%
\pgfsetroundjoin%
\definecolor{currentfill}{rgb}{0.773958,0.589844,0.604427}%
\pgfsetfillcolor{currentfill}%
\pgfsetlinewidth{0.000000pt}%
\definecolor{currentstroke}{rgb}{0.000000,0.000000,0.000000}%
\pgfsetstrokecolor{currentstroke}%
\pgfsetdash{}{0pt}%
\pgfpathmoveto{\pgfqpoint{1.575014in}{0.487845in}}%
\pgfpathlineto{\pgfqpoint{1.613577in}{0.531909in}}%
\pgfpathlineto{\pgfqpoint{1.574893in}{0.535305in}}%
\pgfpathlineto{\pgfqpoint{1.536486in}{0.491398in}}%
\pgfpathclose%
\pgfusepath{fill}%
\end{pgfscope}%
\begin{pgfscope}%
\pgfpathrectangle{\pgfqpoint{0.150000in}{0.150000in}}{\pgfqpoint{2.700000in}{1.950000in}}%
\pgfusepath{clip}%
\pgfsetbuttcap%
\pgfsetroundjoin%
\definecolor{currentfill}{rgb}{0.872733,0.769072,0.777282}%
\pgfsetfillcolor{currentfill}%
\pgfsetlinewidth{0.000000pt}%
\definecolor{currentstroke}{rgb}{0.000000,0.000000,0.000000}%
\pgfsetstrokecolor{currentstroke}%
\pgfsetdash{}{0pt}%
\pgfpathmoveto{\pgfqpoint{1.767928in}{0.696750in}}%
\pgfpathlineto{\pgfqpoint{1.806156in}{0.723177in}}%
\pgfpathlineto{\pgfqpoint{1.766663in}{0.719984in}}%
\pgfpathlineto{\pgfqpoint{1.728658in}{0.699553in}}%
\pgfpathclose%
\pgfusepath{fill}%
\end{pgfscope}%
\begin{pgfscope}%
\pgfpathrectangle{\pgfqpoint{0.150000in}{0.150000in}}{\pgfqpoint{2.700000in}{1.950000in}}%
\pgfusepath{clip}%
\pgfsetbuttcap%
\pgfsetroundjoin%
\definecolor{currentfill}{rgb}{0.878722,0.893658,0.914568}%
\pgfsetfillcolor{currentfill}%
\pgfsetlinewidth{0.000000pt}%
\definecolor{currentstroke}{rgb}{0.000000,0.000000,0.000000}%
\pgfsetstrokecolor{currentstroke}%
\pgfsetdash{}{0pt}%
\pgfpathmoveto{\pgfqpoint{1.459379in}{1.085304in}}%
\pgfpathlineto{\pgfqpoint{1.498027in}{1.098946in}}%
\pgfpathlineto{\pgfqpoint{1.459711in}{1.118648in}}%
\pgfpathlineto{\pgfqpoint{1.420974in}{1.111219in}}%
\pgfpathclose%
\pgfusepath{fill}%
\end{pgfscope}%
\begin{pgfscope}%
\pgfpathrectangle{\pgfqpoint{0.150000in}{0.150000in}}{\pgfqpoint{2.700000in}{1.950000in}}%
\pgfusepath{clip}%
\pgfsetbuttcap%
\pgfsetroundjoin%
\definecolor{currentfill}{rgb}{0.779213,0.806403,0.844470}%
\pgfsetfillcolor{currentfill}%
\pgfsetlinewidth{0.000000pt}%
\definecolor{currentstroke}{rgb}{0.000000,0.000000,0.000000}%
\pgfsetstrokecolor{currentstroke}%
\pgfsetdash{}{0pt}%
\pgfpathmoveto{\pgfqpoint{1.035455in}{1.199931in}}%
\pgfpathlineto{\pgfqpoint{1.074858in}{1.219486in}}%
\pgfpathlineto{\pgfqpoint{1.037028in}{1.245173in}}%
\pgfpathlineto{\pgfqpoint{0.997605in}{1.225691in}}%
\pgfpathclose%
\pgfusepath{fill}%
\end{pgfscope}%
\begin{pgfscope}%
\pgfpathrectangle{\pgfqpoint{0.150000in}{0.150000in}}{\pgfqpoint{2.700000in}{1.950000in}}%
\pgfusepath{clip}%
\pgfsetbuttcap%
\pgfsetroundjoin%
\definecolor{currentfill}{rgb}{0.972013,0.975460,0.980285}%
\pgfsetfillcolor{currentfill}%
\pgfsetlinewidth{0.000000pt}%
\definecolor{currentstroke}{rgb}{0.000000,0.000000,0.000000}%
\pgfsetstrokecolor{currentstroke}%
\pgfsetdash{}{0pt}%
\pgfpathmoveto{\pgfqpoint{1.574918in}{0.934301in}}%
\pgfpathlineto{\pgfqpoint{1.613251in}{0.960327in}}%
\pgfpathlineto{\pgfqpoint{1.575022in}{1.040967in}}%
\pgfpathlineto{\pgfqpoint{1.536486in}{1.014940in}}%
\pgfpathclose%
\pgfusepath{fill}%
\end{pgfscope}%
\begin{pgfscope}%
\pgfpathrectangle{\pgfqpoint{0.150000in}{0.150000in}}{\pgfqpoint{2.700000in}{1.950000in}}%
\pgfusepath{clip}%
\pgfsetbuttcap%
\pgfsetroundjoin%
\definecolor{currentfill}{rgb}{0.796752,0.631204,0.644317}%
\pgfsetfillcolor{currentfill}%
\pgfsetlinewidth{0.000000pt}%
\definecolor{currentstroke}{rgb}{0.000000,0.000000,0.000000}%
\pgfsetstrokecolor{currentstroke}%
\pgfsetdash{}{0pt}%
\pgfpathmoveto{\pgfqpoint{1.613577in}{0.531909in}}%
\pgfpathlineto{\pgfqpoint{1.652178in}{0.576016in}}%
\pgfpathlineto{\pgfqpoint{1.613335in}{0.579254in}}%
\pgfpathlineto{\pgfqpoint{1.574893in}{0.535305in}}%
\pgfpathclose%
\pgfusepath{fill}%
\end{pgfscope}%
\begin{pgfscope}%
\pgfpathrectangle{\pgfqpoint{0.150000in}{0.150000in}}{\pgfqpoint{2.700000in}{1.950000in}}%
\pgfusepath{clip}%
\pgfsetbuttcap%
\pgfsetroundjoin%
\definecolor{currentfill}{rgb}{0.779213,0.806403,0.844470}%
\pgfsetfillcolor{currentfill}%
\pgfsetlinewidth{0.000000pt}%
\definecolor{currentstroke}{rgb}{0.000000,0.000000,0.000000}%
\pgfsetstrokecolor{currentstroke}%
\pgfsetdash{}{0pt}%
\pgfpathmoveto{\pgfqpoint{0.804210in}{1.199931in}}%
\pgfpathlineto{\pgfqpoint{0.843638in}{1.225691in}}%
\pgfpathlineto{\pgfqpoint{0.806082in}{1.251385in}}%
\pgfpathlineto{\pgfqpoint{0.767105in}{1.219486in}}%
\pgfpathclose%
\pgfusepath{fill}%
\end{pgfscope}%
\begin{pgfscope}%
\pgfpathrectangle{\pgfqpoint{0.150000in}{0.150000in}}{\pgfqpoint{2.700000in}{1.950000in}}%
\pgfusepath{clip}%
\pgfsetbuttcap%
\pgfsetroundjoin%
\definecolor{currentfill}{rgb}{0.909819,0.920925,0.936474}%
\pgfsetfillcolor{currentfill}%
\pgfsetlinewidth{0.000000pt}%
\definecolor{currentstroke}{rgb}{0.000000,0.000000,0.000000}%
\pgfsetstrokecolor{currentstroke}%
\pgfsetdash{}{0pt}%
\pgfpathmoveto{\pgfqpoint{1.575022in}{1.040967in}}%
\pgfpathlineto{\pgfqpoint{1.613414in}{1.060816in}}%
\pgfpathlineto{\pgfqpoint{1.574901in}{1.086702in}}%
\pgfpathlineto{\pgfqpoint{1.536486in}{1.079171in}}%
\pgfpathclose%
\pgfusepath{fill}%
\end{pgfscope}%
\begin{pgfscope}%
\pgfpathrectangle{\pgfqpoint{0.150000in}{0.150000in}}{\pgfqpoint{2.700000in}{1.950000in}}%
\pgfusepath{clip}%
\pgfsetbuttcap%
\pgfsetroundjoin%
\definecolor{currentfill}{rgb}{0.937316,0.886259,0.890303}%
\pgfsetfillcolor{currentfill}%
\pgfsetlinewidth{0.000000pt}%
\definecolor{currentstroke}{rgb}{0.000000,0.000000,0.000000}%
\pgfsetstrokecolor{currentstroke}%
\pgfsetdash{}{0pt}%
\pgfpathmoveto{\pgfqpoint{1.498061in}{0.775827in}}%
\pgfpathlineto{\pgfqpoint{1.536486in}{0.807999in}}%
\pgfpathlineto{\pgfqpoint{1.497934in}{0.888066in}}%
\pgfpathlineto{\pgfqpoint{1.459328in}{0.855822in}}%
\pgfpathclose%
\pgfusepath{fill}%
\end{pgfscope}%
\begin{pgfscope}%
\pgfpathrectangle{\pgfqpoint{0.150000in}{0.150000in}}{\pgfqpoint{2.700000in}{1.950000in}}%
\pgfusepath{clip}%
\pgfsetbuttcap%
\pgfsetroundjoin%
\definecolor{currentfill}{rgb}{0.772993,0.800950,0.840089}%
\pgfsetfillcolor{currentfill}%
\pgfsetlinewidth{0.000000pt}%
\definecolor{currentstroke}{rgb}{0.000000,0.000000,0.000000}%
\pgfsetstrokecolor{currentstroke}%
\pgfsetdash{}{0pt}%
\pgfpathmoveto{\pgfqpoint{0.958373in}{1.199931in}}%
\pgfpathlineto{\pgfqpoint{0.997605in}{1.225691in}}%
\pgfpathlineto{\pgfqpoint{0.959851in}{1.251385in}}%
\pgfpathlineto{\pgfqpoint{0.920622in}{1.225691in}}%
\pgfpathclose%
\pgfusepath{fill}%
\end{pgfscope}%
\begin{pgfscope}%
\pgfpathrectangle{\pgfqpoint{0.150000in}{0.150000in}}{\pgfqpoint{2.700000in}{1.950000in}}%
\pgfusepath{clip}%
\pgfsetbuttcap%
\pgfsetroundjoin%
\definecolor{currentfill}{rgb}{0.772993,0.800950,0.840089}%
\pgfsetfillcolor{currentfill}%
\pgfsetlinewidth{0.000000pt}%
\definecolor{currentstroke}{rgb}{0.000000,0.000000,0.000000}%
\pgfsetstrokecolor{currentstroke}%
\pgfsetdash{}{0pt}%
\pgfpathmoveto{\pgfqpoint{0.881291in}{1.199931in}}%
\pgfpathlineto{\pgfqpoint{0.920622in}{1.225691in}}%
\pgfpathlineto{\pgfqpoint{0.882967in}{1.251385in}}%
\pgfpathlineto{\pgfqpoint{0.843638in}{1.225691in}}%
\pgfpathclose%
\pgfusepath{fill}%
\end{pgfscope}%
\begin{pgfscope}%
\pgfpathrectangle{\pgfqpoint{0.150000in}{0.150000in}}{\pgfqpoint{2.700000in}{1.950000in}}%
\pgfusepath{clip}%
\pgfsetbuttcap%
\pgfsetroundjoin%
\definecolor{currentfill}{rgb}{0.804090,0.828217,0.861994}%
\pgfsetfillcolor{currentfill}%
\pgfsetlinewidth{0.000000pt}%
\definecolor{currentstroke}{rgb}{0.000000,0.000000,0.000000}%
\pgfsetstrokecolor{currentstroke}%
\pgfsetdash{}{0pt}%
\pgfpathmoveto{\pgfqpoint{1.150809in}{1.167914in}}%
\pgfpathlineto{\pgfqpoint{1.190227in}{1.181359in}}%
\pgfpathlineto{\pgfqpoint{1.152245in}{1.207098in}}%
\pgfpathlineto{\pgfqpoint{1.112785in}{1.193733in}}%
\pgfpathclose%
\pgfusepath{fill}%
\end{pgfscope}%
\begin{pgfscope}%
\pgfpathrectangle{\pgfqpoint{0.150000in}{0.150000in}}{\pgfqpoint{2.700000in}{1.950000in}}%
\pgfusepath{clip}%
\pgfsetbuttcap%
\pgfsetroundjoin%
\definecolor{currentfill}{rgb}{0.835187,0.855484,0.883900}%
\pgfsetfillcolor{currentfill}%
\pgfsetlinewidth{0.000000pt}%
\definecolor{currentstroke}{rgb}{0.000000,0.000000,0.000000}%
\pgfsetstrokecolor{currentstroke}%
\pgfsetdash{}{0pt}%
\pgfpathmoveto{\pgfqpoint{1.266483in}{1.129683in}}%
\pgfpathlineto{\pgfqpoint{1.305621in}{1.143223in}}%
\pgfpathlineto{\pgfqpoint{1.267488in}{1.169014in}}%
\pgfpathlineto{\pgfqpoint{1.228306in}{1.155554in}}%
\pgfpathclose%
\pgfusepath{fill}%
\end{pgfscope}%
\begin{pgfscope}%
\pgfpathrectangle{\pgfqpoint{0.150000in}{0.150000in}}{\pgfqpoint{2.700000in}{1.950000in}}%
\pgfusepath{clip}%
\pgfsetbuttcap%
\pgfsetroundjoin%
\definecolor{currentfill}{rgb}{0.823346,0.679458,0.690855}%
\pgfsetfillcolor{currentfill}%
\pgfsetlinewidth{0.000000pt}%
\definecolor{currentstroke}{rgb}{0.000000,0.000000,0.000000}%
\pgfsetstrokecolor{currentstroke}%
\pgfsetdash{}{0pt}%
\pgfpathmoveto{\pgfqpoint{1.652178in}{0.576016in}}%
\pgfpathlineto{\pgfqpoint{1.690815in}{0.620164in}}%
\pgfpathlineto{\pgfqpoint{1.651815in}{0.623245in}}%
\pgfpathlineto{\pgfqpoint{1.613335in}{0.579254in}}%
\pgfpathclose%
\pgfusepath{fill}%
\end{pgfscope}%
\begin{pgfscope}%
\pgfpathrectangle{\pgfqpoint{0.150000in}{0.150000in}}{\pgfqpoint{2.700000in}{1.950000in}}%
\pgfusepath{clip}%
\pgfsetbuttcap%
\pgfsetroundjoin%
\definecolor{currentfill}{rgb}{0.922120,0.858686,0.863710}%
\pgfsetfillcolor{currentfill}%
\pgfsetlinewidth{0.000000pt}%
\definecolor{currentstroke}{rgb}{0.000000,0.000000,0.000000}%
\pgfsetstrokecolor{currentstroke}%
\pgfsetdash{}{0pt}%
\pgfpathmoveto{\pgfqpoint{1.459627in}{0.737694in}}%
\pgfpathlineto{\pgfqpoint{1.498061in}{0.775827in}}%
\pgfpathlineto{\pgfqpoint{1.459328in}{0.855822in}}%
\pgfpathlineto{\pgfqpoint{1.420668in}{0.823533in}}%
\pgfpathclose%
\pgfusepath{fill}%
\end{pgfscope}%
\begin{pgfscope}%
\pgfpathrectangle{\pgfqpoint{0.150000in}{0.150000in}}{\pgfqpoint{2.700000in}{1.950000in}}%
\pgfusepath{clip}%
\pgfsetbuttcap%
\pgfsetroundjoin%
\definecolor{currentfill}{rgb}{0.860064,0.877298,0.901425}%
\pgfsetfillcolor{currentfill}%
\pgfsetlinewidth{0.000000pt}%
\definecolor{currentstroke}{rgb}{0.000000,0.000000,0.000000}%
\pgfsetstrokecolor{currentstroke}%
\pgfsetdash{}{0pt}%
\pgfpathmoveto{\pgfqpoint{1.382091in}{1.097591in}}%
\pgfpathlineto{\pgfqpoint{1.420974in}{1.111219in}}%
\pgfpathlineto{\pgfqpoint{1.382756in}{1.130921in}}%
\pgfpathlineto{\pgfqpoint{1.343852in}{1.117367in}}%
\pgfpathclose%
\pgfusepath{fill}%
\end{pgfscope}%
\begin{pgfscope}%
\pgfpathrectangle{\pgfqpoint{0.150000in}{0.150000in}}{\pgfqpoint{2.700000in}{1.950000in}}%
\pgfusepath{clip}%
\pgfsetbuttcap%
\pgfsetroundjoin%
\definecolor{currentfill}{rgb}{0.865135,0.755285,0.763986}%
\pgfsetfillcolor{currentfill}%
\pgfsetlinewidth{0.000000pt}%
\definecolor{currentstroke}{rgb}{0.000000,0.000000,0.000000}%
\pgfsetstrokecolor{currentstroke}%
\pgfsetdash{}{0pt}%
\pgfpathmoveto{\pgfqpoint{1.729489in}{0.664355in}}%
\pgfpathlineto{\pgfqpoint{1.767928in}{0.696750in}}%
\pgfpathlineto{\pgfqpoint{1.728658in}{0.699553in}}%
\pgfpathlineto{\pgfqpoint{1.690330in}{0.667277in}}%
\pgfpathclose%
\pgfusepath{fill}%
\end{pgfscope}%
\begin{pgfscope}%
\pgfpathrectangle{\pgfqpoint{0.150000in}{0.150000in}}{\pgfqpoint{2.700000in}{1.950000in}}%
\pgfusepath{clip}%
\pgfsetbuttcap%
\pgfsetroundjoin%
\definecolor{currentfill}{rgb}{0.846140,0.720818,0.730744}%
\pgfsetfillcolor{currentfill}%
\pgfsetlinewidth{0.000000pt}%
\definecolor{currentstroke}{rgb}{0.000000,0.000000,0.000000}%
\pgfsetstrokecolor{currentstroke}%
\pgfsetdash{}{0pt}%
\pgfpathmoveto{\pgfqpoint{1.690815in}{0.620164in}}%
\pgfpathlineto{\pgfqpoint{1.729489in}{0.664355in}}%
\pgfpathlineto{\pgfqpoint{1.690330in}{0.667277in}}%
\pgfpathlineto{\pgfqpoint{1.651815in}{0.623245in}}%
\pgfpathclose%
\pgfusepath{fill}%
\end{pgfscope}%
\begin{pgfscope}%
\pgfpathrectangle{\pgfqpoint{0.150000in}{0.150000in}}{\pgfqpoint{2.700000in}{1.950000in}}%
\pgfusepath{clip}%
\pgfsetbuttcap%
\pgfsetroundjoin%
\definecolor{currentfill}{rgb}{0.791651,0.817310,0.853232}%
\pgfsetfillcolor{currentfill}%
\pgfsetlinewidth{0.000000pt}%
\definecolor{currentstroke}{rgb}{0.000000,0.000000,0.000000}%
\pgfsetstrokecolor{currentstroke}%
\pgfsetdash{}{0pt}%
\pgfpathmoveto{\pgfqpoint{1.073674in}{1.167914in}}%
\pgfpathlineto{\pgfqpoint{1.112785in}{1.193733in}}%
\pgfpathlineto{\pgfqpoint{1.074858in}{1.219486in}}%
\pgfpathlineto{\pgfqpoint{1.035455in}{1.199931in}}%
\pgfpathclose%
\pgfusepath{fill}%
\end{pgfscope}%
\begin{pgfscope}%
\pgfpathrectangle{\pgfqpoint{0.150000in}{0.150000in}}{\pgfqpoint{2.700000in}{1.950000in}}%
\pgfusepath{clip}%
\pgfsetbuttcap%
\pgfsetroundjoin%
\definecolor{currentfill}{rgb}{0.910723,0.838006,0.843765}%
\pgfsetfillcolor{currentfill}%
\pgfsetlinewidth{0.000000pt}%
\definecolor{currentstroke}{rgb}{0.000000,0.000000,0.000000}%
\pgfsetstrokecolor{currentstroke}%
\pgfsetdash{}{0pt}%
\pgfpathmoveto{\pgfqpoint{1.421116in}{0.705448in}}%
\pgfpathlineto{\pgfqpoint{1.459627in}{0.737694in}}%
\pgfpathlineto{\pgfqpoint{1.420668in}{0.823533in}}%
\pgfpathlineto{\pgfqpoint{1.381954in}{0.791199in}}%
\pgfpathclose%
\pgfusepath{fill}%
\end{pgfscope}%
\begin{pgfscope}%
\pgfpathrectangle{\pgfqpoint{0.150000in}{0.150000in}}{\pgfqpoint{2.700000in}{1.950000in}}%
\pgfusepath{clip}%
\pgfsetbuttcap%
\pgfsetroundjoin%
\definecolor{currentfill}{rgb}{0.984452,0.986366,0.989047}%
\pgfsetfillcolor{currentfill}%
\pgfsetlinewidth{0.000000pt}%
\definecolor{currentstroke}{rgb}{0.000000,0.000000,0.000000}%
\pgfsetstrokecolor{currentstroke}%
\pgfsetdash{}{0pt}%
\pgfpathmoveto{\pgfqpoint{1.536486in}{0.914233in}}%
\pgfpathlineto{\pgfqpoint{1.574918in}{0.934301in}}%
\pgfpathlineto{\pgfqpoint{1.536486in}{1.014940in}}%
\pgfpathlineto{\pgfqpoint{1.497852in}{0.988847in}}%
\pgfpathclose%
\pgfusepath{fill}%
\end{pgfscope}%
\begin{pgfscope}%
\pgfpathrectangle{\pgfqpoint{0.150000in}{0.150000in}}{\pgfqpoint{2.700000in}{1.950000in}}%
\pgfusepath{clip}%
\pgfsetbuttcap%
\pgfsetroundjoin%
\definecolor{currentfill}{rgb}{0.816529,0.839124,0.870757}%
\pgfsetfillcolor{currentfill}%
\pgfsetlinewidth{0.000000pt}%
\definecolor{currentstroke}{rgb}{0.000000,0.000000,0.000000}%
\pgfsetstrokecolor{currentstroke}%
\pgfsetdash{}{0pt}%
\pgfpathmoveto{\pgfqpoint{1.189135in}{1.135852in}}%
\pgfpathlineto{\pgfqpoint{1.228306in}{1.155554in}}%
\pgfpathlineto{\pgfqpoint{1.190227in}{1.181359in}}%
\pgfpathlineto{\pgfqpoint{1.150809in}{1.167914in}}%
\pgfpathclose%
\pgfusepath{fill}%
\end{pgfscope}%
\begin{pgfscope}%
\pgfpathrectangle{\pgfqpoint{0.150000in}{0.150000in}}{\pgfqpoint{2.700000in}{1.950000in}}%
\pgfusepath{clip}%
\pgfsetbuttcap%
\pgfsetroundjoin%
\definecolor{currentfill}{rgb}{0.785432,0.811857,0.848851}%
\pgfsetfillcolor{currentfill}%
\pgfsetlinewidth{0.000000pt}%
\definecolor{currentstroke}{rgb}{0.000000,0.000000,0.000000}%
\pgfsetstrokecolor{currentstroke}%
\pgfsetdash{}{0pt}%
\pgfpathmoveto{\pgfqpoint{0.996222in}{1.174105in}}%
\pgfpathlineto{\pgfqpoint{1.035455in}{1.199931in}}%
\pgfpathlineto{\pgfqpoint{0.997605in}{1.225691in}}%
\pgfpathlineto{\pgfqpoint{0.958373in}{1.199931in}}%
\pgfpathclose%
\pgfusepath{fill}%
\end{pgfscope}%
\begin{pgfscope}%
\pgfpathrectangle{\pgfqpoint{0.150000in}{0.150000in}}{\pgfqpoint{2.700000in}{1.950000in}}%
\pgfusepath{clip}%
\pgfsetbuttcap%
\pgfsetroundjoin%
\definecolor{currentfill}{rgb}{0.785432,0.811857,0.848851}%
\pgfsetfillcolor{currentfill}%
\pgfsetlinewidth{0.000000pt}%
\definecolor{currentstroke}{rgb}{0.000000,0.000000,0.000000}%
\pgfsetstrokecolor{currentstroke}%
\pgfsetdash{}{0pt}%
\pgfpathmoveto{\pgfqpoint{0.919041in}{1.174105in}}%
\pgfpathlineto{\pgfqpoint{0.958373in}{1.199931in}}%
\pgfpathlineto{\pgfqpoint{0.920622in}{1.225691in}}%
\pgfpathlineto{\pgfqpoint{0.881291in}{1.199931in}}%
\pgfpathclose%
\pgfusepath{fill}%
\end{pgfscope}%
\begin{pgfscope}%
\pgfpathrectangle{\pgfqpoint{0.150000in}{0.150000in}}{\pgfqpoint{2.700000in}{1.950000in}}%
\pgfusepath{clip}%
\pgfsetbuttcap%
\pgfsetroundjoin%
\definecolor{currentfill}{rgb}{0.785432,0.811857,0.848851}%
\pgfsetfillcolor{currentfill}%
\pgfsetlinewidth{0.000000pt}%
\definecolor{currentstroke}{rgb}{0.000000,0.000000,0.000000}%
\pgfsetstrokecolor{currentstroke}%
\pgfsetdash{}{0pt}%
\pgfpathmoveto{\pgfqpoint{0.841860in}{1.174105in}}%
\pgfpathlineto{\pgfqpoint{0.881291in}{1.199931in}}%
\pgfpathlineto{\pgfqpoint{0.843638in}{1.225691in}}%
\pgfpathlineto{\pgfqpoint{0.804210in}{1.199931in}}%
\pgfpathclose%
\pgfusepath{fill}%
\end{pgfscope}%
\begin{pgfscope}%
\pgfpathrectangle{\pgfqpoint{0.150000in}{0.150000in}}{\pgfqpoint{2.700000in}{1.950000in}}%
\pgfusepath{clip}%
\pgfsetbuttcap%
\pgfsetroundjoin%
\definecolor{currentfill}{rgb}{0.884942,0.899112,0.918949}%
\pgfsetfillcolor{currentfill}%
\pgfsetlinewidth{0.000000pt}%
\definecolor{currentstroke}{rgb}{0.000000,0.000000,0.000000}%
\pgfsetstrokecolor{currentstroke}%
\pgfsetdash{}{0pt}%
\pgfpathmoveto{\pgfqpoint{1.497861in}{1.065455in}}%
\pgfpathlineto{\pgfqpoint{1.536486in}{1.079171in}}%
\pgfpathlineto{\pgfqpoint{1.498027in}{1.098946in}}%
\pgfpathlineto{\pgfqpoint{1.459379in}{1.085304in}}%
\pgfpathclose%
\pgfusepath{fill}%
\end{pgfscope}%
\begin{pgfscope}%
\pgfpathrectangle{\pgfqpoint{0.150000in}{0.150000in}}{\pgfqpoint{2.700000in}{1.950000in}}%
\pgfusepath{clip}%
\pgfsetbuttcap%
\pgfsetroundjoin%
\definecolor{currentfill}{rgb}{0.847626,0.866391,0.892662}%
\pgfsetfillcolor{currentfill}%
\pgfsetlinewidth{0.000000pt}%
\definecolor{currentstroke}{rgb}{0.000000,0.000000,0.000000}%
\pgfsetstrokecolor{currentstroke}%
\pgfsetdash{}{0pt}%
\pgfpathmoveto{\pgfqpoint{1.304757in}{1.103746in}}%
\pgfpathlineto{\pgfqpoint{1.343852in}{1.117367in}}%
\pgfpathlineto{\pgfqpoint{1.305621in}{1.143223in}}%
\pgfpathlineto{\pgfqpoint{1.266483in}{1.129683in}}%
\pgfpathclose%
\pgfusepath{fill}%
\end{pgfscope}%
\begin{pgfscope}%
\pgfpathrectangle{\pgfqpoint{0.150000in}{0.150000in}}{\pgfqpoint{2.700000in}{1.950000in}}%
\pgfusepath{clip}%
\pgfsetbuttcap%
\pgfsetroundjoin%
\definecolor{currentfill}{rgb}{0.895527,0.810432,0.817172}%
\pgfsetfillcolor{currentfill}%
\pgfsetlinewidth{0.000000pt}%
\definecolor{currentstroke}{rgb}{0.000000,0.000000,0.000000}%
\pgfsetstrokecolor{currentstroke}%
\pgfsetdash{}{0pt}%
\pgfpathmoveto{\pgfqpoint{1.382643in}{0.667277in}}%
\pgfpathlineto{\pgfqpoint{1.421116in}{0.705448in}}%
\pgfpathlineto{\pgfqpoint{1.381954in}{0.791199in}}%
\pgfpathlineto{\pgfqpoint{1.343073in}{0.764786in}}%
\pgfpathclose%
\pgfusepath{fill}%
\end{pgfscope}%
\begin{pgfscope}%
\pgfpathrectangle{\pgfqpoint{0.150000in}{0.150000in}}{\pgfqpoint{2.700000in}{1.950000in}}%
\pgfusepath{clip}%
\pgfsetbuttcap%
\pgfsetroundjoin%
\definecolor{currentfill}{rgb}{0.804090,0.828217,0.861994}%
\pgfsetfillcolor{currentfill}%
\pgfsetlinewidth{0.000000pt}%
\definecolor{currentstroke}{rgb}{0.000000,0.000000,0.000000}%
\pgfsetstrokecolor{currentstroke}%
\pgfsetdash{}{0pt}%
\pgfpathmoveto{\pgfqpoint{1.111697in}{1.142028in}}%
\pgfpathlineto{\pgfqpoint{1.150809in}{1.167914in}}%
\pgfpathlineto{\pgfqpoint{1.112785in}{1.193733in}}%
\pgfpathlineto{\pgfqpoint{1.073674in}{1.167914in}}%
\pgfpathclose%
\pgfusepath{fill}%
\end{pgfscope}%
\begin{pgfscope}%
\pgfpathrectangle{\pgfqpoint{0.150000in}{0.150000in}}{\pgfqpoint{2.700000in}{1.950000in}}%
\pgfusepath{clip}%
\pgfsetbuttcap%
\pgfsetroundjoin%
\definecolor{currentfill}{rgb}{0.903600,0.915472,0.932093}%
\pgfsetfillcolor{currentfill}%
\pgfsetlinewidth{0.000000pt}%
\definecolor{currentstroke}{rgb}{0.000000,0.000000,0.000000}%
\pgfsetstrokecolor{currentstroke}%
\pgfsetdash{}{0pt}%
\pgfpathmoveto{\pgfqpoint{1.536486in}{1.014940in}}%
\pgfpathlineto{\pgfqpoint{1.575022in}{1.040967in}}%
\pgfpathlineto{\pgfqpoint{1.536486in}{1.079171in}}%
\pgfpathlineto{\pgfqpoint{1.497861in}{1.065455in}}%
\pgfpathclose%
\pgfusepath{fill}%
\end{pgfscope}%
\begin{pgfscope}%
\pgfpathrectangle{\pgfqpoint{0.150000in}{0.150000in}}{\pgfqpoint{2.700000in}{1.950000in}}%
\pgfusepath{clip}%
\pgfsetbuttcap%
\pgfsetroundjoin%
\definecolor{currentfill}{rgb}{0.866284,0.882751,0.905806}%
\pgfsetfillcolor{currentfill}%
\pgfsetlinewidth{0.000000pt}%
\definecolor{currentstroke}{rgb}{0.000000,0.000000,0.000000}%
\pgfsetstrokecolor{currentstroke}%
\pgfsetdash{}{0pt}%
\pgfpathmoveto{\pgfqpoint{1.420541in}{1.071594in}}%
\pgfpathlineto{\pgfqpoint{1.459379in}{1.085304in}}%
\pgfpathlineto{\pgfqpoint{1.420974in}{1.111219in}}%
\pgfpathlineto{\pgfqpoint{1.382091in}{1.097591in}}%
\pgfpathclose%
\pgfusepath{fill}%
\end{pgfscope}%
\begin{pgfscope}%
\pgfpathrectangle{\pgfqpoint{0.150000in}{0.150000in}}{\pgfqpoint{2.700000in}{1.950000in}}%
\pgfusepath{clip}%
\pgfsetbuttcap%
\pgfsetroundjoin%
\definecolor{currentfill}{rgb}{0.884130,0.789752,0.797227}%
\pgfsetfillcolor{currentfill}%
\pgfsetlinewidth{0.000000pt}%
\definecolor{currentstroke}{rgb}{0.000000,0.000000,0.000000}%
\pgfsetstrokecolor{currentstroke}%
\pgfsetdash{}{0pt}%
\pgfpathmoveto{\pgfqpoint{1.344161in}{0.629097in}}%
\pgfpathlineto{\pgfqpoint{1.382643in}{0.667277in}}%
\pgfpathlineto{\pgfqpoint{1.343073in}{0.764786in}}%
\pgfpathlineto{\pgfqpoint{1.304228in}{0.732346in}}%
\pgfpathclose%
\pgfusepath{fill}%
\end{pgfscope}%
\begin{pgfscope}%
\pgfpathrectangle{\pgfqpoint{0.150000in}{0.150000in}}{\pgfqpoint{2.700000in}{1.950000in}}%
\pgfusepath{clip}%
\pgfsetbuttcap%
\pgfsetroundjoin%
\definecolor{currentfill}{rgb}{0.828968,0.850031,0.879519}%
\pgfsetfillcolor{currentfill}%
\pgfsetlinewidth{0.000000pt}%
\definecolor{currentstroke}{rgb}{0.000000,0.000000,0.000000}%
\pgfsetstrokecolor{currentstroke}%
\pgfsetdash{}{0pt}%
\pgfpathmoveto{\pgfqpoint{1.227333in}{1.109907in}}%
\pgfpathlineto{\pgfqpoint{1.266483in}{1.129683in}}%
\pgfpathlineto{\pgfqpoint{1.228306in}{1.155554in}}%
\pgfpathlineto{\pgfqpoint{1.189135in}{1.135852in}}%
\pgfpathclose%
\pgfusepath{fill}%
\end{pgfscope}%
\begin{pgfscope}%
\pgfpathrectangle{\pgfqpoint{0.150000in}{0.150000in}}{\pgfqpoint{2.700000in}{1.950000in}}%
\pgfusepath{clip}%
\pgfsetbuttcap%
\pgfsetroundjoin%
\definecolor{currentfill}{rgb}{0.797871,0.822763,0.857613}%
\pgfsetfillcolor{currentfill}%
\pgfsetlinewidth{0.000000pt}%
\definecolor{currentstroke}{rgb}{0.000000,0.000000,0.000000}%
\pgfsetstrokecolor{currentstroke}%
\pgfsetdash{}{0pt}%
\pgfpathmoveto{\pgfqpoint{1.034167in}{1.148212in}}%
\pgfpathlineto{\pgfqpoint{1.073674in}{1.167914in}}%
\pgfpathlineto{\pgfqpoint{1.035455in}{1.199931in}}%
\pgfpathlineto{\pgfqpoint{0.996222in}{1.174105in}}%
\pgfpathclose%
\pgfusepath{fill}%
\end{pgfscope}%
\begin{pgfscope}%
\pgfpathrectangle{\pgfqpoint{0.150000in}{0.150000in}}{\pgfqpoint{2.700000in}{1.950000in}}%
\pgfusepath{clip}%
\pgfsetbuttcap%
\pgfsetroundjoin%
\definecolor{currentfill}{rgb}{0.797871,0.822763,0.857613}%
\pgfsetfillcolor{currentfill}%
\pgfsetlinewidth{0.000000pt}%
\definecolor{currentstroke}{rgb}{0.000000,0.000000,0.000000}%
\pgfsetstrokecolor{currentstroke}%
\pgfsetdash{}{0pt}%
\pgfpathmoveto{\pgfqpoint{0.956887in}{1.148212in}}%
\pgfpathlineto{\pgfqpoint{0.996222in}{1.174105in}}%
\pgfpathlineto{\pgfqpoint{0.958373in}{1.199931in}}%
\pgfpathlineto{\pgfqpoint{0.919041in}{1.174105in}}%
\pgfpathclose%
\pgfusepath{fill}%
\end{pgfscope}%
\begin{pgfscope}%
\pgfpathrectangle{\pgfqpoint{0.150000in}{0.150000in}}{\pgfqpoint{2.700000in}{1.950000in}}%
\pgfusepath{clip}%
\pgfsetbuttcap%
\pgfsetroundjoin%
\definecolor{currentfill}{rgb}{0.797871,0.822763,0.857613}%
\pgfsetfillcolor{currentfill}%
\pgfsetlinewidth{0.000000pt}%
\definecolor{currentstroke}{rgb}{0.000000,0.000000,0.000000}%
\pgfsetstrokecolor{currentstroke}%
\pgfsetdash{}{0pt}%
\pgfpathmoveto{\pgfqpoint{0.879608in}{1.148212in}}%
\pgfpathlineto{\pgfqpoint{0.919041in}{1.174105in}}%
\pgfpathlineto{\pgfqpoint{0.881291in}{1.199931in}}%
\pgfpathlineto{\pgfqpoint{0.841860in}{1.174105in}}%
\pgfpathclose%
\pgfusepath{fill}%
\end{pgfscope}%
\begin{pgfscope}%
\pgfpathrectangle{\pgfqpoint{0.150000in}{0.150000in}}{\pgfqpoint{2.700000in}{1.950000in}}%
\pgfusepath{clip}%
\pgfsetbuttcap%
\pgfsetroundjoin%
\definecolor{currentfill}{rgb}{0.990671,0.991820,0.993428}%
\pgfsetfillcolor{currentfill}%
\pgfsetlinewidth{0.000000pt}%
\definecolor{currentstroke}{rgb}{0.000000,0.000000,0.000000}%
\pgfsetstrokecolor{currentstroke}%
\pgfsetdash{}{0pt}%
\pgfpathmoveto{\pgfqpoint{1.497934in}{0.888066in}}%
\pgfpathlineto{\pgfqpoint{1.536486in}{0.914233in}}%
\pgfpathlineto{\pgfqpoint{1.497852in}{0.988847in}}%
\pgfpathlineto{\pgfqpoint{1.459118in}{0.962686in}}%
\pgfpathclose%
\pgfusepath{fill}%
\end{pgfscope}%
\begin{pgfscope}%
\pgfpathrectangle{\pgfqpoint{0.150000in}{0.150000in}}{\pgfqpoint{2.700000in}{1.950000in}}%
\pgfusepath{clip}%
\pgfsetbuttcap%
\pgfsetroundjoin%
\definecolor{currentfill}{rgb}{0.853845,0.871844,0.897044}%
\pgfsetfillcolor{currentfill}%
\pgfsetlinewidth{0.000000pt}%
\definecolor{currentstroke}{rgb}{0.000000,0.000000,0.000000}%
\pgfsetstrokecolor{currentstroke}%
\pgfsetdash{}{0pt}%
\pgfpathmoveto{\pgfqpoint{1.343130in}{1.077742in}}%
\pgfpathlineto{\pgfqpoint{1.382091in}{1.097591in}}%
\pgfpathlineto{\pgfqpoint{1.343852in}{1.117367in}}%
\pgfpathlineto{\pgfqpoint{1.304757in}{1.103746in}}%
\pgfpathclose%
\pgfusepath{fill}%
\end{pgfscope}%
\begin{pgfscope}%
\pgfpathrectangle{\pgfqpoint{0.150000in}{0.150000in}}{\pgfqpoint{2.700000in}{1.950000in}}%
\pgfusepath{clip}%
\pgfsetbuttcap%
\pgfsetroundjoin%
\definecolor{currentfill}{rgb}{0.816529,0.839124,0.870757}%
\pgfsetfillcolor{currentfill}%
\pgfsetlinewidth{0.000000pt}%
\definecolor{currentstroke}{rgb}{0.000000,0.000000,0.000000}%
\pgfsetstrokecolor{currentstroke}%
\pgfsetdash{}{0pt}%
\pgfpathmoveto{\pgfqpoint{1.149817in}{1.116076in}}%
\pgfpathlineto{\pgfqpoint{1.189135in}{1.135852in}}%
\pgfpathlineto{\pgfqpoint{1.150809in}{1.167914in}}%
\pgfpathlineto{\pgfqpoint{1.111697in}{1.142028in}}%
\pgfpathclose%
\pgfusepath{fill}%
\end{pgfscope}%
\begin{pgfscope}%
\pgfpathrectangle{\pgfqpoint{0.150000in}{0.150000in}}{\pgfqpoint{2.700000in}{1.950000in}}%
\pgfusepath{clip}%
\pgfsetbuttcap%
\pgfsetroundjoin%
\definecolor{currentfill}{rgb}{0.810309,0.833670,0.866376}%
\pgfsetfillcolor{currentfill}%
\pgfsetlinewidth{0.000000pt}%
\definecolor{currentstroke}{rgb}{0.000000,0.000000,0.000000}%
\pgfsetstrokecolor{currentstroke}%
\pgfsetdash{}{0pt}%
\pgfpathmoveto{\pgfqpoint{1.072211in}{1.122253in}}%
\pgfpathlineto{\pgfqpoint{1.111697in}{1.142028in}}%
\pgfpathlineto{\pgfqpoint{1.073674in}{1.167914in}}%
\pgfpathlineto{\pgfqpoint{1.034167in}{1.148212in}}%
\pgfpathclose%
\pgfusepath{fill}%
\end{pgfscope}%
\begin{pgfscope}%
\pgfpathrectangle{\pgfqpoint{0.150000in}{0.150000in}}{\pgfqpoint{2.700000in}{1.950000in}}%
\pgfusepath{clip}%
\pgfsetbuttcap%
\pgfsetroundjoin%
\definecolor{currentfill}{rgb}{0.872503,0.888205,0.910187}%
\pgfsetfillcolor{currentfill}%
\pgfsetlinewidth{0.000000pt}%
\definecolor{currentstroke}{rgb}{0.000000,0.000000,0.000000}%
\pgfsetstrokecolor{currentstroke}%
\pgfsetdash{}{0pt}%
\pgfpathmoveto{\pgfqpoint{1.459090in}{1.045531in}}%
\pgfpathlineto{\pgfqpoint{1.497861in}{1.065455in}}%
\pgfpathlineto{\pgfqpoint{1.459379in}{1.085304in}}%
\pgfpathlineto{\pgfqpoint{1.420541in}{1.071594in}}%
\pgfpathclose%
\pgfusepath{fill}%
\end{pgfscope}%
\begin{pgfscope}%
\pgfpathrectangle{\pgfqpoint{0.150000in}{0.150000in}}{\pgfqpoint{2.700000in}{1.950000in}}%
\pgfusepath{clip}%
\pgfsetbuttcap%
\pgfsetroundjoin%
\definecolor{currentfill}{rgb}{0.841406,0.860938,0.888281}%
\pgfsetfillcolor{currentfill}%
\pgfsetlinewidth{0.000000pt}%
\definecolor{currentstroke}{rgb}{0.000000,0.000000,0.000000}%
\pgfsetstrokecolor{currentstroke}%
\pgfsetdash{}{0pt}%
\pgfpathmoveto{\pgfqpoint{1.265629in}{1.083896in}}%
\pgfpathlineto{\pgfqpoint{1.304757in}{1.103746in}}%
\pgfpathlineto{\pgfqpoint{1.266483in}{1.129683in}}%
\pgfpathlineto{\pgfqpoint{1.227333in}{1.109907in}}%
\pgfpathclose%
\pgfusepath{fill}%
\end{pgfscope}%
\begin{pgfscope}%
\pgfpathrectangle{\pgfqpoint{0.150000in}{0.150000in}}{\pgfqpoint{2.700000in}{1.950000in}}%
\pgfusepath{clip}%
\pgfsetbuttcap%
\pgfsetroundjoin%
\definecolor{currentfill}{rgb}{0.903600,0.915472,0.932093}%
\pgfsetfillcolor{currentfill}%
\pgfsetlinewidth{0.000000pt}%
\definecolor{currentstroke}{rgb}{0.000000,0.000000,0.000000}%
\pgfsetstrokecolor{currentstroke}%
\pgfsetdash{}{0pt}%
\pgfpathmoveto{\pgfqpoint{1.497852in}{0.988847in}}%
\pgfpathlineto{\pgfqpoint{1.536486in}{1.014940in}}%
\pgfpathlineto{\pgfqpoint{1.497861in}{1.065455in}}%
\pgfpathlineto{\pgfqpoint{1.459090in}{1.045531in}}%
\pgfpathclose%
\pgfusepath{fill}%
\end{pgfscope}%
\begin{pgfscope}%
\pgfpathrectangle{\pgfqpoint{0.150000in}{0.150000in}}{\pgfqpoint{2.700000in}{1.950000in}}%
\pgfusepath{clip}%
\pgfsetbuttcap%
\pgfsetroundjoin%
\definecolor{currentfill}{rgb}{0.998100,0.996553,0.996676}%
\pgfsetfillcolor{currentfill}%
\pgfsetlinewidth{0.000000pt}%
\definecolor{currentstroke}{rgb}{0.000000,0.000000,0.000000}%
\pgfsetstrokecolor{currentstroke}%
\pgfsetdash{}{0pt}%
\pgfpathmoveto{\pgfqpoint{1.459328in}{0.855822in}}%
\pgfpathlineto{\pgfqpoint{1.497934in}{0.888066in}}%
\pgfpathlineto{\pgfqpoint{1.459118in}{0.962686in}}%
\pgfpathlineto{\pgfqpoint{1.420285in}{0.936458in}}%
\pgfpathclose%
\pgfusepath{fill}%
\end{pgfscope}%
\begin{pgfscope}%
\pgfpathrectangle{\pgfqpoint{0.150000in}{0.150000in}}{\pgfqpoint{2.700000in}{1.950000in}}%
\pgfusepath{clip}%
\pgfsetbuttcap%
\pgfsetroundjoin%
\definecolor{currentfill}{rgb}{0.804090,0.828217,0.861994}%
\pgfsetfillcolor{currentfill}%
\pgfsetlinewidth{0.000000pt}%
\definecolor{currentstroke}{rgb}{0.000000,0.000000,0.000000}%
\pgfsetstrokecolor{currentstroke}%
\pgfsetdash{}{0pt}%
\pgfpathmoveto{\pgfqpoint{0.994831in}{1.122253in}}%
\pgfpathlineto{\pgfqpoint{1.034167in}{1.148212in}}%
\pgfpathlineto{\pgfqpoint{0.996222in}{1.174105in}}%
\pgfpathlineto{\pgfqpoint{0.956887in}{1.148212in}}%
\pgfpathclose%
\pgfusepath{fill}%
\end{pgfscope}%
\begin{pgfscope}%
\pgfpathrectangle{\pgfqpoint{0.150000in}{0.150000in}}{\pgfqpoint{2.700000in}{1.950000in}}%
\pgfusepath{clip}%
\pgfsetbuttcap%
\pgfsetroundjoin%
\definecolor{currentfill}{rgb}{0.804090,0.828217,0.861994}%
\pgfsetfillcolor{currentfill}%
\pgfsetlinewidth{0.000000pt}%
\definecolor{currentstroke}{rgb}{0.000000,0.000000,0.000000}%
\pgfsetstrokecolor{currentstroke}%
\pgfsetdash{}{0pt}%
\pgfpathmoveto{\pgfqpoint{0.917452in}{1.122253in}}%
\pgfpathlineto{\pgfqpoint{0.956887in}{1.148212in}}%
\pgfpathlineto{\pgfqpoint{0.919041in}{1.174105in}}%
\pgfpathlineto{\pgfqpoint{0.879608in}{1.148212in}}%
\pgfpathclose%
\pgfusepath{fill}%
\end{pgfscope}%
\begin{pgfscope}%
\pgfpathrectangle{\pgfqpoint{0.150000in}{0.150000in}}{\pgfqpoint{2.700000in}{1.950000in}}%
\pgfusepath{clip}%
\pgfsetbuttcap%
\pgfsetroundjoin%
\definecolor{currentfill}{rgb}{0.828968,0.850031,0.879519}%
\pgfsetfillcolor{currentfill}%
\pgfsetlinewidth{0.000000pt}%
\definecolor{currentstroke}{rgb}{0.000000,0.000000,0.000000}%
\pgfsetstrokecolor{currentstroke}%
\pgfsetdash{}{0pt}%
\pgfpathmoveto{\pgfqpoint{1.188036in}{1.090058in}}%
\pgfpathlineto{\pgfqpoint{1.227333in}{1.109907in}}%
\pgfpathlineto{\pgfqpoint{1.189135in}{1.135852in}}%
\pgfpathlineto{\pgfqpoint{1.149817in}{1.116076in}}%
\pgfpathclose%
\pgfusepath{fill}%
\end{pgfscope}%
\begin{pgfscope}%
\pgfpathrectangle{\pgfqpoint{0.150000in}{0.150000in}}{\pgfqpoint{2.700000in}{1.950000in}}%
\pgfusepath{clip}%
\pgfsetbuttcap%
\pgfsetroundjoin%
\definecolor{currentfill}{rgb}{0.860064,0.877298,0.901425}%
\pgfsetfillcolor{currentfill}%
\pgfsetlinewidth{0.000000pt}%
\definecolor{currentstroke}{rgb}{0.000000,0.000000,0.000000}%
\pgfsetstrokecolor{currentstroke}%
\pgfsetdash{}{0pt}%
\pgfpathmoveto{\pgfqpoint{1.381603in}{1.051671in}}%
\pgfpathlineto{\pgfqpoint{1.420541in}{1.071594in}}%
\pgfpathlineto{\pgfqpoint{1.382091in}{1.097591in}}%
\pgfpathlineto{\pgfqpoint{1.343130in}{1.077742in}}%
\pgfpathclose%
\pgfusepath{fill}%
\end{pgfscope}%
\begin{pgfscope}%
\pgfpathrectangle{\pgfqpoint{0.150000in}{0.150000in}}{\pgfqpoint{2.700000in}{1.950000in}}%
\pgfusepath{clip}%
\pgfsetbuttcap%
\pgfsetroundjoin%
\definecolor{currentfill}{rgb}{0.990502,0.982767,0.983379}%
\pgfsetfillcolor{currentfill}%
\pgfsetlinewidth{0.000000pt}%
\definecolor{currentstroke}{rgb}{0.000000,0.000000,0.000000}%
\pgfsetstrokecolor{currentstroke}%
\pgfsetdash{}{0pt}%
\pgfpathmoveto{\pgfqpoint{1.420668in}{0.823533in}}%
\pgfpathlineto{\pgfqpoint{1.459328in}{0.855822in}}%
\pgfpathlineto{\pgfqpoint{1.420285in}{0.936458in}}%
\pgfpathlineto{\pgfqpoint{1.381351in}{0.910163in}}%
\pgfpathclose%
\pgfusepath{fill}%
\end{pgfscope}%
\begin{pgfscope}%
\pgfpathrectangle{\pgfqpoint{0.150000in}{0.150000in}}{\pgfqpoint{2.700000in}{1.950000in}}%
\pgfusepath{clip}%
\pgfsetbuttcap%
\pgfsetroundjoin%
\definecolor{currentfill}{rgb}{0.822748,0.844577,0.875138}%
\pgfsetfillcolor{currentfill}%
\pgfsetlinewidth{0.000000pt}%
\definecolor{currentstroke}{rgb}{0.000000,0.000000,0.000000}%
\pgfsetstrokecolor{currentstroke}%
\pgfsetdash{}{0pt}%
\pgfpathmoveto{\pgfqpoint{1.110352in}{1.096226in}}%
\pgfpathlineto{\pgfqpoint{1.149817in}{1.116076in}}%
\pgfpathlineto{\pgfqpoint{1.111697in}{1.142028in}}%
\pgfpathlineto{\pgfqpoint{1.072211in}{1.122253in}}%
\pgfpathclose%
\pgfusepath{fill}%
\end{pgfscope}%
\begin{pgfscope}%
\pgfpathrectangle{\pgfqpoint{0.150000in}{0.150000in}}{\pgfqpoint{2.700000in}{1.950000in}}%
\pgfusepath{clip}%
\pgfsetbuttcap%
\pgfsetroundjoin%
\definecolor{currentfill}{rgb}{0.853845,0.871844,0.897044}%
\pgfsetfillcolor{currentfill}%
\pgfsetlinewidth{0.000000pt}%
\definecolor{currentstroke}{rgb}{0.000000,0.000000,0.000000}%
\pgfsetstrokecolor{currentstroke}%
\pgfsetdash{}{0pt}%
\pgfpathmoveto{\pgfqpoint{1.304024in}{1.057817in}}%
\pgfpathlineto{\pgfqpoint{1.343130in}{1.077742in}}%
\pgfpathlineto{\pgfqpoint{1.304757in}{1.103746in}}%
\pgfpathlineto{\pgfqpoint{1.265629in}{1.083896in}}%
\pgfpathclose%
\pgfusepath{fill}%
\end{pgfscope}%
\begin{pgfscope}%
\pgfpathrectangle{\pgfqpoint{0.150000in}{0.150000in}}{\pgfqpoint{2.700000in}{1.950000in}}%
\pgfusepath{clip}%
\pgfsetbuttcap%
\pgfsetroundjoin%
\definecolor{currentfill}{rgb}{0.816529,0.839124,0.870757}%
\pgfsetfillcolor{currentfill}%
\pgfsetlinewidth{0.000000pt}%
\definecolor{currentstroke}{rgb}{0.000000,0.000000,0.000000}%
\pgfsetstrokecolor{currentstroke}%
\pgfsetdash{}{0pt}%
\pgfpathmoveto{\pgfqpoint{1.032873in}{1.096226in}}%
\pgfpathlineto{\pgfqpoint{1.072211in}{1.122253in}}%
\pgfpathlineto{\pgfqpoint{1.034167in}{1.148212in}}%
\pgfpathlineto{\pgfqpoint{0.994831in}{1.122253in}}%
\pgfpathclose%
\pgfusepath{fill}%
\end{pgfscope}%
\begin{pgfscope}%
\pgfpathrectangle{\pgfqpoint{0.150000in}{0.150000in}}{\pgfqpoint{2.700000in}{1.950000in}}%
\pgfusepath{clip}%
\pgfsetbuttcap%
\pgfsetroundjoin%
\definecolor{currentfill}{rgb}{0.816529,0.839124,0.870757}%
\pgfsetfillcolor{currentfill}%
\pgfsetlinewidth{0.000000pt}%
\definecolor{currentstroke}{rgb}{0.000000,0.000000,0.000000}%
\pgfsetstrokecolor{currentstroke}%
\pgfsetdash{}{0pt}%
\pgfpathmoveto{\pgfqpoint{0.955394in}{1.096226in}}%
\pgfpathlineto{\pgfqpoint{0.994831in}{1.122253in}}%
\pgfpathlineto{\pgfqpoint{0.956887in}{1.148212in}}%
\pgfpathlineto{\pgfqpoint{0.917452in}{1.122253in}}%
\pgfpathclose%
\pgfusepath{fill}%
\end{pgfscope}%
\begin{pgfscope}%
\pgfpathrectangle{\pgfqpoint{0.150000in}{0.150000in}}{\pgfqpoint{2.700000in}{1.950000in}}%
\pgfusepath{clip}%
\pgfsetbuttcap%
\pgfsetroundjoin%
\definecolor{currentfill}{rgb}{0.909819,0.920925,0.936474}%
\pgfsetfillcolor{currentfill}%
\pgfsetlinewidth{0.000000pt}%
\definecolor{currentstroke}{rgb}{0.000000,0.000000,0.000000}%
\pgfsetstrokecolor{currentstroke}%
\pgfsetdash{}{0pt}%
\pgfpathmoveto{\pgfqpoint{1.459118in}{0.962686in}}%
\pgfpathlineto{\pgfqpoint{1.497852in}{0.988847in}}%
\pgfpathlineto{\pgfqpoint{1.459090in}{1.045531in}}%
\pgfpathlineto{\pgfqpoint{1.420105in}{1.031672in}}%
\pgfpathclose%
\pgfusepath{fill}%
\end{pgfscope}%
\begin{pgfscope}%
\pgfpathrectangle{\pgfqpoint{0.150000in}{0.150000in}}{\pgfqpoint{2.700000in}{1.950000in}}%
\pgfusepath{clip}%
\pgfsetbuttcap%
\pgfsetroundjoin%
\definecolor{currentfill}{rgb}{0.841406,0.860938,0.888281}%
\pgfsetfillcolor{currentfill}%
\pgfsetlinewidth{0.000000pt}%
\definecolor{currentstroke}{rgb}{0.000000,0.000000,0.000000}%
\pgfsetstrokecolor{currentstroke}%
\pgfsetdash{}{0pt}%
\pgfpathmoveto{\pgfqpoint{1.226354in}{1.063972in}}%
\pgfpathlineto{\pgfqpoint{1.265629in}{1.083896in}}%
\pgfpathlineto{\pgfqpoint{1.227333in}{1.109907in}}%
\pgfpathlineto{\pgfqpoint{1.188036in}{1.090058in}}%
\pgfpathclose%
\pgfusepath{fill}%
\end{pgfscope}%
\begin{pgfscope}%
\pgfpathrectangle{\pgfqpoint{0.150000in}{0.150000in}}{\pgfqpoint{2.700000in}{1.950000in}}%
\pgfusepath{clip}%
\pgfsetbuttcap%
\pgfsetroundjoin%
\definecolor{currentfill}{rgb}{0.979105,0.962086,0.963434}%
\pgfsetfillcolor{currentfill}%
\pgfsetlinewidth{0.000000pt}%
\definecolor{currentstroke}{rgb}{0.000000,0.000000,0.000000}%
\pgfsetstrokecolor{currentstroke}%
\pgfsetdash{}{0pt}%
\pgfpathmoveto{\pgfqpoint{1.381954in}{0.791199in}}%
\pgfpathlineto{\pgfqpoint{1.420668in}{0.823533in}}%
\pgfpathlineto{\pgfqpoint{1.381351in}{0.910163in}}%
\pgfpathlineto{\pgfqpoint{1.342317in}{0.883799in}}%
\pgfpathclose%
\pgfusepath{fill}%
\end{pgfscope}%
\begin{pgfscope}%
\pgfpathrectangle{\pgfqpoint{0.150000in}{0.150000in}}{\pgfqpoint{2.700000in}{1.950000in}}%
\pgfusepath{clip}%
\pgfsetbuttcap%
\pgfsetroundjoin%
\definecolor{currentfill}{rgb}{0.872503,0.888205,0.910187}%
\pgfsetfillcolor{currentfill}%
\pgfsetlinewidth{0.000000pt}%
\definecolor{currentstroke}{rgb}{0.000000,0.000000,0.000000}%
\pgfsetstrokecolor{currentstroke}%
\pgfsetdash{}{0pt}%
\pgfpathmoveto{\pgfqpoint{1.420105in}{1.031672in}}%
\pgfpathlineto{\pgfqpoint{1.459090in}{1.045531in}}%
\pgfpathlineto{\pgfqpoint{1.420541in}{1.071594in}}%
\pgfpathlineto{\pgfqpoint{1.381603in}{1.051671in}}%
\pgfpathclose%
\pgfusepath{fill}%
\end{pgfscope}%
\begin{pgfscope}%
\pgfpathrectangle{\pgfqpoint{0.150000in}{0.150000in}}{\pgfqpoint{2.700000in}{1.950000in}}%
\pgfusepath{clip}%
\pgfsetbuttcap%
\pgfsetroundjoin%
\definecolor{currentfill}{rgb}{0.828968,0.850031,0.879519}%
\pgfsetfillcolor{currentfill}%
\pgfsetlinewidth{0.000000pt}%
\definecolor{currentstroke}{rgb}{0.000000,0.000000,0.000000}%
\pgfsetstrokecolor{currentstroke}%
\pgfsetdash{}{0pt}%
\pgfpathmoveto{\pgfqpoint{1.148592in}{1.070133in}}%
\pgfpathlineto{\pgfqpoint{1.188036in}{1.090058in}}%
\pgfpathlineto{\pgfqpoint{1.149817in}{1.116076in}}%
\pgfpathlineto{\pgfqpoint{1.110352in}{1.096226in}}%
\pgfpathclose%
\pgfusepath{fill}%
\end{pgfscope}%
\begin{pgfscope}%
\pgfpathrectangle{\pgfqpoint{0.150000in}{0.150000in}}{\pgfqpoint{2.700000in}{1.950000in}}%
\pgfusepath{clip}%
\pgfsetbuttcap%
\pgfsetroundjoin%
\definecolor{currentfill}{rgb}{0.860064,0.877298,0.901425}%
\pgfsetfillcolor{currentfill}%
\pgfsetlinewidth{0.000000pt}%
\definecolor{currentstroke}{rgb}{0.000000,0.000000,0.000000}%
\pgfsetstrokecolor{currentstroke}%
\pgfsetdash{}{0pt}%
\pgfpathmoveto{\pgfqpoint{1.342403in}{1.037818in}}%
\pgfpathlineto{\pgfqpoint{1.381603in}{1.051671in}}%
\pgfpathlineto{\pgfqpoint{1.343130in}{1.077742in}}%
\pgfpathlineto{\pgfqpoint{1.304024in}{1.057817in}}%
\pgfpathclose%
\pgfusepath{fill}%
\end{pgfscope}%
\begin{pgfscope}%
\pgfpathrectangle{\pgfqpoint{0.150000in}{0.150000in}}{\pgfqpoint{2.700000in}{1.950000in}}%
\pgfusepath{clip}%
\pgfsetbuttcap%
\pgfsetroundjoin%
\definecolor{currentfill}{rgb}{0.828968,0.850031,0.879519}%
\pgfsetfillcolor{currentfill}%
\pgfsetlinewidth{0.000000pt}%
\definecolor{currentstroke}{rgb}{0.000000,0.000000,0.000000}%
\pgfsetstrokecolor{currentstroke}%
\pgfsetdash{}{0pt}%
\pgfpathmoveto{\pgfqpoint{1.071013in}{1.070133in}}%
\pgfpathlineto{\pgfqpoint{1.110352in}{1.096226in}}%
\pgfpathlineto{\pgfqpoint{1.072211in}{1.122253in}}%
\pgfpathlineto{\pgfqpoint{1.032873in}{1.096226in}}%
\pgfpathclose%
\pgfusepath{fill}%
\end{pgfscope}%
\begin{pgfscope}%
\pgfpathrectangle{\pgfqpoint{0.150000in}{0.150000in}}{\pgfqpoint{2.700000in}{1.950000in}}%
\pgfusepath{clip}%
\pgfsetbuttcap%
\pgfsetroundjoin%
\definecolor{currentfill}{rgb}{0.828968,0.850031,0.879519}%
\pgfsetfillcolor{currentfill}%
\pgfsetlinewidth{0.000000pt}%
\definecolor{currentstroke}{rgb}{0.000000,0.000000,0.000000}%
\pgfsetstrokecolor{currentstroke}%
\pgfsetdash{}{0pt}%
\pgfpathmoveto{\pgfqpoint{0.993434in}{1.070133in}}%
\pgfpathlineto{\pgfqpoint{1.032873in}{1.096226in}}%
\pgfpathlineto{\pgfqpoint{0.994831in}{1.122253in}}%
\pgfpathlineto{\pgfqpoint{0.955394in}{1.096226in}}%
\pgfpathclose%
\pgfusepath{fill}%
\end{pgfscope}%
\begin{pgfscope}%
\pgfpathrectangle{\pgfqpoint{0.150000in}{0.150000in}}{\pgfqpoint{2.700000in}{1.950000in}}%
\pgfusepath{clip}%
\pgfsetbuttcap%
\pgfsetroundjoin%
\definecolor{currentfill}{rgb}{0.853845,0.871844,0.897044}%
\pgfsetfillcolor{currentfill}%
\pgfsetlinewidth{0.000000pt}%
\definecolor{currentstroke}{rgb}{0.000000,0.000000,0.000000}%
\pgfsetstrokecolor{currentstroke}%
\pgfsetdash{}{0pt}%
\pgfpathmoveto{\pgfqpoint{1.264770in}{1.037818in}}%
\pgfpathlineto{\pgfqpoint{1.304024in}{1.057817in}}%
\pgfpathlineto{\pgfqpoint{1.265629in}{1.083896in}}%
\pgfpathlineto{\pgfqpoint{1.226354in}{1.063972in}}%
\pgfpathclose%
\pgfusepath{fill}%
\end{pgfscope}%
\begin{pgfscope}%
\pgfpathrectangle{\pgfqpoint{0.150000in}{0.150000in}}{\pgfqpoint{2.700000in}{1.950000in}}%
\pgfusepath{clip}%
\pgfsetbuttcap%
\pgfsetroundjoin%
\definecolor{currentfill}{rgb}{0.971507,0.948300,0.950138}%
\pgfsetfillcolor{currentfill}%
\pgfsetlinewidth{0.000000pt}%
\definecolor{currentstroke}{rgb}{0.000000,0.000000,0.000000}%
\pgfsetstrokecolor{currentstroke}%
\pgfsetdash{}{0pt}%
\pgfpathmoveto{\pgfqpoint{1.343073in}{0.764786in}}%
\pgfpathlineto{\pgfqpoint{1.381954in}{0.791199in}}%
\pgfpathlineto{\pgfqpoint{1.342317in}{0.883799in}}%
\pgfpathlineto{\pgfqpoint{1.303182in}{0.857368in}}%
\pgfpathclose%
\pgfusepath{fill}%
\end{pgfscope}%
\begin{pgfscope}%
\pgfpathrectangle{\pgfqpoint{0.150000in}{0.150000in}}{\pgfqpoint{2.700000in}{1.950000in}}%
\pgfusepath{clip}%
\pgfsetbuttcap%
\pgfsetroundjoin%
\definecolor{currentfill}{rgb}{0.909819,0.920925,0.936474}%
\pgfsetfillcolor{currentfill}%
\pgfsetlinewidth{0.000000pt}%
\definecolor{currentstroke}{rgb}{0.000000,0.000000,0.000000}%
\pgfsetstrokecolor{currentstroke}%
\pgfsetdash{}{0pt}%
\pgfpathmoveto{\pgfqpoint{1.420285in}{0.936458in}}%
\pgfpathlineto{\pgfqpoint{1.459118in}{0.962686in}}%
\pgfpathlineto{\pgfqpoint{1.420105in}{1.031672in}}%
\pgfpathlineto{\pgfqpoint{1.381019in}{1.011597in}}%
\pgfpathclose%
\pgfusepath{fill}%
\end{pgfscope}%
\begin{pgfscope}%
\pgfpathrectangle{\pgfqpoint{0.150000in}{0.150000in}}{\pgfqpoint{2.700000in}{1.950000in}}%
\pgfusepath{clip}%
\pgfsetbuttcap%
\pgfsetroundjoin%
\definecolor{currentfill}{rgb}{0.841406,0.860938,0.888281}%
\pgfsetfillcolor{currentfill}%
\pgfsetlinewidth{0.000000pt}%
\definecolor{currentstroke}{rgb}{0.000000,0.000000,0.000000}%
\pgfsetstrokecolor{currentstroke}%
\pgfsetdash{}{0pt}%
\pgfpathmoveto{\pgfqpoint{1.186930in}{1.043972in}}%
\pgfpathlineto{\pgfqpoint{1.226354in}{1.063972in}}%
\pgfpathlineto{\pgfqpoint{1.188036in}{1.090058in}}%
\pgfpathlineto{\pgfqpoint{1.148592in}{1.070133in}}%
\pgfpathclose%
\pgfusepath{fill}%
\end{pgfscope}%
\begin{pgfscope}%
\pgfpathrectangle{\pgfqpoint{0.150000in}{0.150000in}}{\pgfqpoint{2.700000in}{1.950000in}}%
\pgfusepath{clip}%
\pgfsetbuttcap%
\pgfsetroundjoin%
\definecolor{currentfill}{rgb}{0.835187,0.855484,0.883900}%
\pgfsetfillcolor{currentfill}%
\pgfsetlinewidth{0.000000pt}%
\definecolor{currentstroke}{rgb}{0.000000,0.000000,0.000000}%
\pgfsetstrokecolor{currentstroke}%
\pgfsetdash{}{0pt}%
\pgfpathmoveto{\pgfqpoint{1.109251in}{1.043972in}}%
\pgfpathlineto{\pgfqpoint{1.148592in}{1.070133in}}%
\pgfpathlineto{\pgfqpoint{1.110352in}{1.096226in}}%
\pgfpathlineto{\pgfqpoint{1.071013in}{1.070133in}}%
\pgfpathclose%
\pgfusepath{fill}%
\end{pgfscope}%
\begin{pgfscope}%
\pgfpathrectangle{\pgfqpoint{0.150000in}{0.150000in}}{\pgfqpoint{2.700000in}{1.950000in}}%
\pgfusepath{clip}%
\pgfsetbuttcap%
\pgfsetroundjoin%
\definecolor{currentfill}{rgb}{0.835187,0.855484,0.883900}%
\pgfsetfillcolor{currentfill}%
\pgfsetlinewidth{0.000000pt}%
\definecolor{currentstroke}{rgb}{0.000000,0.000000,0.000000}%
\pgfsetstrokecolor{currentstroke}%
\pgfsetdash{}{0pt}%
\pgfpathmoveto{\pgfqpoint{1.031572in}{1.043972in}}%
\pgfpathlineto{\pgfqpoint{1.071013in}{1.070133in}}%
\pgfpathlineto{\pgfqpoint{1.032873in}{1.096226in}}%
\pgfpathlineto{\pgfqpoint{0.993434in}{1.070133in}}%
\pgfpathclose%
\pgfusepath{fill}%
\end{pgfscope}%
\begin{pgfscope}%
\pgfpathrectangle{\pgfqpoint{0.150000in}{0.150000in}}{\pgfqpoint{2.700000in}{1.950000in}}%
\pgfusepath{clip}%
\pgfsetbuttcap%
\pgfsetroundjoin%
\definecolor{currentfill}{rgb}{0.866284,0.882751,0.905806}%
\pgfsetfillcolor{currentfill}%
\pgfsetlinewidth{0.000000pt}%
\definecolor{currentstroke}{rgb}{0.000000,0.000000,0.000000}%
\pgfsetstrokecolor{currentstroke}%
\pgfsetdash{}{0pt}%
\pgfpathmoveto{\pgfqpoint{1.381019in}{1.011597in}}%
\pgfpathlineto{\pgfqpoint{1.420105in}{1.031672in}}%
\pgfpathlineto{\pgfqpoint{1.381603in}{1.051671in}}%
\pgfpathlineto{\pgfqpoint{1.342403in}{1.037818in}}%
\pgfpathclose%
\pgfusepath{fill}%
\end{pgfscope}%
\begin{pgfscope}%
\pgfpathrectangle{\pgfqpoint{0.150000in}{0.150000in}}{\pgfqpoint{2.700000in}{1.950000in}}%
\pgfusepath{clip}%
\pgfsetbuttcap%
\pgfsetroundjoin%
\definecolor{currentfill}{rgb}{0.963909,0.934513,0.936841}%
\pgfsetfillcolor{currentfill}%
\pgfsetlinewidth{0.000000pt}%
\definecolor{currentstroke}{rgb}{0.000000,0.000000,0.000000}%
\pgfsetstrokecolor{currentstroke}%
\pgfsetdash{}{0pt}%
\pgfpathmoveto{\pgfqpoint{1.304228in}{0.732346in}}%
\pgfpathlineto{\pgfqpoint{1.343073in}{0.764786in}}%
\pgfpathlineto{\pgfqpoint{1.303182in}{0.857368in}}%
\pgfpathlineto{\pgfqpoint{1.263945in}{0.830867in}}%
\pgfpathclose%
\pgfusepath{fill}%
\end{pgfscope}%
\begin{pgfscope}%
\pgfpathrectangle{\pgfqpoint{0.150000in}{0.150000in}}{\pgfqpoint{2.700000in}{1.950000in}}%
\pgfusepath{clip}%
\pgfsetbuttcap%
\pgfsetroundjoin%
\definecolor{currentfill}{rgb}{0.860064,0.877298,0.901425}%
\pgfsetfillcolor{currentfill}%
\pgfsetlinewidth{0.000000pt}%
\definecolor{currentstroke}{rgb}{0.000000,0.000000,0.000000}%
\pgfsetstrokecolor{currentstroke}%
\pgfsetdash{}{0pt}%
\pgfpathmoveto{\pgfqpoint{1.303148in}{1.017744in}}%
\pgfpathlineto{\pgfqpoint{1.342403in}{1.037818in}}%
\pgfpathlineto{\pgfqpoint{1.304024in}{1.057817in}}%
\pgfpathlineto{\pgfqpoint{1.264770in}{1.037818in}}%
\pgfpathclose%
\pgfusepath{fill}%
\end{pgfscope}%
\begin{pgfscope}%
\pgfpathrectangle{\pgfqpoint{0.150000in}{0.150000in}}{\pgfqpoint{2.700000in}{1.950000in}}%
\pgfusepath{clip}%
\pgfsetbuttcap%
\pgfsetroundjoin%
\definecolor{currentfill}{rgb}{0.922258,0.931832,0.945236}%
\pgfsetfillcolor{currentfill}%
\pgfsetlinewidth{0.000000pt}%
\definecolor{currentstroke}{rgb}{0.000000,0.000000,0.000000}%
\pgfsetstrokecolor{currentstroke}%
\pgfsetdash{}{0pt}%
\pgfpathmoveto{\pgfqpoint{1.381351in}{0.910163in}}%
\pgfpathlineto{\pgfqpoint{1.420285in}{0.936458in}}%
\pgfpathlineto{\pgfqpoint{1.381019in}{1.011597in}}%
\pgfpathlineto{\pgfqpoint{1.342016in}{0.979177in}}%
\pgfpathclose%
\pgfusepath{fill}%
\end{pgfscope}%
\begin{pgfscope}%
\pgfpathrectangle{\pgfqpoint{0.150000in}{0.150000in}}{\pgfqpoint{2.700000in}{1.950000in}}%
\pgfusepath{clip}%
\pgfsetbuttcap%
\pgfsetroundjoin%
\definecolor{currentfill}{rgb}{0.853845,0.871844,0.897044}%
\pgfsetfillcolor{currentfill}%
\pgfsetlinewidth{0.000000pt}%
\definecolor{currentstroke}{rgb}{0.000000,0.000000,0.000000}%
\pgfsetstrokecolor{currentstroke}%
\pgfsetdash{}{0pt}%
\pgfpathmoveto{\pgfqpoint{1.225368in}{1.017744in}}%
\pgfpathlineto{\pgfqpoint{1.264770in}{1.037818in}}%
\pgfpathlineto{\pgfqpoint{1.226354in}{1.063972in}}%
\pgfpathlineto{\pgfqpoint{1.186930in}{1.043972in}}%
\pgfpathclose%
\pgfusepath{fill}%
\end{pgfscope}%
\begin{pgfscope}%
\pgfpathrectangle{\pgfqpoint{0.150000in}{0.150000in}}{\pgfqpoint{2.700000in}{1.950000in}}%
\pgfusepath{clip}%
\pgfsetbuttcap%
\pgfsetroundjoin%
\definecolor{currentfill}{rgb}{0.847626,0.866391,0.892662}%
\pgfsetfillcolor{currentfill}%
\pgfsetlinewidth{0.000000pt}%
\definecolor{currentstroke}{rgb}{0.000000,0.000000,0.000000}%
\pgfsetstrokecolor{currentstroke}%
\pgfsetdash{}{0pt}%
\pgfpathmoveto{\pgfqpoint{1.147589in}{1.017744in}}%
\pgfpathlineto{\pgfqpoint{1.186930in}{1.043972in}}%
\pgfpathlineto{\pgfqpoint{1.148592in}{1.070133in}}%
\pgfpathlineto{\pgfqpoint{1.109251in}{1.043972in}}%
\pgfpathclose%
\pgfusepath{fill}%
\end{pgfscope}%
\begin{pgfscope}%
\pgfpathrectangle{\pgfqpoint{0.150000in}{0.150000in}}{\pgfqpoint{2.700000in}{1.950000in}}%
\pgfusepath{clip}%
\pgfsetbuttcap%
\pgfsetroundjoin%
\definecolor{currentfill}{rgb}{0.847626,0.866391,0.892662}%
\pgfsetfillcolor{currentfill}%
\pgfsetlinewidth{0.000000pt}%
\definecolor{currentstroke}{rgb}{0.000000,0.000000,0.000000}%
\pgfsetstrokecolor{currentstroke}%
\pgfsetdash{}{0pt}%
\pgfpathmoveto{\pgfqpoint{1.069809in}{1.017744in}}%
\pgfpathlineto{\pgfqpoint{1.109251in}{1.043972in}}%
\pgfpathlineto{\pgfqpoint{1.071013in}{1.070133in}}%
\pgfpathlineto{\pgfqpoint{1.031572in}{1.043972in}}%
\pgfpathclose%
\pgfusepath{fill}%
\end{pgfscope}%
\begin{pgfscope}%
\pgfpathrectangle{\pgfqpoint{0.150000in}{0.150000in}}{\pgfqpoint{2.700000in}{1.950000in}}%
\pgfusepath{clip}%
\pgfsetbuttcap%
\pgfsetroundjoin%
\definecolor{currentfill}{rgb}{0.934697,0.942739,0.953998}%
\pgfsetfillcolor{currentfill}%
\pgfsetlinewidth{0.000000pt}%
\definecolor{currentstroke}{rgb}{0.000000,0.000000,0.000000}%
\pgfsetstrokecolor{currentstroke}%
\pgfsetdash{}{0pt}%
\pgfpathmoveto{\pgfqpoint{1.342317in}{0.883799in}}%
\pgfpathlineto{\pgfqpoint{1.381351in}{0.910163in}}%
\pgfpathlineto{\pgfqpoint{1.342016in}{0.979177in}}%
\pgfpathlineto{\pgfqpoint{1.302820in}{0.952828in}}%
\pgfpathclose%
\pgfusepath{fill}%
\end{pgfscope}%
\begin{pgfscope}%
\pgfpathrectangle{\pgfqpoint{0.150000in}{0.150000in}}{\pgfqpoint{2.700000in}{1.950000in}}%
\pgfusepath{clip}%
\pgfsetbuttcap%
\pgfsetroundjoin%
\definecolor{currentfill}{rgb}{0.866284,0.882751,0.905806}%
\pgfsetfillcolor{currentfill}%
\pgfsetlinewidth{0.000000pt}%
\definecolor{currentstroke}{rgb}{0.000000,0.000000,0.000000}%
\pgfsetstrokecolor{currentstroke}%
\pgfsetdash{}{0pt}%
\pgfpathmoveto{\pgfqpoint{1.342016in}{0.979177in}}%
\pgfpathlineto{\pgfqpoint{1.381019in}{1.011597in}}%
\pgfpathlineto{\pgfqpoint{1.342403in}{1.037818in}}%
\pgfpathlineto{\pgfqpoint{1.303148in}{1.017744in}}%
\pgfpathclose%
\pgfusepath{fill}%
\end{pgfscope}%
\begin{pgfscope}%
\pgfpathrectangle{\pgfqpoint{0.150000in}{0.150000in}}{\pgfqpoint{2.700000in}{1.950000in}}%
\pgfusepath{clip}%
\pgfsetbuttcap%
\pgfsetroundjoin%
\definecolor{currentfill}{rgb}{0.860064,0.877298,0.901425}%
\pgfsetfillcolor{currentfill}%
\pgfsetlinewidth{0.000000pt}%
\definecolor{currentstroke}{rgb}{0.000000,0.000000,0.000000}%
\pgfsetstrokecolor{currentstroke}%
\pgfsetdash{}{0pt}%
\pgfpathmoveto{\pgfqpoint{1.263905in}{0.991447in}}%
\pgfpathlineto{\pgfqpoint{1.303148in}{1.017744in}}%
\pgfpathlineto{\pgfqpoint{1.264770in}{1.037818in}}%
\pgfpathlineto{\pgfqpoint{1.225368in}{1.017744in}}%
\pgfpathclose%
\pgfusepath{fill}%
\end{pgfscope}%
\begin{pgfscope}%
\pgfpathrectangle{\pgfqpoint{0.150000in}{0.150000in}}{\pgfqpoint{2.700000in}{1.950000in}}%
\pgfusepath{clip}%
\pgfsetbuttcap%
\pgfsetroundjoin%
\definecolor{currentfill}{rgb}{0.853845,0.871844,0.897044}%
\pgfsetfillcolor{currentfill}%
\pgfsetlinewidth{0.000000pt}%
\definecolor{currentstroke}{rgb}{0.000000,0.000000,0.000000}%
\pgfsetstrokecolor{currentstroke}%
\pgfsetdash{}{0pt}%
\pgfpathmoveto{\pgfqpoint{1.185818in}{0.997594in}}%
\pgfpathlineto{\pgfqpoint{1.225368in}{1.017744in}}%
\pgfpathlineto{\pgfqpoint{1.186930in}{1.043972in}}%
\pgfpathlineto{\pgfqpoint{1.147589in}{1.017744in}}%
\pgfpathclose%
\pgfusepath{fill}%
\end{pgfscope}%
\begin{pgfscope}%
\pgfpathrectangle{\pgfqpoint{0.150000in}{0.150000in}}{\pgfqpoint{2.700000in}{1.950000in}}%
\pgfusepath{clip}%
\pgfsetbuttcap%
\pgfsetroundjoin%
\definecolor{currentfill}{rgb}{0.853845,0.871844,0.897044}%
\pgfsetfillcolor{currentfill}%
\pgfsetlinewidth{0.000000pt}%
\definecolor{currentstroke}{rgb}{0.000000,0.000000,0.000000}%
\pgfsetstrokecolor{currentstroke}%
\pgfsetdash{}{0pt}%
\pgfpathmoveto{\pgfqpoint{1.107891in}{0.997594in}}%
\pgfpathlineto{\pgfqpoint{1.147589in}{1.017744in}}%
\pgfpathlineto{\pgfqpoint{1.109251in}{1.043972in}}%
\pgfpathlineto{\pgfqpoint{1.069809in}{1.017744in}}%
\pgfpathclose%
\pgfusepath{fill}%
\end{pgfscope}%
\begin{pgfscope}%
\pgfpathrectangle{\pgfqpoint{0.150000in}{0.150000in}}{\pgfqpoint{2.700000in}{1.950000in}}%
\pgfusepath{clip}%
\pgfsetbuttcap%
\pgfsetroundjoin%
\definecolor{currentfill}{rgb}{0.940916,0.948192,0.958379}%
\pgfsetfillcolor{currentfill}%
\pgfsetlinewidth{0.000000pt}%
\definecolor{currentstroke}{rgb}{0.000000,0.000000,0.000000}%
\pgfsetstrokecolor{currentstroke}%
\pgfsetdash{}{0pt}%
\pgfpathmoveto{\pgfqpoint{1.303182in}{0.857368in}}%
\pgfpathlineto{\pgfqpoint{1.342317in}{0.883799in}}%
\pgfpathlineto{\pgfqpoint{1.302820in}{0.952828in}}%
\pgfpathlineto{\pgfqpoint{1.263522in}{0.926410in}}%
\pgfpathclose%
\pgfusepath{fill}%
\end{pgfscope}%
\begin{pgfscope}%
\pgfpathrectangle{\pgfqpoint{0.150000in}{0.150000in}}{\pgfqpoint{2.700000in}{1.950000in}}%
\pgfusepath{clip}%
\pgfsetbuttcap%
\pgfsetroundjoin%
\definecolor{currentfill}{rgb}{0.878722,0.893658,0.914568}%
\pgfsetfillcolor{currentfill}%
\pgfsetlinewidth{0.000000pt}%
\definecolor{currentstroke}{rgb}{0.000000,0.000000,0.000000}%
\pgfsetstrokecolor{currentstroke}%
\pgfsetdash{}{0pt}%
\pgfpathmoveto{\pgfqpoint{1.302820in}{0.952828in}}%
\pgfpathlineto{\pgfqpoint{1.342016in}{0.979177in}}%
\pgfpathlineto{\pgfqpoint{1.303148in}{1.017744in}}%
\pgfpathlineto{\pgfqpoint{1.263905in}{0.991447in}}%
\pgfpathclose%
\pgfusepath{fill}%
\end{pgfscope}%
\begin{pgfscope}%
\pgfpathrectangle{\pgfqpoint{0.150000in}{0.150000in}}{\pgfqpoint{2.700000in}{1.950000in}}%
\pgfusepath{clip}%
\pgfsetbuttcap%
\pgfsetroundjoin%
\definecolor{currentfill}{rgb}{0.860064,0.877298,0.901425}%
\pgfsetfillcolor{currentfill}%
\pgfsetlinewidth{0.000000pt}%
\definecolor{currentstroke}{rgb}{0.000000,0.000000,0.000000}%
\pgfsetstrokecolor{currentstroke}%
\pgfsetdash{}{0pt}%
\pgfpathmoveto{\pgfqpoint{1.224376in}{0.971221in}}%
\pgfpathlineto{\pgfqpoint{1.263905in}{0.991447in}}%
\pgfpathlineto{\pgfqpoint{1.225368in}{1.017744in}}%
\pgfpathlineto{\pgfqpoint{1.185818in}{0.997594in}}%
\pgfpathclose%
\pgfusepath{fill}%
\end{pgfscope}%
\begin{pgfscope}%
\pgfpathrectangle{\pgfqpoint{0.150000in}{0.150000in}}{\pgfqpoint{2.700000in}{1.950000in}}%
\pgfusepath{clip}%
\pgfsetbuttcap%
\pgfsetroundjoin%
\definecolor{currentfill}{rgb}{0.860064,0.877298,0.901425}%
\pgfsetfillcolor{currentfill}%
\pgfsetlinewidth{0.000000pt}%
\definecolor{currentstroke}{rgb}{0.000000,0.000000,0.000000}%
\pgfsetstrokecolor{currentstroke}%
\pgfsetdash{}{0pt}%
\pgfpathmoveto{\pgfqpoint{1.146349in}{0.971221in}}%
\pgfpathlineto{\pgfqpoint{1.185818in}{0.997594in}}%
\pgfpathlineto{\pgfqpoint{1.147589in}{1.017744in}}%
\pgfpathlineto{\pgfqpoint{1.107891in}{0.997594in}}%
\pgfpathclose%
\pgfusepath{fill}%
\end{pgfscope}%
\begin{pgfscope}%
\pgfpathrectangle{\pgfqpoint{0.150000in}{0.150000in}}{\pgfqpoint{2.700000in}{1.950000in}}%
\pgfusepath{clip}%
\pgfsetbuttcap%
\pgfsetroundjoin%
\definecolor{currentfill}{rgb}{0.953355,0.959099,0.967142}%
\pgfsetfillcolor{currentfill}%
\pgfsetlinewidth{0.000000pt}%
\definecolor{currentstroke}{rgb}{0.000000,0.000000,0.000000}%
\pgfsetstrokecolor{currentstroke}%
\pgfsetdash{}{0pt}%
\pgfpathmoveto{\pgfqpoint{1.263945in}{0.830867in}}%
\pgfpathlineto{\pgfqpoint{1.303182in}{0.857368in}}%
\pgfpathlineto{\pgfqpoint{1.263522in}{0.926410in}}%
\pgfpathlineto{\pgfqpoint{1.224121in}{0.899924in}}%
\pgfpathclose%
\pgfusepath{fill}%
\end{pgfscope}%
\begin{pgfscope}%
\pgfpathrectangle{\pgfqpoint{0.150000in}{0.150000in}}{\pgfqpoint{2.700000in}{1.950000in}}%
\pgfusepath{clip}%
\pgfsetbuttcap%
\pgfsetroundjoin%
\definecolor{currentfill}{rgb}{0.884942,0.899112,0.918949}%
\pgfsetfillcolor{currentfill}%
\pgfsetlinewidth{0.000000pt}%
\definecolor{currentstroke}{rgb}{0.000000,0.000000,0.000000}%
\pgfsetstrokecolor{currentstroke}%
\pgfsetdash{}{0pt}%
\pgfpathmoveto{\pgfqpoint{1.263522in}{0.926410in}}%
\pgfpathlineto{\pgfqpoint{1.302820in}{0.952828in}}%
\pgfpathlineto{\pgfqpoint{1.263905in}{0.991447in}}%
\pgfpathlineto{\pgfqpoint{1.224376in}{0.971221in}}%
\pgfpathclose%
\pgfusepath{fill}%
\end{pgfscope}%
\begin{pgfscope}%
\pgfpathrectangle{\pgfqpoint{0.150000in}{0.150000in}}{\pgfqpoint{2.700000in}{1.950000in}}%
\pgfusepath{clip}%
\pgfsetbuttcap%
\pgfsetroundjoin%
\definecolor{currentfill}{rgb}{0.866284,0.882751,0.905806}%
\pgfsetfillcolor{currentfill}%
\pgfsetlinewidth{0.000000pt}%
\definecolor{currentstroke}{rgb}{0.000000,0.000000,0.000000}%
\pgfsetstrokecolor{currentstroke}%
\pgfsetdash{}{0pt}%
\pgfpathmoveto{\pgfqpoint{1.184907in}{0.944781in}}%
\pgfpathlineto{\pgfqpoint{1.224376in}{0.971221in}}%
\pgfpathlineto{\pgfqpoint{1.185818in}{0.997594in}}%
\pgfpathlineto{\pgfqpoint{1.146349in}{0.971221in}}%
\pgfpathclose%
\pgfusepath{fill}%
\end{pgfscope}%
\begin{pgfscope}%
\pgfpathrectangle{\pgfqpoint{0.150000in}{0.150000in}}{\pgfqpoint{2.700000in}{1.950000in}}%
\pgfusepath{clip}%
\pgfsetbuttcap%
\pgfsetroundjoin%
\definecolor{currentfill}{rgb}{0.891161,0.904565,0.923330}%
\pgfsetfillcolor{currentfill}%
\pgfsetlinewidth{0.000000pt}%
\definecolor{currentstroke}{rgb}{0.000000,0.000000,0.000000}%
\pgfsetstrokecolor{currentstroke}%
\pgfsetdash{}{0pt}%
\pgfpathmoveto{\pgfqpoint{1.224121in}{0.899924in}}%
\pgfpathlineto{\pgfqpoint{1.263522in}{0.926410in}}%
\pgfpathlineto{\pgfqpoint{1.224376in}{0.971221in}}%
\pgfpathlineto{\pgfqpoint{1.184907in}{0.944781in}}%
\pgfpathclose%
\pgfusepath{fill}%
\end{pgfscope}%
\end{pgfpicture}%
\makeatother%
\endgroup%
}
        \caption[Game Values of Parameterised Incompetent Games]{The dependence of the game value $\val(G_{\lambda, \mu})$ on learning parameters $\lambda, \mu \in [0, 1]$ for each parameterised incompetent game defined in \autoref{tab:parameterised-incompetent-games}. Generated using \texttt{incompetent\_game\_plot.py}.}
        \label{fig:parameterised-incompetent-games}
        \end{minipage}
    \end{figure}