% ---------------------------------------------------------------------------- %
% Honours Thesis                                                               %
% Chapter 4 - Incremental Learning                                             %
% ---------------------------------------------------------------------------- %

\chapter{Incremental Learning}  \label{chp:incremental-learning}
    It is often necessary to model the interaction of learning and playing through complex mechanisms.
    Consider, for instance, an athlete who must train between competitions or a poker player who must practice between tournaments.
    How can we describe these situations game theoretically?
    Our proposed solution is an incremental learning game in which, between repeated plays of an incompetent game, a player may increment their learning parameters to modify their incompetence.

    We will begin in \autoref{sec:stochastic-games} by explaining stochastic games, the general framework used in the formulation of incremental learning.
    Then, \autoref{sec:incremental-learning-games} defines the incremental learning model and \autoref{sec:backward-induction} and \autoref{sec:finding-extensions} propose a backward induction algorithm for computing equilibria.
    Lastly, this procedure is applied in \autoref{sec:incremental-learning-in-tennis} to analyse learning strategies in a simplified tennis game.



\section{Stochastic Games} \label{sec:stochastic-games}
    Firstly, before adding incremental learning to an incompetent game, we should explain the stochastic game model introduced by Shapley \parencite{Shapley1953}.
    This is a mathematical framework for describing situations wherein multiple bimatrix games are played sequentially with player-influenced transitions.
    We will adopt the conventions used in Chapter 3 and Chapter 4 of \parencite{Filar1997} to accommodate some additional notation and, in particular, player association in a stochastic game will be indicated using a superscript.

    A two-player \emph{stochastic game} $\Gamma = (\calS, \calA, \calB, r^1, r^2, p)$ evolves over a collection of \emph{stages} $t = 0, 1, 2, \ldots$ and visits a \emph{state} $S_t$ at every stage.
    \nomenclature[F, 01]{$\Gamma$}{A stochastic game. \nomrefpage}%
    We view $\{S_t\}_{t = 0}^{\infty}$ as a stochastic process that takes its values from a finite set $\calS$ called the \emph{state space}.
    \nomenclature[F, 02]{$\calS$}{The state space of $\Gamma$. \nomrefpage}%
    \nomenclature[F, 03]{$S_t$}{The state realised at stage $t = 0, 1, 2, \ldots$ of $\Gamma$. \nomrefpage}%
    If the game is in state $s \in \calS$, then Player 1 and Player 2 must simultaneously choose \emph{actions} from the finite sets $\calA(s)$ and $\calB(s)$, respectively.
    \nomenclature[F, 04]{$\calA(s)$}{The set of actions belonging to Player 1 at $s \in \calS$ in $\Gamma$. \nomrefpage}%
    \nomenclature[F, 05]{$\calB(s)$}{The set of actions belonging to Player 2 at $s \in \calS$ in $\Gamma$. \nomrefpage}%
    These choices are captured in the stochastic processes $\{A_t\}_{t = 0}^\infty$ and $\{B_t\}_{t = 0}^\infty$ where, at any stage $t = 0, 1, 2, \ldots$, the random variable $A_t$ gives Player 1's action and the random variable $B_t$ gives Player 2's action.
    \nomenclature[F, 06]{$A_t$}{The action realised by Player 1 at state $t = 0, 1, 2, \ldots$ of $\Gamma$. \nomrefpage}%
    \nomenclature[F, 07]{$B_t$}{The action realised by Player 2 at state $t = 0, 1, 2, \ldots$ of $\Gamma$. \nomrefpage}%
    Consequently, the event $\{S_t = s, A_t = a, B_t = b\}$ corresponds to Player 1 choosing action $a \in \calA(s)$ and Player 2 choosing action $b \in \calB(s)$ at stage $t = 0, 1, 2, \ldots$ and in state $s \in \calS$.

    After a pair of actions have been selected, the players are awarded utility and the game transitions to a potentially different state.
    Suppose that, at the stage $t = 0, 1, 2, \ldots$, the game is in state $S_t = s \in \calS$, Player 1's action is $A_t = a \in \calA(s)$, and Player 2's action is $B_t = b \in \calB(s)$.
    The \emph{immediate utility} allocations are denoted by $r^1(s, a, b)$ for Player 1 and $r^2(s, a, b)$ for Player 2.
    \nomenclature[F, 08]{$r^k$}{The immediate utility function belonging to Player $k \in \{1, 2\}$ in $\Gamma$. \nomrefpage}%
    Then, the next state $S_{t + 1}$ is determined randomly with the \emph{transition probability}
    \begin{equation} \label{eq:transition-probabilities}
        p(s' | s, a, b)
            = \PP (S_{t + 1} = s' | S_t = s, A_t = a, B_t = b)
    \end{equation}
    being the probability that $s' \in \calS$ is drawn.
    \nomenclature[F, 09]{$p$}{The transition probability function of $\Gamma$. \nomrefpage}%
    The quantity $p(s' | s, a, b)$ is well-defined because it is assumed that the transition dynamics are Markovian; that is, they are   calculated using only the current state $S_t$, Player 1's action $A_t$, and Player 2's action $B_t$.

    We are mostly interested in the space of \emph{stationary strategies}, which are strategies that depend only on the current state.\footnote{This
        definition of a strategy can be generalised to obtain Markov strategies, which may depend on the current state, and behaviour strategies, which may depend on the game's history.
        Fortunately, a suitable equilibrium solution always exists in the space of stationary strategies for the games we consider (see, for example, \parencite[Theorem 3.1.1, Theorem 4.6.4]{Filar1997}).
    }
    Typically, the block row vectors $\vec{f} = (\vec{f}(s))_{s \in \calS}$ and $\vec{g} = (\vec{g}(s))_{s \in \calS}$ represent stationary strategies belonging to Player 1 and Player 2, respectively.
    If $s \in \calS$ is the game's current state, then the stochastic row vectors $\vec{f}(s) = (f(s, a))_{a \in \calA(s)}$ and $\vec{g}(s) = (g(s, b))_{b \in \calB(s)}$ contain the probability $f(s, a)$ that Player 1 chooses action $a \in \calA(s)$ and the probability $g(s, b)$ that Player 2 chooses action $b \in \calB(s)$.\footnote{
        We are implicitly imposing a canonical ordering on the state space $\calS$ and, for all $s \in \calS$, the action sets $\calA(s)$ and $\calB(s)$.
        This allows $\calS$, $\calA(s)$, and $\calB(s)$ to be used as index sets for various vectors and matrices.
    }
    Aligning with our previous terminology for normal-form games, we call $\vec{f}$ and $\vec{g}$ \emph{pure stationary strategies} whenever $f(s, a) \in \{0, 1\}$ and $g(s, b) \in \{0, 1\}$ for every $s \in \calS$, $a \in \calA(s)$, and $b \in \calB(s)$.
    The set of available stationary strategies is denoted by $\vec{F}$ for Player 1 and $\vec{G}$ for Player 2.
    \nomenclature[F, 10]{$\vec{F}$}{The set of stationary strategies belonging to Player 1 in $\Gamma$. \nomrefpage}%
    \nomenclature[F, 11]{$\vec{G}$}{The set of stationary strategies belonging to Player 2 in $\Gamma$. \nomrefpage}%

    Fix a strategy profile $(\vec{f}, \vec{g}) \in \vec{F} \times \vec{G}$.
    Seeing that the players' behaviours are entirely determined by these strategies, we extend the immediate utility of Player $k = 1, 2$ to
    \begin{equation}  \label{eq:expected-immediate-utility}
        r^k(s, \vec{f}, \vec{g})
            = \sum_{a \in \calA(s)} \sum_{b \in \calB(s)} f(s, a) r^k(s, a, b) g(s, b)
    \end{equation}
    and the transition probabilities to
    \begin{equation}  \label{eq:expected-transition-probabilities}
        p(s' | s, a, b)
            = \sum_{a \in \calA(s)} \sum_{b \in \calB(s)} f(s, a) p(s' | s, a, b) g(s, b)
    \end{equation}
    for all $s, s' \in \calS$.
    Evidently, since these probabilities depend only on the present state, the process $\{S_t\}_{t = 0}^\infty$ becomes a Markov chain whose one-step probability transition matrix is $P(\vec{f}, \vec{g}) = (p(s' | s, \vec{f}, \vec{g}))_{s, s' \in \calS}$.
    This allows us to encode the streams of immediate utility rewards for Player 1 and Player 2 as stochastic processes $\{R^1_t\}_{t = 0}^\infty$ and $\{R^2_t\}_{t = 0}^\infty$.
    \nomenclature[F, 12]{$R^k$}{The utility received by Player $k \in \{1, 2\}$ at stage $t = 0, 1, 2, \ldots$ of $\Gamma$. \nomrefpage}%
    So, after starting at state $s \in \calS$, the expected utility allocation for Player $k = 1, 2$ at state $t = 0, 1, 2, \ldots$ is
    \begin{equation}  \label{eq:utility-process-expectation}
    \begin{split} 
        \EE_{s \vec{f} \vec{g}} \big[R^k_t\big]
            & = \sum_{s' \in \calS} \EE_{s \vec{f} \vec{g}} \big[ R^k_t \big| S_t = s' \big] \PP_{s \vec{f} \vec{g}} (S_t = s) \\
            & = \sum_{s' \in \calS} r^k(s', \vec{f}, \vec{g}) P(\vec{f}, \vec{g})^t [s, s']
    \end{split}
    \end{equation}
    where $\PP_{s \vec{f} \vec{g}}$ is the probability measure induced by the dynamics of $(\vec{f}, \vec{g})$ with initial state $S_0 = s$.
    How can this be used to value an arbitrary strategy profile $(\vec{f}, \vec{g}) \in \vec{F} \times \vec{G}$?
    Naively, we want to compute the expected total utility received throughout the game; however, the summations
    \[
        \sum_{t = 0}^\infty \EE_{s \vec{f} \vec{g}} \big[ R^1_t \big]
        \quad\text{and}\quad
        \sum_{t = 0}^\infty \EE_{s \vec{f} \vec{g}} \big[ R^2_t \big]
    \]
    might not converge.

    Instead, consider a two-player \emph{discounted stochastic game} $\Gamma_\beta$ wherein future utility rewards are progressively diminished by a \emph{discount factor} $\beta \in [0, 1)$.
    \nomenclature[F, 13]{$\Gamma_\beta$}{A $\beta$-discounted stochastic game. \nomrefpage}%
    The \emph{discounted value} of $(\vec{f}, \vec{g}) \in \vec{F} \times \vec{G}$ to Player $k = 1, 2$ is
    \begin{equation}  \label{eq:discounted-value}
        v_\beta^k (s, \vec{f}, \vec{g})
            = \sum_{t = 0}^\infty \beta^t \EE_{s \vec{f} \vec{g}} \big[ R^k_t \big]
    \end{equation}
    after starting at the initial state $s \in \calS$.\footnote{A
        different solution to this problem is to construct a limiting average stochastic game $\Gamma_\ms{\alpha}$ (see \parencite[Section 3.4, Chapter 5]{Filar1997}).
        Here, the value of a strategy profile is the limit of its average rewards over finite time horizons.
        This alternative approach is ignored because, unlike in discounted stochastic games, the existence of stationary equilibria is not guaranteed in limiting average stochastic games (see \parencite[Example 3.4.1]{Filar1997}).
    }
    \nomenclature[F, 14]{$v_\sbeta^k$}{The value function belonging to Player $k \in \{1, 2\}$ in $\Gamma_\beta$. \nomrefpage}%
    The convergence of the summation in \eqref{eq:discounted-value} is guaranteed because the sequence of expected utilities is bounded between the minimum and maximum utility allotments.
    We call the vector $\vec{v}_\sbeta^k(\vec{f}, \vec{g}) = (v_\sbeta^k(s, \vec{f}, \vec{g}))_{s \in \calS}$ the \emph{discounted value vector} of $(\vec{f}, \vec{g}) \in \vec{F} \times \vec{G}$ to Player $k = 1, 2$.
    \nomenclature[F, 15]{$\vec{v}_\sbeta^k$}{The value vector function belonging to Player $k \in \{1, 2\}$ in $\Gamma_\beta$. \nomrefpage}%
    Applying this valuation of player strategies, we say that a strategy profile $(\vec{f}^*, \vec{g}^*) \in \vec{F} \times \vec{G}$ is a \emph{Nash equilibrium} of $\Gamma_\beta$ whenever the componentwise inequalities
    \begin{equation}  \label{eq:stochastic-nash-equilibrium}
        \vec{v}_\beta^1(\vec{f}, \vec{g}^*)
            \le \vec{v}_\beta^1(\vec{f}^*, \vec{g}^*)
        \quad\text{and}\quad
        \vec{v}_\beta^2(\vec{f}^*, \vec{g})
            \le \vec{v}_\beta^2(\vec{f}^*, \vec{g}^*)
    \end{equation}
    hold for every $\vec{f} \in \vec{F}$ and $\vec{g} \in \vec{G}$.
    Notice that, by requiring that the inequalities are satisfied regardless of the initial state, this condition eliminates incredible threats and mirrors the subgame perfect equilibrium in extensive-form games.

    The existence of stationary equilibria in two-player zero-sum stochastic games---where $r^1(s, a, b) + r^2(s, a, b) = 0$ for all $s \in \calS$, $a \in \calA(s)$, and $b \in \calB(s)$---was established by Shapley \parencite{Shapley1953}.
    Again, if $(\vec{f}, \vec{g}) \in \vec{F} \times \vec{G}$ is a Nash equilibrium of a zero-sum stochastic game, then $\vec{f}^*$ and $\vec{g}^*$ are called \emph{optimal strategies}.
    Generally, we will not require that our stochastic games are zero-sum and, as a consequence, we must leverage Fink's \parencite{Fink1964} result showing that a stationary equilibrium always exists in a finite-player general-sum stochastic game.



\section{Incremental Learning Games} \label{sec:incremental-learning-games}
    We are now prepared to model the process of incremental learning in incompetent games.
    Consider a parameterised incompetent game $G_\ms{Q_1(\funcdot), Q_2(\funcdot)}$ with learning trajectories $Q_1 : [0, 1] \to \RR^{m_1 \times m_1}$ and $Q_2 : [0, 1] \to \RR^{m_2 \times m_2}$.
    An incremental learning game allows Player 1 and Player 2 to increment their learning parameters through the ordered sets 
    \[
        \Lambda 
            = \{\lambda_1, \lambda_2, \ldots, \lambda_{n_1}\} \subset [0, 1]
        \quad\text{and}\quad
        \Mu 
            = \{\mu_1, \mu_2, \ldots, \mu_{n_2}\} \subset [0, 1]
    \]
    for some $n_1, n_2 \in \ZZ^+$.
    \nomenclature[G, 01]{$\Gamma_\beta$}{A $\beta$-discounted incremental learning game. \nomrefpage}%
    \nomenclature[G, 02]{$\Lambda$}{The set of attainable parameters belonging to Player 1 in $\Gamma_\beta$. \nomrefpage}%
    \nomenclature[G, 03]{$\Mu$}{The set of attainable parameters belonging to Player 2 in $\Gamma_\beta$. \nomrefpage}%
    \nomenclature[G, 04]{$n_k$}{The number of attainable parameters belonging to Player $k \in \{1, 2\}$. \nomrefpage}%
    Recall that the learning parameters $\lambda_i$ and $\mu_j$ correspond to the incompetence matrices $Q_1(\lambda_i)$ and $Q_2(\mu_j)$, as defined in \autoref{sec:incompetent-games}.
    The elements of $\Lambda$ and $\Mu$ are called \emph{attainable learning parameters}.

    Fix $i \in \{1, 2, \ldots, n_1\}$ and $j \in \{1, 2, \ldots, n_2\}$ such that $\lambda_i$ is Player 1's current learning parameter and $\mu_j$ is Player 2's current learning parameter.
    We divide gameplay into two distinct phases: a \emph{playing phase} and a \emph{learning phase}.
    First, the playing phase involves playing the incompetent game $G_{\lambda_i, \mu_j}$ and receiving the utility allotments associated with its realised outcome.
    Second, unless a player's attainable learning parameters have been exhausted, the learning phase gives Player 1 and Player 2 the option to advance their learning parameters to $\lambda_{i + 1}$ and $\mu_{j + 1}$, respectively.
    The decision to increment a learning parameter might incur state-dependent \emph{learning costs} $c^1(\lambda_i, \mu_j)$ and $c^2(\lambda_i, \mu_j)$.
    \nomenclature[G, 05]{$c^k$}{The cost function belonging to Player $k \in \{1, 2\}$ in $\Gamma_\beta$. \nomrefpage}%
    This process is repeated using the updated learning parameters and utility is gradually accumulated over time.
    The structure of an incremental learning game is illustrated in \autoref{fig:incremental-learning-game}, which represents learning parameter pairs as nodes and possible transitions as arcs.
    Note that a transition can only occur as a result of actions in the learning phase.

    \begin{figure}[t]
        \centering
        % ---------------------------------------------------------------------------- %
% Honours Thesis                                                               %
% Figure: Structure of an Incremental Learning Game                            %
% ---------------------------------------------------------------------------- %

\begin{tikzpicture}
    % Vertex Style
    \tikzset{vertex/.style = {
        draw,
        shape=circle,
        inner sep=0pt,
        minimum size=7pt
    }}
    
    % Empty Vertex Style
    \tikzset{empty/.style = {
        inner sep=0pt,
        minimum size=7pt,=
    }}

    % Edge Style
    \tikzset{edge/.style = {
        ->,
        > = latex',
        shorten <=7pt,
        shorten >=7pt
    }}

    % Diagonal Edge Style
    \tikzset{diagonal/.style = {
        ->,
        > = latex',
        shorten <=10pt,
        shorten >=10pt
    }}

    % Nodes
    \foreach \i in {0,1,2,3,4,6}
        \foreach \j in {0,1,2,3,4,6}
            \node[vertex] (\i, \j) at (\i, \j) {};

    \foreach \i in {0,1,2,3,4,5,6}
        \node[empty] (\i, 4) at (\i, 4) {};

    \foreach \j in {0,1,2,3,4,6}
        \node[empty] (4, \j) at (4, \j) {};

    % Edges
    \foreach \i in {0,1,2,3,4,5}
        \foreach \j in {0,1,2,3,4,6}
            \draw[edge] (\i, \j) to (\i + 1, \j);
    
    \foreach \j in {0,1,2,3,4,5}
        \foreach \i in {0,1,2,3,4,6}
            \draw[edge] (\i, \j) to (\i, \j + 1);
    
    \foreach \j in {0,1,2,3,4,5}
        \foreach \i in {\j,...,5}
            \draw[diagonal] (\i, \j) to (\i + 1, \j + 1);
    
    \foreach \j in {1,2,3,4,5}
        \foreach \i in {1,...,\j}
            \draw[diagonal] (\i - 1, \j) to (\i, \j + 1);

    \foreach \i in {0,1,2,3,4,6}
        \foreach \j in {0,1,2,3,4,6}
            \draw[edge] (\i, \j) to [out=330,in=300,loop] (\i, \j);

    % Labels
    \node[rotate=90] (P1) at (-1.5, 3) {Player 1};
    \node[] (L1) at (-0.5, 0) {$\lambda_1$};
    \node[] (L2) at (-0.5, 1) {$\lambda_2$};
    \node[] (L3) at (-0.5, 2) {$\lambda_3$};
    \node[] (L4) at (-0.5, 3) {$\lambda_4$};
    \node[] (L5) at (-0.5, 4) {$\lambda_5$};
    \node[] (L6) at (-0.5, 6) {$\lambda_{n_1}$};

    \node[] (P2) at (3, 7.5) {Player 2};
    \node[] (M1) at (0, 6.5) {$\mu_1$};
    \node[] (M2) at (1, 6.5) {$\mu_2$};
    \node[] (M3) at (2, 6.5) {$\mu_3$};
    \node[] (M4) at (3, 6.5) {$\mu_4$};
    \node[] (M5) at (4, 6.5) {$\mu_5$};
    \node[] (M6) at (6, 6.5) {$\mu_{n_2}$};
\end{tikzpicture}
        \caption[Structure of an Incremental Learning Game]{The transition structure of a general incremental learning game.}
        \label{fig:incremental-learning-game}
    \end{figure}

    We are going to address incremental learning games with an infinite time horizon.
    The motivation, as it appears in \parencite[Section 2.2]{Filar1997}, behind our exclusion of finite time horizons is twofold:
    \begin{itemize}
        \item a stochastic game with a ``short'' time horizon can already be solved easily using dynammic programming, and
        \item a stochastic game with a ``long'' time horizon is computationally expensive to solve using the same method.
    \end{itemize}
    An alternative approach to finding equilibria in a ``long'' time horizon stochastic game is to approximate it over an infinite time horizon and leverage the various mathematical programming techniques capable of solving these models.
    This brings the added benefit of producing stationary strategies, which are often easier to implement than the time-dependent strategies produced when solving stochastic games with a finite time horizon \parencite{Filar1997}.

    Next, to develop a formal description of incremental learning, we define an \emph{incremental learning game} as a stochastic game $\Gamma$ constructed by the following process.
    The state space
    \[
        \calS 
            = \big\{(i, j) : i = 1, 2, \ldots, n_1 \text{ and } j = 1, 2, \ldots, n_2\big\}
    \]
    is chosen to index the attainable learning parameters; the state $(i, j) \in \calS$ corresponds to the parameters $(\lambda_i, \mu_j) \in \Lambda \times \Mu$.
    Fix a state $s = (i, j) \in \calS$.
    It is convenient to simplify our notation by writing $i$ instead of $\lambda_i$ and $j$ instead $\mu_j$ whenever learning parameters are referenced.
    A player's action should consist of an action in the playing phase and an action in the learning phase.
    So, Player 1's action set is
    \[
        \calA(s)
            = 
            \begin{cases}
                \{1, 2, \ldots, m_1\} \times \{0, 1\}, & i \neq n_1, \\
                \{1, 2, \ldots, m_1\} \times \{0\}, & i = n_1, \\
            \end{cases}
    \]
    and Player 2's action set is
    \[
        \calB(s)
            = 
            \begin{cases}
                \{1, 2, \ldots, m_2\} \times \{0, 1\}, & j \neq n_2, \\
                \{1, 2, \ldots, m_2\} \times \{0\}, & j = n_2. \\
            \end{cases}
    \]
    If $a = (a_P, a_L) \in \calA(s)$ and $b = (b_P, b_L) \in \calB(s)$ are selected, then $a_P, b_P$ are interpreted as playing phase actions and $a_L, b_L$ are interpreted as learning phase actions.
    These actions award the utilities
    \[
        r^1(s, a, b)
            = u_{i, j}(a_P, b_P) - a_L c^1(i, j)
        \quad\text{and}\quad
        r^2(s, a, b)
            = -u_{i, j}(a_P, b_P) - b_L c^2(i, j)
    \]
    to Player 1 and Player 2, respectively.\footnote{Clearly,
        an incremental learning game is zero-sum if and only if there are no learning costs, or $c^1(i, j) = 0$ and $c^2(i, j) = 0$ for all $i = 1, 2, \ldots, n_1$ and $j = 1, 2, \ldots, n_2$.
    }
    Furthermore, they cause a guaranteed transition to the next state $(i + a_L, j + b_L)$ such that the transition probabilities are given by
    \[
        p(s' | s, a, b)
            =
            \begin{cases}
                1, & i' = i + a_L \text{ and } j' = j + b_L, \\
                0, & i' \neq i + a_L \text{ or } j' \neq j + b_L, \\
            \end{cases}
    \]
    for every $s' = (i', j') \in \calS$.
    The incremental learning game $\Gamma$ is defined as the stochastic game $(\calS, \calA, \calB, r^1, r^2, p)$.
    Again, to value a strategy profile, we use the \emph{discounted incremental learning game} $\Gamma_\beta$ with a predetermined discount factor $\beta \in [0, 1)$.

    Notice that, for all $s \in \calS$ and $(a, b) \in \calA(s) \times \calB(s)$, the utilities $r^1(s, a, b)$ and $r^2(s, a, b)$ contain two terms: one that depends only on the playing phase actions and one that depends only on the learning phase actions.
    Thus, since these different types of actions do not interact, they are selected using independent probability distributions.
    Fix a state $s = (i, j) \in \calS$ and a strategy profile $(\vec{f}, \vec{g}) \in \vec{F} \times \vec{G}$.
    We know that Player 1 chooses an action $a = (a_P, a_L)$ by independently drawing $a_P$ from the distribution $\vec{f}_P(s)$ over $\{1, 2, \ldots, m_1\}$ and $a_L$ from the distribution $\vec{f}_L(s)$ over $\{0, 1\}$.
    Similarly, Player 2 chooses an action $b = (b_P, b_L)$ by independently drawing $b_P$ from the distribution $\vec{g}_P(s)$ over $\{1, 2, \ldots, m_2\}$ and $b_L$ from the distribution $\vec{g}_L(s)$ over $\{0, 1\}$.
    This decomposition of player strategies is exploited in \autoref{prop:playing-phase-optimality} to show that, given an equilibrium $(\vec{f}^*, \vec{g}^*) \in \vec{F} \times \vec{G}$ of $\Gamma_{\beta}$, the strategy profile $(\vec{f}^*_P(s), \vec{g}^*_P(s))$ is an equilibrium of $G_{i, j}$.

    \begin{proposition} \label{prop:playing-phase-optimality}
        Let the strategy profile $(\vec{f}^*, \vec{g}^*) \in \vec{F} \times \vec{G}$ be an equilibrium of the discounted incremental learning game $\Gamma_\beta$.
        Then, at any state $s = (i, j) \in \calS$, the strategy profile $(\vec{f}^*_P(s), \vec{g}^*_P(s))$ is an equilibrium of the incompetent game $G_{i, j}$.
        
    \end{proposition}

    \begin{proof}
        Consider a strategy profile $(\vec{f}, \vec{g}) \in \vec{F} \times \vec{G}$ that is identical to $(\vec{f}^*, \vec{g}^*)$ except at $\vec{f}_P(s) \neq \vec{f}^*_P(s)$ and $\vec{g}_P(s) \neq \vec{g}^*_P(s)$.
        Observe that, after applying \eqref{eq:expected-immediate-utility} and \eqref{eq:discounted-value}, we obtain
        \begin{equation}  \label{temp:playing-phase-optimality-1}
        \begin{split}
            v^k_\beta(& s, \vec{f}, \vec{g}) - v^k_\beta(s, \vec{f}^*, \vec{g}^*)
                = \sum_{t = 0}^\infty \beta^t \big( \EE_{s \vec{f} \vec{g}} \big[R^k_t\big] - \EE_{s \vec{f}^* \vec{g}^*} \big[R^k_t\big]\big) \\
                & = \sum_{t = 0}^\infty \sum_{s' \in \calS} \beta^t \big( \EE_{s \vec{f} \vec{g}} \big[ R^k_t \big| S_t = s' \big] \PP_{s \vec{f} \vec{g}} (S_t = s') - \EE_{s \vec{f}^* \vec{g}^*} \big[ R^k_t \big| S_t = s' \big] \PP_{s \vec{f}^* \vec{g}^*} (S_t = s')\big).
        \end{split}
        \end{equation}
        Now, since $\vec{f}_L(s') = \vec{f}^*_L(s')$ and $\vec{g}_L(s') = \vec{g}^*_L(s')$ for all $s' \in \calS$, the strategy profiles $(\vec{f}, \vec{g})$ and $(\vec{f}^*, \vec{g}^*)$ induce the same transition dynamics such that $\PP_{s \vec{f} \vec{g}} = \PP_{s \vec{f}^* \vec{g}^*}$.
        Moreover, the similarities between $(\vec{f}, \vec{g})$ and $(\vec{f}^*, \vec{g}^*)$ imply that their expected playing phase utility is equal on $\calS \setminus \{s\}$ and their expected learning phase utility is equal on $\calS$.
        This allows us to reduce \eqref{temp:playing-phase-optimality-1} to
        \begin{equation}  \label{temp:playing-phase-optimality-2}
        \begin{split}
            v^k_\beta(& s, \vec{f}, \vec{g}) - v^k_\beta(s, \vec{f}^*, \vec{g}^*) \\
                & = \sum_{t = 0}^\infty \sum_{s' \in \calS} \beta^t \PP_{s \vec{f}^* \vec{g}^*} (S_t = s') \big( \EE_{s \vec{f} \vec{g}} \big[ R^k_t \big| S_t = s' \big] - \EE_{s \vec{f}^* \vec{g}^*} \big[ R^k_t \big| S_t = s' \big]\big) \\
                & = (-1)^{k - 1} \big( v_{i, j}(\vec{f}_P(s), \vec{g}_P(s)) - v_{i, j}(\vec{f}^*_P(s), \vec{g}^*_P(s)) \big) \sum_{t = 0}^\infty \beta^t \PP_{s \vec{f}^* \vec{g}^*} (S_t = s)
        \end{split}
        \end{equation}
        because the only remaining contribution is from the difference in expected playing phase utilities at the state $s$.
        Note that $\PP_{s \vec{f}^* \vec{g}^*}(S_0 = s) = 1$ and, as a consequence, the summation in \eqref{temp:playing-phase-optimality-2} is strictly positive.
        So,
        \begin{equation} \label{temp:playing-phase-optimality-3}
            v_\beta^1(s, \vec{f}, \vec{g})
                \le v_\beta^1(s, \vec{f}^*, \vec{g}^*)
            \quad\text{implies}\quad
            v_{i, j}(\vec{f}_P(s), \vec{g}_P(s))
                \le v_{i, j}(\vec{f}^*_P(s), \vec{g}^*_P(s))
        \end{equation}
        and
        \begin{equation} \label{temp:playing-phase-optimality-4}
            v_\beta^2(s, \vec{f}, \vec{g})
                \le v_\beta^2(s, \vec{f}^*, \vec{g}^*)
            \quad\text{implies}\quad
            v_{i, j}(\vec{f}_P(s), \vec{g}_P(s))
                \ge v_{i, j}(\vec{f}^*_P(s), \vec{g}^*_P(s)).
        \end{equation}
        Set $\vec{g} = \vec{g}^*$ in \eqref{temp:playing-phase-optimality-3} and $\vec{f} = \vec{f}^*$ in \eqref{temp:playing-phase-optimality-4}.
        Then, by the stochastic game equilibrium inequalities for $(\vec{f}^*, \vec{g}^*)$ in $\Gamma_\beta$, we conclude that $(\vec{f}^*_P(s), \vec{g}^*_P(s))$ satisfies the matrix game equilibrium inequalities in $G_{i, j}$.
    \end{proof}

    The message of \autoref{prop:playing-phase-optimality} is that, whenever we are interested in finding equilibria of a discounted incremental learning game, we are able to assume that both players always select optimal strategies of $G_{i, j}$ in the playing phase at state $s = (i, j) \in \calS$.
    Next, this realisation is used to simplify the description of incremental learning and ``remove'' the playing phase.

    Fix an arbitrary state $s = (i, j) \in \calS$.
    Noting that each player's behaviour during a playing phase is entirely determined by \autoref{prop:playing-phase-optimality}, they only need to select a learning phase action from the sets
    \[
        \calA(s)
            =
            \begin{cases}
                \{0, 1\}, & i \neq n_1, \\
                \{0\}, & i = n_1, \\
            \end{cases}
        \quad\text{and}\quad
        \calB(s)
            =
            \begin{cases}
                \{0, 1\}, & j \neq n_2, \\
                \{0\}, & j = n_2, \\
            \end{cases}.
    \]
    If Player 1 selects  $a \in \calA(s)$ and Player 2 selects $b \in \calB(s)$, then they receive the immediate utilities
    \[
        r^1(s, a, b)
            = \val\big(G_{i, j}\big) - a c^1(i, j)
        \quad\text{and}\quad
        r^2(s, a, b)
            = -\val\big(G_{i, j}\big) - b c^2(i, j),
    \]
    respectively.
    Additionally, the game transitions to the state $(i + a, j + b)$ and the transition probabilities are given by
    \[
        p(s' | s, a, b)
            =
            \begin{cases}
                1, & i' = i + a \text{ and } j' = j + b, \\
                0, & i' \neq i + a \text{ or } j' \neq j + b, \\
            \end{cases}
    \]
    for all $s' = (i', j') \in \calS$.
    Henceforth, we will use this simplified formulation to describe an incremental learning game.
    Although \autoref{prop:playing-phase-optimality} shows that both formulations are equivalent, the simplified version does not explicitly model the playing phase behaviour.
    The playing phase equilibrium strategies at the state $s = (i, j) \in \calS$ can be found separately by computing the optimal strategies of $G_{i, j}$.



\section{Backward Induction} \label{sec:backward-induction}
    After developing a mathematical model that captures the features of incremental learning in incompetent games, our immediate task is to identify a procedure to compute its equilibrium solutions.
    Typically, a single equilibrium of a general-sum stochastic game can be found using nonlinear programming (see, for example, \parencite[Section 3.8]{Filar1997}); however, this approach complicates the process of finding multiple equilibria.
    Instead, we will propose a modified backward induction algorithm that, under certain conditions, is capable of exhaustively identifying every equilibrium in a discounted incremental learning game. 

    The process of \emph{backward induction}, which develops a rational strategy by reasoning backward through time, is commonly used to solve game-theoretic problems.
    Consider a finite extensive-form game with perfect information; that is, a game wherein every information set is a singleton.
    Here, backward induction produces a subgame perfect equilibrium by proceeding upward through the game tree and assigning optimal actions contingent on future play \parencite{Osborne1994}.
    \autoref{fig:backward-induction} shows a backward induction solution to a variant of ``Battle of the Sexes'' with perfect information.
    A selected action is indicated by a solid arc between nodes and an unselected action is indicated by a dashed arc between nodes.
    Notice that, at every decision point, the controlling player is always maximising their utility conditional on the already determined future behaviour.

    \begin{figure}[t]
        \centering
        % ---------------------------------------------------------------------------- %
% Honours Thesis                                                               %
% Figure: Backward Induction in Battle of the Sexes                            %
% ---------------------------------------------------------------------------- %


\begin{istgame}[font=\small]
    % Grow East
    \setistgrowdirection'{east}



    % Player 1
    \xtdistance{30mm}{25mm}

    \istroot(0)<[yshift=15pt]>{Player 1}
        % Football
        \istb[draw=color2]{Football}[above, sloped, yshift=-2pt, font=\scriptsize]
        
        % Concert
        \istb[dotted]{Concert}[above, sloped, yshift=-2pt, font=\scriptsize]
    \endist



    % Player 2
    \xtdistance{30mm}{12mm}
    
    \istroot(1)(0-1)<[yshift=15pt]>{Player 2}
        % Football
        \istb[draw=color4]{Football}[above, sloped, yshift=-2pt, font=\scriptsize]{$(2, 1)$}
        
        % Concert
        \istb[dotted]{Concert}[above, sloped, yshift=-2pt, font=\scriptsize]{$(0, 0)$}
    \endist

    \istroot(2)(0-2){}
        % Football 
        \istb[dotted]{Football}[above, sloped, yshift=-2pt, font=\scriptsize]{$(0, 0)$}
        
        % Concert
        \istb[draw=color4]{Concert}[above, sloped, yshift=-2pt, font=\scriptsize]{$(1, 2)$}
    \endist
    
    % Information Sets
    \xtInfosetO[fill=color2!25](0)(0) % Player 1
    \xtInfosetO[fill=color4!25](1)(1) % Player 2
    \xtInfosetO[fill=color4!25](2)(2) % Player 2

\end{istgame}
        \caption[Backward Induction in ``Battle of the Sexes'']{A backward induction solution to perfect-information ``Battle of the Sexes''.}
        \label{fig:backward-induction}
    \end{figure}

    A discounted incremental learning game, unlike the previous example of ``Battle of the Sexes'', cannot be solved using a standard backward induction procedure because it has infinitely many stages and no ``last'' stage to serve as a starting point.
    So, as an alternative to iterating through the game's stages, we will build a stationary equilibrium $(\vec{f}^*, \vec{g}^*) \in \vec{F} \times \vec{G}$ by iterating through the game's states.
    Precisely, a formal description of this procedure requires an ordering $s_1, s_2, \ldots, s_n$ (with $n = n_1 n_2$) of the state space $\calS$ and, for all $\ell = 1, 2, \ldots, n$, a method to find $(\vec{f}^*(s_\ell), \vec{g}^*(s_\ell))$ given an equilibrium of $\Gamma_\beta$ restricted to $\{s_{\ell + 1}, s_{\ell + 2}, \ldots, s_n\}$.

    First, to construct a suitable notion of ``past'' and ``future'' states, we want a sequence $s_1, s_2, \ldots, s_n$ of states such that
    \begin{equation}  \label{eq:state-ordering-condition}
        \ell' < \ell
        \quad\text{implies}\quad
        p(s_{\ell'} | s_\ell, a, b) = 0
    \end{equation}
    for every $\ell, \ell' = 1, 2, \ldots, n$ and $(a, b) \in \calA(s_\ell) \times \calB(s_\ell)$.
    The purpose of the condition in \eqref{eq:state-ordering-condition} is to ensure that a state $s_\ell$ can never transition to a preceding state $s_{\ell'}$.
    Equivalently, we want a topological ordering of the directed graph $D = (V, E)$ where $V = \calS$ and
    \[
        E 
        =
        \big\{
        (s, s') \in \calS \times \calS 
        : s \neq s' \text{ and } p(s' | s, a, b) > 0 \text{ for some } (a, b) \in \calA(s) \times \calB(s)
        \big\},
    \]
    A \emph{topological ordering} in $D$ is a sequence of states such that $s$ appears before $s'$ for all $(s, s') \in E$.
    This sequence exists if and only if $D$ is an acyclic directed graph \parencite{Erciyes2018}.
    Therefore, having constructed $D$ to explicitly exclude self-loops, the structure of an incremental learning game (see \autoref{fig:incremental-learning-game}) suggests that a topological ordering is possible.
    Indeed, we can always use the lexicographic ordering $s_1, s_2, \ldots, s_n$ where, for any states $s_\ell = (i, j) \in \calS$ and $s_{\ell'} = (i', j') \in \calS$ with $\ell, \ell' = 1, 2, \ldots, n$, we have
    \begin{equation} \label{eq:lexicographic ordering}
        \ell < \ell'
        \quad\text{if and only if}\quad
        (i < i') \text{ or } (i = i' \text{ and } j < j').
    \end{equation}
    It is straightforward to check that, when arranged in lexicographic order, the successors $(i + 1, j)$, $(i, j + 1)$, and $(i + 1, j + 1)$ must appear after $(i, j)$.\footnote{Although
        a lexicographic ordering of the state space satisfies the required conditions, it is often possible to backward induct through the states in a different order.
        A general algorithm to compute topological orderings for an arbitrary acyclic directed graph is given by \parencite[Algorithm 6.11]{Erciyes2018}.
    }
    So, assuming that a suitable ordering has been selected, we will simplify our notation by relabelling the state $s_\ell$ as $\ell$ for every $\ell = 1, 2, \ldots, n$.

    \begin{proposition}  \label{prop:general-ordered-state-value}
        Let $(\vec{f}, \vec{g}) \in \vec{F} \times \vec{G}$ be a strategy profile in the discounted incremental learning game $\Gamma_\beta$.
        Then, for any $\ell = 1, 2, \ldots, n$ and $k = 1, 2$, we have
        \begin{equation}  \label{eq:general-ordered-state-value}
            v_\beta^k(\ell, \vec{f}, \vec{g})
                = \frac{r^k(\ell, \vec{f}, \vec{g}) + \beta \sum_{\ell' = \ell + 1}^n v_\beta^k(\ell', \vec{f}, \vec{g}) p(\ell' | \ell, \vec{f}, \vec{g})}{1 - \beta p(\ell | \ell, \vec{f}, \vec{g})}.
        \end{equation}
    \end{proposition}

    \begin{proof} 
        Observe that, by conditioning on the outcome of the random variable $S_1$, the discounted value of $(\vec{f}, \vec{g})$ at $\ell$ can be written as
        \begin{equation} \label{temp:general-ordered-state-value-1}
        \begin{split}
            v_\beta^k(\ell, \vec{f}, \vec{g})
                &= \sum_{t = 0}^\infty \beta^t \EE_{\ell \vec{f} \vec{g}} \big[ R^k_t \big] \\
                &= \EE_{\ell \vec{f} \vec{g}} \big[ R^k_0 \big] + \sum_{t = 0}^\infty \sum_{\ell' = 1}^n \beta^t \EE_{\ell \vec{f} \vec{g}} \big[ R^k_t \big| S_1 = \ell' \big] \PP_{\ell \vec{f} \vec{g}}\big(S_1 = \ell'\big) \\
                & = \EE_{\ell \vec{f} \vec{g}} \big[ R^k_0 \big] + \sum_{\ell' = 1}^n  \PP_{\ell \vec{f} \vec{g}} \big(S_1 = \ell'\big) \sum_{t = 1}^\infty \beta^t \EE_{\ell \vec{f} \vec{g}} \big[ R^k_t \big| S_1 = \ell' \big] \\
                & = \EE_{\ell \vec{f} \vec{g}} \big[ R^k_0 \big] + \beta \sum_{\ell' = 1}^n  \PP_{\ell \vec{f} \vec{g}} \big(S_1 = \ell'\big) \sum_{t = 1}^\infty \beta^t \EE_{\ell' \vec{f} \vec{g}} \big[ R^k_t \big] \\
                &=  r^k(\ell, \vec{f}, \vec{g}) + \beta \sum_{\ell' = 1}^n v^k_\beta(\ell', \vec{f}, \vec{g}) p(\ell' | \ell, \vec{f}, \vec{g}).
        \end{split}
        \end{equation}
        Of course, an immediate consequence of the ordering condition in \eqref{eq:state-ordering-condition} is that, for all $\ell' = 1, 2, \ldots, \ell - 1$, we obtain
        \begin{equation} \label{temp:general-ordered-state-value-2}
            p(\ell' | \ell, \vec{f}, \vec{g})
                = \sum_{a \in \calA(\ell)} \sum_{b \in \calB(\ell)} f(\ell, a) p(\ell' | \ell, a, b) g(\ell, b)
                = 0
        \end{equation}
        and
        \begin{equation} \label{temp:general-ordered-state-value-3}
            v_\beta^k(\ell, \vec{f}, \vec{g})
                = r^k(\ell, \vec{f}, \vec{g}) + \beta \sum_{\ell' = \ell}^n v_\beta^k(\ell', \vec{f}, \vec{g}) p(\ell' | \ell, \vec{f}, \vec{g}).
        \end{equation}
        Lastly, we can isolate the $v^k_\sbeta(\ell, \vec{f}, \vec{g})$ terms on the left-hand side of \eqref{temp:general-ordered-state-value-3} to obtain \eqref{eq:general-ordered-state-value}, as required.
    \end{proof}

    Fix an index $\ell = 1, 2, \ldots, n$.
    We are allowed to restrict the discounted incremental learning game $\Gamma_\beta$ to the smaller state space $\calS_\ell = \{\ell, \ell + 1, \ldots, n\}$.
    Why?
    The ordering condition in \eqref{eq:state-ordering-condition} guarantees that the excluded states cannot be accessed from within this restricted game.
    Thus, as in \autoref{prop:general-ordered-state-value}, a value can be assigned to incomplete strategy profiles that only specify a player's behaviour on $\calS_\ell$.
    Precisely, \emph{incomplete strategies} belonging to Player 1 and Player 2 are block row vectors
    \[
        \vec{f}_\ell
            = \big(\vec{f}_\ell(\ell')\big)_{\ell' = \ell}^n
        \quad\text{and}\quad
        \vec{g}_\ell
            = \big(\vec{g}_\ell(\ell')\big)_{\ell' = \ell}^n
    \]
    where, for all $\ell' = \ell, \ell + 1, \ldots, n$, the blocks
    \[
        \vec{f}_\ell(\ell')
            = \big(f_\ell(\ell', a)\big)_{a \in \calA(\ell')}
        \quad\text{and}\quad
        \vec{g}_\ell(\ell')
            = \big(g_\ell(\ell', b)\big)_{b \in \calB(\ell')}
    \]
    are stochastic row vectors.
    The set of Player 1's incomplete strategies is denoted by $\vec{F}_\ell$ and the set of Player 2's incomplete strategies is denoted by $\vec{G}_\ell$.
    Clearly, using our previous observation, the discounted value $v_\sbeta^k(\ell', \vec{f}_\ell, \vec{g}_\ell)$ of $(\vec{f}_\ell, \vec{g}_\ell) \in \vec{F}_\ell \times \vec{G}_\ell$ to Player $k = 1, 2$ is well-defined for every $\ell' = \ell, \ell + 1, \ldots, n$.
    An \emph{incomplete equilibrium} $(\vec{f}^*_\ell, \vec{g}^*_\ell) \in \vec{F}_\ell \times \vec{G}_\ell$ satisfies
    \begin{equation}  \label{eq:incomplete-nash-equilibrium}
        v_\beta^1\big(\ell', \vec{f}_\ell, \vec{g}^*_\ell\big)
            \le v_\beta^1\big(\ell', \vec{f}^*_\ell, \vec{g}^*_\ell\big)
        \quad\text{and}\quad
        v_\beta^2\big(\ell', \vec{f}^*_\ell, \vec{g}_\ell\big)
            \le v_\beta^2\big(\ell', \vec{f}^*_\ell, \vec{g}^*_\ell\big)
    \end{equation}
    for all states $\ell' = \ell, \ell + 1, \ldots, n$, Player 1's alternatives $\vec{f}_\ell \in \vec{F}_\ell$, and Player 2's alternatives $\vec{g}_\ell \in \vec{G}_\ell$.

    Now, since a backward induction algorithm builds an equilibrium by progressively adding to an incomplete equilibrium, some additional notation is needed to mathematically describe this process.
    Fix an index $\ell = 1, 2, \ldots, n - 1$.
    A strategy $\vec{f}_\ell \in \vec{F}_\ell$ is said to \emph{extend} $\vec{f}_{\ell + 1} \in \vec{F}_{\ell + 1}$ whenever $\vec{f}_\ell(\ell') = \vec{f}_{\ell + 1}(\ell')$ for all $\ell' = \ell + 1, \ell + 2, \ldots, n$.
    Similarly, a strategy $\vec{g}_\ell \in \vec{G}_\ell$ is said to \emph{extend} $\vec{g}_{\ell + 1} \in \vec{G}_{\ell + 1}$ whenever $\vec{g}_{\ell}(\ell') = \vec{g}_{\ell + 1}(\ell')$ for all $\ell' = \ell + 1, \ell + 2, \ldots, n$.
    The sets of strategies that extend $\vec{f}_{\ell + 1} \in \vec{F}_{\ell + 1}$ and $\vec{g}_{\ell + 1} \in \vec{G}_{\ell + 1}$ are denoted by $\vec{F}_\ell(\vec{f}_{\ell + 1})$ and $\vec{G}_\ell(\vec{g}_{\ell + 1})$, respectively.

    The modified backward induction procedure is outlined in \autoref{alg:incremental-learning-backward-induction} using this terminology.
    Although the problem of finding a suitable extension in Step 2 is not resolved until \autoref{sec:finding-extensions}, we assume that a method exists that is capable of computing these extensions.
    Below, \autoref{thm:backward-induction-verification} verifies that, depending on the choice of extensions, any stationary equilibrium of $\Gamma_\beta$ can be returned by \autoref{alg:incremental-learning-backward-induction}.

    \begin{algorithm} \label{alg:incremental-learning-backward-induction}
    \begin{enumerate}[
        leftmargin=*,
        align=left,
        label=\textbf{Step \arabic*.}
    ]
        \item[]
    
        \item[\textbf{Input.}] An incremental learning game $\Gamma_\beta$ with a state space $\calS = \{s_1, s_2, \ldots, s_n\}$ satisfying the condition in \eqref{eq:state-ordering-condition}.
        
        \item[\textbf{Output.}] An equilibrium $(\vec{f}^*, \vec{g}^*) \in \vec{F} \times \vec{G}$ of $\Gamma_\beta$.
        
        \item (\textit{Initialisation}) Set $(\vec{f}^*_n, \vec{g}^*_n) \in \vec{F}_n \times \vec{G}_n$ such that $f^*_n(n, 0) = 1$ and $g^*_n(n, 0) = 1$.

        \item (\textit{Extension}) Next, for each $\ell = n - 1, n - 2, \ldots, 1$, find a strategy profile $(\vec{f}^*_\ell, \vec{g}^*_\ell) \in \vec{F}_\ell(\vec{f}_{\ell + 1}) \times \vec{G}_\ell(\vec{g}_{\ell + 1})$ satisfying the inequalities
        \begin{equation}  \label{eq:extension-nash-equilibrium}
            v_\beta^1\big(\ell, \vec{f}_\ell, \vec{g}^*_\ell\big)
                \le v_\beta^1\big(\ell, \vec{f}^*_\ell, \vec{g}^*_\ell\big)
            \quad\text{and}\quad
            v_\beta^2\big(\ell, \vec{f}^*_\ell, \vec{g}_\ell\big)
                \le v_\beta^2\big(\ell, \vec{f}^*_\ell, \vec{g}^*_\ell\big)
        \end{equation}
        for all $\vec{f}_\ell \in \vec{F}_\ell(\vec{f}_{\ell + 1})$ and $\vec{g}_\ell \in \vec{G}_\ell(\vec{g}_{\ell + 1})$.

        \item (\textit{Result}) Return  $(\vec{f}^*_1, \vec{g}^*_1)$ as an equilibrium of $\Gamma_\beta$.
    \end{enumerate}
    \end{algorithm}

    \begin{theorem}  \label{thm:backward-induction-verification}
        Assume that we are able to find every solution to Step 2 in \autoref{alg:incremental-learning-backward-induction}.
        Then, a strategy profile $(\vec{f}^*, \vec{g}^*) \in \vec{F} \times \vec{G}$ is an equilibrium of $\Gamma_\beta$ if and only if it can be returned in Step 3.
    \end{theorem}

    \begin{proof}
        We want to prove that, for any $\ell = 1, 2, \ldots, n$, the strategy profile $(\vec{f}^*_\ell, \vec{g}^*_\ell) \in \vec{F}_\ell \times \vec{G}_\ell$ is an equilibrium of $\Gamma_\beta$ restricted to $\calS_\ell$ if and only if it can be produced during the execution of \autoref{alg:incremental-learning-backward-induction}.
        First, in the base case ($\ell = n$), the only available actions are $\calA(n) = \{0\}$ and $\calB(n) = \{0\}$.
        Therefore, because the unique incomplete equilibrium $(\vec{f}^*_n, \vec{g}^*_n) \in \vec{F}_n \times \vec{G}_n$ has $f^*_n(n, 0) = 1$ and $g^*_n(n, 0) = 1$, the previous assertion holds trivially.

        Second, taking an arbitrary index $\ell = n - 1, n - 2, \ldots, 1$, we will assume the validity of the inductive hypothesis for $\ell + 1$; that is, $(\vec{f}^*_{\ell + 1}, \vec{g}^*_{\ell + 1}) \in \vec{F}_{\ell + 1} \times \vec{G}_{\ell + 1}$ is an incomplete equilibrium if and only if it is produced during the execution of \autoref{alg:incremental-learning-backward-induction}.
        It remains to be shown that, under this assumption, the assertion also holds for the preceding state $\ell$.

        ($\Longrightarrow$)
        Suppose $(\vec{f}^*_\ell, \vec{g}^*_\ell) \in \vec{F}_\ell \times \vec{G}_\ell$ is an incomplete equilibrium of the discounted incremental learning game.
        Define $(\vec{f}^*_{\ell + 1}, \vec{g}^*_{\ell + 1}) \in \vec{F}_{\ell + 1} \times \vec{G}_{\ell + 1}$ such that
        \[
            \vec{f}^*_{\ell + 1}(\ell')
                = \vec{f}^*_\ell(\ell')
            \quad\text{and}\quad
            \vec{g}^*_{\ell + 1}(\ell')
                = \vec{g}^*_\ell(\ell')
        \]
        for all $\ell' = \ell + 1, \ell + 2, \ldots, n$.
        Note that $\vec{f}^*_\ell \in \vec{F}_\ell(\vec{f}^*_{\ell + 1})$ and $\vec{g}^*_\ell \in \vec{G}_\ell(\vec{g}^*_{\ell + 1})$.
        We know that $(\vec{f}^*_{\ell + 1}, \vec{g}^*_{\ell + 1})$ satisfies the incomplete equilibrium inequalities in \eqref{eq:incomplete-nash-equilibrium} and, by the inductive hypothesis, it can be produced during the previous iteration of the extension procedure.
        Additionally, since $(\vec{f}^*_\ell, \vec{g}^*_\ell)$ satisfies the extension conditions in \eqref{eq:extension-nash-equilibrium}, it can be produced as a solution to Step 2, as required.

        ($\Longleftarrow$)
        Assume that, during the previous extension iteration, an incomplete strategy profile $(\vec{f}^*_{\ell + 1}, \vec{g}^*_{\ell + 1}) \in \vec{F}_{\ell + 1} \times \vec{G}_{\ell + 1}$ is created.
        Obviously, by the inductive hypothesis, this strategy profile is an incomplete equilibrium of the discounted incremental learning game.
        Find a suitable extension $(\vec{f}^*_\ell, \vec{g}^*_\ell) \in \vec{F}_\ell(\vec{f}^*_{\ell + 1}) \times \vec{G}_\ell(\vec{g}^*_{\ell + 1})$.
        We already know that $(\vec{f}^*_\ell, \vec{g}^*_\ell)$ satisfies the equilibrium inequalities at $\ell + 1, \ell + 2, \ldots, n$, so it only remains to be shown that these conditions hold at the state $\ell$.
        Observe that, when $\ell' = \ell + 1, \ell + 2, \ldots, n$, we have
        \begin{equation}  \label{temp:backward-induction-verification-1}
            v^1_\beta\big(\ell', \vec{f}'_\ell, \vec{g}^*_\ell\big)
                \le v^1_\beta\big(\ell', \vec{f}^*_{\ell + 1}, \vec{g}^*_{\ell + 1}\big)
                = v^1_\beta\big(\ell', \vec{f}_\ell, \vec{g}^*_\ell\big)
        \end{equation}
        for all $\vec{f}_\ell \in \vec{F}_\ell(\vec{f}^*_{\ell + 1})$ and $\vec{f}'_\ell \in \vec{F}_\ell$ with $\vec{f}_\ell(\ell) = \vec{f}'_\ell(\ell)$ and
        \begin{equation}  \label{temp:backward-induction-verification-2}
            v^2_\beta\big(\ell', \vec{f}^*_\ell, \vec{g}'_\ell\big)
                \le v^2_\beta\big(\ell', \vec{f}^*_{\ell + 1}, \vec{g}^*_{\ell + 1}\big)
                = v^2_\beta\big(\ell', \vec{f}^*_\ell, \vec{g}_\ell\big)
        \end{equation}
        for all $\vec{g}_\ell \in \vec{G}_\ell(\vec{g}^*_{\ell + 1})$ and $\vec{g}'_\ell \in \vec{G}_\ell$ with $\vec{g}_\ell(\ell) = \vec{g}'_\ell(\ell)$.
        Then, as an immediate consequence of \eqref{temp:backward-induction-verification-1} and $\vec{f}_\ell(\ell) = \vec{f}'_\ell(\ell)$, it follows that
        \begin{equation}  \label{temp:backward-induction-verification-3}
        \begin{split}
            v^1_\beta\big(\ell, \vec{f}'_\ell, \vec{g}^*_\ell\big)
                & = \frac{r^1\big(\ell, \vec{f}'_\ell, \vec{g}^*_\ell\big) + \beta \sum_{\ell' = \ell + 1}^n v^1_\beta\big(\ell', \vec{f}'_\ell, \vec{g}^*_\ell\big) p\big(\ell' \big| \ell, \vec{f}'_\ell, \vec{g}^*_\ell\big)}{1 - \beta p\big(\ell \big| \ell, \vec{f}'_\ell, \vec{g}^*_\ell\big)} \\
                & \le \frac{r^1\big(\ell, \vec{f}_\ell, \vec{g}^*_\ell\big) + \beta \sum_{\ell' = \ell + 1}^n v^1_\beta\big(\ell', \vec{f}_\ell, \vec{g}^*_\ell\big) p\big(\ell' \big| \ell, \vec{f}_\ell, \vec{g}^*_\ell\big)}{1 - \beta p\big(\ell \big| \ell, \vec{f}_\ell, \vec{g}^*_\ell\big)}
                = v^1_\beta\big(\ell, \vec{f}_\ell, \vec{g}^*_\ell\big).
        \end{split}
        \end{equation}
        Analogously, as an immediate consequence of \eqref{temp:backward-induction-verification-2} and $\vec{g}_\ell(\ell) = \vec{g}'_\ell(\ell)$, we obtain
        \begin{equation}  \label{temp:backward-induction-verification-4}
        \begin{split}
            v^2_\beta\big(\ell, \vec{f}^*_\ell, \vec{g}'_\ell\big)
                & = \frac{r^2\big(\ell, \vec{f}^*_\ell, \vec{g}'_\ell\big) + \beta \sum_{\ell' = \ell + 1}^n v^2_\beta\big(\ell', \vec{f}^*_\ell, \vec{g}'_\ell\big) p\big(\ell' \big| \ell, \vec{f}^*_\ell, \vec{g}'_\ell\big)}{1 - \beta p\big(\ell \big| \ell, \vec{f}^*_\ell, \vec{g}'_\ell\big)} \\
                & \le \frac{r^2\big(\ell, \vec{f}^*_\ell, \vec{g}_\ell\big) + \beta \sum_{\ell' = \ell + 1}^n v^2_\beta\big(\ell', \vec{f}^*_\ell, \vec{g}_\ell\big) p\big(\ell' \big| \ell, \vec{f}^*_\ell, \vec{g}_\ell\big)}{1 - \beta p\big(\ell \big| \ell, \vec{f}^*_\ell, \vec{g}_\ell\big)}
                = v^2_\beta\big(\ell, \vec{f}^*_\ell, \vec{g}_\ell\big).
        \end{split}
        \end{equation}
        This means that, for any $\vec{f}'_\ell \in \vec{F}_\ell$ and $\vec{g}'_\ell \in \vec{F}_\ell$, the strategy profile $(\vec{f}^*_\ell, \vec{g}^*_\ell)$ satisfies the familiar equilibrium inequalities
        \begin{equation}  \label{temp:backward-induction-verification-5}
            v^1_\beta\big(\ell, \vec{f}'_\ell, \vec{g}^*_\ell\big)
                \le v^1_\beta\big(\ell, \vec{f}_\ell, \vec{g}^*_\ell\big)
                \le v^1_\beta\big(\ell, \vec{f}^*_\ell, \vec{g}^*_\ell\big)
        \end{equation}
        and
        \begin{equation}  \label{temp:backward-induction-verification-6}
            v^2_\beta\big(\ell, \vec{f}^*_\ell, \vec{g}'_\ell\big)
                \le v^2_\beta\big(\ell, \vec{f}^*_\ell, \vec{g}_\ell\big)
                \le v^2_\beta\big(\ell, \vec{f}^*_\ell, \vec{g}^*_\ell\big)
        \end{equation}
        where $\vec{f}_\ell \in \vec{F}_\ell(\vec{f}^*_{\ell + 1})$ and $\vec{g}_\ell \in \vec{G}_\ell(\vec{g}^*_{\ell + 1})$ are chosen such that $\vec{f}_\ell(\ell) = \vec{f}'_\ell(\ell)$ and $\vec{g}_\ell(\ell) = \vec{g}'_\ell(\ell)$.
        Hence, the extended strategy profile $(\vec{f}^*_\ell, \vec{g}^*_\ell)$ is an incomplete equilibrium of the incremental learning game.

        We conclude by noting that $(\vec{f}^*_1, \vec{g}^*_1) \in \vec{F}_1 \times \vec{G}_1$, or equivalently $(\vec{f}^*, \vec{g}^*) \in \vec{F} \times \vec{G}$, is an equilibrium of $\Gamma_\beta$ if and only if it can be returned at the termination of \autoref{alg:incremental-learning-backward-induction}.
    \end{proof}


\section{Finding Extensions} \label{sec:finding-extensions}
    Lastly, before implementing \autoref{alg:incremental-learning-backward-induction}, we must identify a method for solving the inequalities in \eqref{eq:extension-nash-equilibrium}.
    Suppose that the modified backward induction algorithm has reached the current state $s_\ell = (i, j)$ with $\ell = 1, 2, \ldots, n - 1$ and, during previous iterations of the extension method, has produced an incomplete equilibrium $(\vec{f}^*_{\ell + 1}, \vec{g}^*_{\ell + 1}) \in \vec{F}_{\ell + 1} \times \vec{G}_{\ell + 1}$.
    Player 1's set of available actions is either $\calA(\ell) = \{0\}$ or $\calA(\ell) = \{0, 1\}$ and Player 2's set of available actions is either $\calB(\ell) = \{0\}$ or $\calB(\ell) = \{0, 1\}$.
    So, the generic extensions $\vec{f}_\ell \in \vec{F}_{\ell}(\vec{f}^*_{\ell + 1})$ and $\vec{g}_\ell \in \vec{G}_{\ell}(\vec{g}^*_{\ell + 1})$ are entirely determined by the probabilities
    \[
        f_\ell(\ell, 0) 
            = p_0
            \in [0, 1]
        \quad\text{and}\quad
        g_\ell(\ell, 0) 
            = q_0
            \in [0, 1]
    \]
    of forgoing learning.
    Define $p_1 = 1 - p_0$ and $q_1 = 1 - q_0$ for the sake of notational compactness.
    Also, letting $s_{a, b} = (i + a, b + j)$ for all $a \in \calA(\ell)$ and $b \in \calB(\ell)$, define some additional auxiliary quantities
    \begin{equation}  \label{eq:extension-constants}
        V^k_{a, b}
            =
            \begin{cases}
                r^k(\ell, a, b) + \beta v^k_\beta\big(s_{a, b}, \vec{f}_{\ell + 1}, \vec{g}_{\ell + 1}\big), & (a, b) \neq (0, 0), \\
                r^k(\ell, a, b), & (a, b) = (0, 0), \\
            \end{cases}
    \end{equation}
    for all $k = 1, 2$ and $(a, b) \in \calA(\ell) \times \calB(\ell)$.
    \autoref{lem:ordered-state-value} expresses the discounted value of $(\vec{f}_\ell, \vec{g}_\ell)$ in terms of the implemented strategies.
    Note that, since the expressions in \eqref{eq:extension-constants} do not depend on the current strategies $\vec{f}_\ell(\ell)$ or $\vec{g}_\ell(\ell)$, they can be treated as constants throughout the process of extending the incomplete equilibrium.

    \begin{lemma}  \label{lem:ordered-state-value}
        Fix a state $\ell = 1, 2, \ldots, n - 1$ and an incomplete equilibrium $(\vec{f}^*_{\ell + 1}, \vec{g}^*_{\ell + 1}) \in \vec{F}_{\ell + 1} \times \vec{G}_{\ell + 1}$.
        Then,
        \begin{equation} \label{eq:ordered-state-value}
            v^k_\beta(\ell, \vec{f}_\ell, \vec{g}_\ell)
                = \frac{1}{1 - \beta p_0 q_0} \sum_{a \in \calA(\ell)} \sum_{b \in \calB(\ell)} p_a V^k_{a, b} q_b
        \end{equation}
        is the discounted value of $(\vec{f}_\ell, \vec{g}_\ell) \in \vec{F}_\ell(\vec{f}^*_{\ell + 1}) \times \vec{G}_\ell(\vec{g}^*_{\ell + 1})$ to Player $k = 1, 2$.
    \end{lemma}

    \begin{proof}
        Recall from \eqref{eq:expected-immediate-utility} that the expected immediate utility of $(\vec{f}_\ell, \vec{g}_\ell)$ to Player $k = 1, 2$ is
        \begin{equation}  \label{temp:ordered-state-value-1}
            r^k(\ell, \vec{f}_\ell, \vec{g}_\ell)
                = \sum_{a \in \calA(\ell)} \sum_{b \in \calB(\ell)} f_\ell(\ell, a) r^k(\ell, a, b) g_\ell(\ell, b)
                = \sum_{a \in \calA(\ell)} \sum_{b \in \calB(\ell)} p_a r^k(\ell, a, b) q_b
        \end{equation}
        and from \eqref{eq:expected-transition-probabilities} that the transition probabilities are
        \begin{equation}  \label{temp:ordered-state-value-2}
        \begin{split}
            p(\ell' | \ell, \vec{f}_\ell, \vec{g}_\ell)
                &=
                \begin{cases}
                    f_\ell(\ell, a) g_\ell(\ell, b), & i' = i + a \text{ for some } a \in \calA(\ell) \text{ and} \\
                                                     & j' = j + b \text{ for some } b \in \calB(\ell), \\
                    0, & \text{otherwise}, \\
                \end{cases} \\
                &=
                \begin{cases}
                    p_a q_b, & i' = i + a \text{ for some } a \in \calA(\ell) \text{ and}\\
                                                     & j' = j + b \text{ for some } b \in \calB(\ell), \\
                    0, & \text{otherwise}, \\
                \end{cases} \\
        \end{split}
        \end{equation}
        for all $\ell' = 1, 2, \ldots, n$ with $s_\ell = (i, j)$ and $s_{\ell'} = (i', j')$.
        Observe that, after considering the possible actions that can be selected to reach the future states, we have
        \begin{equation} \label{temp:ordered-state-value-3}
            \sum_{\ell' = \ell + 1}^n v_\beta^k(\ell', \vec{f}_\ell, \vec{g}_\ell) p(\ell' | \ell, \vec{f}_\ell, \vec{g}_\ell)
                = \mathop{\displaystyle\sum_{a \in \calA(\ell)} \displaystyle\sum_{b \in \calB(\ell)}}_{(a, b) \neq (0, 0)} p_a v^k_\beta(s_{a, b}, \vec{f}_\ell, \vec{g}_\ell) q_b.
        \end{equation}
        Substitute \eqref{temp:ordered-state-value-1}, \eqref{temp:ordered-state-value-2}, and \eqref{temp:ordered-state-value-3} into the expression for the discounted value of $(\vec{f}_\ell, \vec{g}_\ell)$ in \eqref{eq:general-ordered-state-value} to obtain
        \begin{equation}  \label{temp:ordered-state-value-4}
        \begin{split}
            v^k_\beta(\ell, \vec{f}_\ell, \vec{g}_\ell)
                &= \frac{r^k(\ell, \vec{f}_\ell, \vec{g}_\ell) + \beta \sum_{\ell' = \ell + 1}^n v_\beta^k(\ell', \vec{f}_\ell, \vec{g}_\ell) p(\ell' | \ell, \vec{f}_\ell, \vec{g}_\ell)}{1 - \beta p(\ell | \ell, \vec{f}_\ell, \vec{g}_\ell)} \\
                &= \frac{1}{1 - \beta p_0 q_0} \left(\vcenter{\hbox{$\displaystyle\sum_{a \in \calA(\ell)} \displaystyle\sum_{b \in \calB(\ell)} p_a r^k(\ell, a, b) q_b + \beta \mathop{\displaystyle\sum_{a \in \calA(\ell)} \displaystyle\sum_{b \in \calB(\ell)}}_{(a, b) \neq (0, 0)} p_a v^k_\beta(s_{a, b}, \vec{f}_\ell, \vec{g}_\ell) q_b$}}\right) \\
                &= \frac{1}{1 - \beta p_0 q_0} \sum_{a \in \calA(\ell)} \sum_{b \in \calB(\ell)} p_a V^k_{a, b} q_b.
        \end{split}
        \end{equation}
        Clearly, \eqref{temp:ordered-state-value-4} gives the desired valuation of the strategy profile $(\vec{f}_\ell, \vec{g}_\ell)$ to Player $k = 1, 2$.
    \end{proof}

    Next, we will reformulate the inequalities in \eqref{eq:extension-nash-equilibrium} to develop a coupled pair of equivalent maximisation problems.
    If $(\vec{f}^*_\ell, \vec{g}^*_\ell) \in \vec{F}_\ell(\vec{f}^*_{\ell + 1}) \times \vec{G}_\ell(\vec{g}^*_{\ell + 1})$ is an incomplete equilibrium, then we will write
    \[
        f^*_\ell(\ell, 0) 
            = p^*_0 
            \in [0, 1]
        \quad\text{and}\quad
        g^*_\ell(\ell, 0)
            = q^*_0
            \in [0, 1]
    \]
    with $p^*_1 = 1 - p^*_0$ and $q^*_1 = 1 - q^*_0$.
    Evidently, from the valuation of player strategies in \autoref{lem:ordered-state-value}, the strategy profile $(\vec{f}^*_\ell, \vec{g}^*_\ell)$ satisfies \eqref{eq:extension-nash-equilibrium} if and only if the probabilities $p^*_0$ and $q^*_0$ simultaneously solve the maximisation problems
    \begin{equation}  \label{eq:extension-maximisation-1}
        p^*_0
            = \argmax_{\vec{f}_\ell \in \vec{F}_\ell(\vec{f}^*_{\ell + 1})} v^1_\beta\big(\ell, \vec{f}_\ell, \vec{g}^*_\ell\big)
            = \argmax_{p_0 = 1 - p_1 \in [0, 1]} \frac{1}{1 - \beta p_0 q^*_0} \sum_{a \in \calA(\ell)} \sum_{b \in \calB(\ell)} p_a V^1_{a, b} q^*_b
    \end{equation}
    and
    \begin{equation}  \label{eq:extension-maximisation-2}
        q^*_0
            = \argmax_{\vec{g}_\ell \in \vec{G}_\ell(\vec{g}^*_{\ell + 1})} v^2_\beta\big(\ell, \vec{f}^*_\ell, \vec{g}_\ell\big)
            = \argmax_{q_0 = 1 - q_1 \in [0, 1]} \frac{1}{1 - \beta p^*_0 q_0} \sum_{a \in \calA(\ell)} \sum_{b \in \calB(\ell)} p^*_a V^2_{a, b} q_b.
    \end{equation}
    A method for calculating solutions to \eqref{eq:extension-maximisation-1} and \eqref{eq:extension-maximisation-2} is provided in \autoref{alg:extension-algorithm} and verified in \autoref{prop:extension-algorithm-verification}.
    Note that a \emph{best response} is a strategy that maximises the player's utility conditional on their opponent's selected strategy.
    We claim that this method computes every solution when the discounted incremental learning game is \emph{non-degenerate} at the state $\ell$; that is, when every pure strategy has no completely mixed best responses.
    This assumption ensures that there are finitely many solutions, which allows us to exhaustively find every equilibrium.\footnote{The
        assumption of non-degeneracy is used in several game-theoretic algorithms, including the support enumeration algorithm that inspires \autoref{alg:extension-algorithm} (see, for example, \parencite[Algorithm 3.4]{vonStengel2007}).
    }

    \begin{algorithm}  \label{alg:extension-algorithm}
    \begin{enumerate}[
        leftmargin=*,
        align=left,
        label=\textbf{Step \arabic*.}
    ]
        \item[]
    
        \item[\textbf{Input.}]
        A state $\ell = 1, 2, \ldots, n - 1$ and an incomplete equilibrium $(\vec{f}^*_{\ell + 1}, \vec{g}^*_{\ell + 1}) \in \vec{F}_{\ell + 1} \times \vec{G}_{\ell + 1}$.
        
        \item[\textbf{Output.}]
        A set of strategy profiles $(\vec{f}^*_\ell, \vec{g}^*_\ell) \in \vec{F}_\ell(\vec{f}^*_{\ell + 1}) \times \vec{G}_\ell(\vec{g}^*_{\ell + 1})$ that solve \eqref{eq:extension-nash-equilibrium}.

        \item (\textit{Pure Strategies})
        Return every strategy profile $(\vec{f}^*_\ell, \vec{g}^*_\ell) \in \vec{F}_\ell(\vec{f}^*_{\ell + 1}) \times \vec{G}_\ell(\vec{g}^*_{\ell + 1})$ such that
        \begin{equation}  \label{eq:pure-strategy-extension-condition-1}
            p^*_0
                = \argmax_{p_0 = 1 - p_1 \in \{0, 1\}} \frac{1}{1 - \beta p_0 q^*_0} \sum_{a \in \calA(\ell)} \sum_{b \in \calB(\ell)} p_a V^1_{a, b} q^*_b
        \end{equation}
        and
        \begin{equation} \label{eq:pure-strategy-extension-condition-2}
            q^*_0
                = \argmax_{q_0 = 1 - q_1 \in \{0, 1\}} \frac{1}{1 - \beta p^*_0 q_0} \sum_{a \in \calA(\ell)} \sum_{b \in \calB(\ell)} p^*_a V^2_{a, b} q_b.
        \end{equation}
        Observe that, because $\vec{f}^*_\ell$ and $\vec{g}^*_\ell$ are pure strategies, these conditions can be checked by exhaustion.

        \item (\textit{Mixed Strategies})
        Return every strategy profile $(\vec{f}^*_\ell, \vec{g}^*_\ell) \in \vec{F}_\ell(\vec{f}^*_{\ell + 1}) \times \vec{G}_\ell(\vec{g}^*_{\ell + 1})$ such that
        \begin{equation} \label{eq:mixed-strategy-extension-condition}
            \sum_{a \in \calA(\ell)} \big( p^*_a V_{a, 0}^2 - p^*_a (1 - \beta p^*_0) V_{a, 1}^2 \big) = 0
            \quad\text{and}\quad
            \sum_{b \in \calB(\ell)}  \big( q^*_b V^1_{0, b} - q^*_b (1 - \beta q^*_0) V^1_{1, b} \big) = 0
        \end{equation}
        for some $p^*_0 = 1 - p^*_1 \in (0, 1)$ and $q^*_0 = 1 - q^*_1 \in (0, 1)$.
        These quadratic equations in $p^*_0$ and $q^*_0$ can be solved via the quadratic formula.
    \end{enumerate}
    \end{algorithm}

    \begin{proposition} \label{prop:extension-algorithm-verification}
        If $\Gamma_\beta$ is non-degenerate at the present state $\ell = 1, 2, \ldots, n$, then there are finitely many solutions to the extension conditions in \eqref{eq:extension-nash-equilibrium}.
        Moreover, every solution is returned by \autoref{alg:extension-algorithm}.
    \end{proposition}
    
    \begin{proof}
        First, by the assumption that $\Gamma_\beta$ is non-degenerate, a pure strategy can never have a best response in completely mixed strategies.
        It follows that a pure strategy solution to \eqref{eq:extension-nash-equilibrium} can be found after only checking the necessary inequalities over the space of other pure strategies.
        Hence, every pure strategy solution---of which there are finitely many---is returned during Step 1 of \autoref{alg:extension-algorithm}.

        Second, given that the extension conditions in \eqref{eq:extension-nash-equilibrium} have been reduced to a pair of maximisation problems in \eqref{eq:extension-maximisation-1} and \eqref{eq:extension-maximisation-2}, it is useful to compute the partial derivatives of $v^k_\sbeta(\ell, \vec{f}_\ell, \vec{g}_\ell)$ with respect to $p_0$ and $q_0$.
        So, by applying the quotient rule to the strategy valuation from \autoref{lem:ordered-state-value}, we obtain
        \begin{equation} \label{temp:extension-algorithm-verification-1}
            \frac{\partial}{\partial p_0} v^1_\beta(\ell, \vec{f}_\ell, \vec{g}_\ell)
                = \frac{1}{(1 - \beta p_0 q_0)^2} \sum_{b \in \calB(\ell)}  \big( q_b V^1_{0, b} - q_b (1 - \beta q_0) V^1_{1, b} \big)
        \end{equation}
        and
        \begin{equation} \label{temp:extension-algorithm-verification-2}
            \frac{\partial}{\partial q_0} v^2_\beta(\ell, \vec{f}_\ell, \vec{g}_\ell)
                = \frac{1}{(1 - \beta p_0 q_0)^2} \sum_{a \in \calA(\ell)} \big( p_a V_{a, 0}^2 - p_a (1 - \beta p_0) V_{a, 1}^2 \big).
        \end{equation}
        The signs and zeros of these partial derivatives are entirely determined by the opponent's choice of strategy.
        Explicitly, $q^*_0$ and $p^*_0$ are roots of \eqref{temp:extension-algorithm-verification-1} and \eqref{temp:extension-algorithm-verification-2} if and only if
        \begin{equation} \label{temp:extension-algorithm-verification-3}
            \sum_{a \in \calA(\ell)} \big( p^*_a V_{a, 0}^2 - p^*_a (1 - \beta p^*_0) V_{a, 1}^2 \big) = 0
            \quad\text{and}\quad
            \sum_{b \in \calB(\ell)}  \big( q^*_b V^1_{0, b} - q^*_b (1 - \beta q^*_0) V^1_{1, b} \big) = 0.
        \end{equation}
        Clearly, under the conditions in \eqref{temp:extension-algorithm-verification-3}, the partial derivative of Player 1's valuation in \eqref{temp:extension-algorithm-verification-1} is zero regardless of $p_0$ and the partial derivative of Player 2's valuation in \eqref{temp:extension-algorithm-verification-2} is zero regardless of $q_0$.
        This guarantees that, for all $\vec{f}_\ell \in \vec{F}_\ell(\vec{f}^*_{\ell + 1})$ and $\vec{g}_\ell, \vec{G}_\ell(\vec{g}^*_{\ell + 1})$, we have
        \begin{equation} \label{temp:extension-algorithm-verification-4}
            v^1_\beta\big(\ell, \vec{f}_\ell, \vec{g}^*_\ell\big)
                = v^1_\beta\big(\ell, \vec{f}^*_\ell, \vec{g}^*_\ell\big)
            \quad\text{and}\quad
            v^2_\beta\big(\ell, \vec{f}^*_\ell, \vec{g}_\ell\big)
                = v^2_\beta\big(\ell, \vec{f}^*_\ell, \vec{g}^*_\ell\big).
        \end{equation}
        Thus, since the strategy profile $(\vec{f}^*_\ell, \vec{g}^*_\ell)$ satisfies the extension conditions whenever $p^*_0$ and $q^*_0$ solve \eqref{temp:extension-algorithm-verification-3}, every completely mixed solution to \eqref{eq:extension-nash-equilibrium} is returned in Step 2 of \autoref{alg:extension-algorithm}.
        
        Lastly, to show that there are finitely many solutions in completely mixed strategies, notice that the existence of infinitely many solutions requires either $p^*_0$ or $q^*_0$ to solve \eqref{temp:extension-algorithm-verification-3} for every $p^*_0 \in [0, 1]$ or $q^*_0 \in [0, 1]$.
        If this requirement holds for all $p^*_0 \in [0, 1]$, then Player 2 is always indifferent between their actions irrespective of Player 1's strategy.
        This contradicts our assumption of non-degeneracy because Player 2 has completely mixed best responses to Player 1's pure strategy choices: either $p^*_0 = 0$ and $p^*_0 = 1$.
        Therefore, there are only finitely many solutions to the quadratic equations in Step 2.
    \end{proof}

    Note that, as a consequence of its validity, we can use \autoref{alg:extension-algorithm} to find the solutions to \eqref{eq:extension-nash-equilibrium} in \autoref{alg:incremental-learning-backward-induction}.
    The fact that we are capable of giving every suitable solution to these inequalities means that a backward induction procedure can exhaustively compute the equilibria of a non-degenerate incremental learning game.
    In particular, to find every equilibrium, we can simply branch \autoref{alg:incremental-learning-backward-induction} whenever multiple solutions are returned from \autoref{alg:extension-algorithm}.
    Unfortunately, because there is an uncountably infinite number of solutions to \eqref{eq:extension-nash-equilibrium} in a degenerate game, we cannot possibly return every equilibrium.
    Instead, a compromise might involve only taking the pure strategy solutions whenever a continuum of solutions exists.
    The resulting backward induction procedure, with this compromise included, is implemented using Python in \texttt{incremental\_solver.py}.



\section{Incremental Learning in Tennis} \label{sec:incremental-learning-in-tennis}
    \begin{figure}[b]
        \centering
        % ---------------------------------------------------------------------------- %
% Honours Thesis                                                               %
% Figure: Winning Probabilities in a Simple Tennis Game                        %
% ---------------------------------------------------------------------------- %

\setlength{\extrarowheight}{3pt}

\begin{tabular}{cc|c|c|c|}
                            & \multicolumn{1}{c}{} & \multicolumn{3}{c}{Player 2}                                                            \\
                            & \multicolumn{1}{c}{} & \multicolumn{1}{c}{Good Shot}     & \multicolumn{1}{c}{Safe Shot} & \multicolumn{1}{c}{Out} \\ \cline{3-5}
    \multirow{3}*{Player 1} & Good Shot                 & $50\%, 50\%$                       & $70\%, 30\%$                   & $100\%, 0\%$                      \\ \cline{3-5}
                            & Safe Shot                 & $30\%, 70\%$                     & $50\%, 50\%$                     & $100\%, 0\%$                       \\ \cline{3-5}
                            & Out                       & $0\%, 100\%$                      & $0\%, 100\%$                     & $50\%, 50\%$                       \\ \cline{3-5}
\end{tabular}

\setlength{\extrarowheight}{0pt}
        \caption[Winning Probabilities in a Simple Tennis Game]{The probabilities of Player 1 and Player 2 winning a tennis point for each combination of actions.}
        \label{fig:tennis-point}
    \end{figure}

    Finally, to apply the recently described backward induction algorithm and illustrate an example of equilibrium behaviour in incremental learning games, we will look at a simple tennis game borrowed form \parencite[Section 4.2]{Beck2013}.
    Consider a repeated sequence of tennis points in which Player 1 and Player 2 can select from among the actions ``Good Shot'', ``Safe Shot'', and ``Out''.
    After Player 1 and Player 2 select a pair of actions, the probability of winning the point is determined by \autoref{fig:tennis-point}.

    \begin{figure}[t]
        \centering
        %% Creator: Matplotlib, PGF backend
%%
%% To include the figure in your LaTeX document, write
%%   \input{<filename>.pgf}
%%
%% Make sure the required packages are loaded in your preamble
%%   \usepackage{pgf}
%%
%% Figures using additional raster images can only be included by \input if
%% they are in the same directory as the main LaTeX file. For loading figures
%% from other directories you can use the `import` package
%%   \usepackage{import}
%% and then include the figures with
%%   \import{<path to file>}{<filename>.pgf}
%%
%% Matplotlib used the following preamble
%%   \usepackage{fontspec}
%%   \setmainfont{DejaVuSerif.ttf}[Path=C:/Users/Thomas/anaconda3/lib/site-packages/matplotlib/mpl-data/fonts/ttf/]
%%   \setsansfont{DejaVuSans.ttf}[Path=C:/Users/Thomas/anaconda3/lib/site-packages/matplotlib/mpl-data/fonts/ttf/]
%%   \setmonofont{DejaVuSansMono.ttf}[Path=C:/Users/Thomas/anaconda3/lib/site-packages/matplotlib/mpl-data/fonts/ttf/]
%%
\begingroup%
\makeatletter%
\begin{pgfpicture}%
\pgfpathrectangle{\pgfpointorigin}{\pgfqpoint{5.000000in}{3.750000in}}%
\pgfusepath{use as bounding box, clip}%
\begin{pgfscope}%
\pgfsetbuttcap%
\pgfsetmiterjoin%
\definecolor{currentfill}{rgb}{1.000000,1.000000,1.000000}%
\pgfsetfillcolor{currentfill}%
\pgfsetlinewidth{0.000000pt}%
\definecolor{currentstroke}{rgb}{1.000000,1.000000,1.000000}%
\pgfsetstrokecolor{currentstroke}%
\pgfsetdash{}{0pt}%
\pgfpathmoveto{\pgfqpoint{0.000000in}{0.000000in}}%
\pgfpathlineto{\pgfqpoint{5.000000in}{0.000000in}}%
\pgfpathlineto{\pgfqpoint{5.000000in}{3.750000in}}%
\pgfpathlineto{\pgfqpoint{0.000000in}{3.750000in}}%
\pgfpathclose%
\pgfusepath{fill}%
\end{pgfscope}%
\begin{pgfscope}%
\pgfsetbuttcap%
\pgfsetmiterjoin%
\definecolor{currentfill}{rgb}{1.000000,1.000000,1.000000}%
\pgfsetfillcolor{currentfill}%
\pgfsetlinewidth{0.000000pt}%
\definecolor{currentstroke}{rgb}{0.000000,0.000000,0.000000}%
\pgfsetstrokecolor{currentstroke}%
\pgfsetstrokeopacity{0.000000}%
\pgfsetdash{}{0pt}%
\pgfpathmoveto{\pgfqpoint{0.150000in}{0.150000in}}%
\pgfpathlineto{\pgfqpoint{4.850000in}{0.150000in}}%
\pgfpathlineto{\pgfqpoint{4.850000in}{3.600000in}}%
\pgfpathlineto{\pgfqpoint{0.150000in}{3.600000in}}%
\pgfpathclose%
\pgfusepath{fill}%
\end{pgfscope}%
\begin{pgfscope}%
\pgfsetbuttcap%
\pgfsetmiterjoin%
\definecolor{currentfill}{rgb}{0.950000,0.950000,0.950000}%
\pgfsetfillcolor{currentfill}%
\pgfsetfillopacity{0.500000}%
\pgfsetlinewidth{1.003750pt}%
\definecolor{currentstroke}{rgb}{0.950000,0.950000,0.950000}%
\pgfsetstrokecolor{currentstroke}%
\pgfsetstrokeopacity{0.500000}%
\pgfsetdash{}{0pt}%
\pgfpathmoveto{\pgfqpoint{4.368330in}{1.259216in}}%
\pgfpathlineto{\pgfqpoint{2.563514in}{2.182875in}}%
\pgfpathlineto{\pgfqpoint{2.563514in}{3.554793in}}%
\pgfpathlineto{\pgfqpoint{4.506350in}{2.634684in}}%
\pgfusepath{stroke,fill}%
\end{pgfscope}%
\begin{pgfscope}%
\pgfsetbuttcap%
\pgfsetmiterjoin%
\definecolor{currentfill}{rgb}{0.900000,0.900000,0.900000}%
\pgfsetfillcolor{currentfill}%
\pgfsetfillopacity{0.500000}%
\pgfsetlinewidth{1.003750pt}%
\definecolor{currentstroke}{rgb}{0.900000,0.900000,0.900000}%
\pgfsetstrokecolor{currentstroke}%
\pgfsetstrokeopacity{0.500000}%
\pgfsetdash{}{0pt}%
\pgfpathmoveto{\pgfqpoint{0.758697in}{1.259216in}}%
\pgfpathlineto{\pgfqpoint{2.563514in}{2.182875in}}%
\pgfpathlineto{\pgfqpoint{2.563514in}{3.554793in}}%
\pgfpathlineto{\pgfqpoint{0.620677in}{2.634684in}}%
\pgfusepath{stroke,fill}%
\end{pgfscope}%
\begin{pgfscope}%
\pgfsetbuttcap%
\pgfsetmiterjoin%
\definecolor{currentfill}{rgb}{0.925000,0.925000,0.925000}%
\pgfsetfillcolor{currentfill}%
\pgfsetfillopacity{0.500000}%
\pgfsetlinewidth{1.003750pt}%
\definecolor{currentstroke}{rgb}{0.925000,0.925000,0.925000}%
\pgfsetstrokecolor{currentstroke}%
\pgfsetstrokeopacity{0.500000}%
\pgfsetdash{}{0pt}%
\pgfpathmoveto{\pgfqpoint{2.563514in}{0.237851in}}%
\pgfpathlineto{\pgfqpoint{4.368330in}{1.259216in}}%
\pgfpathlineto{\pgfqpoint{2.563514in}{2.182875in}}%
\pgfpathlineto{\pgfqpoint{0.758697in}{1.259216in}}%
\pgfusepath{stroke,fill}%
\end{pgfscope}%
\begin{pgfscope}%
\pgfsetrectcap%
\pgfsetroundjoin%
\pgfsetlinewidth{0.803000pt}%
\definecolor{currentstroke}{rgb}{0.000000,0.000000,0.000000}%
\pgfsetstrokecolor{currentstroke}%
\pgfsetdash{}{0pt}%
\pgfpathmoveto{\pgfqpoint{4.368330in}{1.259216in}}%
\pgfpathlineto{\pgfqpoint{2.563514in}{0.237851in}}%
\pgfusepath{stroke}%
\end{pgfscope}%
\begin{pgfscope}%
\definecolor{textcolor}{rgb}{0.000000,0.000000,0.000000}%
\pgfsetstrokecolor{textcolor}%
\pgfsetfillcolor{textcolor}%
\pgftext[x=3.438881in,y=0.306559in,left,base,rotate=29.505987]{\color{textcolor}\sffamily\fontsize{8.000000}{9.600000}\selectfont Player 2 (\(\displaystyle \mu\))}%
\end{pgfscope}%
\begin{pgfscope}%
\pgfsetbuttcap%
\pgfsetroundjoin%
\pgfsetlinewidth{0.803000pt}%
\definecolor{currentstroke}{rgb}{0.690196,0.690196,0.690196}%
\pgfsetstrokecolor{currentstroke}%
\pgfsetdash{}{0pt}%
\pgfpathmoveto{\pgfqpoint{2.684034in}{0.306055in}}%
\pgfpathlineto{\pgfqpoint{0.878934in}{1.320750in}}%
\pgfpathlineto{\pgfqpoint{0.750551in}{2.696191in}}%
\pgfusepath{stroke}%
\end{pgfscope}%
\begin{pgfscope}%
\pgfsetbuttcap%
\pgfsetroundjoin%
\pgfsetlinewidth{0.803000pt}%
\definecolor{currentstroke}{rgb}{0.690196,0.690196,0.690196}%
\pgfsetstrokecolor{currentstroke}%
\pgfsetdash{}{0pt}%
\pgfpathmoveto{\pgfqpoint{3.010493in}{0.490801in}}%
\pgfpathlineto{\pgfqpoint{1.204828in}{1.487534in}}%
\pgfpathlineto{\pgfqpoint{1.102246in}{2.862750in}}%
\pgfusepath{stroke}%
\end{pgfscope}%
\begin{pgfscope}%
\pgfsetbuttcap%
\pgfsetroundjoin%
\pgfsetlinewidth{0.803000pt}%
\definecolor{currentstroke}{rgb}{0.690196,0.690196,0.690196}%
\pgfsetstrokecolor{currentstroke}%
\pgfsetdash{}{0pt}%
\pgfpathmoveto{\pgfqpoint{3.331053in}{0.672210in}}%
\pgfpathlineto{\pgfqpoint{1.525123in}{1.651453in}}%
\pgfpathlineto{\pgfqpoint{1.447447in}{3.026234in}}%
\pgfusepath{stroke}%
\end{pgfscope}%
\begin{pgfscope}%
\pgfsetbuttcap%
\pgfsetroundjoin%
\pgfsetlinewidth{0.803000pt}%
\definecolor{currentstroke}{rgb}{0.690196,0.690196,0.690196}%
\pgfsetstrokecolor{currentstroke}%
\pgfsetdash{}{0pt}%
\pgfpathmoveto{\pgfqpoint{3.645873in}{0.850370in}}%
\pgfpathlineto{\pgfqpoint{1.839962in}{1.812580in}}%
\pgfpathlineto{\pgfqpoint{1.786333in}{3.186727in}}%
\pgfusepath{stroke}%
\end{pgfscope}%
\begin{pgfscope}%
\pgfsetbuttcap%
\pgfsetroundjoin%
\pgfsetlinewidth{0.803000pt}%
\definecolor{currentstroke}{rgb}{0.690196,0.690196,0.690196}%
\pgfsetstrokecolor{currentstroke}%
\pgfsetdash{}{0pt}%
\pgfpathmoveto{\pgfqpoint{3.955105in}{1.025368in}}%
\pgfpathlineto{\pgfqpoint{2.149483in}{1.970985in}}%
\pgfpathlineto{\pgfqpoint{2.119074in}{3.344311in}}%
\pgfusepath{stroke}%
\end{pgfscope}%
\begin{pgfscope}%
\pgfsetbuttcap%
\pgfsetroundjoin%
\pgfsetlinewidth{0.803000pt}%
\definecolor{currentstroke}{rgb}{0.690196,0.690196,0.690196}%
\pgfsetstrokecolor{currentstroke}%
\pgfsetdash{}{0pt}%
\pgfpathmoveto{\pgfqpoint{4.258897in}{1.197287in}}%
\pgfpathlineto{\pgfqpoint{2.453821in}{2.126738in}}%
\pgfpathlineto{\pgfqpoint{2.445837in}{3.499062in}}%
\pgfusepath{stroke}%
\end{pgfscope}%
\begin{pgfscope}%
\pgfsetrectcap%
\pgfsetroundjoin%
\pgfsetlinewidth{0.803000pt}%
\definecolor{currentstroke}{rgb}{0.000000,0.000000,0.000000}%
\pgfsetstrokecolor{currentstroke}%
\pgfsetdash{}{0pt}%
\pgfpathmoveto{\pgfqpoint{2.668839in}{0.314597in}}%
\pgfpathlineto{\pgfqpoint{2.714464in}{0.288949in}}%
\pgfusepath{stroke}%
\end{pgfscope}%
\begin{pgfscope}%
\definecolor{textcolor}{rgb}{0.000000,0.000000,0.000000}%
\pgfsetstrokecolor{textcolor}%
\pgfsetfillcolor{textcolor}%
\pgftext[x=2.758229in,y=0.214337in,,top]{\color{textcolor}\sffamily\fontsize{6.000000}{7.200000}\selectfont \(\displaystyle 0.0\)}%
\end{pgfscope}%
\begin{pgfscope}%
\pgfsetrectcap%
\pgfsetroundjoin%
\pgfsetlinewidth{0.803000pt}%
\definecolor{currentstroke}{rgb}{0.000000,0.000000,0.000000}%
\pgfsetstrokecolor{currentstroke}%
\pgfsetdash{}{0pt}%
\pgfpathmoveto{\pgfqpoint{2.995300in}{0.499188in}}%
\pgfpathlineto{\pgfqpoint{3.040918in}{0.474007in}}%
\pgfusepath{stroke}%
\end{pgfscope}%
\begin{pgfscope}%
\definecolor{textcolor}{rgb}{0.000000,0.000000,0.000000}%
\pgfsetstrokecolor{textcolor}%
\pgfsetfillcolor{textcolor}%
\pgftext[x=3.083735in,y=0.399762in,,top]{\color{textcolor}\sffamily\fontsize{6.000000}{7.200000}\selectfont \(\displaystyle 0.2\)}%
\end{pgfscope}%
\begin{pgfscope}%
\pgfsetrectcap%
\pgfsetroundjoin%
\pgfsetlinewidth{0.803000pt}%
\definecolor{currentstroke}{rgb}{0.000000,0.000000,0.000000}%
\pgfsetstrokecolor{currentstroke}%
\pgfsetdash{}{0pt}%
\pgfpathmoveto{\pgfqpoint{3.315864in}{0.680446in}}%
\pgfpathlineto{\pgfqpoint{3.361469in}{0.655717in}}%
\pgfusepath{stroke}%
\end{pgfscope}%
\begin{pgfscope}%
\definecolor{textcolor}{rgb}{0.000000,0.000000,0.000000}%
\pgfsetstrokecolor{textcolor}%
\pgfsetfillcolor{textcolor}%
\pgftext[x=3.403364in,y=0.581840in,,top]{\color{textcolor}\sffamily\fontsize{6.000000}{7.200000}\selectfont \(\displaystyle 0.4\)}%
\end{pgfscope}%
\begin{pgfscope}%
\pgfsetrectcap%
\pgfsetroundjoin%
\pgfsetlinewidth{0.803000pt}%
\definecolor{currentstroke}{rgb}{0.000000,0.000000,0.000000}%
\pgfsetstrokecolor{currentstroke}%
\pgfsetdash{}{0pt}%
\pgfpathmoveto{\pgfqpoint{3.630691in}{0.858459in}}%
\pgfpathlineto{\pgfqpoint{3.676275in}{0.834171in}}%
\pgfusepath{stroke}%
\end{pgfscope}%
\begin{pgfscope}%
\definecolor{textcolor}{rgb}{0.000000,0.000000,0.000000}%
\pgfsetstrokecolor{textcolor}%
\pgfsetfillcolor{textcolor}%
\pgftext[x=3.717275in,y=0.760659in,,top]{\color{textcolor}\sffamily\fontsize{6.000000}{7.200000}\selectfont \(\displaystyle 0.6\)}%
\end{pgfscope}%
\begin{pgfscope}%
\pgfsetrectcap%
\pgfsetroundjoin%
\pgfsetlinewidth{0.803000pt}%
\definecolor{currentstroke}{rgb}{0.000000,0.000000,0.000000}%
\pgfsetstrokecolor{currentstroke}%
\pgfsetdash{}{0pt}%
\pgfpathmoveto{\pgfqpoint{3.939932in}{1.033314in}}%
\pgfpathlineto{\pgfqpoint{3.985489in}{1.009455in}}%
\pgfusepath{stroke}%
\end{pgfscope}%
\begin{pgfscope}%
\definecolor{textcolor}{rgb}{0.000000,0.000000,0.000000}%
\pgfsetstrokecolor{textcolor}%
\pgfsetfillcolor{textcolor}%
\pgftext[x=4.025618in,y=0.936308in,,top]{\color{textcolor}\sffamily\fontsize{6.000000}{7.200000}\selectfont \(\displaystyle 0.8\)}%
\end{pgfscope}%
\begin{pgfscope}%
\pgfsetrectcap%
\pgfsetroundjoin%
\pgfsetlinewidth{0.803000pt}%
\definecolor{currentstroke}{rgb}{0.000000,0.000000,0.000000}%
\pgfsetstrokecolor{currentstroke}%
\pgfsetdash{}{0pt}%
\pgfpathmoveto{\pgfqpoint{4.243735in}{1.205094in}}%
\pgfpathlineto{\pgfqpoint{4.289259in}{1.181653in}}%
\pgfusepath{stroke}%
\end{pgfscope}%
\begin{pgfscope}%
\definecolor{textcolor}{rgb}{0.000000,0.000000,0.000000}%
\pgfsetstrokecolor{textcolor}%
\pgfsetfillcolor{textcolor}%
\pgftext[x=4.328542in,y=1.108869in,,top]{\color{textcolor}\sffamily\fontsize{6.000000}{7.200000}\selectfont \(\displaystyle 1.0\)}%
\end{pgfscope}%
\begin{pgfscope}%
\pgfsetrectcap%
\pgfsetroundjoin%
\pgfsetlinewidth{0.803000pt}%
\definecolor{currentstroke}{rgb}{0.000000,0.000000,0.000000}%
\pgfsetstrokecolor{currentstroke}%
\pgfsetdash{}{0pt}%
\pgfpathmoveto{\pgfqpoint{0.758697in}{1.259216in}}%
\pgfpathlineto{\pgfqpoint{2.563514in}{0.237851in}}%
\pgfusepath{stroke}%
\end{pgfscope}%
\begin{pgfscope}%
\definecolor{textcolor}{rgb}{0.000000,0.000000,0.000000}%
\pgfsetstrokecolor{textcolor}%
\pgfsetfillcolor{textcolor}%
\pgftext[x=1.127907in,y=0.623604in,left,base,rotate=330.494013]{\color{textcolor}\sffamily\fontsize{8.000000}{9.600000}\selectfont Player 1 (\(\displaystyle \lambda\))}%
\end{pgfscope}%
\begin{pgfscope}%
\pgfsetbuttcap%
\pgfsetroundjoin%
\pgfsetlinewidth{0.803000pt}%
\definecolor{currentstroke}{rgb}{0.690196,0.690196,0.690196}%
\pgfsetstrokecolor{currentstroke}%
\pgfsetdash{}{0pt}%
\pgfpathmoveto{\pgfqpoint{4.376476in}{2.696191in}}%
\pgfpathlineto{\pgfqpoint{4.248093in}{1.320750in}}%
\pgfpathlineto{\pgfqpoint{2.442993in}{0.306055in}}%
\pgfusepath{stroke}%
\end{pgfscope}%
\begin{pgfscope}%
\pgfsetbuttcap%
\pgfsetroundjoin%
\pgfsetlinewidth{0.803000pt}%
\definecolor{currentstroke}{rgb}{0.690196,0.690196,0.690196}%
\pgfsetstrokecolor{currentstroke}%
\pgfsetdash{}{0pt}%
\pgfpathmoveto{\pgfqpoint{4.024781in}{2.862750in}}%
\pgfpathlineto{\pgfqpoint{3.922199in}{1.487534in}}%
\pgfpathlineto{\pgfqpoint{2.116534in}{0.490801in}}%
\pgfusepath{stroke}%
\end{pgfscope}%
\begin{pgfscope}%
\pgfsetbuttcap%
\pgfsetroundjoin%
\pgfsetlinewidth{0.803000pt}%
\definecolor{currentstroke}{rgb}{0.690196,0.690196,0.690196}%
\pgfsetstrokecolor{currentstroke}%
\pgfsetdash{}{0pt}%
\pgfpathmoveto{\pgfqpoint{3.679580in}{3.026234in}}%
\pgfpathlineto{\pgfqpoint{3.601904in}{1.651453in}}%
\pgfpathlineto{\pgfqpoint{1.795974in}{0.672210in}}%
\pgfusepath{stroke}%
\end{pgfscope}%
\begin{pgfscope}%
\pgfsetbuttcap%
\pgfsetroundjoin%
\pgfsetlinewidth{0.803000pt}%
\definecolor{currentstroke}{rgb}{0.690196,0.690196,0.690196}%
\pgfsetstrokecolor{currentstroke}%
\pgfsetdash{}{0pt}%
\pgfpathmoveto{\pgfqpoint{3.340694in}{3.186727in}}%
\pgfpathlineto{\pgfqpoint{3.287065in}{1.812580in}}%
\pgfpathlineto{\pgfqpoint{1.481154in}{0.850370in}}%
\pgfusepath{stroke}%
\end{pgfscope}%
\begin{pgfscope}%
\pgfsetbuttcap%
\pgfsetroundjoin%
\pgfsetlinewidth{0.803000pt}%
\definecolor{currentstroke}{rgb}{0.690196,0.690196,0.690196}%
\pgfsetstrokecolor{currentstroke}%
\pgfsetdash{}{0pt}%
\pgfpathmoveto{\pgfqpoint{3.007953in}{3.344311in}}%
\pgfpathlineto{\pgfqpoint{2.977544in}{1.970985in}}%
\pgfpathlineto{\pgfqpoint{1.171922in}{1.025368in}}%
\pgfusepath{stroke}%
\end{pgfscope}%
\begin{pgfscope}%
\pgfsetbuttcap%
\pgfsetroundjoin%
\pgfsetlinewidth{0.803000pt}%
\definecolor{currentstroke}{rgb}{0.690196,0.690196,0.690196}%
\pgfsetstrokecolor{currentstroke}%
\pgfsetdash{}{0pt}%
\pgfpathmoveto{\pgfqpoint{2.681190in}{3.499062in}}%
\pgfpathlineto{\pgfqpoint{2.673206in}{2.126738in}}%
\pgfpathlineto{\pgfqpoint{0.868130in}{1.197287in}}%
\pgfusepath{stroke}%
\end{pgfscope}%
\begin{pgfscope}%
\pgfsetrectcap%
\pgfsetroundjoin%
\pgfsetlinewidth{0.803000pt}%
\definecolor{currentstroke}{rgb}{0.000000,0.000000,0.000000}%
\pgfsetstrokecolor{currentstroke}%
\pgfsetdash{}{0pt}%
\pgfpathmoveto{\pgfqpoint{2.458188in}{0.314597in}}%
\pgfpathlineto{\pgfqpoint{2.412563in}{0.288949in}}%
\pgfusepath{stroke}%
\end{pgfscope}%
\begin{pgfscope}%
\definecolor{textcolor}{rgb}{0.000000,0.000000,0.000000}%
\pgfsetstrokecolor{textcolor}%
\pgfsetfillcolor{textcolor}%
\pgftext[x=2.368798in,y=0.214337in,,top]{\color{textcolor}\sffamily\fontsize{6.000000}{7.200000}\selectfont \(\displaystyle 0.0\)}%
\end{pgfscope}%
\begin{pgfscope}%
\pgfsetrectcap%
\pgfsetroundjoin%
\pgfsetlinewidth{0.803000pt}%
\definecolor{currentstroke}{rgb}{0.000000,0.000000,0.000000}%
\pgfsetstrokecolor{currentstroke}%
\pgfsetdash{}{0pt}%
\pgfpathmoveto{\pgfqpoint{2.131727in}{0.499188in}}%
\pgfpathlineto{\pgfqpoint{2.086109in}{0.474007in}}%
\pgfusepath{stroke}%
\end{pgfscope}%
\begin{pgfscope}%
\definecolor{textcolor}{rgb}{0.000000,0.000000,0.000000}%
\pgfsetstrokecolor{textcolor}%
\pgfsetfillcolor{textcolor}%
\pgftext[x=2.043292in,y=0.399762in,,top]{\color{textcolor}\sffamily\fontsize{6.000000}{7.200000}\selectfont \(\displaystyle 0.2\)}%
\end{pgfscope}%
\begin{pgfscope}%
\pgfsetrectcap%
\pgfsetroundjoin%
\pgfsetlinewidth{0.803000pt}%
\definecolor{currentstroke}{rgb}{0.000000,0.000000,0.000000}%
\pgfsetstrokecolor{currentstroke}%
\pgfsetdash{}{0pt}%
\pgfpathmoveto{\pgfqpoint{1.811163in}{0.680446in}}%
\pgfpathlineto{\pgfqpoint{1.765558in}{0.655717in}}%
\pgfusepath{stroke}%
\end{pgfscope}%
\begin{pgfscope}%
\definecolor{textcolor}{rgb}{0.000000,0.000000,0.000000}%
\pgfsetstrokecolor{textcolor}%
\pgfsetfillcolor{textcolor}%
\pgftext[x=1.723663in,y=0.581840in,,top]{\color{textcolor}\sffamily\fontsize{6.000000}{7.200000}\selectfont \(\displaystyle 0.4\)}%
\end{pgfscope}%
\begin{pgfscope}%
\pgfsetrectcap%
\pgfsetroundjoin%
\pgfsetlinewidth{0.803000pt}%
\definecolor{currentstroke}{rgb}{0.000000,0.000000,0.000000}%
\pgfsetstrokecolor{currentstroke}%
\pgfsetdash{}{0pt}%
\pgfpathmoveto{\pgfqpoint{1.496336in}{0.858459in}}%
\pgfpathlineto{\pgfqpoint{1.450752in}{0.834171in}}%
\pgfusepath{stroke}%
\end{pgfscope}%
\begin{pgfscope}%
\definecolor{textcolor}{rgb}{0.000000,0.000000,0.000000}%
\pgfsetstrokecolor{textcolor}%
\pgfsetfillcolor{textcolor}%
\pgftext[x=1.409752in,y=0.760659in,,top]{\color{textcolor}\sffamily\fontsize{6.000000}{7.200000}\selectfont \(\displaystyle 0.6\)}%
\end{pgfscope}%
\begin{pgfscope}%
\pgfsetrectcap%
\pgfsetroundjoin%
\pgfsetlinewidth{0.803000pt}%
\definecolor{currentstroke}{rgb}{0.000000,0.000000,0.000000}%
\pgfsetstrokecolor{currentstroke}%
\pgfsetdash{}{0pt}%
\pgfpathmoveto{\pgfqpoint{1.187095in}{1.033314in}}%
\pgfpathlineto{\pgfqpoint{1.141538in}{1.009455in}}%
\pgfusepath{stroke}%
\end{pgfscope}%
\begin{pgfscope}%
\definecolor{textcolor}{rgb}{0.000000,0.000000,0.000000}%
\pgfsetstrokecolor{textcolor}%
\pgfsetfillcolor{textcolor}%
\pgftext[x=1.101409in,y=0.936308in,,top]{\color{textcolor}\sffamily\fontsize{6.000000}{7.200000}\selectfont \(\displaystyle 0.8\)}%
\end{pgfscope}%
\begin{pgfscope}%
\pgfsetrectcap%
\pgfsetroundjoin%
\pgfsetlinewidth{0.803000pt}%
\definecolor{currentstroke}{rgb}{0.000000,0.000000,0.000000}%
\pgfsetstrokecolor{currentstroke}%
\pgfsetdash{}{0pt}%
\pgfpathmoveto{\pgfqpoint{0.883292in}{1.205094in}}%
\pgfpathlineto{\pgfqpoint{0.837769in}{1.181653in}}%
\pgfusepath{stroke}%
\end{pgfscope}%
\begin{pgfscope}%
\definecolor{textcolor}{rgb}{0.000000,0.000000,0.000000}%
\pgfsetstrokecolor{textcolor}%
\pgfsetfillcolor{textcolor}%
\pgftext[x=0.798485in,y=1.108869in,,top]{\color{textcolor}\sffamily\fontsize{6.000000}{7.200000}\selectfont \(\displaystyle 1.0\)}%
\end{pgfscope}%
\begin{pgfscope}%
\pgfsetrectcap%
\pgfsetroundjoin%
\pgfsetlinewidth{0.803000pt}%
\definecolor{currentstroke}{rgb}{0.000000,0.000000,0.000000}%
\pgfsetstrokecolor{currentstroke}%
\pgfsetdash{}{0pt}%
\pgfpathmoveto{\pgfqpoint{0.758697in}{1.259216in}}%
\pgfpathlineto{\pgfqpoint{0.620677in}{2.634684in}}%
\pgfusepath{stroke}%
\end{pgfscope}%
\begin{pgfscope}%
\definecolor{textcolor}{rgb}{0.000000,0.000000,0.000000}%
\pgfsetstrokecolor{textcolor}%
\pgfsetfillcolor{textcolor}%
\pgftext[x=0.275294in,y=2.172118in,left,base,rotate=275.730122]{\color{textcolor}\sffamily\fontsize{8.000000}{9.600000}\selectfont \(\displaystyle \mathsf{val}(G_{\lambda, \mu}\))}%
\end{pgfscope}%
\begin{pgfscope}%
\pgfsetbuttcap%
\pgfsetroundjoin%
\pgfsetlinewidth{0.803000pt}%
\definecolor{currentstroke}{rgb}{0.690196,0.690196,0.690196}%
\pgfsetstrokecolor{currentstroke}%
\pgfsetdash{}{0pt}%
\pgfpathmoveto{\pgfqpoint{0.739985in}{1.445694in}}%
\pgfpathlineto{\pgfqpoint{2.563514in}{2.369460in}}%
\pgfpathlineto{\pgfqpoint{4.387042in}{1.445694in}}%
\pgfusepath{stroke}%
\end{pgfscope}%
\begin{pgfscope}%
\pgfsetbuttcap%
\pgfsetroundjoin%
\pgfsetlinewidth{0.803000pt}%
\definecolor{currentstroke}{rgb}{0.690196,0.690196,0.690196}%
\pgfsetstrokecolor{currentstroke}%
\pgfsetdash{}{0pt}%
\pgfpathmoveto{\pgfqpoint{0.716416in}{1.680582in}}%
\pgfpathlineto{\pgfqpoint{2.563514in}{2.604219in}}%
\pgfpathlineto{\pgfqpoint{4.410611in}{1.680582in}}%
\pgfusepath{stroke}%
\end{pgfscope}%
\begin{pgfscope}%
\pgfsetbuttcap%
\pgfsetroundjoin%
\pgfsetlinewidth{0.803000pt}%
\definecolor{currentstroke}{rgb}{0.690196,0.690196,0.690196}%
\pgfsetstrokecolor{currentstroke}%
\pgfsetdash{}{0pt}%
\pgfpathmoveto{\pgfqpoint{0.692229in}{1.921622in}}%
\pgfpathlineto{\pgfqpoint{2.563514in}{2.844822in}}%
\pgfpathlineto{\pgfqpoint{4.434798in}{1.921622in}}%
\pgfusepath{stroke}%
\end{pgfscope}%
\begin{pgfscope}%
\pgfsetbuttcap%
\pgfsetroundjoin%
\pgfsetlinewidth{0.803000pt}%
\definecolor{currentstroke}{rgb}{0.690196,0.690196,0.690196}%
\pgfsetstrokecolor{currentstroke}%
\pgfsetdash{}{0pt}%
\pgfpathmoveto{\pgfqpoint{0.667400in}{2.169058in}}%
\pgfpathlineto{\pgfqpoint{2.563514in}{3.091490in}}%
\pgfpathlineto{\pgfqpoint{4.459627in}{2.169058in}}%
\pgfusepath{stroke}%
\end{pgfscope}%
\begin{pgfscope}%
\pgfsetbuttcap%
\pgfsetroundjoin%
\pgfsetlinewidth{0.803000pt}%
\definecolor{currentstroke}{rgb}{0.690196,0.690196,0.690196}%
\pgfsetstrokecolor{currentstroke}%
\pgfsetdash{}{0pt}%
\pgfpathmoveto{\pgfqpoint{0.641903in}{2.423148in}}%
\pgfpathlineto{\pgfqpoint{2.563514in}{3.344454in}}%
\pgfpathlineto{\pgfqpoint{4.485124in}{2.423148in}}%
\pgfusepath{stroke}%
\end{pgfscope}%
\begin{pgfscope}%
\pgfsetrectcap%
\pgfsetroundjoin%
\pgfsetlinewidth{0.803000pt}%
\definecolor{currentstroke}{rgb}{0.000000,0.000000,0.000000}%
\pgfsetstrokecolor{currentstroke}%
\pgfsetdash{}{0pt}%
\pgfpathmoveto{\pgfqpoint{0.755308in}{1.453456in}}%
\pgfpathlineto{\pgfqpoint{0.709303in}{1.430151in}}%
\pgfusepath{stroke}%
\end{pgfscope}%
\begin{pgfscope}%
\definecolor{textcolor}{rgb}{0.000000,0.000000,0.000000}%
\pgfsetstrokecolor{textcolor}%
\pgfsetfillcolor{textcolor}%
\pgftext[x=0.596788in,y=1.445694in,,top]{\color{textcolor}\sffamily\fontsize{6.000000}{7.200000}\selectfont \(\displaystyle -40\)}%
\end{pgfscope}%
\begin{pgfscope}%
\pgfsetrectcap%
\pgfsetroundjoin%
\pgfsetlinewidth{0.803000pt}%
\definecolor{currentstroke}{rgb}{0.000000,0.000000,0.000000}%
\pgfsetstrokecolor{currentstroke}%
\pgfsetdash{}{0pt}%
\pgfpathmoveto{\pgfqpoint{0.731946in}{1.688348in}}%
\pgfpathlineto{\pgfqpoint{0.685317in}{1.665031in}}%
\pgfusepath{stroke}%
\end{pgfscope}%
\begin{pgfscope}%
\definecolor{textcolor}{rgb}{0.000000,0.000000,0.000000}%
\pgfsetstrokecolor{textcolor}%
\pgfsetfillcolor{textcolor}%
\pgftext[x=0.571367in,y=1.680582in,,top]{\color{textcolor}\sffamily\fontsize{6.000000}{7.200000}\selectfont \(\displaystyle -20\)}%
\end{pgfscope}%
\begin{pgfscope}%
\pgfsetrectcap%
\pgfsetroundjoin%
\pgfsetlinewidth{0.803000pt}%
\definecolor{currentstroke}{rgb}{0.000000,0.000000,0.000000}%
\pgfsetstrokecolor{currentstroke}%
\pgfsetdash{}{0pt}%
\pgfpathmoveto{\pgfqpoint{0.707972in}{1.929389in}}%
\pgfpathlineto{\pgfqpoint{0.660702in}{1.906068in}}%
\pgfusepath{stroke}%
\end{pgfscope}%
\begin{pgfscope}%
\definecolor{textcolor}{rgb}{0.000000,0.000000,0.000000}%
\pgfsetstrokecolor{textcolor}%
\pgfsetfillcolor{textcolor}%
\pgftext[x=0.545281in,y=1.921622in,,top]{\color{textcolor}\sffamily\fontsize{6.000000}{7.200000}\selectfont \(\displaystyle 0\)}%
\end{pgfscope}%
\begin{pgfscope}%
\pgfsetrectcap%
\pgfsetroundjoin%
\pgfsetlinewidth{0.803000pt}%
\definecolor{currentstroke}{rgb}{0.000000,0.000000,0.000000}%
\pgfsetstrokecolor{currentstroke}%
\pgfsetdash{}{0pt}%
\pgfpathmoveto{\pgfqpoint{0.683363in}{2.176823in}}%
\pgfpathlineto{\pgfqpoint{0.635434in}{2.153507in}}%
\pgfusepath{stroke}%
\end{pgfscope}%
\begin{pgfscope}%
\definecolor{textcolor}{rgb}{0.000000,0.000000,0.000000}%
\pgfsetstrokecolor{textcolor}%
\pgfsetfillcolor{textcolor}%
\pgftext[x=0.518503in,y=2.169058in,,top]{\color{textcolor}\sffamily\fontsize{6.000000}{7.200000}\selectfont \(\displaystyle 20\)}%
\end{pgfscope}%
\begin{pgfscope}%
\pgfsetrectcap%
\pgfsetroundjoin%
\pgfsetlinewidth{0.803000pt}%
\definecolor{currentstroke}{rgb}{0.000000,0.000000,0.000000}%
\pgfsetstrokecolor{currentstroke}%
\pgfsetdash{}{0pt}%
\pgfpathmoveto{\pgfqpoint{0.658092in}{2.430910in}}%
\pgfpathlineto{\pgfqpoint{0.609485in}{2.407606in}}%
\pgfusepath{stroke}%
\end{pgfscope}%
\begin{pgfscope}%
\definecolor{textcolor}{rgb}{0.000000,0.000000,0.000000}%
\pgfsetstrokecolor{textcolor}%
\pgfsetfillcolor{textcolor}%
\pgftext[x=0.491004in,y=2.423148in,,top]{\color{textcolor}\sffamily\fontsize{6.000000}{7.200000}\selectfont \(\displaystyle 40\)}%
\end{pgfscope}%
\begin{pgfscope}%
\pgfpathrectangle{\pgfqpoint{0.150000in}{0.150000in}}{\pgfqpoint{4.700000in}{3.450000in}}%
\pgfusepath{clip}%
\pgfsetbuttcap%
\pgfsetroundjoin%
\definecolor{currentfill}{rgb}{0.996890,0.997273,0.997809}%
\pgfsetfillcolor{currentfill}%
\pgfsetlinewidth{0.000000pt}%
\definecolor{currentstroke}{rgb}{0.000000,0.000000,0.000000}%
\pgfsetstrokecolor{currentstroke}%
\pgfsetdash{}{0pt}%
\pgfpathmoveto{\pgfqpoint{2.563514in}{2.670171in}}%
\pgfpathlineto{\pgfqpoint{2.626007in}{2.670204in}}%
\pgfpathlineto{\pgfqpoint{2.563514in}{2.732432in}}%
\pgfpathlineto{\pgfqpoint{2.500819in}{2.732609in}}%
\pgfpathclose%
\pgfusepath{fill}%
\end{pgfscope}%
\begin{pgfscope}%
\pgfpathrectangle{\pgfqpoint{0.150000in}{0.150000in}}{\pgfqpoint{4.700000in}{3.450000in}}%
\pgfusepath{clip}%
\pgfsetbuttcap%
\pgfsetroundjoin%
\definecolor{currentfill}{rgb}{0.975306,0.955193,0.956786}%
\pgfsetfillcolor{currentfill}%
\pgfsetlinewidth{0.000000pt}%
\definecolor{currentstroke}{rgb}{0.000000,0.000000,0.000000}%
\pgfsetstrokecolor{currentstroke}%
\pgfsetdash{}{0pt}%
\pgfpathmoveto{\pgfqpoint{2.626225in}{2.607717in}}%
\pgfpathlineto{\pgfqpoint{2.688517in}{2.607958in}}%
\pgfpathlineto{\pgfqpoint{2.626007in}{2.670204in}}%
\pgfpathlineto{\pgfqpoint{2.563514in}{2.670171in}}%
\pgfpathclose%
\pgfusepath{fill}%
\end{pgfscope}%
\begin{pgfscope}%
\pgfpathrectangle{\pgfqpoint{0.150000in}{0.150000in}}{\pgfqpoint{4.700000in}{3.450000in}}%
\pgfusepath{clip}%
\pgfsetbuttcap%
\pgfsetroundjoin%
\definecolor{currentfill}{rgb}{0.952512,0.913833,0.916896}%
\pgfsetfillcolor{currentfill}%
\pgfsetlinewidth{0.000000pt}%
\definecolor{currentstroke}{rgb}{0.000000,0.000000,0.000000}%
\pgfsetstrokecolor{currentstroke}%
\pgfsetdash{}{0pt}%
\pgfpathmoveto{\pgfqpoint{2.688953in}{2.545245in}}%
\pgfpathlineto{\pgfqpoint{2.751044in}{2.545695in}}%
\pgfpathlineto{\pgfqpoint{2.688517in}{2.607958in}}%
\pgfpathlineto{\pgfqpoint{2.626225in}{2.607717in}}%
\pgfpathclose%
\pgfusepath{fill}%
\end{pgfscope}%
\begin{pgfscope}%
\pgfpathrectangle{\pgfqpoint{0.150000in}{0.150000in}}{\pgfqpoint{4.700000in}{3.450000in}}%
\pgfusepath{clip}%
\pgfsetbuttcap%
\pgfsetroundjoin%
\definecolor{currentfill}{rgb}{0.929718,0.872472,0.877007}%
\pgfsetfillcolor{currentfill}%
\pgfsetlinewidth{0.000000pt}%
\definecolor{currentstroke}{rgb}{0.000000,0.000000,0.000000}%
\pgfsetstrokecolor{currentstroke}%
\pgfsetdash{}{0pt}%
\pgfpathmoveto{\pgfqpoint{2.751699in}{2.482757in}}%
\pgfpathlineto{\pgfqpoint{2.813588in}{2.483416in}}%
\pgfpathlineto{\pgfqpoint{2.751044in}{2.545695in}}%
\pgfpathlineto{\pgfqpoint{2.688953in}{2.545245in}}%
\pgfpathclose%
\pgfusepath{fill}%
\end{pgfscope}%
\begin{pgfscope}%
\pgfpathrectangle{\pgfqpoint{0.150000in}{0.150000in}}{\pgfqpoint{4.700000in}{3.450000in}}%
\pgfusepath{clip}%
\pgfsetbuttcap%
\pgfsetroundjoin%
\definecolor{currentfill}{rgb}{0.906924,0.831112,0.837117}%
\pgfsetfillcolor{currentfill}%
\pgfsetlinewidth{0.000000pt}%
\definecolor{currentstroke}{rgb}{0.000000,0.000000,0.000000}%
\pgfsetstrokecolor{currentstroke}%
\pgfsetdash{}{0pt}%
\pgfpathmoveto{\pgfqpoint{2.814461in}{2.420251in}}%
\pgfpathlineto{\pgfqpoint{2.876149in}{2.421119in}}%
\pgfpathlineto{\pgfqpoint{2.813588in}{2.483416in}}%
\pgfpathlineto{\pgfqpoint{2.751699in}{2.482757in}}%
\pgfpathclose%
\pgfusepath{fill}%
\end{pgfscope}%
\begin{pgfscope}%
\pgfpathrectangle{\pgfqpoint{0.150000in}{0.150000in}}{\pgfqpoint{4.700000in}{3.450000in}}%
\pgfusepath{clip}%
\pgfsetbuttcap%
\pgfsetroundjoin%
\definecolor{currentfill}{rgb}{0.884130,0.789752,0.797227}%
\pgfsetfillcolor{currentfill}%
\pgfsetlinewidth{0.000000pt}%
\definecolor{currentstroke}{rgb}{0.000000,0.000000,0.000000}%
\pgfsetstrokecolor{currentstroke}%
\pgfsetdash{}{0pt}%
\pgfpathmoveto{\pgfqpoint{2.877241in}{2.357728in}}%
\pgfpathlineto{\pgfqpoint{2.938727in}{2.358806in}}%
\pgfpathlineto{\pgfqpoint{2.876149in}{2.421119in}}%
\pgfpathlineto{\pgfqpoint{2.814461in}{2.420251in}}%
\pgfpathclose%
\pgfusepath{fill}%
\end{pgfscope}%
\begin{pgfscope}%
\pgfpathrectangle{\pgfqpoint{0.150000in}{0.150000in}}{\pgfqpoint{4.700000in}{3.450000in}}%
\pgfusepath{clip}%
\pgfsetbuttcap%
\pgfsetroundjoin%
\definecolor{currentfill}{rgb}{0.861336,0.748392,0.757338}%
\pgfsetfillcolor{currentfill}%
\pgfsetlinewidth{0.000000pt}%
\definecolor{currentstroke}{rgb}{0.000000,0.000000,0.000000}%
\pgfsetstrokecolor{currentstroke}%
\pgfsetdash{}{0pt}%
\pgfpathmoveto{\pgfqpoint{2.940038in}{2.295188in}}%
\pgfpathlineto{\pgfqpoint{3.001322in}{2.296475in}}%
\pgfpathlineto{\pgfqpoint{2.938727in}{2.358806in}}%
\pgfpathlineto{\pgfqpoint{2.877241in}{2.357728in}}%
\pgfpathclose%
\pgfusepath{fill}%
\end{pgfscope}%
\begin{pgfscope}%
\pgfpathrectangle{\pgfqpoint{0.150000in}{0.150000in}}{\pgfqpoint{4.700000in}{3.450000in}}%
\pgfusepath{clip}%
\pgfsetbuttcap%
\pgfsetroundjoin%
\definecolor{currentfill}{rgb}{0.838542,0.707031,0.717448}%
\pgfsetfillcolor{currentfill}%
\pgfsetlinewidth{0.000000pt}%
\definecolor{currentstroke}{rgb}{0.000000,0.000000,0.000000}%
\pgfsetstrokecolor{currentstroke}%
\pgfsetdash{}{0pt}%
\pgfpathmoveto{\pgfqpoint{3.002852in}{2.232632in}}%
\pgfpathlineto{\pgfqpoint{3.063935in}{2.234128in}}%
\pgfpathlineto{\pgfqpoint{3.001322in}{2.296475in}}%
\pgfpathlineto{\pgfqpoint{2.940038in}{2.295188in}}%
\pgfpathclose%
\pgfusepath{fill}%
\end{pgfscope}%
\begin{pgfscope}%
\pgfpathrectangle{\pgfqpoint{0.150000in}{0.150000in}}{\pgfqpoint{4.700000in}{3.450000in}}%
\pgfusepath{clip}%
\pgfsetbuttcap%
\pgfsetroundjoin%
\definecolor{currentfill}{rgb}{0.815748,0.665671,0.677558}%
\pgfsetfillcolor{currentfill}%
\pgfsetlinewidth{0.000000pt}%
\definecolor{currentstroke}{rgb}{0.000000,0.000000,0.000000}%
\pgfsetstrokecolor{currentstroke}%
\pgfsetdash{}{0pt}%
\pgfpathmoveto{\pgfqpoint{3.065683in}{2.170058in}}%
\pgfpathlineto{\pgfqpoint{3.126564in}{2.171764in}}%
\pgfpathlineto{\pgfqpoint{3.063935in}{2.234128in}}%
\pgfpathlineto{\pgfqpoint{3.002852in}{2.232632in}}%
\pgfpathclose%
\pgfusepath{fill}%
\end{pgfscope}%
\begin{pgfscope}%
\pgfpathrectangle{\pgfqpoint{0.150000in}{0.150000in}}{\pgfqpoint{4.700000in}{3.450000in}}%
\pgfusepath{clip}%
\pgfsetbuttcap%
\pgfsetroundjoin%
\definecolor{currentfill}{rgb}{0.792953,0.624311,0.637669}%
\pgfsetfillcolor{currentfill}%
\pgfsetlinewidth{0.000000pt}%
\definecolor{currentstroke}{rgb}{0.000000,0.000000,0.000000}%
\pgfsetstrokecolor{currentstroke}%
\pgfsetdash{}{0pt}%
\pgfpathmoveto{\pgfqpoint{3.128532in}{2.107467in}}%
\pgfpathlineto{\pgfqpoint{3.189210in}{2.109382in}}%
\pgfpathlineto{\pgfqpoint{3.126564in}{2.171764in}}%
\pgfpathlineto{\pgfqpoint{3.065683in}{2.170058in}}%
\pgfpathclose%
\pgfusepath{fill}%
\end{pgfscope}%
\begin{pgfscope}%
\pgfpathrectangle{\pgfqpoint{0.150000in}{0.150000in}}{\pgfqpoint{4.700000in}{3.450000in}}%
\pgfusepath{clip}%
\pgfsetbuttcap%
\pgfsetroundjoin%
\definecolor{currentfill}{rgb}{0.770159,0.582950,0.597779}%
\pgfsetfillcolor{currentfill}%
\pgfsetlinewidth{0.000000pt}%
\definecolor{currentstroke}{rgb}{0.000000,0.000000,0.000000}%
\pgfsetstrokecolor{currentstroke}%
\pgfsetdash{}{0pt}%
\pgfpathmoveto{\pgfqpoint{3.191397in}{2.044858in}}%
\pgfpathlineto{\pgfqpoint{3.251874in}{2.046984in}}%
\pgfpathlineto{\pgfqpoint{3.189210in}{2.109382in}}%
\pgfpathlineto{\pgfqpoint{3.128532in}{2.107467in}}%
\pgfpathclose%
\pgfusepath{fill}%
\end{pgfscope}%
\begin{pgfscope}%
\pgfpathrectangle{\pgfqpoint{0.150000in}{0.150000in}}{\pgfqpoint{4.700000in}{3.450000in}}%
\pgfusepath{clip}%
\pgfsetbuttcap%
\pgfsetroundjoin%
\definecolor{currentfill}{rgb}{0.747365,0.541590,0.557889}%
\pgfsetfillcolor{currentfill}%
\pgfsetlinewidth{0.000000pt}%
\definecolor{currentstroke}{rgb}{0.000000,0.000000,0.000000}%
\pgfsetstrokecolor{currentstroke}%
\pgfsetdash{}{0pt}%
\pgfpathmoveto{\pgfqpoint{3.254280in}{1.982233in}}%
\pgfpathlineto{\pgfqpoint{3.314554in}{1.984569in}}%
\pgfpathlineto{\pgfqpoint{3.251874in}{2.046984in}}%
\pgfpathlineto{\pgfqpoint{3.191397in}{2.044858in}}%
\pgfpathclose%
\pgfusepath{fill}%
\end{pgfscope}%
\begin{pgfscope}%
\pgfpathrectangle{\pgfqpoint{0.150000in}{0.150000in}}{\pgfqpoint{4.700000in}{3.450000in}}%
\pgfusepath{clip}%
\pgfsetbuttcap%
\pgfsetroundjoin%
\definecolor{currentfill}{rgb}{0.724571,0.500230,0.517999}%
\pgfsetfillcolor{currentfill}%
\pgfsetlinewidth{0.000000pt}%
\definecolor{currentstroke}{rgb}{0.000000,0.000000,0.000000}%
\pgfsetstrokecolor{currentstroke}%
\pgfsetdash{}{0pt}%
\pgfpathmoveto{\pgfqpoint{3.317180in}{1.919591in}}%
\pgfpathlineto{\pgfqpoint{3.377251in}{1.922137in}}%
\pgfpathlineto{\pgfqpoint{3.314554in}{1.984569in}}%
\pgfpathlineto{\pgfqpoint{3.254280in}{1.982233in}}%
\pgfpathclose%
\pgfusepath{fill}%
\end{pgfscope}%
\begin{pgfscope}%
\pgfpathrectangle{\pgfqpoint{0.150000in}{0.150000in}}{\pgfqpoint{4.700000in}{3.450000in}}%
\pgfusepath{clip}%
\pgfsetbuttcap%
\pgfsetroundjoin%
\definecolor{currentfill}{rgb}{0.701777,0.458869,0.478110}%
\pgfsetfillcolor{currentfill}%
\pgfsetlinewidth{0.000000pt}%
\definecolor{currentstroke}{rgb}{0.000000,0.000000,0.000000}%
\pgfsetstrokecolor{currentstroke}%
\pgfsetdash{}{0pt}%
\pgfpathmoveto{\pgfqpoint{3.380097in}{1.856931in}}%
\pgfpathlineto{\pgfqpoint{3.439966in}{1.859687in}}%
\pgfpathlineto{\pgfqpoint{3.377251in}{1.922137in}}%
\pgfpathlineto{\pgfqpoint{3.317180in}{1.919591in}}%
\pgfpathclose%
\pgfusepath{fill}%
\end{pgfscope}%
\begin{pgfscope}%
\pgfpathrectangle{\pgfqpoint{0.150000in}{0.150000in}}{\pgfqpoint{4.700000in}{3.450000in}}%
\pgfusepath{clip}%
\pgfsetbuttcap%
\pgfsetroundjoin%
\definecolor{currentfill}{rgb}{0.678983,0.417509,0.438220}%
\pgfsetfillcolor{currentfill}%
\pgfsetlinewidth{0.000000pt}%
\definecolor{currentstroke}{rgb}{0.000000,0.000000,0.000000}%
\pgfsetstrokecolor{currentstroke}%
\pgfsetdash{}{0pt}%
\pgfpathmoveto{\pgfqpoint{3.443031in}{1.794255in}}%
\pgfpathlineto{\pgfqpoint{3.502698in}{1.797221in}}%
\pgfpathlineto{\pgfqpoint{3.439966in}{1.859687in}}%
\pgfpathlineto{\pgfqpoint{3.380097in}{1.856931in}}%
\pgfpathclose%
\pgfusepath{fill}%
\end{pgfscope}%
\begin{pgfscope}%
\pgfpathrectangle{\pgfqpoint{0.150000in}{0.150000in}}{\pgfqpoint{4.700000in}{3.450000in}}%
\pgfusepath{clip}%
\pgfsetbuttcap%
\pgfsetroundjoin%
\definecolor{currentfill}{rgb}{0.656189,0.376149,0.398330}%
\pgfsetfillcolor{currentfill}%
\pgfsetlinewidth{0.000000pt}%
\definecolor{currentstroke}{rgb}{0.000000,0.000000,0.000000}%
\pgfsetstrokecolor{currentstroke}%
\pgfsetdash{}{0pt}%
\pgfpathmoveto{\pgfqpoint{3.505983in}{1.731561in}}%
\pgfpathlineto{\pgfqpoint{3.565446in}{1.734738in}}%
\pgfpathlineto{\pgfqpoint{3.502698in}{1.797221in}}%
\pgfpathlineto{\pgfqpoint{3.443031in}{1.794255in}}%
\pgfpathclose%
\pgfusepath{fill}%
\end{pgfscope}%
\begin{pgfscope}%
\pgfpathrectangle{\pgfqpoint{0.150000in}{0.150000in}}{\pgfqpoint{4.700000in}{3.450000in}}%
\pgfusepath{clip}%
\pgfsetbuttcap%
\pgfsetroundjoin%
\definecolor{currentfill}{rgb}{0.633395,0.334789,0.358441}%
\pgfsetfillcolor{currentfill}%
\pgfsetlinewidth{0.000000pt}%
\definecolor{currentstroke}{rgb}{0.000000,0.000000,0.000000}%
\pgfsetstrokecolor{currentstroke}%
\pgfsetdash{}{0pt}%
\pgfpathmoveto{\pgfqpoint{3.568951in}{1.668850in}}%
\pgfpathlineto{\pgfqpoint{3.628212in}{1.672237in}}%
\pgfpathlineto{\pgfqpoint{3.565446in}{1.734738in}}%
\pgfpathlineto{\pgfqpoint{3.505983in}{1.731561in}}%
\pgfpathclose%
\pgfusepath{fill}%
\end{pgfscope}%
\begin{pgfscope}%
\pgfpathrectangle{\pgfqpoint{0.150000in}{0.150000in}}{\pgfqpoint{4.700000in}{3.450000in}}%
\pgfusepath{clip}%
\pgfsetbuttcap%
\pgfsetroundjoin%
\definecolor{currentfill}{rgb}{0.610600,0.293428,0.318551}%
\pgfsetfillcolor{currentfill}%
\pgfsetlinewidth{0.000000pt}%
\definecolor{currentstroke}{rgb}{0.000000,0.000000,0.000000}%
\pgfsetstrokecolor{currentstroke}%
\pgfsetdash{}{0pt}%
\pgfpathmoveto{\pgfqpoint{3.631937in}{1.606122in}}%
\pgfpathlineto{\pgfqpoint{3.690995in}{1.609720in}}%
\pgfpathlineto{\pgfqpoint{3.628212in}{1.672237in}}%
\pgfpathlineto{\pgfqpoint{3.568951in}{1.668850in}}%
\pgfpathclose%
\pgfusepath{fill}%
\end{pgfscope}%
\begin{pgfscope}%
\pgfpathrectangle{\pgfqpoint{0.150000in}{0.150000in}}{\pgfqpoint{4.700000in}{3.450000in}}%
\pgfusepath{clip}%
\pgfsetbuttcap%
\pgfsetroundjoin%
\definecolor{currentfill}{rgb}{0.584007,0.245175,0.272013}%
\pgfsetfillcolor{currentfill}%
\pgfsetlinewidth{0.000000pt}%
\definecolor{currentstroke}{rgb}{0.000000,0.000000,0.000000}%
\pgfsetstrokecolor{currentstroke}%
\pgfsetdash{}{0pt}%
\pgfpathmoveto{\pgfqpoint{3.694940in}{1.543377in}}%
\pgfpathlineto{\pgfqpoint{3.753795in}{1.547186in}}%
\pgfpathlineto{\pgfqpoint{3.690995in}{1.609720in}}%
\pgfpathlineto{\pgfqpoint{3.631937in}{1.606122in}}%
\pgfpathclose%
\pgfusepath{fill}%
\end{pgfscope}%
\begin{pgfscope}%
\pgfpathrectangle{\pgfqpoint{0.150000in}{0.150000in}}{\pgfqpoint{4.700000in}{3.450000in}}%
\pgfusepath{clip}%
\pgfsetbuttcap%
\pgfsetroundjoin%
\definecolor{currentfill}{rgb}{0.565012,0.210708,0.238771}%
\pgfsetfillcolor{currentfill}%
\pgfsetlinewidth{0.000000pt}%
\definecolor{currentstroke}{rgb}{0.000000,0.000000,0.000000}%
\pgfsetstrokecolor{currentstroke}%
\pgfsetdash{}{0pt}%
\pgfpathmoveto{\pgfqpoint{3.758420in}{1.487616in}}%
\pgfpathlineto{\pgfqpoint{3.816612in}{1.484634in}}%
\pgfpathlineto{\pgfqpoint{3.753795in}{1.547186in}}%
\pgfpathlineto{\pgfqpoint{3.694940in}{1.543377in}}%
\pgfpathclose%
\pgfusepath{fill}%
\end{pgfscope}%
\begin{pgfscope}%
\pgfpathrectangle{\pgfqpoint{0.150000in}{0.150000in}}{\pgfqpoint{4.700000in}{3.450000in}}%
\pgfusepath{clip}%
\pgfsetbuttcap%
\pgfsetroundjoin%
\definecolor{currentfill}{rgb}{0.959574,0.964553,0.971523}%
\pgfsetfillcolor{currentfill}%
\pgfsetlinewidth{0.000000pt}%
\definecolor{currentstroke}{rgb}{0.000000,0.000000,0.000000}%
\pgfsetstrokecolor{currentstroke}%
\pgfsetdash{}{0pt}%
\pgfpathmoveto{\pgfqpoint{2.500600in}{2.670139in}}%
\pgfpathlineto{\pgfqpoint{2.563514in}{2.670171in}}%
\pgfpathlineto{\pgfqpoint{2.500819in}{2.732609in}}%
\pgfpathlineto{\pgfqpoint{2.437703in}{2.732786in}}%
\pgfpathclose%
\pgfusepath{fill}%
\end{pgfscope}%
\begin{pgfscope}%
\pgfpathrectangle{\pgfqpoint{0.150000in}{0.150000in}}{\pgfqpoint{4.700000in}{3.450000in}}%
\pgfusepath{clip}%
\pgfsetbuttcap%
\pgfsetroundjoin%
\definecolor{currentfill}{rgb}{0.996890,0.997273,0.997809}%
\pgfsetfillcolor{currentfill}%
\pgfsetlinewidth{0.000000pt}%
\definecolor{currentstroke}{rgb}{0.000000,0.000000,0.000000}%
\pgfsetstrokecolor{currentstroke}%
\pgfsetdash{}{0pt}%
\pgfpathmoveto{\pgfqpoint{2.563514in}{2.607474in}}%
\pgfpathlineto{\pgfqpoint{2.626225in}{2.607717in}}%
\pgfpathlineto{\pgfqpoint{2.563514in}{2.670171in}}%
\pgfpathlineto{\pgfqpoint{2.500600in}{2.670139in}}%
\pgfpathclose%
\pgfusepath{fill}%
\end{pgfscope}%
\begin{pgfscope}%
\pgfpathrectangle{\pgfqpoint{0.150000in}{0.150000in}}{\pgfqpoint{4.700000in}{3.450000in}}%
\pgfusepath{clip}%
\pgfsetbuttcap%
\pgfsetroundjoin%
\definecolor{currentfill}{rgb}{0.549816,0.183134,0.212178}%
\pgfsetfillcolor{currentfill}%
\pgfsetlinewidth{0.000000pt}%
\definecolor{currentstroke}{rgb}{0.000000,0.000000,0.000000}%
\pgfsetstrokecolor{currentstroke}%
\pgfsetdash{}{0pt}%
\pgfpathmoveto{\pgfqpoint{3.823062in}{1.447614in}}%
\pgfpathlineto{\pgfqpoint{3.881102in}{1.444892in}}%
\pgfpathlineto{\pgfqpoint{3.816612in}{1.484634in}}%
\pgfpathlineto{\pgfqpoint{3.758420in}{1.487616in}}%
\pgfpathclose%
\pgfusepath{fill}%
\end{pgfscope}%
\begin{pgfscope}%
\pgfpathrectangle{\pgfqpoint{0.150000in}{0.150000in}}{\pgfqpoint{4.700000in}{3.450000in}}%
\pgfusepath{clip}%
\pgfsetbuttcap%
\pgfsetroundjoin%
\definecolor{currentfill}{rgb}{0.975306,0.955193,0.956786}%
\pgfsetfillcolor{currentfill}%
\pgfsetlinewidth{0.000000pt}%
\definecolor{currentstroke}{rgb}{0.000000,0.000000,0.000000}%
\pgfsetstrokecolor{currentstroke}%
\pgfsetdash{}{0pt}%
\pgfpathmoveto{\pgfqpoint{2.626444in}{2.544792in}}%
\pgfpathlineto{\pgfqpoint{2.688953in}{2.545245in}}%
\pgfpathlineto{\pgfqpoint{2.626225in}{2.607717in}}%
\pgfpathlineto{\pgfqpoint{2.563514in}{2.607474in}}%
\pgfpathclose%
\pgfusepath{fill}%
\end{pgfscope}%
\begin{pgfscope}%
\pgfpathrectangle{\pgfqpoint{0.150000in}{0.150000in}}{\pgfqpoint{4.700000in}{3.450000in}}%
\pgfusepath{clip}%
\pgfsetbuttcap%
\pgfsetroundjoin%
\definecolor{currentfill}{rgb}{0.952512,0.913833,0.916896}%
\pgfsetfillcolor{currentfill}%
\pgfsetlinewidth{0.000000pt}%
\definecolor{currentstroke}{rgb}{0.000000,0.000000,0.000000}%
\pgfsetstrokecolor{currentstroke}%
\pgfsetdash{}{0pt}%
\pgfpathmoveto{\pgfqpoint{2.689393in}{2.482093in}}%
\pgfpathlineto{\pgfqpoint{2.751699in}{2.482757in}}%
\pgfpathlineto{\pgfqpoint{2.688953in}{2.545245in}}%
\pgfpathlineto{\pgfqpoint{2.626444in}{2.544792in}}%
\pgfpathclose%
\pgfusepath{fill}%
\end{pgfscope}%
\begin{pgfscope}%
\pgfpathrectangle{\pgfqpoint{0.150000in}{0.150000in}}{\pgfqpoint{4.700000in}{3.450000in}}%
\pgfusepath{clip}%
\pgfsetbuttcap%
\pgfsetroundjoin%
\definecolor{currentfill}{rgb}{0.929718,0.872472,0.877007}%
\pgfsetfillcolor{currentfill}%
\pgfsetlinewidth{0.000000pt}%
\definecolor{currentstroke}{rgb}{0.000000,0.000000,0.000000}%
\pgfsetstrokecolor{currentstroke}%
\pgfsetdash{}{0pt}%
\pgfpathmoveto{\pgfqpoint{2.752358in}{2.419377in}}%
\pgfpathlineto{\pgfqpoint{2.814461in}{2.420251in}}%
\pgfpathlineto{\pgfqpoint{2.751699in}{2.482757in}}%
\pgfpathlineto{\pgfqpoint{2.689393in}{2.482093in}}%
\pgfpathclose%
\pgfusepath{fill}%
\end{pgfscope}%
\begin{pgfscope}%
\pgfpathrectangle{\pgfqpoint{0.150000in}{0.150000in}}{\pgfqpoint{4.700000in}{3.450000in}}%
\pgfusepath{clip}%
\pgfsetbuttcap%
\pgfsetroundjoin%
\definecolor{currentfill}{rgb}{0.906924,0.831112,0.837117}%
\pgfsetfillcolor{currentfill}%
\pgfsetlinewidth{0.000000pt}%
\definecolor{currentstroke}{rgb}{0.000000,0.000000,0.000000}%
\pgfsetstrokecolor{currentstroke}%
\pgfsetdash{}{0pt}%
\pgfpathmoveto{\pgfqpoint{2.815341in}{2.356643in}}%
\pgfpathlineto{\pgfqpoint{2.877241in}{2.357728in}}%
\pgfpathlineto{\pgfqpoint{2.814461in}{2.420251in}}%
\pgfpathlineto{\pgfqpoint{2.752358in}{2.419377in}}%
\pgfpathclose%
\pgfusepath{fill}%
\end{pgfscope}%
\begin{pgfscope}%
\pgfpathrectangle{\pgfqpoint{0.150000in}{0.150000in}}{\pgfqpoint{4.700000in}{3.450000in}}%
\pgfusepath{clip}%
\pgfsetbuttcap%
\pgfsetroundjoin%
\definecolor{currentfill}{rgb}{0.884130,0.789752,0.797227}%
\pgfsetfillcolor{currentfill}%
\pgfsetlinewidth{0.000000pt}%
\definecolor{currentstroke}{rgb}{0.000000,0.000000,0.000000}%
\pgfsetstrokecolor{currentstroke}%
\pgfsetdash{}{0pt}%
\pgfpathmoveto{\pgfqpoint{2.878341in}{2.293893in}}%
\pgfpathlineto{\pgfqpoint{2.940038in}{2.295188in}}%
\pgfpathlineto{\pgfqpoint{2.877241in}{2.357728in}}%
\pgfpathlineto{\pgfqpoint{2.815341in}{2.356643in}}%
\pgfpathclose%
\pgfusepath{fill}%
\end{pgfscope}%
\begin{pgfscope}%
\pgfpathrectangle{\pgfqpoint{0.150000in}{0.150000in}}{\pgfqpoint{4.700000in}{3.450000in}}%
\pgfusepath{clip}%
\pgfsetbuttcap%
\pgfsetroundjoin%
\definecolor{currentfill}{rgb}{0.861336,0.748392,0.757338}%
\pgfsetfillcolor{currentfill}%
\pgfsetlinewidth{0.000000pt}%
\definecolor{currentstroke}{rgb}{0.000000,0.000000,0.000000}%
\pgfsetstrokecolor{currentstroke}%
\pgfsetdash{}{0pt}%
\pgfpathmoveto{\pgfqpoint{2.941358in}{2.231125in}}%
\pgfpathlineto{\pgfqpoint{3.002852in}{2.232632in}}%
\pgfpathlineto{\pgfqpoint{2.940038in}{2.295188in}}%
\pgfpathlineto{\pgfqpoint{2.878341in}{2.293893in}}%
\pgfpathclose%
\pgfusepath{fill}%
\end{pgfscope}%
\begin{pgfscope}%
\pgfpathrectangle{\pgfqpoint{0.150000in}{0.150000in}}{\pgfqpoint{4.700000in}{3.450000in}}%
\pgfusepath{clip}%
\pgfsetbuttcap%
\pgfsetroundjoin%
\definecolor{currentfill}{rgb}{0.838542,0.707031,0.717448}%
\pgfsetfillcolor{currentfill}%
\pgfsetlinewidth{0.000000pt}%
\definecolor{currentstroke}{rgb}{0.000000,0.000000,0.000000}%
\pgfsetstrokecolor{currentstroke}%
\pgfsetdash{}{0pt}%
\pgfpathmoveto{\pgfqpoint{3.004392in}{2.168340in}}%
\pgfpathlineto{\pgfqpoint{3.065683in}{2.170058in}}%
\pgfpathlineto{\pgfqpoint{3.002852in}{2.232632in}}%
\pgfpathlineto{\pgfqpoint{2.941358in}{2.231125in}}%
\pgfpathclose%
\pgfusepath{fill}%
\end{pgfscope}%
\begin{pgfscope}%
\pgfpathrectangle{\pgfqpoint{0.150000in}{0.150000in}}{\pgfqpoint{4.700000in}{3.450000in}}%
\pgfusepath{clip}%
\pgfsetbuttcap%
\pgfsetroundjoin%
\definecolor{currentfill}{rgb}{0.815748,0.665671,0.677558}%
\pgfsetfillcolor{currentfill}%
\pgfsetlinewidth{0.000000pt}%
\definecolor{currentstroke}{rgb}{0.000000,0.000000,0.000000}%
\pgfsetstrokecolor{currentstroke}%
\pgfsetdash{}{0pt}%
\pgfpathmoveto{\pgfqpoint{3.067444in}{2.105538in}}%
\pgfpathlineto{\pgfqpoint{3.128532in}{2.107467in}}%
\pgfpathlineto{\pgfqpoint{3.065683in}{2.170058in}}%
\pgfpathlineto{\pgfqpoint{3.004392in}{2.168340in}}%
\pgfpathclose%
\pgfusepath{fill}%
\end{pgfscope}%
\begin{pgfscope}%
\pgfpathrectangle{\pgfqpoint{0.150000in}{0.150000in}}{\pgfqpoint{4.700000in}{3.450000in}}%
\pgfusepath{clip}%
\pgfsetbuttcap%
\pgfsetroundjoin%
\definecolor{currentfill}{rgb}{0.792953,0.624311,0.637669}%
\pgfsetfillcolor{currentfill}%
\pgfsetlinewidth{0.000000pt}%
\definecolor{currentstroke}{rgb}{0.000000,0.000000,0.000000}%
\pgfsetstrokecolor{currentstroke}%
\pgfsetdash{}{0pt}%
\pgfpathmoveto{\pgfqpoint{3.130513in}{2.042719in}}%
\pgfpathlineto{\pgfqpoint{3.191397in}{2.044858in}}%
\pgfpathlineto{\pgfqpoint{3.128532in}{2.107467in}}%
\pgfpathlineto{\pgfqpoint{3.067444in}{2.105538in}}%
\pgfpathclose%
\pgfusepath{fill}%
\end{pgfscope}%
\begin{pgfscope}%
\pgfpathrectangle{\pgfqpoint{0.150000in}{0.150000in}}{\pgfqpoint{4.700000in}{3.450000in}}%
\pgfusepath{clip}%
\pgfsetbuttcap%
\pgfsetroundjoin%
\definecolor{currentfill}{rgb}{0.770159,0.582950,0.597779}%
\pgfsetfillcolor{currentfill}%
\pgfsetlinewidth{0.000000pt}%
\definecolor{currentstroke}{rgb}{0.000000,0.000000,0.000000}%
\pgfsetstrokecolor{currentstroke}%
\pgfsetdash{}{0pt}%
\pgfpathmoveto{\pgfqpoint{3.193599in}{1.979882in}}%
\pgfpathlineto{\pgfqpoint{3.254280in}{1.982233in}}%
\pgfpathlineto{\pgfqpoint{3.191397in}{2.044858in}}%
\pgfpathlineto{\pgfqpoint{3.130513in}{2.042719in}}%
\pgfpathclose%
\pgfusepath{fill}%
\end{pgfscope}%
\begin{pgfscope}%
\pgfpathrectangle{\pgfqpoint{0.150000in}{0.150000in}}{\pgfqpoint{4.700000in}{3.450000in}}%
\pgfusepath{clip}%
\pgfsetbuttcap%
\pgfsetroundjoin%
\definecolor{currentfill}{rgb}{0.546017,0.176241,0.205530}%
\pgfsetfillcolor{currentfill}%
\pgfsetlinewidth{0.000000pt}%
\definecolor{currentstroke}{rgb}{0.000000,0.000000,0.000000}%
\pgfsetstrokecolor{currentstroke}%
\pgfsetdash{}{0pt}%
\pgfpathmoveto{\pgfqpoint{3.887876in}{1.407385in}}%
\pgfpathlineto{\pgfqpoint{3.945771in}{1.405038in}}%
\pgfpathlineto{\pgfqpoint{3.881102in}{1.444892in}}%
\pgfpathlineto{\pgfqpoint{3.823062in}{1.447614in}}%
\pgfpathclose%
\pgfusepath{fill}%
\end{pgfscope}%
\begin{pgfscope}%
\pgfpathrectangle{\pgfqpoint{0.150000in}{0.150000in}}{\pgfqpoint{4.700000in}{3.450000in}}%
\pgfusepath{clip}%
\pgfsetbuttcap%
\pgfsetroundjoin%
\definecolor{currentfill}{rgb}{0.747365,0.541590,0.557889}%
\pgfsetfillcolor{currentfill}%
\pgfsetlinewidth{0.000000pt}%
\definecolor{currentstroke}{rgb}{0.000000,0.000000,0.000000}%
\pgfsetstrokecolor{currentstroke}%
\pgfsetdash{}{0pt}%
\pgfpathmoveto{\pgfqpoint{3.256703in}{1.917028in}}%
\pgfpathlineto{\pgfqpoint{3.317180in}{1.919591in}}%
\pgfpathlineto{\pgfqpoint{3.254280in}{1.982233in}}%
\pgfpathlineto{\pgfqpoint{3.193599in}{1.979882in}}%
\pgfpathclose%
\pgfusepath{fill}%
\end{pgfscope}%
\begin{pgfscope}%
\pgfpathrectangle{\pgfqpoint{0.150000in}{0.150000in}}{\pgfqpoint{4.700000in}{3.450000in}}%
\pgfusepath{clip}%
\pgfsetbuttcap%
\pgfsetroundjoin%
\definecolor{currentfill}{rgb}{0.724571,0.500230,0.517999}%
\pgfsetfillcolor{currentfill}%
\pgfsetlinewidth{0.000000pt}%
\definecolor{currentstroke}{rgb}{0.000000,0.000000,0.000000}%
\pgfsetstrokecolor{currentstroke}%
\pgfsetdash{}{0pt}%
\pgfpathmoveto{\pgfqpoint{3.319824in}{1.854157in}}%
\pgfpathlineto{\pgfqpoint{3.380097in}{1.856931in}}%
\pgfpathlineto{\pgfqpoint{3.317180in}{1.919591in}}%
\pgfpathlineto{\pgfqpoint{3.256703in}{1.917028in}}%
\pgfpathclose%
\pgfusepath{fill}%
\end{pgfscope}%
\begin{pgfscope}%
\pgfpathrectangle{\pgfqpoint{0.150000in}{0.150000in}}{\pgfqpoint{4.700000in}{3.450000in}}%
\pgfusepath{clip}%
\pgfsetbuttcap%
\pgfsetroundjoin%
\definecolor{currentfill}{rgb}{0.701777,0.458869,0.478110}%
\pgfsetfillcolor{currentfill}%
\pgfsetlinewidth{0.000000pt}%
\definecolor{currentstroke}{rgb}{0.000000,0.000000,0.000000}%
\pgfsetstrokecolor{currentstroke}%
\pgfsetdash{}{0pt}%
\pgfpathmoveto{\pgfqpoint{3.382962in}{1.791269in}}%
\pgfpathlineto{\pgfqpoint{3.443031in}{1.794255in}}%
\pgfpathlineto{\pgfqpoint{3.380097in}{1.856931in}}%
\pgfpathlineto{\pgfqpoint{3.319824in}{1.854157in}}%
\pgfpathclose%
\pgfusepath{fill}%
\end{pgfscope}%
\begin{pgfscope}%
\pgfpathrectangle{\pgfqpoint{0.150000in}{0.150000in}}{\pgfqpoint{4.700000in}{3.450000in}}%
\pgfusepath{clip}%
\pgfsetbuttcap%
\pgfsetroundjoin%
\definecolor{currentfill}{rgb}{0.678983,0.417509,0.438220}%
\pgfsetfillcolor{currentfill}%
\pgfsetlinewidth{0.000000pt}%
\definecolor{currentstroke}{rgb}{0.000000,0.000000,0.000000}%
\pgfsetstrokecolor{currentstroke}%
\pgfsetdash{}{0pt}%
\pgfpathmoveto{\pgfqpoint{3.446118in}{1.728363in}}%
\pgfpathlineto{\pgfqpoint{3.505983in}{1.731561in}}%
\pgfpathlineto{\pgfqpoint{3.443031in}{1.794255in}}%
\pgfpathlineto{\pgfqpoint{3.382962in}{1.791269in}}%
\pgfpathclose%
\pgfusepath{fill}%
\end{pgfscope}%
\begin{pgfscope}%
\pgfpathrectangle{\pgfqpoint{0.150000in}{0.150000in}}{\pgfqpoint{4.700000in}{3.450000in}}%
\pgfusepath{clip}%
\pgfsetbuttcap%
\pgfsetroundjoin%
\definecolor{currentfill}{rgb}{0.656189,0.376149,0.398330}%
\pgfsetfillcolor{currentfill}%
\pgfsetlinewidth{0.000000pt}%
\definecolor{currentstroke}{rgb}{0.000000,0.000000,0.000000}%
\pgfsetstrokecolor{currentstroke}%
\pgfsetdash{}{0pt}%
\pgfpathmoveto{\pgfqpoint{3.509290in}{1.665441in}}%
\pgfpathlineto{\pgfqpoint{3.568951in}{1.668850in}}%
\pgfpathlineto{\pgfqpoint{3.505983in}{1.731561in}}%
\pgfpathlineto{\pgfqpoint{3.446118in}{1.728363in}}%
\pgfpathclose%
\pgfusepath{fill}%
\end{pgfscope}%
\begin{pgfscope}%
\pgfpathrectangle{\pgfqpoint{0.150000in}{0.150000in}}{\pgfqpoint{4.700000in}{3.450000in}}%
\pgfusepath{clip}%
\pgfsetbuttcap%
\pgfsetroundjoin%
\definecolor{currentfill}{rgb}{0.633395,0.334789,0.358441}%
\pgfsetfillcolor{currentfill}%
\pgfsetlinewidth{0.000000pt}%
\definecolor{currentstroke}{rgb}{0.000000,0.000000,0.000000}%
\pgfsetstrokecolor{currentstroke}%
\pgfsetdash{}{0pt}%
\pgfpathmoveto{\pgfqpoint{3.572481in}{1.602500in}}%
\pgfpathlineto{\pgfqpoint{3.631937in}{1.606122in}}%
\pgfpathlineto{\pgfqpoint{3.568951in}{1.668850in}}%
\pgfpathlineto{\pgfqpoint{3.509290in}{1.665441in}}%
\pgfpathclose%
\pgfusepath{fill}%
\end{pgfscope}%
\begin{pgfscope}%
\pgfpathrectangle{\pgfqpoint{0.150000in}{0.150000in}}{\pgfqpoint{4.700000in}{3.450000in}}%
\pgfusepath{clip}%
\pgfsetbuttcap%
\pgfsetroundjoin%
\definecolor{currentfill}{rgb}{0.610600,0.293428,0.318551}%
\pgfsetfillcolor{currentfill}%
\pgfsetlinewidth{0.000000pt}%
\definecolor{currentstroke}{rgb}{0.000000,0.000000,0.000000}%
\pgfsetstrokecolor{currentstroke}%
\pgfsetdash{}{0pt}%
\pgfpathmoveto{\pgfqpoint{3.635688in}{1.539543in}}%
\pgfpathlineto{\pgfqpoint{3.694940in}{1.543377in}}%
\pgfpathlineto{\pgfqpoint{3.631937in}{1.606122in}}%
\pgfpathlineto{\pgfqpoint{3.572481in}{1.602500in}}%
\pgfpathclose%
\pgfusepath{fill}%
\end{pgfscope}%
\begin{pgfscope}%
\pgfpathrectangle{\pgfqpoint{0.150000in}{0.150000in}}{\pgfqpoint{4.700000in}{3.450000in}}%
\pgfusepath{clip}%
\pgfsetbuttcap%
\pgfsetroundjoin%
\definecolor{currentfill}{rgb}{0.591605,0.258961,0.285309}%
\pgfsetfillcolor{currentfill}%
\pgfsetlinewidth{0.000000pt}%
\definecolor{currentstroke}{rgb}{0.000000,0.000000,0.000000}%
\pgfsetstrokecolor{currentstroke}%
\pgfsetdash{}{0pt}%
\pgfpathmoveto{\pgfqpoint{3.699796in}{1.490735in}}%
\pgfpathlineto{\pgfqpoint{3.758420in}{1.487616in}}%
\pgfpathlineto{\pgfqpoint{3.694940in}{1.543377in}}%
\pgfpathlineto{\pgfqpoint{3.635688in}{1.539543in}}%
\pgfpathclose%
\pgfusepath{fill}%
\end{pgfscope}%
\begin{pgfscope}%
\pgfpathrectangle{\pgfqpoint{0.150000in}{0.150000in}}{\pgfqpoint{4.700000in}{3.450000in}}%
\pgfusepath{clip}%
\pgfsetbuttcap%
\pgfsetroundjoin%
\definecolor{currentfill}{rgb}{0.542218,0.169347,0.198882}%
\pgfsetfillcolor{currentfill}%
\pgfsetlinewidth{0.000000pt}%
\definecolor{currentstroke}{rgb}{0.000000,0.000000,0.000000}%
\pgfsetstrokecolor{currentstroke}%
\pgfsetdash{}{0pt}%
\pgfpathmoveto{\pgfqpoint{3.952870in}{1.367045in}}%
\pgfpathlineto{\pgfqpoint{4.010621in}{1.365073in}}%
\pgfpathlineto{\pgfqpoint{3.945771in}{1.405038in}}%
\pgfpathlineto{\pgfqpoint{3.887876in}{1.407385in}}%
\pgfpathclose%
\pgfusepath{fill}%
\end{pgfscope}%
\begin{pgfscope}%
\pgfpathrectangle{\pgfqpoint{0.150000in}{0.150000in}}{\pgfqpoint{4.700000in}{3.450000in}}%
\pgfusepath{clip}%
\pgfsetbuttcap%
\pgfsetroundjoin%
\definecolor{currentfill}{rgb}{0.922258,0.931832,0.945236}%
\pgfsetfillcolor{currentfill}%
\pgfsetlinewidth{0.000000pt}%
\definecolor{currentstroke}{rgb}{0.000000,0.000000,0.000000}%
\pgfsetstrokecolor{currentstroke}%
\pgfsetdash{}{0pt}%
\pgfpathmoveto{\pgfqpoint{2.437261in}{2.670106in}}%
\pgfpathlineto{\pgfqpoint{2.500600in}{2.670139in}}%
\pgfpathlineto{\pgfqpoint{2.437703in}{2.732786in}}%
\pgfpathlineto{\pgfqpoint{2.374161in}{2.732965in}}%
\pgfpathclose%
\pgfusepath{fill}%
\end{pgfscope}%
\begin{pgfscope}%
\pgfpathrectangle{\pgfqpoint{0.150000in}{0.150000in}}{\pgfqpoint{4.700000in}{3.450000in}}%
\pgfusepath{clip}%
\pgfsetbuttcap%
\pgfsetroundjoin%
\definecolor{currentfill}{rgb}{0.959574,0.964553,0.971523}%
\pgfsetfillcolor{currentfill}%
\pgfsetlinewidth{0.000000pt}%
\definecolor{currentstroke}{rgb}{0.000000,0.000000,0.000000}%
\pgfsetstrokecolor{currentstroke}%
\pgfsetdash{}{0pt}%
\pgfpathmoveto{\pgfqpoint{2.500379in}{2.607229in}}%
\pgfpathlineto{\pgfqpoint{2.563514in}{2.607474in}}%
\pgfpathlineto{\pgfqpoint{2.500600in}{2.670139in}}%
\pgfpathlineto{\pgfqpoint{2.437261in}{2.670106in}}%
\pgfpathclose%
\pgfusepath{fill}%
\end{pgfscope}%
\begin{pgfscope}%
\pgfpathrectangle{\pgfqpoint{0.150000in}{0.150000in}}{\pgfqpoint{4.700000in}{3.450000in}}%
\pgfusepath{clip}%
\pgfsetbuttcap%
\pgfsetroundjoin%
\definecolor{currentfill}{rgb}{0.996890,0.997273,0.997809}%
\pgfsetfillcolor{currentfill}%
\pgfsetlinewidth{0.000000pt}%
\definecolor{currentstroke}{rgb}{0.000000,0.000000,0.000000}%
\pgfsetstrokecolor{currentstroke}%
\pgfsetdash{}{0pt}%
\pgfpathmoveto{\pgfqpoint{2.563514in}{2.544336in}}%
\pgfpathlineto{\pgfqpoint{2.626444in}{2.544792in}}%
\pgfpathlineto{\pgfqpoint{2.563514in}{2.607474in}}%
\pgfpathlineto{\pgfqpoint{2.500379in}{2.607229in}}%
\pgfpathclose%
\pgfusepath{fill}%
\end{pgfscope}%
\begin{pgfscope}%
\pgfpathrectangle{\pgfqpoint{0.150000in}{0.150000in}}{\pgfqpoint{4.700000in}{3.450000in}}%
\pgfusepath{clip}%
\pgfsetbuttcap%
\pgfsetroundjoin%
\definecolor{currentfill}{rgb}{0.975306,0.955193,0.956786}%
\pgfsetfillcolor{currentfill}%
\pgfsetlinewidth{0.000000pt}%
\definecolor{currentstroke}{rgb}{0.000000,0.000000,0.000000}%
\pgfsetstrokecolor{currentstroke}%
\pgfsetdash{}{0pt}%
\pgfpathmoveto{\pgfqpoint{2.626666in}{2.481425in}}%
\pgfpathlineto{\pgfqpoint{2.689393in}{2.482093in}}%
\pgfpathlineto{\pgfqpoint{2.626444in}{2.544792in}}%
\pgfpathlineto{\pgfqpoint{2.563514in}{2.544336in}}%
\pgfpathclose%
\pgfusepath{fill}%
\end{pgfscope}%
\begin{pgfscope}%
\pgfpathrectangle{\pgfqpoint{0.150000in}{0.150000in}}{\pgfqpoint{4.700000in}{3.450000in}}%
\pgfusepath{clip}%
\pgfsetbuttcap%
\pgfsetroundjoin%
\definecolor{currentfill}{rgb}{0.952512,0.913833,0.916896}%
\pgfsetfillcolor{currentfill}%
\pgfsetlinewidth{0.000000pt}%
\definecolor{currentstroke}{rgb}{0.000000,0.000000,0.000000}%
\pgfsetstrokecolor{currentstroke}%
\pgfsetdash{}{0pt}%
\pgfpathmoveto{\pgfqpoint{2.689835in}{2.418497in}}%
\pgfpathlineto{\pgfqpoint{2.752358in}{2.419377in}}%
\pgfpathlineto{\pgfqpoint{2.689393in}{2.482093in}}%
\pgfpathlineto{\pgfqpoint{2.626666in}{2.481425in}}%
\pgfpathclose%
\pgfusepath{fill}%
\end{pgfscope}%
\begin{pgfscope}%
\pgfpathrectangle{\pgfqpoint{0.150000in}{0.150000in}}{\pgfqpoint{4.700000in}{3.450000in}}%
\pgfusepath{clip}%
\pgfsetbuttcap%
\pgfsetroundjoin%
\definecolor{currentfill}{rgb}{0.580208,0.238281,0.265365}%
\pgfsetfillcolor{currentfill}%
\pgfsetlinewidth{0.000000pt}%
\definecolor{currentstroke}{rgb}{0.000000,0.000000,0.000000}%
\pgfsetstrokecolor{currentstroke}%
\pgfsetdash{}{0pt}%
\pgfpathmoveto{\pgfqpoint{3.764578in}{1.450241in}}%
\pgfpathlineto{\pgfqpoint{3.823062in}{1.447614in}}%
\pgfpathlineto{\pgfqpoint{3.758420in}{1.487616in}}%
\pgfpathlineto{\pgfqpoint{3.699796in}{1.490735in}}%
\pgfpathclose%
\pgfusepath{fill}%
\end{pgfscope}%
\begin{pgfscope}%
\pgfpathrectangle{\pgfqpoint{0.150000in}{0.150000in}}{\pgfqpoint{4.700000in}{3.450000in}}%
\pgfusepath{clip}%
\pgfsetbuttcap%
\pgfsetroundjoin%
\definecolor{currentfill}{rgb}{0.929718,0.872472,0.877007}%
\pgfsetfillcolor{currentfill}%
\pgfsetlinewidth{0.000000pt}%
\definecolor{currentstroke}{rgb}{0.000000,0.000000,0.000000}%
\pgfsetstrokecolor{currentstroke}%
\pgfsetdash{}{0pt}%
\pgfpathmoveto{\pgfqpoint{2.753022in}{2.355551in}}%
\pgfpathlineto{\pgfqpoint{2.815341in}{2.356643in}}%
\pgfpathlineto{\pgfqpoint{2.752358in}{2.419377in}}%
\pgfpathlineto{\pgfqpoint{2.689835in}{2.418497in}}%
\pgfpathclose%
\pgfusepath{fill}%
\end{pgfscope}%
\begin{pgfscope}%
\pgfpathrectangle{\pgfqpoint{0.150000in}{0.150000in}}{\pgfqpoint{4.700000in}{3.450000in}}%
\pgfusepath{clip}%
\pgfsetbuttcap%
\pgfsetroundjoin%
\definecolor{currentfill}{rgb}{0.906924,0.831112,0.837117}%
\pgfsetfillcolor{currentfill}%
\pgfsetlinewidth{0.000000pt}%
\definecolor{currentstroke}{rgb}{0.000000,0.000000,0.000000}%
\pgfsetstrokecolor{currentstroke}%
\pgfsetdash{}{0pt}%
\pgfpathmoveto{\pgfqpoint{2.816226in}{2.292588in}}%
\pgfpathlineto{\pgfqpoint{2.878341in}{2.293893in}}%
\pgfpathlineto{\pgfqpoint{2.815341in}{2.356643in}}%
\pgfpathlineto{\pgfqpoint{2.753022in}{2.355551in}}%
\pgfpathclose%
\pgfusepath{fill}%
\end{pgfscope}%
\begin{pgfscope}%
\pgfpathrectangle{\pgfqpoint{0.150000in}{0.150000in}}{\pgfqpoint{4.700000in}{3.450000in}}%
\pgfusepath{clip}%
\pgfsetbuttcap%
\pgfsetroundjoin%
\definecolor{currentfill}{rgb}{0.884130,0.789752,0.797227}%
\pgfsetfillcolor{currentfill}%
\pgfsetlinewidth{0.000000pt}%
\definecolor{currentstroke}{rgb}{0.000000,0.000000,0.000000}%
\pgfsetstrokecolor{currentstroke}%
\pgfsetdash{}{0pt}%
\pgfpathmoveto{\pgfqpoint{2.879448in}{2.229608in}}%
\pgfpathlineto{\pgfqpoint{2.941358in}{2.231125in}}%
\pgfpathlineto{\pgfqpoint{2.878341in}{2.293893in}}%
\pgfpathlineto{\pgfqpoint{2.816226in}{2.292588in}}%
\pgfpathclose%
\pgfusepath{fill}%
\end{pgfscope}%
\begin{pgfscope}%
\pgfpathrectangle{\pgfqpoint{0.150000in}{0.150000in}}{\pgfqpoint{4.700000in}{3.450000in}}%
\pgfusepath{clip}%
\pgfsetbuttcap%
\pgfsetroundjoin%
\definecolor{currentfill}{rgb}{0.534620,0.155561,0.185585}%
\pgfsetfillcolor{currentfill}%
\pgfsetlinewidth{0.000000pt}%
\definecolor{currentstroke}{rgb}{0.000000,0.000000,0.000000}%
\pgfsetstrokecolor{currentstroke}%
\pgfsetdash{}{0pt}%
\pgfpathmoveto{\pgfqpoint{4.018054in}{1.326706in}}%
\pgfpathlineto{\pgfqpoint{4.075653in}{1.324996in}}%
\pgfpathlineto{\pgfqpoint{4.010621in}{1.365073in}}%
\pgfpathlineto{\pgfqpoint{3.952870in}{1.367045in}}%
\pgfpathclose%
\pgfusepath{fill}%
\end{pgfscope}%
\begin{pgfscope}%
\pgfpathrectangle{\pgfqpoint{0.150000in}{0.150000in}}{\pgfqpoint{4.700000in}{3.450000in}}%
\pgfusepath{clip}%
\pgfsetbuttcap%
\pgfsetroundjoin%
\definecolor{currentfill}{rgb}{0.861336,0.748392,0.757338}%
\pgfsetfillcolor{currentfill}%
\pgfsetlinewidth{0.000000pt}%
\definecolor{currentstroke}{rgb}{0.000000,0.000000,0.000000}%
\pgfsetstrokecolor{currentstroke}%
\pgfsetdash{}{0pt}%
\pgfpathmoveto{\pgfqpoint{2.942687in}{2.166611in}}%
\pgfpathlineto{\pgfqpoint{3.004392in}{2.168340in}}%
\pgfpathlineto{\pgfqpoint{2.941358in}{2.231125in}}%
\pgfpathlineto{\pgfqpoint{2.879448in}{2.229608in}}%
\pgfpathclose%
\pgfusepath{fill}%
\end{pgfscope}%
\begin{pgfscope}%
\pgfpathrectangle{\pgfqpoint{0.150000in}{0.150000in}}{\pgfqpoint{4.700000in}{3.450000in}}%
\pgfusepath{clip}%
\pgfsetbuttcap%
\pgfsetroundjoin%
\definecolor{currentfill}{rgb}{0.838542,0.707031,0.717448}%
\pgfsetfillcolor{currentfill}%
\pgfsetlinewidth{0.000000pt}%
\definecolor{currentstroke}{rgb}{0.000000,0.000000,0.000000}%
\pgfsetstrokecolor{currentstroke}%
\pgfsetdash{}{0pt}%
\pgfpathmoveto{\pgfqpoint{3.005944in}{2.103596in}}%
\pgfpathlineto{\pgfqpoint{3.067444in}{2.105538in}}%
\pgfpathlineto{\pgfqpoint{3.004392in}{2.168340in}}%
\pgfpathlineto{\pgfqpoint{2.942687in}{2.166611in}}%
\pgfpathclose%
\pgfusepath{fill}%
\end{pgfscope}%
\begin{pgfscope}%
\pgfpathrectangle{\pgfqpoint{0.150000in}{0.150000in}}{\pgfqpoint{4.700000in}{3.450000in}}%
\pgfusepath{clip}%
\pgfsetbuttcap%
\pgfsetroundjoin%
\definecolor{currentfill}{rgb}{0.815748,0.665671,0.677558}%
\pgfsetfillcolor{currentfill}%
\pgfsetlinewidth{0.000000pt}%
\definecolor{currentstroke}{rgb}{0.000000,0.000000,0.000000}%
\pgfsetstrokecolor{currentstroke}%
\pgfsetdash{}{0pt}%
\pgfpathmoveto{\pgfqpoint{3.069217in}{2.040564in}}%
\pgfpathlineto{\pgfqpoint{3.130513in}{2.042719in}}%
\pgfpathlineto{\pgfqpoint{3.067444in}{2.105538in}}%
\pgfpathlineto{\pgfqpoint{3.005944in}{2.103596in}}%
\pgfpathclose%
\pgfusepath{fill}%
\end{pgfscope}%
\begin{pgfscope}%
\pgfpathrectangle{\pgfqpoint{0.150000in}{0.150000in}}{\pgfqpoint{4.700000in}{3.450000in}}%
\pgfusepath{clip}%
\pgfsetbuttcap%
\pgfsetroundjoin%
\definecolor{currentfill}{rgb}{0.792953,0.624311,0.637669}%
\pgfsetfillcolor{currentfill}%
\pgfsetlinewidth{0.000000pt}%
\definecolor{currentstroke}{rgb}{0.000000,0.000000,0.000000}%
\pgfsetstrokecolor{currentstroke}%
\pgfsetdash{}{0pt}%
\pgfpathmoveto{\pgfqpoint{3.132509in}{1.977515in}}%
\pgfpathlineto{\pgfqpoint{3.193599in}{1.979882in}}%
\pgfpathlineto{\pgfqpoint{3.130513in}{2.042719in}}%
\pgfpathlineto{\pgfqpoint{3.069217in}{2.040564in}}%
\pgfpathclose%
\pgfusepath{fill}%
\end{pgfscope}%
\begin{pgfscope}%
\pgfpathrectangle{\pgfqpoint{0.150000in}{0.150000in}}{\pgfqpoint{4.700000in}{3.450000in}}%
\pgfusepath{clip}%
\pgfsetbuttcap%
\pgfsetroundjoin%
\definecolor{currentfill}{rgb}{0.770159,0.582950,0.597779}%
\pgfsetfillcolor{currentfill}%
\pgfsetlinewidth{0.000000pt}%
\definecolor{currentstroke}{rgb}{0.000000,0.000000,0.000000}%
\pgfsetstrokecolor{currentstroke}%
\pgfsetdash{}{0pt}%
\pgfpathmoveto{\pgfqpoint{3.195817in}{1.914448in}}%
\pgfpathlineto{\pgfqpoint{3.256703in}{1.917028in}}%
\pgfpathlineto{\pgfqpoint{3.193599in}{1.979882in}}%
\pgfpathlineto{\pgfqpoint{3.132509in}{1.977515in}}%
\pgfpathclose%
\pgfusepath{fill}%
\end{pgfscope}%
\begin{pgfscope}%
\pgfpathrectangle{\pgfqpoint{0.150000in}{0.150000in}}{\pgfqpoint{4.700000in}{3.450000in}}%
\pgfusepath{clip}%
\pgfsetbuttcap%
\pgfsetroundjoin%
\definecolor{currentfill}{rgb}{0.747365,0.541590,0.557889}%
\pgfsetfillcolor{currentfill}%
\pgfsetlinewidth{0.000000pt}%
\definecolor{currentstroke}{rgb}{0.000000,0.000000,0.000000}%
\pgfsetstrokecolor{currentstroke}%
\pgfsetdash{}{0pt}%
\pgfpathmoveto{\pgfqpoint{3.259143in}{1.851364in}}%
\pgfpathlineto{\pgfqpoint{3.319824in}{1.854157in}}%
\pgfpathlineto{\pgfqpoint{3.256703in}{1.917028in}}%
\pgfpathlineto{\pgfqpoint{3.195817in}{1.914448in}}%
\pgfpathclose%
\pgfusepath{fill}%
\end{pgfscope}%
\begin{pgfscope}%
\pgfpathrectangle{\pgfqpoint{0.150000in}{0.150000in}}{\pgfqpoint{4.700000in}{3.450000in}}%
\pgfusepath{clip}%
\pgfsetbuttcap%
\pgfsetroundjoin%
\definecolor{currentfill}{rgb}{0.724571,0.500230,0.517999}%
\pgfsetfillcolor{currentfill}%
\pgfsetlinewidth{0.000000pt}%
\definecolor{currentstroke}{rgb}{0.000000,0.000000,0.000000}%
\pgfsetstrokecolor{currentstroke}%
\pgfsetdash{}{0pt}%
\pgfpathmoveto{\pgfqpoint{3.322487in}{1.788263in}}%
\pgfpathlineto{\pgfqpoint{3.382962in}{1.791269in}}%
\pgfpathlineto{\pgfqpoint{3.319824in}{1.854157in}}%
\pgfpathlineto{\pgfqpoint{3.259143in}{1.851364in}}%
\pgfpathclose%
\pgfusepath{fill}%
\end{pgfscope}%
\begin{pgfscope}%
\pgfpathrectangle{\pgfqpoint{0.150000in}{0.150000in}}{\pgfqpoint{4.700000in}{3.450000in}}%
\pgfusepath{clip}%
\pgfsetbuttcap%
\pgfsetroundjoin%
\definecolor{currentfill}{rgb}{0.576409,0.231388,0.258716}%
\pgfsetfillcolor{currentfill}%
\pgfsetlinewidth{0.000000pt}%
\definecolor{currentstroke}{rgb}{0.000000,0.000000,0.000000}%
\pgfsetstrokecolor{currentstroke}%
\pgfsetdash{}{0pt}%
\pgfpathmoveto{\pgfqpoint{3.829546in}{1.409749in}}%
\pgfpathlineto{\pgfqpoint{3.887876in}{1.407385in}}%
\pgfpathlineto{\pgfqpoint{3.823062in}{1.447614in}}%
\pgfpathlineto{\pgfqpoint{3.764578in}{1.450241in}}%
\pgfpathclose%
\pgfusepath{fill}%
\end{pgfscope}%
\begin{pgfscope}%
\pgfpathrectangle{\pgfqpoint{0.150000in}{0.150000in}}{\pgfqpoint{4.700000in}{3.450000in}}%
\pgfusepath{clip}%
\pgfsetbuttcap%
\pgfsetroundjoin%
\definecolor{currentfill}{rgb}{0.701777,0.458869,0.478110}%
\pgfsetfillcolor{currentfill}%
\pgfsetlinewidth{0.000000pt}%
\definecolor{currentstroke}{rgb}{0.000000,0.000000,0.000000}%
\pgfsetstrokecolor{currentstroke}%
\pgfsetdash{}{0pt}%
\pgfpathmoveto{\pgfqpoint{3.385848in}{1.725144in}}%
\pgfpathlineto{\pgfqpoint{3.446118in}{1.728363in}}%
\pgfpathlineto{\pgfqpoint{3.382962in}{1.791269in}}%
\pgfpathlineto{\pgfqpoint{3.322487in}{1.788263in}}%
\pgfpathclose%
\pgfusepath{fill}%
\end{pgfscope}%
\begin{pgfscope}%
\pgfpathrectangle{\pgfqpoint{0.150000in}{0.150000in}}{\pgfqpoint{4.700000in}{3.450000in}}%
\pgfusepath{clip}%
\pgfsetbuttcap%
\pgfsetroundjoin%
\definecolor{currentfill}{rgb}{0.678983,0.417509,0.438220}%
\pgfsetfillcolor{currentfill}%
\pgfsetlinewidth{0.000000pt}%
\definecolor{currentstroke}{rgb}{0.000000,0.000000,0.000000}%
\pgfsetstrokecolor{currentstroke}%
\pgfsetdash{}{0pt}%
\pgfpathmoveto{\pgfqpoint{3.449226in}{1.662008in}}%
\pgfpathlineto{\pgfqpoint{3.509290in}{1.665441in}}%
\pgfpathlineto{\pgfqpoint{3.446118in}{1.728363in}}%
\pgfpathlineto{\pgfqpoint{3.385848in}{1.725144in}}%
\pgfpathclose%
\pgfusepath{fill}%
\end{pgfscope}%
\begin{pgfscope}%
\pgfpathrectangle{\pgfqpoint{0.150000in}{0.150000in}}{\pgfqpoint{4.700000in}{3.450000in}}%
\pgfusepath{clip}%
\pgfsetbuttcap%
\pgfsetroundjoin%
\definecolor{currentfill}{rgb}{0.656189,0.376149,0.398330}%
\pgfsetfillcolor{currentfill}%
\pgfsetlinewidth{0.000000pt}%
\definecolor{currentstroke}{rgb}{0.000000,0.000000,0.000000}%
\pgfsetstrokecolor{currentstroke}%
\pgfsetdash{}{0pt}%
\pgfpathmoveto{\pgfqpoint{3.512622in}{1.598854in}}%
\pgfpathlineto{\pgfqpoint{3.572481in}{1.602500in}}%
\pgfpathlineto{\pgfqpoint{3.509290in}{1.665441in}}%
\pgfpathlineto{\pgfqpoint{3.449226in}{1.662008in}}%
\pgfpathclose%
\pgfusepath{fill}%
\end{pgfscope}%
\begin{pgfscope}%
\pgfpathrectangle{\pgfqpoint{0.150000in}{0.150000in}}{\pgfqpoint{4.700000in}{3.450000in}}%
\pgfusepath{clip}%
\pgfsetbuttcap%
\pgfsetroundjoin%
\definecolor{currentfill}{rgb}{0.530821,0.148667,0.178937}%
\pgfsetfillcolor{currentfill}%
\pgfsetlinewidth{0.000000pt}%
\definecolor{currentstroke}{rgb}{0.000000,0.000000,0.000000}%
\pgfsetstrokecolor{currentstroke}%
\pgfsetdash{}{0pt}%
\pgfpathmoveto{\pgfqpoint{4.083412in}{1.286141in}}%
\pgfpathlineto{\pgfqpoint{4.140867in}{1.284807in}}%
\pgfpathlineto{\pgfqpoint{4.075653in}{1.324996in}}%
\pgfpathlineto{\pgfqpoint{4.018054in}{1.326706in}}%
\pgfpathclose%
\pgfusepath{fill}%
\end{pgfscope}%
\begin{pgfscope}%
\pgfpathrectangle{\pgfqpoint{0.150000in}{0.150000in}}{\pgfqpoint{4.700000in}{3.450000in}}%
\pgfusepath{clip}%
\pgfsetbuttcap%
\pgfsetroundjoin%
\definecolor{currentfill}{rgb}{0.633395,0.334789,0.358441}%
\pgfsetfillcolor{currentfill}%
\pgfsetlinewidth{0.000000pt}%
\definecolor{currentstroke}{rgb}{0.000000,0.000000,0.000000}%
\pgfsetstrokecolor{currentstroke}%
\pgfsetdash{}{0pt}%
\pgfpathmoveto{\pgfqpoint{3.576035in}{1.535683in}}%
\pgfpathlineto{\pgfqpoint{3.635688in}{1.539543in}}%
\pgfpathlineto{\pgfqpoint{3.572481in}{1.602500in}}%
\pgfpathlineto{\pgfqpoint{3.512622in}{1.598854in}}%
\pgfpathclose%
\pgfusepath{fill}%
\end{pgfscope}%
\begin{pgfscope}%
\pgfpathrectangle{\pgfqpoint{0.150000in}{0.150000in}}{\pgfqpoint{4.700000in}{3.450000in}}%
\pgfusepath{clip}%
\pgfsetbuttcap%
\pgfsetroundjoin%
\definecolor{currentfill}{rgb}{0.614400,0.300322,0.325199}%
\pgfsetfillcolor{currentfill}%
\pgfsetlinewidth{0.000000pt}%
\definecolor{currentstroke}{rgb}{0.000000,0.000000,0.000000}%
\pgfsetstrokecolor{currentstroke}%
\pgfsetdash{}{0pt}%
\pgfpathmoveto{\pgfqpoint{3.640723in}{1.493762in}}%
\pgfpathlineto{\pgfqpoint{3.699796in}{1.490735in}}%
\pgfpathlineto{\pgfqpoint{3.635688in}{1.539543in}}%
\pgfpathlineto{\pgfqpoint{3.576035in}{1.535683in}}%
\pgfpathclose%
\pgfusepath{fill}%
\end{pgfscope}%
\begin{pgfscope}%
\pgfpathrectangle{\pgfqpoint{0.150000in}{0.150000in}}{\pgfqpoint{4.700000in}{3.450000in}}%
\pgfusepath{clip}%
\pgfsetbuttcap%
\pgfsetroundjoin%
\definecolor{currentfill}{rgb}{0.568811,0.217601,0.245420}%
\pgfsetfillcolor{currentfill}%
\pgfsetlinewidth{0.000000pt}%
\definecolor{currentstroke}{rgb}{0.000000,0.000000,0.000000}%
\pgfsetstrokecolor{currentstroke}%
\pgfsetdash{}{0pt}%
\pgfpathmoveto{\pgfqpoint{3.894694in}{1.369146in}}%
\pgfpathlineto{\pgfqpoint{3.952870in}{1.367045in}}%
\pgfpathlineto{\pgfqpoint{3.887876in}{1.407385in}}%
\pgfpathlineto{\pgfqpoint{3.829546in}{1.409749in}}%
\pgfpathclose%
\pgfusepath{fill}%
\end{pgfscope}%
\begin{pgfscope}%
\pgfpathrectangle{\pgfqpoint{0.150000in}{0.150000in}}{\pgfqpoint{4.700000in}{3.450000in}}%
\pgfusepath{clip}%
\pgfsetbuttcap%
\pgfsetroundjoin%
\definecolor{currentfill}{rgb}{0.884942,0.899112,0.918949}%
\pgfsetfillcolor{currentfill}%
\pgfsetlinewidth{0.000000pt}%
\definecolor{currentstroke}{rgb}{0.000000,0.000000,0.000000}%
\pgfsetstrokecolor{currentstroke}%
\pgfsetdash{}{0pt}%
\pgfpathmoveto{\pgfqpoint{2.373493in}{2.670073in}}%
\pgfpathlineto{\pgfqpoint{2.437261in}{2.670106in}}%
\pgfpathlineto{\pgfqpoint{2.374161in}{2.732965in}}%
\pgfpathlineto{\pgfqpoint{2.310188in}{2.733145in}}%
\pgfpathclose%
\pgfusepath{fill}%
\end{pgfscope}%
\begin{pgfscope}%
\pgfpathrectangle{\pgfqpoint{0.150000in}{0.150000in}}{\pgfqpoint{4.700000in}{3.450000in}}%
\pgfusepath{clip}%
\pgfsetbuttcap%
\pgfsetroundjoin%
\definecolor{currentfill}{rgb}{0.922258,0.931832,0.945236}%
\pgfsetfillcolor{currentfill}%
\pgfsetlinewidth{0.000000pt}%
\definecolor{currentstroke}{rgb}{0.000000,0.000000,0.000000}%
\pgfsetstrokecolor{currentstroke}%
\pgfsetdash{}{0pt}%
\pgfpathmoveto{\pgfqpoint{2.436816in}{2.606983in}}%
\pgfpathlineto{\pgfqpoint{2.500379in}{2.607229in}}%
\pgfpathlineto{\pgfqpoint{2.437261in}{2.670106in}}%
\pgfpathlineto{\pgfqpoint{2.373493in}{2.670073in}}%
\pgfpathclose%
\pgfusepath{fill}%
\end{pgfscope}%
\begin{pgfscope}%
\pgfpathrectangle{\pgfqpoint{0.150000in}{0.150000in}}{\pgfqpoint{4.700000in}{3.450000in}}%
\pgfusepath{clip}%
\pgfsetbuttcap%
\pgfsetroundjoin%
\definecolor{currentfill}{rgb}{0.959574,0.964553,0.971523}%
\pgfsetfillcolor{currentfill}%
\pgfsetlinewidth{0.000000pt}%
\definecolor{currentstroke}{rgb}{0.000000,0.000000,0.000000}%
\pgfsetstrokecolor{currentstroke}%
\pgfsetdash{}{0pt}%
\pgfpathmoveto{\pgfqpoint{2.500156in}{2.543876in}}%
\pgfpathlineto{\pgfqpoint{2.563514in}{2.544336in}}%
\pgfpathlineto{\pgfqpoint{2.500379in}{2.607229in}}%
\pgfpathlineto{\pgfqpoint{2.436816in}{2.606983in}}%
\pgfpathclose%
\pgfusepath{fill}%
\end{pgfscope}%
\begin{pgfscope}%
\pgfpathrectangle{\pgfqpoint{0.150000in}{0.150000in}}{\pgfqpoint{4.700000in}{3.450000in}}%
\pgfusepath{clip}%
\pgfsetbuttcap%
\pgfsetroundjoin%
\definecolor{currentfill}{rgb}{0.996890,0.997273,0.997809}%
\pgfsetfillcolor{currentfill}%
\pgfsetlinewidth{0.000000pt}%
\definecolor{currentstroke}{rgb}{0.000000,0.000000,0.000000}%
\pgfsetstrokecolor{currentstroke}%
\pgfsetdash{}{0pt}%
\pgfpathmoveto{\pgfqpoint{2.563514in}{2.480752in}}%
\pgfpathlineto{\pgfqpoint{2.626666in}{2.481425in}}%
\pgfpathlineto{\pgfqpoint{2.563514in}{2.544336in}}%
\pgfpathlineto{\pgfqpoint{2.500156in}{2.543876in}}%
\pgfpathclose%
\pgfusepath{fill}%
\end{pgfscope}%
\begin{pgfscope}%
\pgfpathrectangle{\pgfqpoint{0.150000in}{0.150000in}}{\pgfqpoint{4.700000in}{3.450000in}}%
\pgfusepath{clip}%
\pgfsetbuttcap%
\pgfsetroundjoin%
\definecolor{currentfill}{rgb}{0.975306,0.955193,0.956786}%
\pgfsetfillcolor{currentfill}%
\pgfsetlinewidth{0.000000pt}%
\definecolor{currentstroke}{rgb}{0.000000,0.000000,0.000000}%
\pgfsetstrokecolor{currentstroke}%
\pgfsetdash{}{0pt}%
\pgfpathmoveto{\pgfqpoint{2.626889in}{2.417611in}}%
\pgfpathlineto{\pgfqpoint{2.689835in}{2.418497in}}%
\pgfpathlineto{\pgfqpoint{2.626666in}{2.481425in}}%
\pgfpathlineto{\pgfqpoint{2.563514in}{2.480752in}}%
\pgfpathclose%
\pgfusepath{fill}%
\end{pgfscope}%
\begin{pgfscope}%
\pgfpathrectangle{\pgfqpoint{0.150000in}{0.150000in}}{\pgfqpoint{4.700000in}{3.450000in}}%
\pgfusepath{clip}%
\pgfsetbuttcap%
\pgfsetroundjoin%
\definecolor{currentfill}{rgb}{0.952512,0.913833,0.916896}%
\pgfsetfillcolor{currentfill}%
\pgfsetlinewidth{0.000000pt}%
\definecolor{currentstroke}{rgb}{0.000000,0.000000,0.000000}%
\pgfsetstrokecolor{currentstroke}%
\pgfsetdash{}{0pt}%
\pgfpathmoveto{\pgfqpoint{2.690281in}{2.354452in}}%
\pgfpathlineto{\pgfqpoint{2.753022in}{2.355551in}}%
\pgfpathlineto{\pgfqpoint{2.689835in}{2.418497in}}%
\pgfpathlineto{\pgfqpoint{2.626889in}{2.417611in}}%
\pgfpathclose%
\pgfusepath{fill}%
\end{pgfscope}%
\begin{pgfscope}%
\pgfpathrectangle{\pgfqpoint{0.150000in}{0.150000in}}{\pgfqpoint{4.700000in}{3.450000in}}%
\pgfusepath{clip}%
\pgfsetbuttcap%
\pgfsetroundjoin%
\definecolor{currentfill}{rgb}{0.929718,0.872472,0.877007}%
\pgfsetfillcolor{currentfill}%
\pgfsetlinewidth{0.000000pt}%
\definecolor{currentstroke}{rgb}{0.000000,0.000000,0.000000}%
\pgfsetstrokecolor{currentstroke}%
\pgfsetdash{}{0pt}%
\pgfpathmoveto{\pgfqpoint{2.753691in}{2.291275in}}%
\pgfpathlineto{\pgfqpoint{2.816226in}{2.292588in}}%
\pgfpathlineto{\pgfqpoint{2.753022in}{2.355551in}}%
\pgfpathlineto{\pgfqpoint{2.690281in}{2.354452in}}%
\pgfpathclose%
\pgfusepath{fill}%
\end{pgfscope}%
\begin{pgfscope}%
\pgfpathrectangle{\pgfqpoint{0.150000in}{0.150000in}}{\pgfqpoint{4.700000in}{3.450000in}}%
\pgfusepath{clip}%
\pgfsetbuttcap%
\pgfsetroundjoin%
\definecolor{currentfill}{rgb}{0.606801,0.286535,0.311903}%
\pgfsetfillcolor{currentfill}%
\pgfsetlinewidth{0.000000pt}%
\definecolor{currentstroke}{rgb}{0.000000,0.000000,0.000000}%
\pgfsetstrokecolor{currentstroke}%
\pgfsetdash{}{0pt}%
\pgfpathmoveto{\pgfqpoint{3.705660in}{1.453003in}}%
\pgfpathlineto{\pgfqpoint{3.764578in}{1.450241in}}%
\pgfpathlineto{\pgfqpoint{3.699796in}{1.490735in}}%
\pgfpathlineto{\pgfqpoint{3.640723in}{1.493762in}}%
\pgfpathclose%
\pgfusepath{fill}%
\end{pgfscope}%
\begin{pgfscope}%
\pgfpathrectangle{\pgfqpoint{0.150000in}{0.150000in}}{\pgfqpoint{4.700000in}{3.450000in}}%
\pgfusepath{clip}%
\pgfsetbuttcap%
\pgfsetroundjoin%
\definecolor{currentfill}{rgb}{0.906924,0.831112,0.837117}%
\pgfsetfillcolor{currentfill}%
\pgfsetlinewidth{0.000000pt}%
\definecolor{currentstroke}{rgb}{0.000000,0.000000,0.000000}%
\pgfsetstrokecolor{currentstroke}%
\pgfsetdash{}{0pt}%
\pgfpathmoveto{\pgfqpoint{2.817118in}{2.228081in}}%
\pgfpathlineto{\pgfqpoint{2.879448in}{2.229608in}}%
\pgfpathlineto{\pgfqpoint{2.816226in}{2.292588in}}%
\pgfpathlineto{\pgfqpoint{2.753691in}{2.291275in}}%
\pgfpathclose%
\pgfusepath{fill}%
\end{pgfscope}%
\begin{pgfscope}%
\pgfpathrectangle{\pgfqpoint{0.150000in}{0.150000in}}{\pgfqpoint{4.700000in}{3.450000in}}%
\pgfusepath{clip}%
\pgfsetbuttcap%
\pgfsetroundjoin%
\definecolor{currentfill}{rgb}{0.884130,0.789752,0.797227}%
\pgfsetfillcolor{currentfill}%
\pgfsetlinewidth{0.000000pt}%
\definecolor{currentstroke}{rgb}{0.000000,0.000000,0.000000}%
\pgfsetstrokecolor{currentstroke}%
\pgfsetdash{}{0pt}%
\pgfpathmoveto{\pgfqpoint{2.880563in}{2.164870in}}%
\pgfpathlineto{\pgfqpoint{2.942687in}{2.166611in}}%
\pgfpathlineto{\pgfqpoint{2.879448in}{2.229608in}}%
\pgfpathlineto{\pgfqpoint{2.817118in}{2.228081in}}%
\pgfpathclose%
\pgfusepath{fill}%
\end{pgfscope}%
\begin{pgfscope}%
\pgfpathrectangle{\pgfqpoint{0.150000in}{0.150000in}}{\pgfqpoint{4.700000in}{3.450000in}}%
\pgfusepath{clip}%
\pgfsetbuttcap%
\pgfsetroundjoin%
\definecolor{currentfill}{rgb}{0.861336,0.748392,0.757338}%
\pgfsetfillcolor{currentfill}%
\pgfsetlinewidth{0.000000pt}%
\definecolor{currentstroke}{rgb}{0.000000,0.000000,0.000000}%
\pgfsetstrokecolor{currentstroke}%
\pgfsetdash{}{0pt}%
\pgfpathmoveto{\pgfqpoint{2.944026in}{2.101641in}}%
\pgfpathlineto{\pgfqpoint{3.005944in}{2.103596in}}%
\pgfpathlineto{\pgfqpoint{2.942687in}{2.166611in}}%
\pgfpathlineto{\pgfqpoint{2.880563in}{2.164870in}}%
\pgfpathclose%
\pgfusepath{fill}%
\end{pgfscope}%
\begin{pgfscope}%
\pgfpathrectangle{\pgfqpoint{0.150000in}{0.150000in}}{\pgfqpoint{4.700000in}{3.450000in}}%
\pgfusepath{clip}%
\pgfsetbuttcap%
\pgfsetroundjoin%
\definecolor{currentfill}{rgb}{0.565012,0.210708,0.238771}%
\pgfsetfillcolor{currentfill}%
\pgfsetlinewidth{0.000000pt}%
\definecolor{currentstroke}{rgb}{0.000000,0.000000,0.000000}%
\pgfsetstrokecolor{currentstroke}%
\pgfsetdash{}{0pt}%
\pgfpathmoveto{\pgfqpoint{3.960016in}{1.328314in}}%
\pgfpathlineto{\pgfqpoint{4.018054in}{1.326706in}}%
\pgfpathlineto{\pgfqpoint{3.952870in}{1.367045in}}%
\pgfpathlineto{\pgfqpoint{3.894694in}{1.369146in}}%
\pgfpathclose%
\pgfusepath{fill}%
\end{pgfscope}%
\begin{pgfscope}%
\pgfpathrectangle{\pgfqpoint{0.150000in}{0.150000in}}{\pgfqpoint{4.700000in}{3.450000in}}%
\pgfusepath{clip}%
\pgfsetbuttcap%
\pgfsetroundjoin%
\definecolor{currentfill}{rgb}{0.838542,0.707031,0.717448}%
\pgfsetfillcolor{currentfill}%
\pgfsetlinewidth{0.000000pt}%
\definecolor{currentstroke}{rgb}{0.000000,0.000000,0.000000}%
\pgfsetstrokecolor{currentstroke}%
\pgfsetdash{}{0pt}%
\pgfpathmoveto{\pgfqpoint{3.007506in}{2.038395in}}%
\pgfpathlineto{\pgfqpoint{3.069217in}{2.040564in}}%
\pgfpathlineto{\pgfqpoint{3.005944in}{2.103596in}}%
\pgfpathlineto{\pgfqpoint{2.944026in}{2.101641in}}%
\pgfpathclose%
\pgfusepath{fill}%
\end{pgfscope}%
\begin{pgfscope}%
\pgfpathrectangle{\pgfqpoint{0.150000in}{0.150000in}}{\pgfqpoint{4.700000in}{3.450000in}}%
\pgfusepath{clip}%
\pgfsetbuttcap%
\pgfsetroundjoin%
\definecolor{currentfill}{rgb}{0.815748,0.665671,0.677558}%
\pgfsetfillcolor{currentfill}%
\pgfsetlinewidth{0.000000pt}%
\definecolor{currentstroke}{rgb}{0.000000,0.000000,0.000000}%
\pgfsetstrokecolor{currentstroke}%
\pgfsetdash{}{0pt}%
\pgfpathmoveto{\pgfqpoint{3.071003in}{1.975131in}}%
\pgfpathlineto{\pgfqpoint{3.132509in}{1.977515in}}%
\pgfpathlineto{\pgfqpoint{3.069217in}{2.040564in}}%
\pgfpathlineto{\pgfqpoint{3.007506in}{2.038395in}}%
\pgfpathclose%
\pgfusepath{fill}%
\end{pgfscope}%
\begin{pgfscope}%
\pgfpathrectangle{\pgfqpoint{0.150000in}{0.150000in}}{\pgfqpoint{4.700000in}{3.450000in}}%
\pgfusepath{clip}%
\pgfsetbuttcap%
\pgfsetroundjoin%
\definecolor{currentfill}{rgb}{0.792953,0.624311,0.637669}%
\pgfsetfillcolor{currentfill}%
\pgfsetlinewidth{0.000000pt}%
\definecolor{currentstroke}{rgb}{0.000000,0.000000,0.000000}%
\pgfsetstrokecolor{currentstroke}%
\pgfsetdash{}{0pt}%
\pgfpathmoveto{\pgfqpoint{3.134518in}{1.911850in}}%
\pgfpathlineto{\pgfqpoint{3.195817in}{1.914448in}}%
\pgfpathlineto{\pgfqpoint{3.132509in}{1.977515in}}%
\pgfpathlineto{\pgfqpoint{3.071003in}{1.975131in}}%
\pgfpathclose%
\pgfusepath{fill}%
\end{pgfscope}%
\begin{pgfscope}%
\pgfpathrectangle{\pgfqpoint{0.150000in}{0.150000in}}{\pgfqpoint{4.700000in}{3.450000in}}%
\pgfusepath{clip}%
\pgfsetbuttcap%
\pgfsetroundjoin%
\definecolor{currentfill}{rgb}{0.770159,0.582950,0.597779}%
\pgfsetfillcolor{currentfill}%
\pgfsetlinewidth{0.000000pt}%
\definecolor{currentstroke}{rgb}{0.000000,0.000000,0.000000}%
\pgfsetstrokecolor{currentstroke}%
\pgfsetdash{}{0pt}%
\pgfpathmoveto{\pgfqpoint{3.198051in}{1.848552in}}%
\pgfpathlineto{\pgfqpoint{3.259143in}{1.851364in}}%
\pgfpathlineto{\pgfqpoint{3.195817in}{1.914448in}}%
\pgfpathlineto{\pgfqpoint{3.134518in}{1.911850in}}%
\pgfpathclose%
\pgfusepath{fill}%
\end{pgfscope}%
\begin{pgfscope}%
\pgfpathrectangle{\pgfqpoint{0.150000in}{0.150000in}}{\pgfqpoint{4.700000in}{3.450000in}}%
\pgfusepath{clip}%
\pgfsetbuttcap%
\pgfsetroundjoin%
\definecolor{currentfill}{rgb}{0.747365,0.541590,0.557889}%
\pgfsetfillcolor{currentfill}%
\pgfsetlinewidth{0.000000pt}%
\definecolor{currentstroke}{rgb}{0.000000,0.000000,0.000000}%
\pgfsetstrokecolor{currentstroke}%
\pgfsetdash{}{0pt}%
\pgfpathmoveto{\pgfqpoint{3.261601in}{1.785236in}}%
\pgfpathlineto{\pgfqpoint{3.322487in}{1.788263in}}%
\pgfpathlineto{\pgfqpoint{3.259143in}{1.851364in}}%
\pgfpathlineto{\pgfqpoint{3.198051in}{1.848552in}}%
\pgfpathclose%
\pgfusepath{fill}%
\end{pgfscope}%
\begin{pgfscope}%
\pgfpathrectangle{\pgfqpoint{0.150000in}{0.150000in}}{\pgfqpoint{4.700000in}{3.450000in}}%
\pgfusepath{clip}%
\pgfsetbuttcap%
\pgfsetroundjoin%
\definecolor{currentfill}{rgb}{0.724571,0.500230,0.517999}%
\pgfsetfillcolor{currentfill}%
\pgfsetlinewidth{0.000000pt}%
\definecolor{currentstroke}{rgb}{0.000000,0.000000,0.000000}%
\pgfsetstrokecolor{currentstroke}%
\pgfsetdash{}{0pt}%
\pgfpathmoveto{\pgfqpoint{3.325168in}{1.721902in}}%
\pgfpathlineto{\pgfqpoint{3.385848in}{1.725144in}}%
\pgfpathlineto{\pgfqpoint{3.322487in}{1.788263in}}%
\pgfpathlineto{\pgfqpoint{3.261601in}{1.785236in}}%
\pgfpathclose%
\pgfusepath{fill}%
\end{pgfscope}%
\begin{pgfscope}%
\pgfpathrectangle{\pgfqpoint{0.150000in}{0.150000in}}{\pgfqpoint{4.700000in}{3.450000in}}%
\pgfusepath{clip}%
\pgfsetbuttcap%
\pgfsetroundjoin%
\definecolor{currentfill}{rgb}{0.701777,0.458869,0.478110}%
\pgfsetfillcolor{currentfill}%
\pgfsetlinewidth{0.000000pt}%
\definecolor{currentstroke}{rgb}{0.000000,0.000000,0.000000}%
\pgfsetstrokecolor{currentstroke}%
\pgfsetdash{}{0pt}%
\pgfpathmoveto{\pgfqpoint{3.388753in}{1.658551in}}%
\pgfpathlineto{\pgfqpoint{3.449226in}{1.662008in}}%
\pgfpathlineto{\pgfqpoint{3.385848in}{1.725144in}}%
\pgfpathlineto{\pgfqpoint{3.325168in}{1.721902in}}%
\pgfpathclose%
\pgfusepath{fill}%
\end{pgfscope}%
\begin{pgfscope}%
\pgfpathrectangle{\pgfqpoint{0.150000in}{0.150000in}}{\pgfqpoint{4.700000in}{3.450000in}}%
\pgfusepath{clip}%
\pgfsetbuttcap%
\pgfsetroundjoin%
\definecolor{currentfill}{rgb}{0.678983,0.417509,0.438220}%
\pgfsetfillcolor{currentfill}%
\pgfsetlinewidth{0.000000pt}%
\definecolor{currentstroke}{rgb}{0.000000,0.000000,0.000000}%
\pgfsetstrokecolor{currentstroke}%
\pgfsetdash{}{0pt}%
\pgfpathmoveto{\pgfqpoint{3.452356in}{1.595183in}}%
\pgfpathlineto{\pgfqpoint{3.512622in}{1.598854in}}%
\pgfpathlineto{\pgfqpoint{3.449226in}{1.662008in}}%
\pgfpathlineto{\pgfqpoint{3.388753in}{1.658551in}}%
\pgfpathclose%
\pgfusepath{fill}%
\end{pgfscope}%
\begin{pgfscope}%
\pgfpathrectangle{\pgfqpoint{0.150000in}{0.150000in}}{\pgfqpoint{4.700000in}{3.450000in}}%
\pgfusepath{clip}%
\pgfsetbuttcap%
\pgfsetroundjoin%
\definecolor{currentfill}{rgb}{0.603002,0.279642,0.305254}%
\pgfsetfillcolor{currentfill}%
\pgfsetlinewidth{0.000000pt}%
\definecolor{currentstroke}{rgb}{0.000000,0.000000,0.000000}%
\pgfsetstrokecolor{currentstroke}%
\pgfsetdash{}{0pt}%
\pgfpathmoveto{\pgfqpoint{3.770776in}{1.412132in}}%
\pgfpathlineto{\pgfqpoint{3.829546in}{1.409749in}}%
\pgfpathlineto{\pgfqpoint{3.764578in}{1.450241in}}%
\pgfpathlineto{\pgfqpoint{3.705660in}{1.453003in}}%
\pgfpathclose%
\pgfusepath{fill}%
\end{pgfscope}%
\begin{pgfscope}%
\pgfpathrectangle{\pgfqpoint{0.150000in}{0.150000in}}{\pgfqpoint{4.700000in}{3.450000in}}%
\pgfusepath{clip}%
\pgfsetbuttcap%
\pgfsetroundjoin%
\definecolor{currentfill}{rgb}{0.656189,0.376149,0.398330}%
\pgfsetfillcolor{currentfill}%
\pgfsetlinewidth{0.000000pt}%
\definecolor{currentstroke}{rgb}{0.000000,0.000000,0.000000}%
\pgfsetstrokecolor{currentstroke}%
\pgfsetdash{}{0pt}%
\pgfpathmoveto{\pgfqpoint{3.516298in}{1.537958in}}%
\pgfpathlineto{\pgfqpoint{3.576035in}{1.535683in}}%
\pgfpathlineto{\pgfqpoint{3.512622in}{1.598854in}}%
\pgfpathlineto{\pgfqpoint{3.452356in}{1.595183in}}%
\pgfpathclose%
\pgfusepath{fill}%
\end{pgfscope}%
\begin{pgfscope}%
\pgfpathrectangle{\pgfqpoint{0.150000in}{0.150000in}}{\pgfqpoint{4.700000in}{3.450000in}}%
\pgfusepath{clip}%
\pgfsetbuttcap%
\pgfsetroundjoin%
\definecolor{currentfill}{rgb}{0.557414,0.196921,0.225475}%
\pgfsetfillcolor{currentfill}%
\pgfsetlinewidth{0.000000pt}%
\definecolor{currentstroke}{rgb}{0.000000,0.000000,0.000000}%
\pgfsetstrokecolor{currentstroke}%
\pgfsetdash{}{0pt}%
\pgfpathmoveto{\pgfqpoint{4.025527in}{1.287484in}}%
\pgfpathlineto{\pgfqpoint{4.083412in}{1.286141in}}%
\pgfpathlineto{\pgfqpoint{4.018054in}{1.326706in}}%
\pgfpathlineto{\pgfqpoint{3.960016in}{1.328314in}}%
\pgfpathclose%
\pgfusepath{fill}%
\end{pgfscope}%
\begin{pgfscope}%
\pgfpathrectangle{\pgfqpoint{0.150000in}{0.150000in}}{\pgfqpoint{4.700000in}{3.450000in}}%
\pgfusepath{clip}%
\pgfsetbuttcap%
\pgfsetroundjoin%
\definecolor{currentfill}{rgb}{0.640993,0.348575,0.371737}%
\pgfsetfillcolor{currentfill}%
\pgfsetlinewidth{0.000000pt}%
\definecolor{currentstroke}{rgb}{0.000000,0.000000,0.000000}%
\pgfsetstrokecolor{currentstroke}%
\pgfsetdash{}{0pt}%
\pgfpathmoveto{\pgfqpoint{3.581208in}{1.496928in}}%
\pgfpathlineto{\pgfqpoint{3.640723in}{1.493762in}}%
\pgfpathlineto{\pgfqpoint{3.576035in}{1.535683in}}%
\pgfpathlineto{\pgfqpoint{3.516298in}{1.537958in}}%
\pgfpathclose%
\pgfusepath{fill}%
\end{pgfscope}%
\begin{pgfscope}%
\pgfpathrectangle{\pgfqpoint{0.150000in}{0.150000in}}{\pgfqpoint{4.700000in}{3.450000in}}%
\pgfusepath{clip}%
\pgfsetbuttcap%
\pgfsetroundjoin%
\definecolor{currentfill}{rgb}{0.847626,0.866391,0.892662}%
\pgfsetfillcolor{currentfill}%
\pgfsetlinewidth{0.000000pt}%
\definecolor{currentstroke}{rgb}{0.000000,0.000000,0.000000}%
\pgfsetstrokecolor{currentstroke}%
\pgfsetdash{}{0pt}%
\pgfpathmoveto{\pgfqpoint{2.309292in}{2.670039in}}%
\pgfpathlineto{\pgfqpoint{2.373493in}{2.670073in}}%
\pgfpathlineto{\pgfqpoint{2.310188in}{2.733145in}}%
\pgfpathlineto{\pgfqpoint{2.245780in}{2.733326in}}%
\pgfpathclose%
\pgfusepath{fill}%
\end{pgfscope}%
\begin{pgfscope}%
\pgfpathrectangle{\pgfqpoint{0.150000in}{0.150000in}}{\pgfqpoint{4.700000in}{3.450000in}}%
\pgfusepath{clip}%
\pgfsetbuttcap%
\pgfsetroundjoin%
\definecolor{currentfill}{rgb}{0.884942,0.899112,0.918949}%
\pgfsetfillcolor{currentfill}%
\pgfsetlinewidth{0.000000pt}%
\definecolor{currentstroke}{rgb}{0.000000,0.000000,0.000000}%
\pgfsetstrokecolor{currentstroke}%
\pgfsetdash{}{0pt}%
\pgfpathmoveto{\pgfqpoint{2.372821in}{2.606735in}}%
\pgfpathlineto{\pgfqpoint{2.436816in}{2.606983in}}%
\pgfpathlineto{\pgfqpoint{2.373493in}{2.670073in}}%
\pgfpathlineto{\pgfqpoint{2.309292in}{2.670039in}}%
\pgfpathclose%
\pgfusepath{fill}%
\end{pgfscope}%
\begin{pgfscope}%
\pgfpathrectangle{\pgfqpoint{0.150000in}{0.150000in}}{\pgfqpoint{4.700000in}{3.450000in}}%
\pgfusepath{clip}%
\pgfsetbuttcap%
\pgfsetroundjoin%
\definecolor{currentfill}{rgb}{0.922258,0.931832,0.945236}%
\pgfsetfillcolor{currentfill}%
\pgfsetlinewidth{0.000000pt}%
\definecolor{currentstroke}{rgb}{0.000000,0.000000,0.000000}%
\pgfsetstrokecolor{currentstroke}%
\pgfsetdash{}{0pt}%
\pgfpathmoveto{\pgfqpoint{2.436368in}{2.543414in}}%
\pgfpathlineto{\pgfqpoint{2.500156in}{2.543876in}}%
\pgfpathlineto{\pgfqpoint{2.436816in}{2.606983in}}%
\pgfpathlineto{\pgfqpoint{2.372821in}{2.606735in}}%
\pgfpathclose%
\pgfusepath{fill}%
\end{pgfscope}%
\begin{pgfscope}%
\pgfpathrectangle{\pgfqpoint{0.150000in}{0.150000in}}{\pgfqpoint{4.700000in}{3.450000in}}%
\pgfusepath{clip}%
\pgfsetbuttcap%
\pgfsetroundjoin%
\definecolor{currentfill}{rgb}{0.599203,0.272748,0.298606}%
\pgfsetfillcolor{currentfill}%
\pgfsetlinewidth{0.000000pt}%
\definecolor{currentstroke}{rgb}{0.000000,0.000000,0.000000}%
\pgfsetstrokecolor{currentstroke}%
\pgfsetdash{}{0pt}%
\pgfpathmoveto{\pgfqpoint{3.836073in}{1.371147in}}%
\pgfpathlineto{\pgfqpoint{3.894694in}{1.369146in}}%
\pgfpathlineto{\pgfqpoint{3.829546in}{1.409749in}}%
\pgfpathlineto{\pgfqpoint{3.770776in}{1.412132in}}%
\pgfpathclose%
\pgfusepath{fill}%
\end{pgfscope}%
\begin{pgfscope}%
\pgfpathrectangle{\pgfqpoint{0.150000in}{0.150000in}}{\pgfqpoint{4.700000in}{3.450000in}}%
\pgfusepath{clip}%
\pgfsetbuttcap%
\pgfsetroundjoin%
\definecolor{currentfill}{rgb}{0.959574,0.964553,0.971523}%
\pgfsetfillcolor{currentfill}%
\pgfsetlinewidth{0.000000pt}%
\definecolor{currentstroke}{rgb}{0.000000,0.000000,0.000000}%
\pgfsetstrokecolor{currentstroke}%
\pgfsetdash{}{0pt}%
\pgfpathmoveto{\pgfqpoint{2.499932in}{2.480075in}}%
\pgfpathlineto{\pgfqpoint{2.563514in}{2.480752in}}%
\pgfpathlineto{\pgfqpoint{2.500156in}{2.543876in}}%
\pgfpathlineto{\pgfqpoint{2.436368in}{2.543414in}}%
\pgfpathclose%
\pgfusepath{fill}%
\end{pgfscope}%
\begin{pgfscope}%
\pgfpathrectangle{\pgfqpoint{0.150000in}{0.150000in}}{\pgfqpoint{4.700000in}{3.450000in}}%
\pgfusepath{clip}%
\pgfsetbuttcap%
\pgfsetroundjoin%
\definecolor{currentfill}{rgb}{0.996890,0.997273,0.997809}%
\pgfsetfillcolor{currentfill}%
\pgfsetlinewidth{0.000000pt}%
\definecolor{currentstroke}{rgb}{0.000000,0.000000,0.000000}%
\pgfsetstrokecolor{currentstroke}%
\pgfsetdash{}{0pt}%
\pgfpathmoveto{\pgfqpoint{2.563514in}{2.416718in}}%
\pgfpathlineto{\pgfqpoint{2.626889in}{2.417611in}}%
\pgfpathlineto{\pgfqpoint{2.563514in}{2.480752in}}%
\pgfpathlineto{\pgfqpoint{2.499932in}{2.480075in}}%
\pgfpathclose%
\pgfusepath{fill}%
\end{pgfscope}%
\begin{pgfscope}%
\pgfpathrectangle{\pgfqpoint{0.150000in}{0.150000in}}{\pgfqpoint{4.700000in}{3.450000in}}%
\pgfusepath{clip}%
\pgfsetbuttcap%
\pgfsetroundjoin%
\definecolor{currentfill}{rgb}{0.975306,0.955193,0.956786}%
\pgfsetfillcolor{currentfill}%
\pgfsetlinewidth{0.000000pt}%
\definecolor{currentstroke}{rgb}{0.000000,0.000000,0.000000}%
\pgfsetstrokecolor{currentstroke}%
\pgfsetdash{}{0pt}%
\pgfpathmoveto{\pgfqpoint{2.627113in}{2.353344in}}%
\pgfpathlineto{\pgfqpoint{2.690281in}{2.354452in}}%
\pgfpathlineto{\pgfqpoint{2.626889in}{2.417611in}}%
\pgfpathlineto{\pgfqpoint{2.563514in}{2.416718in}}%
\pgfpathclose%
\pgfusepath{fill}%
\end{pgfscope}%
\begin{pgfscope}%
\pgfpathrectangle{\pgfqpoint{0.150000in}{0.150000in}}{\pgfqpoint{4.700000in}{3.450000in}}%
\pgfusepath{clip}%
\pgfsetbuttcap%
\pgfsetroundjoin%
\definecolor{currentfill}{rgb}{0.952512,0.913833,0.916896}%
\pgfsetfillcolor{currentfill}%
\pgfsetlinewidth{0.000000pt}%
\definecolor{currentstroke}{rgb}{0.000000,0.000000,0.000000}%
\pgfsetstrokecolor{currentstroke}%
\pgfsetdash{}{0pt}%
\pgfpathmoveto{\pgfqpoint{2.690730in}{2.289953in}}%
\pgfpathlineto{\pgfqpoint{2.753691in}{2.291275in}}%
\pgfpathlineto{\pgfqpoint{2.690281in}{2.354452in}}%
\pgfpathlineto{\pgfqpoint{2.627113in}{2.353344in}}%
\pgfpathclose%
\pgfusepath{fill}%
\end{pgfscope}%
\begin{pgfscope}%
\pgfpathrectangle{\pgfqpoint{0.150000in}{0.150000in}}{\pgfqpoint{4.700000in}{3.450000in}}%
\pgfusepath{clip}%
\pgfsetbuttcap%
\pgfsetroundjoin%
\definecolor{currentfill}{rgb}{0.929718,0.872472,0.877007}%
\pgfsetfillcolor{currentfill}%
\pgfsetlinewidth{0.000000pt}%
\definecolor{currentstroke}{rgb}{0.000000,0.000000,0.000000}%
\pgfsetstrokecolor{currentstroke}%
\pgfsetdash{}{0pt}%
\pgfpathmoveto{\pgfqpoint{2.754364in}{2.226544in}}%
\pgfpathlineto{\pgfqpoint{2.817118in}{2.228081in}}%
\pgfpathlineto{\pgfqpoint{2.753691in}{2.291275in}}%
\pgfpathlineto{\pgfqpoint{2.690730in}{2.289953in}}%
\pgfpathclose%
\pgfusepath{fill}%
\end{pgfscope}%
\begin{pgfscope}%
\pgfpathrectangle{\pgfqpoint{0.150000in}{0.150000in}}{\pgfqpoint{4.700000in}{3.450000in}}%
\pgfusepath{clip}%
\pgfsetbuttcap%
\pgfsetroundjoin%
\definecolor{currentfill}{rgb}{0.906924,0.831112,0.837117}%
\pgfsetfillcolor{currentfill}%
\pgfsetlinewidth{0.000000pt}%
\definecolor{currentstroke}{rgb}{0.000000,0.000000,0.000000}%
\pgfsetstrokecolor{currentstroke}%
\pgfsetdash{}{0pt}%
\pgfpathmoveto{\pgfqpoint{2.818017in}{2.163117in}}%
\pgfpathlineto{\pgfqpoint{2.880563in}{2.164870in}}%
\pgfpathlineto{\pgfqpoint{2.817118in}{2.228081in}}%
\pgfpathlineto{\pgfqpoint{2.754364in}{2.226544in}}%
\pgfpathclose%
\pgfusepath{fill}%
\end{pgfscope}%
\begin{pgfscope}%
\pgfpathrectangle{\pgfqpoint{0.150000in}{0.150000in}}{\pgfqpoint{4.700000in}{3.450000in}}%
\pgfusepath{clip}%
\pgfsetbuttcap%
\pgfsetroundjoin%
\definecolor{currentfill}{rgb}{0.884130,0.789752,0.797227}%
\pgfsetfillcolor{currentfill}%
\pgfsetlinewidth{0.000000pt}%
\definecolor{currentstroke}{rgb}{0.000000,0.000000,0.000000}%
\pgfsetstrokecolor{currentstroke}%
\pgfsetdash{}{0pt}%
\pgfpathmoveto{\pgfqpoint{2.881686in}{2.099673in}}%
\pgfpathlineto{\pgfqpoint{2.944026in}{2.101641in}}%
\pgfpathlineto{\pgfqpoint{2.880563in}{2.164870in}}%
\pgfpathlineto{\pgfqpoint{2.818017in}{2.163117in}}%
\pgfpathclose%
\pgfusepath{fill}%
\end{pgfscope}%
\begin{pgfscope}%
\pgfpathrectangle{\pgfqpoint{0.150000in}{0.150000in}}{\pgfqpoint{4.700000in}{3.450000in}}%
\pgfusepath{clip}%
\pgfsetbuttcap%
\pgfsetroundjoin%
\definecolor{currentfill}{rgb}{0.637194,0.341682,0.365089}%
\pgfsetfillcolor{currentfill}%
\pgfsetlinewidth{0.000000pt}%
\definecolor{currentstroke}{rgb}{0.000000,0.000000,0.000000}%
\pgfsetstrokecolor{currentstroke}%
\pgfsetdash{}{0pt}%
\pgfpathmoveto{\pgfqpoint{3.646296in}{1.455787in}}%
\pgfpathlineto{\pgfqpoint{3.705660in}{1.453003in}}%
\pgfpathlineto{\pgfqpoint{3.640723in}{1.493762in}}%
\pgfpathlineto{\pgfqpoint{3.581208in}{1.496928in}}%
\pgfpathclose%
\pgfusepath{fill}%
\end{pgfscope}%
\begin{pgfscope}%
\pgfpathrectangle{\pgfqpoint{0.150000in}{0.150000in}}{\pgfqpoint{4.700000in}{3.450000in}}%
\pgfusepath{clip}%
\pgfsetbuttcap%
\pgfsetroundjoin%
\definecolor{currentfill}{rgb}{0.861336,0.748392,0.757338}%
\pgfsetfillcolor{currentfill}%
\pgfsetlinewidth{0.000000pt}%
\definecolor{currentstroke}{rgb}{0.000000,0.000000,0.000000}%
\pgfsetstrokecolor{currentstroke}%
\pgfsetdash{}{0pt}%
\pgfpathmoveto{\pgfqpoint{2.945374in}{2.036211in}}%
\pgfpathlineto{\pgfqpoint{3.007506in}{2.038395in}}%
\pgfpathlineto{\pgfqpoint{2.944026in}{2.101641in}}%
\pgfpathlineto{\pgfqpoint{2.881686in}{2.099673in}}%
\pgfpathclose%
\pgfusepath{fill}%
\end{pgfscope}%
\begin{pgfscope}%
\pgfpathrectangle{\pgfqpoint{0.150000in}{0.150000in}}{\pgfqpoint{4.700000in}{3.450000in}}%
\pgfusepath{clip}%
\pgfsetbuttcap%
\pgfsetroundjoin%
\definecolor{currentfill}{rgb}{0.838542,0.707031,0.717448}%
\pgfsetfillcolor{currentfill}%
\pgfsetlinewidth{0.000000pt}%
\definecolor{currentstroke}{rgb}{0.000000,0.000000,0.000000}%
\pgfsetstrokecolor{currentstroke}%
\pgfsetdash{}{0pt}%
\pgfpathmoveto{\pgfqpoint{3.009079in}{1.972732in}}%
\pgfpathlineto{\pgfqpoint{3.071003in}{1.975131in}}%
\pgfpathlineto{\pgfqpoint{3.007506in}{2.038395in}}%
\pgfpathlineto{\pgfqpoint{2.945374in}{2.036211in}}%
\pgfpathclose%
\pgfusepath{fill}%
\end{pgfscope}%
\begin{pgfscope}%
\pgfpathrectangle{\pgfqpoint{0.150000in}{0.150000in}}{\pgfqpoint{4.700000in}{3.450000in}}%
\pgfusepath{clip}%
\pgfsetbuttcap%
\pgfsetroundjoin%
\definecolor{currentfill}{rgb}{0.815748,0.665671,0.677558}%
\pgfsetfillcolor{currentfill}%
\pgfsetlinewidth{0.000000pt}%
\definecolor{currentstroke}{rgb}{0.000000,0.000000,0.000000}%
\pgfsetstrokecolor{currentstroke}%
\pgfsetdash{}{0pt}%
\pgfpathmoveto{\pgfqpoint{3.072802in}{1.909235in}}%
\pgfpathlineto{\pgfqpoint{3.134518in}{1.911850in}}%
\pgfpathlineto{\pgfqpoint{3.071003in}{1.975131in}}%
\pgfpathlineto{\pgfqpoint{3.009079in}{1.972732in}}%
\pgfpathclose%
\pgfusepath{fill}%
\end{pgfscope}%
\begin{pgfscope}%
\pgfpathrectangle{\pgfqpoint{0.150000in}{0.150000in}}{\pgfqpoint{4.700000in}{3.450000in}}%
\pgfusepath{clip}%
\pgfsetbuttcap%
\pgfsetroundjoin%
\definecolor{currentfill}{rgb}{0.591605,0.258961,0.285309}%
\pgfsetfillcolor{currentfill}%
\pgfsetlinewidth{0.000000pt}%
\definecolor{currentstroke}{rgb}{0.000000,0.000000,0.000000}%
\pgfsetstrokecolor{currentstroke}%
\pgfsetdash{}{0pt}%
\pgfpathmoveto{\pgfqpoint{3.901550in}{1.330049in}}%
\pgfpathlineto{\pgfqpoint{3.960016in}{1.328314in}}%
\pgfpathlineto{\pgfqpoint{3.894694in}{1.369146in}}%
\pgfpathlineto{\pgfqpoint{3.836073in}{1.371147in}}%
\pgfpathclose%
\pgfusepath{fill}%
\end{pgfscope}%
\begin{pgfscope}%
\pgfpathrectangle{\pgfqpoint{0.150000in}{0.150000in}}{\pgfqpoint{4.700000in}{3.450000in}}%
\pgfusepath{clip}%
\pgfsetbuttcap%
\pgfsetroundjoin%
\definecolor{currentfill}{rgb}{0.792953,0.624311,0.637669}%
\pgfsetfillcolor{currentfill}%
\pgfsetlinewidth{0.000000pt}%
\definecolor{currentstroke}{rgb}{0.000000,0.000000,0.000000}%
\pgfsetstrokecolor{currentstroke}%
\pgfsetdash{}{0pt}%
\pgfpathmoveto{\pgfqpoint{3.136542in}{1.845721in}}%
\pgfpathlineto{\pgfqpoint{3.198051in}{1.848552in}}%
\pgfpathlineto{\pgfqpoint{3.134518in}{1.911850in}}%
\pgfpathlineto{\pgfqpoint{3.072802in}{1.909235in}}%
\pgfpathclose%
\pgfusepath{fill}%
\end{pgfscope}%
\begin{pgfscope}%
\pgfpathrectangle{\pgfqpoint{0.150000in}{0.150000in}}{\pgfqpoint{4.700000in}{3.450000in}}%
\pgfusepath{clip}%
\pgfsetbuttcap%
\pgfsetroundjoin%
\definecolor{currentfill}{rgb}{0.770159,0.582950,0.597779}%
\pgfsetfillcolor{currentfill}%
\pgfsetlinewidth{0.000000pt}%
\definecolor{currentstroke}{rgb}{0.000000,0.000000,0.000000}%
\pgfsetstrokecolor{currentstroke}%
\pgfsetdash{}{0pt}%
\pgfpathmoveto{\pgfqpoint{3.200300in}{1.782188in}}%
\pgfpathlineto{\pgfqpoint{3.261601in}{1.785236in}}%
\pgfpathlineto{\pgfqpoint{3.198051in}{1.848552in}}%
\pgfpathlineto{\pgfqpoint{3.136542in}{1.845721in}}%
\pgfpathclose%
\pgfusepath{fill}%
\end{pgfscope}%
\begin{pgfscope}%
\pgfpathrectangle{\pgfqpoint{0.150000in}{0.150000in}}{\pgfqpoint{4.700000in}{3.450000in}}%
\pgfusepath{clip}%
\pgfsetbuttcap%
\pgfsetroundjoin%
\definecolor{currentfill}{rgb}{0.747365,0.541590,0.557889}%
\pgfsetfillcolor{currentfill}%
\pgfsetlinewidth{0.000000pt}%
\definecolor{currentstroke}{rgb}{0.000000,0.000000,0.000000}%
\pgfsetstrokecolor{currentstroke}%
\pgfsetdash{}{0pt}%
\pgfpathmoveto{\pgfqpoint{3.264076in}{1.718639in}}%
\pgfpathlineto{\pgfqpoint{3.325168in}{1.721902in}}%
\pgfpathlineto{\pgfqpoint{3.261601in}{1.785236in}}%
\pgfpathlineto{\pgfqpoint{3.200300in}{1.782188in}}%
\pgfpathclose%
\pgfusepath{fill}%
\end{pgfscope}%
\begin{pgfscope}%
\pgfpathrectangle{\pgfqpoint{0.150000in}{0.150000in}}{\pgfqpoint{4.700000in}{3.450000in}}%
\pgfusepath{clip}%
\pgfsetbuttcap%
\pgfsetroundjoin%
\definecolor{currentfill}{rgb}{0.724571,0.500230,0.517999}%
\pgfsetfillcolor{currentfill}%
\pgfsetlinewidth{0.000000pt}%
\definecolor{currentstroke}{rgb}{0.000000,0.000000,0.000000}%
\pgfsetstrokecolor{currentstroke}%
\pgfsetdash{}{0pt}%
\pgfpathmoveto{\pgfqpoint{3.327869in}{1.655072in}}%
\pgfpathlineto{\pgfqpoint{3.388753in}{1.658551in}}%
\pgfpathlineto{\pgfqpoint{3.325168in}{1.721902in}}%
\pgfpathlineto{\pgfqpoint{3.264076in}{1.718639in}}%
\pgfpathclose%
\pgfusepath{fill}%
\end{pgfscope}%
\begin{pgfscope}%
\pgfpathrectangle{\pgfqpoint{0.150000in}{0.150000in}}{\pgfqpoint{4.700000in}{3.450000in}}%
\pgfusepath{clip}%
\pgfsetbuttcap%
\pgfsetroundjoin%
\definecolor{currentfill}{rgb}{0.701777,0.458869,0.478110}%
\pgfsetfillcolor{currentfill}%
\pgfsetlinewidth{0.000000pt}%
\definecolor{currentstroke}{rgb}{0.000000,0.000000,0.000000}%
\pgfsetstrokecolor{currentstroke}%
\pgfsetdash{}{0pt}%
\pgfpathmoveto{\pgfqpoint{3.391680in}{1.591487in}}%
\pgfpathlineto{\pgfqpoint{3.452356in}{1.595183in}}%
\pgfpathlineto{\pgfqpoint{3.388753in}{1.658551in}}%
\pgfpathlineto{\pgfqpoint{3.327869in}{1.655072in}}%
\pgfpathclose%
\pgfusepath{fill}%
\end{pgfscope}%
\begin{pgfscope}%
\pgfpathrectangle{\pgfqpoint{0.150000in}{0.150000in}}{\pgfqpoint{4.700000in}{3.450000in}}%
\pgfusepath{clip}%
\pgfsetbuttcap%
\pgfsetroundjoin%
\definecolor{currentfill}{rgb}{0.629596,0.327895,0.351792}%
\pgfsetfillcolor{currentfill}%
\pgfsetlinewidth{0.000000pt}%
\definecolor{currentstroke}{rgb}{0.000000,0.000000,0.000000}%
\pgfsetstrokecolor{currentstroke}%
\pgfsetdash{}{0pt}%
\pgfpathmoveto{\pgfqpoint{3.711562in}{1.414532in}}%
\pgfpathlineto{\pgfqpoint{3.770776in}{1.412132in}}%
\pgfpathlineto{\pgfqpoint{3.705660in}{1.453003in}}%
\pgfpathlineto{\pgfqpoint{3.646296in}{1.455787in}}%
\pgfpathclose%
\pgfusepath{fill}%
\end{pgfscope}%
\begin{pgfscope}%
\pgfpathrectangle{\pgfqpoint{0.150000in}{0.150000in}}{\pgfqpoint{4.700000in}{3.450000in}}%
\pgfusepath{clip}%
\pgfsetbuttcap%
\pgfsetroundjoin%
\definecolor{currentfill}{rgb}{0.682782,0.424403,0.444868}%
\pgfsetfillcolor{currentfill}%
\pgfsetlinewidth{0.000000pt}%
\definecolor{currentstroke}{rgb}{0.000000,0.000000,0.000000}%
\pgfsetstrokecolor{currentstroke}%
\pgfsetdash{}{0pt}%
\pgfpathmoveto{\pgfqpoint{3.456170in}{1.541417in}}%
\pgfpathlineto{\pgfqpoint{3.516298in}{1.537958in}}%
\pgfpathlineto{\pgfqpoint{3.452356in}{1.595183in}}%
\pgfpathlineto{\pgfqpoint{3.391680in}{1.591487in}}%
\pgfpathclose%
\pgfusepath{fill}%
\end{pgfscope}%
\begin{pgfscope}%
\pgfpathrectangle{\pgfqpoint{0.150000in}{0.150000in}}{\pgfqpoint{4.700000in}{3.450000in}}%
\pgfusepath{clip}%
\pgfsetbuttcap%
\pgfsetroundjoin%
\definecolor{currentfill}{rgb}{0.587806,0.252068,0.278661}%
\pgfsetfillcolor{currentfill}%
\pgfsetlinewidth{0.000000pt}%
\definecolor{currentstroke}{rgb}{0.000000,0.000000,0.000000}%
\pgfsetstrokecolor{currentstroke}%
\pgfsetdash{}{0pt}%
\pgfpathmoveto{\pgfqpoint{3.967199in}{1.288722in}}%
\pgfpathlineto{\pgfqpoint{4.025527in}{1.287484in}}%
\pgfpathlineto{\pgfqpoint{3.960016in}{1.328314in}}%
\pgfpathlineto{\pgfqpoint{3.901550in}{1.330049in}}%
\pgfpathclose%
\pgfusepath{fill}%
\end{pgfscope}%
\begin{pgfscope}%
\pgfpathrectangle{\pgfqpoint{0.150000in}{0.150000in}}{\pgfqpoint{4.700000in}{3.450000in}}%
\pgfusepath{clip}%
\pgfsetbuttcap%
\pgfsetroundjoin%
\definecolor{currentfill}{rgb}{0.810309,0.833670,0.866376}%
\pgfsetfillcolor{currentfill}%
\pgfsetlinewidth{0.000000pt}%
\definecolor{currentstroke}{rgb}{0.000000,0.000000,0.000000}%
\pgfsetstrokecolor{currentstroke}%
\pgfsetdash{}{0pt}%
\pgfpathmoveto{\pgfqpoint{2.244652in}{2.670006in}}%
\pgfpathlineto{\pgfqpoint{2.309292in}{2.670039in}}%
\pgfpathlineto{\pgfqpoint{2.245780in}{2.733326in}}%
\pgfpathlineto{\pgfqpoint{2.180933in}{2.733508in}}%
\pgfpathclose%
\pgfusepath{fill}%
\end{pgfscope}%
\begin{pgfscope}%
\pgfpathrectangle{\pgfqpoint{0.150000in}{0.150000in}}{\pgfqpoint{4.700000in}{3.450000in}}%
\pgfusepath{clip}%
\pgfsetbuttcap%
\pgfsetroundjoin%
\definecolor{currentfill}{rgb}{0.847626,0.866391,0.892662}%
\pgfsetfillcolor{currentfill}%
\pgfsetlinewidth{0.000000pt}%
\definecolor{currentstroke}{rgb}{0.000000,0.000000,0.000000}%
\pgfsetstrokecolor{currentstroke}%
\pgfsetdash{}{0pt}%
\pgfpathmoveto{\pgfqpoint{2.308389in}{2.606486in}}%
\pgfpathlineto{\pgfqpoint{2.372821in}{2.606735in}}%
\pgfpathlineto{\pgfqpoint{2.309292in}{2.670039in}}%
\pgfpathlineto{\pgfqpoint{2.244652in}{2.670006in}}%
\pgfpathclose%
\pgfusepath{fill}%
\end{pgfscope}%
\begin{pgfscope}%
\pgfpathrectangle{\pgfqpoint{0.150000in}{0.150000in}}{\pgfqpoint{4.700000in}{3.450000in}}%
\pgfusepath{clip}%
\pgfsetbuttcap%
\pgfsetroundjoin%
\definecolor{currentfill}{rgb}{0.884942,0.899112,0.918949}%
\pgfsetfillcolor{currentfill}%
\pgfsetlinewidth{0.000000pt}%
\definecolor{currentstroke}{rgb}{0.000000,0.000000,0.000000}%
\pgfsetstrokecolor{currentstroke}%
\pgfsetdash{}{0pt}%
\pgfpathmoveto{\pgfqpoint{2.372144in}{2.542948in}}%
\pgfpathlineto{\pgfqpoint{2.436368in}{2.543414in}}%
\pgfpathlineto{\pgfqpoint{2.372821in}{2.606735in}}%
\pgfpathlineto{\pgfqpoint{2.308389in}{2.606486in}}%
\pgfpathclose%
\pgfusepath{fill}%
\end{pgfscope}%
\begin{pgfscope}%
\pgfpathrectangle{\pgfqpoint{0.150000in}{0.150000in}}{\pgfqpoint{4.700000in}{3.450000in}}%
\pgfusepath{clip}%
\pgfsetbuttcap%
\pgfsetroundjoin%
\definecolor{currentfill}{rgb}{0.671385,0.403722,0.424923}%
\pgfsetfillcolor{currentfill}%
\pgfsetlinewidth{0.000000pt}%
\definecolor{currentstroke}{rgb}{0.000000,0.000000,0.000000}%
\pgfsetstrokecolor{currentstroke}%
\pgfsetdash{}{0pt}%
\pgfpathmoveto{\pgfqpoint{3.521233in}{1.500002in}}%
\pgfpathlineto{\pgfqpoint{3.581208in}{1.496928in}}%
\pgfpathlineto{\pgfqpoint{3.516298in}{1.537958in}}%
\pgfpathlineto{\pgfqpoint{3.456170in}{1.541417in}}%
\pgfpathclose%
\pgfusepath{fill}%
\end{pgfscope}%
\begin{pgfscope}%
\pgfpathrectangle{\pgfqpoint{0.150000in}{0.150000in}}{\pgfqpoint{4.700000in}{3.450000in}}%
\pgfusepath{clip}%
\pgfsetbuttcap%
\pgfsetroundjoin%
\definecolor{currentfill}{rgb}{0.922258,0.931832,0.945236}%
\pgfsetfillcolor{currentfill}%
\pgfsetlinewidth{0.000000pt}%
\definecolor{currentstroke}{rgb}{0.000000,0.000000,0.000000}%
\pgfsetstrokecolor{currentstroke}%
\pgfsetdash{}{0pt}%
\pgfpathmoveto{\pgfqpoint{2.435916in}{2.479393in}}%
\pgfpathlineto{\pgfqpoint{2.499932in}{2.480075in}}%
\pgfpathlineto{\pgfqpoint{2.436368in}{2.543414in}}%
\pgfpathlineto{\pgfqpoint{2.372144in}{2.542948in}}%
\pgfpathclose%
\pgfusepath{fill}%
\end{pgfscope}%
\begin{pgfscope}%
\pgfpathrectangle{\pgfqpoint{0.150000in}{0.150000in}}{\pgfqpoint{4.700000in}{3.450000in}}%
\pgfusepath{clip}%
\pgfsetbuttcap%
\pgfsetroundjoin%
\definecolor{currentfill}{rgb}{0.959574,0.964553,0.971523}%
\pgfsetfillcolor{currentfill}%
\pgfsetlinewidth{0.000000pt}%
\definecolor{currentstroke}{rgb}{0.000000,0.000000,0.000000}%
\pgfsetstrokecolor{currentstroke}%
\pgfsetdash{}{0pt}%
\pgfpathmoveto{\pgfqpoint{2.499706in}{2.415820in}}%
\pgfpathlineto{\pgfqpoint{2.563514in}{2.416718in}}%
\pgfpathlineto{\pgfqpoint{2.499932in}{2.480075in}}%
\pgfpathlineto{\pgfqpoint{2.435916in}{2.479393in}}%
\pgfpathclose%
\pgfusepath{fill}%
\end{pgfscope}%
\begin{pgfscope}%
\pgfpathrectangle{\pgfqpoint{0.150000in}{0.150000in}}{\pgfqpoint{4.700000in}{3.450000in}}%
\pgfusepath{clip}%
\pgfsetbuttcap%
\pgfsetroundjoin%
\definecolor{currentfill}{rgb}{0.996890,0.997273,0.997809}%
\pgfsetfillcolor{currentfill}%
\pgfsetlinewidth{0.000000pt}%
\definecolor{currentstroke}{rgb}{0.000000,0.000000,0.000000}%
\pgfsetstrokecolor{currentstroke}%
\pgfsetdash{}{0pt}%
\pgfpathmoveto{\pgfqpoint{2.563514in}{2.352230in}}%
\pgfpathlineto{\pgfqpoint{2.627113in}{2.353344in}}%
\pgfpathlineto{\pgfqpoint{2.563514in}{2.416718in}}%
\pgfpathlineto{\pgfqpoint{2.499706in}{2.415820in}}%
\pgfpathclose%
\pgfusepath{fill}%
\end{pgfscope}%
\begin{pgfscope}%
\pgfpathrectangle{\pgfqpoint{0.150000in}{0.150000in}}{\pgfqpoint{4.700000in}{3.450000in}}%
\pgfusepath{clip}%
\pgfsetbuttcap%
\pgfsetroundjoin%
\definecolor{currentfill}{rgb}{0.625797,0.321002,0.345144}%
\pgfsetfillcolor{currentfill}%
\pgfsetlinewidth{0.000000pt}%
\definecolor{currentstroke}{rgb}{0.000000,0.000000,0.000000}%
\pgfsetstrokecolor{currentstroke}%
\pgfsetdash{}{0pt}%
\pgfpathmoveto{\pgfqpoint{3.777008in}{1.373164in}}%
\pgfpathlineto{\pgfqpoint{3.836073in}{1.371147in}}%
\pgfpathlineto{\pgfqpoint{3.770776in}{1.412132in}}%
\pgfpathlineto{\pgfqpoint{3.711562in}{1.414532in}}%
\pgfpathclose%
\pgfusepath{fill}%
\end{pgfscope}%
\begin{pgfscope}%
\pgfpathrectangle{\pgfqpoint{0.150000in}{0.150000in}}{\pgfqpoint{4.700000in}{3.450000in}}%
\pgfusepath{clip}%
\pgfsetbuttcap%
\pgfsetroundjoin%
\definecolor{currentfill}{rgb}{0.975306,0.955193,0.956786}%
\pgfsetfillcolor{currentfill}%
\pgfsetlinewidth{0.000000pt}%
\definecolor{currentstroke}{rgb}{0.000000,0.000000,0.000000}%
\pgfsetstrokecolor{currentstroke}%
\pgfsetdash{}{0pt}%
\pgfpathmoveto{\pgfqpoint{2.627339in}{2.288622in}}%
\pgfpathlineto{\pgfqpoint{2.690730in}{2.289953in}}%
\pgfpathlineto{\pgfqpoint{2.627113in}{2.353344in}}%
\pgfpathlineto{\pgfqpoint{2.563514in}{2.352230in}}%
\pgfpathclose%
\pgfusepath{fill}%
\end{pgfscope}%
\begin{pgfscope}%
\pgfpathrectangle{\pgfqpoint{0.150000in}{0.150000in}}{\pgfqpoint{4.700000in}{3.450000in}}%
\pgfusepath{clip}%
\pgfsetbuttcap%
\pgfsetroundjoin%
\definecolor{currentfill}{rgb}{0.952512,0.913833,0.916896}%
\pgfsetfillcolor{currentfill}%
\pgfsetlinewidth{0.000000pt}%
\definecolor{currentstroke}{rgb}{0.000000,0.000000,0.000000}%
\pgfsetstrokecolor{currentstroke}%
\pgfsetdash{}{0pt}%
\pgfpathmoveto{\pgfqpoint{2.691182in}{2.224996in}}%
\pgfpathlineto{\pgfqpoint{2.754364in}{2.226544in}}%
\pgfpathlineto{\pgfqpoint{2.690730in}{2.289953in}}%
\pgfpathlineto{\pgfqpoint{2.627339in}{2.288622in}}%
\pgfpathclose%
\pgfusepath{fill}%
\end{pgfscope}%
\begin{pgfscope}%
\pgfpathrectangle{\pgfqpoint{0.150000in}{0.150000in}}{\pgfqpoint{4.700000in}{3.450000in}}%
\pgfusepath{clip}%
\pgfsetbuttcap%
\pgfsetroundjoin%
\definecolor{currentfill}{rgb}{0.929718,0.872472,0.877007}%
\pgfsetfillcolor{currentfill}%
\pgfsetlinewidth{0.000000pt}%
\definecolor{currentstroke}{rgb}{0.000000,0.000000,0.000000}%
\pgfsetstrokecolor{currentstroke}%
\pgfsetdash{}{0pt}%
\pgfpathmoveto{\pgfqpoint{2.755043in}{2.161352in}}%
\pgfpathlineto{\pgfqpoint{2.818017in}{2.163117in}}%
\pgfpathlineto{\pgfqpoint{2.754364in}{2.226544in}}%
\pgfpathlineto{\pgfqpoint{2.691182in}{2.224996in}}%
\pgfpathclose%
\pgfusepath{fill}%
\end{pgfscope}%
\begin{pgfscope}%
\pgfpathrectangle{\pgfqpoint{0.150000in}{0.150000in}}{\pgfqpoint{4.700000in}{3.450000in}}%
\pgfusepath{clip}%
\pgfsetbuttcap%
\pgfsetroundjoin%
\definecolor{currentfill}{rgb}{0.906924,0.831112,0.837117}%
\pgfsetfillcolor{currentfill}%
\pgfsetlinewidth{0.000000pt}%
\definecolor{currentstroke}{rgb}{0.000000,0.000000,0.000000}%
\pgfsetstrokecolor{currentstroke}%
\pgfsetdash{}{0pt}%
\pgfpathmoveto{\pgfqpoint{2.818921in}{2.097691in}}%
\pgfpathlineto{\pgfqpoint{2.881686in}{2.099673in}}%
\pgfpathlineto{\pgfqpoint{2.818017in}{2.163117in}}%
\pgfpathlineto{\pgfqpoint{2.755043in}{2.161352in}}%
\pgfpathclose%
\pgfusepath{fill}%
\end{pgfscope}%
\begin{pgfscope}%
\pgfpathrectangle{\pgfqpoint{0.150000in}{0.150000in}}{\pgfqpoint{4.700000in}{3.450000in}}%
\pgfusepath{clip}%
\pgfsetbuttcap%
\pgfsetroundjoin%
\definecolor{currentfill}{rgb}{0.884130,0.789752,0.797227}%
\pgfsetfillcolor{currentfill}%
\pgfsetlinewidth{0.000000pt}%
\definecolor{currentstroke}{rgb}{0.000000,0.000000,0.000000}%
\pgfsetstrokecolor{currentstroke}%
\pgfsetdash{}{0pt}%
\pgfpathmoveto{\pgfqpoint{2.882817in}{2.034013in}}%
\pgfpathlineto{\pgfqpoint{2.945374in}{2.036211in}}%
\pgfpathlineto{\pgfqpoint{2.881686in}{2.099673in}}%
\pgfpathlineto{\pgfqpoint{2.818921in}{2.097691in}}%
\pgfpathclose%
\pgfusepath{fill}%
\end{pgfscope}%
\begin{pgfscope}%
\pgfpathrectangle{\pgfqpoint{0.150000in}{0.150000in}}{\pgfqpoint{4.700000in}{3.450000in}}%
\pgfusepath{clip}%
\pgfsetbuttcap%
\pgfsetroundjoin%
\definecolor{currentfill}{rgb}{0.861336,0.748392,0.757338}%
\pgfsetfillcolor{currentfill}%
\pgfsetlinewidth{0.000000pt}%
\definecolor{currentstroke}{rgb}{0.000000,0.000000,0.000000}%
\pgfsetstrokecolor{currentstroke}%
\pgfsetdash{}{0pt}%
\pgfpathmoveto{\pgfqpoint{2.946731in}{1.970316in}}%
\pgfpathlineto{\pgfqpoint{3.009079in}{1.972732in}}%
\pgfpathlineto{\pgfqpoint{2.945374in}{2.036211in}}%
\pgfpathlineto{\pgfqpoint{2.882817in}{2.034013in}}%
\pgfpathclose%
\pgfusepath{fill}%
\end{pgfscope}%
\begin{pgfscope}%
\pgfpathrectangle{\pgfqpoint{0.150000in}{0.150000in}}{\pgfqpoint{4.700000in}{3.450000in}}%
\pgfusepath{clip}%
\pgfsetbuttcap%
\pgfsetroundjoin%
\definecolor{currentfill}{rgb}{0.838542,0.707031,0.717448}%
\pgfsetfillcolor{currentfill}%
\pgfsetlinewidth{0.000000pt}%
\definecolor{currentstroke}{rgb}{0.000000,0.000000,0.000000}%
\pgfsetstrokecolor{currentstroke}%
\pgfsetdash{}{0pt}%
\pgfpathmoveto{\pgfqpoint{3.010663in}{1.906602in}}%
\pgfpathlineto{\pgfqpoint{3.072802in}{1.909235in}}%
\pgfpathlineto{\pgfqpoint{3.009079in}{1.972732in}}%
\pgfpathlineto{\pgfqpoint{2.946731in}{1.970316in}}%
\pgfpathclose%
\pgfusepath{fill}%
\end{pgfscope}%
\begin{pgfscope}%
\pgfpathrectangle{\pgfqpoint{0.150000in}{0.150000in}}{\pgfqpoint{4.700000in}{3.450000in}}%
\pgfusepath{clip}%
\pgfsetbuttcap%
\pgfsetroundjoin%
\definecolor{currentfill}{rgb}{0.663787,0.389936,0.411627}%
\pgfsetfillcolor{currentfill}%
\pgfsetlinewidth{0.000000pt}%
\definecolor{currentstroke}{rgb}{0.000000,0.000000,0.000000}%
\pgfsetstrokecolor{currentstroke}%
\pgfsetdash{}{0pt}%
\pgfpathmoveto{\pgfqpoint{3.586480in}{1.458591in}}%
\pgfpathlineto{\pgfqpoint{3.646296in}{1.455787in}}%
\pgfpathlineto{\pgfqpoint{3.581208in}{1.496928in}}%
\pgfpathlineto{\pgfqpoint{3.521233in}{1.500002in}}%
\pgfpathclose%
\pgfusepath{fill}%
\end{pgfscope}%
\begin{pgfscope}%
\pgfpathrectangle{\pgfqpoint{0.150000in}{0.150000in}}{\pgfqpoint{4.700000in}{3.450000in}}%
\pgfusepath{clip}%
\pgfsetbuttcap%
\pgfsetroundjoin%
\definecolor{currentfill}{rgb}{0.815748,0.665671,0.677558}%
\pgfsetfillcolor{currentfill}%
\pgfsetlinewidth{0.000000pt}%
\definecolor{currentstroke}{rgb}{0.000000,0.000000,0.000000}%
\pgfsetstrokecolor{currentstroke}%
\pgfsetdash{}{0pt}%
\pgfpathmoveto{\pgfqpoint{3.074613in}{1.842870in}}%
\pgfpathlineto{\pgfqpoint{3.136542in}{1.845721in}}%
\pgfpathlineto{\pgfqpoint{3.072802in}{1.909235in}}%
\pgfpathlineto{\pgfqpoint{3.010663in}{1.906602in}}%
\pgfpathclose%
\pgfusepath{fill}%
\end{pgfscope}%
\begin{pgfscope}%
\pgfpathrectangle{\pgfqpoint{0.150000in}{0.150000in}}{\pgfqpoint{4.700000in}{3.450000in}}%
\pgfusepath{clip}%
\pgfsetbuttcap%
\pgfsetroundjoin%
\definecolor{currentfill}{rgb}{0.792953,0.624311,0.637669}%
\pgfsetfillcolor{currentfill}%
\pgfsetlinewidth{0.000000pt}%
\definecolor{currentstroke}{rgb}{0.000000,0.000000,0.000000}%
\pgfsetstrokecolor{currentstroke}%
\pgfsetdash{}{0pt}%
\pgfpathmoveto{\pgfqpoint{3.138580in}{1.779120in}}%
\pgfpathlineto{\pgfqpoint{3.200300in}{1.782188in}}%
\pgfpathlineto{\pgfqpoint{3.136542in}{1.845721in}}%
\pgfpathlineto{\pgfqpoint{3.074613in}{1.842870in}}%
\pgfpathclose%
\pgfusepath{fill}%
\end{pgfscope}%
\begin{pgfscope}%
\pgfpathrectangle{\pgfqpoint{0.150000in}{0.150000in}}{\pgfqpoint{4.700000in}{3.450000in}}%
\pgfusepath{clip}%
\pgfsetbuttcap%
\pgfsetroundjoin%
\definecolor{currentfill}{rgb}{0.770159,0.582950,0.597779}%
\pgfsetfillcolor{currentfill}%
\pgfsetlinewidth{0.000000pt}%
\definecolor{currentstroke}{rgb}{0.000000,0.000000,0.000000}%
\pgfsetstrokecolor{currentstroke}%
\pgfsetdash{}{0pt}%
\pgfpathmoveto{\pgfqpoint{3.202565in}{1.715353in}}%
\pgfpathlineto{\pgfqpoint{3.264076in}{1.718639in}}%
\pgfpathlineto{\pgfqpoint{3.200300in}{1.782188in}}%
\pgfpathlineto{\pgfqpoint{3.138580in}{1.779120in}}%
\pgfpathclose%
\pgfusepath{fill}%
\end{pgfscope}%
\begin{pgfscope}%
\pgfpathrectangle{\pgfqpoint{0.150000in}{0.150000in}}{\pgfqpoint{4.700000in}{3.450000in}}%
\pgfusepath{clip}%
\pgfsetbuttcap%
\pgfsetroundjoin%
\definecolor{currentfill}{rgb}{0.621998,0.314108,0.338496}%
\pgfsetfillcolor{currentfill}%
\pgfsetlinewidth{0.000000pt}%
\definecolor{currentstroke}{rgb}{0.000000,0.000000,0.000000}%
\pgfsetstrokecolor{currentstroke}%
\pgfsetdash{}{0pt}%
\pgfpathmoveto{\pgfqpoint{3.842634in}{1.331682in}}%
\pgfpathlineto{\pgfqpoint{3.901550in}{1.330049in}}%
\pgfpathlineto{\pgfqpoint{3.836073in}{1.371147in}}%
\pgfpathlineto{\pgfqpoint{3.777008in}{1.373164in}}%
\pgfpathclose%
\pgfusepath{fill}%
\end{pgfscope}%
\begin{pgfscope}%
\pgfpathrectangle{\pgfqpoint{0.150000in}{0.150000in}}{\pgfqpoint{4.700000in}{3.450000in}}%
\pgfusepath{clip}%
\pgfsetbuttcap%
\pgfsetroundjoin%
\definecolor{currentfill}{rgb}{0.747365,0.541590,0.557889}%
\pgfsetfillcolor{currentfill}%
\pgfsetlinewidth{0.000000pt}%
\definecolor{currentstroke}{rgb}{0.000000,0.000000,0.000000}%
\pgfsetstrokecolor{currentstroke}%
\pgfsetdash{}{0pt}%
\pgfpathmoveto{\pgfqpoint{3.266568in}{1.651568in}}%
\pgfpathlineto{\pgfqpoint{3.327869in}{1.655072in}}%
\pgfpathlineto{\pgfqpoint{3.264076in}{1.718639in}}%
\pgfpathlineto{\pgfqpoint{3.202565in}{1.715353in}}%
\pgfpathclose%
\pgfusepath{fill}%
\end{pgfscope}%
\begin{pgfscope}%
\pgfpathrectangle{\pgfqpoint{0.150000in}{0.150000in}}{\pgfqpoint{4.700000in}{3.450000in}}%
\pgfusepath{clip}%
\pgfsetbuttcap%
\pgfsetroundjoin%
\definecolor{currentfill}{rgb}{0.724571,0.500230,0.517999}%
\pgfsetfillcolor{currentfill}%
\pgfsetlinewidth{0.000000pt}%
\definecolor{currentstroke}{rgb}{0.000000,0.000000,0.000000}%
\pgfsetstrokecolor{currentstroke}%
\pgfsetdash{}{0pt}%
\pgfpathmoveto{\pgfqpoint{3.330589in}{1.587765in}}%
\pgfpathlineto{\pgfqpoint{3.391680in}{1.591487in}}%
\pgfpathlineto{\pgfqpoint{3.327869in}{1.655072in}}%
\pgfpathlineto{\pgfqpoint{3.266568in}{1.651568in}}%
\pgfpathclose%
\pgfusepath{fill}%
\end{pgfscope}%
\begin{pgfscope}%
\pgfpathrectangle{\pgfqpoint{0.150000in}{0.150000in}}{\pgfqpoint{4.700000in}{3.450000in}}%
\pgfusepath{clip}%
\pgfsetbuttcap%
\pgfsetroundjoin%
\definecolor{currentfill}{rgb}{0.709375,0.472656,0.491406}%
\pgfsetfillcolor{currentfill}%
\pgfsetlinewidth{0.000000pt}%
\definecolor{currentstroke}{rgb}{0.000000,0.000000,0.000000}%
\pgfsetstrokecolor{currentstroke}%
\pgfsetdash{}{0pt}%
\pgfpathmoveto{\pgfqpoint{3.395582in}{1.544902in}}%
\pgfpathlineto{\pgfqpoint{3.456170in}{1.541417in}}%
\pgfpathlineto{\pgfqpoint{3.391680in}{1.591487in}}%
\pgfpathlineto{\pgfqpoint{3.330589in}{1.587765in}}%
\pgfpathclose%
\pgfusepath{fill}%
\end{pgfscope}%
\begin{pgfscope}%
\pgfpathrectangle{\pgfqpoint{0.150000in}{0.150000in}}{\pgfqpoint{4.700000in}{3.450000in}}%
\pgfusepath{clip}%
\pgfsetbuttcap%
\pgfsetroundjoin%
\definecolor{currentfill}{rgb}{0.659988,0.383042,0.404979}%
\pgfsetfillcolor{currentfill}%
\pgfsetlinewidth{0.000000pt}%
\definecolor{currentstroke}{rgb}{0.000000,0.000000,0.000000}%
\pgfsetstrokecolor{currentstroke}%
\pgfsetdash{}{0pt}%
\pgfpathmoveto{\pgfqpoint{3.651898in}{1.416950in}}%
\pgfpathlineto{\pgfqpoint{3.711562in}{1.414532in}}%
\pgfpathlineto{\pgfqpoint{3.646296in}{1.455787in}}%
\pgfpathlineto{\pgfqpoint{3.586480in}{1.458591in}}%
\pgfpathclose%
\pgfusepath{fill}%
\end{pgfscope}%
\begin{pgfscope}%
\pgfpathrectangle{\pgfqpoint{0.150000in}{0.150000in}}{\pgfqpoint{4.700000in}{3.450000in}}%
\pgfusepath{clip}%
\pgfsetbuttcap%
\pgfsetroundjoin%
\definecolor{currentfill}{rgb}{0.772993,0.800950,0.840089}%
\pgfsetfillcolor{currentfill}%
\pgfsetlinewidth{0.000000pt}%
\definecolor{currentstroke}{rgb}{0.000000,0.000000,0.000000}%
\pgfsetstrokecolor{currentstroke}%
\pgfsetdash{}{0pt}%
\pgfpathmoveto{\pgfqpoint{2.179571in}{2.669972in}}%
\pgfpathlineto{\pgfqpoint{2.244652in}{2.670006in}}%
\pgfpathlineto{\pgfqpoint{2.180933in}{2.733508in}}%
\pgfpathlineto{\pgfqpoint{2.115642in}{2.733692in}}%
\pgfpathclose%
\pgfusepath{fill}%
\end{pgfscope}%
\begin{pgfscope}%
\pgfpathrectangle{\pgfqpoint{0.150000in}{0.150000in}}{\pgfqpoint{4.700000in}{3.450000in}}%
\pgfusepath{clip}%
\pgfsetbuttcap%
\pgfsetroundjoin%
\definecolor{currentfill}{rgb}{0.614400,0.300322,0.325199}%
\pgfsetfillcolor{currentfill}%
\pgfsetlinewidth{0.000000pt}%
\definecolor{currentstroke}{rgb}{0.000000,0.000000,0.000000}%
\pgfsetstrokecolor{currentstroke}%
\pgfsetdash{}{0pt}%
\pgfpathmoveto{\pgfqpoint{3.908441in}{1.290085in}}%
\pgfpathlineto{\pgfqpoint{3.967199in}{1.288722in}}%
\pgfpathlineto{\pgfqpoint{3.901550in}{1.330049in}}%
\pgfpathlineto{\pgfqpoint{3.842634in}{1.331682in}}%
\pgfpathclose%
\pgfusepath{fill}%
\end{pgfscope}%
\begin{pgfscope}%
\pgfpathrectangle{\pgfqpoint{0.150000in}{0.150000in}}{\pgfqpoint{4.700000in}{3.450000in}}%
\pgfusepath{clip}%
\pgfsetbuttcap%
\pgfsetroundjoin%
\definecolor{currentfill}{rgb}{0.810309,0.833670,0.866376}%
\pgfsetfillcolor{currentfill}%
\pgfsetlinewidth{0.000000pt}%
\definecolor{currentstroke}{rgb}{0.000000,0.000000,0.000000}%
\pgfsetstrokecolor{currentstroke}%
\pgfsetdash{}{0pt}%
\pgfpathmoveto{\pgfqpoint{2.243517in}{2.606235in}}%
\pgfpathlineto{\pgfqpoint{2.308389in}{2.606486in}}%
\pgfpathlineto{\pgfqpoint{2.244652in}{2.670006in}}%
\pgfpathlineto{\pgfqpoint{2.179571in}{2.669972in}}%
\pgfpathclose%
\pgfusepath{fill}%
\end{pgfscope}%
\begin{pgfscope}%
\pgfpathrectangle{\pgfqpoint{0.150000in}{0.150000in}}{\pgfqpoint{4.700000in}{3.450000in}}%
\pgfusepath{clip}%
\pgfsetbuttcap%
\pgfsetroundjoin%
\definecolor{currentfill}{rgb}{0.847626,0.866391,0.892662}%
\pgfsetfillcolor{currentfill}%
\pgfsetlinewidth{0.000000pt}%
\definecolor{currentstroke}{rgb}{0.000000,0.000000,0.000000}%
\pgfsetstrokecolor{currentstroke}%
\pgfsetdash{}{0pt}%
\pgfpathmoveto{\pgfqpoint{2.307480in}{2.542480in}}%
\pgfpathlineto{\pgfqpoint{2.372144in}{2.542948in}}%
\pgfpathlineto{\pgfqpoint{2.308389in}{2.606486in}}%
\pgfpathlineto{\pgfqpoint{2.243517in}{2.606235in}}%
\pgfpathclose%
\pgfusepath{fill}%
\end{pgfscope}%
\begin{pgfscope}%
\pgfpathrectangle{\pgfqpoint{0.150000in}{0.150000in}}{\pgfqpoint{4.700000in}{3.450000in}}%
\pgfusepath{clip}%
\pgfsetbuttcap%
\pgfsetroundjoin%
\definecolor{currentfill}{rgb}{0.884942,0.899112,0.918949}%
\pgfsetfillcolor{currentfill}%
\pgfsetlinewidth{0.000000pt}%
\definecolor{currentstroke}{rgb}{0.000000,0.000000,0.000000}%
\pgfsetstrokecolor{currentstroke}%
\pgfsetdash{}{0pt}%
\pgfpathmoveto{\pgfqpoint{2.371462in}{2.478707in}}%
\pgfpathlineto{\pgfqpoint{2.435916in}{2.479393in}}%
\pgfpathlineto{\pgfqpoint{2.372144in}{2.542948in}}%
\pgfpathlineto{\pgfqpoint{2.307480in}{2.542480in}}%
\pgfpathclose%
\pgfusepath{fill}%
\end{pgfscope}%
\begin{pgfscope}%
\pgfpathrectangle{\pgfqpoint{0.150000in}{0.150000in}}{\pgfqpoint{4.700000in}{3.450000in}}%
\pgfusepath{clip}%
\pgfsetbuttcap%
\pgfsetroundjoin%
\definecolor{currentfill}{rgb}{0.922258,0.931832,0.945236}%
\pgfsetfillcolor{currentfill}%
\pgfsetlinewidth{0.000000pt}%
\definecolor{currentstroke}{rgb}{0.000000,0.000000,0.000000}%
\pgfsetstrokecolor{currentstroke}%
\pgfsetdash{}{0pt}%
\pgfpathmoveto{\pgfqpoint{2.435461in}{2.414916in}}%
\pgfpathlineto{\pgfqpoint{2.499706in}{2.415820in}}%
\pgfpathlineto{\pgfqpoint{2.435916in}{2.479393in}}%
\pgfpathlineto{\pgfqpoint{2.371462in}{2.478707in}}%
\pgfpathclose%
\pgfusepath{fill}%
\end{pgfscope}%
\begin{pgfscope}%
\pgfpathrectangle{\pgfqpoint{0.150000in}{0.150000in}}{\pgfqpoint{4.700000in}{3.450000in}}%
\pgfusepath{clip}%
\pgfsetbuttcap%
\pgfsetroundjoin%
\definecolor{currentfill}{rgb}{0.959574,0.964553,0.971523}%
\pgfsetfillcolor{currentfill}%
\pgfsetlinewidth{0.000000pt}%
\definecolor{currentstroke}{rgb}{0.000000,0.000000,0.000000}%
\pgfsetstrokecolor{currentstroke}%
\pgfsetdash{}{0pt}%
\pgfpathmoveto{\pgfqpoint{2.499478in}{2.351108in}}%
\pgfpathlineto{\pgfqpoint{2.563514in}{2.352230in}}%
\pgfpathlineto{\pgfqpoint{2.499706in}{2.415820in}}%
\pgfpathlineto{\pgfqpoint{2.435461in}{2.414916in}}%
\pgfpathclose%
\pgfusepath{fill}%
\end{pgfscope}%
\begin{pgfscope}%
\pgfpathrectangle{\pgfqpoint{0.150000in}{0.150000in}}{\pgfqpoint{4.700000in}{3.450000in}}%
\pgfusepath{clip}%
\pgfsetbuttcap%
\pgfsetroundjoin%
\definecolor{currentfill}{rgb}{0.697978,0.451976,0.471461}%
\pgfsetfillcolor{currentfill}%
\pgfsetlinewidth{0.000000pt}%
\definecolor{currentstroke}{rgb}{0.000000,0.000000,0.000000}%
\pgfsetstrokecolor{currentstroke}%
\pgfsetdash{}{0pt}%
\pgfpathmoveto{\pgfqpoint{3.460805in}{1.503216in}}%
\pgfpathlineto{\pgfqpoint{3.521233in}{1.500002in}}%
\pgfpathlineto{\pgfqpoint{3.456170in}{1.541417in}}%
\pgfpathlineto{\pgfqpoint{3.395582in}{1.544902in}}%
\pgfpathclose%
\pgfusepath{fill}%
\end{pgfscope}%
\begin{pgfscope}%
\pgfpathrectangle{\pgfqpoint{0.150000in}{0.150000in}}{\pgfqpoint{4.700000in}{3.450000in}}%
\pgfusepath{clip}%
\pgfsetbuttcap%
\pgfsetroundjoin%
\definecolor{currentfill}{rgb}{0.996890,0.997273,0.997809}%
\pgfsetfillcolor{currentfill}%
\pgfsetlinewidth{0.000000pt}%
\definecolor{currentstroke}{rgb}{0.000000,0.000000,0.000000}%
\pgfsetstrokecolor{currentstroke}%
\pgfsetdash{}{0pt}%
\pgfpathmoveto{\pgfqpoint{2.563514in}{2.287281in}}%
\pgfpathlineto{\pgfqpoint{2.627339in}{2.288622in}}%
\pgfpathlineto{\pgfqpoint{2.563514in}{2.352230in}}%
\pgfpathlineto{\pgfqpoint{2.499478in}{2.351108in}}%
\pgfpathclose%
\pgfusepath{fill}%
\end{pgfscope}%
\begin{pgfscope}%
\pgfpathrectangle{\pgfqpoint{0.150000in}{0.150000in}}{\pgfqpoint{4.700000in}{3.450000in}}%
\pgfusepath{clip}%
\pgfsetbuttcap%
\pgfsetroundjoin%
\definecolor{currentfill}{rgb}{0.975306,0.955193,0.956786}%
\pgfsetfillcolor{currentfill}%
\pgfsetlinewidth{0.000000pt}%
\definecolor{currentstroke}{rgb}{0.000000,0.000000,0.000000}%
\pgfsetstrokecolor{currentstroke}%
\pgfsetdash{}{0pt}%
\pgfpathmoveto{\pgfqpoint{2.627566in}{2.223437in}}%
\pgfpathlineto{\pgfqpoint{2.691182in}{2.224996in}}%
\pgfpathlineto{\pgfqpoint{2.627339in}{2.288622in}}%
\pgfpathlineto{\pgfqpoint{2.563514in}{2.287281in}}%
\pgfpathclose%
\pgfusepath{fill}%
\end{pgfscope}%
\begin{pgfscope}%
\pgfpathrectangle{\pgfqpoint{0.150000in}{0.150000in}}{\pgfqpoint{4.700000in}{3.450000in}}%
\pgfusepath{clip}%
\pgfsetbuttcap%
\pgfsetroundjoin%
\definecolor{currentfill}{rgb}{0.652390,0.369256,0.391682}%
\pgfsetfillcolor{currentfill}%
\pgfsetlinewidth{0.000000pt}%
\definecolor{currentstroke}{rgb}{0.000000,0.000000,0.000000}%
\pgfsetstrokecolor{currentstroke}%
\pgfsetdash{}{0pt}%
\pgfpathmoveto{\pgfqpoint{3.717496in}{1.375196in}}%
\pgfpathlineto{\pgfqpoint{3.777008in}{1.373164in}}%
\pgfpathlineto{\pgfqpoint{3.711562in}{1.414532in}}%
\pgfpathlineto{\pgfqpoint{3.651898in}{1.416950in}}%
\pgfpathclose%
\pgfusepath{fill}%
\end{pgfscope}%
\begin{pgfscope}%
\pgfpathrectangle{\pgfqpoint{0.150000in}{0.150000in}}{\pgfqpoint{4.700000in}{3.450000in}}%
\pgfusepath{clip}%
\pgfsetbuttcap%
\pgfsetroundjoin%
\definecolor{currentfill}{rgb}{0.952512,0.913833,0.916896}%
\pgfsetfillcolor{currentfill}%
\pgfsetlinewidth{0.000000pt}%
\definecolor{currentstroke}{rgb}{0.000000,0.000000,0.000000}%
\pgfsetstrokecolor{currentstroke}%
\pgfsetdash{}{0pt}%
\pgfpathmoveto{\pgfqpoint{2.691637in}{2.159576in}}%
\pgfpathlineto{\pgfqpoint{2.755043in}{2.161352in}}%
\pgfpathlineto{\pgfqpoint{2.691182in}{2.224996in}}%
\pgfpathlineto{\pgfqpoint{2.627566in}{2.223437in}}%
\pgfpathclose%
\pgfusepath{fill}%
\end{pgfscope}%
\begin{pgfscope}%
\pgfpathrectangle{\pgfqpoint{0.150000in}{0.150000in}}{\pgfqpoint{4.700000in}{3.450000in}}%
\pgfusepath{clip}%
\pgfsetbuttcap%
\pgfsetroundjoin%
\definecolor{currentfill}{rgb}{0.929718,0.872472,0.877007}%
\pgfsetfillcolor{currentfill}%
\pgfsetlinewidth{0.000000pt}%
\definecolor{currentstroke}{rgb}{0.000000,0.000000,0.000000}%
\pgfsetstrokecolor{currentstroke}%
\pgfsetdash{}{0pt}%
\pgfpathmoveto{\pgfqpoint{2.755726in}{2.095696in}}%
\pgfpathlineto{\pgfqpoint{2.818921in}{2.097691in}}%
\pgfpathlineto{\pgfqpoint{2.755043in}{2.161352in}}%
\pgfpathlineto{\pgfqpoint{2.691637in}{2.159576in}}%
\pgfpathclose%
\pgfusepath{fill}%
\end{pgfscope}%
\begin{pgfscope}%
\pgfpathrectangle{\pgfqpoint{0.150000in}{0.150000in}}{\pgfqpoint{4.700000in}{3.450000in}}%
\pgfusepath{clip}%
\pgfsetbuttcap%
\pgfsetroundjoin%
\definecolor{currentfill}{rgb}{0.906924,0.831112,0.837117}%
\pgfsetfillcolor{currentfill}%
\pgfsetlinewidth{0.000000pt}%
\definecolor{currentstroke}{rgb}{0.000000,0.000000,0.000000}%
\pgfsetstrokecolor{currentstroke}%
\pgfsetdash{}{0pt}%
\pgfpathmoveto{\pgfqpoint{2.819832in}{2.031799in}}%
\pgfpathlineto{\pgfqpoint{2.882817in}{2.034013in}}%
\pgfpathlineto{\pgfqpoint{2.818921in}{2.097691in}}%
\pgfpathlineto{\pgfqpoint{2.755726in}{2.095696in}}%
\pgfpathclose%
\pgfusepath{fill}%
\end{pgfscope}%
\begin{pgfscope}%
\pgfpathrectangle{\pgfqpoint{0.150000in}{0.150000in}}{\pgfqpoint{4.700000in}{3.450000in}}%
\pgfusepath{clip}%
\pgfsetbuttcap%
\pgfsetroundjoin%
\definecolor{currentfill}{rgb}{0.884130,0.789752,0.797227}%
\pgfsetfillcolor{currentfill}%
\pgfsetlinewidth{0.000000pt}%
\definecolor{currentstroke}{rgb}{0.000000,0.000000,0.000000}%
\pgfsetstrokecolor{currentstroke}%
\pgfsetdash{}{0pt}%
\pgfpathmoveto{\pgfqpoint{2.883957in}{1.967884in}}%
\pgfpathlineto{\pgfqpoint{2.946731in}{1.970316in}}%
\pgfpathlineto{\pgfqpoint{2.882817in}{2.034013in}}%
\pgfpathlineto{\pgfqpoint{2.819832in}{2.031799in}}%
\pgfpathclose%
\pgfusepath{fill}%
\end{pgfscope}%
\begin{pgfscope}%
\pgfpathrectangle{\pgfqpoint{0.150000in}{0.150000in}}{\pgfqpoint{4.700000in}{3.450000in}}%
\pgfusepath{clip}%
\pgfsetbuttcap%
\pgfsetroundjoin%
\definecolor{currentfill}{rgb}{0.861336,0.748392,0.757338}%
\pgfsetfillcolor{currentfill}%
\pgfsetlinewidth{0.000000pt}%
\definecolor{currentstroke}{rgb}{0.000000,0.000000,0.000000}%
\pgfsetstrokecolor{currentstroke}%
\pgfsetdash{}{0pt}%
\pgfpathmoveto{\pgfqpoint{2.948099in}{1.903951in}}%
\pgfpathlineto{\pgfqpoint{3.010663in}{1.906602in}}%
\pgfpathlineto{\pgfqpoint{2.946731in}{1.970316in}}%
\pgfpathlineto{\pgfqpoint{2.883957in}{1.967884in}}%
\pgfpathclose%
\pgfusepath{fill}%
\end{pgfscope}%
\begin{pgfscope}%
\pgfpathrectangle{\pgfqpoint{0.150000in}{0.150000in}}{\pgfqpoint{4.700000in}{3.450000in}}%
\pgfusepath{clip}%
\pgfsetbuttcap%
\pgfsetroundjoin%
\definecolor{currentfill}{rgb}{0.838542,0.707031,0.717448}%
\pgfsetfillcolor{currentfill}%
\pgfsetlinewidth{0.000000pt}%
\definecolor{currentstroke}{rgb}{0.000000,0.000000,0.000000}%
\pgfsetstrokecolor{currentstroke}%
\pgfsetdash{}{0pt}%
\pgfpathmoveto{\pgfqpoint{3.012259in}{1.840000in}}%
\pgfpathlineto{\pgfqpoint{3.074613in}{1.842870in}}%
\pgfpathlineto{\pgfqpoint{3.010663in}{1.906602in}}%
\pgfpathlineto{\pgfqpoint{2.948099in}{1.903951in}}%
\pgfpathclose%
\pgfusepath{fill}%
\end{pgfscope}%
\begin{pgfscope}%
\pgfpathrectangle{\pgfqpoint{0.150000in}{0.150000in}}{\pgfqpoint{4.700000in}{3.450000in}}%
\pgfusepath{clip}%
\pgfsetbuttcap%
\pgfsetroundjoin%
\definecolor{currentfill}{rgb}{0.815748,0.665671,0.677558}%
\pgfsetfillcolor{currentfill}%
\pgfsetlinewidth{0.000000pt}%
\definecolor{currentstroke}{rgb}{0.000000,0.000000,0.000000}%
\pgfsetstrokecolor{currentstroke}%
\pgfsetdash{}{0pt}%
\pgfpathmoveto{\pgfqpoint{3.076437in}{1.776031in}}%
\pgfpathlineto{\pgfqpoint{3.138580in}{1.779120in}}%
\pgfpathlineto{\pgfqpoint{3.074613in}{1.842870in}}%
\pgfpathlineto{\pgfqpoint{3.012259in}{1.840000in}}%
\pgfpathclose%
\pgfusepath{fill}%
\end{pgfscope}%
\begin{pgfscope}%
\pgfpathrectangle{\pgfqpoint{0.150000in}{0.150000in}}{\pgfqpoint{4.700000in}{3.450000in}}%
\pgfusepath{clip}%
\pgfsetbuttcap%
\pgfsetroundjoin%
\definecolor{currentfill}{rgb}{0.792953,0.624311,0.637669}%
\pgfsetfillcolor{currentfill}%
\pgfsetlinewidth{0.000000pt}%
\definecolor{currentstroke}{rgb}{0.000000,0.000000,0.000000}%
\pgfsetstrokecolor{currentstroke}%
\pgfsetdash{}{0pt}%
\pgfpathmoveto{\pgfqpoint{3.140633in}{1.712045in}}%
\pgfpathlineto{\pgfqpoint{3.202565in}{1.715353in}}%
\pgfpathlineto{\pgfqpoint{3.138580in}{1.779120in}}%
\pgfpathlineto{\pgfqpoint{3.076437in}{1.776031in}}%
\pgfpathclose%
\pgfusepath{fill}%
\end{pgfscope}%
\begin{pgfscope}%
\pgfpathrectangle{\pgfqpoint{0.150000in}{0.150000in}}{\pgfqpoint{4.700000in}{3.450000in}}%
\pgfusepath{clip}%
\pgfsetbuttcap%
\pgfsetroundjoin%
\definecolor{currentfill}{rgb}{0.694179,0.445083,0.464813}%
\pgfsetfillcolor{currentfill}%
\pgfsetlinewidth{0.000000pt}%
\definecolor{currentstroke}{rgb}{0.000000,0.000000,0.000000}%
\pgfsetstrokecolor{currentstroke}%
\pgfsetdash{}{0pt}%
\pgfpathmoveto{\pgfqpoint{3.526201in}{1.461300in}}%
\pgfpathlineto{\pgfqpoint{3.586480in}{1.458591in}}%
\pgfpathlineto{\pgfqpoint{3.521233in}{1.500002in}}%
\pgfpathlineto{\pgfqpoint{3.460805in}{1.503216in}}%
\pgfpathclose%
\pgfusepath{fill}%
\end{pgfscope}%
\begin{pgfscope}%
\pgfpathrectangle{\pgfqpoint{0.150000in}{0.150000in}}{\pgfqpoint{4.700000in}{3.450000in}}%
\pgfusepath{clip}%
\pgfsetbuttcap%
\pgfsetroundjoin%
\definecolor{currentfill}{rgb}{0.770159,0.582950,0.597779}%
\pgfsetfillcolor{currentfill}%
\pgfsetlinewidth{0.000000pt}%
\definecolor{currentstroke}{rgb}{0.000000,0.000000,0.000000}%
\pgfsetstrokecolor{currentstroke}%
\pgfsetdash{}{0pt}%
\pgfpathmoveto{\pgfqpoint{3.204847in}{1.648040in}}%
\pgfpathlineto{\pgfqpoint{3.266568in}{1.651568in}}%
\pgfpathlineto{\pgfqpoint{3.202565in}{1.715353in}}%
\pgfpathlineto{\pgfqpoint{3.140633in}{1.712045in}}%
\pgfpathclose%
\pgfusepath{fill}%
\end{pgfscope}%
\begin{pgfscope}%
\pgfpathrectangle{\pgfqpoint{0.150000in}{0.150000in}}{\pgfqpoint{4.700000in}{3.450000in}}%
\pgfusepath{clip}%
\pgfsetbuttcap%
\pgfsetroundjoin%
\definecolor{currentfill}{rgb}{0.648591,0.362362,0.385034}%
\pgfsetfillcolor{currentfill}%
\pgfsetlinewidth{0.000000pt}%
\definecolor{currentstroke}{rgb}{0.000000,0.000000,0.000000}%
\pgfsetstrokecolor{currentstroke}%
\pgfsetdash{}{0pt}%
\pgfpathmoveto{\pgfqpoint{3.783281in}{1.333443in}}%
\pgfpathlineto{\pgfqpoint{3.842634in}{1.331682in}}%
\pgfpathlineto{\pgfqpoint{3.777008in}{1.373164in}}%
\pgfpathlineto{\pgfqpoint{3.717496in}{1.375196in}}%
\pgfpathclose%
\pgfusepath{fill}%
\end{pgfscope}%
\begin{pgfscope}%
\pgfpathrectangle{\pgfqpoint{0.150000in}{0.150000in}}{\pgfqpoint{4.700000in}{3.450000in}}%
\pgfusepath{clip}%
\pgfsetbuttcap%
\pgfsetroundjoin%
\definecolor{currentfill}{rgb}{0.747365,0.541590,0.557889}%
\pgfsetfillcolor{currentfill}%
\pgfsetlinewidth{0.000000pt}%
\definecolor{currentstroke}{rgb}{0.000000,0.000000,0.000000}%
\pgfsetstrokecolor{currentstroke}%
\pgfsetdash{}{0pt}%
\pgfpathmoveto{\pgfqpoint{3.269323in}{1.590377in}}%
\pgfpathlineto{\pgfqpoint{3.330589in}{1.587765in}}%
\pgfpathlineto{\pgfqpoint{3.266568in}{1.651568in}}%
\pgfpathlineto{\pgfqpoint{3.204847in}{1.648040in}}%
\pgfpathclose%
\pgfusepath{fill}%
\end{pgfscope}%
\begin{pgfscope}%
\pgfpathrectangle{\pgfqpoint{0.150000in}{0.150000in}}{\pgfqpoint{4.700000in}{3.450000in}}%
\pgfusepath{clip}%
\pgfsetbuttcap%
\pgfsetroundjoin%
\definecolor{currentfill}{rgb}{0.735968,0.520910,0.537944}%
\pgfsetfillcolor{currentfill}%
\pgfsetlinewidth{0.000000pt}%
\definecolor{currentstroke}{rgb}{0.000000,0.000000,0.000000}%
\pgfsetstrokecolor{currentstroke}%
\pgfsetdash{}{0pt}%
\pgfpathmoveto{\pgfqpoint{3.334528in}{1.548414in}}%
\pgfpathlineto{\pgfqpoint{3.395582in}{1.544902in}}%
\pgfpathlineto{\pgfqpoint{3.330589in}{1.587765in}}%
\pgfpathlineto{\pgfqpoint{3.269323in}{1.590377in}}%
\pgfpathclose%
\pgfusepath{fill}%
\end{pgfscope}%
\begin{pgfscope}%
\pgfpathrectangle{\pgfqpoint{0.150000in}{0.150000in}}{\pgfqpoint{4.700000in}{3.450000in}}%
\pgfusepath{clip}%
\pgfsetbuttcap%
\pgfsetroundjoin%
\definecolor{currentfill}{rgb}{0.735677,0.768229,0.813802}%
\pgfsetfillcolor{currentfill}%
\pgfsetlinewidth{0.000000pt}%
\definecolor{currentstroke}{rgb}{0.000000,0.000000,0.000000}%
\pgfsetstrokecolor{currentstroke}%
\pgfsetdash{}{0pt}%
\pgfpathmoveto{\pgfqpoint{2.114042in}{2.669938in}}%
\pgfpathlineto{\pgfqpoint{2.179571in}{2.669972in}}%
\pgfpathlineto{\pgfqpoint{2.115642in}{2.733692in}}%
\pgfpathlineto{\pgfqpoint{2.049903in}{2.733876in}}%
\pgfpathclose%
\pgfusepath{fill}%
\end{pgfscope}%
\begin{pgfscope}%
\pgfpathrectangle{\pgfqpoint{0.150000in}{0.150000in}}{\pgfqpoint{4.700000in}{3.450000in}}%
\pgfusepath{clip}%
\pgfsetbuttcap%
\pgfsetroundjoin%
\definecolor{currentfill}{rgb}{0.686581,0.431296,0.451517}%
\pgfsetfillcolor{currentfill}%
\pgfsetlinewidth{0.000000pt}%
\definecolor{currentstroke}{rgb}{0.000000,0.000000,0.000000}%
\pgfsetstrokecolor{currentstroke}%
\pgfsetdash{}{0pt}%
\pgfpathmoveto{\pgfqpoint{3.591780in}{1.419387in}}%
\pgfpathlineto{\pgfqpoint{3.651898in}{1.416950in}}%
\pgfpathlineto{\pgfqpoint{3.586480in}{1.458591in}}%
\pgfpathlineto{\pgfqpoint{3.526201in}{1.461300in}}%
\pgfpathclose%
\pgfusepath{fill}%
\end{pgfscope}%
\begin{pgfscope}%
\pgfpathrectangle{\pgfqpoint{0.150000in}{0.150000in}}{\pgfqpoint{4.700000in}{3.450000in}}%
\pgfusepath{clip}%
\pgfsetbuttcap%
\pgfsetroundjoin%
\definecolor{currentfill}{rgb}{0.772993,0.800950,0.840089}%
\pgfsetfillcolor{currentfill}%
\pgfsetlinewidth{0.000000pt}%
\definecolor{currentstroke}{rgb}{0.000000,0.000000,0.000000}%
\pgfsetstrokecolor{currentstroke}%
\pgfsetdash{}{0pt}%
\pgfpathmoveto{\pgfqpoint{2.178198in}{2.605982in}}%
\pgfpathlineto{\pgfqpoint{2.243517in}{2.606235in}}%
\pgfpathlineto{\pgfqpoint{2.179571in}{2.669972in}}%
\pgfpathlineto{\pgfqpoint{2.114042in}{2.669938in}}%
\pgfpathclose%
\pgfusepath{fill}%
\end{pgfscope}%
\begin{pgfscope}%
\pgfpathrectangle{\pgfqpoint{0.150000in}{0.150000in}}{\pgfqpoint{4.700000in}{3.450000in}}%
\pgfusepath{clip}%
\pgfsetbuttcap%
\pgfsetroundjoin%
\definecolor{currentfill}{rgb}{0.810309,0.833670,0.866376}%
\pgfsetfillcolor{currentfill}%
\pgfsetlinewidth{0.000000pt}%
\definecolor{currentstroke}{rgb}{0.000000,0.000000,0.000000}%
\pgfsetstrokecolor{currentstroke}%
\pgfsetdash{}{0pt}%
\pgfpathmoveto{\pgfqpoint{2.242372in}{2.542007in}}%
\pgfpathlineto{\pgfqpoint{2.307480in}{2.542480in}}%
\pgfpathlineto{\pgfqpoint{2.243517in}{2.606235in}}%
\pgfpathlineto{\pgfqpoint{2.178198in}{2.605982in}}%
\pgfpathclose%
\pgfusepath{fill}%
\end{pgfscope}%
\begin{pgfscope}%
\pgfpathrectangle{\pgfqpoint{0.150000in}{0.150000in}}{\pgfqpoint{4.700000in}{3.450000in}}%
\pgfusepath{clip}%
\pgfsetbuttcap%
\pgfsetroundjoin%
\definecolor{currentfill}{rgb}{0.640993,0.348575,0.371737}%
\pgfsetfillcolor{currentfill}%
\pgfsetlinewidth{0.000000pt}%
\definecolor{currentstroke}{rgb}{0.000000,0.000000,0.000000}%
\pgfsetstrokecolor{currentstroke}%
\pgfsetdash{}{0pt}%
\pgfpathmoveto{\pgfqpoint{3.849239in}{1.291459in}}%
\pgfpathlineto{\pgfqpoint{3.908441in}{1.290085in}}%
\pgfpathlineto{\pgfqpoint{3.842634in}{1.331682in}}%
\pgfpathlineto{\pgfqpoint{3.783281in}{1.333443in}}%
\pgfpathclose%
\pgfusepath{fill}%
\end{pgfscope}%
\begin{pgfscope}%
\pgfpathrectangle{\pgfqpoint{0.150000in}{0.150000in}}{\pgfqpoint{4.700000in}{3.450000in}}%
\pgfusepath{clip}%
\pgfsetbuttcap%
\pgfsetroundjoin%
\definecolor{currentfill}{rgb}{0.847626,0.866391,0.892662}%
\pgfsetfillcolor{currentfill}%
\pgfsetlinewidth{0.000000pt}%
\definecolor{currentstroke}{rgb}{0.000000,0.000000,0.000000}%
\pgfsetstrokecolor{currentstroke}%
\pgfsetdash{}{0pt}%
\pgfpathmoveto{\pgfqpoint{2.306565in}{2.478015in}}%
\pgfpathlineto{\pgfqpoint{2.371462in}{2.478707in}}%
\pgfpathlineto{\pgfqpoint{2.307480in}{2.542480in}}%
\pgfpathlineto{\pgfqpoint{2.242372in}{2.542007in}}%
\pgfpathclose%
\pgfusepath{fill}%
\end{pgfscope}%
\begin{pgfscope}%
\pgfpathrectangle{\pgfqpoint{0.150000in}{0.150000in}}{\pgfqpoint{4.700000in}{3.450000in}}%
\pgfusepath{clip}%
\pgfsetbuttcap%
\pgfsetroundjoin%
\definecolor{currentfill}{rgb}{0.884942,0.899112,0.918949}%
\pgfsetfillcolor{currentfill}%
\pgfsetlinewidth{0.000000pt}%
\definecolor{currentstroke}{rgb}{0.000000,0.000000,0.000000}%
\pgfsetstrokecolor{currentstroke}%
\pgfsetdash{}{0pt}%
\pgfpathmoveto{\pgfqpoint{2.370775in}{2.414005in}}%
\pgfpathlineto{\pgfqpoint{2.435461in}{2.414916in}}%
\pgfpathlineto{\pgfqpoint{2.371462in}{2.478707in}}%
\pgfpathlineto{\pgfqpoint{2.306565in}{2.478015in}}%
\pgfpathclose%
\pgfusepath{fill}%
\end{pgfscope}%
\begin{pgfscope}%
\pgfpathrectangle{\pgfqpoint{0.150000in}{0.150000in}}{\pgfqpoint{4.700000in}{3.450000in}}%
\pgfusepath{clip}%
\pgfsetbuttcap%
\pgfsetroundjoin%
\definecolor{currentfill}{rgb}{0.922258,0.931832,0.945236}%
\pgfsetfillcolor{currentfill}%
\pgfsetlinewidth{0.000000pt}%
\definecolor{currentstroke}{rgb}{0.000000,0.000000,0.000000}%
\pgfsetstrokecolor{currentstroke}%
\pgfsetdash{}{0pt}%
\pgfpathmoveto{\pgfqpoint{2.435003in}{2.349978in}}%
\pgfpathlineto{\pgfqpoint{2.499478in}{2.351108in}}%
\pgfpathlineto{\pgfqpoint{2.435461in}{2.414916in}}%
\pgfpathlineto{\pgfqpoint{2.370775in}{2.414005in}}%
\pgfpathclose%
\pgfusepath{fill}%
\end{pgfscope}%
\begin{pgfscope}%
\pgfpathrectangle{\pgfqpoint{0.150000in}{0.150000in}}{\pgfqpoint{4.700000in}{3.450000in}}%
\pgfusepath{clip}%
\pgfsetbuttcap%
\pgfsetroundjoin%
\definecolor{currentfill}{rgb}{0.959574,0.964553,0.971523}%
\pgfsetfillcolor{currentfill}%
\pgfsetlinewidth{0.000000pt}%
\definecolor{currentstroke}{rgb}{0.000000,0.000000,0.000000}%
\pgfsetstrokecolor{currentstroke}%
\pgfsetdash{}{0pt}%
\pgfpathmoveto{\pgfqpoint{2.499249in}{2.285932in}}%
\pgfpathlineto{\pgfqpoint{2.563514in}{2.287281in}}%
\pgfpathlineto{\pgfqpoint{2.499478in}{2.351108in}}%
\pgfpathlineto{\pgfqpoint{2.435003in}{2.349978in}}%
\pgfpathclose%
\pgfusepath{fill}%
\end{pgfscope}%
\begin{pgfscope}%
\pgfpathrectangle{\pgfqpoint{0.150000in}{0.150000in}}{\pgfqpoint{4.700000in}{3.450000in}}%
\pgfusepath{clip}%
\pgfsetbuttcap%
\pgfsetroundjoin%
\definecolor{currentfill}{rgb}{0.996890,0.997273,0.997809}%
\pgfsetfillcolor{currentfill}%
\pgfsetlinewidth{0.000000pt}%
\definecolor{currentstroke}{rgb}{0.000000,0.000000,0.000000}%
\pgfsetstrokecolor{currentstroke}%
\pgfsetdash{}{0pt}%
\pgfpathmoveto{\pgfqpoint{2.563514in}{2.221868in}}%
\pgfpathlineto{\pgfqpoint{2.627566in}{2.223437in}}%
\pgfpathlineto{\pgfqpoint{2.563514in}{2.287281in}}%
\pgfpathlineto{\pgfqpoint{2.499249in}{2.285932in}}%
\pgfpathclose%
\pgfusepath{fill}%
\end{pgfscope}%
\begin{pgfscope}%
\pgfpathrectangle{\pgfqpoint{0.150000in}{0.150000in}}{\pgfqpoint{4.700000in}{3.450000in}}%
\pgfusepath{clip}%
\pgfsetbuttcap%
\pgfsetroundjoin%
\definecolor{currentfill}{rgb}{0.975306,0.955193,0.956786}%
\pgfsetfillcolor{currentfill}%
\pgfsetlinewidth{0.000000pt}%
\definecolor{currentstroke}{rgb}{0.000000,0.000000,0.000000}%
\pgfsetstrokecolor{currentstroke}%
\pgfsetdash{}{0pt}%
\pgfpathmoveto{\pgfqpoint{2.627796in}{2.157787in}}%
\pgfpathlineto{\pgfqpoint{2.691637in}{2.159576in}}%
\pgfpathlineto{\pgfqpoint{2.627566in}{2.223437in}}%
\pgfpathlineto{\pgfqpoint{2.563514in}{2.221868in}}%
\pgfpathclose%
\pgfusepath{fill}%
\end{pgfscope}%
\begin{pgfscope}%
\pgfpathrectangle{\pgfqpoint{0.150000in}{0.150000in}}{\pgfqpoint{4.700000in}{3.450000in}}%
\pgfusepath{clip}%
\pgfsetbuttcap%
\pgfsetroundjoin%
\definecolor{currentfill}{rgb}{0.728370,0.507123,0.524648}%
\pgfsetfillcolor{currentfill}%
\pgfsetlinewidth{0.000000pt}%
\definecolor{currentstroke}{rgb}{0.000000,0.000000,0.000000}%
\pgfsetstrokecolor{currentstroke}%
\pgfsetdash{}{0pt}%
\pgfpathmoveto{\pgfqpoint{3.399908in}{1.506338in}}%
\pgfpathlineto{\pgfqpoint{3.460805in}{1.503216in}}%
\pgfpathlineto{\pgfqpoint{3.395582in}{1.544902in}}%
\pgfpathlineto{\pgfqpoint{3.334528in}{1.548414in}}%
\pgfpathclose%
\pgfusepath{fill}%
\end{pgfscope}%
\begin{pgfscope}%
\pgfpathrectangle{\pgfqpoint{0.150000in}{0.150000in}}{\pgfqpoint{4.700000in}{3.450000in}}%
\pgfusepath{clip}%
\pgfsetbuttcap%
\pgfsetroundjoin%
\definecolor{currentfill}{rgb}{0.952512,0.913833,0.916896}%
\pgfsetfillcolor{currentfill}%
\pgfsetlinewidth{0.000000pt}%
\definecolor{currentstroke}{rgb}{0.000000,0.000000,0.000000}%
\pgfsetstrokecolor{currentstroke}%
\pgfsetdash{}{0pt}%
\pgfpathmoveto{\pgfqpoint{2.692096in}{2.093687in}}%
\pgfpathlineto{\pgfqpoint{2.755726in}{2.095696in}}%
\pgfpathlineto{\pgfqpoint{2.691637in}{2.159576in}}%
\pgfpathlineto{\pgfqpoint{2.627796in}{2.157787in}}%
\pgfpathclose%
\pgfusepath{fill}%
\end{pgfscope}%
\begin{pgfscope}%
\pgfpathrectangle{\pgfqpoint{0.150000in}{0.150000in}}{\pgfqpoint{4.700000in}{3.450000in}}%
\pgfusepath{clip}%
\pgfsetbuttcap%
\pgfsetroundjoin%
\definecolor{currentfill}{rgb}{0.929718,0.872472,0.877007}%
\pgfsetfillcolor{currentfill}%
\pgfsetlinewidth{0.000000pt}%
\definecolor{currentstroke}{rgb}{0.000000,0.000000,0.000000}%
\pgfsetstrokecolor{currentstroke}%
\pgfsetdash{}{0pt}%
\pgfpathmoveto{\pgfqpoint{2.756414in}{2.029570in}}%
\pgfpathlineto{\pgfqpoint{2.819832in}{2.031799in}}%
\pgfpathlineto{\pgfqpoint{2.755726in}{2.095696in}}%
\pgfpathlineto{\pgfqpoint{2.692096in}{2.093687in}}%
\pgfpathclose%
\pgfusepath{fill}%
\end{pgfscope}%
\begin{pgfscope}%
\pgfpathrectangle{\pgfqpoint{0.150000in}{0.150000in}}{\pgfqpoint{4.700000in}{3.450000in}}%
\pgfusepath{clip}%
\pgfsetbuttcap%
\pgfsetroundjoin%
\definecolor{currentfill}{rgb}{0.682782,0.424403,0.444868}%
\pgfsetfillcolor{currentfill}%
\pgfsetlinewidth{0.000000pt}%
\definecolor{currentstroke}{rgb}{0.000000,0.000000,0.000000}%
\pgfsetstrokecolor{currentstroke}%
\pgfsetdash{}{0pt}%
\pgfpathmoveto{\pgfqpoint{3.657538in}{1.377360in}}%
\pgfpathlineto{\pgfqpoint{3.717496in}{1.375196in}}%
\pgfpathlineto{\pgfqpoint{3.651898in}{1.416950in}}%
\pgfpathlineto{\pgfqpoint{3.591780in}{1.419387in}}%
\pgfpathclose%
\pgfusepath{fill}%
\end{pgfscope}%
\begin{pgfscope}%
\pgfpathrectangle{\pgfqpoint{0.150000in}{0.150000in}}{\pgfqpoint{4.700000in}{3.450000in}}%
\pgfusepath{clip}%
\pgfsetbuttcap%
\pgfsetroundjoin%
\definecolor{currentfill}{rgb}{0.906924,0.831112,0.837117}%
\pgfsetfillcolor{currentfill}%
\pgfsetlinewidth{0.000000pt}%
\definecolor{currentstroke}{rgb}{0.000000,0.000000,0.000000}%
\pgfsetstrokecolor{currentstroke}%
\pgfsetdash{}{0pt}%
\pgfpathmoveto{\pgfqpoint{2.820750in}{1.965434in}}%
\pgfpathlineto{\pgfqpoint{2.883957in}{1.967884in}}%
\pgfpathlineto{\pgfqpoint{2.819832in}{2.031799in}}%
\pgfpathlineto{\pgfqpoint{2.756414in}{2.029570in}}%
\pgfpathclose%
\pgfusepath{fill}%
\end{pgfscope}%
\begin{pgfscope}%
\pgfpathrectangle{\pgfqpoint{0.150000in}{0.150000in}}{\pgfqpoint{4.700000in}{3.450000in}}%
\pgfusepath{clip}%
\pgfsetbuttcap%
\pgfsetroundjoin%
\definecolor{currentfill}{rgb}{0.884130,0.789752,0.797227}%
\pgfsetfillcolor{currentfill}%
\pgfsetlinewidth{0.000000pt}%
\definecolor{currentstroke}{rgb}{0.000000,0.000000,0.000000}%
\pgfsetstrokecolor{currentstroke}%
\pgfsetdash{}{0pt}%
\pgfpathmoveto{\pgfqpoint{2.885104in}{1.901281in}}%
\pgfpathlineto{\pgfqpoint{2.948099in}{1.903951in}}%
\pgfpathlineto{\pgfqpoint{2.883957in}{1.967884in}}%
\pgfpathlineto{\pgfqpoint{2.820750in}{1.965434in}}%
\pgfpathclose%
\pgfusepath{fill}%
\end{pgfscope}%
\begin{pgfscope}%
\pgfpathrectangle{\pgfqpoint{0.150000in}{0.150000in}}{\pgfqpoint{4.700000in}{3.450000in}}%
\pgfusepath{clip}%
\pgfsetbuttcap%
\pgfsetroundjoin%
\definecolor{currentfill}{rgb}{0.861336,0.748392,0.757338}%
\pgfsetfillcolor{currentfill}%
\pgfsetlinewidth{0.000000pt}%
\definecolor{currentstroke}{rgb}{0.000000,0.000000,0.000000}%
\pgfsetstrokecolor{currentstroke}%
\pgfsetdash{}{0pt}%
\pgfpathmoveto{\pgfqpoint{2.949476in}{1.837110in}}%
\pgfpathlineto{\pgfqpoint{3.012259in}{1.840000in}}%
\pgfpathlineto{\pgfqpoint{2.948099in}{1.903951in}}%
\pgfpathlineto{\pgfqpoint{2.885104in}{1.901281in}}%
\pgfpathclose%
\pgfusepath{fill}%
\end{pgfscope}%
\begin{pgfscope}%
\pgfpathrectangle{\pgfqpoint{0.150000in}{0.150000in}}{\pgfqpoint{4.700000in}{3.450000in}}%
\pgfusepath{clip}%
\pgfsetbuttcap%
\pgfsetroundjoin%
\definecolor{currentfill}{rgb}{0.838542,0.707031,0.717448}%
\pgfsetfillcolor{currentfill}%
\pgfsetlinewidth{0.000000pt}%
\definecolor{currentstroke}{rgb}{0.000000,0.000000,0.000000}%
\pgfsetstrokecolor{currentstroke}%
\pgfsetdash{}{0pt}%
\pgfpathmoveto{\pgfqpoint{3.013866in}{1.772921in}}%
\pgfpathlineto{\pgfqpoint{3.076437in}{1.776031in}}%
\pgfpathlineto{\pgfqpoint{3.012259in}{1.840000in}}%
\pgfpathlineto{\pgfqpoint{2.949476in}{1.837110in}}%
\pgfpathclose%
\pgfusepath{fill}%
\end{pgfscope}%
\begin{pgfscope}%
\pgfpathrectangle{\pgfqpoint{0.150000in}{0.150000in}}{\pgfqpoint{4.700000in}{3.450000in}}%
\pgfusepath{clip}%
\pgfsetbuttcap%
\pgfsetroundjoin%
\definecolor{currentfill}{rgb}{0.815748,0.665671,0.677558}%
\pgfsetfillcolor{currentfill}%
\pgfsetlinewidth{0.000000pt}%
\definecolor{currentstroke}{rgb}{0.000000,0.000000,0.000000}%
\pgfsetstrokecolor{currentstroke}%
\pgfsetdash{}{0pt}%
\pgfpathmoveto{\pgfqpoint{3.078274in}{1.708713in}}%
\pgfpathlineto{\pgfqpoint{3.140633in}{1.712045in}}%
\pgfpathlineto{\pgfqpoint{3.076437in}{1.776031in}}%
\pgfpathlineto{\pgfqpoint{3.013866in}{1.772921in}}%
\pgfpathclose%
\pgfusepath{fill}%
\end{pgfscope}%
\begin{pgfscope}%
\pgfpathrectangle{\pgfqpoint{0.150000in}{0.150000in}}{\pgfqpoint{4.700000in}{3.450000in}}%
\pgfusepath{clip}%
\pgfsetbuttcap%
\pgfsetroundjoin%
\definecolor{currentfill}{rgb}{0.792953,0.624311,0.637669}%
\pgfsetfillcolor{currentfill}%
\pgfsetlinewidth{0.000000pt}%
\definecolor{currentstroke}{rgb}{0.000000,0.000000,0.000000}%
\pgfsetstrokecolor{currentstroke}%
\pgfsetdash{}{0pt}%
\pgfpathmoveto{\pgfqpoint{3.142700in}{1.644488in}}%
\pgfpathlineto{\pgfqpoint{3.204847in}{1.648040in}}%
\pgfpathlineto{\pgfqpoint{3.140633in}{1.712045in}}%
\pgfpathlineto{\pgfqpoint{3.078274in}{1.708713in}}%
\pgfpathclose%
\pgfusepath{fill}%
\end{pgfscope}%
\begin{pgfscope}%
\pgfpathrectangle{\pgfqpoint{0.150000in}{0.150000in}}{\pgfqpoint{4.700000in}{3.450000in}}%
\pgfusepath{clip}%
\pgfsetbuttcap%
\pgfsetroundjoin%
\definecolor{currentfill}{rgb}{0.720772,0.493336,0.511351}%
\pgfsetfillcolor{currentfill}%
\pgfsetlinewidth{0.000000pt}%
\definecolor{currentstroke}{rgb}{0.000000,0.000000,0.000000}%
\pgfsetstrokecolor{currentstroke}%
\pgfsetdash{}{0pt}%
\pgfpathmoveto{\pgfqpoint{3.465466in}{1.464147in}}%
\pgfpathlineto{\pgfqpoint{3.526201in}{1.461300in}}%
\pgfpathlineto{\pgfqpoint{3.460805in}{1.503216in}}%
\pgfpathlineto{\pgfqpoint{3.399908in}{1.506338in}}%
\pgfpathclose%
\pgfusepath{fill}%
\end{pgfscope}%
\begin{pgfscope}%
\pgfpathrectangle{\pgfqpoint{0.150000in}{0.150000in}}{\pgfqpoint{4.700000in}{3.450000in}}%
\pgfusepath{clip}%
\pgfsetbuttcap%
\pgfsetroundjoin%
\definecolor{currentfill}{rgb}{0.773958,0.589844,0.604427}%
\pgfsetfillcolor{currentfill}%
\pgfsetlinewidth{0.000000pt}%
\definecolor{currentstroke}{rgb}{0.000000,0.000000,0.000000}%
\pgfsetstrokecolor{currentstroke}%
\pgfsetdash{}{0pt}%
\pgfpathmoveto{\pgfqpoint{3.207639in}{1.594310in}}%
\pgfpathlineto{\pgfqpoint{3.269323in}{1.590377in}}%
\pgfpathlineto{\pgfqpoint{3.204847in}{1.648040in}}%
\pgfpathlineto{\pgfqpoint{3.142700in}{1.644488in}}%
\pgfpathclose%
\pgfusepath{fill}%
\end{pgfscope}%
\begin{pgfscope}%
\pgfpathrectangle{\pgfqpoint{0.150000in}{0.150000in}}{\pgfqpoint{4.700000in}{3.450000in}}%
\pgfusepath{clip}%
\pgfsetbuttcap%
\pgfsetroundjoin%
\definecolor{currentfill}{rgb}{0.675184,0.410616,0.431572}%
\pgfsetfillcolor{currentfill}%
\pgfsetlinewidth{0.000000pt}%
\definecolor{currentstroke}{rgb}{0.000000,0.000000,0.000000}%
\pgfsetstrokecolor{currentstroke}%
\pgfsetdash{}{0pt}%
\pgfpathmoveto{\pgfqpoint{3.723469in}{1.335101in}}%
\pgfpathlineto{\pgfqpoint{3.783281in}{1.333443in}}%
\pgfpathlineto{\pgfqpoint{3.717496in}{1.375196in}}%
\pgfpathlineto{\pgfqpoint{3.657538in}{1.377360in}}%
\pgfpathclose%
\pgfusepath{fill}%
\end{pgfscope}%
\begin{pgfscope}%
\pgfpathrectangle{\pgfqpoint{0.150000in}{0.150000in}}{\pgfqpoint{4.700000in}{3.450000in}}%
\pgfusepath{clip}%
\pgfsetbuttcap%
\pgfsetroundjoin%
\definecolor{currentfill}{rgb}{0.698361,0.735509,0.787515}%
\pgfsetfillcolor{currentfill}%
\pgfsetlinewidth{0.000000pt}%
\definecolor{currentstroke}{rgb}{0.000000,0.000000,0.000000}%
\pgfsetstrokecolor{currentstroke}%
\pgfsetdash{}{0pt}%
\pgfpathmoveto{\pgfqpoint{2.048061in}{2.669904in}}%
\pgfpathlineto{\pgfqpoint{2.114042in}{2.669938in}}%
\pgfpathlineto{\pgfqpoint{2.049903in}{2.733876in}}%
\pgfpathlineto{\pgfqpoint{1.983711in}{2.734063in}}%
\pgfpathclose%
\pgfusepath{fill}%
\end{pgfscope}%
\begin{pgfscope}%
\pgfpathrectangle{\pgfqpoint{0.150000in}{0.150000in}}{\pgfqpoint{4.700000in}{3.450000in}}%
\pgfusepath{clip}%
\pgfsetbuttcap%
\pgfsetroundjoin%
\definecolor{currentfill}{rgb}{0.735677,0.768229,0.813802}%
\pgfsetfillcolor{currentfill}%
\pgfsetlinewidth{0.000000pt}%
\definecolor{currentstroke}{rgb}{0.000000,0.000000,0.000000}%
\pgfsetstrokecolor{currentstroke}%
\pgfsetdash{}{0pt}%
\pgfpathmoveto{\pgfqpoint{2.112429in}{2.605727in}}%
\pgfpathlineto{\pgfqpoint{2.178198in}{2.605982in}}%
\pgfpathlineto{\pgfqpoint{2.114042in}{2.669938in}}%
\pgfpathlineto{\pgfqpoint{2.048061in}{2.669904in}}%
\pgfpathclose%
\pgfusepath{fill}%
\end{pgfscope}%
\begin{pgfscope}%
\pgfpathrectangle{\pgfqpoint{0.150000in}{0.150000in}}{\pgfqpoint{4.700000in}{3.450000in}}%
\pgfusepath{clip}%
\pgfsetbuttcap%
\pgfsetroundjoin%
\definecolor{currentfill}{rgb}{0.762561,0.569164,0.584482}%
\pgfsetfillcolor{currentfill}%
\pgfsetlinewidth{0.000000pt}%
\definecolor{currentstroke}{rgb}{0.000000,0.000000,0.000000}%
\pgfsetstrokecolor{currentstroke}%
\pgfsetdash{}{0pt}%
\pgfpathmoveto{\pgfqpoint{3.273003in}{1.551953in}}%
\pgfpathlineto{\pgfqpoint{3.334528in}{1.548414in}}%
\pgfpathlineto{\pgfqpoint{3.269323in}{1.590377in}}%
\pgfpathlineto{\pgfqpoint{3.207639in}{1.594310in}}%
\pgfpathclose%
\pgfusepath{fill}%
\end{pgfscope}%
\begin{pgfscope}%
\pgfpathrectangle{\pgfqpoint{0.150000in}{0.150000in}}{\pgfqpoint{4.700000in}{3.450000in}}%
\pgfusepath{clip}%
\pgfsetbuttcap%
\pgfsetroundjoin%
\definecolor{currentfill}{rgb}{0.772993,0.800950,0.840089}%
\pgfsetfillcolor{currentfill}%
\pgfsetlinewidth{0.000000pt}%
\definecolor{currentstroke}{rgb}{0.000000,0.000000,0.000000}%
\pgfsetstrokecolor{currentstroke}%
\pgfsetdash{}{0pt}%
\pgfpathmoveto{\pgfqpoint{2.176816in}{2.541532in}}%
\pgfpathlineto{\pgfqpoint{2.242372in}{2.542007in}}%
\pgfpathlineto{\pgfqpoint{2.178198in}{2.605982in}}%
\pgfpathlineto{\pgfqpoint{2.112429in}{2.605727in}}%
\pgfpathclose%
\pgfusepath{fill}%
\end{pgfscope}%
\begin{pgfscope}%
\pgfpathrectangle{\pgfqpoint{0.150000in}{0.150000in}}{\pgfqpoint{4.700000in}{3.450000in}}%
\pgfusepath{clip}%
\pgfsetbuttcap%
\pgfsetroundjoin%
\definecolor{currentfill}{rgb}{0.810309,0.833670,0.866376}%
\pgfsetfillcolor{currentfill}%
\pgfsetlinewidth{0.000000pt}%
\definecolor{currentstroke}{rgb}{0.000000,0.000000,0.000000}%
\pgfsetstrokecolor{currentstroke}%
\pgfsetdash{}{0pt}%
\pgfpathmoveto{\pgfqpoint{2.241220in}{2.477319in}}%
\pgfpathlineto{\pgfqpoint{2.306565in}{2.478015in}}%
\pgfpathlineto{\pgfqpoint{2.242372in}{2.542007in}}%
\pgfpathlineto{\pgfqpoint{2.176816in}{2.541532in}}%
\pgfpathclose%
\pgfusepath{fill}%
\end{pgfscope}%
\begin{pgfscope}%
\pgfpathrectangle{\pgfqpoint{0.150000in}{0.150000in}}{\pgfqpoint{4.700000in}{3.450000in}}%
\pgfusepath{clip}%
\pgfsetbuttcap%
\pgfsetroundjoin%
\definecolor{currentfill}{rgb}{0.716973,0.486443,0.504703}%
\pgfsetfillcolor{currentfill}%
\pgfsetlinewidth{0.000000pt}%
\definecolor{currentstroke}{rgb}{0.000000,0.000000,0.000000}%
\pgfsetstrokecolor{currentstroke}%
\pgfsetdash{}{0pt}%
\pgfpathmoveto{\pgfqpoint{3.531201in}{1.421843in}}%
\pgfpathlineto{\pgfqpoint{3.591780in}{1.419387in}}%
\pgfpathlineto{\pgfqpoint{3.526201in}{1.461300in}}%
\pgfpathlineto{\pgfqpoint{3.465466in}{1.464147in}}%
\pgfpathclose%
\pgfusepath{fill}%
\end{pgfscope}%
\begin{pgfscope}%
\pgfpathrectangle{\pgfqpoint{0.150000in}{0.150000in}}{\pgfqpoint{4.700000in}{3.450000in}}%
\pgfusepath{clip}%
\pgfsetbuttcap%
\pgfsetroundjoin%
\definecolor{currentfill}{rgb}{0.847626,0.866391,0.892662}%
\pgfsetfillcolor{currentfill}%
\pgfsetlinewidth{0.000000pt}%
\definecolor{currentstroke}{rgb}{0.000000,0.000000,0.000000}%
\pgfsetstrokecolor{currentstroke}%
\pgfsetdash{}{0pt}%
\pgfpathmoveto{\pgfqpoint{2.305643in}{2.413089in}}%
\pgfpathlineto{\pgfqpoint{2.370775in}{2.414005in}}%
\pgfpathlineto{\pgfqpoint{2.306565in}{2.478015in}}%
\pgfpathlineto{\pgfqpoint{2.241220in}{2.477319in}}%
\pgfpathclose%
\pgfusepath{fill}%
\end{pgfscope}%
\begin{pgfscope}%
\pgfpathrectangle{\pgfqpoint{0.150000in}{0.150000in}}{\pgfqpoint{4.700000in}{3.450000in}}%
\pgfusepath{clip}%
\pgfsetbuttcap%
\pgfsetroundjoin%
\definecolor{currentfill}{rgb}{0.884942,0.899112,0.918949}%
\pgfsetfillcolor{currentfill}%
\pgfsetlinewidth{0.000000pt}%
\definecolor{currentstroke}{rgb}{0.000000,0.000000,0.000000}%
\pgfsetstrokecolor{currentstroke}%
\pgfsetdash{}{0pt}%
\pgfpathmoveto{\pgfqpoint{2.370083in}{2.348840in}}%
\pgfpathlineto{\pgfqpoint{2.435003in}{2.349978in}}%
\pgfpathlineto{\pgfqpoint{2.370775in}{2.414005in}}%
\pgfpathlineto{\pgfqpoint{2.305643in}{2.413089in}}%
\pgfpathclose%
\pgfusepath{fill}%
\end{pgfscope}%
\begin{pgfscope}%
\pgfpathrectangle{\pgfqpoint{0.150000in}{0.150000in}}{\pgfqpoint{4.700000in}{3.450000in}}%
\pgfusepath{clip}%
\pgfsetbuttcap%
\pgfsetroundjoin%
\definecolor{currentfill}{rgb}{0.671385,0.403722,0.424923}%
\pgfsetfillcolor{currentfill}%
\pgfsetlinewidth{0.000000pt}%
\definecolor{currentstroke}{rgb}{0.000000,0.000000,0.000000}%
\pgfsetstrokecolor{currentstroke}%
\pgfsetdash{}{0pt}%
\pgfpathmoveto{\pgfqpoint{3.789588in}{1.292843in}}%
\pgfpathlineto{\pgfqpoint{3.849239in}{1.291459in}}%
\pgfpathlineto{\pgfqpoint{3.783281in}{1.333443in}}%
\pgfpathlineto{\pgfqpoint{3.723469in}{1.335101in}}%
\pgfpathclose%
\pgfusepath{fill}%
\end{pgfscope}%
\begin{pgfscope}%
\pgfpathrectangle{\pgfqpoint{0.150000in}{0.150000in}}{\pgfqpoint{4.700000in}{3.450000in}}%
\pgfusepath{clip}%
\pgfsetbuttcap%
\pgfsetroundjoin%
\definecolor{currentfill}{rgb}{0.922258,0.931832,0.945236}%
\pgfsetfillcolor{currentfill}%
\pgfsetlinewidth{0.000000pt}%
\definecolor{currentstroke}{rgb}{0.000000,0.000000,0.000000}%
\pgfsetstrokecolor{currentstroke}%
\pgfsetdash{}{0pt}%
\pgfpathmoveto{\pgfqpoint{2.434542in}{2.284573in}}%
\pgfpathlineto{\pgfqpoint{2.499249in}{2.285932in}}%
\pgfpathlineto{\pgfqpoint{2.435003in}{2.349978in}}%
\pgfpathlineto{\pgfqpoint{2.370083in}{2.348840in}}%
\pgfpathclose%
\pgfusepath{fill}%
\end{pgfscope}%
\begin{pgfscope}%
\pgfpathrectangle{\pgfqpoint{0.150000in}{0.150000in}}{\pgfqpoint{4.700000in}{3.450000in}}%
\pgfusepath{clip}%
\pgfsetbuttcap%
\pgfsetroundjoin%
\definecolor{currentfill}{rgb}{0.959574,0.964553,0.971523}%
\pgfsetfillcolor{currentfill}%
\pgfsetlinewidth{0.000000pt}%
\definecolor{currentstroke}{rgb}{0.000000,0.000000,0.000000}%
\pgfsetstrokecolor{currentstroke}%
\pgfsetdash{}{0pt}%
\pgfpathmoveto{\pgfqpoint{2.499019in}{2.220288in}}%
\pgfpathlineto{\pgfqpoint{2.563514in}{2.221868in}}%
\pgfpathlineto{\pgfqpoint{2.499249in}{2.285932in}}%
\pgfpathlineto{\pgfqpoint{2.434542in}{2.284573in}}%
\pgfpathclose%
\pgfusepath{fill}%
\end{pgfscope}%
\begin{pgfscope}%
\pgfpathrectangle{\pgfqpoint{0.150000in}{0.150000in}}{\pgfqpoint{4.700000in}{3.450000in}}%
\pgfusepath{clip}%
\pgfsetbuttcap%
\pgfsetroundjoin%
\definecolor{currentfill}{rgb}{0.996890,0.997273,0.997809}%
\pgfsetfillcolor{currentfill}%
\pgfsetlinewidth{0.000000pt}%
\definecolor{currentstroke}{rgb}{0.000000,0.000000,0.000000}%
\pgfsetstrokecolor{currentstroke}%
\pgfsetdash{}{0pt}%
\pgfpathmoveto{\pgfqpoint{2.563514in}{2.155985in}}%
\pgfpathlineto{\pgfqpoint{2.627796in}{2.157787in}}%
\pgfpathlineto{\pgfqpoint{2.563514in}{2.221868in}}%
\pgfpathlineto{\pgfqpoint{2.499019in}{2.220288in}}%
\pgfpathclose%
\pgfusepath{fill}%
\end{pgfscope}%
\begin{pgfscope}%
\pgfpathrectangle{\pgfqpoint{0.150000in}{0.150000in}}{\pgfqpoint{4.700000in}{3.450000in}}%
\pgfusepath{clip}%
\pgfsetbuttcap%
\pgfsetroundjoin%
\definecolor{currentfill}{rgb}{0.975306,0.955193,0.956786}%
\pgfsetfillcolor{currentfill}%
\pgfsetlinewidth{0.000000pt}%
\definecolor{currentstroke}{rgb}{0.000000,0.000000,0.000000}%
\pgfsetstrokecolor{currentstroke}%
\pgfsetdash{}{0pt}%
\pgfpathmoveto{\pgfqpoint{2.628026in}{2.091664in}}%
\pgfpathlineto{\pgfqpoint{2.692096in}{2.093687in}}%
\pgfpathlineto{\pgfqpoint{2.627796in}{2.157787in}}%
\pgfpathlineto{\pgfqpoint{2.563514in}{2.155985in}}%
\pgfpathclose%
\pgfusepath{fill}%
\end{pgfscope}%
\begin{pgfscope}%
\pgfpathrectangle{\pgfqpoint{0.150000in}{0.150000in}}{\pgfqpoint{4.700000in}{3.450000in}}%
\pgfusepath{clip}%
\pgfsetbuttcap%
\pgfsetroundjoin%
\definecolor{currentfill}{rgb}{0.952512,0.913833,0.916896}%
\pgfsetfillcolor{currentfill}%
\pgfsetlinewidth{0.000000pt}%
\definecolor{currentstroke}{rgb}{0.000000,0.000000,0.000000}%
\pgfsetstrokecolor{currentstroke}%
\pgfsetdash{}{0pt}%
\pgfpathmoveto{\pgfqpoint{2.692558in}{2.027325in}}%
\pgfpathlineto{\pgfqpoint{2.756414in}{2.029570in}}%
\pgfpathlineto{\pgfqpoint{2.692096in}{2.093687in}}%
\pgfpathlineto{\pgfqpoint{2.628026in}{2.091664in}}%
\pgfpathclose%
\pgfusepath{fill}%
\end{pgfscope}%
\begin{pgfscope}%
\pgfpathrectangle{\pgfqpoint{0.150000in}{0.150000in}}{\pgfqpoint{4.700000in}{3.450000in}}%
\pgfusepath{clip}%
\pgfsetbuttcap%
\pgfsetroundjoin%
\definecolor{currentfill}{rgb}{0.929718,0.872472,0.877007}%
\pgfsetfillcolor{currentfill}%
\pgfsetlinewidth{0.000000pt}%
\definecolor{currentstroke}{rgb}{0.000000,0.000000,0.000000}%
\pgfsetstrokecolor{currentstroke}%
\pgfsetdash{}{0pt}%
\pgfpathmoveto{\pgfqpoint{2.757107in}{1.962968in}}%
\pgfpathlineto{\pgfqpoint{2.820750in}{1.965434in}}%
\pgfpathlineto{\pgfqpoint{2.756414in}{2.029570in}}%
\pgfpathlineto{\pgfqpoint{2.692558in}{2.027325in}}%
\pgfpathclose%
\pgfusepath{fill}%
\end{pgfscope}%
\begin{pgfscope}%
\pgfpathrectangle{\pgfqpoint{0.150000in}{0.150000in}}{\pgfqpoint{4.700000in}{3.450000in}}%
\pgfusepath{clip}%
\pgfsetbuttcap%
\pgfsetroundjoin%
\definecolor{currentfill}{rgb}{0.758762,0.562270,0.577834}%
\pgfsetfillcolor{currentfill}%
\pgfsetlinewidth{0.000000pt}%
\definecolor{currentstroke}{rgb}{0.000000,0.000000,0.000000}%
\pgfsetstrokecolor{currentstroke}%
\pgfsetdash{}{0pt}%
\pgfpathmoveto{\pgfqpoint{3.338547in}{1.509601in}}%
\pgfpathlineto{\pgfqpoint{3.399908in}{1.506338in}}%
\pgfpathlineto{\pgfqpoint{3.334528in}{1.548414in}}%
\pgfpathlineto{\pgfqpoint{3.273003in}{1.551953in}}%
\pgfpathclose%
\pgfusepath{fill}%
\end{pgfscope}%
\begin{pgfscope}%
\pgfpathrectangle{\pgfqpoint{0.150000in}{0.150000in}}{\pgfqpoint{4.700000in}{3.450000in}}%
\pgfusepath{clip}%
\pgfsetbuttcap%
\pgfsetroundjoin%
\definecolor{currentfill}{rgb}{0.906924,0.831112,0.837117}%
\pgfsetfillcolor{currentfill}%
\pgfsetlinewidth{0.000000pt}%
\definecolor{currentstroke}{rgb}{0.000000,0.000000,0.000000}%
\pgfsetstrokecolor{currentstroke}%
\pgfsetdash{}{0pt}%
\pgfpathmoveto{\pgfqpoint{2.821674in}{1.898593in}}%
\pgfpathlineto{\pgfqpoint{2.885104in}{1.901281in}}%
\pgfpathlineto{\pgfqpoint{2.820750in}{1.965434in}}%
\pgfpathlineto{\pgfqpoint{2.757107in}{1.962968in}}%
\pgfpathclose%
\pgfusepath{fill}%
\end{pgfscope}%
\begin{pgfscope}%
\pgfpathrectangle{\pgfqpoint{0.150000in}{0.150000in}}{\pgfqpoint{4.700000in}{3.450000in}}%
\pgfusepath{clip}%
\pgfsetbuttcap%
\pgfsetroundjoin%
\definecolor{currentfill}{rgb}{0.709375,0.472656,0.491406}%
\pgfsetfillcolor{currentfill}%
\pgfsetlinewidth{0.000000pt}%
\definecolor{currentstroke}{rgb}{0.000000,0.000000,0.000000}%
\pgfsetstrokecolor{currentstroke}%
\pgfsetdash{}{0pt}%
\pgfpathmoveto{\pgfqpoint{3.597115in}{1.379423in}}%
\pgfpathlineto{\pgfqpoint{3.657538in}{1.377360in}}%
\pgfpathlineto{\pgfqpoint{3.591780in}{1.419387in}}%
\pgfpathlineto{\pgfqpoint{3.531201in}{1.421843in}}%
\pgfpathclose%
\pgfusepath{fill}%
\end{pgfscope}%
\begin{pgfscope}%
\pgfpathrectangle{\pgfqpoint{0.150000in}{0.150000in}}{\pgfqpoint{4.700000in}{3.450000in}}%
\pgfusepath{clip}%
\pgfsetbuttcap%
\pgfsetroundjoin%
\definecolor{currentfill}{rgb}{0.884130,0.789752,0.797227}%
\pgfsetfillcolor{currentfill}%
\pgfsetlinewidth{0.000000pt}%
\definecolor{currentstroke}{rgb}{0.000000,0.000000,0.000000}%
\pgfsetstrokecolor{currentstroke}%
\pgfsetdash{}{0pt}%
\pgfpathmoveto{\pgfqpoint{2.886259in}{1.834200in}}%
\pgfpathlineto{\pgfqpoint{2.949476in}{1.837110in}}%
\pgfpathlineto{\pgfqpoint{2.885104in}{1.901281in}}%
\pgfpathlineto{\pgfqpoint{2.821674in}{1.898593in}}%
\pgfpathclose%
\pgfusepath{fill}%
\end{pgfscope}%
\begin{pgfscope}%
\pgfpathrectangle{\pgfqpoint{0.150000in}{0.150000in}}{\pgfqpoint{4.700000in}{3.450000in}}%
\pgfusepath{clip}%
\pgfsetbuttcap%
\pgfsetroundjoin%
\definecolor{currentfill}{rgb}{0.861336,0.748392,0.757338}%
\pgfsetfillcolor{currentfill}%
\pgfsetlinewidth{0.000000pt}%
\definecolor{currentstroke}{rgb}{0.000000,0.000000,0.000000}%
\pgfsetstrokecolor{currentstroke}%
\pgfsetdash{}{0pt}%
\pgfpathmoveto{\pgfqpoint{2.950863in}{1.769789in}}%
\pgfpathlineto{\pgfqpoint{3.013866in}{1.772921in}}%
\pgfpathlineto{\pgfqpoint{2.949476in}{1.837110in}}%
\pgfpathlineto{\pgfqpoint{2.886259in}{1.834200in}}%
\pgfpathclose%
\pgfusepath{fill}%
\end{pgfscope}%
\begin{pgfscope}%
\pgfpathrectangle{\pgfqpoint{0.150000in}{0.150000in}}{\pgfqpoint{4.700000in}{3.450000in}}%
\pgfusepath{clip}%
\pgfsetbuttcap%
\pgfsetroundjoin%
\definecolor{currentfill}{rgb}{0.838542,0.707031,0.717448}%
\pgfsetfillcolor{currentfill}%
\pgfsetlinewidth{0.000000pt}%
\definecolor{currentstroke}{rgb}{0.000000,0.000000,0.000000}%
\pgfsetstrokecolor{currentstroke}%
\pgfsetdash{}{0pt}%
\pgfpathmoveto{\pgfqpoint{3.015485in}{1.705359in}}%
\pgfpathlineto{\pgfqpoint{3.078274in}{1.708713in}}%
\pgfpathlineto{\pgfqpoint{3.013866in}{1.772921in}}%
\pgfpathlineto{\pgfqpoint{2.950863in}{1.769789in}}%
\pgfpathclose%
\pgfusepath{fill}%
\end{pgfscope}%
\begin{pgfscope}%
\pgfpathrectangle{\pgfqpoint{0.150000in}{0.150000in}}{\pgfqpoint{4.700000in}{3.450000in}}%
\pgfusepath{clip}%
\pgfsetbuttcap%
\pgfsetroundjoin%
\definecolor{currentfill}{rgb}{0.815748,0.665671,0.677558}%
\pgfsetfillcolor{currentfill}%
\pgfsetlinewidth{0.000000pt}%
\definecolor{currentstroke}{rgb}{0.000000,0.000000,0.000000}%
\pgfsetstrokecolor{currentstroke}%
\pgfsetdash{}{0pt}%
\pgfpathmoveto{\pgfqpoint{3.080124in}{1.640912in}}%
\pgfpathlineto{\pgfqpoint{3.142700in}{1.644488in}}%
\pgfpathlineto{\pgfqpoint{3.078274in}{1.708713in}}%
\pgfpathlineto{\pgfqpoint{3.015485in}{1.705359in}}%
\pgfpathclose%
\pgfusepath{fill}%
\end{pgfscope}%
\begin{pgfscope}%
\pgfpathrectangle{\pgfqpoint{0.150000in}{0.150000in}}{\pgfqpoint{4.700000in}{3.450000in}}%
\pgfusepath{clip}%
\pgfsetbuttcap%
\pgfsetroundjoin%
\definecolor{currentfill}{rgb}{0.800551,0.638097,0.650965}%
\pgfsetfillcolor{currentfill}%
\pgfsetlinewidth{0.000000pt}%
\definecolor{currentstroke}{rgb}{0.000000,0.000000,0.000000}%
\pgfsetstrokecolor{currentstroke}%
\pgfsetdash{}{0pt}%
\pgfpathmoveto{\pgfqpoint{3.145476in}{1.598273in}}%
\pgfpathlineto{\pgfqpoint{3.207639in}{1.594310in}}%
\pgfpathlineto{\pgfqpoint{3.142700in}{1.644488in}}%
\pgfpathlineto{\pgfqpoint{3.080124in}{1.640912in}}%
\pgfpathclose%
\pgfusepath{fill}%
\end{pgfscope}%
\begin{pgfscope}%
\pgfpathrectangle{\pgfqpoint{0.150000in}{0.150000in}}{\pgfqpoint{4.700000in}{3.450000in}}%
\pgfusepath{clip}%
\pgfsetbuttcap%
\pgfsetroundjoin%
\definecolor{currentfill}{rgb}{0.751164,0.548483,0.564537}%
\pgfsetfillcolor{currentfill}%
\pgfsetlinewidth{0.000000pt}%
\definecolor{currentstroke}{rgb}{0.000000,0.000000,0.000000}%
\pgfsetstrokecolor{currentstroke}%
\pgfsetdash{}{0pt}%
\pgfpathmoveto{\pgfqpoint{3.404264in}{1.467017in}}%
\pgfpathlineto{\pgfqpoint{3.465466in}{1.464147in}}%
\pgfpathlineto{\pgfqpoint{3.399908in}{1.506338in}}%
\pgfpathlineto{\pgfqpoint{3.338547in}{1.509601in}}%
\pgfpathclose%
\pgfusepath{fill}%
\end{pgfscope}%
\begin{pgfscope}%
\pgfpathrectangle{\pgfqpoint{0.150000in}{0.150000in}}{\pgfqpoint{4.700000in}{3.450000in}}%
\pgfusepath{clip}%
\pgfsetbuttcap%
\pgfsetroundjoin%
\definecolor{currentfill}{rgb}{0.705576,0.465763,0.484758}%
\pgfsetfillcolor{currentfill}%
\pgfsetlinewidth{0.000000pt}%
\definecolor{currentstroke}{rgb}{0.000000,0.000000,0.000000}%
\pgfsetstrokecolor{currentstroke}%
\pgfsetdash{}{0pt}%
\pgfpathmoveto{\pgfqpoint{3.663209in}{1.336888in}}%
\pgfpathlineto{\pgfqpoint{3.723469in}{1.335101in}}%
\pgfpathlineto{\pgfqpoint{3.657538in}{1.377360in}}%
\pgfpathlineto{\pgfqpoint{3.597115in}{1.379423in}}%
\pgfpathclose%
\pgfusepath{fill}%
\end{pgfscope}%
\begin{pgfscope}%
\pgfpathrectangle{\pgfqpoint{0.150000in}{0.150000in}}{\pgfqpoint{4.700000in}{3.450000in}}%
\pgfusepath{clip}%
\pgfsetbuttcap%
\pgfsetroundjoin%
\definecolor{currentfill}{rgb}{0.661045,0.702788,0.761229}%
\pgfsetfillcolor{currentfill}%
\pgfsetlinewidth{0.000000pt}%
\definecolor{currentstroke}{rgb}{0.000000,0.000000,0.000000}%
\pgfsetstrokecolor{currentstroke}%
\pgfsetdash{}{0pt}%
\pgfpathmoveto{\pgfqpoint{1.981624in}{2.669869in}}%
\pgfpathlineto{\pgfqpoint{2.048061in}{2.669904in}}%
\pgfpathlineto{\pgfqpoint{1.983711in}{2.734063in}}%
\pgfpathlineto{\pgfqpoint{1.917060in}{2.734250in}}%
\pgfpathclose%
\pgfusepath{fill}%
\end{pgfscope}%
\begin{pgfscope}%
\pgfpathrectangle{\pgfqpoint{0.150000in}{0.150000in}}{\pgfqpoint{4.700000in}{3.450000in}}%
\pgfusepath{clip}%
\pgfsetbuttcap%
\pgfsetroundjoin%
\definecolor{currentfill}{rgb}{0.698361,0.735509,0.787515}%
\pgfsetfillcolor{currentfill}%
\pgfsetlinewidth{0.000000pt}%
\definecolor{currentstroke}{rgb}{0.000000,0.000000,0.000000}%
\pgfsetstrokecolor{currentstroke}%
\pgfsetdash{}{0pt}%
\pgfpathmoveto{\pgfqpoint{2.046205in}{2.605471in}}%
\pgfpathlineto{\pgfqpoint{2.112429in}{2.605727in}}%
\pgfpathlineto{\pgfqpoint{2.048061in}{2.669904in}}%
\pgfpathlineto{\pgfqpoint{1.981624in}{2.669869in}}%
\pgfpathclose%
\pgfusepath{fill}%
\end{pgfscope}%
\begin{pgfscope}%
\pgfpathrectangle{\pgfqpoint{0.150000in}{0.150000in}}{\pgfqpoint{4.700000in}{3.450000in}}%
\pgfusepath{clip}%
\pgfsetbuttcap%
\pgfsetroundjoin%
\definecolor{currentfill}{rgb}{0.735677,0.768229,0.813802}%
\pgfsetfillcolor{currentfill}%
\pgfsetlinewidth{0.000000pt}%
\definecolor{currentstroke}{rgb}{0.000000,0.000000,0.000000}%
\pgfsetstrokecolor{currentstroke}%
\pgfsetdash{}{0pt}%
\pgfpathmoveto{\pgfqpoint{2.110805in}{2.541054in}}%
\pgfpathlineto{\pgfqpoint{2.176816in}{2.541532in}}%
\pgfpathlineto{\pgfqpoint{2.112429in}{2.605727in}}%
\pgfpathlineto{\pgfqpoint{2.046205in}{2.605471in}}%
\pgfpathclose%
\pgfusepath{fill}%
\end{pgfscope}%
\begin{pgfscope}%
\pgfpathrectangle{\pgfqpoint{0.150000in}{0.150000in}}{\pgfqpoint{4.700000in}{3.450000in}}%
\pgfusepath{clip}%
\pgfsetbuttcap%
\pgfsetroundjoin%
\definecolor{currentfill}{rgb}{0.772993,0.800950,0.840089}%
\pgfsetfillcolor{currentfill}%
\pgfsetlinewidth{0.000000pt}%
\definecolor{currentstroke}{rgb}{0.000000,0.000000,0.000000}%
\pgfsetstrokecolor{currentstroke}%
\pgfsetdash{}{0pt}%
\pgfpathmoveto{\pgfqpoint{2.175423in}{2.476619in}}%
\pgfpathlineto{\pgfqpoint{2.241220in}{2.477319in}}%
\pgfpathlineto{\pgfqpoint{2.176816in}{2.541532in}}%
\pgfpathlineto{\pgfqpoint{2.110805in}{2.541054in}}%
\pgfpathclose%
\pgfusepath{fill}%
\end{pgfscope}%
\begin{pgfscope}%
\pgfpathrectangle{\pgfqpoint{0.150000in}{0.150000in}}{\pgfqpoint{4.700000in}{3.450000in}}%
\pgfusepath{clip}%
\pgfsetbuttcap%
\pgfsetroundjoin%
\definecolor{currentfill}{rgb}{0.810309,0.833670,0.866376}%
\pgfsetfillcolor{currentfill}%
\pgfsetlinewidth{0.000000pt}%
\definecolor{currentstroke}{rgb}{0.000000,0.000000,0.000000}%
\pgfsetstrokecolor{currentstroke}%
\pgfsetdash{}{0pt}%
\pgfpathmoveto{\pgfqpoint{2.240060in}{2.412165in}}%
\pgfpathlineto{\pgfqpoint{2.305643in}{2.413089in}}%
\pgfpathlineto{\pgfqpoint{2.241220in}{2.477319in}}%
\pgfpathlineto{\pgfqpoint{2.175423in}{2.476619in}}%
\pgfpathclose%
\pgfusepath{fill}%
\end{pgfscope}%
\begin{pgfscope}%
\pgfpathrectangle{\pgfqpoint{0.150000in}{0.150000in}}{\pgfqpoint{4.700000in}{3.450000in}}%
\pgfusepath{clip}%
\pgfsetbuttcap%
\pgfsetroundjoin%
\definecolor{currentfill}{rgb}{0.792953,0.624311,0.637669}%
\pgfsetfillcolor{currentfill}%
\pgfsetlinewidth{0.000000pt}%
\definecolor{currentstroke}{rgb}{0.000000,0.000000,0.000000}%
\pgfsetstrokecolor{currentstroke}%
\pgfsetdash{}{0pt}%
\pgfpathmoveto{\pgfqpoint{3.211005in}{1.555639in}}%
\pgfpathlineto{\pgfqpoint{3.273003in}{1.551953in}}%
\pgfpathlineto{\pgfqpoint{3.207639in}{1.594310in}}%
\pgfpathlineto{\pgfqpoint{3.145476in}{1.598273in}}%
\pgfpathclose%
\pgfusepath{fill}%
\end{pgfscope}%
\begin{pgfscope}%
\pgfpathrectangle{\pgfqpoint{0.150000in}{0.150000in}}{\pgfqpoint{4.700000in}{3.450000in}}%
\pgfusepath{clip}%
\pgfsetbuttcap%
\pgfsetroundjoin%
\definecolor{currentfill}{rgb}{0.847626,0.866391,0.892662}%
\pgfsetfillcolor{currentfill}%
\pgfsetlinewidth{0.000000pt}%
\definecolor{currentstroke}{rgb}{0.000000,0.000000,0.000000}%
\pgfsetstrokecolor{currentstroke}%
\pgfsetdash{}{0pt}%
\pgfpathmoveto{\pgfqpoint{2.304714in}{2.347694in}}%
\pgfpathlineto{\pgfqpoint{2.370083in}{2.348840in}}%
\pgfpathlineto{\pgfqpoint{2.305643in}{2.413089in}}%
\pgfpathlineto{\pgfqpoint{2.240060in}{2.412165in}}%
\pgfpathclose%
\pgfusepath{fill}%
\end{pgfscope}%
\begin{pgfscope}%
\pgfpathrectangle{\pgfqpoint{0.150000in}{0.150000in}}{\pgfqpoint{4.700000in}{3.450000in}}%
\pgfusepath{clip}%
\pgfsetbuttcap%
\pgfsetroundjoin%
\definecolor{currentfill}{rgb}{0.743566,0.534697,0.551241}%
\pgfsetfillcolor{currentfill}%
\pgfsetlinewidth{0.000000pt}%
\definecolor{currentstroke}{rgb}{0.000000,0.000000,0.000000}%
\pgfsetstrokecolor{currentstroke}%
\pgfsetdash{}{0pt}%
\pgfpathmoveto{\pgfqpoint{3.470158in}{1.424317in}}%
\pgfpathlineto{\pgfqpoint{3.531201in}{1.421843in}}%
\pgfpathlineto{\pgfqpoint{3.465466in}{1.464147in}}%
\pgfpathlineto{\pgfqpoint{3.404264in}{1.467017in}}%
\pgfpathclose%
\pgfusepath{fill}%
\end{pgfscope}%
\begin{pgfscope}%
\pgfpathrectangle{\pgfqpoint{0.150000in}{0.150000in}}{\pgfqpoint{4.700000in}{3.450000in}}%
\pgfusepath{clip}%
\pgfsetbuttcap%
\pgfsetroundjoin%
\definecolor{currentfill}{rgb}{0.884942,0.899112,0.918949}%
\pgfsetfillcolor{currentfill}%
\pgfsetlinewidth{0.000000pt}%
\definecolor{currentstroke}{rgb}{0.000000,0.000000,0.000000}%
\pgfsetstrokecolor{currentstroke}%
\pgfsetdash{}{0pt}%
\pgfpathmoveto{\pgfqpoint{2.369386in}{2.283205in}}%
\pgfpathlineto{\pgfqpoint{2.434542in}{2.284573in}}%
\pgfpathlineto{\pgfqpoint{2.370083in}{2.348840in}}%
\pgfpathlineto{\pgfqpoint{2.304714in}{2.347694in}}%
\pgfpathclose%
\pgfusepath{fill}%
\end{pgfscope}%
\begin{pgfscope}%
\pgfpathrectangle{\pgfqpoint{0.150000in}{0.150000in}}{\pgfqpoint{4.700000in}{3.450000in}}%
\pgfusepath{clip}%
\pgfsetbuttcap%
\pgfsetroundjoin%
\definecolor{currentfill}{rgb}{0.922258,0.931832,0.945236}%
\pgfsetfillcolor{currentfill}%
\pgfsetlinewidth{0.000000pt}%
\definecolor{currentstroke}{rgb}{0.000000,0.000000,0.000000}%
\pgfsetstrokecolor{currentstroke}%
\pgfsetdash{}{0pt}%
\pgfpathmoveto{\pgfqpoint{2.434077in}{2.218697in}}%
\pgfpathlineto{\pgfqpoint{2.499019in}{2.220288in}}%
\pgfpathlineto{\pgfqpoint{2.434542in}{2.284573in}}%
\pgfpathlineto{\pgfqpoint{2.369386in}{2.283205in}}%
\pgfpathclose%
\pgfusepath{fill}%
\end{pgfscope}%
\begin{pgfscope}%
\pgfpathrectangle{\pgfqpoint{0.150000in}{0.150000in}}{\pgfqpoint{4.700000in}{3.450000in}}%
\pgfusepath{clip}%
\pgfsetbuttcap%
\pgfsetroundjoin%
\definecolor{currentfill}{rgb}{0.697978,0.451976,0.471461}%
\pgfsetfillcolor{currentfill}%
\pgfsetlinewidth{0.000000pt}%
\definecolor{currentstroke}{rgb}{0.000000,0.000000,0.000000}%
\pgfsetstrokecolor{currentstroke}%
\pgfsetdash{}{0pt}%
\pgfpathmoveto{\pgfqpoint{3.729482in}{1.294238in}}%
\pgfpathlineto{\pgfqpoint{3.789588in}{1.292843in}}%
\pgfpathlineto{\pgfqpoint{3.723469in}{1.335101in}}%
\pgfpathlineto{\pgfqpoint{3.663209in}{1.336888in}}%
\pgfpathclose%
\pgfusepath{fill}%
\end{pgfscope}%
\begin{pgfscope}%
\pgfpathrectangle{\pgfqpoint{0.150000in}{0.150000in}}{\pgfqpoint{4.700000in}{3.450000in}}%
\pgfusepath{clip}%
\pgfsetbuttcap%
\pgfsetroundjoin%
\definecolor{currentfill}{rgb}{0.959574,0.964553,0.971523}%
\pgfsetfillcolor{currentfill}%
\pgfsetlinewidth{0.000000pt}%
\definecolor{currentstroke}{rgb}{0.000000,0.000000,0.000000}%
\pgfsetstrokecolor{currentstroke}%
\pgfsetdash{}{0pt}%
\pgfpathmoveto{\pgfqpoint{2.498786in}{2.154171in}}%
\pgfpathlineto{\pgfqpoint{2.563514in}{2.155985in}}%
\pgfpathlineto{\pgfqpoint{2.499019in}{2.220288in}}%
\pgfpathlineto{\pgfqpoint{2.434077in}{2.218697in}}%
\pgfpathclose%
\pgfusepath{fill}%
\end{pgfscope}%
\begin{pgfscope}%
\pgfpathrectangle{\pgfqpoint{0.150000in}{0.150000in}}{\pgfqpoint{4.700000in}{3.450000in}}%
\pgfusepath{clip}%
\pgfsetbuttcap%
\pgfsetroundjoin%
\definecolor{currentfill}{rgb}{0.996890,0.997273,0.997809}%
\pgfsetfillcolor{currentfill}%
\pgfsetlinewidth{0.000000pt}%
\definecolor{currentstroke}{rgb}{0.000000,0.000000,0.000000}%
\pgfsetstrokecolor{currentstroke}%
\pgfsetdash{}{0pt}%
\pgfpathmoveto{\pgfqpoint{2.563514in}{2.089627in}}%
\pgfpathlineto{\pgfqpoint{2.628026in}{2.091664in}}%
\pgfpathlineto{\pgfqpoint{2.563514in}{2.155985in}}%
\pgfpathlineto{\pgfqpoint{2.498786in}{2.154171in}}%
\pgfpathclose%
\pgfusepath{fill}%
\end{pgfscope}%
\begin{pgfscope}%
\pgfpathrectangle{\pgfqpoint{0.150000in}{0.150000in}}{\pgfqpoint{4.700000in}{3.450000in}}%
\pgfusepath{clip}%
\pgfsetbuttcap%
\pgfsetroundjoin%
\definecolor{currentfill}{rgb}{0.975306,0.955193,0.956786}%
\pgfsetfillcolor{currentfill}%
\pgfsetlinewidth{0.000000pt}%
\definecolor{currentstroke}{rgb}{0.000000,0.000000,0.000000}%
\pgfsetstrokecolor{currentstroke}%
\pgfsetdash{}{0pt}%
\pgfpathmoveto{\pgfqpoint{2.628259in}{2.025065in}}%
\pgfpathlineto{\pgfqpoint{2.692558in}{2.027325in}}%
\pgfpathlineto{\pgfqpoint{2.628026in}{2.091664in}}%
\pgfpathlineto{\pgfqpoint{2.563514in}{2.089627in}}%
\pgfpathclose%
\pgfusepath{fill}%
\end{pgfscope}%
\begin{pgfscope}%
\pgfpathrectangle{\pgfqpoint{0.150000in}{0.150000in}}{\pgfqpoint{4.700000in}{3.450000in}}%
\pgfusepath{clip}%
\pgfsetbuttcap%
\pgfsetroundjoin%
\definecolor{currentfill}{rgb}{0.952512,0.913833,0.916896}%
\pgfsetfillcolor{currentfill}%
\pgfsetlinewidth{0.000000pt}%
\definecolor{currentstroke}{rgb}{0.000000,0.000000,0.000000}%
\pgfsetstrokecolor{currentstroke}%
\pgfsetdash{}{0pt}%
\pgfpathmoveto{\pgfqpoint{2.693023in}{1.960485in}}%
\pgfpathlineto{\pgfqpoint{2.757107in}{1.962968in}}%
\pgfpathlineto{\pgfqpoint{2.692558in}{2.027325in}}%
\pgfpathlineto{\pgfqpoint{2.628259in}{2.025065in}}%
\pgfpathclose%
\pgfusepath{fill}%
\end{pgfscope}%
\begin{pgfscope}%
\pgfpathrectangle{\pgfqpoint{0.150000in}{0.150000in}}{\pgfqpoint{4.700000in}{3.450000in}}%
\pgfusepath{clip}%
\pgfsetbuttcap%
\pgfsetroundjoin%
\definecolor{currentfill}{rgb}{0.929718,0.872472,0.877007}%
\pgfsetfillcolor{currentfill}%
\pgfsetlinewidth{0.000000pt}%
\definecolor{currentstroke}{rgb}{0.000000,0.000000,0.000000}%
\pgfsetstrokecolor{currentstroke}%
\pgfsetdash{}{0pt}%
\pgfpathmoveto{\pgfqpoint{2.757805in}{1.895887in}}%
\pgfpathlineto{\pgfqpoint{2.821674in}{1.898593in}}%
\pgfpathlineto{\pgfqpoint{2.757107in}{1.962968in}}%
\pgfpathlineto{\pgfqpoint{2.693023in}{1.960485in}}%
\pgfpathclose%
\pgfusepath{fill}%
\end{pgfscope}%
\begin{pgfscope}%
\pgfpathrectangle{\pgfqpoint{0.150000in}{0.150000in}}{\pgfqpoint{4.700000in}{3.450000in}}%
\pgfusepath{clip}%
\pgfsetbuttcap%
\pgfsetroundjoin%
\definecolor{currentfill}{rgb}{0.906924,0.831112,0.837117}%
\pgfsetfillcolor{currentfill}%
\pgfsetlinewidth{0.000000pt}%
\definecolor{currentstroke}{rgb}{0.000000,0.000000,0.000000}%
\pgfsetstrokecolor{currentstroke}%
\pgfsetdash{}{0pt}%
\pgfpathmoveto{\pgfqpoint{2.822605in}{1.831270in}}%
\pgfpathlineto{\pgfqpoint{2.886259in}{1.834200in}}%
\pgfpathlineto{\pgfqpoint{2.821674in}{1.898593in}}%
\pgfpathlineto{\pgfqpoint{2.757805in}{1.895887in}}%
\pgfpathclose%
\pgfusepath{fill}%
\end{pgfscope}%
\begin{pgfscope}%
\pgfpathrectangle{\pgfqpoint{0.150000in}{0.150000in}}{\pgfqpoint{4.700000in}{3.450000in}}%
\pgfusepath{clip}%
\pgfsetbuttcap%
\pgfsetroundjoin%
\definecolor{currentfill}{rgb}{0.884130,0.789752,0.797227}%
\pgfsetfillcolor{currentfill}%
\pgfsetlinewidth{0.000000pt}%
\definecolor{currentstroke}{rgb}{0.000000,0.000000,0.000000}%
\pgfsetstrokecolor{currentstroke}%
\pgfsetdash{}{0pt}%
\pgfpathmoveto{\pgfqpoint{2.887423in}{1.766635in}}%
\pgfpathlineto{\pgfqpoint{2.950863in}{1.769789in}}%
\pgfpathlineto{\pgfqpoint{2.886259in}{1.834200in}}%
\pgfpathlineto{\pgfqpoint{2.822605in}{1.831270in}}%
\pgfpathclose%
\pgfusepath{fill}%
\end{pgfscope}%
\begin{pgfscope}%
\pgfpathrectangle{\pgfqpoint{0.150000in}{0.150000in}}{\pgfqpoint{4.700000in}{3.450000in}}%
\pgfusepath{clip}%
\pgfsetbuttcap%
\pgfsetroundjoin%
\definecolor{currentfill}{rgb}{0.785355,0.610524,0.624372}%
\pgfsetfillcolor{currentfill}%
\pgfsetlinewidth{0.000000pt}%
\definecolor{currentstroke}{rgb}{0.000000,0.000000,0.000000}%
\pgfsetstrokecolor{currentstroke}%
\pgfsetdash{}{0pt}%
\pgfpathmoveto{\pgfqpoint{3.276707in}{1.512772in}}%
\pgfpathlineto{\pgfqpoint{3.338547in}{1.509601in}}%
\pgfpathlineto{\pgfqpoint{3.273003in}{1.551953in}}%
\pgfpathlineto{\pgfqpoint{3.211005in}{1.555639in}}%
\pgfpathclose%
\pgfusepath{fill}%
\end{pgfscope}%
\begin{pgfscope}%
\pgfpathrectangle{\pgfqpoint{0.150000in}{0.150000in}}{\pgfqpoint{4.700000in}{3.450000in}}%
\pgfusepath{clip}%
\pgfsetbuttcap%
\pgfsetroundjoin%
\definecolor{currentfill}{rgb}{0.861336,0.748392,0.757338}%
\pgfsetfillcolor{currentfill}%
\pgfsetlinewidth{0.000000pt}%
\definecolor{currentstroke}{rgb}{0.000000,0.000000,0.000000}%
\pgfsetstrokecolor{currentstroke}%
\pgfsetdash{}{0pt}%
\pgfpathmoveto{\pgfqpoint{2.952260in}{1.701982in}}%
\pgfpathlineto{\pgfqpoint{3.015485in}{1.705359in}}%
\pgfpathlineto{\pgfqpoint{2.950863in}{1.769789in}}%
\pgfpathlineto{\pgfqpoint{2.887423in}{1.766635in}}%
\pgfpathclose%
\pgfusepath{fill}%
\end{pgfscope}%
\begin{pgfscope}%
\pgfpathrectangle{\pgfqpoint{0.150000in}{0.150000in}}{\pgfqpoint{4.700000in}{3.450000in}}%
\pgfusepath{clip}%
\pgfsetbuttcap%
\pgfsetroundjoin%
\definecolor{currentfill}{rgb}{0.739767,0.527803,0.544593}%
\pgfsetfillcolor{currentfill}%
\pgfsetlinewidth{0.000000pt}%
\definecolor{currentstroke}{rgb}{0.000000,0.000000,0.000000}%
\pgfsetstrokecolor{currentstroke}%
\pgfsetdash{}{0pt}%
\pgfpathmoveto{\pgfqpoint{3.536229in}{1.381502in}}%
\pgfpathlineto{\pgfqpoint{3.597115in}{1.379423in}}%
\pgfpathlineto{\pgfqpoint{3.531201in}{1.421843in}}%
\pgfpathlineto{\pgfqpoint{3.470158in}{1.424317in}}%
\pgfpathclose%
\pgfusepath{fill}%
\end{pgfscope}%
\begin{pgfscope}%
\pgfpathrectangle{\pgfqpoint{0.150000in}{0.150000in}}{\pgfqpoint{4.700000in}{3.450000in}}%
\pgfusepath{clip}%
\pgfsetbuttcap%
\pgfsetroundjoin%
\definecolor{currentfill}{rgb}{0.838542,0.707031,0.717448}%
\pgfsetfillcolor{currentfill}%
\pgfsetlinewidth{0.000000pt}%
\definecolor{currentstroke}{rgb}{0.000000,0.000000,0.000000}%
\pgfsetstrokecolor{currentstroke}%
\pgfsetdash{}{0pt}%
\pgfpathmoveto{\pgfqpoint{3.017309in}{1.645186in}}%
\pgfpathlineto{\pgfqpoint{3.080124in}{1.640912in}}%
\pgfpathlineto{\pgfqpoint{3.015485in}{1.705359in}}%
\pgfpathlineto{\pgfqpoint{2.952260in}{1.701982in}}%
\pgfpathclose%
\pgfusepath{fill}%
\end{pgfscope}%
\begin{pgfscope}%
\pgfpathrectangle{\pgfqpoint{0.150000in}{0.150000in}}{\pgfqpoint{4.700000in}{3.450000in}}%
\pgfusepath{clip}%
\pgfsetbuttcap%
\pgfsetroundjoin%
\definecolor{currentfill}{rgb}{0.827145,0.686351,0.697503}%
\pgfsetfillcolor{currentfill}%
\pgfsetlinewidth{0.000000pt}%
\definecolor{currentstroke}{rgb}{0.000000,0.000000,0.000000}%
\pgfsetstrokecolor{currentstroke}%
\pgfsetdash{}{0pt}%
\pgfpathmoveto{\pgfqpoint{3.082828in}{1.602267in}}%
\pgfpathlineto{\pgfqpoint{3.145476in}{1.598273in}}%
\pgfpathlineto{\pgfqpoint{3.080124in}{1.640912in}}%
\pgfpathlineto{\pgfqpoint{3.017309in}{1.645186in}}%
\pgfpathclose%
\pgfusepath{fill}%
\end{pgfscope}%
\begin{pgfscope}%
\pgfpathrectangle{\pgfqpoint{0.150000in}{0.150000in}}{\pgfqpoint{4.700000in}{3.450000in}}%
\pgfusepath{clip}%
\pgfsetbuttcap%
\pgfsetroundjoin%
\definecolor{currentfill}{rgb}{0.777757,0.596737,0.611075}%
\pgfsetfillcolor{currentfill}%
\pgfsetlinewidth{0.000000pt}%
\definecolor{currentstroke}{rgb}{0.000000,0.000000,0.000000}%
\pgfsetstrokecolor{currentstroke}%
\pgfsetdash{}{0pt}%
\pgfpathmoveto{\pgfqpoint{3.342584in}{1.469790in}}%
\pgfpathlineto{\pgfqpoint{3.404264in}{1.467017in}}%
\pgfpathlineto{\pgfqpoint{3.338547in}{1.509601in}}%
\pgfpathlineto{\pgfqpoint{3.276707in}{1.512772in}}%
\pgfpathclose%
\pgfusepath{fill}%
\end{pgfscope}%
\begin{pgfscope}%
\pgfpathrectangle{\pgfqpoint{0.150000in}{0.150000in}}{\pgfqpoint{4.700000in}{3.450000in}}%
\pgfusepath{clip}%
\pgfsetbuttcap%
\pgfsetroundjoin%
\definecolor{currentfill}{rgb}{0.623729,0.670067,0.734942}%
\pgfsetfillcolor{currentfill}%
\pgfsetlinewidth{0.000000pt}%
\definecolor{currentstroke}{rgb}{0.000000,0.000000,0.000000}%
\pgfsetstrokecolor{currentstroke}%
\pgfsetdash{}{0pt}%
\pgfpathmoveto{\pgfqpoint{1.914726in}{2.669835in}}%
\pgfpathlineto{\pgfqpoint{1.981624in}{2.669869in}}%
\pgfpathlineto{\pgfqpoint{1.917060in}{2.734250in}}%
\pgfpathlineto{\pgfqpoint{1.849948in}{2.734439in}}%
\pgfpathclose%
\pgfusepath{fill}%
\end{pgfscope}%
\begin{pgfscope}%
\pgfpathrectangle{\pgfqpoint{0.150000in}{0.150000in}}{\pgfqpoint{4.700000in}{3.450000in}}%
\pgfusepath{clip}%
\pgfsetbuttcap%
\pgfsetroundjoin%
\definecolor{currentfill}{rgb}{0.732169,0.514017,0.531296}%
\pgfsetfillcolor{currentfill}%
\pgfsetlinewidth{0.000000pt}%
\definecolor{currentstroke}{rgb}{0.000000,0.000000,0.000000}%
\pgfsetstrokecolor{currentstroke}%
\pgfsetdash{}{0pt}%
\pgfpathmoveto{\pgfqpoint{3.602479in}{1.338572in}}%
\pgfpathlineto{\pgfqpoint{3.663209in}{1.336888in}}%
\pgfpathlineto{\pgfqpoint{3.597115in}{1.379423in}}%
\pgfpathlineto{\pgfqpoint{3.536229in}{1.381502in}}%
\pgfpathclose%
\pgfusepath{fill}%
\end{pgfscope}%
\begin{pgfscope}%
\pgfpathrectangle{\pgfqpoint{0.150000in}{0.150000in}}{\pgfqpoint{4.700000in}{3.450000in}}%
\pgfusepath{clip}%
\pgfsetbuttcap%
\pgfsetroundjoin%
\definecolor{currentfill}{rgb}{0.661045,0.702788,0.761229}%
\pgfsetfillcolor{currentfill}%
\pgfsetlinewidth{0.000000pt}%
\definecolor{currentstroke}{rgb}{0.000000,0.000000,0.000000}%
\pgfsetstrokecolor{currentstroke}%
\pgfsetdash{}{0pt}%
\pgfpathmoveto{\pgfqpoint{1.979522in}{2.605212in}}%
\pgfpathlineto{\pgfqpoint{2.046205in}{2.605471in}}%
\pgfpathlineto{\pgfqpoint{1.981624in}{2.669869in}}%
\pgfpathlineto{\pgfqpoint{1.914726in}{2.669835in}}%
\pgfpathclose%
\pgfusepath{fill}%
\end{pgfscope}%
\begin{pgfscope}%
\pgfpathrectangle{\pgfqpoint{0.150000in}{0.150000in}}{\pgfqpoint{4.700000in}{3.450000in}}%
\pgfusepath{clip}%
\pgfsetbuttcap%
\pgfsetroundjoin%
\definecolor{currentfill}{rgb}{0.698361,0.735509,0.787515}%
\pgfsetfillcolor{currentfill}%
\pgfsetlinewidth{0.000000pt}%
\definecolor{currentstroke}{rgb}{0.000000,0.000000,0.000000}%
\pgfsetstrokecolor{currentstroke}%
\pgfsetdash{}{0pt}%
\pgfpathmoveto{\pgfqpoint{2.044337in}{2.540572in}}%
\pgfpathlineto{\pgfqpoint{2.110805in}{2.541054in}}%
\pgfpathlineto{\pgfqpoint{2.046205in}{2.605471in}}%
\pgfpathlineto{\pgfqpoint{1.979522in}{2.605212in}}%
\pgfpathclose%
\pgfusepath{fill}%
\end{pgfscope}%
\begin{pgfscope}%
\pgfpathrectangle{\pgfqpoint{0.150000in}{0.150000in}}{\pgfqpoint{4.700000in}{3.450000in}}%
\pgfusepath{clip}%
\pgfsetbuttcap%
\pgfsetroundjoin%
\definecolor{currentfill}{rgb}{0.735677,0.768229,0.813802}%
\pgfsetfillcolor{currentfill}%
\pgfsetlinewidth{0.000000pt}%
\definecolor{currentstroke}{rgb}{0.000000,0.000000,0.000000}%
\pgfsetstrokecolor{currentstroke}%
\pgfsetdash{}{0pt}%
\pgfpathmoveto{\pgfqpoint{2.109170in}{2.475913in}}%
\pgfpathlineto{\pgfqpoint{2.175423in}{2.476619in}}%
\pgfpathlineto{\pgfqpoint{2.110805in}{2.541054in}}%
\pgfpathlineto{\pgfqpoint{2.044337in}{2.540572in}}%
\pgfpathclose%
\pgfusepath{fill}%
\end{pgfscope}%
\begin{pgfscope}%
\pgfpathrectangle{\pgfqpoint{0.150000in}{0.150000in}}{\pgfqpoint{4.700000in}{3.450000in}}%
\pgfusepath{clip}%
\pgfsetbuttcap%
\pgfsetroundjoin%
\definecolor{currentfill}{rgb}{0.772993,0.800950,0.840089}%
\pgfsetfillcolor{currentfill}%
\pgfsetlinewidth{0.000000pt}%
\definecolor{currentstroke}{rgb}{0.000000,0.000000,0.000000}%
\pgfsetstrokecolor{currentstroke}%
\pgfsetdash{}{0pt}%
\pgfpathmoveto{\pgfqpoint{2.174021in}{2.411236in}}%
\pgfpathlineto{\pgfqpoint{2.240060in}{2.412165in}}%
\pgfpathlineto{\pgfqpoint{2.175423in}{2.476619in}}%
\pgfpathlineto{\pgfqpoint{2.109170in}{2.475913in}}%
\pgfpathclose%
\pgfusepath{fill}%
\end{pgfscope}%
\begin{pgfscope}%
\pgfpathrectangle{\pgfqpoint{0.150000in}{0.150000in}}{\pgfqpoint{4.700000in}{3.450000in}}%
\pgfusepath{clip}%
\pgfsetbuttcap%
\pgfsetroundjoin%
\definecolor{currentfill}{rgb}{0.810309,0.833670,0.866376}%
\pgfsetfillcolor{currentfill}%
\pgfsetlinewidth{0.000000pt}%
\definecolor{currentstroke}{rgb}{0.000000,0.000000,0.000000}%
\pgfsetstrokecolor{currentstroke}%
\pgfsetdash{}{0pt}%
\pgfpathmoveto{\pgfqpoint{2.238891in}{2.346541in}}%
\pgfpathlineto{\pgfqpoint{2.304714in}{2.347694in}}%
\pgfpathlineto{\pgfqpoint{2.240060in}{2.412165in}}%
\pgfpathlineto{\pgfqpoint{2.174021in}{2.411236in}}%
\pgfpathclose%
\pgfusepath{fill}%
\end{pgfscope}%
\begin{pgfscope}%
\pgfpathrectangle{\pgfqpoint{0.150000in}{0.150000in}}{\pgfqpoint{4.700000in}{3.450000in}}%
\pgfusepath{clip}%
\pgfsetbuttcap%
\pgfsetroundjoin%
\definecolor{currentfill}{rgb}{0.847626,0.866391,0.892662}%
\pgfsetfillcolor{currentfill}%
\pgfsetlinewidth{0.000000pt}%
\definecolor{currentstroke}{rgb}{0.000000,0.000000,0.000000}%
\pgfsetstrokecolor{currentstroke}%
\pgfsetdash{}{0pt}%
\pgfpathmoveto{\pgfqpoint{2.303778in}{2.281827in}}%
\pgfpathlineto{\pgfqpoint{2.369386in}{2.283205in}}%
\pgfpathlineto{\pgfqpoint{2.304714in}{2.347694in}}%
\pgfpathlineto{\pgfqpoint{2.238891in}{2.346541in}}%
\pgfpathclose%
\pgfusepath{fill}%
\end{pgfscope}%
\begin{pgfscope}%
\pgfpathrectangle{\pgfqpoint{0.150000in}{0.150000in}}{\pgfqpoint{4.700000in}{3.450000in}}%
\pgfusepath{clip}%
\pgfsetbuttcap%
\pgfsetroundjoin%
\definecolor{currentfill}{rgb}{0.884942,0.899112,0.918949}%
\pgfsetfillcolor{currentfill}%
\pgfsetlinewidth{0.000000pt}%
\definecolor{currentstroke}{rgb}{0.000000,0.000000,0.000000}%
\pgfsetstrokecolor{currentstroke}%
\pgfsetdash{}{0pt}%
\pgfpathmoveto{\pgfqpoint{2.368685in}{2.217095in}}%
\pgfpathlineto{\pgfqpoint{2.434077in}{2.218697in}}%
\pgfpathlineto{\pgfqpoint{2.369386in}{2.283205in}}%
\pgfpathlineto{\pgfqpoint{2.303778in}{2.281827in}}%
\pgfpathclose%
\pgfusepath{fill}%
\end{pgfscope}%
\begin{pgfscope}%
\pgfpathrectangle{\pgfqpoint{0.150000in}{0.150000in}}{\pgfqpoint{4.700000in}{3.450000in}}%
\pgfusepath{clip}%
\pgfsetbuttcap%
\pgfsetroundjoin%
\definecolor{currentfill}{rgb}{0.819547,0.672564,0.684206}%
\pgfsetfillcolor{currentfill}%
\pgfsetlinewidth{0.000000pt}%
\definecolor{currentstroke}{rgb}{0.000000,0.000000,0.000000}%
\pgfsetstrokecolor{currentstroke}%
\pgfsetdash{}{0pt}%
\pgfpathmoveto{\pgfqpoint{3.148522in}{1.559234in}}%
\pgfpathlineto{\pgfqpoint{3.211005in}{1.555639in}}%
\pgfpathlineto{\pgfqpoint{3.145476in}{1.598273in}}%
\pgfpathlineto{\pgfqpoint{3.082828in}{1.602267in}}%
\pgfpathclose%
\pgfusepath{fill}%
\end{pgfscope}%
\begin{pgfscope}%
\pgfpathrectangle{\pgfqpoint{0.150000in}{0.150000in}}{\pgfqpoint{4.700000in}{3.450000in}}%
\pgfusepath{clip}%
\pgfsetbuttcap%
\pgfsetroundjoin%
\definecolor{currentfill}{rgb}{0.922258,0.931832,0.945236}%
\pgfsetfillcolor{currentfill}%
\pgfsetlinewidth{0.000000pt}%
\definecolor{currentstroke}{rgb}{0.000000,0.000000,0.000000}%
\pgfsetstrokecolor{currentstroke}%
\pgfsetdash{}{0pt}%
\pgfpathmoveto{\pgfqpoint{2.433609in}{2.152345in}}%
\pgfpathlineto{\pgfqpoint{2.498786in}{2.154171in}}%
\pgfpathlineto{\pgfqpoint{2.434077in}{2.218697in}}%
\pgfpathlineto{\pgfqpoint{2.368685in}{2.217095in}}%
\pgfpathclose%
\pgfusepath{fill}%
\end{pgfscope}%
\begin{pgfscope}%
\pgfpathrectangle{\pgfqpoint{0.150000in}{0.150000in}}{\pgfqpoint{4.700000in}{3.450000in}}%
\pgfusepath{clip}%
\pgfsetbuttcap%
\pgfsetroundjoin%
\definecolor{currentfill}{rgb}{0.773958,0.589844,0.604427}%
\pgfsetfillcolor{currentfill}%
\pgfsetlinewidth{0.000000pt}%
\definecolor{currentstroke}{rgb}{0.000000,0.000000,0.000000}%
\pgfsetstrokecolor{currentstroke}%
\pgfsetdash{}{0pt}%
\pgfpathmoveto{\pgfqpoint{3.408643in}{1.426811in}}%
\pgfpathlineto{\pgfqpoint{3.470158in}{1.424317in}}%
\pgfpathlineto{\pgfqpoint{3.404264in}{1.467017in}}%
\pgfpathlineto{\pgfqpoint{3.342584in}{1.469790in}}%
\pgfpathclose%
\pgfusepath{fill}%
\end{pgfscope}%
\begin{pgfscope}%
\pgfpathrectangle{\pgfqpoint{0.150000in}{0.150000in}}{\pgfqpoint{4.700000in}{3.450000in}}%
\pgfusepath{clip}%
\pgfsetbuttcap%
\pgfsetroundjoin%
\definecolor{currentfill}{rgb}{0.959574,0.964553,0.971523}%
\pgfsetfillcolor{currentfill}%
\pgfsetlinewidth{0.000000pt}%
\definecolor{currentstroke}{rgb}{0.000000,0.000000,0.000000}%
\pgfsetstrokecolor{currentstroke}%
\pgfsetdash{}{0pt}%
\pgfpathmoveto{\pgfqpoint{2.498552in}{2.087576in}}%
\pgfpathlineto{\pgfqpoint{2.563514in}{2.089627in}}%
\pgfpathlineto{\pgfqpoint{2.498786in}{2.154171in}}%
\pgfpathlineto{\pgfqpoint{2.433609in}{2.152345in}}%
\pgfpathclose%
\pgfusepath{fill}%
\end{pgfscope}%
\begin{pgfscope}%
\pgfpathrectangle{\pgfqpoint{0.150000in}{0.150000in}}{\pgfqpoint{4.700000in}{3.450000in}}%
\pgfusepath{clip}%
\pgfsetbuttcap%
\pgfsetroundjoin%
\definecolor{currentfill}{rgb}{0.996890,0.997273,0.997809}%
\pgfsetfillcolor{currentfill}%
\pgfsetlinewidth{0.000000pt}%
\definecolor{currentstroke}{rgb}{0.000000,0.000000,0.000000}%
\pgfsetstrokecolor{currentstroke}%
\pgfsetdash{}{0pt}%
\pgfpathmoveto{\pgfqpoint{2.563514in}{2.022790in}}%
\pgfpathlineto{\pgfqpoint{2.628259in}{2.025065in}}%
\pgfpathlineto{\pgfqpoint{2.563514in}{2.089627in}}%
\pgfpathlineto{\pgfqpoint{2.498552in}{2.087576in}}%
\pgfpathclose%
\pgfusepath{fill}%
\end{pgfscope}%
\begin{pgfscope}%
\pgfpathrectangle{\pgfqpoint{0.150000in}{0.150000in}}{\pgfqpoint{4.700000in}{3.450000in}}%
\pgfusepath{clip}%
\pgfsetbuttcap%
\pgfsetroundjoin%
\definecolor{currentfill}{rgb}{0.724571,0.500230,0.517999}%
\pgfsetfillcolor{currentfill}%
\pgfsetlinewidth{0.000000pt}%
\definecolor{currentstroke}{rgb}{0.000000,0.000000,0.000000}%
\pgfsetstrokecolor{currentstroke}%
\pgfsetdash{}{0pt}%
\pgfpathmoveto{\pgfqpoint{3.668909in}{1.295525in}}%
\pgfpathlineto{\pgfqpoint{3.729482in}{1.294238in}}%
\pgfpathlineto{\pgfqpoint{3.663209in}{1.336888in}}%
\pgfpathlineto{\pgfqpoint{3.602479in}{1.338572in}}%
\pgfpathclose%
\pgfusepath{fill}%
\end{pgfscope}%
\begin{pgfscope}%
\pgfpathrectangle{\pgfqpoint{0.150000in}{0.150000in}}{\pgfqpoint{4.700000in}{3.450000in}}%
\pgfusepath{clip}%
\pgfsetbuttcap%
\pgfsetroundjoin%
\definecolor{currentfill}{rgb}{0.975306,0.955193,0.956786}%
\pgfsetfillcolor{currentfill}%
\pgfsetlinewidth{0.000000pt}%
\definecolor{currentstroke}{rgb}{0.000000,0.000000,0.000000}%
\pgfsetstrokecolor{currentstroke}%
\pgfsetdash{}{0pt}%
\pgfpathmoveto{\pgfqpoint{2.628493in}{1.957985in}}%
\pgfpathlineto{\pgfqpoint{2.693023in}{1.960485in}}%
\pgfpathlineto{\pgfqpoint{2.628259in}{2.025065in}}%
\pgfpathlineto{\pgfqpoint{2.563514in}{2.022790in}}%
\pgfpathclose%
\pgfusepath{fill}%
\end{pgfscope}%
\begin{pgfscope}%
\pgfpathrectangle{\pgfqpoint{0.150000in}{0.150000in}}{\pgfqpoint{4.700000in}{3.450000in}}%
\pgfusepath{clip}%
\pgfsetbuttcap%
\pgfsetroundjoin%
\definecolor{currentfill}{rgb}{0.952512,0.913833,0.916896}%
\pgfsetfillcolor{currentfill}%
\pgfsetlinewidth{0.000000pt}%
\definecolor{currentstroke}{rgb}{0.000000,0.000000,0.000000}%
\pgfsetstrokecolor{currentstroke}%
\pgfsetdash{}{0pt}%
\pgfpathmoveto{\pgfqpoint{2.693491in}{1.893161in}}%
\pgfpathlineto{\pgfqpoint{2.757805in}{1.895887in}}%
\pgfpathlineto{\pgfqpoint{2.693023in}{1.960485in}}%
\pgfpathlineto{\pgfqpoint{2.628493in}{1.957985in}}%
\pgfpathclose%
\pgfusepath{fill}%
\end{pgfscope}%
\begin{pgfscope}%
\pgfpathrectangle{\pgfqpoint{0.150000in}{0.150000in}}{\pgfqpoint{4.700000in}{3.450000in}}%
\pgfusepath{clip}%
\pgfsetbuttcap%
\pgfsetroundjoin%
\definecolor{currentfill}{rgb}{0.929718,0.872472,0.877007}%
\pgfsetfillcolor{currentfill}%
\pgfsetlinewidth{0.000000pt}%
\definecolor{currentstroke}{rgb}{0.000000,0.000000,0.000000}%
\pgfsetstrokecolor{currentstroke}%
\pgfsetdash{}{0pt}%
\pgfpathmoveto{\pgfqpoint{2.758508in}{1.828319in}}%
\pgfpathlineto{\pgfqpoint{2.822605in}{1.831270in}}%
\pgfpathlineto{\pgfqpoint{2.757805in}{1.895887in}}%
\pgfpathlineto{\pgfqpoint{2.693491in}{1.893161in}}%
\pgfpathclose%
\pgfusepath{fill}%
\end{pgfscope}%
\begin{pgfscope}%
\pgfpathrectangle{\pgfqpoint{0.150000in}{0.150000in}}{\pgfqpoint{4.700000in}{3.450000in}}%
\pgfusepath{clip}%
\pgfsetbuttcap%
\pgfsetroundjoin%
\definecolor{currentfill}{rgb}{0.906924,0.831112,0.837117}%
\pgfsetfillcolor{currentfill}%
\pgfsetlinewidth{0.000000pt}%
\definecolor{currentstroke}{rgb}{0.000000,0.000000,0.000000}%
\pgfsetstrokecolor{currentstroke}%
\pgfsetdash{}{0pt}%
\pgfpathmoveto{\pgfqpoint{2.823542in}{1.763459in}}%
\pgfpathlineto{\pgfqpoint{2.887423in}{1.766635in}}%
\pgfpathlineto{\pgfqpoint{2.822605in}{1.831270in}}%
\pgfpathlineto{\pgfqpoint{2.758508in}{1.828319in}}%
\pgfpathclose%
\pgfusepath{fill}%
\end{pgfscope}%
\begin{pgfscope}%
\pgfpathrectangle{\pgfqpoint{0.150000in}{0.150000in}}{\pgfqpoint{4.700000in}{3.450000in}}%
\pgfusepath{clip}%
\pgfsetbuttcap%
\pgfsetroundjoin%
\definecolor{currentfill}{rgb}{0.884130,0.789752,0.797227}%
\pgfsetfillcolor{currentfill}%
\pgfsetlinewidth{0.000000pt}%
\definecolor{currentstroke}{rgb}{0.000000,0.000000,0.000000}%
\pgfsetstrokecolor{currentstroke}%
\pgfsetdash{}{0pt}%
\pgfpathmoveto{\pgfqpoint{2.888596in}{1.698581in}}%
\pgfpathlineto{\pgfqpoint{2.952260in}{1.701982in}}%
\pgfpathlineto{\pgfqpoint{2.887423in}{1.766635in}}%
\pgfpathlineto{\pgfqpoint{2.823542in}{1.763459in}}%
\pgfpathclose%
\pgfusepath{fill}%
\end{pgfscope}%
\begin{pgfscope}%
\pgfpathrectangle{\pgfqpoint{0.150000in}{0.150000in}}{\pgfqpoint{4.700000in}{3.450000in}}%
\pgfusepath{clip}%
\pgfsetbuttcap%
\pgfsetroundjoin%
\definecolor{currentfill}{rgb}{0.815748,0.665671,0.677558}%
\pgfsetfillcolor{currentfill}%
\pgfsetlinewidth{0.000000pt}%
\definecolor{currentstroke}{rgb}{0.000000,0.000000,0.000000}%
\pgfsetstrokecolor{currentstroke}%
\pgfsetdash{}{0pt}%
\pgfpathmoveto{\pgfqpoint{3.214386in}{1.515966in}}%
\pgfpathlineto{\pgfqpoint{3.276707in}{1.512772in}}%
\pgfpathlineto{\pgfqpoint{3.211005in}{1.555639in}}%
\pgfpathlineto{\pgfqpoint{3.148522in}{1.559234in}}%
\pgfpathclose%
\pgfusepath{fill}%
\end{pgfscope}%
\begin{pgfscope}%
\pgfpathrectangle{\pgfqpoint{0.150000in}{0.150000in}}{\pgfqpoint{4.700000in}{3.450000in}}%
\pgfusepath{clip}%
\pgfsetbuttcap%
\pgfsetroundjoin%
\definecolor{currentfill}{rgb}{0.766360,0.576057,0.591131}%
\pgfsetfillcolor{currentfill}%
\pgfsetlinewidth{0.000000pt}%
\definecolor{currentstroke}{rgb}{0.000000,0.000000,0.000000}%
\pgfsetstrokecolor{currentstroke}%
\pgfsetdash{}{0pt}%
\pgfpathmoveto{\pgfqpoint{3.474875in}{1.383598in}}%
\pgfpathlineto{\pgfqpoint{3.536229in}{1.381502in}}%
\pgfpathlineto{\pgfqpoint{3.470158in}{1.424317in}}%
\pgfpathlineto{\pgfqpoint{3.408643in}{1.426811in}}%
\pgfpathclose%
\pgfusepath{fill}%
\end{pgfscope}%
\begin{pgfscope}%
\pgfpathrectangle{\pgfqpoint{0.150000in}{0.150000in}}{\pgfqpoint{4.700000in}{3.450000in}}%
\pgfusepath{clip}%
\pgfsetbuttcap%
\pgfsetroundjoin%
\definecolor{currentfill}{rgb}{0.865135,0.755285,0.763986}%
\pgfsetfillcolor{currentfill}%
\pgfsetlinewidth{0.000000pt}%
\definecolor{currentstroke}{rgb}{0.000000,0.000000,0.000000}%
\pgfsetstrokecolor{currentstroke}%
\pgfsetdash{}{0pt}%
\pgfpathmoveto{\pgfqpoint{2.954006in}{1.649613in}}%
\pgfpathlineto{\pgfqpoint{3.017309in}{1.645186in}}%
\pgfpathlineto{\pgfqpoint{2.952260in}{1.701982in}}%
\pgfpathlineto{\pgfqpoint{2.888596in}{1.698581in}}%
\pgfpathclose%
\pgfusepath{fill}%
\end{pgfscope}%
\begin{pgfscope}%
\pgfpathrectangle{\pgfqpoint{0.150000in}{0.150000in}}{\pgfqpoint{4.700000in}{3.450000in}}%
\pgfusepath{clip}%
\pgfsetbuttcap%
\pgfsetroundjoin%
\definecolor{currentfill}{rgb}{0.586412,0.637347,0.708655}%
\pgfsetfillcolor{currentfill}%
\pgfsetlinewidth{0.000000pt}%
\definecolor{currentstroke}{rgb}{0.000000,0.000000,0.000000}%
\pgfsetstrokecolor{currentstroke}%
\pgfsetdash{}{0pt}%
\pgfpathmoveto{\pgfqpoint{1.847361in}{2.669800in}}%
\pgfpathlineto{\pgfqpoint{1.914726in}{2.669835in}}%
\pgfpathlineto{\pgfqpoint{1.849948in}{2.734439in}}%
\pgfpathlineto{\pgfqpoint{1.782367in}{2.734629in}}%
\pgfpathclose%
\pgfusepath{fill}%
\end{pgfscope}%
\begin{pgfscope}%
\pgfpathrectangle{\pgfqpoint{0.150000in}{0.150000in}}{\pgfqpoint{4.700000in}{3.450000in}}%
\pgfusepath{clip}%
\pgfsetbuttcap%
\pgfsetroundjoin%
\definecolor{currentfill}{rgb}{0.623729,0.670067,0.734942}%
\pgfsetfillcolor{currentfill}%
\pgfsetlinewidth{0.000000pt}%
\definecolor{currentstroke}{rgb}{0.000000,0.000000,0.000000}%
\pgfsetstrokecolor{currentstroke}%
\pgfsetdash{}{0pt}%
\pgfpathmoveto{\pgfqpoint{1.912374in}{2.604952in}}%
\pgfpathlineto{\pgfqpoint{1.979522in}{2.605212in}}%
\pgfpathlineto{\pgfqpoint{1.914726in}{2.669835in}}%
\pgfpathlineto{\pgfqpoint{1.847361in}{2.669800in}}%
\pgfpathclose%
\pgfusepath{fill}%
\end{pgfscope}%
\begin{pgfscope}%
\pgfpathrectangle{\pgfqpoint{0.150000in}{0.150000in}}{\pgfqpoint{4.700000in}{3.450000in}}%
\pgfusepath{clip}%
\pgfsetbuttcap%
\pgfsetroundjoin%
\definecolor{currentfill}{rgb}{0.857537,0.741498,0.750689}%
\pgfsetfillcolor{currentfill}%
\pgfsetlinewidth{0.000000pt}%
\definecolor{currentstroke}{rgb}{0.000000,0.000000,0.000000}%
\pgfsetstrokecolor{currentstroke}%
\pgfsetdash{}{0pt}%
\pgfpathmoveto{\pgfqpoint{3.019691in}{1.606292in}}%
\pgfpathlineto{\pgfqpoint{3.082828in}{1.602267in}}%
\pgfpathlineto{\pgfqpoint{3.017309in}{1.645186in}}%
\pgfpathlineto{\pgfqpoint{2.954006in}{1.649613in}}%
\pgfpathclose%
\pgfusepath{fill}%
\end{pgfscope}%
\begin{pgfscope}%
\pgfpathrectangle{\pgfqpoint{0.150000in}{0.150000in}}{\pgfqpoint{4.700000in}{3.450000in}}%
\pgfusepath{clip}%
\pgfsetbuttcap%
\pgfsetroundjoin%
\definecolor{currentfill}{rgb}{0.808150,0.651884,0.664262}%
\pgfsetfillcolor{currentfill}%
\pgfsetlinewidth{0.000000pt}%
\definecolor{currentstroke}{rgb}{0.000000,0.000000,0.000000}%
\pgfsetstrokecolor{currentstroke}%
\pgfsetdash{}{0pt}%
\pgfpathmoveto{\pgfqpoint{3.280431in}{1.472703in}}%
\pgfpathlineto{\pgfqpoint{3.342584in}{1.469790in}}%
\pgfpathlineto{\pgfqpoint{3.276707in}{1.512772in}}%
\pgfpathlineto{\pgfqpoint{3.214386in}{1.515966in}}%
\pgfpathclose%
\pgfusepath{fill}%
\end{pgfscope}%
\begin{pgfscope}%
\pgfpathrectangle{\pgfqpoint{0.150000in}{0.150000in}}{\pgfqpoint{4.700000in}{3.450000in}}%
\pgfusepath{clip}%
\pgfsetbuttcap%
\pgfsetroundjoin%
\definecolor{currentfill}{rgb}{0.661045,0.702788,0.761229}%
\pgfsetfillcolor{currentfill}%
\pgfsetlinewidth{0.000000pt}%
\definecolor{currentstroke}{rgb}{0.000000,0.000000,0.000000}%
\pgfsetstrokecolor{currentstroke}%
\pgfsetdash{}{0pt}%
\pgfpathmoveto{\pgfqpoint{1.977405in}{2.540086in}}%
\pgfpathlineto{\pgfqpoint{2.044337in}{2.540572in}}%
\pgfpathlineto{\pgfqpoint{1.979522in}{2.605212in}}%
\pgfpathlineto{\pgfqpoint{1.912374in}{2.604952in}}%
\pgfpathclose%
\pgfusepath{fill}%
\end{pgfscope}%
\begin{pgfscope}%
\pgfpathrectangle{\pgfqpoint{0.150000in}{0.150000in}}{\pgfqpoint{4.700000in}{3.450000in}}%
\pgfusepath{clip}%
\pgfsetbuttcap%
\pgfsetroundjoin%
\definecolor{currentfill}{rgb}{0.698361,0.735509,0.787515}%
\pgfsetfillcolor{currentfill}%
\pgfsetlinewidth{0.000000pt}%
\definecolor{currentstroke}{rgb}{0.000000,0.000000,0.000000}%
\pgfsetstrokecolor{currentstroke}%
\pgfsetdash{}{0pt}%
\pgfpathmoveto{\pgfqpoint{2.042454in}{2.475202in}}%
\pgfpathlineto{\pgfqpoint{2.109170in}{2.475913in}}%
\pgfpathlineto{\pgfqpoint{2.044337in}{2.540572in}}%
\pgfpathlineto{\pgfqpoint{1.977405in}{2.540086in}}%
\pgfpathclose%
\pgfusepath{fill}%
\end{pgfscope}%
\begin{pgfscope}%
\pgfpathrectangle{\pgfqpoint{0.150000in}{0.150000in}}{\pgfqpoint{4.700000in}{3.450000in}}%
\pgfusepath{clip}%
\pgfsetbuttcap%
\pgfsetroundjoin%
\definecolor{currentfill}{rgb}{0.758762,0.562270,0.577834}%
\pgfsetfillcolor{currentfill}%
\pgfsetlinewidth{0.000000pt}%
\definecolor{currentstroke}{rgb}{0.000000,0.000000,0.000000}%
\pgfsetstrokecolor{currentstroke}%
\pgfsetdash{}{0pt}%
\pgfpathmoveto{\pgfqpoint{3.541290in}{1.340387in}}%
\pgfpathlineto{\pgfqpoint{3.602479in}{1.338572in}}%
\pgfpathlineto{\pgfqpoint{3.536229in}{1.381502in}}%
\pgfpathlineto{\pgfqpoint{3.474875in}{1.383598in}}%
\pgfpathclose%
\pgfusepath{fill}%
\end{pgfscope}%
\begin{pgfscope}%
\pgfpathrectangle{\pgfqpoint{0.150000in}{0.150000in}}{\pgfqpoint{4.700000in}{3.450000in}}%
\pgfusepath{clip}%
\pgfsetbuttcap%
\pgfsetroundjoin%
\definecolor{currentfill}{rgb}{0.735677,0.768229,0.813802}%
\pgfsetfillcolor{currentfill}%
\pgfsetlinewidth{0.000000pt}%
\definecolor{currentstroke}{rgb}{0.000000,0.000000,0.000000}%
\pgfsetstrokecolor{currentstroke}%
\pgfsetdash{}{0pt}%
\pgfpathmoveto{\pgfqpoint{2.107522in}{2.410300in}}%
\pgfpathlineto{\pgfqpoint{2.174021in}{2.411236in}}%
\pgfpathlineto{\pgfqpoint{2.109170in}{2.475913in}}%
\pgfpathlineto{\pgfqpoint{2.042454in}{2.475202in}}%
\pgfpathclose%
\pgfusepath{fill}%
\end{pgfscope}%
\begin{pgfscope}%
\pgfpathrectangle{\pgfqpoint{0.150000in}{0.150000in}}{\pgfqpoint{4.700000in}{3.450000in}}%
\pgfusepath{clip}%
\pgfsetbuttcap%
\pgfsetroundjoin%
\definecolor{currentfill}{rgb}{0.772993,0.800950,0.840089}%
\pgfsetfillcolor{currentfill}%
\pgfsetlinewidth{0.000000pt}%
\definecolor{currentstroke}{rgb}{0.000000,0.000000,0.000000}%
\pgfsetstrokecolor{currentstroke}%
\pgfsetdash{}{0pt}%
\pgfpathmoveto{\pgfqpoint{2.172608in}{2.345379in}}%
\pgfpathlineto{\pgfqpoint{2.238891in}{2.346541in}}%
\pgfpathlineto{\pgfqpoint{2.174021in}{2.411236in}}%
\pgfpathlineto{\pgfqpoint{2.107522in}{2.410300in}}%
\pgfpathclose%
\pgfusepath{fill}%
\end{pgfscope}%
\begin{pgfscope}%
\pgfpathrectangle{\pgfqpoint{0.150000in}{0.150000in}}{\pgfqpoint{4.700000in}{3.450000in}}%
\pgfusepath{clip}%
\pgfsetbuttcap%
\pgfsetroundjoin%
\definecolor{currentfill}{rgb}{0.810309,0.833670,0.866376}%
\pgfsetfillcolor{currentfill}%
\pgfsetlinewidth{0.000000pt}%
\definecolor{currentstroke}{rgb}{0.000000,0.000000,0.000000}%
\pgfsetstrokecolor{currentstroke}%
\pgfsetdash{}{0pt}%
\pgfpathmoveto{\pgfqpoint{2.237713in}{2.280440in}}%
\pgfpathlineto{\pgfqpoint{2.303778in}{2.281827in}}%
\pgfpathlineto{\pgfqpoint{2.238891in}{2.346541in}}%
\pgfpathlineto{\pgfqpoint{2.172608in}{2.345379in}}%
\pgfpathclose%
\pgfusepath{fill}%
\end{pgfscope}%
\begin{pgfscope}%
\pgfpathrectangle{\pgfqpoint{0.150000in}{0.150000in}}{\pgfqpoint{4.700000in}{3.450000in}}%
\pgfusepath{clip}%
\pgfsetbuttcap%
\pgfsetroundjoin%
\definecolor{currentfill}{rgb}{0.847626,0.866391,0.892662}%
\pgfsetfillcolor{currentfill}%
\pgfsetlinewidth{0.000000pt}%
\definecolor{currentstroke}{rgb}{0.000000,0.000000,0.000000}%
\pgfsetstrokecolor{currentstroke}%
\pgfsetdash{}{0pt}%
\pgfpathmoveto{\pgfqpoint{2.302836in}{2.215482in}}%
\pgfpathlineto{\pgfqpoint{2.368685in}{2.217095in}}%
\pgfpathlineto{\pgfqpoint{2.303778in}{2.281827in}}%
\pgfpathlineto{\pgfqpoint{2.237713in}{2.280440in}}%
\pgfpathclose%
\pgfusepath{fill}%
\end{pgfscope}%
\begin{pgfscope}%
\pgfpathrectangle{\pgfqpoint{0.150000in}{0.150000in}}{\pgfqpoint{4.700000in}{3.450000in}}%
\pgfusepath{clip}%
\pgfsetbuttcap%
\pgfsetroundjoin%
\definecolor{currentfill}{rgb}{0.884942,0.899112,0.918949}%
\pgfsetfillcolor{currentfill}%
\pgfsetlinewidth{0.000000pt}%
\definecolor{currentstroke}{rgb}{0.000000,0.000000,0.000000}%
\pgfsetstrokecolor{currentstroke}%
\pgfsetdash{}{0pt}%
\pgfpathmoveto{\pgfqpoint{2.367978in}{2.150506in}}%
\pgfpathlineto{\pgfqpoint{2.433609in}{2.152345in}}%
\pgfpathlineto{\pgfqpoint{2.368685in}{2.217095in}}%
\pgfpathlineto{\pgfqpoint{2.302836in}{2.215482in}}%
\pgfpathclose%
\pgfusepath{fill}%
\end{pgfscope}%
\begin{pgfscope}%
\pgfpathrectangle{\pgfqpoint{0.150000in}{0.150000in}}{\pgfqpoint{4.700000in}{3.450000in}}%
\pgfusepath{clip}%
\pgfsetbuttcap%
\pgfsetroundjoin%
\definecolor{currentfill}{rgb}{0.922258,0.931832,0.945236}%
\pgfsetfillcolor{currentfill}%
\pgfsetlinewidth{0.000000pt}%
\definecolor{currentstroke}{rgb}{0.000000,0.000000,0.000000}%
\pgfsetstrokecolor{currentstroke}%
\pgfsetdash{}{0pt}%
\pgfpathmoveto{\pgfqpoint{2.433138in}{2.085511in}}%
\pgfpathlineto{\pgfqpoint{2.498552in}{2.087576in}}%
\pgfpathlineto{\pgfqpoint{2.433609in}{2.152345in}}%
\pgfpathlineto{\pgfqpoint{2.367978in}{2.150506in}}%
\pgfpathclose%
\pgfusepath{fill}%
\end{pgfscope}%
\begin{pgfscope}%
\pgfpathrectangle{\pgfqpoint{0.150000in}{0.150000in}}{\pgfqpoint{4.700000in}{3.450000in}}%
\pgfusepath{clip}%
\pgfsetbuttcap%
\pgfsetroundjoin%
\definecolor{currentfill}{rgb}{0.959574,0.964553,0.971523}%
\pgfsetfillcolor{currentfill}%
\pgfsetlinewidth{0.000000pt}%
\definecolor{currentstroke}{rgb}{0.000000,0.000000,0.000000}%
\pgfsetstrokecolor{currentstroke}%
\pgfsetdash{}{0pt}%
\pgfpathmoveto{\pgfqpoint{2.498316in}{2.020498in}}%
\pgfpathlineto{\pgfqpoint{2.563514in}{2.022790in}}%
\pgfpathlineto{\pgfqpoint{2.498552in}{2.087576in}}%
\pgfpathlineto{\pgfqpoint{2.433138in}{2.085511in}}%
\pgfpathclose%
\pgfusepath{fill}%
\end{pgfscope}%
\begin{pgfscope}%
\pgfpathrectangle{\pgfqpoint{0.150000in}{0.150000in}}{\pgfqpoint{4.700000in}{3.450000in}}%
\pgfusepath{clip}%
\pgfsetbuttcap%
\pgfsetroundjoin%
\definecolor{currentfill}{rgb}{0.849939,0.727711,0.737393}%
\pgfsetfillcolor{currentfill}%
\pgfsetlinewidth{0.000000pt}%
\definecolor{currentstroke}{rgb}{0.000000,0.000000,0.000000}%
\pgfsetstrokecolor{currentstroke}%
\pgfsetdash{}{0pt}%
\pgfpathmoveto{\pgfqpoint{3.085550in}{1.562857in}}%
\pgfpathlineto{\pgfqpoint{3.148522in}{1.559234in}}%
\pgfpathlineto{\pgfqpoint{3.082828in}{1.602267in}}%
\pgfpathlineto{\pgfqpoint{3.019691in}{1.606292in}}%
\pgfpathclose%
\pgfusepath{fill}%
\end{pgfscope}%
\begin{pgfscope}%
\pgfpathrectangle{\pgfqpoint{0.150000in}{0.150000in}}{\pgfqpoint{4.700000in}{3.450000in}}%
\pgfusepath{clip}%
\pgfsetbuttcap%
\pgfsetroundjoin%
\definecolor{currentfill}{rgb}{0.996890,0.997273,0.997809}%
\pgfsetfillcolor{currentfill}%
\pgfsetlinewidth{0.000000pt}%
\definecolor{currentstroke}{rgb}{0.000000,0.000000,0.000000}%
\pgfsetstrokecolor{currentstroke}%
\pgfsetdash{}{0pt}%
\pgfpathmoveto{\pgfqpoint{2.563514in}{1.955467in}}%
\pgfpathlineto{\pgfqpoint{2.628493in}{1.957985in}}%
\pgfpathlineto{\pgfqpoint{2.563514in}{2.022790in}}%
\pgfpathlineto{\pgfqpoint{2.498316in}{2.020498in}}%
\pgfpathclose%
\pgfusepath{fill}%
\end{pgfscope}%
\begin{pgfscope}%
\pgfpathrectangle{\pgfqpoint{0.150000in}{0.150000in}}{\pgfqpoint{4.700000in}{3.450000in}}%
\pgfusepath{clip}%
\pgfsetbuttcap%
\pgfsetroundjoin%
\definecolor{currentfill}{rgb}{0.800551,0.638097,0.650965}%
\pgfsetfillcolor{currentfill}%
\pgfsetlinewidth{0.000000pt}%
\definecolor{currentstroke}{rgb}{0.000000,0.000000,0.000000}%
\pgfsetstrokecolor{currentstroke}%
\pgfsetdash{}{0pt}%
\pgfpathmoveto{\pgfqpoint{3.346652in}{1.429323in}}%
\pgfpathlineto{\pgfqpoint{3.408643in}{1.426811in}}%
\pgfpathlineto{\pgfqpoint{3.342584in}{1.469790in}}%
\pgfpathlineto{\pgfqpoint{3.280431in}{1.472703in}}%
\pgfpathclose%
\pgfusepath{fill}%
\end{pgfscope}%
\begin{pgfscope}%
\pgfpathrectangle{\pgfqpoint{0.150000in}{0.150000in}}{\pgfqpoint{4.700000in}{3.450000in}}%
\pgfusepath{clip}%
\pgfsetbuttcap%
\pgfsetroundjoin%
\definecolor{currentfill}{rgb}{0.975306,0.955193,0.956786}%
\pgfsetfillcolor{currentfill}%
\pgfsetlinewidth{0.000000pt}%
\definecolor{currentstroke}{rgb}{0.000000,0.000000,0.000000}%
\pgfsetstrokecolor{currentstroke}%
\pgfsetdash{}{0pt}%
\pgfpathmoveto{\pgfqpoint{2.628729in}{1.890417in}}%
\pgfpathlineto{\pgfqpoint{2.693491in}{1.893161in}}%
\pgfpathlineto{\pgfqpoint{2.628493in}{1.957985in}}%
\pgfpathlineto{\pgfqpoint{2.563514in}{1.955467in}}%
\pgfpathclose%
\pgfusepath{fill}%
\end{pgfscope}%
\begin{pgfscope}%
\pgfpathrectangle{\pgfqpoint{0.150000in}{0.150000in}}{\pgfqpoint{4.700000in}{3.450000in}}%
\pgfusepath{clip}%
\pgfsetbuttcap%
\pgfsetroundjoin%
\definecolor{currentfill}{rgb}{0.754963,0.555377,0.571186}%
\pgfsetfillcolor{currentfill}%
\pgfsetlinewidth{0.000000pt}%
\definecolor{currentstroke}{rgb}{0.000000,0.000000,0.000000}%
\pgfsetstrokecolor{currentstroke}%
\pgfsetdash{}{0pt}%
\pgfpathmoveto{\pgfqpoint{3.607879in}{1.296941in}}%
\pgfpathlineto{\pgfqpoint{3.668909in}{1.295525in}}%
\pgfpathlineto{\pgfqpoint{3.602479in}{1.338572in}}%
\pgfpathlineto{\pgfqpoint{3.541290in}{1.340387in}}%
\pgfpathclose%
\pgfusepath{fill}%
\end{pgfscope}%
\begin{pgfscope}%
\pgfpathrectangle{\pgfqpoint{0.150000in}{0.150000in}}{\pgfqpoint{4.700000in}{3.450000in}}%
\pgfusepath{clip}%
\pgfsetbuttcap%
\pgfsetroundjoin%
\definecolor{currentfill}{rgb}{0.952512,0.913833,0.916896}%
\pgfsetfillcolor{currentfill}%
\pgfsetlinewidth{0.000000pt}%
\definecolor{currentstroke}{rgb}{0.000000,0.000000,0.000000}%
\pgfsetstrokecolor{currentstroke}%
\pgfsetdash{}{0pt}%
\pgfpathmoveto{\pgfqpoint{2.693963in}{1.825348in}}%
\pgfpathlineto{\pgfqpoint{2.758508in}{1.828319in}}%
\pgfpathlineto{\pgfqpoint{2.693491in}{1.893161in}}%
\pgfpathlineto{\pgfqpoint{2.628729in}{1.890417in}}%
\pgfpathclose%
\pgfusepath{fill}%
\end{pgfscope}%
\begin{pgfscope}%
\pgfpathrectangle{\pgfqpoint{0.150000in}{0.150000in}}{\pgfqpoint{4.700000in}{3.450000in}}%
\pgfusepath{clip}%
\pgfsetbuttcap%
\pgfsetroundjoin%
\definecolor{currentfill}{rgb}{0.929718,0.872472,0.877007}%
\pgfsetfillcolor{currentfill}%
\pgfsetlinewidth{0.000000pt}%
\definecolor{currentstroke}{rgb}{0.000000,0.000000,0.000000}%
\pgfsetstrokecolor{currentstroke}%
\pgfsetdash{}{0pt}%
\pgfpathmoveto{\pgfqpoint{2.759216in}{1.760262in}}%
\pgfpathlineto{\pgfqpoint{2.823542in}{1.763459in}}%
\pgfpathlineto{\pgfqpoint{2.758508in}{1.828319in}}%
\pgfpathlineto{\pgfqpoint{2.693963in}{1.825348in}}%
\pgfpathclose%
\pgfusepath{fill}%
\end{pgfscope}%
\begin{pgfscope}%
\pgfpathrectangle{\pgfqpoint{0.150000in}{0.150000in}}{\pgfqpoint{4.700000in}{3.450000in}}%
\pgfusepath{clip}%
\pgfsetbuttcap%
\pgfsetroundjoin%
\definecolor{currentfill}{rgb}{0.906924,0.831112,0.837117}%
\pgfsetfillcolor{currentfill}%
\pgfsetlinewidth{0.000000pt}%
\definecolor{currentstroke}{rgb}{0.000000,0.000000,0.000000}%
\pgfsetstrokecolor{currentstroke}%
\pgfsetdash{}{0pt}%
\pgfpathmoveto{\pgfqpoint{2.824521in}{1.697566in}}%
\pgfpathlineto{\pgfqpoint{2.888596in}{1.698581in}}%
\pgfpathlineto{\pgfqpoint{2.823542in}{1.763459in}}%
\pgfpathlineto{\pgfqpoint{2.759216in}{1.760262in}}%
\pgfpathclose%
\pgfusepath{fill}%
\end{pgfscope}%
\begin{pgfscope}%
\pgfpathrectangle{\pgfqpoint{0.150000in}{0.150000in}}{\pgfqpoint{4.700000in}{3.450000in}}%
\pgfusepath{clip}%
\pgfsetbuttcap%
\pgfsetroundjoin%
\definecolor{currentfill}{rgb}{0.842341,0.713925,0.724096}%
\pgfsetfillcolor{currentfill}%
\pgfsetlinewidth{0.000000pt}%
\definecolor{currentstroke}{rgb}{0.000000,0.000000,0.000000}%
\pgfsetstrokecolor{currentstroke}%
\pgfsetdash{}{0pt}%
\pgfpathmoveto{\pgfqpoint{3.151584in}{1.519306in}}%
\pgfpathlineto{\pgfqpoint{3.214386in}{1.515966in}}%
\pgfpathlineto{\pgfqpoint{3.148522in}{1.559234in}}%
\pgfpathlineto{\pgfqpoint{3.085550in}{1.562857in}}%
\pgfpathclose%
\pgfusepath{fill}%
\end{pgfscope}%
\begin{pgfscope}%
\pgfpathrectangle{\pgfqpoint{0.150000in}{0.150000in}}{\pgfqpoint{4.700000in}{3.450000in}}%
\pgfusepath{clip}%
\pgfsetbuttcap%
\pgfsetroundjoin%
\definecolor{currentfill}{rgb}{0.891728,0.803539,0.810524}%
\pgfsetfillcolor{currentfill}%
\pgfsetlinewidth{0.000000pt}%
\definecolor{currentstroke}{rgb}{0.000000,0.000000,0.000000}%
\pgfsetstrokecolor{currentstroke}%
\pgfsetdash{}{0pt}%
\pgfpathmoveto{\pgfqpoint{2.890202in}{1.653955in}}%
\pgfpathlineto{\pgfqpoint{2.954006in}{1.649613in}}%
\pgfpathlineto{\pgfqpoint{2.888596in}{1.698581in}}%
\pgfpathlineto{\pgfqpoint{2.824521in}{1.697566in}}%
\pgfpathclose%
\pgfusepath{fill}%
\end{pgfscope}%
\begin{pgfscope}%
\pgfpathrectangle{\pgfqpoint{0.150000in}{0.150000in}}{\pgfqpoint{4.700000in}{3.450000in}}%
\pgfusepath{clip}%
\pgfsetbuttcap%
\pgfsetroundjoin%
\definecolor{currentfill}{rgb}{0.792953,0.624311,0.637669}%
\pgfsetfillcolor{currentfill}%
\pgfsetlinewidth{0.000000pt}%
\definecolor{currentstroke}{rgb}{0.000000,0.000000,0.000000}%
\pgfsetstrokecolor{currentstroke}%
\pgfsetdash{}{0pt}%
\pgfpathmoveto{\pgfqpoint{3.413051in}{1.385828in}}%
\pgfpathlineto{\pgfqpoint{3.474875in}{1.383598in}}%
\pgfpathlineto{\pgfqpoint{3.408643in}{1.426811in}}%
\pgfpathlineto{\pgfqpoint{3.346652in}{1.429323in}}%
\pgfpathclose%
\pgfusepath{fill}%
\end{pgfscope}%
\begin{pgfscope}%
\pgfpathrectangle{\pgfqpoint{0.150000in}{0.150000in}}{\pgfqpoint{4.700000in}{3.450000in}}%
\pgfusepath{clip}%
\pgfsetbuttcap%
\pgfsetroundjoin%
\definecolor{currentfill}{rgb}{0.549096,0.604626,0.682368}%
\pgfsetfillcolor{currentfill}%
\pgfsetlinewidth{0.000000pt}%
\definecolor{currentstroke}{rgb}{0.000000,0.000000,0.000000}%
\pgfsetstrokecolor{currentstroke}%
\pgfsetdash{}{0pt}%
\pgfpathmoveto{\pgfqpoint{1.779526in}{2.669764in}}%
\pgfpathlineto{\pgfqpoint{1.847361in}{2.669800in}}%
\pgfpathlineto{\pgfqpoint{1.782367in}{2.734629in}}%
\pgfpathlineto{\pgfqpoint{1.714315in}{2.734820in}}%
\pgfpathclose%
\pgfusepath{fill}%
\end{pgfscope}%
\begin{pgfscope}%
\pgfpathrectangle{\pgfqpoint{0.150000in}{0.150000in}}{\pgfqpoint{4.700000in}{3.450000in}}%
\pgfusepath{clip}%
\pgfsetbuttcap%
\pgfsetroundjoin%
\definecolor{currentfill}{rgb}{0.586412,0.637347,0.708655}%
\pgfsetfillcolor{currentfill}%
\pgfsetlinewidth{0.000000pt}%
\definecolor{currentstroke}{rgb}{0.000000,0.000000,0.000000}%
\pgfsetstrokecolor{currentstroke}%
\pgfsetdash{}{0pt}%
\pgfpathmoveto{\pgfqpoint{1.844757in}{2.604690in}}%
\pgfpathlineto{\pgfqpoint{1.912374in}{2.604952in}}%
\pgfpathlineto{\pgfqpoint{1.847361in}{2.669800in}}%
\pgfpathlineto{\pgfqpoint{1.779526in}{2.669764in}}%
\pgfpathclose%
\pgfusepath{fill}%
\end{pgfscope}%
\begin{pgfscope}%
\pgfpathrectangle{\pgfqpoint{0.150000in}{0.150000in}}{\pgfqpoint{4.700000in}{3.450000in}}%
\pgfusepath{clip}%
\pgfsetbuttcap%
\pgfsetroundjoin%
\definecolor{currentfill}{rgb}{0.623729,0.670067,0.734942}%
\pgfsetfillcolor{currentfill}%
\pgfsetlinewidth{0.000000pt}%
\definecolor{currentstroke}{rgb}{0.000000,0.000000,0.000000}%
\pgfsetstrokecolor{currentstroke}%
\pgfsetdash{}{0pt}%
\pgfpathmoveto{\pgfqpoint{1.910005in}{2.539598in}}%
\pgfpathlineto{\pgfqpoint{1.977405in}{2.540086in}}%
\pgfpathlineto{\pgfqpoint{1.912374in}{2.604952in}}%
\pgfpathlineto{\pgfqpoint{1.844757in}{2.604690in}}%
\pgfpathclose%
\pgfusepath{fill}%
\end{pgfscope}%
\begin{pgfscope}%
\pgfpathrectangle{\pgfqpoint{0.150000in}{0.150000in}}{\pgfqpoint{4.700000in}{3.450000in}}%
\pgfusepath{clip}%
\pgfsetbuttcap%
\pgfsetroundjoin%
\definecolor{currentfill}{rgb}{0.661045,0.702788,0.761229}%
\pgfsetfillcolor{currentfill}%
\pgfsetlinewidth{0.000000pt}%
\definecolor{currentstroke}{rgb}{0.000000,0.000000,0.000000}%
\pgfsetstrokecolor{currentstroke}%
\pgfsetdash{}{0pt}%
\pgfpathmoveto{\pgfqpoint{1.975272in}{2.474487in}}%
\pgfpathlineto{\pgfqpoint{2.042454in}{2.475202in}}%
\pgfpathlineto{\pgfqpoint{1.977405in}{2.540086in}}%
\pgfpathlineto{\pgfqpoint{1.910005in}{2.539598in}}%
\pgfpathclose%
\pgfusepath{fill}%
\end{pgfscope}%
\begin{pgfscope}%
\pgfpathrectangle{\pgfqpoint{0.150000in}{0.150000in}}{\pgfqpoint{4.700000in}{3.450000in}}%
\pgfusepath{clip}%
\pgfsetbuttcap%
\pgfsetroundjoin%
\definecolor{currentfill}{rgb}{0.698361,0.735509,0.787515}%
\pgfsetfillcolor{currentfill}%
\pgfsetlinewidth{0.000000pt}%
\definecolor{currentstroke}{rgb}{0.000000,0.000000,0.000000}%
\pgfsetstrokecolor{currentstroke}%
\pgfsetdash{}{0pt}%
\pgfpathmoveto{\pgfqpoint{2.040558in}{2.409357in}}%
\pgfpathlineto{\pgfqpoint{2.107522in}{2.410300in}}%
\pgfpathlineto{\pgfqpoint{2.042454in}{2.475202in}}%
\pgfpathlineto{\pgfqpoint{1.975272in}{2.474487in}}%
\pgfpathclose%
\pgfusepath{fill}%
\end{pgfscope}%
\begin{pgfscope}%
\pgfpathrectangle{\pgfqpoint{0.150000in}{0.150000in}}{\pgfqpoint{4.700000in}{3.450000in}}%
\pgfusepath{clip}%
\pgfsetbuttcap%
\pgfsetroundjoin%
\definecolor{currentfill}{rgb}{0.884130,0.789752,0.797227}%
\pgfsetfillcolor{currentfill}%
\pgfsetlinewidth{0.000000pt}%
\definecolor{currentstroke}{rgb}{0.000000,0.000000,0.000000}%
\pgfsetstrokecolor{currentstroke}%
\pgfsetdash{}{0pt}%
\pgfpathmoveto{\pgfqpoint{2.956058in}{1.610349in}}%
\pgfpathlineto{\pgfqpoint{3.019691in}{1.606292in}}%
\pgfpathlineto{\pgfqpoint{2.954006in}{1.649613in}}%
\pgfpathlineto{\pgfqpoint{2.890202in}{1.653955in}}%
\pgfpathclose%
\pgfusepath{fill}%
\end{pgfscope}%
\begin{pgfscope}%
\pgfpathrectangle{\pgfqpoint{0.150000in}{0.150000in}}{\pgfqpoint{4.700000in}{3.450000in}}%
\pgfusepath{clip}%
\pgfsetbuttcap%
\pgfsetroundjoin%
\definecolor{currentfill}{rgb}{0.834743,0.700138,0.710800}%
\pgfsetfillcolor{currentfill}%
\pgfsetlinewidth{0.000000pt}%
\definecolor{currentstroke}{rgb}{0.000000,0.000000,0.000000}%
\pgfsetstrokecolor{currentstroke}%
\pgfsetdash{}{0pt}%
\pgfpathmoveto{\pgfqpoint{3.217794in}{1.475639in}}%
\pgfpathlineto{\pgfqpoint{3.280431in}{1.472703in}}%
\pgfpathlineto{\pgfqpoint{3.214386in}{1.515966in}}%
\pgfpathlineto{\pgfqpoint{3.151584in}{1.519306in}}%
\pgfpathclose%
\pgfusepath{fill}%
\end{pgfscope}%
\begin{pgfscope}%
\pgfpathrectangle{\pgfqpoint{0.150000in}{0.150000in}}{\pgfqpoint{4.700000in}{3.450000in}}%
\pgfusepath{clip}%
\pgfsetbuttcap%
\pgfsetroundjoin%
\definecolor{currentfill}{rgb}{0.735677,0.768229,0.813802}%
\pgfsetfillcolor{currentfill}%
\pgfsetlinewidth{0.000000pt}%
\definecolor{currentstroke}{rgb}{0.000000,0.000000,0.000000}%
\pgfsetstrokecolor{currentstroke}%
\pgfsetdash{}{0pt}%
\pgfpathmoveto{\pgfqpoint{2.105863in}{2.344209in}}%
\pgfpathlineto{\pgfqpoint{2.172608in}{2.345379in}}%
\pgfpathlineto{\pgfqpoint{2.107522in}{2.410300in}}%
\pgfpathlineto{\pgfqpoint{2.040558in}{2.409357in}}%
\pgfpathclose%
\pgfusepath{fill}%
\end{pgfscope}%
\begin{pgfscope}%
\pgfpathrectangle{\pgfqpoint{0.150000in}{0.150000in}}{\pgfqpoint{4.700000in}{3.450000in}}%
\pgfusepath{clip}%
\pgfsetbuttcap%
\pgfsetroundjoin%
\definecolor{currentfill}{rgb}{0.789154,0.617417,0.631020}%
\pgfsetfillcolor{currentfill}%
\pgfsetlinewidth{0.000000pt}%
\definecolor{currentstroke}{rgb}{0.000000,0.000000,0.000000}%
\pgfsetstrokecolor{currentstroke}%
\pgfsetdash{}{0pt}%
\pgfpathmoveto{\pgfqpoint{3.479623in}{1.342097in}}%
\pgfpathlineto{\pgfqpoint{3.541290in}{1.340387in}}%
\pgfpathlineto{\pgfqpoint{3.474875in}{1.383598in}}%
\pgfpathlineto{\pgfqpoint{3.413051in}{1.385828in}}%
\pgfpathclose%
\pgfusepath{fill}%
\end{pgfscope}%
\begin{pgfscope}%
\pgfpathrectangle{\pgfqpoint{0.150000in}{0.150000in}}{\pgfqpoint{4.700000in}{3.450000in}}%
\pgfusepath{clip}%
\pgfsetbuttcap%
\pgfsetroundjoin%
\definecolor{currentfill}{rgb}{0.772993,0.800950,0.840089}%
\pgfsetfillcolor{currentfill}%
\pgfsetlinewidth{0.000000pt}%
\definecolor{currentstroke}{rgb}{0.000000,0.000000,0.000000}%
\pgfsetstrokecolor{currentstroke}%
\pgfsetdash{}{0pt}%
\pgfpathmoveto{\pgfqpoint{2.171186in}{2.279043in}}%
\pgfpathlineto{\pgfqpoint{2.237713in}{2.280440in}}%
\pgfpathlineto{\pgfqpoint{2.172608in}{2.345379in}}%
\pgfpathlineto{\pgfqpoint{2.105863in}{2.344209in}}%
\pgfpathclose%
\pgfusepath{fill}%
\end{pgfscope}%
\begin{pgfscope}%
\pgfpathrectangle{\pgfqpoint{0.150000in}{0.150000in}}{\pgfqpoint{4.700000in}{3.450000in}}%
\pgfusepath{clip}%
\pgfsetbuttcap%
\pgfsetroundjoin%
\definecolor{currentfill}{rgb}{0.810309,0.833670,0.866376}%
\pgfsetfillcolor{currentfill}%
\pgfsetlinewidth{0.000000pt}%
\definecolor{currentstroke}{rgb}{0.000000,0.000000,0.000000}%
\pgfsetstrokecolor{currentstroke}%
\pgfsetdash{}{0pt}%
\pgfpathmoveto{\pgfqpoint{2.236527in}{2.213857in}}%
\pgfpathlineto{\pgfqpoint{2.302836in}{2.215482in}}%
\pgfpathlineto{\pgfqpoint{2.237713in}{2.280440in}}%
\pgfpathlineto{\pgfqpoint{2.171186in}{2.279043in}}%
\pgfpathclose%
\pgfusepath{fill}%
\end{pgfscope}%
\begin{pgfscope}%
\pgfpathrectangle{\pgfqpoint{0.150000in}{0.150000in}}{\pgfqpoint{4.700000in}{3.450000in}}%
\pgfusepath{clip}%
\pgfsetbuttcap%
\pgfsetroundjoin%
\definecolor{currentfill}{rgb}{0.847626,0.866391,0.892662}%
\pgfsetfillcolor{currentfill}%
\pgfsetlinewidth{0.000000pt}%
\definecolor{currentstroke}{rgb}{0.000000,0.000000,0.000000}%
\pgfsetstrokecolor{currentstroke}%
\pgfsetdash{}{0pt}%
\pgfpathmoveto{\pgfqpoint{2.301887in}{2.148654in}}%
\pgfpathlineto{\pgfqpoint{2.367978in}{2.150506in}}%
\pgfpathlineto{\pgfqpoint{2.302836in}{2.215482in}}%
\pgfpathlineto{\pgfqpoint{2.236527in}{2.213857in}}%
\pgfpathclose%
\pgfusepath{fill}%
\end{pgfscope}%
\begin{pgfscope}%
\pgfpathrectangle{\pgfqpoint{0.150000in}{0.150000in}}{\pgfqpoint{4.700000in}{3.450000in}}%
\pgfusepath{clip}%
\pgfsetbuttcap%
\pgfsetroundjoin%
\definecolor{currentfill}{rgb}{0.884942,0.899112,0.918949}%
\pgfsetfillcolor{currentfill}%
\pgfsetlinewidth{0.000000pt}%
\definecolor{currentstroke}{rgb}{0.000000,0.000000,0.000000}%
\pgfsetstrokecolor{currentstroke}%
\pgfsetdash{}{0pt}%
\pgfpathmoveto{\pgfqpoint{2.367266in}{2.083431in}}%
\pgfpathlineto{\pgfqpoint{2.433138in}{2.085511in}}%
\pgfpathlineto{\pgfqpoint{2.367978in}{2.150506in}}%
\pgfpathlineto{\pgfqpoint{2.301887in}{2.148654in}}%
\pgfpathclose%
\pgfusepath{fill}%
\end{pgfscope}%
\begin{pgfscope}%
\pgfpathrectangle{\pgfqpoint{0.150000in}{0.150000in}}{\pgfqpoint{4.700000in}{3.450000in}}%
\pgfusepath{clip}%
\pgfsetbuttcap%
\pgfsetroundjoin%
\definecolor{currentfill}{rgb}{0.922258,0.931832,0.945236}%
\pgfsetfillcolor{currentfill}%
\pgfsetlinewidth{0.000000pt}%
\definecolor{currentstroke}{rgb}{0.000000,0.000000,0.000000}%
\pgfsetstrokecolor{currentstroke}%
\pgfsetdash{}{0pt}%
\pgfpathmoveto{\pgfqpoint{2.432663in}{2.018191in}}%
\pgfpathlineto{\pgfqpoint{2.498316in}{2.020498in}}%
\pgfpathlineto{\pgfqpoint{2.433138in}{2.085511in}}%
\pgfpathlineto{\pgfqpoint{2.367266in}{2.083431in}}%
\pgfpathclose%
\pgfusepath{fill}%
\end{pgfscope}%
\begin{pgfscope}%
\pgfpathrectangle{\pgfqpoint{0.150000in}{0.150000in}}{\pgfqpoint{4.700000in}{3.450000in}}%
\pgfusepath{clip}%
\pgfsetbuttcap%
\pgfsetroundjoin%
\definecolor{currentfill}{rgb}{0.959574,0.964553,0.971523}%
\pgfsetfillcolor{currentfill}%
\pgfsetlinewidth{0.000000pt}%
\definecolor{currentstroke}{rgb}{0.000000,0.000000,0.000000}%
\pgfsetstrokecolor{currentstroke}%
\pgfsetdash{}{0pt}%
\pgfpathmoveto{\pgfqpoint{2.498079in}{1.952931in}}%
\pgfpathlineto{\pgfqpoint{2.563514in}{1.955467in}}%
\pgfpathlineto{\pgfqpoint{2.498316in}{2.020498in}}%
\pgfpathlineto{\pgfqpoint{2.432663in}{2.018191in}}%
\pgfpathclose%
\pgfusepath{fill}%
\end{pgfscope}%
\begin{pgfscope}%
\pgfpathrectangle{\pgfqpoint{0.150000in}{0.150000in}}{\pgfqpoint{4.700000in}{3.450000in}}%
\pgfusepath{clip}%
\pgfsetbuttcap%
\pgfsetroundjoin%
\definecolor{currentfill}{rgb}{0.996890,0.997273,0.997809}%
\pgfsetfillcolor{currentfill}%
\pgfsetlinewidth{0.000000pt}%
\definecolor{currentstroke}{rgb}{0.000000,0.000000,0.000000}%
\pgfsetstrokecolor{currentstroke}%
\pgfsetdash{}{0pt}%
\pgfpathmoveto{\pgfqpoint{2.563514in}{1.887653in}}%
\pgfpathlineto{\pgfqpoint{2.628729in}{1.890417in}}%
\pgfpathlineto{\pgfqpoint{2.563514in}{1.955467in}}%
\pgfpathlineto{\pgfqpoint{2.498079in}{1.952931in}}%
\pgfpathclose%
\pgfusepath{fill}%
\end{pgfscope}%
\begin{pgfscope}%
\pgfpathrectangle{\pgfqpoint{0.150000in}{0.150000in}}{\pgfqpoint{4.700000in}{3.450000in}}%
\pgfusepath{clip}%
\pgfsetbuttcap%
\pgfsetroundjoin%
\definecolor{currentfill}{rgb}{0.975306,0.955193,0.956786}%
\pgfsetfillcolor{currentfill}%
\pgfsetlinewidth{0.000000pt}%
\definecolor{currentstroke}{rgb}{0.000000,0.000000,0.000000}%
\pgfsetstrokecolor{currentstroke}%
\pgfsetdash{}{0pt}%
\pgfpathmoveto{\pgfqpoint{2.628967in}{1.822357in}}%
\pgfpathlineto{\pgfqpoint{2.693963in}{1.825348in}}%
\pgfpathlineto{\pgfqpoint{2.628729in}{1.890417in}}%
\pgfpathlineto{\pgfqpoint{2.563514in}{1.887653in}}%
\pgfpathclose%
\pgfusepath{fill}%
\end{pgfscope}%
\begin{pgfscope}%
\pgfpathrectangle{\pgfqpoint{0.150000in}{0.150000in}}{\pgfqpoint{4.700000in}{3.450000in}}%
\pgfusepath{clip}%
\pgfsetbuttcap%
\pgfsetroundjoin%
\definecolor{currentfill}{rgb}{0.876532,0.775965,0.783931}%
\pgfsetfillcolor{currentfill}%
\pgfsetlinewidth{0.000000pt}%
\definecolor{currentstroke}{rgb}{0.000000,0.000000,0.000000}%
\pgfsetstrokecolor{currentstroke}%
\pgfsetdash{}{0pt}%
\pgfpathmoveto{\pgfqpoint{3.022085in}{1.566508in}}%
\pgfpathlineto{\pgfqpoint{3.085550in}{1.562857in}}%
\pgfpathlineto{\pgfqpoint{3.019691in}{1.606292in}}%
\pgfpathlineto{\pgfqpoint{2.956058in}{1.610349in}}%
\pgfpathclose%
\pgfusepath{fill}%
\end{pgfscope}%
\begin{pgfscope}%
\pgfpathrectangle{\pgfqpoint{0.150000in}{0.150000in}}{\pgfqpoint{4.700000in}{3.450000in}}%
\pgfusepath{clip}%
\pgfsetbuttcap%
\pgfsetroundjoin%
\definecolor{currentfill}{rgb}{0.952512,0.913833,0.916896}%
\pgfsetfillcolor{currentfill}%
\pgfsetlinewidth{0.000000pt}%
\definecolor{currentstroke}{rgb}{0.000000,0.000000,0.000000}%
\pgfsetstrokecolor{currentstroke}%
\pgfsetdash{}{0pt}%
\pgfpathmoveto{\pgfqpoint{2.694438in}{1.757042in}}%
\pgfpathlineto{\pgfqpoint{2.759216in}{1.760262in}}%
\pgfpathlineto{\pgfqpoint{2.693963in}{1.825348in}}%
\pgfpathlineto{\pgfqpoint{2.628967in}{1.822357in}}%
\pgfpathclose%
\pgfusepath{fill}%
\end{pgfscope}%
\begin{pgfscope}%
\pgfpathrectangle{\pgfqpoint{0.150000in}{0.150000in}}{\pgfqpoint{4.700000in}{3.450000in}}%
\pgfusepath{clip}%
\pgfsetbuttcap%
\pgfsetroundjoin%
\definecolor{currentfill}{rgb}{0.830944,0.693244,0.704151}%
\pgfsetfillcolor{currentfill}%
\pgfsetlinewidth{0.000000pt}%
\definecolor{currentstroke}{rgb}{0.000000,0.000000,0.000000}%
\pgfsetstrokecolor{currentstroke}%
\pgfsetdash{}{0pt}%
\pgfpathmoveto{\pgfqpoint{3.284180in}{1.431856in}}%
\pgfpathlineto{\pgfqpoint{3.346652in}{1.429323in}}%
\pgfpathlineto{\pgfqpoint{3.280431in}{1.472703in}}%
\pgfpathlineto{\pgfqpoint{3.217794in}{1.475639in}}%
\pgfpathclose%
\pgfusepath{fill}%
\end{pgfscope}%
\begin{pgfscope}%
\pgfpathrectangle{\pgfqpoint{0.150000in}{0.150000in}}{\pgfqpoint{4.700000in}{3.450000in}}%
\pgfusepath{clip}%
\pgfsetbuttcap%
\pgfsetroundjoin%
\definecolor{currentfill}{rgb}{0.781556,0.603631,0.617724}%
\pgfsetfillcolor{currentfill}%
\pgfsetlinewidth{0.000000pt}%
\definecolor{currentstroke}{rgb}{0.000000,0.000000,0.000000}%
\pgfsetstrokecolor{currentstroke}%
\pgfsetdash{}{0pt}%
\pgfpathmoveto{\pgfqpoint{3.546379in}{1.298367in}}%
\pgfpathlineto{\pgfqpoint{3.607879in}{1.296941in}}%
\pgfpathlineto{\pgfqpoint{3.541290in}{1.340387in}}%
\pgfpathlineto{\pgfqpoint{3.479623in}{1.342097in}}%
\pgfpathclose%
\pgfusepath{fill}%
\end{pgfscope}%
\begin{pgfscope}%
\pgfpathrectangle{\pgfqpoint{0.150000in}{0.150000in}}{\pgfqpoint{4.700000in}{3.450000in}}%
\pgfusepath{clip}%
\pgfsetbuttcap%
\pgfsetroundjoin%
\definecolor{currentfill}{rgb}{0.933517,0.879366,0.883655}%
\pgfsetfillcolor{currentfill}%
\pgfsetlinewidth{0.000000pt}%
\definecolor{currentstroke}{rgb}{0.000000,0.000000,0.000000}%
\pgfsetstrokecolor{currentstroke}%
\pgfsetdash{}{0pt}%
\pgfpathmoveto{\pgfqpoint{2.760044in}{1.702471in}}%
\pgfpathlineto{\pgfqpoint{2.824521in}{1.697566in}}%
\pgfpathlineto{\pgfqpoint{2.759216in}{1.760262in}}%
\pgfpathlineto{\pgfqpoint{2.694438in}{1.757042in}}%
\pgfpathclose%
\pgfusepath{fill}%
\end{pgfscope}%
\begin{pgfscope}%
\pgfpathrectangle{\pgfqpoint{0.150000in}{0.150000in}}{\pgfqpoint{4.700000in}{3.450000in}}%
\pgfusepath{clip}%
\pgfsetbuttcap%
\pgfsetroundjoin%
\definecolor{currentfill}{rgb}{0.922120,0.858686,0.863710}%
\pgfsetfillcolor{currentfill}%
\pgfsetlinewidth{0.000000pt}%
\definecolor{currentstroke}{rgb}{0.000000,0.000000,0.000000}%
\pgfsetstrokecolor{currentstroke}%
\pgfsetdash{}{0pt}%
\pgfpathmoveto{\pgfqpoint{2.825896in}{1.658452in}}%
\pgfpathlineto{\pgfqpoint{2.890202in}{1.653955in}}%
\pgfpathlineto{\pgfqpoint{2.824521in}{1.697566in}}%
\pgfpathlineto{\pgfqpoint{2.760044in}{1.702471in}}%
\pgfpathclose%
\pgfusepath{fill}%
\end{pgfscope}%
\begin{pgfscope}%
\pgfpathrectangle{\pgfqpoint{0.150000in}{0.150000in}}{\pgfqpoint{4.700000in}{3.450000in}}%
\pgfusepath{clip}%
\pgfsetbuttcap%
\pgfsetroundjoin%
\definecolor{currentfill}{rgb}{0.872733,0.769072,0.777282}%
\pgfsetfillcolor{currentfill}%
\pgfsetlinewidth{0.000000pt}%
\definecolor{currentstroke}{rgb}{0.000000,0.000000,0.000000}%
\pgfsetstrokecolor{currentstroke}%
\pgfsetdash{}{0pt}%
\pgfpathmoveto{\pgfqpoint{3.088287in}{1.522551in}}%
\pgfpathlineto{\pgfqpoint{3.151584in}{1.519306in}}%
\pgfpathlineto{\pgfqpoint{3.085550in}{1.562857in}}%
\pgfpathlineto{\pgfqpoint{3.022085in}{1.566508in}}%
\pgfpathclose%
\pgfusepath{fill}%
\end{pgfscope}%
\begin{pgfscope}%
\pgfpathrectangle{\pgfqpoint{0.150000in}{0.150000in}}{\pgfqpoint{4.700000in}{3.450000in}}%
\pgfusepath{clip}%
\pgfsetbuttcap%
\pgfsetroundjoin%
\definecolor{currentfill}{rgb}{0.823346,0.679458,0.690855}%
\pgfsetfillcolor{currentfill}%
\pgfsetlinewidth{0.000000pt}%
\definecolor{currentstroke}{rgb}{0.000000,0.000000,0.000000}%
\pgfsetstrokecolor{currentstroke}%
\pgfsetdash{}{0pt}%
\pgfpathmoveto{\pgfqpoint{3.350743in}{1.387956in}}%
\pgfpathlineto{\pgfqpoint{3.413051in}{1.385828in}}%
\pgfpathlineto{\pgfqpoint{3.346652in}{1.429323in}}%
\pgfpathlineto{\pgfqpoint{3.284180in}{1.431856in}}%
\pgfpathclose%
\pgfusepath{fill}%
\end{pgfscope}%
\begin{pgfscope}%
\pgfpathrectangle{\pgfqpoint{0.150000in}{0.150000in}}{\pgfqpoint{4.700000in}{3.450000in}}%
\pgfusepath{clip}%
\pgfsetbuttcap%
\pgfsetroundjoin%
\definecolor{currentfill}{rgb}{0.511780,0.571906,0.656081}%
\pgfsetfillcolor{currentfill}%
\pgfsetlinewidth{0.000000pt}%
\definecolor{currentstroke}{rgb}{0.000000,0.000000,0.000000}%
\pgfsetstrokecolor{currentstroke}%
\pgfsetdash{}{0pt}%
\pgfpathmoveto{\pgfqpoint{1.711216in}{2.669729in}}%
\pgfpathlineto{\pgfqpoint{1.779526in}{2.669764in}}%
\pgfpathlineto{\pgfqpoint{1.714315in}{2.734820in}}%
\pgfpathlineto{\pgfqpoint{1.645785in}{2.735013in}}%
\pgfpathclose%
\pgfusepath{fill}%
\end{pgfscope}%
\begin{pgfscope}%
\pgfpathrectangle{\pgfqpoint{0.150000in}{0.150000in}}{\pgfqpoint{4.700000in}{3.450000in}}%
\pgfusepath{clip}%
\pgfsetbuttcap%
\pgfsetroundjoin%
\definecolor{currentfill}{rgb}{0.549096,0.604626,0.682368}%
\pgfsetfillcolor{currentfill}%
\pgfsetlinewidth{0.000000pt}%
\definecolor{currentstroke}{rgb}{0.000000,0.000000,0.000000}%
\pgfsetstrokecolor{currentstroke}%
\pgfsetdash{}{0pt}%
\pgfpathmoveto{\pgfqpoint{1.776665in}{2.604427in}}%
\pgfpathlineto{\pgfqpoint{1.844757in}{2.604690in}}%
\pgfpathlineto{\pgfqpoint{1.779526in}{2.669764in}}%
\pgfpathlineto{\pgfqpoint{1.711216in}{2.669729in}}%
\pgfpathclose%
\pgfusepath{fill}%
\end{pgfscope}%
\begin{pgfscope}%
\pgfpathrectangle{\pgfqpoint{0.150000in}{0.150000in}}{\pgfqpoint{4.700000in}{3.450000in}}%
\pgfusepath{clip}%
\pgfsetbuttcap%
\pgfsetroundjoin%
\definecolor{currentfill}{rgb}{0.586412,0.637347,0.708655}%
\pgfsetfillcolor{currentfill}%
\pgfsetlinewidth{0.000000pt}%
\definecolor{currentstroke}{rgb}{0.000000,0.000000,0.000000}%
\pgfsetstrokecolor{currentstroke}%
\pgfsetdash{}{0pt}%
\pgfpathmoveto{\pgfqpoint{1.842133in}{2.539106in}}%
\pgfpathlineto{\pgfqpoint{1.910005in}{2.539598in}}%
\pgfpathlineto{\pgfqpoint{1.844757in}{2.604690in}}%
\pgfpathlineto{\pgfqpoint{1.776665in}{2.604427in}}%
\pgfpathclose%
\pgfusepath{fill}%
\end{pgfscope}%
\begin{pgfscope}%
\pgfpathrectangle{\pgfqpoint{0.150000in}{0.150000in}}{\pgfqpoint{4.700000in}{3.450000in}}%
\pgfusepath{clip}%
\pgfsetbuttcap%
\pgfsetroundjoin%
\definecolor{currentfill}{rgb}{0.623729,0.670067,0.734942}%
\pgfsetfillcolor{currentfill}%
\pgfsetlinewidth{0.000000pt}%
\definecolor{currentstroke}{rgb}{0.000000,0.000000,0.000000}%
\pgfsetstrokecolor{currentstroke}%
\pgfsetdash{}{0pt}%
\pgfpathmoveto{\pgfqpoint{1.907619in}{2.473766in}}%
\pgfpathlineto{\pgfqpoint{1.975272in}{2.474487in}}%
\pgfpathlineto{\pgfqpoint{1.910005in}{2.539598in}}%
\pgfpathlineto{\pgfqpoint{1.842133in}{2.539106in}}%
\pgfpathclose%
\pgfusepath{fill}%
\end{pgfscope}%
\begin{pgfscope}%
\pgfpathrectangle{\pgfqpoint{0.150000in}{0.150000in}}{\pgfqpoint{4.700000in}{3.450000in}}%
\pgfusepath{clip}%
\pgfsetbuttcap%
\pgfsetroundjoin%
\definecolor{currentfill}{rgb}{0.661045,0.702788,0.761229}%
\pgfsetfillcolor{currentfill}%
\pgfsetlinewidth{0.000000pt}%
\definecolor{currentstroke}{rgb}{0.000000,0.000000,0.000000}%
\pgfsetstrokecolor{currentstroke}%
\pgfsetdash{}{0pt}%
\pgfpathmoveto{\pgfqpoint{1.973124in}{2.408408in}}%
\pgfpathlineto{\pgfqpoint{2.040558in}{2.409357in}}%
\pgfpathlineto{\pgfqpoint{1.975272in}{2.474487in}}%
\pgfpathlineto{\pgfqpoint{1.907619in}{2.473766in}}%
\pgfpathclose%
\pgfusepath{fill}%
\end{pgfscope}%
\begin{pgfscope}%
\pgfpathrectangle{\pgfqpoint{0.150000in}{0.150000in}}{\pgfqpoint{4.700000in}{3.450000in}}%
\pgfusepath{clip}%
\pgfsetbuttcap%
\pgfsetroundjoin%
\definecolor{currentfill}{rgb}{0.698361,0.735509,0.787515}%
\pgfsetfillcolor{currentfill}%
\pgfsetlinewidth{0.000000pt}%
\definecolor{currentstroke}{rgb}{0.000000,0.000000,0.000000}%
\pgfsetstrokecolor{currentstroke}%
\pgfsetdash{}{0pt}%
\pgfpathmoveto{\pgfqpoint{2.038648in}{2.343031in}}%
\pgfpathlineto{\pgfqpoint{2.105863in}{2.344209in}}%
\pgfpathlineto{\pgfqpoint{2.040558in}{2.409357in}}%
\pgfpathlineto{\pgfqpoint{1.973124in}{2.408408in}}%
\pgfpathclose%
\pgfusepath{fill}%
\end{pgfscope}%
\begin{pgfscope}%
\pgfpathrectangle{\pgfqpoint{0.150000in}{0.150000in}}{\pgfqpoint{4.700000in}{3.450000in}}%
\pgfusepath{clip}%
\pgfsetbuttcap%
\pgfsetroundjoin%
\definecolor{currentfill}{rgb}{0.735677,0.768229,0.813802}%
\pgfsetfillcolor{currentfill}%
\pgfsetlinewidth{0.000000pt}%
\definecolor{currentstroke}{rgb}{0.000000,0.000000,0.000000}%
\pgfsetstrokecolor{currentstroke}%
\pgfsetdash{}{0pt}%
\pgfpathmoveto{\pgfqpoint{2.104191in}{2.277636in}}%
\pgfpathlineto{\pgfqpoint{2.171186in}{2.279043in}}%
\pgfpathlineto{\pgfqpoint{2.105863in}{2.344209in}}%
\pgfpathlineto{\pgfqpoint{2.038648in}{2.343031in}}%
\pgfpathclose%
\pgfusepath{fill}%
\end{pgfscope}%
\begin{pgfscope}%
\pgfpathrectangle{\pgfqpoint{0.150000in}{0.150000in}}{\pgfqpoint{4.700000in}{3.450000in}}%
\pgfusepath{clip}%
\pgfsetbuttcap%
\pgfsetroundjoin%
\definecolor{currentfill}{rgb}{0.772993,0.800950,0.840089}%
\pgfsetfillcolor{currentfill}%
\pgfsetlinewidth{0.000000pt}%
\definecolor{currentstroke}{rgb}{0.000000,0.000000,0.000000}%
\pgfsetstrokecolor{currentstroke}%
\pgfsetdash{}{0pt}%
\pgfpathmoveto{\pgfqpoint{2.169752in}{2.212221in}}%
\pgfpathlineto{\pgfqpoint{2.236527in}{2.213857in}}%
\pgfpathlineto{\pgfqpoint{2.171186in}{2.279043in}}%
\pgfpathlineto{\pgfqpoint{2.104191in}{2.277636in}}%
\pgfpathclose%
\pgfusepath{fill}%
\end{pgfscope}%
\begin{pgfscope}%
\pgfpathrectangle{\pgfqpoint{0.150000in}{0.150000in}}{\pgfqpoint{4.700000in}{3.450000in}}%
\pgfusepath{clip}%
\pgfsetbuttcap%
\pgfsetroundjoin%
\definecolor{currentfill}{rgb}{0.914522,0.844899,0.850414}%
\pgfsetfillcolor{currentfill}%
\pgfsetlinewidth{0.000000pt}%
\definecolor{currentstroke}{rgb}{0.000000,0.000000,0.000000}%
\pgfsetstrokecolor{currentstroke}%
\pgfsetdash{}{0pt}%
\pgfpathmoveto{\pgfqpoint{2.891923in}{1.614438in}}%
\pgfpathlineto{\pgfqpoint{2.956058in}{1.610349in}}%
\pgfpathlineto{\pgfqpoint{2.890202in}{1.653955in}}%
\pgfpathlineto{\pgfqpoint{2.825896in}{1.658452in}}%
\pgfpathclose%
\pgfusepath{fill}%
\end{pgfscope}%
\begin{pgfscope}%
\pgfpathrectangle{\pgfqpoint{0.150000in}{0.150000in}}{\pgfqpoint{4.700000in}{3.450000in}}%
\pgfusepath{clip}%
\pgfsetbuttcap%
\pgfsetroundjoin%
\definecolor{currentfill}{rgb}{0.865135,0.755285,0.763986}%
\pgfsetfillcolor{currentfill}%
\pgfsetlinewidth{0.000000pt}%
\definecolor{currentstroke}{rgb}{0.000000,0.000000,0.000000}%
\pgfsetstrokecolor{currentstroke}%
\pgfsetdash{}{0pt}%
\pgfpathmoveto{\pgfqpoint{3.154667in}{1.478598in}}%
\pgfpathlineto{\pgfqpoint{3.217794in}{1.475639in}}%
\pgfpathlineto{\pgfqpoint{3.151584in}{1.519306in}}%
\pgfpathlineto{\pgfqpoint{3.088287in}{1.522551in}}%
\pgfpathclose%
\pgfusepath{fill}%
\end{pgfscope}%
\begin{pgfscope}%
\pgfpathrectangle{\pgfqpoint{0.150000in}{0.150000in}}{\pgfqpoint{4.700000in}{3.450000in}}%
\pgfusepath{clip}%
\pgfsetbuttcap%
\pgfsetroundjoin%
\definecolor{currentfill}{rgb}{0.810309,0.833670,0.866376}%
\pgfsetfillcolor{currentfill}%
\pgfsetlinewidth{0.000000pt}%
\definecolor{currentstroke}{rgb}{0.000000,0.000000,0.000000}%
\pgfsetstrokecolor{currentstroke}%
\pgfsetdash{}{0pt}%
\pgfpathmoveto{\pgfqpoint{2.235332in}{2.146789in}}%
\pgfpathlineto{\pgfqpoint{2.301887in}{2.148654in}}%
\pgfpathlineto{\pgfqpoint{2.236527in}{2.213857in}}%
\pgfpathlineto{\pgfqpoint{2.169752in}{2.212221in}}%
\pgfpathclose%
\pgfusepath{fill}%
\end{pgfscope}%
\begin{pgfscope}%
\pgfpathrectangle{\pgfqpoint{0.150000in}{0.150000in}}{\pgfqpoint{4.700000in}{3.450000in}}%
\pgfusepath{clip}%
\pgfsetbuttcap%
\pgfsetroundjoin%
\definecolor{currentfill}{rgb}{0.815748,0.665671,0.677558}%
\pgfsetfillcolor{currentfill}%
\pgfsetlinewidth{0.000000pt}%
\definecolor{currentstroke}{rgb}{0.000000,0.000000,0.000000}%
\pgfsetstrokecolor{currentstroke}%
\pgfsetdash{}{0pt}%
\pgfpathmoveto{\pgfqpoint{3.417484in}{1.343939in}}%
\pgfpathlineto{\pgfqpoint{3.479623in}{1.342097in}}%
\pgfpathlineto{\pgfqpoint{3.413051in}{1.385828in}}%
\pgfpathlineto{\pgfqpoint{3.350743in}{1.387956in}}%
\pgfpathclose%
\pgfusepath{fill}%
\end{pgfscope}%
\begin{pgfscope}%
\pgfpathrectangle{\pgfqpoint{0.150000in}{0.150000in}}{\pgfqpoint{4.700000in}{3.450000in}}%
\pgfusepath{clip}%
\pgfsetbuttcap%
\pgfsetroundjoin%
\definecolor{currentfill}{rgb}{0.847626,0.866391,0.892662}%
\pgfsetfillcolor{currentfill}%
\pgfsetlinewidth{0.000000pt}%
\definecolor{currentstroke}{rgb}{0.000000,0.000000,0.000000}%
\pgfsetstrokecolor{currentstroke}%
\pgfsetdash{}{0pt}%
\pgfpathmoveto{\pgfqpoint{2.300931in}{2.081337in}}%
\pgfpathlineto{\pgfqpoint{2.367266in}{2.083431in}}%
\pgfpathlineto{\pgfqpoint{2.301887in}{2.148654in}}%
\pgfpathlineto{\pgfqpoint{2.235332in}{2.146789in}}%
\pgfpathclose%
\pgfusepath{fill}%
\end{pgfscope}%
\begin{pgfscope}%
\pgfpathrectangle{\pgfqpoint{0.150000in}{0.150000in}}{\pgfqpoint{4.700000in}{3.450000in}}%
\pgfusepath{clip}%
\pgfsetbuttcap%
\pgfsetroundjoin%
\definecolor{currentfill}{rgb}{0.884942,0.899112,0.918949}%
\pgfsetfillcolor{currentfill}%
\pgfsetlinewidth{0.000000pt}%
\definecolor{currentstroke}{rgb}{0.000000,0.000000,0.000000}%
\pgfsetstrokecolor{currentstroke}%
\pgfsetdash{}{0pt}%
\pgfpathmoveto{\pgfqpoint{2.366549in}{2.015867in}}%
\pgfpathlineto{\pgfqpoint{2.432663in}{2.018191in}}%
\pgfpathlineto{\pgfqpoint{2.367266in}{2.083431in}}%
\pgfpathlineto{\pgfqpoint{2.300931in}{2.081337in}}%
\pgfpathclose%
\pgfusepath{fill}%
\end{pgfscope}%
\begin{pgfscope}%
\pgfpathrectangle{\pgfqpoint{0.150000in}{0.150000in}}{\pgfqpoint{4.700000in}{3.450000in}}%
\pgfusepath{clip}%
\pgfsetbuttcap%
\pgfsetroundjoin%
\definecolor{currentfill}{rgb}{0.922258,0.931832,0.945236}%
\pgfsetfillcolor{currentfill}%
\pgfsetlinewidth{0.000000pt}%
\definecolor{currentstroke}{rgb}{0.000000,0.000000,0.000000}%
\pgfsetstrokecolor{currentstroke}%
\pgfsetdash{}{0pt}%
\pgfpathmoveto{\pgfqpoint{2.432185in}{1.950378in}}%
\pgfpathlineto{\pgfqpoint{2.498079in}{1.952931in}}%
\pgfpathlineto{\pgfqpoint{2.432663in}{2.018191in}}%
\pgfpathlineto{\pgfqpoint{2.366549in}{2.015867in}}%
\pgfpathclose%
\pgfusepath{fill}%
\end{pgfscope}%
\begin{pgfscope}%
\pgfpathrectangle{\pgfqpoint{0.150000in}{0.150000in}}{\pgfqpoint{4.700000in}{3.450000in}}%
\pgfusepath{clip}%
\pgfsetbuttcap%
\pgfsetroundjoin%
\definecolor{currentfill}{rgb}{0.959574,0.964553,0.971523}%
\pgfsetfillcolor{currentfill}%
\pgfsetlinewidth{0.000000pt}%
\definecolor{currentstroke}{rgb}{0.000000,0.000000,0.000000}%
\pgfsetstrokecolor{currentstroke}%
\pgfsetdash{}{0pt}%
\pgfpathmoveto{\pgfqpoint{2.497840in}{1.884870in}}%
\pgfpathlineto{\pgfqpoint{2.563514in}{1.887653in}}%
\pgfpathlineto{\pgfqpoint{2.498079in}{1.952931in}}%
\pgfpathlineto{\pgfqpoint{2.432185in}{1.950378in}}%
\pgfpathclose%
\pgfusepath{fill}%
\end{pgfscope}%
\begin{pgfscope}%
\pgfpathrectangle{\pgfqpoint{0.150000in}{0.150000in}}{\pgfqpoint{4.700000in}{3.450000in}}%
\pgfusepath{clip}%
\pgfsetbuttcap%
\pgfsetroundjoin%
\definecolor{currentfill}{rgb}{0.996890,0.997273,0.997809}%
\pgfsetfillcolor{currentfill}%
\pgfsetlinewidth{0.000000pt}%
\definecolor{currentstroke}{rgb}{0.000000,0.000000,0.000000}%
\pgfsetstrokecolor{currentstroke}%
\pgfsetdash{}{0pt}%
\pgfpathmoveto{\pgfqpoint{2.563514in}{1.819344in}}%
\pgfpathlineto{\pgfqpoint{2.628967in}{1.822357in}}%
\pgfpathlineto{\pgfqpoint{2.563514in}{1.887653in}}%
\pgfpathlineto{\pgfqpoint{2.497840in}{1.884870in}}%
\pgfpathclose%
\pgfusepath{fill}%
\end{pgfscope}%
\begin{pgfscope}%
\pgfpathrectangle{\pgfqpoint{0.150000in}{0.150000in}}{\pgfqpoint{4.700000in}{3.450000in}}%
\pgfusepath{clip}%
\pgfsetbuttcap%
\pgfsetroundjoin%
\definecolor{currentfill}{rgb}{0.975306,0.955193,0.956786}%
\pgfsetfillcolor{currentfill}%
\pgfsetlinewidth{0.000000pt}%
\definecolor{currentstroke}{rgb}{0.000000,0.000000,0.000000}%
\pgfsetstrokecolor{currentstroke}%
\pgfsetdash{}{0pt}%
\pgfpathmoveto{\pgfqpoint{2.629206in}{1.753799in}}%
\pgfpathlineto{\pgfqpoint{2.694438in}{1.757042in}}%
\pgfpathlineto{\pgfqpoint{2.628967in}{1.822357in}}%
\pgfpathlineto{\pgfqpoint{2.563514in}{1.819344in}}%
\pgfpathclose%
\pgfusepath{fill}%
\end{pgfscope}%
\begin{pgfscope}%
\pgfpathrectangle{\pgfqpoint{0.150000in}{0.150000in}}{\pgfqpoint{4.700000in}{3.450000in}}%
\pgfusepath{clip}%
\pgfsetbuttcap%
\pgfsetroundjoin%
\definecolor{currentfill}{rgb}{0.906924,0.831112,0.837117}%
\pgfsetfillcolor{currentfill}%
\pgfsetlinewidth{0.000000pt}%
\definecolor{currentstroke}{rgb}{0.000000,0.000000,0.000000}%
\pgfsetstrokecolor{currentstroke}%
\pgfsetdash{}{0pt}%
\pgfpathmoveto{\pgfqpoint{2.958121in}{1.570188in}}%
\pgfpathlineto{\pgfqpoint{3.022085in}{1.566508in}}%
\pgfpathlineto{\pgfqpoint{2.956058in}{1.610349in}}%
\pgfpathlineto{\pgfqpoint{2.891923in}{1.614438in}}%
\pgfpathclose%
\pgfusepath{fill}%
\end{pgfscope}%
\begin{pgfscope}%
\pgfpathrectangle{\pgfqpoint{0.150000in}{0.150000in}}{\pgfqpoint{4.700000in}{3.450000in}}%
\pgfusepath{clip}%
\pgfsetbuttcap%
\pgfsetroundjoin%
\definecolor{currentfill}{rgb}{0.857537,0.741498,0.750689}%
\pgfsetfillcolor{currentfill}%
\pgfsetlinewidth{0.000000pt}%
\definecolor{currentstroke}{rgb}{0.000000,0.000000,0.000000}%
\pgfsetstrokecolor{currentstroke}%
\pgfsetdash{}{0pt}%
\pgfpathmoveto{\pgfqpoint{3.221220in}{1.434408in}}%
\pgfpathlineto{\pgfqpoint{3.284180in}{1.431856in}}%
\pgfpathlineto{\pgfqpoint{3.217794in}{1.475639in}}%
\pgfpathlineto{\pgfqpoint{3.154667in}{1.478598in}}%
\pgfpathclose%
\pgfusepath{fill}%
\end{pgfscope}%
\begin{pgfscope}%
\pgfpathrectangle{\pgfqpoint{0.150000in}{0.150000in}}{\pgfqpoint{4.700000in}{3.450000in}}%
\pgfusepath{clip}%
\pgfsetbuttcap%
\pgfsetroundjoin%
\definecolor{currentfill}{rgb}{0.808150,0.651884,0.664262}%
\pgfsetfillcolor{currentfill}%
\pgfsetlinewidth{0.000000pt}%
\definecolor{currentstroke}{rgb}{0.000000,0.000000,0.000000}%
\pgfsetstrokecolor{currentstroke}%
\pgfsetdash{}{0pt}%
\pgfpathmoveto{\pgfqpoint{3.484402in}{1.299805in}}%
\pgfpathlineto{\pgfqpoint{3.546379in}{1.298367in}}%
\pgfpathlineto{\pgfqpoint{3.479623in}{1.342097in}}%
\pgfpathlineto{\pgfqpoint{3.417484in}{1.343939in}}%
\pgfpathclose%
\pgfusepath{fill}%
\end{pgfscope}%
\begin{pgfscope}%
\pgfpathrectangle{\pgfqpoint{0.150000in}{0.150000in}}{\pgfqpoint{4.700000in}{3.450000in}}%
\pgfusepath{clip}%
\pgfsetbuttcap%
\pgfsetroundjoin%
\definecolor{currentfill}{rgb}{0.960110,0.927619,0.930193}%
\pgfsetfillcolor{currentfill}%
\pgfsetlinewidth{0.000000pt}%
\definecolor{currentstroke}{rgb}{0.000000,0.000000,0.000000}%
\pgfsetstrokecolor{currentstroke}%
\pgfsetdash{}{0pt}%
\pgfpathmoveto{\pgfqpoint{2.695053in}{1.707293in}}%
\pgfpathlineto{\pgfqpoint{2.760044in}{1.702471in}}%
\pgfpathlineto{\pgfqpoint{2.694438in}{1.757042in}}%
\pgfpathlineto{\pgfqpoint{2.629206in}{1.753799in}}%
\pgfpathclose%
\pgfusepath{fill}%
\end{pgfscope}%
\begin{pgfscope}%
\pgfpathrectangle{\pgfqpoint{0.150000in}{0.150000in}}{\pgfqpoint{4.700000in}{3.450000in}}%
\pgfusepath{clip}%
\pgfsetbuttcap%
\pgfsetroundjoin%
\definecolor{currentfill}{rgb}{0.474464,0.539185,0.629795}%
\pgfsetfillcolor{currentfill}%
\pgfsetlinewidth{0.000000pt}%
\definecolor{currentstroke}{rgb}{0.000000,0.000000,0.000000}%
\pgfsetstrokecolor{currentstroke}%
\pgfsetdash{}{0pt}%
\pgfpathmoveto{\pgfqpoint{1.642424in}{2.669693in}}%
\pgfpathlineto{\pgfqpoint{1.711216in}{2.669729in}}%
\pgfpathlineto{\pgfqpoint{1.645785in}{2.735013in}}%
\pgfpathlineto{\pgfqpoint{1.576773in}{2.735207in}}%
\pgfpathclose%
\pgfusepath{fill}%
\end{pgfscope}%
\begin{pgfscope}%
\pgfpathrectangle{\pgfqpoint{0.150000in}{0.150000in}}{\pgfqpoint{4.700000in}{3.450000in}}%
\pgfusepath{clip}%
\pgfsetbuttcap%
\pgfsetroundjoin%
\definecolor{currentfill}{rgb}{0.511780,0.571906,0.656081}%
\pgfsetfillcolor{currentfill}%
\pgfsetlinewidth{0.000000pt}%
\definecolor{currentstroke}{rgb}{0.000000,0.000000,0.000000}%
\pgfsetstrokecolor{currentstroke}%
\pgfsetdash{}{0pt}%
\pgfpathmoveto{\pgfqpoint{1.708094in}{2.604161in}}%
\pgfpathlineto{\pgfqpoint{1.776665in}{2.604427in}}%
\pgfpathlineto{\pgfqpoint{1.711216in}{2.669729in}}%
\pgfpathlineto{\pgfqpoint{1.642424in}{2.669693in}}%
\pgfpathclose%
\pgfusepath{fill}%
\end{pgfscope}%
\begin{pgfscope}%
\pgfpathrectangle{\pgfqpoint{0.150000in}{0.150000in}}{\pgfqpoint{4.700000in}{3.450000in}}%
\pgfusepath{clip}%
\pgfsetbuttcap%
\pgfsetroundjoin%
\definecolor{currentfill}{rgb}{0.899326,0.817325,0.823820}%
\pgfsetfillcolor{currentfill}%
\pgfsetlinewidth{0.000000pt}%
\definecolor{currentstroke}{rgb}{0.000000,0.000000,0.000000}%
\pgfsetstrokecolor{currentstroke}%
\pgfsetdash{}{0pt}%
\pgfpathmoveto{\pgfqpoint{3.024496in}{1.525942in}}%
\pgfpathlineto{\pgfqpoint{3.088287in}{1.522551in}}%
\pgfpathlineto{\pgfqpoint{3.022085in}{1.566508in}}%
\pgfpathlineto{\pgfqpoint{2.958121in}{1.570188in}}%
\pgfpathclose%
\pgfusepath{fill}%
\end{pgfscope}%
\begin{pgfscope}%
\pgfpathrectangle{\pgfqpoint{0.150000in}{0.150000in}}{\pgfqpoint{4.700000in}{3.450000in}}%
\pgfusepath{clip}%
\pgfsetbuttcap%
\pgfsetroundjoin%
\definecolor{currentfill}{rgb}{0.948713,0.906939,0.910248}%
\pgfsetfillcolor{currentfill}%
\pgfsetlinewidth{0.000000pt}%
\definecolor{currentstroke}{rgb}{0.000000,0.000000,0.000000}%
\pgfsetstrokecolor{currentstroke}%
\pgfsetdash{}{0pt}%
\pgfpathmoveto{\pgfqpoint{2.761081in}{1.662984in}}%
\pgfpathlineto{\pgfqpoint{2.825896in}{1.658452in}}%
\pgfpathlineto{\pgfqpoint{2.760044in}{1.702471in}}%
\pgfpathlineto{\pgfqpoint{2.695053in}{1.707293in}}%
\pgfpathclose%
\pgfusepath{fill}%
\end{pgfscope}%
\begin{pgfscope}%
\pgfpathrectangle{\pgfqpoint{0.150000in}{0.150000in}}{\pgfqpoint{4.700000in}{3.450000in}}%
\pgfusepath{clip}%
\pgfsetbuttcap%
\pgfsetroundjoin%
\definecolor{currentfill}{rgb}{0.849939,0.727711,0.737393}%
\pgfsetfillcolor{currentfill}%
\pgfsetlinewidth{0.000000pt}%
\definecolor{currentstroke}{rgb}{0.000000,0.000000,0.000000}%
\pgfsetstrokecolor{currentstroke}%
\pgfsetdash{}{0pt}%
\pgfpathmoveto{\pgfqpoint{3.287949in}{1.390101in}}%
\pgfpathlineto{\pgfqpoint{3.350743in}{1.387956in}}%
\pgfpathlineto{\pgfqpoint{3.284180in}{1.431856in}}%
\pgfpathlineto{\pgfqpoint{3.221220in}{1.434408in}}%
\pgfpathclose%
\pgfusepath{fill}%
\end{pgfscope}%
\begin{pgfscope}%
\pgfpathrectangle{\pgfqpoint{0.150000in}{0.150000in}}{\pgfqpoint{4.700000in}{3.450000in}}%
\pgfusepath{clip}%
\pgfsetbuttcap%
\pgfsetroundjoin%
\definecolor{currentfill}{rgb}{0.549096,0.604626,0.682368}%
\pgfsetfillcolor{currentfill}%
\pgfsetlinewidth{0.000000pt}%
\definecolor{currentstroke}{rgb}{0.000000,0.000000,0.000000}%
\pgfsetstrokecolor{currentstroke}%
\pgfsetdash{}{0pt}%
\pgfpathmoveto{\pgfqpoint{1.773782in}{2.538610in}}%
\pgfpathlineto{\pgfqpoint{1.842133in}{2.539106in}}%
\pgfpathlineto{\pgfqpoint{1.776665in}{2.604427in}}%
\pgfpathlineto{\pgfqpoint{1.708094in}{2.604161in}}%
\pgfpathclose%
\pgfusepath{fill}%
\end{pgfscope}%
\begin{pgfscope}%
\pgfpathrectangle{\pgfqpoint{0.150000in}{0.150000in}}{\pgfqpoint{4.700000in}{3.450000in}}%
\pgfusepath{clip}%
\pgfsetbuttcap%
\pgfsetroundjoin%
\definecolor{currentfill}{rgb}{0.586412,0.637347,0.708655}%
\pgfsetfillcolor{currentfill}%
\pgfsetlinewidth{0.000000pt}%
\definecolor{currentstroke}{rgb}{0.000000,0.000000,0.000000}%
\pgfsetstrokecolor{currentstroke}%
\pgfsetdash{}{0pt}%
\pgfpathmoveto{\pgfqpoint{1.839490in}{2.473041in}}%
\pgfpathlineto{\pgfqpoint{1.907619in}{2.473766in}}%
\pgfpathlineto{\pgfqpoint{1.842133in}{2.539106in}}%
\pgfpathlineto{\pgfqpoint{1.773782in}{2.538610in}}%
\pgfpathclose%
\pgfusepath{fill}%
\end{pgfscope}%
\begin{pgfscope}%
\pgfpathrectangle{\pgfqpoint{0.150000in}{0.150000in}}{\pgfqpoint{4.700000in}{3.450000in}}%
\pgfusepath{clip}%
\pgfsetbuttcap%
\pgfsetroundjoin%
\definecolor{currentfill}{rgb}{0.623729,0.670067,0.734942}%
\pgfsetfillcolor{currentfill}%
\pgfsetlinewidth{0.000000pt}%
\definecolor{currentstroke}{rgb}{0.000000,0.000000,0.000000}%
\pgfsetstrokecolor{currentstroke}%
\pgfsetdash{}{0pt}%
\pgfpathmoveto{\pgfqpoint{1.905216in}{2.407452in}}%
\pgfpathlineto{\pgfqpoint{1.973124in}{2.408408in}}%
\pgfpathlineto{\pgfqpoint{1.907619in}{2.473766in}}%
\pgfpathlineto{\pgfqpoint{1.839490in}{2.473041in}}%
\pgfpathclose%
\pgfusepath{fill}%
\end{pgfscope}%
\begin{pgfscope}%
\pgfpathrectangle{\pgfqpoint{0.150000in}{0.150000in}}{\pgfqpoint{4.700000in}{3.450000in}}%
\pgfusepath{clip}%
\pgfsetbuttcap%
\pgfsetroundjoin%
\definecolor{currentfill}{rgb}{0.661045,0.702788,0.761229}%
\pgfsetfillcolor{currentfill}%
\pgfsetlinewidth{0.000000pt}%
\definecolor{currentstroke}{rgb}{0.000000,0.000000,0.000000}%
\pgfsetstrokecolor{currentstroke}%
\pgfsetdash{}{0pt}%
\pgfpathmoveto{\pgfqpoint{1.970961in}{2.341845in}}%
\pgfpathlineto{\pgfqpoint{2.038648in}{2.343031in}}%
\pgfpathlineto{\pgfqpoint{1.973124in}{2.408408in}}%
\pgfpathlineto{\pgfqpoint{1.905216in}{2.407452in}}%
\pgfpathclose%
\pgfusepath{fill}%
\end{pgfscope}%
\begin{pgfscope}%
\pgfpathrectangle{\pgfqpoint{0.150000in}{0.150000in}}{\pgfqpoint{4.700000in}{3.450000in}}%
\pgfusepath{clip}%
\pgfsetbuttcap%
\pgfsetroundjoin%
\definecolor{currentfill}{rgb}{0.698361,0.735509,0.787515}%
\pgfsetfillcolor{currentfill}%
\pgfsetlinewidth{0.000000pt}%
\definecolor{currentstroke}{rgb}{0.000000,0.000000,0.000000}%
\pgfsetstrokecolor{currentstroke}%
\pgfsetdash{}{0pt}%
\pgfpathmoveto{\pgfqpoint{2.036724in}{2.276219in}}%
\pgfpathlineto{\pgfqpoint{2.104191in}{2.277636in}}%
\pgfpathlineto{\pgfqpoint{2.038648in}{2.343031in}}%
\pgfpathlineto{\pgfqpoint{1.970961in}{2.341845in}}%
\pgfpathclose%
\pgfusepath{fill}%
\end{pgfscope}%
\begin{pgfscope}%
\pgfpathrectangle{\pgfqpoint{0.150000in}{0.150000in}}{\pgfqpoint{4.700000in}{3.450000in}}%
\pgfusepath{clip}%
\pgfsetbuttcap%
\pgfsetroundjoin%
\definecolor{currentfill}{rgb}{0.735677,0.768229,0.813802}%
\pgfsetfillcolor{currentfill}%
\pgfsetlinewidth{0.000000pt}%
\definecolor{currentstroke}{rgb}{0.000000,0.000000,0.000000}%
\pgfsetstrokecolor{currentstroke}%
\pgfsetdash{}{0pt}%
\pgfpathmoveto{\pgfqpoint{2.102507in}{2.210574in}}%
\pgfpathlineto{\pgfqpoint{2.169752in}{2.212221in}}%
\pgfpathlineto{\pgfqpoint{2.104191in}{2.277636in}}%
\pgfpathlineto{\pgfqpoint{2.036724in}{2.276219in}}%
\pgfpathclose%
\pgfusepath{fill}%
\end{pgfscope}%
\begin{pgfscope}%
\pgfpathrectangle{\pgfqpoint{0.150000in}{0.150000in}}{\pgfqpoint{4.700000in}{3.450000in}}%
\pgfusepath{clip}%
\pgfsetbuttcap%
\pgfsetroundjoin%
\definecolor{currentfill}{rgb}{0.772993,0.800950,0.840089}%
\pgfsetfillcolor{currentfill}%
\pgfsetlinewidth{0.000000pt}%
\definecolor{currentstroke}{rgb}{0.000000,0.000000,0.000000}%
\pgfsetstrokecolor{currentstroke}%
\pgfsetdash{}{0pt}%
\pgfpathmoveto{\pgfqpoint{2.168309in}{2.144910in}}%
\pgfpathlineto{\pgfqpoint{2.235332in}{2.146789in}}%
\pgfpathlineto{\pgfqpoint{2.169752in}{2.212221in}}%
\pgfpathlineto{\pgfqpoint{2.102507in}{2.210574in}}%
\pgfpathclose%
\pgfusepath{fill}%
\end{pgfscope}%
\begin{pgfscope}%
\pgfpathrectangle{\pgfqpoint{0.150000in}{0.150000in}}{\pgfqpoint{4.700000in}{3.450000in}}%
\pgfusepath{clip}%
\pgfsetbuttcap%
\pgfsetroundjoin%
\definecolor{currentfill}{rgb}{0.810309,0.833670,0.866376}%
\pgfsetfillcolor{currentfill}%
\pgfsetlinewidth{0.000000pt}%
\definecolor{currentstroke}{rgb}{0.000000,0.000000,0.000000}%
\pgfsetstrokecolor{currentstroke}%
\pgfsetdash{}{0pt}%
\pgfpathmoveto{\pgfqpoint{2.234129in}{2.079228in}}%
\pgfpathlineto{\pgfqpoint{2.300931in}{2.081337in}}%
\pgfpathlineto{\pgfqpoint{2.235332in}{2.146789in}}%
\pgfpathlineto{\pgfqpoint{2.168309in}{2.144910in}}%
\pgfpathclose%
\pgfusepath{fill}%
\end{pgfscope}%
\begin{pgfscope}%
\pgfpathrectangle{\pgfqpoint{0.150000in}{0.150000in}}{\pgfqpoint{4.700000in}{3.450000in}}%
\pgfusepath{clip}%
\pgfsetbuttcap%
\pgfsetroundjoin%
\definecolor{currentfill}{rgb}{0.847626,0.866391,0.892662}%
\pgfsetfillcolor{currentfill}%
\pgfsetlinewidth{0.000000pt}%
\definecolor{currentstroke}{rgb}{0.000000,0.000000,0.000000}%
\pgfsetstrokecolor{currentstroke}%
\pgfsetdash{}{0pt}%
\pgfpathmoveto{\pgfqpoint{2.299968in}{2.013527in}}%
\pgfpathlineto{\pgfqpoint{2.366549in}{2.015867in}}%
\pgfpathlineto{\pgfqpoint{2.300931in}{2.081337in}}%
\pgfpathlineto{\pgfqpoint{2.234129in}{2.079228in}}%
\pgfpathclose%
\pgfusepath{fill}%
\end{pgfscope}%
\begin{pgfscope}%
\pgfpathrectangle{\pgfqpoint{0.150000in}{0.150000in}}{\pgfqpoint{4.700000in}{3.450000in}}%
\pgfusepath{clip}%
\pgfsetbuttcap%
\pgfsetroundjoin%
\definecolor{currentfill}{rgb}{0.941115,0.893153,0.896952}%
\pgfsetfillcolor{currentfill}%
\pgfsetlinewidth{0.000000pt}%
\definecolor{currentstroke}{rgb}{0.000000,0.000000,0.000000}%
\pgfsetstrokecolor{currentstroke}%
\pgfsetdash{}{0pt}%
\pgfpathmoveto{\pgfqpoint{2.827281in}{1.618559in}}%
\pgfpathlineto{\pgfqpoint{2.891923in}{1.614438in}}%
\pgfpathlineto{\pgfqpoint{2.825896in}{1.658452in}}%
\pgfpathlineto{\pgfqpoint{2.761081in}{1.662984in}}%
\pgfpathclose%
\pgfusepath{fill}%
\end{pgfscope}%
\begin{pgfscope}%
\pgfpathrectangle{\pgfqpoint{0.150000in}{0.150000in}}{\pgfqpoint{4.700000in}{3.450000in}}%
\pgfusepath{clip}%
\pgfsetbuttcap%
\pgfsetroundjoin%
\definecolor{currentfill}{rgb}{0.891728,0.803539,0.810524}%
\pgfsetfillcolor{currentfill}%
\pgfsetlinewidth{0.000000pt}%
\definecolor{currentstroke}{rgb}{0.000000,0.000000,0.000000}%
\pgfsetstrokecolor{currentstroke}%
\pgfsetdash{}{0pt}%
\pgfpathmoveto{\pgfqpoint{3.091042in}{1.481460in}}%
\pgfpathlineto{\pgfqpoint{3.154667in}{1.478598in}}%
\pgfpathlineto{\pgfqpoint{3.088287in}{1.522551in}}%
\pgfpathlineto{\pgfqpoint{3.024496in}{1.525942in}}%
\pgfpathclose%
\pgfusepath{fill}%
\end{pgfscope}%
\begin{pgfscope}%
\pgfpathrectangle{\pgfqpoint{0.150000in}{0.150000in}}{\pgfqpoint{4.700000in}{3.450000in}}%
\pgfusepath{clip}%
\pgfsetbuttcap%
\pgfsetroundjoin%
\definecolor{currentfill}{rgb}{0.842341,0.713925,0.724096}%
\pgfsetfillcolor{currentfill}%
\pgfsetlinewidth{0.000000pt}%
\definecolor{currentstroke}{rgb}{0.000000,0.000000,0.000000}%
\pgfsetstrokecolor{currentstroke}%
\pgfsetdash{}{0pt}%
\pgfpathmoveto{\pgfqpoint{3.354855in}{1.345676in}}%
\pgfpathlineto{\pgfqpoint{3.417484in}{1.343939in}}%
\pgfpathlineto{\pgfqpoint{3.350743in}{1.387956in}}%
\pgfpathlineto{\pgfqpoint{3.287949in}{1.390101in}}%
\pgfpathclose%
\pgfusepath{fill}%
\end{pgfscope}%
\begin{pgfscope}%
\pgfpathrectangle{\pgfqpoint{0.150000in}{0.150000in}}{\pgfqpoint{4.700000in}{3.450000in}}%
\pgfusepath{clip}%
\pgfsetbuttcap%
\pgfsetroundjoin%
\definecolor{currentfill}{rgb}{0.884942,0.899112,0.918949}%
\pgfsetfillcolor{currentfill}%
\pgfsetlinewidth{0.000000pt}%
\definecolor{currentstroke}{rgb}{0.000000,0.000000,0.000000}%
\pgfsetstrokecolor{currentstroke}%
\pgfsetdash{}{0pt}%
\pgfpathmoveto{\pgfqpoint{2.365826in}{1.947807in}}%
\pgfpathlineto{\pgfqpoint{2.432185in}{1.950378in}}%
\pgfpathlineto{\pgfqpoint{2.366549in}{2.015867in}}%
\pgfpathlineto{\pgfqpoint{2.299968in}{2.013527in}}%
\pgfpathclose%
\pgfusepath{fill}%
\end{pgfscope}%
\begin{pgfscope}%
\pgfpathrectangle{\pgfqpoint{0.150000in}{0.150000in}}{\pgfqpoint{4.700000in}{3.450000in}}%
\pgfusepath{clip}%
\pgfsetbuttcap%
\pgfsetroundjoin%
\definecolor{currentfill}{rgb}{0.922258,0.931832,0.945236}%
\pgfsetfillcolor{currentfill}%
\pgfsetlinewidth{0.000000pt}%
\definecolor{currentstroke}{rgb}{0.000000,0.000000,0.000000}%
\pgfsetstrokecolor{currentstroke}%
\pgfsetdash{}{0pt}%
\pgfpathmoveto{\pgfqpoint{2.431703in}{1.882068in}}%
\pgfpathlineto{\pgfqpoint{2.497840in}{1.884870in}}%
\pgfpathlineto{\pgfqpoint{2.432185in}{1.950378in}}%
\pgfpathlineto{\pgfqpoint{2.365826in}{1.947807in}}%
\pgfpathclose%
\pgfusepath{fill}%
\end{pgfscope}%
\begin{pgfscope}%
\pgfpathrectangle{\pgfqpoint{0.150000in}{0.150000in}}{\pgfqpoint{4.700000in}{3.450000in}}%
\pgfusepath{clip}%
\pgfsetbuttcap%
\pgfsetroundjoin%
\definecolor{currentfill}{rgb}{0.959574,0.964553,0.971523}%
\pgfsetfillcolor{currentfill}%
\pgfsetlinewidth{0.000000pt}%
\definecolor{currentstroke}{rgb}{0.000000,0.000000,0.000000}%
\pgfsetstrokecolor{currentstroke}%
\pgfsetdash{}{0pt}%
\pgfpathmoveto{\pgfqpoint{2.497599in}{1.816310in}}%
\pgfpathlineto{\pgfqpoint{2.563514in}{1.819344in}}%
\pgfpathlineto{\pgfqpoint{2.497840in}{1.884870in}}%
\pgfpathlineto{\pgfqpoint{2.431703in}{1.882068in}}%
\pgfpathclose%
\pgfusepath{fill}%
\end{pgfscope}%
\begin{pgfscope}%
\pgfpathrectangle{\pgfqpoint{0.150000in}{0.150000in}}{\pgfqpoint{4.700000in}{3.450000in}}%
\pgfusepath{clip}%
\pgfsetbuttcap%
\pgfsetroundjoin%
\definecolor{currentfill}{rgb}{0.996890,0.997273,0.997809}%
\pgfsetfillcolor{currentfill}%
\pgfsetlinewidth{0.000000pt}%
\definecolor{currentstroke}{rgb}{0.000000,0.000000,0.000000}%
\pgfsetstrokecolor{currentstroke}%
\pgfsetdash{}{0pt}%
\pgfpathmoveto{\pgfqpoint{2.563514in}{1.750533in}}%
\pgfpathlineto{\pgfqpoint{2.629206in}{1.753799in}}%
\pgfpathlineto{\pgfqpoint{2.563514in}{1.819344in}}%
\pgfpathlineto{\pgfqpoint{2.497599in}{1.816310in}}%
\pgfpathclose%
\pgfusepath{fill}%
\end{pgfscope}%
\begin{pgfscope}%
\pgfpathrectangle{\pgfqpoint{0.150000in}{0.150000in}}{\pgfqpoint{4.700000in}{3.450000in}}%
\pgfusepath{clip}%
\pgfsetbuttcap%
\pgfsetroundjoin%
\definecolor{currentfill}{rgb}{0.982904,0.968980,0.970083}%
\pgfsetfillcolor{currentfill}%
\pgfsetlinewidth{0.000000pt}%
\definecolor{currentstroke}{rgb}{0.000000,0.000000,0.000000}%
\pgfsetstrokecolor{currentstroke}%
\pgfsetdash{}{0pt}%
\pgfpathmoveto{\pgfqpoint{2.629513in}{1.703250in}}%
\pgfpathlineto{\pgfqpoint{2.695053in}{1.707293in}}%
\pgfpathlineto{\pgfqpoint{2.629206in}{1.753799in}}%
\pgfpathlineto{\pgfqpoint{2.563514in}{1.750533in}}%
\pgfpathclose%
\pgfusepath{fill}%
\end{pgfscope}%
\begin{pgfscope}%
\pgfpathrectangle{\pgfqpoint{0.150000in}{0.150000in}}{\pgfqpoint{4.700000in}{3.450000in}}%
\pgfusepath{clip}%
\pgfsetbuttcap%
\pgfsetroundjoin%
\definecolor{currentfill}{rgb}{0.937316,0.886259,0.890303}%
\pgfsetfillcolor{currentfill}%
\pgfsetlinewidth{0.000000pt}%
\definecolor{currentstroke}{rgb}{0.000000,0.000000,0.000000}%
\pgfsetstrokecolor{currentstroke}%
\pgfsetdash{}{0pt}%
\pgfpathmoveto{\pgfqpoint{2.893652in}{1.573897in}}%
\pgfpathlineto{\pgfqpoint{2.958121in}{1.570188in}}%
\pgfpathlineto{\pgfqpoint{2.891923in}{1.614438in}}%
\pgfpathlineto{\pgfqpoint{2.827281in}{1.618559in}}%
\pgfpathclose%
\pgfusepath{fill}%
\end{pgfscope}%
\begin{pgfscope}%
\pgfpathrectangle{\pgfqpoint{0.150000in}{0.150000in}}{\pgfqpoint{4.700000in}{3.450000in}}%
\pgfusepath{clip}%
\pgfsetbuttcap%
\pgfsetroundjoin%
\definecolor{currentfill}{rgb}{0.884130,0.789752,0.797227}%
\pgfsetfillcolor{currentfill}%
\pgfsetlinewidth{0.000000pt}%
\definecolor{currentstroke}{rgb}{0.000000,0.000000,0.000000}%
\pgfsetstrokecolor{currentstroke}%
\pgfsetdash{}{0pt}%
\pgfpathmoveto{\pgfqpoint{3.157767in}{1.436980in}}%
\pgfpathlineto{\pgfqpoint{3.221220in}{1.434408in}}%
\pgfpathlineto{\pgfqpoint{3.154667in}{1.478598in}}%
\pgfpathlineto{\pgfqpoint{3.091042in}{1.481460in}}%
\pgfpathclose%
\pgfusepath{fill}%
\end{pgfscope}%
\begin{pgfscope}%
\pgfpathrectangle{\pgfqpoint{0.150000in}{0.150000in}}{\pgfqpoint{4.700000in}{3.450000in}}%
\pgfusepath{clip}%
\pgfsetbuttcap%
\pgfsetroundjoin%
\definecolor{currentfill}{rgb}{0.838542,0.707031,0.717448}%
\pgfsetfillcolor{currentfill}%
\pgfsetlinewidth{0.000000pt}%
\definecolor{currentstroke}{rgb}{0.000000,0.000000,0.000000}%
\pgfsetstrokecolor{currentstroke}%
\pgfsetdash{}{0pt}%
\pgfpathmoveto{\pgfqpoint{3.421945in}{1.301253in}}%
\pgfpathlineto{\pgfqpoint{3.484402in}{1.299805in}}%
\pgfpathlineto{\pgfqpoint{3.417484in}{1.343939in}}%
\pgfpathlineto{\pgfqpoint{3.354855in}{1.345676in}}%
\pgfpathclose%
\pgfusepath{fill}%
\end{pgfscope}%
\begin{pgfscope}%
\pgfpathrectangle{\pgfqpoint{0.150000in}{0.150000in}}{\pgfqpoint{4.700000in}{3.450000in}}%
\pgfusepath{clip}%
\pgfsetbuttcap%
\pgfsetroundjoin%
\definecolor{currentfill}{rgb}{0.437148,0.506464,0.603508}%
\pgfsetfillcolor{currentfill}%
\pgfsetlinewidth{0.000000pt}%
\definecolor{currentstroke}{rgb}{0.000000,0.000000,0.000000}%
\pgfsetstrokecolor{currentstroke}%
\pgfsetdash{}{0pt}%
\pgfpathmoveto{\pgfqpoint{1.573147in}{2.669657in}}%
\pgfpathlineto{\pgfqpoint{1.642424in}{2.669693in}}%
\pgfpathlineto{\pgfqpoint{1.576773in}{2.735207in}}%
\pgfpathlineto{\pgfqpoint{1.507274in}{2.735402in}}%
\pgfpathclose%
\pgfusepath{fill}%
\end{pgfscope}%
\begin{pgfscope}%
\pgfpathrectangle{\pgfqpoint{0.150000in}{0.150000in}}{\pgfqpoint{4.700000in}{3.450000in}}%
\pgfusepath{clip}%
\pgfsetbuttcap%
\pgfsetroundjoin%
\definecolor{currentfill}{rgb}{0.474464,0.539185,0.629795}%
\pgfsetfillcolor{currentfill}%
\pgfsetlinewidth{0.000000pt}%
\definecolor{currentstroke}{rgb}{0.000000,0.000000,0.000000}%
\pgfsetstrokecolor{currentstroke}%
\pgfsetdash{}{0pt}%
\pgfpathmoveto{\pgfqpoint{1.639038in}{2.603894in}}%
\pgfpathlineto{\pgfqpoint{1.708094in}{2.604161in}}%
\pgfpathlineto{\pgfqpoint{1.642424in}{2.669693in}}%
\pgfpathlineto{\pgfqpoint{1.573147in}{2.669657in}}%
\pgfpathclose%
\pgfusepath{fill}%
\end{pgfscope}%
\begin{pgfscope}%
\pgfpathrectangle{\pgfqpoint{0.150000in}{0.150000in}}{\pgfqpoint{4.700000in}{3.450000in}}%
\pgfusepath{clip}%
\pgfsetbuttcap%
\pgfsetroundjoin%
\definecolor{currentfill}{rgb}{0.511780,0.571906,0.656081}%
\pgfsetfillcolor{currentfill}%
\pgfsetlinewidth{0.000000pt}%
\definecolor{currentstroke}{rgb}{0.000000,0.000000,0.000000}%
\pgfsetstrokecolor{currentstroke}%
\pgfsetdash{}{0pt}%
\pgfpathmoveto{\pgfqpoint{1.704949in}{2.538111in}}%
\pgfpathlineto{\pgfqpoint{1.773782in}{2.538610in}}%
\pgfpathlineto{\pgfqpoint{1.708094in}{2.604161in}}%
\pgfpathlineto{\pgfqpoint{1.639038in}{2.603894in}}%
\pgfpathclose%
\pgfusepath{fill}%
\end{pgfscope}%
\begin{pgfscope}%
\pgfpathrectangle{\pgfqpoint{0.150000in}{0.150000in}}{\pgfqpoint{4.700000in}{3.450000in}}%
\pgfusepath{clip}%
\pgfsetbuttcap%
\pgfsetroundjoin%
\definecolor{currentfill}{rgb}{0.975306,0.955193,0.956786}%
\pgfsetfillcolor{currentfill}%
\pgfsetlinewidth{0.000000pt}%
\definecolor{currentstroke}{rgb}{0.000000,0.000000,0.000000}%
\pgfsetstrokecolor{currentstroke}%
\pgfsetdash{}{0pt}%
\pgfpathmoveto{\pgfqpoint{2.695665in}{1.655858in}}%
\pgfpathlineto{\pgfqpoint{2.761081in}{1.662984in}}%
\pgfpathlineto{\pgfqpoint{2.695053in}{1.707293in}}%
\pgfpathlineto{\pgfqpoint{2.629513in}{1.703250in}}%
\pgfpathclose%
\pgfusepath{fill}%
\end{pgfscope}%
\begin{pgfscope}%
\pgfpathrectangle{\pgfqpoint{0.150000in}{0.150000in}}{\pgfqpoint{4.700000in}{3.450000in}}%
\pgfusepath{clip}%
\pgfsetbuttcap%
\pgfsetroundjoin%
\definecolor{currentfill}{rgb}{0.549096,0.604626,0.682368}%
\pgfsetfillcolor{currentfill}%
\pgfsetlinewidth{0.000000pt}%
\definecolor{currentstroke}{rgb}{0.000000,0.000000,0.000000}%
\pgfsetstrokecolor{currentstroke}%
\pgfsetdash{}{0pt}%
\pgfpathmoveto{\pgfqpoint{1.770879in}{2.472310in}}%
\pgfpathlineto{\pgfqpoint{1.839490in}{2.473041in}}%
\pgfpathlineto{\pgfqpoint{1.773782in}{2.538610in}}%
\pgfpathlineto{\pgfqpoint{1.704949in}{2.538111in}}%
\pgfpathclose%
\pgfusepath{fill}%
\end{pgfscope}%
\begin{pgfscope}%
\pgfpathrectangle{\pgfqpoint{0.150000in}{0.150000in}}{\pgfqpoint{4.700000in}{3.450000in}}%
\pgfusepath{clip}%
\pgfsetbuttcap%
\pgfsetroundjoin%
\definecolor{currentfill}{rgb}{0.929718,0.872472,0.877007}%
\pgfsetfillcolor{currentfill}%
\pgfsetlinewidth{0.000000pt}%
\definecolor{currentstroke}{rgb}{0.000000,0.000000,0.000000}%
\pgfsetstrokecolor{currentstroke}%
\pgfsetdash{}{0pt}%
\pgfpathmoveto{\pgfqpoint{2.960198in}{1.529240in}}%
\pgfpathlineto{\pgfqpoint{3.024496in}{1.525942in}}%
\pgfpathlineto{\pgfqpoint{2.958121in}{1.570188in}}%
\pgfpathlineto{\pgfqpoint{2.893652in}{1.573897in}}%
\pgfpathclose%
\pgfusepath{fill}%
\end{pgfscope}%
\begin{pgfscope}%
\pgfpathrectangle{\pgfqpoint{0.150000in}{0.150000in}}{\pgfqpoint{4.700000in}{3.450000in}}%
\pgfusepath{clip}%
\pgfsetbuttcap%
\pgfsetroundjoin%
\definecolor{currentfill}{rgb}{0.586412,0.637347,0.708655}%
\pgfsetfillcolor{currentfill}%
\pgfsetlinewidth{0.000000pt}%
\definecolor{currentstroke}{rgb}{0.000000,0.000000,0.000000}%
\pgfsetstrokecolor{currentstroke}%
\pgfsetdash{}{0pt}%
\pgfpathmoveto{\pgfqpoint{1.836827in}{2.406489in}}%
\pgfpathlineto{\pgfqpoint{1.905216in}{2.407452in}}%
\pgfpathlineto{\pgfqpoint{1.839490in}{2.473041in}}%
\pgfpathlineto{\pgfqpoint{1.770879in}{2.472310in}}%
\pgfpathclose%
\pgfusepath{fill}%
\end{pgfscope}%
\begin{pgfscope}%
\pgfpathrectangle{\pgfqpoint{0.150000in}{0.150000in}}{\pgfqpoint{4.700000in}{3.450000in}}%
\pgfusepath{clip}%
\pgfsetbuttcap%
\pgfsetroundjoin%
\definecolor{currentfill}{rgb}{0.880331,0.782858,0.790579}%
\pgfsetfillcolor{currentfill}%
\pgfsetlinewidth{0.000000pt}%
\definecolor{currentstroke}{rgb}{0.000000,0.000000,0.000000}%
\pgfsetstrokecolor{currentstroke}%
\pgfsetdash{}{0pt}%
\pgfpathmoveto{\pgfqpoint{3.224665in}{1.392263in}}%
\pgfpathlineto{\pgfqpoint{3.287949in}{1.390101in}}%
\pgfpathlineto{\pgfqpoint{3.221220in}{1.434408in}}%
\pgfpathlineto{\pgfqpoint{3.157767in}{1.436980in}}%
\pgfpathclose%
\pgfusepath{fill}%
\end{pgfscope}%
\begin{pgfscope}%
\pgfpathrectangle{\pgfqpoint{0.150000in}{0.150000in}}{\pgfqpoint{4.700000in}{3.450000in}}%
\pgfusepath{clip}%
\pgfsetbuttcap%
\pgfsetroundjoin%
\definecolor{currentfill}{rgb}{0.623729,0.670067,0.734942}%
\pgfsetfillcolor{currentfill}%
\pgfsetlinewidth{0.000000pt}%
\definecolor{currentstroke}{rgb}{0.000000,0.000000,0.000000}%
\pgfsetstrokecolor{currentstroke}%
\pgfsetdash{}{0pt}%
\pgfpathmoveto{\pgfqpoint{1.902795in}{2.340650in}}%
\pgfpathlineto{\pgfqpoint{1.970961in}{2.341845in}}%
\pgfpathlineto{\pgfqpoint{1.905216in}{2.407452in}}%
\pgfpathlineto{\pgfqpoint{1.836827in}{2.406489in}}%
\pgfpathclose%
\pgfusepath{fill}%
\end{pgfscope}%
\begin{pgfscope}%
\pgfpathrectangle{\pgfqpoint{0.150000in}{0.150000in}}{\pgfqpoint{4.700000in}{3.450000in}}%
\pgfusepath{clip}%
\pgfsetbuttcap%
\pgfsetroundjoin%
\definecolor{currentfill}{rgb}{0.661045,0.702788,0.761229}%
\pgfsetfillcolor{currentfill}%
\pgfsetlinewidth{0.000000pt}%
\definecolor{currentstroke}{rgb}{0.000000,0.000000,0.000000}%
\pgfsetstrokecolor{currentstroke}%
\pgfsetdash{}{0pt}%
\pgfpathmoveto{\pgfqpoint{1.968781in}{2.274792in}}%
\pgfpathlineto{\pgfqpoint{2.036724in}{2.276219in}}%
\pgfpathlineto{\pgfqpoint{1.970961in}{2.341845in}}%
\pgfpathlineto{\pgfqpoint{1.902795in}{2.340650in}}%
\pgfpathclose%
\pgfusepath{fill}%
\end{pgfscope}%
\begin{pgfscope}%
\pgfpathrectangle{\pgfqpoint{0.150000in}{0.150000in}}{\pgfqpoint{4.700000in}{3.450000in}}%
\pgfusepath{clip}%
\pgfsetbuttcap%
\pgfsetroundjoin%
\definecolor{currentfill}{rgb}{0.698361,0.735509,0.787515}%
\pgfsetfillcolor{currentfill}%
\pgfsetlinewidth{0.000000pt}%
\definecolor{currentstroke}{rgb}{0.000000,0.000000,0.000000}%
\pgfsetstrokecolor{currentstroke}%
\pgfsetdash{}{0pt}%
\pgfpathmoveto{\pgfqpoint{2.034786in}{2.208915in}}%
\pgfpathlineto{\pgfqpoint{2.102507in}{2.210574in}}%
\pgfpathlineto{\pgfqpoint{2.036724in}{2.276219in}}%
\pgfpathlineto{\pgfqpoint{1.968781in}{2.274792in}}%
\pgfpathclose%
\pgfusepath{fill}%
\end{pgfscope}%
\begin{pgfscope}%
\pgfpathrectangle{\pgfqpoint{0.150000in}{0.150000in}}{\pgfqpoint{4.700000in}{3.450000in}}%
\pgfusepath{clip}%
\pgfsetbuttcap%
\pgfsetroundjoin%
\definecolor{currentfill}{rgb}{0.735677,0.768229,0.813802}%
\pgfsetfillcolor{currentfill}%
\pgfsetlinewidth{0.000000pt}%
\definecolor{currentstroke}{rgb}{0.000000,0.000000,0.000000}%
\pgfsetstrokecolor{currentstroke}%
\pgfsetdash{}{0pt}%
\pgfpathmoveto{\pgfqpoint{2.100811in}{2.143019in}}%
\pgfpathlineto{\pgfqpoint{2.168309in}{2.144910in}}%
\pgfpathlineto{\pgfqpoint{2.102507in}{2.210574in}}%
\pgfpathlineto{\pgfqpoint{2.034786in}{2.208915in}}%
\pgfpathclose%
\pgfusepath{fill}%
\end{pgfscope}%
\begin{pgfscope}%
\pgfpathrectangle{\pgfqpoint{0.150000in}{0.150000in}}{\pgfqpoint{4.700000in}{3.450000in}}%
\pgfusepath{clip}%
\pgfsetbuttcap%
\pgfsetroundjoin%
\definecolor{currentfill}{rgb}{0.772993,0.800950,0.840089}%
\pgfsetfillcolor{currentfill}%
\pgfsetlinewidth{0.000000pt}%
\definecolor{currentstroke}{rgb}{0.000000,0.000000,0.000000}%
\pgfsetstrokecolor{currentstroke}%
\pgfsetdash{}{0pt}%
\pgfpathmoveto{\pgfqpoint{2.166854in}{2.077104in}}%
\pgfpathlineto{\pgfqpoint{2.234129in}{2.079228in}}%
\pgfpathlineto{\pgfqpoint{2.168309in}{2.144910in}}%
\pgfpathlineto{\pgfqpoint{2.100811in}{2.143019in}}%
\pgfpathclose%
\pgfusepath{fill}%
\end{pgfscope}%
\begin{pgfscope}%
\pgfpathrectangle{\pgfqpoint{0.150000in}{0.150000in}}{\pgfqpoint{4.700000in}{3.450000in}}%
\pgfusepath{clip}%
\pgfsetbuttcap%
\pgfsetroundjoin%
\definecolor{currentfill}{rgb}{0.810309,0.833670,0.866376}%
\pgfsetfillcolor{currentfill}%
\pgfsetlinewidth{0.000000pt}%
\definecolor{currentstroke}{rgb}{0.000000,0.000000,0.000000}%
\pgfsetstrokecolor{currentstroke}%
\pgfsetdash{}{0pt}%
\pgfpathmoveto{\pgfqpoint{2.232917in}{2.011170in}}%
\pgfpathlineto{\pgfqpoint{2.299968in}{2.013527in}}%
\pgfpathlineto{\pgfqpoint{2.234129in}{2.079228in}}%
\pgfpathlineto{\pgfqpoint{2.166854in}{2.077104in}}%
\pgfpathclose%
\pgfusepath{fill}%
\end{pgfscope}%
\begin{pgfscope}%
\pgfpathrectangle{\pgfqpoint{0.150000in}{0.150000in}}{\pgfqpoint{4.700000in}{3.450000in}}%
\pgfusepath{clip}%
\pgfsetbuttcap%
\pgfsetroundjoin%
\definecolor{currentfill}{rgb}{0.934697,0.942739,0.953998}%
\pgfsetfillcolor{currentfill}%
\pgfsetlinewidth{0.000000pt}%
\definecolor{currentstroke}{rgb}{0.000000,0.000000,0.000000}%
\pgfsetstrokecolor{currentstroke}%
\pgfsetdash{}{0pt}%
\pgfpathmoveto{\pgfqpoint{2.431355in}{1.793976in}}%
\pgfpathlineto{\pgfqpoint{2.497599in}{1.816310in}}%
\pgfpathlineto{\pgfqpoint{2.431703in}{1.882068in}}%
\pgfpathlineto{\pgfqpoint{2.365215in}{1.868275in}}%
\pgfpathclose%
\pgfusepath{fill}%
\end{pgfscope}%
\begin{pgfscope}%
\pgfpathrectangle{\pgfqpoint{0.150000in}{0.150000in}}{\pgfqpoint{4.700000in}{3.450000in}}%
\pgfusepath{clip}%
\pgfsetbuttcap%
\pgfsetroundjoin%
\definecolor{currentfill}{rgb}{0.972013,0.975460,0.980285}%
\pgfsetfillcolor{currentfill}%
\pgfsetlinewidth{0.000000pt}%
\definecolor{currentstroke}{rgb}{0.000000,0.000000,0.000000}%
\pgfsetstrokecolor{currentstroke}%
\pgfsetdash{}{0pt}%
\pgfpathmoveto{\pgfqpoint{2.497423in}{1.728643in}}%
\pgfpathlineto{\pgfqpoint{2.563514in}{1.750533in}}%
\pgfpathlineto{\pgfqpoint{2.497599in}{1.816310in}}%
\pgfpathlineto{\pgfqpoint{2.431355in}{1.793976in}}%
\pgfpathclose%
\pgfusepath{fill}%
\end{pgfscope}%
\begin{pgfscope}%
\pgfpathrectangle{\pgfqpoint{0.150000in}{0.150000in}}{\pgfqpoint{4.700000in}{3.450000in}}%
\pgfusepath{clip}%
\pgfsetbuttcap%
\pgfsetroundjoin%
\definecolor{currentfill}{rgb}{0.891161,0.904565,0.923330}%
\pgfsetfillcolor{currentfill}%
\pgfsetlinewidth{0.000000pt}%
\definecolor{currentstroke}{rgb}{0.000000,0.000000,0.000000}%
\pgfsetstrokecolor{currentstroke}%
\pgfsetdash{}{0pt}%
\pgfpathmoveto{\pgfqpoint{2.365215in}{1.868275in}}%
\pgfpathlineto{\pgfqpoint{2.431703in}{1.882068in}}%
\pgfpathlineto{\pgfqpoint{2.365826in}{1.947807in}}%
\pgfpathlineto{\pgfqpoint{2.299033in}{1.942742in}}%
\pgfpathclose%
\pgfusepath{fill}%
\end{pgfscope}%
\begin{pgfscope}%
\pgfpathrectangle{\pgfqpoint{0.150000in}{0.150000in}}{\pgfqpoint{4.700000in}{3.450000in}}%
\pgfusepath{clip}%
\pgfsetbuttcap%
\pgfsetroundjoin%
\definecolor{currentfill}{rgb}{0.847626,0.866391,0.892662}%
\pgfsetfillcolor{currentfill}%
\pgfsetlinewidth{0.000000pt}%
\definecolor{currentstroke}{rgb}{0.000000,0.000000,0.000000}%
\pgfsetstrokecolor{currentstroke}%
\pgfsetdash{}{0pt}%
\pgfpathmoveto{\pgfqpoint{2.299033in}{1.942742in}}%
\pgfpathlineto{\pgfqpoint{2.365826in}{1.947807in}}%
\pgfpathlineto{\pgfqpoint{2.299968in}{2.013527in}}%
\pgfpathlineto{\pgfqpoint{2.232917in}{2.011170in}}%
\pgfpathclose%
\pgfusepath{fill}%
\end{pgfscope}%
\begin{pgfscope}%
\pgfpathrectangle{\pgfqpoint{0.150000in}{0.150000in}}{\pgfqpoint{4.700000in}{3.450000in}}%
\pgfusepath{clip}%
\pgfsetbuttcap%
\pgfsetroundjoin%
\definecolor{currentfill}{rgb}{0.967708,0.941406,0.943490}%
\pgfsetfillcolor{currentfill}%
\pgfsetlinewidth{0.000000pt}%
\definecolor{currentstroke}{rgb}{0.000000,0.000000,0.000000}%
\pgfsetstrokecolor{currentstroke}%
\pgfsetdash{}{0pt}%
\pgfpathmoveto{\pgfqpoint{2.761969in}{1.608357in}}%
\pgfpathlineto{\pgfqpoint{2.827281in}{1.618559in}}%
\pgfpathlineto{\pgfqpoint{2.761081in}{1.662984in}}%
\pgfpathlineto{\pgfqpoint{2.695665in}{1.655858in}}%
\pgfpathclose%
\pgfusepath{fill}%
\end{pgfscope}%
\begin{pgfscope}%
\pgfpathrectangle{\pgfqpoint{0.150000in}{0.150000in}}{\pgfqpoint{4.700000in}{3.450000in}}%
\pgfusepath{clip}%
\pgfsetbuttcap%
\pgfsetroundjoin%
\definecolor{currentfill}{rgb}{0.922120,0.858686,0.863710}%
\pgfsetfillcolor{currentfill}%
\pgfsetlinewidth{0.000000pt}%
\definecolor{currentstroke}{rgb}{0.000000,0.000000,0.000000}%
\pgfsetstrokecolor{currentstroke}%
\pgfsetdash{}{0pt}%
\pgfpathmoveto{\pgfqpoint{3.026919in}{1.484465in}}%
\pgfpathlineto{\pgfqpoint{3.091042in}{1.481460in}}%
\pgfpathlineto{\pgfqpoint{3.024496in}{1.525942in}}%
\pgfpathlineto{\pgfqpoint{2.960198in}{1.529240in}}%
\pgfpathclose%
\pgfusepath{fill}%
\end{pgfscope}%
\begin{pgfscope}%
\pgfpathrectangle{\pgfqpoint{0.150000in}{0.150000in}}{\pgfqpoint{4.700000in}{3.450000in}}%
\pgfusepath{clip}%
\pgfsetbuttcap%
\pgfsetroundjoin%
\definecolor{currentfill}{rgb}{0.872733,0.769072,0.777282}%
\pgfsetfillcolor{currentfill}%
\pgfsetlinewidth{0.000000pt}%
\definecolor{currentstroke}{rgb}{0.000000,0.000000,0.000000}%
\pgfsetstrokecolor{currentstroke}%
\pgfsetdash{}{0pt}%
\pgfpathmoveto{\pgfqpoint{3.291744in}{1.347547in}}%
\pgfpathlineto{\pgfqpoint{3.354855in}{1.345676in}}%
\pgfpathlineto{\pgfqpoint{3.287949in}{1.390101in}}%
\pgfpathlineto{\pgfqpoint{3.224665in}{1.392263in}}%
\pgfpathclose%
\pgfusepath{fill}%
\end{pgfscope}%
\begin{pgfscope}%
\pgfpathrectangle{\pgfqpoint{0.150000in}{0.150000in}}{\pgfqpoint{4.700000in}{3.450000in}}%
\pgfusepath{clip}%
\pgfsetbuttcap%
\pgfsetroundjoin%
\definecolor{currentfill}{rgb}{0.996890,0.997273,0.997809}%
\pgfsetfillcolor{currentfill}%
\pgfsetlinewidth{0.000000pt}%
\definecolor{currentstroke}{rgb}{0.000000,0.000000,0.000000}%
\pgfsetstrokecolor{currentstroke}%
\pgfsetdash{}{0pt}%
\pgfpathmoveto{\pgfqpoint{2.563514in}{1.681215in}}%
\pgfpathlineto{\pgfqpoint{2.629513in}{1.703250in}}%
\pgfpathlineto{\pgfqpoint{2.563514in}{1.750533in}}%
\pgfpathlineto{\pgfqpoint{2.497423in}{1.728643in}}%
\pgfpathclose%
\pgfusepath{fill}%
\end{pgfscope}%
\begin{pgfscope}%
\pgfpathrectangle{\pgfqpoint{0.150000in}{0.150000in}}{\pgfqpoint{4.700000in}{3.450000in}}%
\pgfusepath{clip}%
\pgfsetbuttcap%
\pgfsetroundjoin%
\definecolor{currentfill}{rgb}{0.960110,0.927619,0.930193}%
\pgfsetfillcolor{currentfill}%
\pgfsetlinewidth{0.000000pt}%
\definecolor{currentstroke}{rgb}{0.000000,0.000000,0.000000}%
\pgfsetstrokecolor{currentstroke}%
\pgfsetdash{}{0pt}%
\pgfpathmoveto{\pgfqpoint{2.828426in}{1.560747in}}%
\pgfpathlineto{\pgfqpoint{2.893652in}{1.573897in}}%
\pgfpathlineto{\pgfqpoint{2.827281in}{1.618559in}}%
\pgfpathlineto{\pgfqpoint{2.761969in}{1.608357in}}%
\pgfpathclose%
\pgfusepath{fill}%
\end{pgfscope}%
\begin{pgfscope}%
\pgfpathrectangle{\pgfqpoint{0.150000in}{0.150000in}}{\pgfqpoint{4.700000in}{3.450000in}}%
\pgfusepath{clip}%
\pgfsetbuttcap%
\pgfsetroundjoin%
\definecolor{currentfill}{rgb}{0.914522,0.844899,0.850414}%
\pgfsetfillcolor{currentfill}%
\pgfsetlinewidth{0.000000pt}%
\definecolor{currentstroke}{rgb}{0.000000,0.000000,0.000000}%
\pgfsetstrokecolor{currentstroke}%
\pgfsetdash{}{0pt}%
\pgfpathmoveto{\pgfqpoint{3.093816in}{1.439572in}}%
\pgfpathlineto{\pgfqpoint{3.157767in}{1.436980in}}%
\pgfpathlineto{\pgfqpoint{3.091042in}{1.481460in}}%
\pgfpathlineto{\pgfqpoint{3.026919in}{1.484465in}}%
\pgfpathclose%
\pgfusepath{fill}%
\end{pgfscope}%
\begin{pgfscope}%
\pgfpathrectangle{\pgfqpoint{0.150000in}{0.150000in}}{\pgfqpoint{4.700000in}{3.450000in}}%
\pgfusepath{clip}%
\pgfsetbuttcap%
\pgfsetroundjoin%
\definecolor{currentfill}{rgb}{0.865135,0.755285,0.763986}%
\pgfsetfillcolor{currentfill}%
\pgfsetlinewidth{0.000000pt}%
\definecolor{currentstroke}{rgb}{0.000000,0.000000,0.000000}%
\pgfsetstrokecolor{currentstroke}%
\pgfsetdash{}{0pt}%
\pgfpathmoveto{\pgfqpoint{3.358995in}{1.302593in}}%
\pgfpathlineto{\pgfqpoint{3.421945in}{1.301253in}}%
\pgfpathlineto{\pgfqpoint{3.354855in}{1.345676in}}%
\pgfpathlineto{\pgfqpoint{3.291744in}{1.347547in}}%
\pgfpathclose%
\pgfusepath{fill}%
\end{pgfscope}%
\begin{pgfscope}%
\pgfpathrectangle{\pgfqpoint{0.150000in}{0.150000in}}{\pgfqpoint{4.700000in}{3.450000in}}%
\pgfusepath{clip}%
\pgfsetbuttcap%
\pgfsetroundjoin%
\definecolor{currentfill}{rgb}{0.990502,0.982767,0.983379}%
\pgfsetfillcolor{currentfill}%
\pgfsetlinewidth{0.000000pt}%
\definecolor{currentstroke}{rgb}{0.000000,0.000000,0.000000}%
\pgfsetstrokecolor{currentstroke}%
\pgfsetdash{}{0pt}%
\pgfpathmoveto{\pgfqpoint{2.629757in}{1.633678in}}%
\pgfpathlineto{\pgfqpoint{2.695665in}{1.655858in}}%
\pgfpathlineto{\pgfqpoint{2.629513in}{1.703250in}}%
\pgfpathlineto{\pgfqpoint{2.563514in}{1.681215in}}%
\pgfpathclose%
\pgfusepath{fill}%
\end{pgfscope}%
\begin{pgfscope}%
\pgfpathrectangle{\pgfqpoint{0.150000in}{0.150000in}}{\pgfqpoint{4.700000in}{3.450000in}}%
\pgfusepath{clip}%
\pgfsetbuttcap%
\pgfsetroundjoin%
\definecolor{currentfill}{rgb}{0.399831,0.473744,0.577221}%
\pgfsetfillcolor{currentfill}%
\pgfsetlinewidth{0.000000pt}%
\definecolor{currentstroke}{rgb}{0.000000,0.000000,0.000000}%
\pgfsetstrokecolor{currentstroke}%
\pgfsetdash{}{0pt}%
\pgfpathmoveto{\pgfqpoint{1.503378in}{2.669621in}}%
\pgfpathlineto{\pgfqpoint{1.573147in}{2.669657in}}%
\pgfpathlineto{\pgfqpoint{1.507274in}{2.735402in}}%
\pgfpathlineto{\pgfqpoint{1.437282in}{2.735599in}}%
\pgfpathclose%
\pgfusepath{fill}%
\end{pgfscope}%
\begin{pgfscope}%
\pgfpathrectangle{\pgfqpoint{0.150000in}{0.150000in}}{\pgfqpoint{4.700000in}{3.450000in}}%
\pgfusepath{clip}%
\pgfsetbuttcap%
\pgfsetroundjoin%
\definecolor{currentfill}{rgb}{0.437148,0.506464,0.603508}%
\pgfsetfillcolor{currentfill}%
\pgfsetlinewidth{0.000000pt}%
\definecolor{currentstroke}{rgb}{0.000000,0.000000,0.000000}%
\pgfsetstrokecolor{currentstroke}%
\pgfsetdash{}{0pt}%
\pgfpathmoveto{\pgfqpoint{1.569493in}{2.603624in}}%
\pgfpathlineto{\pgfqpoint{1.639038in}{2.603894in}}%
\pgfpathlineto{\pgfqpoint{1.573147in}{2.669657in}}%
\pgfpathlineto{\pgfqpoint{1.503378in}{2.669621in}}%
\pgfpathclose%
\pgfusepath{fill}%
\end{pgfscope}%
\begin{pgfscope}%
\pgfpathrectangle{\pgfqpoint{0.150000in}{0.150000in}}{\pgfqpoint{4.700000in}{3.450000in}}%
\pgfusepath{clip}%
\pgfsetbuttcap%
\pgfsetroundjoin%
\definecolor{currentfill}{rgb}{0.474464,0.539185,0.629795}%
\pgfsetfillcolor{currentfill}%
\pgfsetlinewidth{0.000000pt}%
\definecolor{currentstroke}{rgb}{0.000000,0.000000,0.000000}%
\pgfsetstrokecolor{currentstroke}%
\pgfsetdash{}{0pt}%
\pgfpathmoveto{\pgfqpoint{1.635628in}{2.537609in}}%
\pgfpathlineto{\pgfqpoint{1.704949in}{2.538111in}}%
\pgfpathlineto{\pgfqpoint{1.639038in}{2.603894in}}%
\pgfpathlineto{\pgfqpoint{1.569493in}{2.603624in}}%
\pgfpathclose%
\pgfusepath{fill}%
\end{pgfscope}%
\begin{pgfscope}%
\pgfpathrectangle{\pgfqpoint{0.150000in}{0.150000in}}{\pgfqpoint{4.700000in}{3.450000in}}%
\pgfusepath{clip}%
\pgfsetbuttcap%
\pgfsetroundjoin%
\definecolor{currentfill}{rgb}{0.511780,0.571906,0.656081}%
\pgfsetfillcolor{currentfill}%
\pgfsetlinewidth{0.000000pt}%
\definecolor{currentstroke}{rgb}{0.000000,0.000000,0.000000}%
\pgfsetstrokecolor{currentstroke}%
\pgfsetdash{}{0pt}%
\pgfpathmoveto{\pgfqpoint{1.701781in}{2.471574in}}%
\pgfpathlineto{\pgfqpoint{1.770879in}{2.472310in}}%
\pgfpathlineto{\pgfqpoint{1.704949in}{2.538111in}}%
\pgfpathlineto{\pgfqpoint{1.635628in}{2.537609in}}%
\pgfpathclose%
\pgfusepath{fill}%
\end{pgfscope}%
\begin{pgfscope}%
\pgfpathrectangle{\pgfqpoint{0.150000in}{0.150000in}}{\pgfqpoint{4.700000in}{3.450000in}}%
\pgfusepath{clip}%
\pgfsetbuttcap%
\pgfsetroundjoin%
\definecolor{currentfill}{rgb}{0.959574,0.964553,0.971523}%
\pgfsetfillcolor{currentfill}%
\pgfsetlinewidth{0.000000pt}%
\definecolor{currentstroke}{rgb}{0.000000,0.000000,0.000000}%
\pgfsetstrokecolor{currentstroke}%
\pgfsetdash{}{0pt}%
\pgfpathmoveto{\pgfqpoint{2.430997in}{1.706643in}}%
\pgfpathlineto{\pgfqpoint{2.497423in}{1.728643in}}%
\pgfpathlineto{\pgfqpoint{2.431355in}{1.793976in}}%
\pgfpathlineto{\pgfqpoint{2.364805in}{1.769198in}}%
\pgfpathclose%
\pgfusepath{fill}%
\end{pgfscope}%
\begin{pgfscope}%
\pgfpathrectangle{\pgfqpoint{0.150000in}{0.150000in}}{\pgfqpoint{4.700000in}{3.450000in}}%
\pgfusepath{clip}%
\pgfsetbuttcap%
\pgfsetroundjoin%
\definecolor{currentfill}{rgb}{0.916039,0.926379,0.940855}%
\pgfsetfillcolor{currentfill}%
\pgfsetlinewidth{0.000000pt}%
\definecolor{currentstroke}{rgb}{0.000000,0.000000,0.000000}%
\pgfsetstrokecolor{currentstroke}%
\pgfsetdash{}{0pt}%
\pgfpathmoveto{\pgfqpoint{2.364805in}{1.769198in}}%
\pgfpathlineto{\pgfqpoint{2.431355in}{1.793976in}}%
\pgfpathlineto{\pgfqpoint{2.365215in}{1.868275in}}%
\pgfpathlineto{\pgfqpoint{2.298490in}{1.843541in}}%
\pgfpathclose%
\pgfusepath{fill}%
\end{pgfscope}%
\begin{pgfscope}%
\pgfpathrectangle{\pgfqpoint{0.150000in}{0.150000in}}{\pgfqpoint{4.700000in}{3.450000in}}%
\pgfusepath{clip}%
\pgfsetbuttcap%
\pgfsetroundjoin%
\definecolor{currentfill}{rgb}{0.948713,0.906939,0.910248}%
\pgfsetfillcolor{currentfill}%
\pgfsetlinewidth{0.000000pt}%
\definecolor{currentstroke}{rgb}{0.000000,0.000000,0.000000}%
\pgfsetstrokecolor{currentstroke}%
\pgfsetdash{}{0pt}%
\pgfpathmoveto{\pgfqpoint{2.895037in}{1.513026in}}%
\pgfpathlineto{\pgfqpoint{2.960198in}{1.529240in}}%
\pgfpathlineto{\pgfqpoint{2.893652in}{1.573897in}}%
\pgfpathlineto{\pgfqpoint{2.828426in}{1.560747in}}%
\pgfpathclose%
\pgfusepath{fill}%
\end{pgfscope}%
\begin{pgfscope}%
\pgfpathrectangle{\pgfqpoint{0.150000in}{0.150000in}}{\pgfqpoint{4.700000in}{3.450000in}}%
\pgfusepath{clip}%
\pgfsetbuttcap%
\pgfsetroundjoin%
\definecolor{currentfill}{rgb}{0.549096,0.604626,0.682368}%
\pgfsetfillcolor{currentfill}%
\pgfsetlinewidth{0.000000pt}%
\definecolor{currentstroke}{rgb}{0.000000,0.000000,0.000000}%
\pgfsetstrokecolor{currentstroke}%
\pgfsetdash{}{0pt}%
\pgfpathmoveto{\pgfqpoint{1.767953in}{2.405520in}}%
\pgfpathlineto{\pgfqpoint{1.836827in}{2.406489in}}%
\pgfpathlineto{\pgfqpoint{1.770879in}{2.472310in}}%
\pgfpathlineto{\pgfqpoint{1.701781in}{2.471574in}}%
\pgfpathclose%
\pgfusepath{fill}%
\end{pgfscope}%
\begin{pgfscope}%
\pgfpathrectangle{\pgfqpoint{0.150000in}{0.150000in}}{\pgfqpoint{4.700000in}{3.450000in}}%
\pgfusepath{clip}%
\pgfsetbuttcap%
\pgfsetroundjoin%
\definecolor{currentfill}{rgb}{0.586412,0.637347,0.708655}%
\pgfsetfillcolor{currentfill}%
\pgfsetlinewidth{0.000000pt}%
\definecolor{currentstroke}{rgb}{0.000000,0.000000,0.000000}%
\pgfsetstrokecolor{currentstroke}%
\pgfsetdash{}{0pt}%
\pgfpathmoveto{\pgfqpoint{1.834145in}{2.339447in}}%
\pgfpathlineto{\pgfqpoint{1.902795in}{2.340650in}}%
\pgfpathlineto{\pgfqpoint{1.836827in}{2.406489in}}%
\pgfpathlineto{\pgfqpoint{1.767953in}{2.405520in}}%
\pgfpathclose%
\pgfusepath{fill}%
\end{pgfscope}%
\begin{pgfscope}%
\pgfpathrectangle{\pgfqpoint{0.150000in}{0.150000in}}{\pgfqpoint{4.700000in}{3.450000in}}%
\pgfusepath{clip}%
\pgfsetbuttcap%
\pgfsetroundjoin%
\definecolor{currentfill}{rgb}{0.872503,0.888205,0.910187}%
\pgfsetfillcolor{currentfill}%
\pgfsetlinewidth{0.000000pt}%
\definecolor{currentstroke}{rgb}{0.000000,0.000000,0.000000}%
\pgfsetstrokecolor{currentstroke}%
\pgfsetdash{}{0pt}%
\pgfpathmoveto{\pgfqpoint{2.298490in}{1.843541in}}%
\pgfpathlineto{\pgfqpoint{2.365215in}{1.868275in}}%
\pgfpathlineto{\pgfqpoint{2.299033in}{1.942742in}}%
\pgfpathlineto{\pgfqpoint{2.232134in}{1.917930in}}%
\pgfpathclose%
\pgfusepath{fill}%
\end{pgfscope}%
\begin{pgfscope}%
\pgfpathrectangle{\pgfqpoint{0.150000in}{0.150000in}}{\pgfqpoint{4.700000in}{3.450000in}}%
\pgfusepath{clip}%
\pgfsetbuttcap%
\pgfsetroundjoin%
\definecolor{currentfill}{rgb}{0.623729,0.670067,0.734942}%
\pgfsetfillcolor{currentfill}%
\pgfsetlinewidth{0.000000pt}%
\definecolor{currentstroke}{rgb}{0.000000,0.000000,0.000000}%
\pgfsetstrokecolor{currentstroke}%
\pgfsetdash{}{0pt}%
\pgfpathmoveto{\pgfqpoint{1.900356in}{2.273355in}}%
\pgfpathlineto{\pgfqpoint{1.968781in}{2.274792in}}%
\pgfpathlineto{\pgfqpoint{1.902795in}{2.340650in}}%
\pgfpathlineto{\pgfqpoint{1.834145in}{2.339447in}}%
\pgfpathclose%
\pgfusepath{fill}%
\end{pgfscope}%
\begin{pgfscope}%
\pgfpathrectangle{\pgfqpoint{0.150000in}{0.150000in}}{\pgfqpoint{4.700000in}{3.450000in}}%
\pgfusepath{clip}%
\pgfsetbuttcap%
\pgfsetroundjoin%
\definecolor{currentfill}{rgb}{0.982904,0.968980,0.970083}%
\pgfsetfillcolor{currentfill}%
\pgfsetlinewidth{0.000000pt}%
\definecolor{currentstroke}{rgb}{0.000000,0.000000,0.000000}%
\pgfsetstrokecolor{currentstroke}%
\pgfsetdash{}{0pt}%
\pgfpathmoveto{\pgfqpoint{2.696153in}{1.586031in}}%
\pgfpathlineto{\pgfqpoint{2.761969in}{1.608357in}}%
\pgfpathlineto{\pgfqpoint{2.695665in}{1.655858in}}%
\pgfpathlineto{\pgfqpoint{2.629757in}{1.633678in}}%
\pgfpathclose%
\pgfusepath{fill}%
\end{pgfscope}%
\begin{pgfscope}%
\pgfpathrectangle{\pgfqpoint{0.150000in}{0.150000in}}{\pgfqpoint{4.700000in}{3.450000in}}%
\pgfusepath{clip}%
\pgfsetbuttcap%
\pgfsetroundjoin%
\definecolor{currentfill}{rgb}{0.906924,0.831112,0.837117}%
\pgfsetfillcolor{currentfill}%
\pgfsetlinewidth{0.000000pt}%
\definecolor{currentstroke}{rgb}{0.000000,0.000000,0.000000}%
\pgfsetstrokecolor{currentstroke}%
\pgfsetdash{}{0pt}%
\pgfpathmoveto{\pgfqpoint{3.160809in}{1.392151in}}%
\pgfpathlineto{\pgfqpoint{3.224665in}{1.392263in}}%
\pgfpathlineto{\pgfqpoint{3.157767in}{1.436980in}}%
\pgfpathlineto{\pgfqpoint{3.093816in}{1.439572in}}%
\pgfpathclose%
\pgfusepath{fill}%
\end{pgfscope}%
\begin{pgfscope}%
\pgfpathrectangle{\pgfqpoint{0.150000in}{0.150000in}}{\pgfqpoint{4.700000in}{3.450000in}}%
\pgfusepath{clip}%
\pgfsetbuttcap%
\pgfsetroundjoin%
\definecolor{currentfill}{rgb}{0.822748,0.844577,0.875138}%
\pgfsetfillcolor{currentfill}%
\pgfsetlinewidth{0.000000pt}%
\definecolor{currentstroke}{rgb}{0.000000,0.000000,0.000000}%
\pgfsetstrokecolor{currentstroke}%
\pgfsetdash{}{0pt}%
\pgfpathmoveto{\pgfqpoint{2.232134in}{1.917930in}}%
\pgfpathlineto{\pgfqpoint{2.299033in}{1.942742in}}%
\pgfpathlineto{\pgfqpoint{2.232917in}{2.011170in}}%
\pgfpathlineto{\pgfqpoint{2.165741in}{1.992239in}}%
\pgfpathclose%
\pgfusepath{fill}%
\end{pgfscope}%
\begin{pgfscope}%
\pgfpathrectangle{\pgfqpoint{0.150000in}{0.150000in}}{\pgfqpoint{4.700000in}{3.450000in}}%
\pgfusepath{clip}%
\pgfsetbuttcap%
\pgfsetroundjoin%
\definecolor{currentfill}{rgb}{0.661045,0.702788,0.761229}%
\pgfsetfillcolor{currentfill}%
\pgfsetlinewidth{0.000000pt}%
\definecolor{currentstroke}{rgb}{0.000000,0.000000,0.000000}%
\pgfsetstrokecolor{currentstroke}%
\pgfsetdash{}{0pt}%
\pgfpathmoveto{\pgfqpoint{1.966585in}{2.207244in}}%
\pgfpathlineto{\pgfqpoint{2.034786in}{2.208915in}}%
\pgfpathlineto{\pgfqpoint{1.968781in}{2.274792in}}%
\pgfpathlineto{\pgfqpoint{1.900356in}{2.273355in}}%
\pgfpathclose%
\pgfusepath{fill}%
\end{pgfscope}%
\begin{pgfscope}%
\pgfpathrectangle{\pgfqpoint{0.150000in}{0.150000in}}{\pgfqpoint{4.700000in}{3.450000in}}%
\pgfusepath{clip}%
\pgfsetbuttcap%
\pgfsetroundjoin%
\definecolor{currentfill}{rgb}{0.779213,0.806403,0.844470}%
\pgfsetfillcolor{currentfill}%
\pgfsetlinewidth{0.000000pt}%
\definecolor{currentstroke}{rgb}{0.000000,0.000000,0.000000}%
\pgfsetstrokecolor{currentstroke}%
\pgfsetdash{}{0pt}%
\pgfpathmoveto{\pgfqpoint{2.165741in}{1.992239in}}%
\pgfpathlineto{\pgfqpoint{2.232917in}{2.011170in}}%
\pgfpathlineto{\pgfqpoint{2.166854in}{2.077104in}}%
\pgfpathlineto{\pgfqpoint{2.099306in}{2.066717in}}%
\pgfpathclose%
\pgfusepath{fill}%
\end{pgfscope}%
\begin{pgfscope}%
\pgfpathrectangle{\pgfqpoint{0.150000in}{0.150000in}}{\pgfqpoint{4.700000in}{3.450000in}}%
\pgfusepath{clip}%
\pgfsetbuttcap%
\pgfsetroundjoin%
\definecolor{currentfill}{rgb}{0.984452,0.986366,0.989047}%
\pgfsetfillcolor{currentfill}%
\pgfsetlinewidth{0.000000pt}%
\definecolor{currentstroke}{rgb}{0.000000,0.000000,0.000000}%
\pgfsetstrokecolor{currentstroke}%
\pgfsetdash{}{0pt}%
\pgfpathmoveto{\pgfqpoint{2.497179in}{1.659069in}}%
\pgfpathlineto{\pgfqpoint{2.563514in}{1.681215in}}%
\pgfpathlineto{\pgfqpoint{2.497423in}{1.728643in}}%
\pgfpathlineto{\pgfqpoint{2.430997in}{1.706643in}}%
\pgfpathclose%
\pgfusepath{fill}%
\end{pgfscope}%
\begin{pgfscope}%
\pgfpathrectangle{\pgfqpoint{0.150000in}{0.150000in}}{\pgfqpoint{4.700000in}{3.450000in}}%
\pgfusepath{clip}%
\pgfsetbuttcap%
\pgfsetroundjoin%
\definecolor{currentfill}{rgb}{0.698361,0.735509,0.787515}%
\pgfsetfillcolor{currentfill}%
\pgfsetlinewidth{0.000000pt}%
\definecolor{currentstroke}{rgb}{0.000000,0.000000,0.000000}%
\pgfsetstrokecolor{currentstroke}%
\pgfsetdash{}{0pt}%
\pgfpathmoveto{\pgfqpoint{2.032834in}{2.141114in}}%
\pgfpathlineto{\pgfqpoint{2.100811in}{2.143019in}}%
\pgfpathlineto{\pgfqpoint{2.034786in}{2.208915in}}%
\pgfpathlineto{\pgfqpoint{1.966585in}{2.207244in}}%
\pgfpathclose%
\pgfusepath{fill}%
\end{pgfscope}%
\begin{pgfscope}%
\pgfpathrectangle{\pgfqpoint{0.150000in}{0.150000in}}{\pgfqpoint{4.700000in}{3.450000in}}%
\pgfusepath{clip}%
\pgfsetbuttcap%
\pgfsetroundjoin%
\definecolor{currentfill}{rgb}{0.735677,0.768229,0.813802}%
\pgfsetfillcolor{currentfill}%
\pgfsetlinewidth{0.000000pt}%
\definecolor{currentstroke}{rgb}{0.000000,0.000000,0.000000}%
\pgfsetstrokecolor{currentstroke}%
\pgfsetdash{}{0pt}%
\pgfpathmoveto{\pgfqpoint{2.099306in}{2.066717in}}%
\pgfpathlineto{\pgfqpoint{2.166854in}{2.077104in}}%
\pgfpathlineto{\pgfqpoint{2.100811in}{2.143019in}}%
\pgfpathlineto{\pgfqpoint{2.032834in}{2.141114in}}%
\pgfpathclose%
\pgfusepath{fill}%
\end{pgfscope}%
\begin{pgfscope}%
\pgfpathrectangle{\pgfqpoint{0.150000in}{0.150000in}}{\pgfqpoint{4.700000in}{3.450000in}}%
\pgfusepath{clip}%
\pgfsetbuttcap%
\pgfsetroundjoin%
\definecolor{currentfill}{rgb}{0.941115,0.893153,0.896952}%
\pgfsetfillcolor{currentfill}%
\pgfsetlinewidth{0.000000pt}%
\definecolor{currentstroke}{rgb}{0.000000,0.000000,0.000000}%
\pgfsetstrokecolor{currentstroke}%
\pgfsetdash{}{0pt}%
\pgfpathmoveto{\pgfqpoint{2.961801in}{1.465196in}}%
\pgfpathlineto{\pgfqpoint{3.026919in}{1.484465in}}%
\pgfpathlineto{\pgfqpoint{2.960198in}{1.529240in}}%
\pgfpathlineto{\pgfqpoint{2.895037in}{1.513026in}}%
\pgfpathclose%
\pgfusepath{fill}%
\end{pgfscope}%
\begin{pgfscope}%
\pgfpathrectangle{\pgfqpoint{0.150000in}{0.150000in}}{\pgfqpoint{4.700000in}{3.450000in}}%
\pgfusepath{clip}%
\pgfsetbuttcap%
\pgfsetroundjoin%
\definecolor{currentfill}{rgb}{0.971507,0.948300,0.950138}%
\pgfsetfillcolor{currentfill}%
\pgfsetlinewidth{0.000000pt}%
\definecolor{currentstroke}{rgb}{0.000000,0.000000,0.000000}%
\pgfsetstrokecolor{currentstroke}%
\pgfsetdash{}{0pt}%
\pgfpathmoveto{\pgfqpoint{2.762703in}{1.538274in}}%
\pgfpathlineto{\pgfqpoint{2.828426in}{1.560747in}}%
\pgfpathlineto{\pgfqpoint{2.761969in}{1.608357in}}%
\pgfpathlineto{\pgfqpoint{2.696153in}{1.586031in}}%
\pgfpathclose%
\pgfusepath{fill}%
\end{pgfscope}%
\begin{pgfscope}%
\pgfpathrectangle{\pgfqpoint{0.150000in}{0.150000in}}{\pgfqpoint{4.700000in}{3.450000in}}%
\pgfusepath{clip}%
\pgfsetbuttcap%
\pgfsetroundjoin%
\definecolor{currentfill}{rgb}{0.899326,0.817325,0.823820}%
\pgfsetfillcolor{currentfill}%
\pgfsetlinewidth{0.000000pt}%
\definecolor{currentstroke}{rgb}{0.000000,0.000000,0.000000}%
\pgfsetstrokecolor{currentstroke}%
\pgfsetdash{}{0pt}%
\pgfpathmoveto{\pgfqpoint{3.227943in}{1.344136in}}%
\pgfpathlineto{\pgfqpoint{3.291744in}{1.347547in}}%
\pgfpathlineto{\pgfqpoint{3.224665in}{1.392263in}}%
\pgfpathlineto{\pgfqpoint{3.160809in}{1.392151in}}%
\pgfpathclose%
\pgfusepath{fill}%
\end{pgfscope}%
\begin{pgfscope}%
\pgfpathrectangle{\pgfqpoint{0.150000in}{0.150000in}}{\pgfqpoint{4.700000in}{3.450000in}}%
\pgfusepath{clip}%
\pgfsetbuttcap%
\pgfsetroundjoin%
\definecolor{currentfill}{rgb}{0.996890,0.997273,0.997809}%
\pgfsetfillcolor{currentfill}%
\pgfsetlinewidth{0.000000pt}%
\definecolor{currentstroke}{rgb}{0.000000,0.000000,0.000000}%
\pgfsetstrokecolor{currentstroke}%
\pgfsetdash{}{0pt}%
\pgfpathmoveto{\pgfqpoint{2.563514in}{1.611385in}}%
\pgfpathlineto{\pgfqpoint{2.629757in}{1.633678in}}%
\pgfpathlineto{\pgfqpoint{2.563514in}{1.681215in}}%
\pgfpathlineto{\pgfqpoint{2.497179in}{1.659069in}}%
\pgfpathclose%
\pgfusepath{fill}%
\end{pgfscope}%
\begin{pgfscope}%
\pgfpathrectangle{\pgfqpoint{0.150000in}{0.150000in}}{\pgfqpoint{4.700000in}{3.450000in}}%
\pgfusepath{clip}%
\pgfsetbuttcap%
\pgfsetroundjoin%
\definecolor{currentfill}{rgb}{0.903600,0.915472,0.932093}%
\pgfsetfillcolor{currentfill}%
\pgfsetlinewidth{0.000000pt}%
\definecolor{currentstroke}{rgb}{0.000000,0.000000,0.000000}%
\pgfsetstrokecolor{currentstroke}%
\pgfsetdash{}{0pt}%
\pgfpathmoveto{\pgfqpoint{2.297939in}{1.744301in}}%
\pgfpathlineto{\pgfqpoint{2.364805in}{1.769198in}}%
\pgfpathlineto{\pgfqpoint{2.298490in}{1.843541in}}%
\pgfpathlineto{\pgfqpoint{2.231449in}{1.818567in}}%
\pgfpathclose%
\pgfusepath{fill}%
\end{pgfscope}%
\begin{pgfscope}%
\pgfpathrectangle{\pgfqpoint{0.150000in}{0.150000in}}{\pgfqpoint{4.700000in}{3.450000in}}%
\pgfusepath{clip}%
\pgfsetbuttcap%
\pgfsetroundjoin%
\definecolor{currentfill}{rgb}{0.947135,0.953646,0.962760}%
\pgfsetfillcolor{currentfill}%
\pgfsetlinewidth{0.000000pt}%
\definecolor{currentstroke}{rgb}{0.000000,0.000000,0.000000}%
\pgfsetstrokecolor{currentstroke}%
\pgfsetdash{}{0pt}%
\pgfpathmoveto{\pgfqpoint{2.364233in}{1.684530in}}%
\pgfpathlineto{\pgfqpoint{2.430997in}{1.706643in}}%
\pgfpathlineto{\pgfqpoint{2.364805in}{1.769198in}}%
\pgfpathlineto{\pgfqpoint{2.297939in}{1.744301in}}%
\pgfpathclose%
\pgfusepath{fill}%
\end{pgfscope}%
\begin{pgfscope}%
\pgfpathrectangle{\pgfqpoint{0.150000in}{0.150000in}}{\pgfqpoint{4.700000in}{3.450000in}}%
\pgfusepath{clip}%
\pgfsetbuttcap%
\pgfsetroundjoin%
\definecolor{currentfill}{rgb}{0.933517,0.879366,0.883655}%
\pgfsetfillcolor{currentfill}%
\pgfsetlinewidth{0.000000pt}%
\definecolor{currentstroke}{rgb}{0.000000,0.000000,0.000000}%
\pgfsetstrokecolor{currentstroke}%
\pgfsetdash{}{0pt}%
\pgfpathmoveto{\pgfqpoint{3.028720in}{1.417254in}}%
\pgfpathlineto{\pgfqpoint{3.093816in}{1.439572in}}%
\pgfpathlineto{\pgfqpoint{3.026919in}{1.484465in}}%
\pgfpathlineto{\pgfqpoint{2.961801in}{1.465196in}}%
\pgfpathclose%
\pgfusepath{fill}%
\end{pgfscope}%
\begin{pgfscope}%
\pgfpathrectangle{\pgfqpoint{0.150000in}{0.150000in}}{\pgfqpoint{4.700000in}{3.450000in}}%
\pgfusepath{clip}%
\pgfsetbuttcap%
\pgfsetroundjoin%
\definecolor{currentfill}{rgb}{0.860064,0.877298,0.901425}%
\pgfsetfillcolor{currentfill}%
\pgfsetlinewidth{0.000000pt}%
\definecolor{currentstroke}{rgb}{0.000000,0.000000,0.000000}%
\pgfsetstrokecolor{currentstroke}%
\pgfsetdash{}{0pt}%
\pgfpathmoveto{\pgfqpoint{2.231449in}{1.818567in}}%
\pgfpathlineto{\pgfqpoint{2.298490in}{1.843541in}}%
\pgfpathlineto{\pgfqpoint{2.232134in}{1.917930in}}%
\pgfpathlineto{\pgfqpoint{2.164921in}{1.892877in}}%
\pgfpathclose%
\pgfusepath{fill}%
\end{pgfscope}%
\begin{pgfscope}%
\pgfpathrectangle{\pgfqpoint{0.150000in}{0.150000in}}{\pgfqpoint{4.700000in}{3.450000in}}%
\pgfusepath{clip}%
\pgfsetbuttcap%
\pgfsetroundjoin%
\definecolor{currentfill}{rgb}{0.963909,0.934513,0.936841}%
\pgfsetfillcolor{currentfill}%
\pgfsetlinewidth{0.000000pt}%
\definecolor{currentstroke}{rgb}{0.000000,0.000000,0.000000}%
\pgfsetstrokecolor{currentstroke}%
\pgfsetdash{}{0pt}%
\pgfpathmoveto{\pgfqpoint{2.829407in}{1.490406in}}%
\pgfpathlineto{\pgfqpoint{2.895037in}{1.513026in}}%
\pgfpathlineto{\pgfqpoint{2.828426in}{1.560747in}}%
\pgfpathlineto{\pgfqpoint{2.762703in}{1.538274in}}%
\pgfpathclose%
\pgfusepath{fill}%
\end{pgfscope}%
\begin{pgfscope}%
\pgfpathrectangle{\pgfqpoint{0.150000in}{0.150000in}}{\pgfqpoint{4.700000in}{3.450000in}}%
\pgfusepath{clip}%
\pgfsetbuttcap%
\pgfsetroundjoin%
\definecolor{currentfill}{rgb}{0.810309,0.833670,0.866376}%
\pgfsetfillcolor{currentfill}%
\pgfsetlinewidth{0.000000pt}%
\definecolor{currentstroke}{rgb}{0.000000,0.000000,0.000000}%
\pgfsetstrokecolor{currentstroke}%
\pgfsetdash{}{0pt}%
\pgfpathmoveto{\pgfqpoint{2.164921in}{1.892877in}}%
\pgfpathlineto{\pgfqpoint{2.232134in}{1.917930in}}%
\pgfpathlineto{\pgfqpoint{2.165741in}{1.992239in}}%
\pgfpathlineto{\pgfqpoint{2.098355in}{1.967229in}}%
\pgfpathclose%
\pgfusepath{fill}%
\end{pgfscope}%
\begin{pgfscope}%
\pgfpathrectangle{\pgfqpoint{0.150000in}{0.150000in}}{\pgfqpoint{4.700000in}{3.450000in}}%
\pgfusepath{clip}%
\pgfsetbuttcap%
\pgfsetroundjoin%
\definecolor{currentfill}{rgb}{0.891728,0.803539,0.810524}%
\pgfsetfillcolor{currentfill}%
\pgfsetlinewidth{0.000000pt}%
\definecolor{currentstroke}{rgb}{0.000000,0.000000,0.000000}%
\pgfsetstrokecolor{currentstroke}%
\pgfsetdash{}{0pt}%
\pgfpathmoveto{\pgfqpoint{3.295232in}{1.296010in}}%
\pgfpathlineto{\pgfqpoint{3.358995in}{1.302593in}}%
\pgfpathlineto{\pgfqpoint{3.291744in}{1.347547in}}%
\pgfpathlineto{\pgfqpoint{3.227943in}{1.344136in}}%
\pgfpathclose%
\pgfusepath{fill}%
\end{pgfscope}%
\begin{pgfscope}%
\pgfpathrectangle{\pgfqpoint{0.150000in}{0.150000in}}{\pgfqpoint{4.700000in}{3.450000in}}%
\pgfusepath{clip}%
\pgfsetbuttcap%
\pgfsetroundjoin%
\definecolor{currentfill}{rgb}{0.990502,0.982767,0.983379}%
\pgfsetfillcolor{currentfill}%
\pgfsetlinewidth{0.000000pt}%
\definecolor{currentstroke}{rgb}{0.000000,0.000000,0.000000}%
\pgfsetstrokecolor{currentstroke}%
\pgfsetdash{}{0pt}%
\pgfpathmoveto{\pgfqpoint{2.630002in}{1.563591in}}%
\pgfpathlineto{\pgfqpoint{2.696153in}{1.586031in}}%
\pgfpathlineto{\pgfqpoint{2.629757in}{1.633678in}}%
\pgfpathlineto{\pgfqpoint{2.563514in}{1.611385in}}%
\pgfpathclose%
\pgfusepath{fill}%
\end{pgfscope}%
\begin{pgfscope}%
\pgfpathrectangle{\pgfqpoint{0.150000in}{0.150000in}}{\pgfqpoint{4.700000in}{3.450000in}}%
\pgfusepath{clip}%
\pgfsetbuttcap%
\pgfsetroundjoin%
\definecolor{currentfill}{rgb}{0.362515,0.441023,0.550934}%
\pgfsetfillcolor{currentfill}%
\pgfsetlinewidth{0.000000pt}%
\definecolor{currentstroke}{rgb}{0.000000,0.000000,0.000000}%
\pgfsetstrokecolor{currentstroke}%
\pgfsetdash{}{0pt}%
\pgfpathmoveto{\pgfqpoint{1.433113in}{2.669585in}}%
\pgfpathlineto{\pgfqpoint{1.503378in}{2.669621in}}%
\pgfpathlineto{\pgfqpoint{1.437282in}{2.735599in}}%
\pgfpathlineto{\pgfqpoint{1.366792in}{2.735797in}}%
\pgfpathclose%
\pgfusepath{fill}%
\end{pgfscope}%
\begin{pgfscope}%
\pgfpathrectangle{\pgfqpoint{0.150000in}{0.150000in}}{\pgfqpoint{4.700000in}{3.450000in}}%
\pgfusepath{clip}%
\pgfsetbuttcap%
\pgfsetroundjoin%
\definecolor{currentfill}{rgb}{0.972013,0.975460,0.980285}%
\pgfsetfillcolor{currentfill}%
\pgfsetlinewidth{0.000000pt}%
\definecolor{currentstroke}{rgb}{0.000000,0.000000,0.000000}%
\pgfsetstrokecolor{currentstroke}%
\pgfsetdash{}{0pt}%
\pgfpathmoveto{\pgfqpoint{2.430506in}{1.636810in}}%
\pgfpathlineto{\pgfqpoint{2.497179in}{1.659069in}}%
\pgfpathlineto{\pgfqpoint{2.430997in}{1.706643in}}%
\pgfpathlineto{\pgfqpoint{2.364233in}{1.684530in}}%
\pgfpathclose%
\pgfusepath{fill}%
\end{pgfscope}%
\begin{pgfscope}%
\pgfpathrectangle{\pgfqpoint{0.150000in}{0.150000in}}{\pgfqpoint{4.700000in}{3.450000in}}%
\pgfusepath{clip}%
\pgfsetbuttcap%
\pgfsetroundjoin%
\definecolor{currentfill}{rgb}{0.399831,0.473744,0.577221}%
\pgfsetfillcolor{currentfill}%
\pgfsetlinewidth{0.000000pt}%
\definecolor{currentstroke}{rgb}{0.000000,0.000000,0.000000}%
\pgfsetstrokecolor{currentstroke}%
\pgfsetdash{}{0pt}%
\pgfpathmoveto{\pgfqpoint{1.499454in}{2.603353in}}%
\pgfpathlineto{\pgfqpoint{1.569493in}{2.603624in}}%
\pgfpathlineto{\pgfqpoint{1.503378in}{2.669621in}}%
\pgfpathlineto{\pgfqpoint{1.433113in}{2.669585in}}%
\pgfpathclose%
\pgfusepath{fill}%
\end{pgfscope}%
\begin{pgfscope}%
\pgfpathrectangle{\pgfqpoint{0.150000in}{0.150000in}}{\pgfqpoint{4.700000in}{3.450000in}}%
\pgfusepath{clip}%
\pgfsetbuttcap%
\pgfsetroundjoin%
\definecolor{currentfill}{rgb}{0.766774,0.795496,0.835708}%
\pgfsetfillcolor{currentfill}%
\pgfsetlinewidth{0.000000pt}%
\definecolor{currentstroke}{rgb}{0.000000,0.000000,0.000000}%
\pgfsetstrokecolor{currentstroke}%
\pgfsetdash{}{0pt}%
\pgfpathmoveto{\pgfqpoint{2.098355in}{1.967229in}}%
\pgfpathlineto{\pgfqpoint{2.165741in}{1.992239in}}%
\pgfpathlineto{\pgfqpoint{2.099306in}{2.066717in}}%
\pgfpathlineto{\pgfqpoint{2.031751in}{2.041624in}}%
\pgfpathclose%
\pgfusepath{fill}%
\end{pgfscope}%
\begin{pgfscope}%
\pgfpathrectangle{\pgfqpoint{0.150000in}{0.150000in}}{\pgfqpoint{4.700000in}{3.450000in}}%
\pgfusepath{clip}%
\pgfsetbuttcap%
\pgfsetroundjoin%
\definecolor{currentfill}{rgb}{0.437148,0.506464,0.603508}%
\pgfsetfillcolor{currentfill}%
\pgfsetlinewidth{0.000000pt}%
\definecolor{currentstroke}{rgb}{0.000000,0.000000,0.000000}%
\pgfsetstrokecolor{currentstroke}%
\pgfsetdash{}{0pt}%
\pgfpathmoveto{\pgfqpoint{1.565813in}{2.537102in}}%
\pgfpathlineto{\pgfqpoint{1.635628in}{2.537609in}}%
\pgfpathlineto{\pgfqpoint{1.569493in}{2.603624in}}%
\pgfpathlineto{\pgfqpoint{1.499454in}{2.603353in}}%
\pgfpathclose%
\pgfusepath{fill}%
\end{pgfscope}%
\begin{pgfscope}%
\pgfpathrectangle{\pgfqpoint{0.150000in}{0.150000in}}{\pgfqpoint{4.700000in}{3.450000in}}%
\pgfusepath{clip}%
\pgfsetbuttcap%
\pgfsetroundjoin%
\definecolor{currentfill}{rgb}{0.474464,0.539185,0.629795}%
\pgfsetfillcolor{currentfill}%
\pgfsetlinewidth{0.000000pt}%
\definecolor{currentstroke}{rgb}{0.000000,0.000000,0.000000}%
\pgfsetstrokecolor{currentstroke}%
\pgfsetdash{}{0pt}%
\pgfpathmoveto{\pgfqpoint{1.632192in}{2.470833in}}%
\pgfpathlineto{\pgfqpoint{1.701781in}{2.471574in}}%
\pgfpathlineto{\pgfqpoint{1.635628in}{2.537609in}}%
\pgfpathlineto{\pgfqpoint{1.565813in}{2.537102in}}%
\pgfpathclose%
\pgfusepath{fill}%
\end{pgfscope}%
\begin{pgfscope}%
\pgfpathrectangle{\pgfqpoint{0.150000in}{0.150000in}}{\pgfqpoint{4.700000in}{3.450000in}}%
\pgfusepath{clip}%
\pgfsetbuttcap%
\pgfsetroundjoin%
\definecolor{currentfill}{rgb}{0.717019,0.751869,0.800659}%
\pgfsetfillcolor{currentfill}%
\pgfsetlinewidth{0.000000pt}%
\definecolor{currentstroke}{rgb}{0.000000,0.000000,0.000000}%
\pgfsetstrokecolor{currentstroke}%
\pgfsetdash{}{0pt}%
\pgfpathmoveto{\pgfqpoint{2.031751in}{2.041624in}}%
\pgfpathlineto{\pgfqpoint{2.099306in}{2.066717in}}%
\pgfpathlineto{\pgfqpoint{2.032834in}{2.141114in}}%
\pgfpathlineto{\pgfqpoint{1.965108in}{2.116062in}}%
\pgfpathclose%
\pgfusepath{fill}%
\end{pgfscope}%
\begin{pgfscope}%
\pgfpathrectangle{\pgfqpoint{0.150000in}{0.150000in}}{\pgfqpoint{4.700000in}{3.450000in}}%
\pgfusepath{clip}%
\pgfsetbuttcap%
\pgfsetroundjoin%
\definecolor{currentfill}{rgb}{0.511780,0.571906,0.656081}%
\pgfsetfillcolor{currentfill}%
\pgfsetlinewidth{0.000000pt}%
\definecolor{currentstroke}{rgb}{0.000000,0.000000,0.000000}%
\pgfsetstrokecolor{currentstroke}%
\pgfsetdash{}{0pt}%
\pgfpathmoveto{\pgfqpoint{1.698589in}{2.404544in}}%
\pgfpathlineto{\pgfqpoint{1.767953in}{2.405520in}}%
\pgfpathlineto{\pgfqpoint{1.701781in}{2.471574in}}%
\pgfpathlineto{\pgfqpoint{1.632192in}{2.470833in}}%
\pgfpathclose%
\pgfusepath{fill}%
\end{pgfscope}%
\begin{pgfscope}%
\pgfpathrectangle{\pgfqpoint{0.150000in}{0.150000in}}{\pgfqpoint{4.700000in}{3.450000in}}%
\pgfusepath{clip}%
\pgfsetbuttcap%
\pgfsetroundjoin%
\definecolor{currentfill}{rgb}{0.925919,0.865579,0.870358}%
\pgfsetfillcolor{currentfill}%
\pgfsetlinewidth{0.000000pt}%
\definecolor{currentstroke}{rgb}{0.000000,0.000000,0.000000}%
\pgfsetstrokecolor{currentstroke}%
\pgfsetdash{}{0pt}%
\pgfpathmoveto{\pgfqpoint{3.095794in}{1.369202in}}%
\pgfpathlineto{\pgfqpoint{3.160809in}{1.392151in}}%
\pgfpathlineto{\pgfqpoint{3.093816in}{1.439572in}}%
\pgfpathlineto{\pgfqpoint{3.028720in}{1.417254in}}%
\pgfpathclose%
\pgfusepath{fill}%
\end{pgfscope}%
\begin{pgfscope}%
\pgfpathrectangle{\pgfqpoint{0.150000in}{0.150000in}}{\pgfqpoint{4.700000in}{3.450000in}}%
\pgfusepath{clip}%
\pgfsetbuttcap%
\pgfsetroundjoin%
\definecolor{currentfill}{rgb}{0.673483,0.713695,0.769991}%
\pgfsetfillcolor{currentfill}%
\pgfsetlinewidth{0.000000pt}%
\definecolor{currentstroke}{rgb}{0.000000,0.000000,0.000000}%
\pgfsetstrokecolor{currentstroke}%
\pgfsetdash{}{0pt}%
\pgfpathmoveto{\pgfqpoint{1.965108in}{2.116062in}}%
\pgfpathlineto{\pgfqpoint{2.032834in}{2.141114in}}%
\pgfpathlineto{\pgfqpoint{1.966585in}{2.207244in}}%
\pgfpathlineto{\pgfqpoint{1.898426in}{2.190543in}}%
\pgfpathclose%
\pgfusepath{fill}%
\end{pgfscope}%
\begin{pgfscope}%
\pgfpathrectangle{\pgfqpoint{0.150000in}{0.150000in}}{\pgfqpoint{4.700000in}{3.450000in}}%
\pgfusepath{clip}%
\pgfsetbuttcap%
\pgfsetroundjoin%
\definecolor{currentfill}{rgb}{0.549096,0.604626,0.682368}%
\pgfsetfillcolor{currentfill}%
\pgfsetlinewidth{0.000000pt}%
\definecolor{currentstroke}{rgb}{0.000000,0.000000,0.000000}%
\pgfsetstrokecolor{currentstroke}%
\pgfsetdash{}{0pt}%
\pgfpathmoveto{\pgfqpoint{1.765007in}{2.338235in}}%
\pgfpathlineto{\pgfqpoint{1.834145in}{2.339447in}}%
\pgfpathlineto{\pgfqpoint{1.767953in}{2.405520in}}%
\pgfpathlineto{\pgfqpoint{1.698589in}{2.404544in}}%
\pgfpathclose%
\pgfusepath{fill}%
\end{pgfscope}%
\begin{pgfscope}%
\pgfpathrectangle{\pgfqpoint{0.150000in}{0.150000in}}{\pgfqpoint{4.700000in}{3.450000in}}%
\pgfusepath{clip}%
\pgfsetbuttcap%
\pgfsetroundjoin%
\definecolor{currentfill}{rgb}{0.629948,0.675521,0.739323}%
\pgfsetfillcolor{currentfill}%
\pgfsetlinewidth{0.000000pt}%
\definecolor{currentstroke}{rgb}{0.000000,0.000000,0.000000}%
\pgfsetstrokecolor{currentstroke}%
\pgfsetdash{}{0pt}%
\pgfpathmoveto{\pgfqpoint{1.898426in}{2.190543in}}%
\pgfpathlineto{\pgfqpoint{1.966585in}{2.207244in}}%
\pgfpathlineto{\pgfqpoint{1.900356in}{2.273355in}}%
\pgfpathlineto{\pgfqpoint{1.831706in}{2.265067in}}%
\pgfpathclose%
\pgfusepath{fill}%
\end{pgfscope}%
\begin{pgfscope}%
\pgfpathrectangle{\pgfqpoint{0.150000in}{0.150000in}}{\pgfqpoint{4.700000in}{3.450000in}}%
\pgfusepath{clip}%
\pgfsetbuttcap%
\pgfsetroundjoin%
\definecolor{currentfill}{rgb}{0.586412,0.637347,0.708655}%
\pgfsetfillcolor{currentfill}%
\pgfsetlinewidth{0.000000pt}%
\definecolor{currentstroke}{rgb}{0.000000,0.000000,0.000000}%
\pgfsetstrokecolor{currentstroke}%
\pgfsetdash{}{0pt}%
\pgfpathmoveto{\pgfqpoint{1.831706in}{2.265067in}}%
\pgfpathlineto{\pgfqpoint{1.900356in}{2.273355in}}%
\pgfpathlineto{\pgfqpoint{1.834145in}{2.339447in}}%
\pgfpathlineto{\pgfqpoint{1.765007in}{2.338235in}}%
\pgfpathclose%
\pgfusepath{fill}%
\end{pgfscope}%
\begin{pgfscope}%
\pgfpathrectangle{\pgfqpoint{0.150000in}{0.150000in}}{\pgfqpoint{4.700000in}{3.450000in}}%
\pgfusepath{clip}%
\pgfsetbuttcap%
\pgfsetroundjoin%
\definecolor{currentfill}{rgb}{0.952512,0.913833,0.916896}%
\pgfsetfillcolor{currentfill}%
\pgfsetlinewidth{0.000000pt}%
\definecolor{currentstroke}{rgb}{0.000000,0.000000,0.000000}%
\pgfsetstrokecolor{currentstroke}%
\pgfsetdash{}{0pt}%
\pgfpathmoveto{\pgfqpoint{2.896265in}{1.442427in}}%
\pgfpathlineto{\pgfqpoint{2.961801in}{1.465196in}}%
\pgfpathlineto{\pgfqpoint{2.895037in}{1.513026in}}%
\pgfpathlineto{\pgfqpoint{2.829407in}{1.490406in}}%
\pgfpathclose%
\pgfusepath{fill}%
\end{pgfscope}%
\begin{pgfscope}%
\pgfpathrectangle{\pgfqpoint{0.150000in}{0.150000in}}{\pgfqpoint{4.700000in}{3.450000in}}%
\pgfusepath{clip}%
\pgfsetbuttcap%
\pgfsetroundjoin%
\definecolor{currentfill}{rgb}{0.982904,0.968980,0.970083}%
\pgfsetfillcolor{currentfill}%
\pgfsetlinewidth{0.000000pt}%
\definecolor{currentstroke}{rgb}{0.000000,0.000000,0.000000}%
\pgfsetstrokecolor{currentstroke}%
\pgfsetdash{}{0pt}%
\pgfpathmoveto{\pgfqpoint{2.696645in}{1.515686in}}%
\pgfpathlineto{\pgfqpoint{2.762703in}{1.538274in}}%
\pgfpathlineto{\pgfqpoint{2.696153in}{1.586031in}}%
\pgfpathlineto{\pgfqpoint{2.630002in}{1.563591in}}%
\pgfpathclose%
\pgfusepath{fill}%
\end{pgfscope}%
\begin{pgfscope}%
\pgfpathrectangle{\pgfqpoint{0.150000in}{0.150000in}}{\pgfqpoint{4.700000in}{3.450000in}}%
\pgfusepath{clip}%
\pgfsetbuttcap%
\pgfsetroundjoin%
\definecolor{currentfill}{rgb}{0.984452,0.986366,0.989047}%
\pgfsetfillcolor{currentfill}%
\pgfsetlinewidth{0.000000pt}%
\definecolor{currentstroke}{rgb}{0.000000,0.000000,0.000000}%
\pgfsetstrokecolor{currentstroke}%
\pgfsetdash{}{0pt}%
\pgfpathmoveto{\pgfqpoint{2.496933in}{1.588979in}}%
\pgfpathlineto{\pgfqpoint{2.563514in}{1.611385in}}%
\pgfpathlineto{\pgfqpoint{2.497179in}{1.659069in}}%
\pgfpathlineto{\pgfqpoint{2.430506in}{1.636810in}}%
\pgfpathclose%
\pgfusepath{fill}%
\end{pgfscope}%
\begin{pgfscope}%
\pgfpathrectangle{\pgfqpoint{0.150000in}{0.150000in}}{\pgfqpoint{4.700000in}{3.450000in}}%
\pgfusepath{clip}%
\pgfsetbuttcap%
\pgfsetroundjoin%
\definecolor{currentfill}{rgb}{0.897381,0.910018,0.927711}%
\pgfsetfillcolor{currentfill}%
\pgfsetlinewidth{0.000000pt}%
\definecolor{currentstroke}{rgb}{0.000000,0.000000,0.000000}%
\pgfsetstrokecolor{currentstroke}%
\pgfsetdash{}{0pt}%
\pgfpathmoveto{\pgfqpoint{2.230751in}{1.719409in}}%
\pgfpathlineto{\pgfqpoint{2.297939in}{1.744301in}}%
\pgfpathlineto{\pgfqpoint{2.231449in}{1.818567in}}%
\pgfpathlineto{\pgfqpoint{2.164088in}{1.793599in}}%
\pgfpathclose%
\pgfusepath{fill}%
\end{pgfscope}%
\begin{pgfscope}%
\pgfpathrectangle{\pgfqpoint{0.150000in}{0.150000in}}{\pgfqpoint{4.700000in}{3.450000in}}%
\pgfusepath{clip}%
\pgfsetbuttcap%
\pgfsetroundjoin%
\definecolor{currentfill}{rgb}{0.914522,0.844899,0.850414}%
\pgfsetfillcolor{currentfill}%
\pgfsetlinewidth{0.000000pt}%
\definecolor{currentstroke}{rgb}{0.000000,0.000000,0.000000}%
\pgfsetstrokecolor{currentstroke}%
\pgfsetdash{}{0pt}%
\pgfpathmoveto{\pgfqpoint{3.163023in}{1.321038in}}%
\pgfpathlineto{\pgfqpoint{3.227943in}{1.344136in}}%
\pgfpathlineto{\pgfqpoint{3.160809in}{1.392151in}}%
\pgfpathlineto{\pgfqpoint{3.095794in}{1.369202in}}%
\pgfpathclose%
\pgfusepath{fill}%
\end{pgfscope}%
\begin{pgfscope}%
\pgfpathrectangle{\pgfqpoint{0.150000in}{0.150000in}}{\pgfqpoint{4.700000in}{3.450000in}}%
\pgfusepath{clip}%
\pgfsetbuttcap%
\pgfsetroundjoin%
\definecolor{currentfill}{rgb}{0.934697,0.942739,0.953998}%
\pgfsetfillcolor{currentfill}%
\pgfsetlinewidth{0.000000pt}%
\definecolor{currentstroke}{rgb}{0.000000,0.000000,0.000000}%
\pgfsetstrokecolor{currentstroke}%
\pgfsetdash{}{0pt}%
\pgfpathmoveto{\pgfqpoint{2.297130in}{1.662305in}}%
\pgfpathlineto{\pgfqpoint{2.364233in}{1.684530in}}%
\pgfpathlineto{\pgfqpoint{2.297939in}{1.744301in}}%
\pgfpathlineto{\pgfqpoint{2.230751in}{1.719409in}}%
\pgfpathclose%
\pgfusepath{fill}%
\end{pgfscope}%
\begin{pgfscope}%
\pgfpathrectangle{\pgfqpoint{0.150000in}{0.150000in}}{\pgfqpoint{4.700000in}{3.450000in}}%
\pgfusepath{clip}%
\pgfsetbuttcap%
\pgfsetroundjoin%
\definecolor{currentfill}{rgb}{0.847626,0.866391,0.892662}%
\pgfsetfillcolor{currentfill}%
\pgfsetlinewidth{0.000000pt}%
\definecolor{currentstroke}{rgb}{0.000000,0.000000,0.000000}%
\pgfsetstrokecolor{currentstroke}%
\pgfsetdash{}{0pt}%
\pgfpathmoveto{\pgfqpoint{2.164088in}{1.793599in}}%
\pgfpathlineto{\pgfqpoint{2.231449in}{1.818567in}}%
\pgfpathlineto{\pgfqpoint{2.164921in}{1.892877in}}%
\pgfpathlineto{\pgfqpoint{2.097389in}{1.867829in}}%
\pgfpathclose%
\pgfusepath{fill}%
\end{pgfscope}%
\begin{pgfscope}%
\pgfpathrectangle{\pgfqpoint{0.150000in}{0.150000in}}{\pgfqpoint{4.700000in}{3.450000in}}%
\pgfusepath{clip}%
\pgfsetbuttcap%
\pgfsetroundjoin%
\definecolor{currentfill}{rgb}{0.944914,0.900046,0.903600}%
\pgfsetfillcolor{currentfill}%
\pgfsetlinewidth{0.000000pt}%
\definecolor{currentstroke}{rgb}{0.000000,0.000000,0.000000}%
\pgfsetstrokecolor{currentstroke}%
\pgfsetdash{}{0pt}%
\pgfpathmoveto{\pgfqpoint{2.963278in}{1.394337in}}%
\pgfpathlineto{\pgfqpoint{3.028720in}{1.417254in}}%
\pgfpathlineto{\pgfqpoint{2.961801in}{1.465196in}}%
\pgfpathlineto{\pgfqpoint{2.896265in}{1.442427in}}%
\pgfpathclose%
\pgfusepath{fill}%
\end{pgfscope}%
\begin{pgfscope}%
\pgfpathrectangle{\pgfqpoint{0.150000in}{0.150000in}}{\pgfqpoint{4.700000in}{3.450000in}}%
\pgfusepath{clip}%
\pgfsetbuttcap%
\pgfsetroundjoin%
\definecolor{currentfill}{rgb}{0.971507,0.948300,0.950138}%
\pgfsetfillcolor{currentfill}%
\pgfsetlinewidth{0.000000pt}%
\definecolor{currentstroke}{rgb}{0.000000,0.000000,0.000000}%
\pgfsetstrokecolor{currentstroke}%
\pgfsetdash{}{0pt}%
\pgfpathmoveto{\pgfqpoint{2.763442in}{1.467670in}}%
\pgfpathlineto{\pgfqpoint{2.829407in}{1.490406in}}%
\pgfpathlineto{\pgfqpoint{2.762703in}{1.538274in}}%
\pgfpathlineto{\pgfqpoint{2.696645in}{1.515686in}}%
\pgfpathclose%
\pgfusepath{fill}%
\end{pgfscope}%
\begin{pgfscope}%
\pgfpathrectangle{\pgfqpoint{0.150000in}{0.150000in}}{\pgfqpoint{4.700000in}{3.450000in}}%
\pgfusepath{clip}%
\pgfsetbuttcap%
\pgfsetroundjoin%
\definecolor{currentfill}{rgb}{0.797871,0.822763,0.857613}%
\pgfsetfillcolor{currentfill}%
\pgfsetlinewidth{0.000000pt}%
\definecolor{currentstroke}{rgb}{0.000000,0.000000,0.000000}%
\pgfsetstrokecolor{currentstroke}%
\pgfsetdash{}{0pt}%
\pgfpathmoveto{\pgfqpoint{2.097389in}{1.867829in}}%
\pgfpathlineto{\pgfqpoint{2.164921in}{1.892877in}}%
\pgfpathlineto{\pgfqpoint{2.098355in}{1.967229in}}%
\pgfpathlineto{\pgfqpoint{2.030652in}{1.942101in}}%
\pgfpathclose%
\pgfusepath{fill}%
\end{pgfscope}%
\begin{pgfscope}%
\pgfpathrectangle{\pgfqpoint{0.150000in}{0.150000in}}{\pgfqpoint{4.700000in}{3.450000in}}%
\pgfusepath{clip}%
\pgfsetbuttcap%
\pgfsetroundjoin%
\definecolor{currentfill}{rgb}{0.996890,0.997273,0.997809}%
\pgfsetfillcolor{currentfill}%
\pgfsetlinewidth{0.000000pt}%
\definecolor{currentstroke}{rgb}{0.000000,0.000000,0.000000}%
\pgfsetstrokecolor{currentstroke}%
\pgfsetdash{}{0pt}%
\pgfpathmoveto{\pgfqpoint{2.563514in}{1.541037in}}%
\pgfpathlineto{\pgfqpoint{2.630002in}{1.563591in}}%
\pgfpathlineto{\pgfqpoint{2.563514in}{1.611385in}}%
\pgfpathlineto{\pgfqpoint{2.496933in}{1.588979in}}%
\pgfpathclose%
\pgfusepath{fill}%
\end{pgfscope}%
\begin{pgfscope}%
\pgfpathrectangle{\pgfqpoint{0.150000in}{0.150000in}}{\pgfqpoint{4.700000in}{3.450000in}}%
\pgfusepath{clip}%
\pgfsetbuttcap%
\pgfsetroundjoin%
\definecolor{currentfill}{rgb}{0.754335,0.784589,0.826945}%
\pgfsetfillcolor{currentfill}%
\pgfsetlinewidth{0.000000pt}%
\definecolor{currentstroke}{rgb}{0.000000,0.000000,0.000000}%
\pgfsetstrokecolor{currentstroke}%
\pgfsetdash{}{0pt}%
\pgfpathmoveto{\pgfqpoint{2.030652in}{1.942101in}}%
\pgfpathlineto{\pgfqpoint{2.098355in}{1.967229in}}%
\pgfpathlineto{\pgfqpoint{2.031751in}{2.041624in}}%
\pgfpathlineto{\pgfqpoint{1.963879in}{2.016414in}}%
\pgfpathclose%
\pgfusepath{fill}%
\end{pgfscope}%
\begin{pgfscope}%
\pgfpathrectangle{\pgfqpoint{0.150000in}{0.150000in}}{\pgfqpoint{4.700000in}{3.450000in}}%
\pgfusepath{clip}%
\pgfsetbuttcap%
\pgfsetroundjoin%
\definecolor{currentfill}{rgb}{0.953355,0.959099,0.967142}%
\pgfsetfillcolor{currentfill}%
\pgfsetlinewidth{0.000000pt}%
\definecolor{currentstroke}{rgb}{0.000000,0.000000,0.000000}%
\pgfsetstrokecolor{currentstroke}%
\pgfsetdash{}{0pt}%
\pgfpathmoveto{\pgfqpoint{2.363494in}{1.614437in}}%
\pgfpathlineto{\pgfqpoint{2.430506in}{1.636810in}}%
\pgfpathlineto{\pgfqpoint{2.364233in}{1.684530in}}%
\pgfpathlineto{\pgfqpoint{2.297130in}{1.662305in}}%
\pgfpathclose%
\pgfusepath{fill}%
\end{pgfscope}%
\begin{pgfscope}%
\pgfpathrectangle{\pgfqpoint{0.150000in}{0.150000in}}{\pgfqpoint{4.700000in}{3.450000in}}%
\pgfusepath{clip}%
\pgfsetbuttcap%
\pgfsetroundjoin%
\definecolor{currentfill}{rgb}{0.704580,0.740962,0.791896}%
\pgfsetfillcolor{currentfill}%
\pgfsetlinewidth{0.000000pt}%
\definecolor{currentstroke}{rgb}{0.000000,0.000000,0.000000}%
\pgfsetstrokecolor{currentstroke}%
\pgfsetdash{}{0pt}%
\pgfpathmoveto{\pgfqpoint{1.963879in}{2.016414in}}%
\pgfpathlineto{\pgfqpoint{2.031751in}{2.041624in}}%
\pgfpathlineto{\pgfqpoint{1.965108in}{2.116062in}}%
\pgfpathlineto{\pgfqpoint{1.897070in}{2.090768in}}%
\pgfpathclose%
\pgfusepath{fill}%
\end{pgfscope}%
\begin{pgfscope}%
\pgfpathrectangle{\pgfqpoint{0.150000in}{0.150000in}}{\pgfqpoint{4.700000in}{3.450000in}}%
\pgfusepath{clip}%
\pgfsetbuttcap%
\pgfsetroundjoin%
\definecolor{currentfill}{rgb}{0.906924,0.831112,0.837117}%
\pgfsetfillcolor{currentfill}%
\pgfsetlinewidth{0.000000pt}%
\definecolor{currentstroke}{rgb}{0.000000,0.000000,0.000000}%
\pgfsetstrokecolor{currentstroke}%
\pgfsetdash{}{0pt}%
\pgfpathmoveto{\pgfqpoint{3.230409in}{1.272762in}}%
\pgfpathlineto{\pgfqpoint{3.295232in}{1.296010in}}%
\pgfpathlineto{\pgfqpoint{3.227943in}{1.344136in}}%
\pgfpathlineto{\pgfqpoint{3.163023in}{1.321038in}}%
\pgfpathclose%
\pgfusepath{fill}%
\end{pgfscope}%
\begin{pgfscope}%
\pgfpathrectangle{\pgfqpoint{0.150000in}{0.150000in}}{\pgfqpoint{4.700000in}{3.450000in}}%
\pgfusepath{clip}%
\pgfsetbuttcap%
\pgfsetroundjoin%
\definecolor{currentfill}{rgb}{0.661045,0.702788,0.761229}%
\pgfsetfillcolor{currentfill}%
\pgfsetlinewidth{0.000000pt}%
\definecolor{currentstroke}{rgb}{0.000000,0.000000,0.000000}%
\pgfsetstrokecolor{currentstroke}%
\pgfsetdash{}{0pt}%
\pgfpathmoveto{\pgfqpoint{1.897070in}{2.090768in}}%
\pgfpathlineto{\pgfqpoint{1.965108in}{2.116062in}}%
\pgfpathlineto{\pgfqpoint{1.898426in}{2.190543in}}%
\pgfpathlineto{\pgfqpoint{1.830218in}{2.165288in}}%
\pgfpathclose%
\pgfusepath{fill}%
\end{pgfscope}%
\begin{pgfscope}%
\pgfpathrectangle{\pgfqpoint{0.150000in}{0.150000in}}{\pgfqpoint{4.700000in}{3.450000in}}%
\pgfusepath{clip}%
\pgfsetbuttcap%
\pgfsetroundjoin%
\definecolor{currentfill}{rgb}{0.933517,0.879366,0.883655}%
\pgfsetfillcolor{currentfill}%
\pgfsetlinewidth{0.000000pt}%
\definecolor{currentstroke}{rgb}{0.000000,0.000000,0.000000}%
\pgfsetstrokecolor{currentstroke}%
\pgfsetdash{}{0pt}%
\pgfpathmoveto{\pgfqpoint{3.030447in}{1.346136in}}%
\pgfpathlineto{\pgfqpoint{3.095794in}{1.369202in}}%
\pgfpathlineto{\pgfqpoint{3.028720in}{1.417254in}}%
\pgfpathlineto{\pgfqpoint{2.963278in}{1.394337in}}%
\pgfpathclose%
\pgfusepath{fill}%
\end{pgfscope}%
\begin{pgfscope}%
\pgfpathrectangle{\pgfqpoint{0.150000in}{0.150000in}}{\pgfqpoint{4.700000in}{3.450000in}}%
\pgfusepath{clip}%
\pgfsetbuttcap%
\pgfsetroundjoin%
\definecolor{currentfill}{rgb}{0.611290,0.659161,0.726180}%
\pgfsetfillcolor{currentfill}%
\pgfsetlinewidth{0.000000pt}%
\definecolor{currentstroke}{rgb}{0.000000,0.000000,0.000000}%
\pgfsetstrokecolor{currentstroke}%
\pgfsetdash{}{0pt}%
\pgfpathmoveto{\pgfqpoint{1.830218in}{2.165288in}}%
\pgfpathlineto{\pgfqpoint{1.898426in}{2.190543in}}%
\pgfpathlineto{\pgfqpoint{1.831706in}{2.265067in}}%
\pgfpathlineto{\pgfqpoint{1.763334in}{2.239725in}}%
\pgfpathclose%
\pgfusepath{fill}%
\end{pgfscope}%
\begin{pgfscope}%
\pgfpathrectangle{\pgfqpoint{0.150000in}{0.150000in}}{\pgfqpoint{4.700000in}{3.450000in}}%
\pgfusepath{clip}%
\pgfsetbuttcap%
\pgfsetroundjoin%
\definecolor{currentfill}{rgb}{0.963909,0.934513,0.936841}%
\pgfsetfillcolor{currentfill}%
\pgfsetlinewidth{0.000000pt}%
\definecolor{currentstroke}{rgb}{0.000000,0.000000,0.000000}%
\pgfsetstrokecolor{currentstroke}%
\pgfsetdash{}{0pt}%
\pgfpathmoveto{\pgfqpoint{2.830394in}{1.419543in}}%
\pgfpathlineto{\pgfqpoint{2.896265in}{1.442427in}}%
\pgfpathlineto{\pgfqpoint{2.829407in}{1.490406in}}%
\pgfpathlineto{\pgfqpoint{2.763442in}{1.467670in}}%
\pgfpathclose%
\pgfusepath{fill}%
\end{pgfscope}%
\begin{pgfscope}%
\pgfpathrectangle{\pgfqpoint{0.150000in}{0.150000in}}{\pgfqpoint{4.700000in}{3.450000in}}%
\pgfusepath{clip}%
\pgfsetbuttcap%
\pgfsetroundjoin%
\definecolor{currentfill}{rgb}{0.318980,0.402849,0.520267}%
\pgfsetfillcolor{currentfill}%
\pgfsetlinewidth{0.000000pt}%
\definecolor{currentstroke}{rgb}{0.000000,0.000000,0.000000}%
\pgfsetstrokecolor{currentstroke}%
\pgfsetdash{}{0pt}%
\pgfpathmoveto{\pgfqpoint{1.362347in}{2.669548in}}%
\pgfpathlineto{\pgfqpoint{1.433113in}{2.669585in}}%
\pgfpathlineto{\pgfqpoint{1.366792in}{2.735797in}}%
\pgfpathlineto{\pgfqpoint{1.295799in}{2.735997in}}%
\pgfpathclose%
\pgfusepath{fill}%
\end{pgfscope}%
\begin{pgfscope}%
\pgfpathrectangle{\pgfqpoint{0.150000in}{0.150000in}}{\pgfqpoint{4.700000in}{3.450000in}}%
\pgfusepath{clip}%
\pgfsetbuttcap%
\pgfsetroundjoin%
\definecolor{currentfill}{rgb}{0.990502,0.982767,0.983379}%
\pgfsetfillcolor{currentfill}%
\pgfsetlinewidth{0.000000pt}%
\definecolor{currentstroke}{rgb}{0.000000,0.000000,0.000000}%
\pgfsetstrokecolor{currentstroke}%
\pgfsetdash{}{0pt}%
\pgfpathmoveto{\pgfqpoint{2.630249in}{1.492984in}}%
\pgfpathlineto{\pgfqpoint{2.696645in}{1.515686in}}%
\pgfpathlineto{\pgfqpoint{2.630002in}{1.563591in}}%
\pgfpathlineto{\pgfqpoint{2.563514in}{1.541037in}}%
\pgfpathclose%
\pgfusepath{fill}%
\end{pgfscope}%
\begin{pgfscope}%
\pgfpathrectangle{\pgfqpoint{0.150000in}{0.150000in}}{\pgfqpoint{4.700000in}{3.450000in}}%
\pgfusepath{clip}%
\pgfsetbuttcap%
\pgfsetroundjoin%
\definecolor{currentfill}{rgb}{0.567754,0.620987,0.695512}%
\pgfsetfillcolor{currentfill}%
\pgfsetlinewidth{0.000000pt}%
\definecolor{currentstroke}{rgb}{0.000000,0.000000,0.000000}%
\pgfsetstrokecolor{currentstroke}%
\pgfsetdash{}{0pt}%
\pgfpathmoveto{\pgfqpoint{1.763334in}{2.239725in}}%
\pgfpathlineto{\pgfqpoint{1.831706in}{2.265067in}}%
\pgfpathlineto{\pgfqpoint{1.765007in}{2.338235in}}%
\pgfpathlineto{\pgfqpoint{1.696413in}{2.314202in}}%
\pgfpathclose%
\pgfusepath{fill}%
\end{pgfscope}%
\begin{pgfscope}%
\pgfpathrectangle{\pgfqpoint{0.150000in}{0.150000in}}{\pgfqpoint{4.700000in}{3.450000in}}%
\pgfusepath{clip}%
\pgfsetbuttcap%
\pgfsetroundjoin%
\definecolor{currentfill}{rgb}{0.362515,0.441023,0.550934}%
\pgfsetfillcolor{currentfill}%
\pgfsetlinewidth{0.000000pt}%
\definecolor{currentstroke}{rgb}{0.000000,0.000000,0.000000}%
\pgfsetstrokecolor{currentstroke}%
\pgfsetdash{}{0pt}%
\pgfpathmoveto{\pgfqpoint{1.428914in}{2.603080in}}%
\pgfpathlineto{\pgfqpoint{1.499454in}{2.603353in}}%
\pgfpathlineto{\pgfqpoint{1.433113in}{2.669585in}}%
\pgfpathlineto{\pgfqpoint{1.362347in}{2.669548in}}%
\pgfpathclose%
\pgfusepath{fill}%
\end{pgfscope}%
\begin{pgfscope}%
\pgfpathrectangle{\pgfqpoint{0.150000in}{0.150000in}}{\pgfqpoint{4.700000in}{3.450000in}}%
\pgfusepath{clip}%
\pgfsetbuttcap%
\pgfsetroundjoin%
\definecolor{currentfill}{rgb}{0.524219,0.582812,0.664844}%
\pgfsetfillcolor{currentfill}%
\pgfsetlinewidth{0.000000pt}%
\definecolor{currentstroke}{rgb}{0.000000,0.000000,0.000000}%
\pgfsetstrokecolor{currentstroke}%
\pgfsetdash{}{0pt}%
\pgfpathmoveto{\pgfqpoint{1.696413in}{2.314202in}}%
\pgfpathlineto{\pgfqpoint{1.765007in}{2.338235in}}%
\pgfpathlineto{\pgfqpoint{1.698589in}{2.404544in}}%
\pgfpathlineto{\pgfqpoint{1.629455in}{2.388721in}}%
\pgfpathclose%
\pgfusepath{fill}%
\end{pgfscope}%
\begin{pgfscope}%
\pgfpathrectangle{\pgfqpoint{0.150000in}{0.150000in}}{\pgfqpoint{4.700000in}{3.450000in}}%
\pgfusepath{clip}%
\pgfsetbuttcap%
\pgfsetroundjoin%
\definecolor{currentfill}{rgb}{0.972013,0.975460,0.980285}%
\pgfsetfillcolor{currentfill}%
\pgfsetlinewidth{0.000000pt}%
\definecolor{currentstroke}{rgb}{0.000000,0.000000,0.000000}%
\pgfsetstrokecolor{currentstroke}%
\pgfsetdash{}{0pt}%
\pgfpathmoveto{\pgfqpoint{2.430012in}{1.566458in}}%
\pgfpathlineto{\pgfqpoint{2.496933in}{1.588979in}}%
\pgfpathlineto{\pgfqpoint{2.430506in}{1.636810in}}%
\pgfpathlineto{\pgfqpoint{2.363494in}{1.614437in}}%
\pgfpathclose%
\pgfusepath{fill}%
\end{pgfscope}%
\begin{pgfscope}%
\pgfpathrectangle{\pgfqpoint{0.150000in}{0.150000in}}{\pgfqpoint{4.700000in}{3.450000in}}%
\pgfusepath{clip}%
\pgfsetbuttcap%
\pgfsetroundjoin%
\definecolor{currentfill}{rgb}{0.399831,0.473744,0.577221}%
\pgfsetfillcolor{currentfill}%
\pgfsetlinewidth{0.000000pt}%
\definecolor{currentstroke}{rgb}{0.000000,0.000000,0.000000}%
\pgfsetstrokecolor{currentstroke}%
\pgfsetdash{}{0pt}%
\pgfpathmoveto{\pgfqpoint{1.495500in}{2.536593in}}%
\pgfpathlineto{\pgfqpoint{1.565813in}{2.537102in}}%
\pgfpathlineto{\pgfqpoint{1.499454in}{2.603353in}}%
\pgfpathlineto{\pgfqpoint{1.428914in}{2.603080in}}%
\pgfpathclose%
\pgfusepath{fill}%
\end{pgfscope}%
\begin{pgfscope}%
\pgfpathrectangle{\pgfqpoint{0.150000in}{0.150000in}}{\pgfqpoint{4.700000in}{3.450000in}}%
\pgfusepath{clip}%
\pgfsetbuttcap%
\pgfsetroundjoin%
\definecolor{currentfill}{rgb}{0.480683,0.544638,0.634176}%
\pgfsetfillcolor{currentfill}%
\pgfsetlinewidth{0.000000pt}%
\definecolor{currentstroke}{rgb}{0.000000,0.000000,0.000000}%
\pgfsetstrokecolor{currentstroke}%
\pgfsetdash{}{0pt}%
\pgfpathmoveto{\pgfqpoint{1.629455in}{2.388721in}}%
\pgfpathlineto{\pgfqpoint{1.698589in}{2.404544in}}%
\pgfpathlineto{\pgfqpoint{1.632192in}{2.470833in}}%
\pgfpathlineto{\pgfqpoint{1.562460in}{2.463281in}}%
\pgfpathclose%
\pgfusepath{fill}%
\end{pgfscope}%
\begin{pgfscope}%
\pgfpathrectangle{\pgfqpoint{0.150000in}{0.150000in}}{\pgfqpoint{4.700000in}{3.450000in}}%
\pgfusepath{clip}%
\pgfsetbuttcap%
\pgfsetroundjoin%
\definecolor{currentfill}{rgb}{0.437148,0.506464,0.603508}%
\pgfsetfillcolor{currentfill}%
\pgfsetlinewidth{0.000000pt}%
\definecolor{currentstroke}{rgb}{0.000000,0.000000,0.000000}%
\pgfsetstrokecolor{currentstroke}%
\pgfsetdash{}{0pt}%
\pgfpathmoveto{\pgfqpoint{1.562460in}{2.463281in}}%
\pgfpathlineto{\pgfqpoint{1.632192in}{2.470833in}}%
\pgfpathlineto{\pgfqpoint{1.565813in}{2.537102in}}%
\pgfpathlineto{\pgfqpoint{1.495500in}{2.536593in}}%
\pgfpathclose%
\pgfusepath{fill}%
\end{pgfscope}%
\begin{pgfscope}%
\pgfpathrectangle{\pgfqpoint{0.150000in}{0.150000in}}{\pgfqpoint{4.700000in}{3.450000in}}%
\pgfusepath{clip}%
\pgfsetbuttcap%
\pgfsetroundjoin%
\definecolor{currentfill}{rgb}{0.884942,0.899112,0.918949}%
\pgfsetfillcolor{currentfill}%
\pgfsetlinewidth{0.000000pt}%
\definecolor{currentstroke}{rgb}{0.000000,0.000000,0.000000}%
\pgfsetstrokecolor{currentstroke}%
\pgfsetdash{}{0pt}%
\pgfpathmoveto{\pgfqpoint{2.163243in}{1.694274in}}%
\pgfpathlineto{\pgfqpoint{2.230751in}{1.719409in}}%
\pgfpathlineto{\pgfqpoint{2.164088in}{1.793599in}}%
\pgfpathlineto{\pgfqpoint{2.096408in}{1.768387in}}%
\pgfpathclose%
\pgfusepath{fill}%
\end{pgfscope}%
\begin{pgfscope}%
\pgfpathrectangle{\pgfqpoint{0.150000in}{0.150000in}}{\pgfqpoint{4.700000in}{3.450000in}}%
\pgfusepath{clip}%
\pgfsetbuttcap%
\pgfsetroundjoin%
\definecolor{currentfill}{rgb}{0.916039,0.926379,0.940855}%
\pgfsetfillcolor{currentfill}%
\pgfsetlinewidth{0.000000pt}%
\definecolor{currentstroke}{rgb}{0.000000,0.000000,0.000000}%
\pgfsetstrokecolor{currentstroke}%
\pgfsetdash{}{0pt}%
\pgfpathmoveto{\pgfqpoint{2.229683in}{1.639966in}}%
\pgfpathlineto{\pgfqpoint{2.297130in}{1.662305in}}%
\pgfpathlineto{\pgfqpoint{2.230751in}{1.719409in}}%
\pgfpathlineto{\pgfqpoint{2.163243in}{1.694274in}}%
\pgfpathclose%
\pgfusepath{fill}%
\end{pgfscope}%
\begin{pgfscope}%
\pgfpathrectangle{\pgfqpoint{0.150000in}{0.150000in}}{\pgfqpoint{4.700000in}{3.450000in}}%
\pgfusepath{clip}%
\pgfsetbuttcap%
\pgfsetroundjoin%
\definecolor{currentfill}{rgb}{0.925919,0.865579,0.870358}%
\pgfsetfillcolor{currentfill}%
\pgfsetlinewidth{0.000000pt}%
\definecolor{currentstroke}{rgb}{0.000000,0.000000,0.000000}%
\pgfsetstrokecolor{currentstroke}%
\pgfsetdash{}{0pt}%
\pgfpathmoveto{\pgfqpoint{3.097773in}{1.297822in}}%
\pgfpathlineto{\pgfqpoint{3.163023in}{1.321038in}}%
\pgfpathlineto{\pgfqpoint{3.095794in}{1.369202in}}%
\pgfpathlineto{\pgfqpoint{3.030447in}{1.346136in}}%
\pgfpathclose%
\pgfusepath{fill}%
\end{pgfscope}%
\begin{pgfscope}%
\pgfpathrectangle{\pgfqpoint{0.150000in}{0.150000in}}{\pgfqpoint{4.700000in}{3.450000in}}%
\pgfusepath{clip}%
\pgfsetbuttcap%
\pgfsetroundjoin%
\definecolor{currentfill}{rgb}{0.835187,0.855484,0.883900}%
\pgfsetfillcolor{currentfill}%
\pgfsetlinewidth{0.000000pt}%
\definecolor{currentstroke}{rgb}{0.000000,0.000000,0.000000}%
\pgfsetstrokecolor{currentstroke}%
\pgfsetdash{}{0pt}%
\pgfpathmoveto{\pgfqpoint{2.096408in}{1.768387in}}%
\pgfpathlineto{\pgfqpoint{2.164088in}{1.793599in}}%
\pgfpathlineto{\pgfqpoint{2.097389in}{1.867829in}}%
\pgfpathlineto{\pgfqpoint{2.029535in}{1.842663in}}%
\pgfpathclose%
\pgfusepath{fill}%
\end{pgfscope}%
\begin{pgfscope}%
\pgfpathrectangle{\pgfqpoint{0.150000in}{0.150000in}}{\pgfqpoint{4.700000in}{3.450000in}}%
\pgfusepath{clip}%
\pgfsetbuttcap%
\pgfsetroundjoin%
\definecolor{currentfill}{rgb}{0.952512,0.913833,0.916896}%
\pgfsetfillcolor{currentfill}%
\pgfsetlinewidth{0.000000pt}%
\definecolor{currentstroke}{rgb}{0.000000,0.000000,0.000000}%
\pgfsetstrokecolor{currentstroke}%
\pgfsetdash{}{0pt}%
\pgfpathmoveto{\pgfqpoint{2.897502in}{1.371303in}}%
\pgfpathlineto{\pgfqpoint{2.963278in}{1.394337in}}%
\pgfpathlineto{\pgfqpoint{2.896265in}{1.442427in}}%
\pgfpathlineto{\pgfqpoint{2.830394in}{1.419543in}}%
\pgfpathclose%
\pgfusepath{fill}%
\end{pgfscope}%
\begin{pgfscope}%
\pgfpathrectangle{\pgfqpoint{0.150000in}{0.150000in}}{\pgfqpoint{4.700000in}{3.450000in}}%
\pgfusepath{clip}%
\pgfsetbuttcap%
\pgfsetroundjoin%
\definecolor{currentfill}{rgb}{0.791651,0.817310,0.853232}%
\pgfsetfillcolor{currentfill}%
\pgfsetlinewidth{0.000000pt}%
\definecolor{currentstroke}{rgb}{0.000000,0.000000,0.000000}%
\pgfsetstrokecolor{currentstroke}%
\pgfsetdash{}{0pt}%
\pgfpathmoveto{\pgfqpoint{2.029535in}{1.842663in}}%
\pgfpathlineto{\pgfqpoint{2.097389in}{1.867829in}}%
\pgfpathlineto{\pgfqpoint{2.030652in}{1.942101in}}%
\pgfpathlineto{\pgfqpoint{1.962630in}{1.916855in}}%
\pgfpathclose%
\pgfusepath{fill}%
\end{pgfscope}%
\begin{pgfscope}%
\pgfpathrectangle{\pgfqpoint{0.150000in}{0.150000in}}{\pgfqpoint{4.700000in}{3.450000in}}%
\pgfusepath{clip}%
\pgfsetbuttcap%
\pgfsetroundjoin%
\definecolor{currentfill}{rgb}{0.982904,0.968980,0.970083}%
\pgfsetfillcolor{currentfill}%
\pgfsetlinewidth{0.000000pt}%
\definecolor{currentstroke}{rgb}{0.000000,0.000000,0.000000}%
\pgfsetstrokecolor{currentstroke}%
\pgfsetdash{}{0pt}%
\pgfpathmoveto{\pgfqpoint{2.697140in}{1.444818in}}%
\pgfpathlineto{\pgfqpoint{2.763442in}{1.467670in}}%
\pgfpathlineto{\pgfqpoint{2.696645in}{1.515686in}}%
\pgfpathlineto{\pgfqpoint{2.630249in}{1.492984in}}%
\pgfpathclose%
\pgfusepath{fill}%
\end{pgfscope}%
\begin{pgfscope}%
\pgfpathrectangle{\pgfqpoint{0.150000in}{0.150000in}}{\pgfqpoint{4.700000in}{3.450000in}}%
\pgfusepath{clip}%
\pgfsetbuttcap%
\pgfsetroundjoin%
\definecolor{currentfill}{rgb}{0.741896,0.773683,0.818183}%
\pgfsetfillcolor{currentfill}%
\pgfsetlinewidth{0.000000pt}%
\definecolor{currentstroke}{rgb}{0.000000,0.000000,0.000000}%
\pgfsetstrokecolor{currentstroke}%
\pgfsetdash{}{0pt}%
\pgfpathmoveto{\pgfqpoint{1.962630in}{1.916855in}}%
\pgfpathlineto{\pgfqpoint{2.030652in}{1.942101in}}%
\pgfpathlineto{\pgfqpoint{1.963879in}{2.016414in}}%
\pgfpathlineto{\pgfqpoint{1.895685in}{1.991210in}}%
\pgfpathclose%
\pgfusepath{fill}%
\end{pgfscope}%
\begin{pgfscope}%
\pgfpathrectangle{\pgfqpoint{0.150000in}{0.150000in}}{\pgfqpoint{4.700000in}{3.450000in}}%
\pgfusepath{clip}%
\pgfsetbuttcap%
\pgfsetroundjoin%
\definecolor{currentfill}{rgb}{0.984452,0.986366,0.989047}%
\pgfsetfillcolor{currentfill}%
\pgfsetlinewidth{0.000000pt}%
\definecolor{currentstroke}{rgb}{0.000000,0.000000,0.000000}%
\pgfsetstrokecolor{currentstroke}%
\pgfsetdash{}{0pt}%
\pgfpathmoveto{\pgfqpoint{2.496685in}{1.518367in}}%
\pgfpathlineto{\pgfqpoint{2.563514in}{1.541037in}}%
\pgfpathlineto{\pgfqpoint{2.496933in}{1.588979in}}%
\pgfpathlineto{\pgfqpoint{2.430012in}{1.566458in}}%
\pgfpathclose%
\pgfusepath{fill}%
\end{pgfscope}%
\begin{pgfscope}%
\pgfpathrectangle{\pgfqpoint{0.150000in}{0.150000in}}{\pgfqpoint{4.700000in}{3.450000in}}%
\pgfusepath{clip}%
\pgfsetbuttcap%
\pgfsetroundjoin%
\definecolor{currentfill}{rgb}{0.698361,0.735509,0.787515}%
\pgfsetfillcolor{currentfill}%
\pgfsetlinewidth{0.000000pt}%
\definecolor{currentstroke}{rgb}{0.000000,0.000000,0.000000}%
\pgfsetstrokecolor{currentstroke}%
\pgfsetdash{}{0pt}%
\pgfpathmoveto{\pgfqpoint{1.895685in}{1.991210in}}%
\pgfpathlineto{\pgfqpoint{1.963879in}{2.016414in}}%
\pgfpathlineto{\pgfqpoint{1.897070in}{2.090768in}}%
\pgfpathlineto{\pgfqpoint{1.828709in}{2.065481in}}%
\pgfpathclose%
\pgfusepath{fill}%
\end{pgfscope}%
\begin{pgfscope}%
\pgfpathrectangle{\pgfqpoint{0.150000in}{0.150000in}}{\pgfqpoint{4.700000in}{3.450000in}}%
\pgfusepath{clip}%
\pgfsetbuttcap%
\pgfsetroundjoin%
\definecolor{currentfill}{rgb}{0.940916,0.948192,0.958379}%
\pgfsetfillcolor{currentfill}%
\pgfsetlinewidth{0.000000pt}%
\definecolor{currentstroke}{rgb}{0.000000,0.000000,0.000000}%
\pgfsetstrokecolor{currentstroke}%
\pgfsetdash{}{0pt}%
\pgfpathmoveto{\pgfqpoint{2.296138in}{1.591950in}}%
\pgfpathlineto{\pgfqpoint{2.363494in}{1.614437in}}%
\pgfpathlineto{\pgfqpoint{2.297130in}{1.662305in}}%
\pgfpathlineto{\pgfqpoint{2.229683in}{1.639966in}}%
\pgfpathclose%
\pgfusepath{fill}%
\end{pgfscope}%
\begin{pgfscope}%
\pgfpathrectangle{\pgfqpoint{0.150000in}{0.150000in}}{\pgfqpoint{4.700000in}{3.450000in}}%
\pgfusepath{clip}%
\pgfsetbuttcap%
\pgfsetroundjoin%
\definecolor{currentfill}{rgb}{0.648606,0.691881,0.752466}%
\pgfsetfillcolor{currentfill}%
\pgfsetlinewidth{0.000000pt}%
\definecolor{currentstroke}{rgb}{0.000000,0.000000,0.000000}%
\pgfsetstrokecolor{currentstroke}%
\pgfsetdash{}{0pt}%
\pgfpathmoveto{\pgfqpoint{1.828709in}{2.065481in}}%
\pgfpathlineto{\pgfqpoint{1.897070in}{2.090768in}}%
\pgfpathlineto{\pgfqpoint{1.830218in}{2.165288in}}%
\pgfpathlineto{\pgfqpoint{1.761697in}{2.139790in}}%
\pgfpathclose%
\pgfusepath{fill}%
\end{pgfscope}%
\begin{pgfscope}%
\pgfpathrectangle{\pgfqpoint{0.150000in}{0.150000in}}{\pgfqpoint{4.700000in}{3.450000in}}%
\pgfusepath{clip}%
\pgfsetbuttcap%
\pgfsetroundjoin%
\definecolor{currentfill}{rgb}{0.914522,0.844899,0.850414}%
\pgfsetfillcolor{currentfill}%
\pgfsetlinewidth{0.000000pt}%
\definecolor{currentstroke}{rgb}{0.000000,0.000000,0.000000}%
\pgfsetstrokecolor{currentstroke}%
\pgfsetdash{}{0pt}%
\pgfpathmoveto{\pgfqpoint{3.165255in}{1.249396in}}%
\pgfpathlineto{\pgfqpoint{3.230409in}{1.272762in}}%
\pgfpathlineto{\pgfqpoint{3.163023in}{1.321038in}}%
\pgfpathlineto{\pgfqpoint{3.097773in}{1.297822in}}%
\pgfpathclose%
\pgfusepath{fill}%
\end{pgfscope}%
\begin{pgfscope}%
\pgfpathrectangle{\pgfqpoint{0.150000in}{0.150000in}}{\pgfqpoint{4.700000in}{3.450000in}}%
\pgfusepath{clip}%
\pgfsetbuttcap%
\pgfsetroundjoin%
\definecolor{currentfill}{rgb}{0.605070,0.653707,0.721798}%
\pgfsetfillcolor{currentfill}%
\pgfsetlinewidth{0.000000pt}%
\definecolor{currentstroke}{rgb}{0.000000,0.000000,0.000000}%
\pgfsetstrokecolor{currentstroke}%
\pgfsetdash{}{0pt}%
\pgfpathmoveto{\pgfqpoint{1.761697in}{2.139790in}}%
\pgfpathlineto{\pgfqpoint{1.830218in}{2.165288in}}%
\pgfpathlineto{\pgfqpoint{1.763334in}{2.239725in}}%
\pgfpathlineto{\pgfqpoint{1.694645in}{2.214265in}}%
\pgfpathclose%
\pgfusepath{fill}%
\end{pgfscope}%
\begin{pgfscope}%
\pgfpathrectangle{\pgfqpoint{0.150000in}{0.150000in}}{\pgfqpoint{4.700000in}{3.450000in}}%
\pgfusepath{clip}%
\pgfsetbuttcap%
\pgfsetroundjoin%
\definecolor{currentfill}{rgb}{0.944914,0.900046,0.903600}%
\pgfsetfillcolor{currentfill}%
\pgfsetlinewidth{0.000000pt}%
\definecolor{currentstroke}{rgb}{0.000000,0.000000,0.000000}%
\pgfsetstrokecolor{currentstroke}%
\pgfsetdash{}{0pt}%
\pgfpathmoveto{\pgfqpoint{2.964767in}{1.322951in}}%
\pgfpathlineto{\pgfqpoint{3.030447in}{1.346136in}}%
\pgfpathlineto{\pgfqpoint{2.963278in}{1.394337in}}%
\pgfpathlineto{\pgfqpoint{2.897502in}{1.371303in}}%
\pgfpathclose%
\pgfusepath{fill}%
\end{pgfscope}%
\begin{pgfscope}%
\pgfpathrectangle{\pgfqpoint{0.150000in}{0.150000in}}{\pgfqpoint{4.700000in}{3.450000in}}%
\pgfusepath{clip}%
\pgfsetbuttcap%
\pgfsetroundjoin%
\definecolor{currentfill}{rgb}{0.971507,0.948300,0.950138}%
\pgfsetfillcolor{currentfill}%
\pgfsetlinewidth{0.000000pt}%
\definecolor{currentstroke}{rgb}{0.000000,0.000000,0.000000}%
\pgfsetstrokecolor{currentstroke}%
\pgfsetdash{}{0pt}%
\pgfpathmoveto{\pgfqpoint{2.764186in}{1.396541in}}%
\pgfpathlineto{\pgfqpoint{2.830394in}{1.419543in}}%
\pgfpathlineto{\pgfqpoint{2.763442in}{1.467670in}}%
\pgfpathlineto{\pgfqpoint{2.697140in}{1.444818in}}%
\pgfpathclose%
\pgfusepath{fill}%
\end{pgfscope}%
\begin{pgfscope}%
\pgfpathrectangle{\pgfqpoint{0.150000in}{0.150000in}}{\pgfqpoint{4.700000in}{3.450000in}}%
\pgfusepath{clip}%
\pgfsetbuttcap%
\pgfsetroundjoin%
\definecolor{currentfill}{rgb}{0.555316,0.610080,0.686749}%
\pgfsetfillcolor{currentfill}%
\pgfsetlinewidth{0.000000pt}%
\definecolor{currentstroke}{rgb}{0.000000,0.000000,0.000000}%
\pgfsetstrokecolor{currentstroke}%
\pgfsetdash{}{0pt}%
\pgfpathmoveto{\pgfqpoint{1.694645in}{2.214265in}}%
\pgfpathlineto{\pgfqpoint{1.763334in}{2.239725in}}%
\pgfpathlineto{\pgfqpoint{1.696413in}{2.314202in}}%
\pgfpathlineto{\pgfqpoint{1.627563in}{2.288653in}}%
\pgfpathclose%
\pgfusepath{fill}%
\end{pgfscope}%
\begin{pgfscope}%
\pgfpathrectangle{\pgfqpoint{0.150000in}{0.150000in}}{\pgfqpoint{4.700000in}{3.450000in}}%
\pgfusepath{clip}%
\pgfsetbuttcap%
\pgfsetroundjoin%
\definecolor{currentfill}{rgb}{0.996890,0.997273,0.997809}%
\pgfsetfillcolor{currentfill}%
\pgfsetlinewidth{0.000000pt}%
\definecolor{currentstroke}{rgb}{0.000000,0.000000,0.000000}%
\pgfsetstrokecolor{currentstroke}%
\pgfsetdash{}{0pt}%
\pgfpathmoveto{\pgfqpoint{2.563514in}{1.470165in}}%
\pgfpathlineto{\pgfqpoint{2.630249in}{1.492984in}}%
\pgfpathlineto{\pgfqpoint{2.563514in}{1.541037in}}%
\pgfpathlineto{\pgfqpoint{2.496685in}{1.518367in}}%
\pgfpathclose%
\pgfusepath{fill}%
\end{pgfscope}%
\begin{pgfscope}%
\pgfpathrectangle{\pgfqpoint{0.150000in}{0.150000in}}{\pgfqpoint{4.700000in}{3.450000in}}%
\pgfusepath{clip}%
\pgfsetbuttcap%
\pgfsetroundjoin%
\definecolor{currentfill}{rgb}{0.505561,0.566452,0.651700}%
\pgfsetfillcolor{currentfill}%
\pgfsetlinewidth{0.000000pt}%
\definecolor{currentstroke}{rgb}{0.000000,0.000000,0.000000}%
\pgfsetstrokecolor{currentstroke}%
\pgfsetdash{}{0pt}%
\pgfpathmoveto{\pgfqpoint{1.627563in}{2.288653in}}%
\pgfpathlineto{\pgfqpoint{1.696413in}{2.314202in}}%
\pgfpathlineto{\pgfqpoint{1.629455in}{2.388721in}}%
\pgfpathlineto{\pgfqpoint{1.560438in}{2.363209in}}%
\pgfpathclose%
\pgfusepath{fill}%
\end{pgfscope}%
\begin{pgfscope}%
\pgfpathrectangle{\pgfqpoint{0.150000in}{0.150000in}}{\pgfqpoint{4.700000in}{3.450000in}}%
\pgfusepath{clip}%
\pgfsetbuttcap%
\pgfsetroundjoin%
\definecolor{currentfill}{rgb}{0.953355,0.959099,0.967142}%
\pgfsetfillcolor{currentfill}%
\pgfsetlinewidth{0.000000pt}%
\definecolor{currentstroke}{rgb}{0.000000,0.000000,0.000000}%
\pgfsetstrokecolor{currentstroke}%
\pgfsetdash{}{0pt}%
\pgfpathmoveto{\pgfqpoint{2.362749in}{1.543822in}}%
\pgfpathlineto{\pgfqpoint{2.430012in}{1.566458in}}%
\pgfpathlineto{\pgfqpoint{2.363494in}{1.614437in}}%
\pgfpathlineto{\pgfqpoint{2.296138in}{1.591950in}}%
\pgfpathclose%
\pgfusepath{fill}%
\end{pgfscope}%
\begin{pgfscope}%
\pgfpathrectangle{\pgfqpoint{0.150000in}{0.150000in}}{\pgfqpoint{4.700000in}{3.450000in}}%
\pgfusepath{clip}%
\pgfsetbuttcap%
\pgfsetroundjoin%
\definecolor{currentfill}{rgb}{0.872503,0.888205,0.910187}%
\pgfsetfillcolor{currentfill}%
\pgfsetlinewidth{0.000000pt}%
\definecolor{currentstroke}{rgb}{0.000000,0.000000,0.000000}%
\pgfsetstrokecolor{currentstroke}%
\pgfsetdash{}{0pt}%
\pgfpathmoveto{\pgfqpoint{2.095412in}{1.669018in}}%
\pgfpathlineto{\pgfqpoint{2.163243in}{1.694274in}}%
\pgfpathlineto{\pgfqpoint{2.096408in}{1.768387in}}%
\pgfpathlineto{\pgfqpoint{2.028403in}{1.743179in}}%
\pgfpathclose%
\pgfusepath{fill}%
\end{pgfscope}%
\begin{pgfscope}%
\pgfpathrectangle{\pgfqpoint{0.150000in}{0.150000in}}{\pgfqpoint{4.700000in}{3.450000in}}%
\pgfusepath{clip}%
\pgfsetbuttcap%
\pgfsetroundjoin%
\definecolor{currentfill}{rgb}{0.462025,0.528278,0.621032}%
\pgfsetfillcolor{currentfill}%
\pgfsetlinewidth{0.000000pt}%
\definecolor{currentstroke}{rgb}{0.000000,0.000000,0.000000}%
\pgfsetstrokecolor{currentstroke}%
\pgfsetdash{}{0pt}%
\pgfpathmoveto{\pgfqpoint{1.560438in}{2.363209in}}%
\pgfpathlineto{\pgfqpoint{1.629455in}{2.388721in}}%
\pgfpathlineto{\pgfqpoint{1.562460in}{2.463281in}}%
\pgfpathlineto{\pgfqpoint{1.493284in}{2.437676in}}%
\pgfpathclose%
\pgfusepath{fill}%
\end{pgfscope}%
\begin{pgfscope}%
\pgfpathrectangle{\pgfqpoint{0.150000in}{0.150000in}}{\pgfqpoint{4.700000in}{3.450000in}}%
\pgfusepath{clip}%
\pgfsetbuttcap%
\pgfsetroundjoin%
\definecolor{currentfill}{rgb}{0.822748,0.844577,0.875138}%
\pgfsetfillcolor{currentfill}%
\pgfsetlinewidth{0.000000pt}%
\definecolor{currentstroke}{rgb}{0.000000,0.000000,0.000000}%
\pgfsetstrokecolor{currentstroke}%
\pgfsetdash{}{0pt}%
\pgfpathmoveto{\pgfqpoint{2.028403in}{1.743179in}}%
\pgfpathlineto{\pgfqpoint{2.096408in}{1.768387in}}%
\pgfpathlineto{\pgfqpoint{2.029535in}{1.842663in}}%
\pgfpathlineto{\pgfqpoint{1.961363in}{1.817253in}}%
\pgfpathclose%
\pgfusepath{fill}%
\end{pgfscope}%
\begin{pgfscope}%
\pgfpathrectangle{\pgfqpoint{0.150000in}{0.150000in}}{\pgfqpoint{4.700000in}{3.450000in}}%
\pgfusepath{clip}%
\pgfsetbuttcap%
\pgfsetroundjoin%
\definecolor{currentfill}{rgb}{0.412270,0.484651,0.585983}%
\pgfsetfillcolor{currentfill}%
\pgfsetlinewidth{0.000000pt}%
\definecolor{currentstroke}{rgb}{0.000000,0.000000,0.000000}%
\pgfsetstrokecolor{currentstroke}%
\pgfsetdash{}{0pt}%
\pgfpathmoveto{\pgfqpoint{1.493284in}{2.437676in}}%
\pgfpathlineto{\pgfqpoint{1.562460in}{2.463281in}}%
\pgfpathlineto{\pgfqpoint{1.495500in}{2.536593in}}%
\pgfpathlineto{\pgfqpoint{1.426088in}{2.512311in}}%
\pgfpathclose%
\pgfusepath{fill}%
\end{pgfscope}%
\begin{pgfscope}%
\pgfpathrectangle{\pgfqpoint{0.150000in}{0.150000in}}{\pgfqpoint{4.700000in}{3.450000in}}%
\pgfusepath{clip}%
\pgfsetbuttcap%
\pgfsetroundjoin%
\definecolor{currentfill}{rgb}{0.903600,0.915472,0.932093}%
\pgfsetfillcolor{currentfill}%
\pgfsetlinewidth{0.000000pt}%
\definecolor{currentstroke}{rgb}{0.000000,0.000000,0.000000}%
\pgfsetstrokecolor{currentstroke}%
\pgfsetdash{}{0pt}%
\pgfpathmoveto{\pgfqpoint{2.161891in}{1.617513in}}%
\pgfpathlineto{\pgfqpoint{2.229683in}{1.639966in}}%
\pgfpathlineto{\pgfqpoint{2.163243in}{1.694274in}}%
\pgfpathlineto{\pgfqpoint{2.095412in}{1.669018in}}%
\pgfpathclose%
\pgfusepath{fill}%
\end{pgfscope}%
\begin{pgfscope}%
\pgfpathrectangle{\pgfqpoint{0.150000in}{0.150000in}}{\pgfqpoint{4.700000in}{3.450000in}}%
\pgfusepath{clip}%
\pgfsetbuttcap%
\pgfsetroundjoin%
\definecolor{currentfill}{rgb}{0.933517,0.879366,0.883655}%
\pgfsetfillcolor{currentfill}%
\pgfsetlinewidth{0.000000pt}%
\definecolor{currentstroke}{rgb}{0.000000,0.000000,0.000000}%
\pgfsetstrokecolor{currentstroke}%
\pgfsetdash{}{0pt}%
\pgfpathmoveto{\pgfqpoint{3.032188in}{1.274487in}}%
\pgfpathlineto{\pgfqpoint{3.097773in}{1.297822in}}%
\pgfpathlineto{\pgfqpoint{3.030447in}{1.346136in}}%
\pgfpathlineto{\pgfqpoint{2.964767in}{1.322951in}}%
\pgfpathclose%
\pgfusepath{fill}%
\end{pgfscope}%
\begin{pgfscope}%
\pgfpathrectangle{\pgfqpoint{0.150000in}{0.150000in}}{\pgfqpoint{4.700000in}{3.450000in}}%
\pgfusepath{clip}%
\pgfsetbuttcap%
\pgfsetroundjoin%
\definecolor{currentfill}{rgb}{0.368735,0.446477,0.555316}%
\pgfsetfillcolor{currentfill}%
\pgfsetlinewidth{0.000000pt}%
\definecolor{currentstroke}{rgb}{0.000000,0.000000,0.000000}%
\pgfsetstrokecolor{currentstroke}%
\pgfsetdash{}{0pt}%
\pgfpathmoveto{\pgfqpoint{1.426088in}{2.512311in}}%
\pgfpathlineto{\pgfqpoint{1.495500in}{2.536593in}}%
\pgfpathlineto{\pgfqpoint{1.428914in}{2.603080in}}%
\pgfpathlineto{\pgfqpoint{1.358863in}{2.586857in}}%
\pgfpathclose%
\pgfusepath{fill}%
\end{pgfscope}%
\begin{pgfscope}%
\pgfpathrectangle{\pgfqpoint{0.150000in}{0.150000in}}{\pgfqpoint{4.700000in}{3.450000in}}%
\pgfusepath{clip}%
\pgfsetbuttcap%
\pgfsetroundjoin%
\definecolor{currentfill}{rgb}{0.779213,0.806403,0.844470}%
\pgfsetfillcolor{currentfill}%
\pgfsetlinewidth{0.000000pt}%
\definecolor{currentstroke}{rgb}{0.000000,0.000000,0.000000}%
\pgfsetstrokecolor{currentstroke}%
\pgfsetdash{}{0pt}%
\pgfpathmoveto{\pgfqpoint{1.961363in}{1.817253in}}%
\pgfpathlineto{\pgfqpoint{2.029535in}{1.842663in}}%
\pgfpathlineto{\pgfqpoint{1.962630in}{1.916855in}}%
\pgfpathlineto{\pgfqpoint{1.894286in}{1.891489in}}%
\pgfpathclose%
\pgfusepath{fill}%
\end{pgfscope}%
\begin{pgfscope}%
\pgfpathrectangle{\pgfqpoint{0.150000in}{0.150000in}}{\pgfqpoint{4.700000in}{3.450000in}}%
\pgfusepath{clip}%
\pgfsetbuttcap%
\pgfsetroundjoin%
\definecolor{currentfill}{rgb}{0.331419,0.413756,0.529029}%
\pgfsetfillcolor{currentfill}%
\pgfsetlinewidth{0.000000pt}%
\definecolor{currentstroke}{rgb}{0.000000,0.000000,0.000000}%
\pgfsetstrokecolor{currentstroke}%
\pgfsetdash{}{0pt}%
\pgfpathmoveto{\pgfqpoint{1.358863in}{2.586857in}}%
\pgfpathlineto{\pgfqpoint{1.428914in}{2.603080in}}%
\pgfpathlineto{\pgfqpoint{1.362347in}{2.669548in}}%
\pgfpathlineto{\pgfqpoint{1.291594in}{2.661572in}}%
\pgfpathclose%
\pgfusepath{fill}%
\end{pgfscope}%
\begin{pgfscope}%
\pgfpathrectangle{\pgfqpoint{0.150000in}{0.150000in}}{\pgfqpoint{4.700000in}{3.450000in}}%
\pgfusepath{clip}%
\pgfsetbuttcap%
\pgfsetroundjoin%
\definecolor{currentfill}{rgb}{0.287883,0.375582,0.498361}%
\pgfsetfillcolor{currentfill}%
\pgfsetlinewidth{0.000000pt}%
\definecolor{currentstroke}{rgb}{0.000000,0.000000,0.000000}%
\pgfsetstrokecolor{currentstroke}%
\pgfsetdash{}{0pt}%
\pgfpathmoveto{\pgfqpoint{1.291594in}{2.661572in}}%
\pgfpathlineto{\pgfqpoint{1.362347in}{2.669548in}}%
\pgfpathlineto{\pgfqpoint{1.295799in}{2.735997in}}%
\pgfpathlineto{\pgfqpoint{1.224298in}{2.736198in}}%
\pgfpathclose%
\pgfusepath{fill}%
\end{pgfscope}%
\begin{pgfscope}%
\pgfpathrectangle{\pgfqpoint{0.150000in}{0.150000in}}{\pgfqpoint{4.700000in}{3.450000in}}%
\pgfusepath{clip}%
\pgfsetbuttcap%
\pgfsetroundjoin%
\definecolor{currentfill}{rgb}{0.963909,0.934513,0.936841}%
\pgfsetfillcolor{currentfill}%
\pgfsetlinewidth{0.000000pt}%
\definecolor{currentstroke}{rgb}{0.000000,0.000000,0.000000}%
\pgfsetstrokecolor{currentstroke}%
\pgfsetdash{}{0pt}%
\pgfpathmoveto{\pgfqpoint{2.831389in}{1.348151in}}%
\pgfpathlineto{\pgfqpoint{2.897502in}{1.371303in}}%
\pgfpathlineto{\pgfqpoint{2.830394in}{1.419543in}}%
\pgfpathlineto{\pgfqpoint{2.764186in}{1.396541in}}%
\pgfpathclose%
\pgfusepath{fill}%
\end{pgfscope}%
\begin{pgfscope}%
\pgfpathrectangle{\pgfqpoint{0.150000in}{0.150000in}}{\pgfqpoint{4.700000in}{3.450000in}}%
\pgfusepath{clip}%
\pgfsetbuttcap%
\pgfsetroundjoin%
\definecolor{currentfill}{rgb}{0.729458,0.762776,0.809421}%
\pgfsetfillcolor{currentfill}%
\pgfsetlinewidth{0.000000pt}%
\definecolor{currentstroke}{rgb}{0.000000,0.000000,0.000000}%
\pgfsetstrokecolor{currentstroke}%
\pgfsetdash{}{0pt}%
\pgfpathmoveto{\pgfqpoint{1.894286in}{1.891489in}}%
\pgfpathlineto{\pgfqpoint{1.962630in}{1.916855in}}%
\pgfpathlineto{\pgfqpoint{1.895685in}{1.991210in}}%
\pgfpathlineto{\pgfqpoint{1.827174in}{1.965763in}}%
\pgfpathclose%
\pgfusepath{fill}%
\end{pgfscope}%
\begin{pgfscope}%
\pgfpathrectangle{\pgfqpoint{0.150000in}{0.150000in}}{\pgfqpoint{4.700000in}{3.450000in}}%
\pgfusepath{clip}%
\pgfsetbuttcap%
\pgfsetroundjoin%
\definecolor{currentfill}{rgb}{0.990502,0.982767,0.983379}%
\pgfsetfillcolor{currentfill}%
\pgfsetlinewidth{0.000000pt}%
\definecolor{currentstroke}{rgb}{0.000000,0.000000,0.000000}%
\pgfsetstrokecolor{currentstroke}%
\pgfsetdash{}{0pt}%
\pgfpathmoveto{\pgfqpoint{2.630498in}{1.421849in}}%
\pgfpathlineto{\pgfqpoint{2.697140in}{1.444818in}}%
\pgfpathlineto{\pgfqpoint{2.630249in}{1.492984in}}%
\pgfpathlineto{\pgfqpoint{2.563514in}{1.470165in}}%
\pgfpathclose%
\pgfusepath{fill}%
\end{pgfscope}%
\begin{pgfscope}%
\pgfpathrectangle{\pgfqpoint{0.150000in}{0.150000in}}{\pgfqpoint{4.700000in}{3.450000in}}%
\pgfusepath{clip}%
\pgfsetbuttcap%
\pgfsetroundjoin%
\definecolor{currentfill}{rgb}{0.685922,0.724602,0.778753}%
\pgfsetfillcolor{currentfill}%
\pgfsetlinewidth{0.000000pt}%
\definecolor{currentstroke}{rgb}{0.000000,0.000000,0.000000}%
\pgfsetstrokecolor{currentstroke}%
\pgfsetdash{}{0pt}%
\pgfpathmoveto{\pgfqpoint{1.827174in}{1.965763in}}%
\pgfpathlineto{\pgfqpoint{1.895685in}{1.991210in}}%
\pgfpathlineto{\pgfqpoint{1.828709in}{2.065481in}}%
\pgfpathlineto{\pgfqpoint{1.760027in}{2.040075in}}%
\pgfpathclose%
\pgfusepath{fill}%
\end{pgfscope}%
\begin{pgfscope}%
\pgfpathrectangle{\pgfqpoint{0.150000in}{0.150000in}}{\pgfqpoint{4.700000in}{3.450000in}}%
\pgfusepath{clip}%
\pgfsetbuttcap%
\pgfsetroundjoin%
\definecolor{currentfill}{rgb}{0.972013,0.975460,0.980285}%
\pgfsetfillcolor{currentfill}%
\pgfsetlinewidth{0.000000pt}%
\definecolor{currentstroke}{rgb}{0.000000,0.000000,0.000000}%
\pgfsetstrokecolor{currentstroke}%
\pgfsetdash{}{0pt}%
\pgfpathmoveto{\pgfqpoint{2.429514in}{1.495582in}}%
\pgfpathlineto{\pgfqpoint{2.496685in}{1.518367in}}%
\pgfpathlineto{\pgfqpoint{2.430012in}{1.566458in}}%
\pgfpathlineto{\pgfqpoint{2.362749in}{1.543822in}}%
\pgfpathclose%
\pgfusepath{fill}%
\end{pgfscope}%
\begin{pgfscope}%
\pgfpathrectangle{\pgfqpoint{0.150000in}{0.150000in}}{\pgfqpoint{4.700000in}{3.450000in}}%
\pgfusepath{clip}%
\pgfsetbuttcap%
\pgfsetroundjoin%
\definecolor{currentfill}{rgb}{0.636167,0.680974,0.743704}%
\pgfsetfillcolor{currentfill}%
\pgfsetlinewidth{0.000000pt}%
\definecolor{currentstroke}{rgb}{0.000000,0.000000,0.000000}%
\pgfsetstrokecolor{currentstroke}%
\pgfsetdash{}{0pt}%
\pgfpathmoveto{\pgfqpoint{1.760027in}{2.040075in}}%
\pgfpathlineto{\pgfqpoint{1.828709in}{2.065481in}}%
\pgfpathlineto{\pgfqpoint{1.761697in}{2.139790in}}%
\pgfpathlineto{\pgfqpoint{1.692852in}{2.114299in}}%
\pgfpathclose%
\pgfusepath{fill}%
\end{pgfscope}%
\begin{pgfscope}%
\pgfpathrectangle{\pgfqpoint{0.150000in}{0.150000in}}{\pgfqpoint{4.700000in}{3.450000in}}%
\pgfusepath{clip}%
\pgfsetbuttcap%
\pgfsetroundjoin%
\definecolor{currentfill}{rgb}{0.922258,0.931832,0.945236}%
\pgfsetfillcolor{currentfill}%
\pgfsetlinewidth{0.000000pt}%
\definecolor{currentstroke}{rgb}{0.000000,0.000000,0.000000}%
\pgfsetstrokecolor{currentstroke}%
\pgfsetdash{}{0pt}%
\pgfpathmoveto{\pgfqpoint{2.228438in}{1.569348in}}%
\pgfpathlineto{\pgfqpoint{2.296138in}{1.591950in}}%
\pgfpathlineto{\pgfqpoint{2.229683in}{1.639966in}}%
\pgfpathlineto{\pgfqpoint{2.161891in}{1.617513in}}%
\pgfpathclose%
\pgfusepath{fill}%
\end{pgfscope}%
\begin{pgfscope}%
\pgfpathrectangle{\pgfqpoint{0.150000in}{0.150000in}}{\pgfqpoint{4.700000in}{3.450000in}}%
\pgfusepath{clip}%
\pgfsetbuttcap%
\pgfsetroundjoin%
\definecolor{currentfill}{rgb}{0.592632,0.642800,0.713036}%
\pgfsetfillcolor{currentfill}%
\pgfsetlinewidth{0.000000pt}%
\definecolor{currentstroke}{rgb}{0.000000,0.000000,0.000000}%
\pgfsetstrokecolor{currentstroke}%
\pgfsetdash{}{0pt}%
\pgfpathmoveto{\pgfqpoint{1.692852in}{2.114299in}}%
\pgfpathlineto{\pgfqpoint{1.761697in}{2.139790in}}%
\pgfpathlineto{\pgfqpoint{1.694645in}{2.214265in}}%
\pgfpathlineto{\pgfqpoint{1.625637in}{2.188687in}}%
\pgfpathclose%
\pgfusepath{fill}%
\end{pgfscope}%
\begin{pgfscope}%
\pgfpathrectangle{\pgfqpoint{0.150000in}{0.150000in}}{\pgfqpoint{4.700000in}{3.450000in}}%
\pgfusepath{clip}%
\pgfsetbuttcap%
\pgfsetroundjoin%
\definecolor{currentfill}{rgb}{0.925919,0.865579,0.870358}%
\pgfsetfillcolor{currentfill}%
\pgfsetlinewidth{0.000000pt}%
\definecolor{currentstroke}{rgb}{0.000000,0.000000,0.000000}%
\pgfsetstrokecolor{currentstroke}%
\pgfsetdash{}{0pt}%
\pgfpathmoveto{\pgfqpoint{3.099766in}{1.225909in}}%
\pgfpathlineto{\pgfqpoint{3.165255in}{1.249396in}}%
\pgfpathlineto{\pgfqpoint{3.097773in}{1.297822in}}%
\pgfpathlineto{\pgfqpoint{3.032188in}{1.274487in}}%
\pgfpathclose%
\pgfusepath{fill}%
\end{pgfscope}%
\begin{pgfscope}%
\pgfpathrectangle{\pgfqpoint{0.150000in}{0.150000in}}{\pgfqpoint{4.700000in}{3.450000in}}%
\pgfusepath{clip}%
\pgfsetbuttcap%
\pgfsetroundjoin%
\definecolor{currentfill}{rgb}{0.952512,0.913833,0.916896}%
\pgfsetfillcolor{currentfill}%
\pgfsetlinewidth{0.000000pt}%
\definecolor{currentstroke}{rgb}{0.000000,0.000000,0.000000}%
\pgfsetstrokecolor{currentstroke}%
\pgfsetdash{}{0pt}%
\pgfpathmoveto{\pgfqpoint{2.898749in}{1.299648in}}%
\pgfpathlineto{\pgfqpoint{2.964767in}{1.322951in}}%
\pgfpathlineto{\pgfqpoint{2.897502in}{1.371303in}}%
\pgfpathlineto{\pgfqpoint{2.831389in}{1.348151in}}%
\pgfpathclose%
\pgfusepath{fill}%
\end{pgfscope}%
\begin{pgfscope}%
\pgfpathrectangle{\pgfqpoint{0.150000in}{0.150000in}}{\pgfqpoint{4.700000in}{3.450000in}}%
\pgfusepath{clip}%
\pgfsetbuttcap%
\pgfsetroundjoin%
\definecolor{currentfill}{rgb}{0.542877,0.599173,0.677987}%
\pgfsetfillcolor{currentfill}%
\pgfsetlinewidth{0.000000pt}%
\definecolor{currentstroke}{rgb}{0.000000,0.000000,0.000000}%
\pgfsetstrokecolor{currentstroke}%
\pgfsetdash{}{0pt}%
\pgfpathmoveto{\pgfqpoint{1.625637in}{2.188687in}}%
\pgfpathlineto{\pgfqpoint{1.694645in}{2.214265in}}%
\pgfpathlineto{\pgfqpoint{1.627563in}{2.288653in}}%
\pgfpathlineto{\pgfqpoint{1.558387in}{2.263114in}}%
\pgfpathclose%
\pgfusepath{fill}%
\end{pgfscope}%
\begin{pgfscope}%
\pgfpathrectangle{\pgfqpoint{0.150000in}{0.150000in}}{\pgfqpoint{4.700000in}{3.450000in}}%
\pgfusepath{clip}%
\pgfsetbuttcap%
\pgfsetroundjoin%
\definecolor{currentfill}{rgb}{0.982904,0.968980,0.970083}%
\pgfsetfillcolor{currentfill}%
\pgfsetlinewidth{0.000000pt}%
\definecolor{currentstroke}{rgb}{0.000000,0.000000,0.000000}%
\pgfsetstrokecolor{currentstroke}%
\pgfsetdash{}{0pt}%
\pgfpathmoveto{\pgfqpoint{2.697639in}{1.373421in}}%
\pgfpathlineto{\pgfqpoint{2.764186in}{1.396541in}}%
\pgfpathlineto{\pgfqpoint{2.697140in}{1.444818in}}%
\pgfpathlineto{\pgfqpoint{2.630498in}{1.421849in}}%
\pgfpathclose%
\pgfusepath{fill}%
\end{pgfscope}%
\begin{pgfscope}%
\pgfpathrectangle{\pgfqpoint{0.150000in}{0.150000in}}{\pgfqpoint{4.700000in}{3.450000in}}%
\pgfusepath{clip}%
\pgfsetbuttcap%
\pgfsetroundjoin%
\definecolor{currentfill}{rgb}{0.499341,0.560999,0.647319}%
\pgfsetfillcolor{currentfill}%
\pgfsetlinewidth{0.000000pt}%
\definecolor{currentstroke}{rgb}{0.000000,0.000000,0.000000}%
\pgfsetstrokecolor{currentstroke}%
\pgfsetdash{}{0pt}%
\pgfpathmoveto{\pgfqpoint{1.558387in}{2.263114in}}%
\pgfpathlineto{\pgfqpoint{1.627563in}{2.288653in}}%
\pgfpathlineto{\pgfqpoint{1.560438in}{2.363209in}}%
\pgfpathlineto{\pgfqpoint{1.491110in}{2.337451in}}%
\pgfpathclose%
\pgfusepath{fill}%
\end{pgfscope}%
\begin{pgfscope}%
\pgfpathrectangle{\pgfqpoint{0.150000in}{0.150000in}}{\pgfqpoint{4.700000in}{3.450000in}}%
\pgfusepath{clip}%
\pgfsetbuttcap%
\pgfsetroundjoin%
\definecolor{currentfill}{rgb}{0.984452,0.986366,0.989047}%
\pgfsetfillcolor{currentfill}%
\pgfsetlinewidth{0.000000pt}%
\definecolor{currentstroke}{rgb}{0.000000,0.000000,0.000000}%
\pgfsetstrokecolor{currentstroke}%
\pgfsetdash{}{0pt}%
\pgfpathmoveto{\pgfqpoint{2.496436in}{1.447228in}}%
\pgfpathlineto{\pgfqpoint{2.563514in}{1.470165in}}%
\pgfpathlineto{\pgfqpoint{2.496685in}{1.518367in}}%
\pgfpathlineto{\pgfqpoint{2.429514in}{1.495582in}}%
\pgfpathclose%
\pgfusepath{fill}%
\end{pgfscope}%
\begin{pgfscope}%
\pgfpathrectangle{\pgfqpoint{0.150000in}{0.150000in}}{\pgfqpoint{4.700000in}{3.450000in}}%
\pgfusepath{clip}%
\pgfsetbuttcap%
\pgfsetroundjoin%
\definecolor{currentfill}{rgb}{0.449586,0.517371,0.612270}%
\pgfsetfillcolor{currentfill}%
\pgfsetlinewidth{0.000000pt}%
\definecolor{currentstroke}{rgb}{0.000000,0.000000,0.000000}%
\pgfsetstrokecolor{currentstroke}%
\pgfsetdash{}{0pt}%
\pgfpathmoveto{\pgfqpoint{1.491110in}{2.337451in}}%
\pgfpathlineto{\pgfqpoint{1.560438in}{2.363209in}}%
\pgfpathlineto{\pgfqpoint{1.493284in}{2.437676in}}%
\pgfpathlineto{\pgfqpoint{1.423792in}{2.411953in}}%
\pgfpathclose%
\pgfusepath{fill}%
\end{pgfscope}%
\begin{pgfscope}%
\pgfpathrectangle{\pgfqpoint{0.150000in}{0.150000in}}{\pgfqpoint{4.700000in}{3.450000in}}%
\pgfusepath{clip}%
\pgfsetbuttcap%
\pgfsetroundjoin%
\definecolor{currentfill}{rgb}{0.860064,0.877298,0.901425}%
\pgfsetfillcolor{currentfill}%
\pgfsetlinewidth{0.000000pt}%
\definecolor{currentstroke}{rgb}{0.000000,0.000000,0.000000}%
\pgfsetstrokecolor{currentstroke}%
\pgfsetdash{}{0pt}%
\pgfpathmoveto{\pgfqpoint{2.027254in}{1.643641in}}%
\pgfpathlineto{\pgfqpoint{2.095412in}{1.669018in}}%
\pgfpathlineto{\pgfqpoint{2.028403in}{1.743179in}}%
\pgfpathlineto{\pgfqpoint{1.960075in}{1.717726in}}%
\pgfpathclose%
\pgfusepath{fill}%
\end{pgfscope}%
\begin{pgfscope}%
\pgfpathrectangle{\pgfqpoint{0.150000in}{0.150000in}}{\pgfqpoint{4.700000in}{3.450000in}}%
\pgfusepath{clip}%
\pgfsetbuttcap%
\pgfsetroundjoin%
\definecolor{currentfill}{rgb}{0.940916,0.948192,0.958379}%
\pgfsetfillcolor{currentfill}%
\pgfsetlinewidth{0.000000pt}%
\definecolor{currentstroke}{rgb}{0.000000,0.000000,0.000000}%
\pgfsetstrokecolor{currentstroke}%
\pgfsetdash{}{0pt}%
\pgfpathmoveto{\pgfqpoint{2.295140in}{1.521070in}}%
\pgfpathlineto{\pgfqpoint{2.362749in}{1.543822in}}%
\pgfpathlineto{\pgfqpoint{2.296138in}{1.591950in}}%
\pgfpathlineto{\pgfqpoint{2.228438in}{1.569348in}}%
\pgfpathclose%
\pgfusepath{fill}%
\end{pgfscope}%
\begin{pgfscope}%
\pgfpathrectangle{\pgfqpoint{0.150000in}{0.150000in}}{\pgfqpoint{4.700000in}{3.450000in}}%
\pgfusepath{clip}%
\pgfsetbuttcap%
\pgfsetroundjoin%
\definecolor{currentfill}{rgb}{0.406051,0.479197,0.581602}%
\pgfsetfillcolor{currentfill}%
\pgfsetlinewidth{0.000000pt}%
\definecolor{currentstroke}{rgb}{0.000000,0.000000,0.000000}%
\pgfsetstrokecolor{currentstroke}%
\pgfsetdash{}{0pt}%
\pgfpathmoveto{\pgfqpoint{1.423792in}{2.411953in}}%
\pgfpathlineto{\pgfqpoint{1.493284in}{2.437676in}}%
\pgfpathlineto{\pgfqpoint{1.426088in}{2.512311in}}%
\pgfpathlineto{\pgfqpoint{1.356439in}{2.486494in}}%
\pgfpathclose%
\pgfusepath{fill}%
\end{pgfscope}%
\begin{pgfscope}%
\pgfpathrectangle{\pgfqpoint{0.150000in}{0.150000in}}{\pgfqpoint{4.700000in}{3.450000in}}%
\pgfusepath{clip}%
\pgfsetbuttcap%
\pgfsetroundjoin%
\definecolor{currentfill}{rgb}{0.810309,0.833670,0.866376}%
\pgfsetfillcolor{currentfill}%
\pgfsetlinewidth{0.000000pt}%
\definecolor{currentstroke}{rgb}{0.000000,0.000000,0.000000}%
\pgfsetstrokecolor{currentstroke}%
\pgfsetdash{}{0pt}%
\pgfpathmoveto{\pgfqpoint{1.960075in}{1.717726in}}%
\pgfpathlineto{\pgfqpoint{2.028403in}{1.743179in}}%
\pgfpathlineto{\pgfqpoint{1.961363in}{1.817253in}}%
\pgfpathlineto{\pgfqpoint{1.892862in}{1.791846in}}%
\pgfpathclose%
\pgfusepath{fill}%
\end{pgfscope}%
\begin{pgfscope}%
\pgfpathrectangle{\pgfqpoint{0.150000in}{0.150000in}}{\pgfqpoint{4.700000in}{3.450000in}}%
\pgfusepath{clip}%
\pgfsetbuttcap%
\pgfsetroundjoin%
\definecolor{currentfill}{rgb}{0.891161,0.904565,0.923330}%
\pgfsetfillcolor{currentfill}%
\pgfsetlinewidth{0.000000pt}%
\definecolor{currentstroke}{rgb}{0.000000,0.000000,0.000000}%
\pgfsetstrokecolor{currentstroke}%
\pgfsetdash{}{0pt}%
\pgfpathmoveto{\pgfqpoint{2.093751in}{1.594945in}}%
\pgfpathlineto{\pgfqpoint{2.161891in}{1.617513in}}%
\pgfpathlineto{\pgfqpoint{2.095412in}{1.669018in}}%
\pgfpathlineto{\pgfqpoint{2.027254in}{1.643641in}}%
\pgfpathclose%
\pgfusepath{fill}%
\end{pgfscope}%
\begin{pgfscope}%
\pgfpathrectangle{\pgfqpoint{0.150000in}{0.150000in}}{\pgfqpoint{4.700000in}{3.450000in}}%
\pgfusepath{clip}%
\pgfsetbuttcap%
\pgfsetroundjoin%
\definecolor{currentfill}{rgb}{0.356296,0.435570,0.546553}%
\pgfsetfillcolor{currentfill}%
\pgfsetlinewidth{0.000000pt}%
\definecolor{currentstroke}{rgb}{0.000000,0.000000,0.000000}%
\pgfsetstrokecolor{currentstroke}%
\pgfsetdash{}{0pt}%
\pgfpathmoveto{\pgfqpoint{1.356439in}{2.486494in}}%
\pgfpathlineto{\pgfqpoint{1.426088in}{2.512311in}}%
\pgfpathlineto{\pgfqpoint{1.358863in}{2.586857in}}%
\pgfpathlineto{\pgfqpoint{1.289051in}{2.561074in}}%
\pgfpathclose%
\pgfusepath{fill}%
\end{pgfscope}%
\begin{pgfscope}%
\pgfpathrectangle{\pgfqpoint{0.150000in}{0.150000in}}{\pgfqpoint{4.700000in}{3.450000in}}%
\pgfusepath{clip}%
\pgfsetbuttcap%
\pgfsetroundjoin%
\definecolor{currentfill}{rgb}{0.766774,0.795496,0.835708}%
\pgfsetfillcolor{currentfill}%
\pgfsetlinewidth{0.000000pt}%
\definecolor{currentstroke}{rgb}{0.000000,0.000000,0.000000}%
\pgfsetstrokecolor{currentstroke}%
\pgfsetdash{}{0pt}%
\pgfpathmoveto{\pgfqpoint{1.892862in}{1.791846in}}%
\pgfpathlineto{\pgfqpoint{1.961363in}{1.817253in}}%
\pgfpathlineto{\pgfqpoint{1.894286in}{1.891489in}}%
\pgfpathlineto{\pgfqpoint{1.825617in}{1.866002in}}%
\pgfpathclose%
\pgfusepath{fill}%
\end{pgfscope}%
\begin{pgfscope}%
\pgfpathrectangle{\pgfqpoint{0.150000in}{0.150000in}}{\pgfqpoint{4.700000in}{3.450000in}}%
\pgfusepath{clip}%
\pgfsetbuttcap%
\pgfsetroundjoin%
\definecolor{currentfill}{rgb}{0.944914,0.900046,0.903600}%
\pgfsetfillcolor{currentfill}%
\pgfsetlinewidth{0.000000pt}%
\definecolor{currentstroke}{rgb}{0.000000,0.000000,0.000000}%
\pgfsetstrokecolor{currentstroke}%
\pgfsetdash{}{0pt}%
\pgfpathmoveto{\pgfqpoint{2.966266in}{1.251032in}}%
\pgfpathlineto{\pgfqpoint{3.032188in}{1.274487in}}%
\pgfpathlineto{\pgfqpoint{2.964767in}{1.322951in}}%
\pgfpathlineto{\pgfqpoint{2.898749in}{1.299648in}}%
\pgfpathclose%
\pgfusepath{fill}%
\end{pgfscope}%
\begin{pgfscope}%
\pgfpathrectangle{\pgfqpoint{0.150000in}{0.150000in}}{\pgfqpoint{4.700000in}{3.450000in}}%
\pgfusepath{clip}%
\pgfsetbuttcap%
\pgfsetroundjoin%
\definecolor{currentfill}{rgb}{0.312760,0.397396,0.515885}%
\pgfsetfillcolor{currentfill}%
\pgfsetlinewidth{0.000000pt}%
\definecolor{currentstroke}{rgb}{0.000000,0.000000,0.000000}%
\pgfsetstrokecolor{currentstroke}%
\pgfsetdash{}{0pt}%
\pgfpathmoveto{\pgfqpoint{1.289051in}{2.561074in}}%
\pgfpathlineto{\pgfqpoint{1.358863in}{2.586857in}}%
\pgfpathlineto{\pgfqpoint{1.291594in}{2.661572in}}%
\pgfpathlineto{\pgfqpoint{1.221638in}{2.635561in}}%
\pgfpathclose%
\pgfusepath{fill}%
\end{pgfscope}%
\begin{pgfscope}%
\pgfpathrectangle{\pgfqpoint{0.150000in}{0.150000in}}{\pgfqpoint{4.700000in}{3.450000in}}%
\pgfusepath{clip}%
\pgfsetbuttcap%
\pgfsetroundjoin%
\definecolor{currentfill}{rgb}{0.723238,0.757322,0.805040}%
\pgfsetfillcolor{currentfill}%
\pgfsetlinewidth{0.000000pt}%
\definecolor{currentstroke}{rgb}{0.000000,0.000000,0.000000}%
\pgfsetstrokecolor{currentstroke}%
\pgfsetdash{}{0pt}%
\pgfpathmoveto{\pgfqpoint{1.825617in}{1.866002in}}%
\pgfpathlineto{\pgfqpoint{1.894286in}{1.891489in}}%
\pgfpathlineto{\pgfqpoint{1.827174in}{1.965763in}}%
\pgfpathlineto{\pgfqpoint{1.758339in}{1.940195in}}%
\pgfpathclose%
\pgfusepath{fill}%
\end{pgfscope}%
\begin{pgfscope}%
\pgfpathrectangle{\pgfqpoint{0.150000in}{0.150000in}}{\pgfqpoint{4.700000in}{3.450000in}}%
\pgfusepath{clip}%
\pgfsetbuttcap%
\pgfsetroundjoin%
\definecolor{currentfill}{rgb}{0.971507,0.948300,0.950138}%
\pgfsetfillcolor{currentfill}%
\pgfsetlinewidth{0.000000pt}%
\definecolor{currentstroke}{rgb}{0.000000,0.000000,0.000000}%
\pgfsetstrokecolor{currentstroke}%
\pgfsetdash{}{0pt}%
\pgfpathmoveto{\pgfqpoint{2.764936in}{1.324880in}}%
\pgfpathlineto{\pgfqpoint{2.831389in}{1.348151in}}%
\pgfpathlineto{\pgfqpoint{2.764186in}{1.396541in}}%
\pgfpathlineto{\pgfqpoint{2.697639in}{1.373421in}}%
\pgfpathclose%
\pgfusepath{fill}%
\end{pgfscope}%
\begin{pgfscope}%
\pgfpathrectangle{\pgfqpoint{0.150000in}{0.150000in}}{\pgfqpoint{4.700000in}{3.450000in}}%
\pgfusepath{clip}%
\pgfsetbuttcap%
\pgfsetroundjoin%
\definecolor{currentfill}{rgb}{0.263006,0.353768,0.480836}%
\pgfsetfillcolor{currentfill}%
\pgfsetlinewidth{0.000000pt}%
\definecolor{currentstroke}{rgb}{0.000000,0.000000,0.000000}%
\pgfsetstrokecolor{currentstroke}%
\pgfsetdash{}{0pt}%
\pgfpathmoveto{\pgfqpoint{1.221638in}{2.635561in}}%
\pgfpathlineto{\pgfqpoint{1.291594in}{2.661572in}}%
\pgfpathlineto{\pgfqpoint{1.224298in}{2.736198in}}%
\pgfpathlineto{\pgfqpoint{1.154181in}{2.710217in}}%
\pgfpathclose%
\pgfusepath{fill}%
\end{pgfscope}%
\begin{pgfscope}%
\pgfpathrectangle{\pgfqpoint{0.150000in}{0.150000in}}{\pgfqpoint{4.700000in}{3.450000in}}%
\pgfusepath{clip}%
\pgfsetbuttcap%
\pgfsetroundjoin%
\definecolor{currentfill}{rgb}{0.673483,0.713695,0.769991}%
\pgfsetfillcolor{currentfill}%
\pgfsetlinewidth{0.000000pt}%
\definecolor{currentstroke}{rgb}{0.000000,0.000000,0.000000}%
\pgfsetstrokecolor{currentstroke}%
\pgfsetdash{}{0pt}%
\pgfpathmoveto{\pgfqpoint{1.758339in}{1.940195in}}%
\pgfpathlineto{\pgfqpoint{1.827174in}{1.965763in}}%
\pgfpathlineto{\pgfqpoint{1.760027in}{2.040075in}}%
\pgfpathlineto{\pgfqpoint{1.691028in}{2.014424in}}%
\pgfpathclose%
\pgfusepath{fill}%
\end{pgfscope}%
\begin{pgfscope}%
\pgfpathrectangle{\pgfqpoint{0.150000in}{0.150000in}}{\pgfqpoint{4.700000in}{3.450000in}}%
\pgfusepath{clip}%
\pgfsetbuttcap%
\pgfsetroundjoin%
\definecolor{currentfill}{rgb}{0.996890,0.997273,0.997809}%
\pgfsetfillcolor{currentfill}%
\pgfsetlinewidth{0.000000pt}%
\definecolor{currentstroke}{rgb}{0.000000,0.000000,0.000000}%
\pgfsetstrokecolor{currentstroke}%
\pgfsetdash{}{0pt}%
\pgfpathmoveto{\pgfqpoint{2.563514in}{1.398762in}}%
\pgfpathlineto{\pgfqpoint{2.630498in}{1.421849in}}%
\pgfpathlineto{\pgfqpoint{2.563514in}{1.470165in}}%
\pgfpathlineto{\pgfqpoint{2.496436in}{1.447228in}}%
\pgfpathclose%
\pgfusepath{fill}%
\end{pgfscope}%
\begin{pgfscope}%
\pgfpathrectangle{\pgfqpoint{0.150000in}{0.150000in}}{\pgfqpoint{4.700000in}{3.450000in}}%
\pgfusepath{clip}%
\pgfsetbuttcap%
\pgfsetroundjoin%
\definecolor{currentfill}{rgb}{0.629948,0.675521,0.739323}%
\pgfsetfillcolor{currentfill}%
\pgfsetlinewidth{0.000000pt}%
\definecolor{currentstroke}{rgb}{0.000000,0.000000,0.000000}%
\pgfsetstrokecolor{currentstroke}%
\pgfsetdash{}{0pt}%
\pgfpathmoveto{\pgfqpoint{1.691028in}{2.014424in}}%
\pgfpathlineto{\pgfqpoint{1.760027in}{2.040075in}}%
\pgfpathlineto{\pgfqpoint{1.692852in}{2.114299in}}%
\pgfpathlineto{\pgfqpoint{1.623685in}{2.088689in}}%
\pgfpathclose%
\pgfusepath{fill}%
\end{pgfscope}%
\begin{pgfscope}%
\pgfpathrectangle{\pgfqpoint{0.150000in}{0.150000in}}{\pgfqpoint{4.700000in}{3.450000in}}%
\pgfusepath{clip}%
\pgfsetbuttcap%
\pgfsetroundjoin%
\definecolor{currentfill}{rgb}{0.953355,0.959099,0.967142}%
\pgfsetfillcolor{currentfill}%
\pgfsetlinewidth{0.000000pt}%
\definecolor{currentstroke}{rgb}{0.000000,0.000000,0.000000}%
\pgfsetstrokecolor{currentstroke}%
\pgfsetdash{}{0pt}%
\pgfpathmoveto{\pgfqpoint{2.361998in}{1.472679in}}%
\pgfpathlineto{\pgfqpoint{2.429514in}{1.495582in}}%
\pgfpathlineto{\pgfqpoint{2.362749in}{1.543822in}}%
\pgfpathlineto{\pgfqpoint{2.295140in}{1.521070in}}%
\pgfpathclose%
\pgfusepath{fill}%
\end{pgfscope}%
\begin{pgfscope}%
\pgfpathrectangle{\pgfqpoint{0.150000in}{0.150000in}}{\pgfqpoint{4.700000in}{3.450000in}}%
\pgfusepath{clip}%
\pgfsetbuttcap%
\pgfsetroundjoin%
\definecolor{currentfill}{rgb}{0.580193,0.631893,0.704274}%
\pgfsetfillcolor{currentfill}%
\pgfsetlinewidth{0.000000pt}%
\definecolor{currentstroke}{rgb}{0.000000,0.000000,0.000000}%
\pgfsetstrokecolor{currentstroke}%
\pgfsetdash{}{0pt}%
\pgfpathmoveto{\pgfqpoint{1.623685in}{2.088689in}}%
\pgfpathlineto{\pgfqpoint{1.692852in}{2.114299in}}%
\pgfpathlineto{\pgfqpoint{1.625637in}{2.188687in}}%
\pgfpathlineto{\pgfqpoint{1.556308in}{2.162991in}}%
\pgfpathclose%
\pgfusepath{fill}%
\end{pgfscope}%
\begin{pgfscope}%
\pgfpathrectangle{\pgfqpoint{0.150000in}{0.150000in}}{\pgfqpoint{4.700000in}{3.450000in}}%
\pgfusepath{clip}%
\pgfsetbuttcap%
\pgfsetroundjoin%
\definecolor{currentfill}{rgb}{0.909819,0.920925,0.936474}%
\pgfsetfillcolor{currentfill}%
\pgfsetlinewidth{0.000000pt}%
\definecolor{currentstroke}{rgb}{0.000000,0.000000,0.000000}%
\pgfsetstrokecolor{currentstroke}%
\pgfsetdash{}{0pt}%
\pgfpathmoveto{\pgfqpoint{2.160389in}{1.546629in}}%
\pgfpathlineto{\pgfqpoint{2.228438in}{1.569348in}}%
\pgfpathlineto{\pgfqpoint{2.161891in}{1.617513in}}%
\pgfpathlineto{\pgfqpoint{2.093751in}{1.594945in}}%
\pgfpathclose%
\pgfusepath{fill}%
\end{pgfscope}%
\begin{pgfscope}%
\pgfpathrectangle{\pgfqpoint{0.150000in}{0.150000in}}{\pgfqpoint{4.700000in}{3.450000in}}%
\pgfusepath{clip}%
\pgfsetbuttcap%
\pgfsetroundjoin%
\definecolor{currentfill}{rgb}{0.536657,0.593719,0.673606}%
\pgfsetfillcolor{currentfill}%
\pgfsetlinewidth{0.000000pt}%
\definecolor{currentstroke}{rgb}{0.000000,0.000000,0.000000}%
\pgfsetstrokecolor{currentstroke}%
\pgfsetdash{}{0pt}%
\pgfpathmoveto{\pgfqpoint{1.556308in}{2.162991in}}%
\pgfpathlineto{\pgfqpoint{1.625637in}{2.188687in}}%
\pgfpathlineto{\pgfqpoint{1.558387in}{2.263114in}}%
\pgfpathlineto{\pgfqpoint{1.488898in}{2.237328in}}%
\pgfpathclose%
\pgfusepath{fill}%
\end{pgfscope}%
\begin{pgfscope}%
\pgfpathrectangle{\pgfqpoint{0.150000in}{0.150000in}}{\pgfqpoint{4.700000in}{3.450000in}}%
\pgfusepath{clip}%
\pgfsetbuttcap%
\pgfsetroundjoin%
\definecolor{currentfill}{rgb}{0.933517,0.879366,0.883655}%
\pgfsetfillcolor{currentfill}%
\pgfsetlinewidth{0.000000pt}%
\definecolor{currentstroke}{rgb}{0.000000,0.000000,0.000000}%
\pgfsetstrokecolor{currentstroke}%
\pgfsetdash{}{0pt}%
\pgfpathmoveto{\pgfqpoint{3.033941in}{1.202302in}}%
\pgfpathlineto{\pgfqpoint{3.099766in}{1.225909in}}%
\pgfpathlineto{\pgfqpoint{3.032188in}{1.274487in}}%
\pgfpathlineto{\pgfqpoint{2.966266in}{1.251032in}}%
\pgfpathclose%
\pgfusepath{fill}%
\end{pgfscope}%
\begin{pgfscope}%
\pgfpathrectangle{\pgfqpoint{0.150000in}{0.150000in}}{\pgfqpoint{4.700000in}{3.450000in}}%
\pgfusepath{clip}%
\pgfsetbuttcap%
\pgfsetroundjoin%
\definecolor{currentfill}{rgb}{0.963909,0.934513,0.936841}%
\pgfsetfillcolor{currentfill}%
\pgfsetlinewidth{0.000000pt}%
\definecolor{currentstroke}{rgb}{0.000000,0.000000,0.000000}%
\pgfsetstrokecolor{currentstroke}%
\pgfsetdash{}{0pt}%
\pgfpathmoveto{\pgfqpoint{2.832391in}{1.276225in}}%
\pgfpathlineto{\pgfqpoint{2.898749in}{1.299648in}}%
\pgfpathlineto{\pgfqpoint{2.831389in}{1.348151in}}%
\pgfpathlineto{\pgfqpoint{2.764936in}{1.324880in}}%
\pgfpathclose%
\pgfusepath{fill}%
\end{pgfscope}%
\begin{pgfscope}%
\pgfpathrectangle{\pgfqpoint{0.150000in}{0.150000in}}{\pgfqpoint{4.700000in}{3.450000in}}%
\pgfusepath{clip}%
\pgfsetbuttcap%
\pgfsetroundjoin%
\definecolor{currentfill}{rgb}{0.486903,0.550092,0.638557}%
\pgfsetfillcolor{currentfill}%
\pgfsetlinewidth{0.000000pt}%
\definecolor{currentstroke}{rgb}{0.000000,0.000000,0.000000}%
\pgfsetstrokecolor{currentstroke}%
\pgfsetdash{}{0pt}%
\pgfpathmoveto{\pgfqpoint{1.488898in}{2.237328in}}%
\pgfpathlineto{\pgfqpoint{1.558387in}{2.263114in}}%
\pgfpathlineto{\pgfqpoint{1.491110in}{2.337451in}}%
\pgfpathlineto{\pgfqpoint{1.421456in}{2.311702in}}%
\pgfpathclose%
\pgfusepath{fill}%
\end{pgfscope}%
\begin{pgfscope}%
\pgfpathrectangle{\pgfqpoint{0.150000in}{0.150000in}}{\pgfqpoint{4.700000in}{3.450000in}}%
\pgfusepath{clip}%
\pgfsetbuttcap%
\pgfsetroundjoin%
\definecolor{currentfill}{rgb}{0.990502,0.982767,0.983379}%
\pgfsetfillcolor{currentfill}%
\pgfsetlinewidth{0.000000pt}%
\definecolor{currentstroke}{rgb}{0.000000,0.000000,0.000000}%
\pgfsetstrokecolor{currentstroke}%
\pgfsetdash{}{0pt}%
\pgfpathmoveto{\pgfqpoint{2.630749in}{1.350183in}}%
\pgfpathlineto{\pgfqpoint{2.697639in}{1.373421in}}%
\pgfpathlineto{\pgfqpoint{2.630498in}{1.421849in}}%
\pgfpathlineto{\pgfqpoint{2.563514in}{1.398762in}}%
\pgfpathclose%
\pgfusepath{fill}%
\end{pgfscope}%
\begin{pgfscope}%
\pgfpathrectangle{\pgfqpoint{0.150000in}{0.150000in}}{\pgfqpoint{4.700000in}{3.450000in}}%
\pgfusepath{clip}%
\pgfsetbuttcap%
\pgfsetroundjoin%
\definecolor{currentfill}{rgb}{0.443367,0.511918,0.607889}%
\pgfsetfillcolor{currentfill}%
\pgfsetlinewidth{0.000000pt}%
\definecolor{currentstroke}{rgb}{0.000000,0.000000,0.000000}%
\pgfsetstrokecolor{currentstroke}%
\pgfsetdash{}{0pt}%
\pgfpathmoveto{\pgfqpoint{1.421456in}{2.311702in}}%
\pgfpathlineto{\pgfqpoint{1.491110in}{2.337451in}}%
\pgfpathlineto{\pgfqpoint{1.423792in}{2.411953in}}%
\pgfpathlineto{\pgfqpoint{1.353980in}{2.386113in}}%
\pgfpathclose%
\pgfusepath{fill}%
\end{pgfscope}%
\begin{pgfscope}%
\pgfpathrectangle{\pgfqpoint{0.150000in}{0.150000in}}{\pgfqpoint{4.700000in}{3.450000in}}%
\pgfusepath{clip}%
\pgfsetbuttcap%
\pgfsetroundjoin%
\definecolor{currentfill}{rgb}{0.847626,0.866391,0.892662}%
\pgfsetfillcolor{currentfill}%
\pgfsetlinewidth{0.000000pt}%
\definecolor{currentstroke}{rgb}{0.000000,0.000000,0.000000}%
\pgfsetstrokecolor{currentstroke}%
\pgfsetdash{}{0pt}%
\pgfpathmoveto{\pgfqpoint{1.958687in}{1.620614in}}%
\pgfpathlineto{\pgfqpoint{2.027254in}{1.643641in}}%
\pgfpathlineto{\pgfqpoint{1.960075in}{1.717726in}}%
\pgfpathlineto{\pgfqpoint{1.891415in}{1.692275in}}%
\pgfpathclose%
\pgfusepath{fill}%
\end{pgfscope}%
\begin{pgfscope}%
\pgfpathrectangle{\pgfqpoint{0.150000in}{0.150000in}}{\pgfqpoint{4.700000in}{3.450000in}}%
\pgfusepath{clip}%
\pgfsetbuttcap%
\pgfsetroundjoin%
\definecolor{currentfill}{rgb}{0.972013,0.975460,0.980285}%
\pgfsetfillcolor{currentfill}%
\pgfsetlinewidth{0.000000pt}%
\definecolor{currentstroke}{rgb}{0.000000,0.000000,0.000000}%
\pgfsetstrokecolor{currentstroke}%
\pgfsetdash{}{0pt}%
\pgfpathmoveto{\pgfqpoint{2.429012in}{1.424174in}}%
\pgfpathlineto{\pgfqpoint{2.496436in}{1.447228in}}%
\pgfpathlineto{\pgfqpoint{2.429514in}{1.495582in}}%
\pgfpathlineto{\pgfqpoint{2.361998in}{1.472679in}}%
\pgfpathclose%
\pgfusepath{fill}%
\end{pgfscope}%
\begin{pgfscope}%
\pgfpathrectangle{\pgfqpoint{0.150000in}{0.150000in}}{\pgfqpoint{4.700000in}{3.450000in}}%
\pgfusepath{clip}%
\pgfsetbuttcap%
\pgfsetroundjoin%
\definecolor{currentfill}{rgb}{0.393612,0.468290,0.572840}%
\pgfsetfillcolor{currentfill}%
\pgfsetlinewidth{0.000000pt}%
\definecolor{currentstroke}{rgb}{0.000000,0.000000,0.000000}%
\pgfsetstrokecolor{currentstroke}%
\pgfsetdash{}{0pt}%
\pgfpathmoveto{\pgfqpoint{1.353980in}{2.386113in}}%
\pgfpathlineto{\pgfqpoint{1.423792in}{2.411953in}}%
\pgfpathlineto{\pgfqpoint{1.356439in}{2.486494in}}%
\pgfpathlineto{\pgfqpoint{1.286472in}{2.460560in}}%
\pgfpathclose%
\pgfusepath{fill}%
\end{pgfscope}%
\begin{pgfscope}%
\pgfpathrectangle{\pgfqpoint{0.150000in}{0.150000in}}{\pgfqpoint{4.700000in}{3.450000in}}%
\pgfusepath{clip}%
\pgfsetbuttcap%
\pgfsetroundjoin%
\definecolor{currentfill}{rgb}{0.804090,0.828217,0.861994}%
\pgfsetfillcolor{currentfill}%
\pgfsetlinewidth{0.000000pt}%
\definecolor{currentstroke}{rgb}{0.000000,0.000000,0.000000}%
\pgfsetstrokecolor{currentstroke}%
\pgfsetdash{}{0pt}%
\pgfpathmoveto{\pgfqpoint{1.891415in}{1.692275in}}%
\pgfpathlineto{\pgfqpoint{1.960075in}{1.717726in}}%
\pgfpathlineto{\pgfqpoint{1.892862in}{1.791846in}}%
\pgfpathlineto{\pgfqpoint{1.824034in}{1.766318in}}%
\pgfpathclose%
\pgfusepath{fill}%
\end{pgfscope}%
\begin{pgfscope}%
\pgfpathrectangle{\pgfqpoint{0.150000in}{0.150000in}}{\pgfqpoint{4.700000in}{3.450000in}}%
\pgfusepath{clip}%
\pgfsetbuttcap%
\pgfsetroundjoin%
\definecolor{currentfill}{rgb}{0.922258,0.931832,0.945236}%
\pgfsetfillcolor{currentfill}%
\pgfsetlinewidth{0.000000pt}%
\definecolor{currentstroke}{rgb}{0.000000,0.000000,0.000000}%
\pgfsetstrokecolor{currentstroke}%
\pgfsetdash{}{0pt}%
\pgfpathmoveto{\pgfqpoint{2.227183in}{1.498200in}}%
\pgfpathlineto{\pgfqpoint{2.295140in}{1.521070in}}%
\pgfpathlineto{\pgfqpoint{2.228438in}{1.569348in}}%
\pgfpathlineto{\pgfqpoint{2.160389in}{1.546629in}}%
\pgfpathclose%
\pgfusepath{fill}%
\end{pgfscope}%
\begin{pgfscope}%
\pgfpathrectangle{\pgfqpoint{0.150000in}{0.150000in}}{\pgfqpoint{4.700000in}{3.450000in}}%
\pgfusepath{clip}%
\pgfsetbuttcap%
\pgfsetroundjoin%
\definecolor{currentfill}{rgb}{0.350077,0.430116,0.542172}%
\pgfsetfillcolor{currentfill}%
\pgfsetlinewidth{0.000000pt}%
\definecolor{currentstroke}{rgb}{0.000000,0.000000,0.000000}%
\pgfsetstrokecolor{currentstroke}%
\pgfsetdash{}{0pt}%
\pgfpathmoveto{\pgfqpoint{1.286472in}{2.460560in}}%
\pgfpathlineto{\pgfqpoint{1.356439in}{2.486494in}}%
\pgfpathlineto{\pgfqpoint{1.289051in}{2.561074in}}%
\pgfpathlineto{\pgfqpoint{1.218930in}{2.535043in}}%
\pgfpathclose%
\pgfusepath{fill}%
\end{pgfscope}%
\begin{pgfscope}%
\pgfpathrectangle{\pgfqpoint{0.150000in}{0.150000in}}{\pgfqpoint{4.700000in}{3.450000in}}%
\pgfusepath{clip}%
\pgfsetbuttcap%
\pgfsetroundjoin%
\definecolor{currentfill}{rgb}{0.754335,0.784589,0.826945}%
\pgfsetfillcolor{currentfill}%
\pgfsetlinewidth{0.000000pt}%
\definecolor{currentstroke}{rgb}{0.000000,0.000000,0.000000}%
\pgfsetstrokecolor{currentstroke}%
\pgfsetdash{}{0pt}%
\pgfpathmoveto{\pgfqpoint{1.824034in}{1.766318in}}%
\pgfpathlineto{\pgfqpoint{1.892862in}{1.791846in}}%
\pgfpathlineto{\pgfqpoint{1.825617in}{1.866002in}}%
\pgfpathlineto{\pgfqpoint{1.756622in}{1.840395in}}%
\pgfpathclose%
\pgfusepath{fill}%
\end{pgfscope}%
\begin{pgfscope}%
\pgfpathrectangle{\pgfqpoint{0.150000in}{0.150000in}}{\pgfqpoint{4.700000in}{3.450000in}}%
\pgfusepath{clip}%
\pgfsetbuttcap%
\pgfsetroundjoin%
\definecolor{currentfill}{rgb}{0.878722,0.893658,0.914568}%
\pgfsetfillcolor{currentfill}%
\pgfsetlinewidth{0.000000pt}%
\definecolor{currentstroke}{rgb}{0.000000,0.000000,0.000000}%
\pgfsetstrokecolor{currentstroke}%
\pgfsetdash{}{0pt}%
\pgfpathmoveto{\pgfqpoint{2.025260in}{1.572260in}}%
\pgfpathlineto{\pgfqpoint{2.093751in}{1.594945in}}%
\pgfpathlineto{\pgfqpoint{2.027254in}{1.643641in}}%
\pgfpathlineto{\pgfqpoint{1.958687in}{1.620614in}}%
\pgfpathclose%
\pgfusepath{fill}%
\end{pgfscope}%
\begin{pgfscope}%
\pgfpathrectangle{\pgfqpoint{0.150000in}{0.150000in}}{\pgfqpoint{4.700000in}{3.450000in}}%
\pgfusepath{clip}%
\pgfsetbuttcap%
\pgfsetroundjoin%
\definecolor{currentfill}{rgb}{0.306541,0.391942,0.511504}%
\pgfsetfillcolor{currentfill}%
\pgfsetlinewidth{0.000000pt}%
\definecolor{currentstroke}{rgb}{0.000000,0.000000,0.000000}%
\pgfsetstrokecolor{currentstroke}%
\pgfsetdash{}{0pt}%
\pgfpathmoveto{\pgfqpoint{1.218930in}{2.535043in}}%
\pgfpathlineto{\pgfqpoint{1.289051in}{2.561074in}}%
\pgfpathlineto{\pgfqpoint{1.221638in}{2.635561in}}%
\pgfpathlineto{\pgfqpoint{1.151356in}{2.609563in}}%
\pgfpathclose%
\pgfusepath{fill}%
\end{pgfscope}%
\begin{pgfscope}%
\pgfpathrectangle{\pgfqpoint{0.150000in}{0.150000in}}{\pgfqpoint{4.700000in}{3.450000in}}%
\pgfusepath{clip}%
\pgfsetbuttcap%
\pgfsetroundjoin%
\definecolor{currentfill}{rgb}{0.710800,0.746415,0.796278}%
\pgfsetfillcolor{currentfill}%
\pgfsetlinewidth{0.000000pt}%
\definecolor{currentstroke}{rgb}{0.000000,0.000000,0.000000}%
\pgfsetstrokecolor{currentstroke}%
\pgfsetdash{}{0pt}%
\pgfpathmoveto{\pgfqpoint{1.756622in}{1.840395in}}%
\pgfpathlineto{\pgfqpoint{1.825617in}{1.866002in}}%
\pgfpathlineto{\pgfqpoint{1.758339in}{1.940195in}}%
\pgfpathlineto{\pgfqpoint{1.689179in}{1.914506in}}%
\pgfpathclose%
\pgfusepath{fill}%
\end{pgfscope}%
\begin{pgfscope}%
\pgfpathrectangle{\pgfqpoint{0.150000in}{0.150000in}}{\pgfqpoint{4.700000in}{3.450000in}}%
\pgfusepath{clip}%
\pgfsetbuttcap%
\pgfsetroundjoin%
\definecolor{currentfill}{rgb}{0.952512,0.913833,0.916896}%
\pgfsetfillcolor{currentfill}%
\pgfsetlinewidth{0.000000pt}%
\definecolor{currentstroke}{rgb}{0.000000,0.000000,0.000000}%
\pgfsetstrokecolor{currentstroke}%
\pgfsetdash{}{0pt}%
\pgfpathmoveto{\pgfqpoint{2.900005in}{1.227456in}}%
\pgfpathlineto{\pgfqpoint{2.966266in}{1.251032in}}%
\pgfpathlineto{\pgfqpoint{2.898749in}{1.299648in}}%
\pgfpathlineto{\pgfqpoint{2.832391in}{1.276225in}}%
\pgfpathclose%
\pgfusepath{fill}%
\end{pgfscope}%
\begin{pgfscope}%
\pgfpathrectangle{\pgfqpoint{0.150000in}{0.150000in}}{\pgfqpoint{4.700000in}{3.450000in}}%
\pgfusepath{clip}%
\pgfsetbuttcap%
\pgfsetroundjoin%
\definecolor{currentfill}{rgb}{0.256786,0.348315,0.476455}%
\pgfsetfillcolor{currentfill}%
\pgfsetlinewidth{0.000000pt}%
\definecolor{currentstroke}{rgb}{0.000000,0.000000,0.000000}%
\pgfsetstrokecolor{currentstroke}%
\pgfsetdash{}{0pt}%
\pgfpathmoveto{\pgfqpoint{1.151356in}{2.609563in}}%
\pgfpathlineto{\pgfqpoint{1.221638in}{2.635561in}}%
\pgfpathlineto{\pgfqpoint{1.154181in}{2.710217in}}%
\pgfpathlineto{\pgfqpoint{1.083748in}{2.684119in}}%
\pgfpathclose%
\pgfusepath{fill}%
\end{pgfscope}%
\begin{pgfscope}%
\pgfpathrectangle{\pgfqpoint{0.150000in}{0.150000in}}{\pgfqpoint{4.700000in}{3.450000in}}%
\pgfusepath{clip}%
\pgfsetbuttcap%
\pgfsetroundjoin%
\definecolor{currentfill}{rgb}{0.661045,0.702788,0.761229}%
\pgfsetfillcolor{currentfill}%
\pgfsetlinewidth{0.000000pt}%
\definecolor{currentstroke}{rgb}{0.000000,0.000000,0.000000}%
\pgfsetstrokecolor{currentstroke}%
\pgfsetdash{}{0pt}%
\pgfpathmoveto{\pgfqpoint{1.689179in}{1.914506in}}%
\pgfpathlineto{\pgfqpoint{1.758339in}{1.940195in}}%
\pgfpathlineto{\pgfqpoint{1.691028in}{2.014424in}}%
\pgfpathlineto{\pgfqpoint{1.621699in}{1.988778in}}%
\pgfpathclose%
\pgfusepath{fill}%
\end{pgfscope}%
\begin{pgfscope}%
\pgfpathrectangle{\pgfqpoint{0.150000in}{0.150000in}}{\pgfqpoint{4.700000in}{3.450000in}}%
\pgfusepath{clip}%
\pgfsetbuttcap%
\pgfsetroundjoin%
\definecolor{currentfill}{rgb}{0.982904,0.968980,0.970083}%
\pgfsetfillcolor{currentfill}%
\pgfsetlinewidth{0.000000pt}%
\definecolor{currentstroke}{rgb}{0.000000,0.000000,0.000000}%
\pgfsetstrokecolor{currentstroke}%
\pgfsetdash{}{0pt}%
\pgfpathmoveto{\pgfqpoint{2.698141in}{1.301489in}}%
\pgfpathlineto{\pgfqpoint{2.764936in}{1.324880in}}%
\pgfpathlineto{\pgfqpoint{2.697639in}{1.373421in}}%
\pgfpathlineto{\pgfqpoint{2.630749in}{1.350183in}}%
\pgfpathclose%
\pgfusepath{fill}%
\end{pgfscope}%
\begin{pgfscope}%
\pgfpathrectangle{\pgfqpoint{0.150000in}{0.150000in}}{\pgfqpoint{4.700000in}{3.450000in}}%
\pgfusepath{clip}%
\pgfsetbuttcap%
\pgfsetroundjoin%
\definecolor{currentfill}{rgb}{0.617509,0.664614,0.730561}%
\pgfsetfillcolor{currentfill}%
\pgfsetlinewidth{0.000000pt}%
\definecolor{currentstroke}{rgb}{0.000000,0.000000,0.000000}%
\pgfsetstrokecolor{currentstroke}%
\pgfsetdash{}{0pt}%
\pgfpathmoveto{\pgfqpoint{1.621699in}{1.988778in}}%
\pgfpathlineto{\pgfqpoint{1.691028in}{2.014424in}}%
\pgfpathlineto{\pgfqpoint{1.623685in}{2.088689in}}%
\pgfpathlineto{\pgfqpoint{1.554193in}{2.062959in}}%
\pgfpathclose%
\pgfusepath{fill}%
\end{pgfscope}%
\begin{pgfscope}%
\pgfpathrectangle{\pgfqpoint{0.150000in}{0.150000in}}{\pgfqpoint{4.700000in}{3.450000in}}%
\pgfusepath{clip}%
\pgfsetbuttcap%
\pgfsetroundjoin%
\definecolor{currentfill}{rgb}{0.984452,0.986366,0.989047}%
\pgfsetfillcolor{currentfill}%
\pgfsetlinewidth{0.000000pt}%
\definecolor{currentstroke}{rgb}{0.000000,0.000000,0.000000}%
\pgfsetstrokecolor{currentstroke}%
\pgfsetdash{}{0pt}%
\pgfpathmoveto{\pgfqpoint{2.496184in}{1.375556in}}%
\pgfpathlineto{\pgfqpoint{2.563514in}{1.398762in}}%
\pgfpathlineto{\pgfqpoint{2.496436in}{1.447228in}}%
\pgfpathlineto{\pgfqpoint{2.429012in}{1.424174in}}%
\pgfpathclose%
\pgfusepath{fill}%
\end{pgfscope}%
\begin{pgfscope}%
\pgfpathrectangle{\pgfqpoint{0.150000in}{0.150000in}}{\pgfqpoint{4.700000in}{3.450000in}}%
\pgfusepath{clip}%
\pgfsetbuttcap%
\pgfsetroundjoin%
\definecolor{currentfill}{rgb}{0.573974,0.626440,0.699893}%
\pgfsetfillcolor{currentfill}%
\pgfsetlinewidth{0.000000pt}%
\definecolor{currentstroke}{rgb}{0.000000,0.000000,0.000000}%
\pgfsetstrokecolor{currentstroke}%
\pgfsetdash{}{0pt}%
\pgfpathmoveto{\pgfqpoint{1.554193in}{2.062959in}}%
\pgfpathlineto{\pgfqpoint{1.623685in}{2.088689in}}%
\pgfpathlineto{\pgfqpoint{1.556308in}{2.162991in}}%
\pgfpathlineto{\pgfqpoint{1.486656in}{2.137174in}}%
\pgfpathclose%
\pgfusepath{fill}%
\end{pgfscope}%
\begin{pgfscope}%
\pgfpathrectangle{\pgfqpoint{0.150000in}{0.150000in}}{\pgfqpoint{4.700000in}{3.450000in}}%
\pgfusepath{clip}%
\pgfsetbuttcap%
\pgfsetroundjoin%
\definecolor{currentfill}{rgb}{0.940916,0.948192,0.958379}%
\pgfsetfillcolor{currentfill}%
\pgfsetlinewidth{0.000000pt}%
\definecolor{currentstroke}{rgb}{0.000000,0.000000,0.000000}%
\pgfsetstrokecolor{currentstroke}%
\pgfsetdash{}{0pt}%
\pgfpathmoveto{\pgfqpoint{2.294134in}{1.449658in}}%
\pgfpathlineto{\pgfqpoint{2.361998in}{1.472679in}}%
\pgfpathlineto{\pgfqpoint{2.295140in}{1.521070in}}%
\pgfpathlineto{\pgfqpoint{2.227183in}{1.498200in}}%
\pgfpathclose%
\pgfusepath{fill}%
\end{pgfscope}%
\begin{pgfscope}%
\pgfpathrectangle{\pgfqpoint{0.150000in}{0.150000in}}{\pgfqpoint{4.700000in}{3.450000in}}%
\pgfusepath{clip}%
\pgfsetbuttcap%
\pgfsetroundjoin%
\definecolor{currentfill}{rgb}{0.524219,0.582812,0.664844}%
\pgfsetfillcolor{currentfill}%
\pgfsetlinewidth{0.000000pt}%
\definecolor{currentstroke}{rgb}{0.000000,0.000000,0.000000}%
\pgfsetstrokecolor{currentstroke}%
\pgfsetdash{}{0pt}%
\pgfpathmoveto{\pgfqpoint{1.486656in}{2.137174in}}%
\pgfpathlineto{\pgfqpoint{1.556308in}{2.162991in}}%
\pgfpathlineto{\pgfqpoint{1.488898in}{2.237328in}}%
\pgfpathlineto{\pgfqpoint{1.419087in}{2.211423in}}%
\pgfpathclose%
\pgfusepath{fill}%
\end{pgfscope}%
\begin{pgfscope}%
\pgfpathrectangle{\pgfqpoint{0.150000in}{0.150000in}}{\pgfqpoint{4.700000in}{3.450000in}}%
\pgfusepath{clip}%
\pgfsetbuttcap%
\pgfsetroundjoin%
\definecolor{currentfill}{rgb}{0.891161,0.904565,0.923330}%
\pgfsetfillcolor{currentfill}%
\pgfsetlinewidth{0.000000pt}%
\definecolor{currentstroke}{rgb}{0.000000,0.000000,0.000000}%
\pgfsetstrokecolor{currentstroke}%
\pgfsetdash{}{0pt}%
\pgfpathmoveto{\pgfqpoint{2.091990in}{1.523793in}}%
\pgfpathlineto{\pgfqpoint{2.160389in}{1.546629in}}%
\pgfpathlineto{\pgfqpoint{2.093751in}{1.594945in}}%
\pgfpathlineto{\pgfqpoint{2.025260in}{1.572260in}}%
\pgfpathclose%
\pgfusepath{fill}%
\end{pgfscope}%
\begin{pgfscope}%
\pgfpathrectangle{\pgfqpoint{0.150000in}{0.150000in}}{\pgfqpoint{4.700000in}{3.450000in}}%
\pgfusepath{clip}%
\pgfsetbuttcap%
\pgfsetroundjoin%
\definecolor{currentfill}{rgb}{0.480683,0.544638,0.634176}%
\pgfsetfillcolor{currentfill}%
\pgfsetlinewidth{0.000000pt}%
\definecolor{currentstroke}{rgb}{0.000000,0.000000,0.000000}%
\pgfsetstrokecolor{currentstroke}%
\pgfsetdash{}{0pt}%
\pgfpathmoveto{\pgfqpoint{1.419087in}{2.211423in}}%
\pgfpathlineto{\pgfqpoint{1.488898in}{2.237328in}}%
\pgfpathlineto{\pgfqpoint{1.421456in}{2.311702in}}%
\pgfpathlineto{\pgfqpoint{1.351479in}{2.285835in}}%
\pgfpathclose%
\pgfusepath{fill}%
\end{pgfscope}%
\begin{pgfscope}%
\pgfpathrectangle{\pgfqpoint{0.150000in}{0.150000in}}{\pgfqpoint{4.700000in}{3.450000in}}%
\pgfusepath{clip}%
\pgfsetbuttcap%
\pgfsetroundjoin%
\definecolor{currentfill}{rgb}{0.944914,0.900046,0.903600}%
\pgfsetfillcolor{currentfill}%
\pgfsetlinewidth{0.000000pt}%
\definecolor{currentstroke}{rgb}{0.000000,0.000000,0.000000}%
\pgfsetstrokecolor{currentstroke}%
\pgfsetdash{}{0pt}%
\pgfpathmoveto{\pgfqpoint{2.967776in}{1.178573in}}%
\pgfpathlineto{\pgfqpoint{3.033941in}{1.202302in}}%
\pgfpathlineto{\pgfqpoint{2.966266in}{1.251032in}}%
\pgfpathlineto{\pgfqpoint{2.900005in}{1.227456in}}%
\pgfpathclose%
\pgfusepath{fill}%
\end{pgfscope}%
\begin{pgfscope}%
\pgfpathrectangle{\pgfqpoint{0.150000in}{0.150000in}}{\pgfqpoint{4.700000in}{3.450000in}}%
\pgfusepath{clip}%
\pgfsetbuttcap%
\pgfsetroundjoin%
\definecolor{currentfill}{rgb}{0.430928,0.501011,0.599127}%
\pgfsetfillcolor{currentfill}%
\pgfsetlinewidth{0.000000pt}%
\definecolor{currentstroke}{rgb}{0.000000,0.000000,0.000000}%
\pgfsetstrokecolor{currentstroke}%
\pgfsetdash{}{0pt}%
\pgfpathmoveto{\pgfqpoint{1.351479in}{2.285835in}}%
\pgfpathlineto{\pgfqpoint{1.421456in}{2.311702in}}%
\pgfpathlineto{\pgfqpoint{1.353980in}{2.386113in}}%
\pgfpathlineto{\pgfqpoint{1.283848in}{2.360154in}}%
\pgfpathclose%
\pgfusepath{fill}%
\end{pgfscope}%
\begin{pgfscope}%
\pgfpathrectangle{\pgfqpoint{0.150000in}{0.150000in}}{\pgfqpoint{4.700000in}{3.450000in}}%
\pgfusepath{clip}%
\pgfsetbuttcap%
\pgfsetroundjoin%
\definecolor{currentfill}{rgb}{0.971507,0.948300,0.950138}%
\pgfsetfillcolor{currentfill}%
\pgfsetlinewidth{0.000000pt}%
\definecolor{currentstroke}{rgb}{0.000000,0.000000,0.000000}%
\pgfsetstrokecolor{currentstroke}%
\pgfsetdash{}{0pt}%
\pgfpathmoveto{\pgfqpoint{2.765692in}{1.252682in}}%
\pgfpathlineto{\pgfqpoint{2.832391in}{1.276225in}}%
\pgfpathlineto{\pgfqpoint{2.764936in}{1.324880in}}%
\pgfpathlineto{\pgfqpoint{2.698141in}{1.301489in}}%
\pgfpathclose%
\pgfusepath{fill}%
\end{pgfscope}%
\begin{pgfscope}%
\pgfpathrectangle{\pgfqpoint{0.150000in}{0.150000in}}{\pgfqpoint{4.700000in}{3.450000in}}%
\pgfusepath{clip}%
\pgfsetbuttcap%
\pgfsetroundjoin%
\definecolor{currentfill}{rgb}{0.387393,0.462837,0.568459}%
\pgfsetfillcolor{currentfill}%
\pgfsetlinewidth{0.000000pt}%
\definecolor{currentstroke}{rgb}{0.000000,0.000000,0.000000}%
\pgfsetstrokecolor{currentstroke}%
\pgfsetdash{}{0pt}%
\pgfpathmoveto{\pgfqpoint{1.283848in}{2.360154in}}%
\pgfpathlineto{\pgfqpoint{1.353980in}{2.386113in}}%
\pgfpathlineto{\pgfqpoint{1.286472in}{2.460560in}}%
\pgfpathlineto{\pgfqpoint{1.216185in}{2.434507in}}%
\pgfpathclose%
\pgfusepath{fill}%
\end{pgfscope}%
\begin{pgfscope}%
\pgfpathrectangle{\pgfqpoint{0.150000in}{0.150000in}}{\pgfqpoint{4.700000in}{3.450000in}}%
\pgfusepath{clip}%
\pgfsetbuttcap%
\pgfsetroundjoin%
\definecolor{currentfill}{rgb}{0.996890,0.997273,0.997809}%
\pgfsetfillcolor{currentfill}%
\pgfsetlinewidth{0.000000pt}%
\definecolor{currentstroke}{rgb}{0.000000,0.000000,0.000000}%
\pgfsetstrokecolor{currentstroke}%
\pgfsetdash{}{0pt}%
\pgfpathmoveto{\pgfqpoint{2.563514in}{1.326824in}}%
\pgfpathlineto{\pgfqpoint{2.630749in}{1.350183in}}%
\pgfpathlineto{\pgfqpoint{2.563514in}{1.398762in}}%
\pgfpathlineto{\pgfqpoint{2.496184in}{1.375556in}}%
\pgfpathclose%
\pgfusepath{fill}%
\end{pgfscope}%
\begin{pgfscope}%
\pgfpathrectangle{\pgfqpoint{0.150000in}{0.150000in}}{\pgfqpoint{4.700000in}{3.450000in}}%
\pgfusepath{clip}%
\pgfsetbuttcap%
\pgfsetroundjoin%
\definecolor{currentfill}{rgb}{0.835187,0.855484,0.883900}%
\pgfsetfillcolor{currentfill}%
\pgfsetlinewidth{0.000000pt}%
\definecolor{currentstroke}{rgb}{0.000000,0.000000,0.000000}%
\pgfsetstrokecolor{currentstroke}%
\pgfsetdash{}{0pt}%
\pgfpathmoveto{\pgfqpoint{1.889752in}{1.597963in}}%
\pgfpathlineto{\pgfqpoint{1.958687in}{1.620614in}}%
\pgfpathlineto{\pgfqpoint{1.891415in}{1.692275in}}%
\pgfpathlineto{\pgfqpoint{1.822429in}{1.666577in}}%
\pgfpathclose%
\pgfusepath{fill}%
\end{pgfscope}%
\begin{pgfscope}%
\pgfpathrectangle{\pgfqpoint{0.150000in}{0.150000in}}{\pgfqpoint{4.700000in}{3.450000in}}%
\pgfusepath{clip}%
\pgfsetbuttcap%
\pgfsetroundjoin%
\definecolor{currentfill}{rgb}{0.343857,0.424663,0.537791}%
\pgfsetfillcolor{currentfill}%
\pgfsetlinewidth{0.000000pt}%
\definecolor{currentstroke}{rgb}{0.000000,0.000000,0.000000}%
\pgfsetstrokecolor{currentstroke}%
\pgfsetdash{}{0pt}%
\pgfpathmoveto{\pgfqpoint{1.216185in}{2.434507in}}%
\pgfpathlineto{\pgfqpoint{1.286472in}{2.460560in}}%
\pgfpathlineto{\pgfqpoint{1.218930in}{2.535043in}}%
\pgfpathlineto{\pgfqpoint{1.148491in}{2.508894in}}%
\pgfpathclose%
\pgfusepath{fill}%
\end{pgfscope}%
\begin{pgfscope}%
\pgfpathrectangle{\pgfqpoint{0.150000in}{0.150000in}}{\pgfqpoint{4.700000in}{3.450000in}}%
\pgfusepath{clip}%
\pgfsetbuttcap%
\pgfsetroundjoin%
\definecolor{currentfill}{rgb}{0.953355,0.959099,0.967142}%
\pgfsetfillcolor{currentfill}%
\pgfsetlinewidth{0.000000pt}%
\definecolor{currentstroke}{rgb}{0.000000,0.000000,0.000000}%
\pgfsetstrokecolor{currentstroke}%
\pgfsetdash{}{0pt}%
\pgfpathmoveto{\pgfqpoint{2.361241in}{1.401001in}}%
\pgfpathlineto{\pgfqpoint{2.429012in}{1.424174in}}%
\pgfpathlineto{\pgfqpoint{2.361998in}{1.472679in}}%
\pgfpathlineto{\pgfqpoint{2.294134in}{1.449658in}}%
\pgfpathclose%
\pgfusepath{fill}%
\end{pgfscope}%
\begin{pgfscope}%
\pgfpathrectangle{\pgfqpoint{0.150000in}{0.150000in}}{\pgfqpoint{4.700000in}{3.450000in}}%
\pgfusepath{clip}%
\pgfsetbuttcap%
\pgfsetroundjoin%
\definecolor{currentfill}{rgb}{0.791651,0.817310,0.853232}%
\pgfsetfillcolor{currentfill}%
\pgfsetlinewidth{0.000000pt}%
\definecolor{currentstroke}{rgb}{0.000000,0.000000,0.000000}%
\pgfsetstrokecolor{currentstroke}%
\pgfsetdash{}{0pt}%
\pgfpathmoveto{\pgfqpoint{1.822429in}{1.666577in}}%
\pgfpathlineto{\pgfqpoint{1.891415in}{1.692275in}}%
\pgfpathlineto{\pgfqpoint{1.824034in}{1.766318in}}%
\pgfpathlineto{\pgfqpoint{1.754876in}{1.740667in}}%
\pgfpathclose%
\pgfusepath{fill}%
\end{pgfscope}%
\begin{pgfscope}%
\pgfpathrectangle{\pgfqpoint{0.150000in}{0.150000in}}{\pgfqpoint{4.700000in}{3.450000in}}%
\pgfusepath{clip}%
\pgfsetbuttcap%
\pgfsetroundjoin%
\definecolor{currentfill}{rgb}{0.909819,0.920925,0.936474}%
\pgfsetfillcolor{currentfill}%
\pgfsetlinewidth{0.000000pt}%
\definecolor{currentstroke}{rgb}{0.000000,0.000000,0.000000}%
\pgfsetstrokecolor{currentstroke}%
\pgfsetdash{}{0pt}%
\pgfpathmoveto{\pgfqpoint{2.158876in}{1.475213in}}%
\pgfpathlineto{\pgfqpoint{2.227183in}{1.498200in}}%
\pgfpathlineto{\pgfqpoint{2.160389in}{1.546629in}}%
\pgfpathlineto{\pgfqpoint{2.091990in}{1.523793in}}%
\pgfpathclose%
\pgfusepath{fill}%
\end{pgfscope}%
\begin{pgfscope}%
\pgfpathrectangle{\pgfqpoint{0.150000in}{0.150000in}}{\pgfqpoint{4.700000in}{3.450000in}}%
\pgfusepath{clip}%
\pgfsetbuttcap%
\pgfsetroundjoin%
\definecolor{currentfill}{rgb}{0.741896,0.773683,0.818183}%
\pgfsetfillcolor{currentfill}%
\pgfsetlinewidth{0.000000pt}%
\definecolor{currentstroke}{rgb}{0.000000,0.000000,0.000000}%
\pgfsetstrokecolor{currentstroke}%
\pgfsetdash{}{0pt}%
\pgfpathmoveto{\pgfqpoint{1.754876in}{1.740667in}}%
\pgfpathlineto{\pgfqpoint{1.824034in}{1.766318in}}%
\pgfpathlineto{\pgfqpoint{1.756622in}{1.840395in}}%
\pgfpathlineto{\pgfqpoint{1.687299in}{1.814665in}}%
\pgfpathclose%
\pgfusepath{fill}%
\end{pgfscope}%
\begin{pgfscope}%
\pgfpathrectangle{\pgfqpoint{0.150000in}{0.150000in}}{\pgfqpoint{4.700000in}{3.450000in}}%
\pgfusepath{clip}%
\pgfsetbuttcap%
\pgfsetroundjoin%
\definecolor{currentfill}{rgb}{0.294102,0.381036,0.502742}%
\pgfsetfillcolor{currentfill}%
\pgfsetlinewidth{0.000000pt}%
\definecolor{currentstroke}{rgb}{0.000000,0.000000,0.000000}%
\pgfsetstrokecolor{currentstroke}%
\pgfsetdash{}{0pt}%
\pgfpathmoveto{\pgfqpoint{1.148491in}{2.508894in}}%
\pgfpathlineto{\pgfqpoint{1.218930in}{2.535043in}}%
\pgfpathlineto{\pgfqpoint{1.151356in}{2.609563in}}%
\pgfpathlineto{\pgfqpoint{1.080755in}{2.583446in}}%
\pgfpathclose%
\pgfusepath{fill}%
\end{pgfscope}%
\begin{pgfscope}%
\pgfpathrectangle{\pgfqpoint{0.150000in}{0.150000in}}{\pgfqpoint{4.700000in}{3.450000in}}%
\pgfusepath{clip}%
\pgfsetbuttcap%
\pgfsetroundjoin%
\definecolor{currentfill}{rgb}{0.698361,0.735509,0.787515}%
\pgfsetfillcolor{currentfill}%
\pgfsetlinewidth{0.000000pt}%
\definecolor{currentstroke}{rgb}{0.000000,0.000000,0.000000}%
\pgfsetstrokecolor{currentstroke}%
\pgfsetdash{}{0pt}%
\pgfpathmoveto{\pgfqpoint{1.687299in}{1.814665in}}%
\pgfpathlineto{\pgfqpoint{1.756622in}{1.840395in}}%
\pgfpathlineto{\pgfqpoint{1.689179in}{1.914506in}}%
\pgfpathlineto{\pgfqpoint{1.619685in}{1.888822in}}%
\pgfpathclose%
\pgfusepath{fill}%
\end{pgfscope}%
\begin{pgfscope}%
\pgfpathrectangle{\pgfqpoint{0.150000in}{0.150000in}}{\pgfqpoint{4.700000in}{3.450000in}}%
\pgfusepath{clip}%
\pgfsetbuttcap%
\pgfsetroundjoin%
\definecolor{currentfill}{rgb}{0.860064,0.877298,0.901425}%
\pgfsetfillcolor{currentfill}%
\pgfsetlinewidth{0.000000pt}%
\definecolor{currentstroke}{rgb}{0.000000,0.000000,0.000000}%
\pgfsetstrokecolor{currentstroke}%
\pgfsetdash{}{0pt}%
\pgfpathmoveto{\pgfqpoint{1.956416in}{1.549459in}}%
\pgfpathlineto{\pgfqpoint{2.025260in}{1.572260in}}%
\pgfpathlineto{\pgfqpoint{1.958687in}{1.620614in}}%
\pgfpathlineto{\pgfqpoint{1.889752in}{1.597963in}}%
\pgfpathclose%
\pgfusepath{fill}%
\end{pgfscope}%
\begin{pgfscope}%
\pgfpathrectangle{\pgfqpoint{0.150000in}{0.150000in}}{\pgfqpoint{4.700000in}{3.450000in}}%
\pgfusepath{clip}%
\pgfsetbuttcap%
\pgfsetroundjoin%
\definecolor{currentfill}{rgb}{0.250567,0.342862,0.472074}%
\pgfsetfillcolor{currentfill}%
\pgfsetlinewidth{0.000000pt}%
\definecolor{currentstroke}{rgb}{0.000000,0.000000,0.000000}%
\pgfsetstrokecolor{currentstroke}%
\pgfsetdash{}{0pt}%
\pgfpathmoveto{\pgfqpoint{1.080755in}{2.583446in}}%
\pgfpathlineto{\pgfqpoint{1.151356in}{2.609563in}}%
\pgfpathlineto{\pgfqpoint{1.083748in}{2.684119in}}%
\pgfpathlineto{\pgfqpoint{1.012998in}{2.657903in}}%
\pgfpathclose%
\pgfusepath{fill}%
\end{pgfscope}%
\begin{pgfscope}%
\pgfpathrectangle{\pgfqpoint{0.150000in}{0.150000in}}{\pgfqpoint{4.700000in}{3.450000in}}%
\pgfusepath{clip}%
\pgfsetbuttcap%
\pgfsetroundjoin%
\definecolor{currentfill}{rgb}{0.654825,0.697335,0.756847}%
\pgfsetfillcolor{currentfill}%
\pgfsetlinewidth{0.000000pt}%
\definecolor{currentstroke}{rgb}{0.000000,0.000000,0.000000}%
\pgfsetstrokecolor{currentstroke}%
\pgfsetdash{}{0pt}%
\pgfpathmoveto{\pgfqpoint{1.619685in}{1.888822in}}%
\pgfpathlineto{\pgfqpoint{1.689179in}{1.914506in}}%
\pgfpathlineto{\pgfqpoint{1.621699in}{1.988778in}}%
\pgfpathlineto{\pgfqpoint{1.552048in}{1.962885in}}%
\pgfpathclose%
\pgfusepath{fill}%
\end{pgfscope}%
\begin{pgfscope}%
\pgfpathrectangle{\pgfqpoint{0.150000in}{0.150000in}}{\pgfqpoint{4.700000in}{3.450000in}}%
\pgfusepath{clip}%
\pgfsetbuttcap%
\pgfsetroundjoin%
\definecolor{currentfill}{rgb}{0.963909,0.934513,0.936841}%
\pgfsetfillcolor{currentfill}%
\pgfsetlinewidth{0.000000pt}%
\definecolor{currentstroke}{rgb}{0.000000,0.000000,0.000000}%
\pgfsetstrokecolor{currentstroke}%
\pgfsetdash{}{0pt}%
\pgfpathmoveto{\pgfqpoint{2.833401in}{1.203759in}}%
\pgfpathlineto{\pgfqpoint{2.900005in}{1.227456in}}%
\pgfpathlineto{\pgfqpoint{2.832391in}{1.276225in}}%
\pgfpathlineto{\pgfqpoint{2.765692in}{1.252682in}}%
\pgfpathclose%
\pgfusepath{fill}%
\end{pgfscope}%
\begin{pgfscope}%
\pgfpathrectangle{\pgfqpoint{0.150000in}{0.150000in}}{\pgfqpoint{4.700000in}{3.450000in}}%
\pgfusepath{clip}%
\pgfsetbuttcap%
\pgfsetroundjoin%
\definecolor{currentfill}{rgb}{0.605070,0.653707,0.721798}%
\pgfsetfillcolor{currentfill}%
\pgfsetlinewidth{0.000000pt}%
\definecolor{currentstroke}{rgb}{0.000000,0.000000,0.000000}%
\pgfsetstrokecolor{currentstroke}%
\pgfsetdash{}{0pt}%
\pgfpathmoveto{\pgfqpoint{1.552048in}{1.962885in}}%
\pgfpathlineto{\pgfqpoint{1.621699in}{1.988778in}}%
\pgfpathlineto{\pgfqpoint{1.554193in}{2.062959in}}%
\pgfpathlineto{\pgfqpoint{1.484374in}{2.037108in}}%
\pgfpathclose%
\pgfusepath{fill}%
\end{pgfscope}%
\begin{pgfscope}%
\pgfpathrectangle{\pgfqpoint{0.150000in}{0.150000in}}{\pgfqpoint{4.700000in}{3.450000in}}%
\pgfusepath{clip}%
\pgfsetbuttcap%
\pgfsetroundjoin%
\definecolor{currentfill}{rgb}{0.990502,0.982767,0.983379}%
\pgfsetfillcolor{currentfill}%
\pgfsetlinewidth{0.000000pt}%
\definecolor{currentstroke}{rgb}{0.000000,0.000000,0.000000}%
\pgfsetstrokecolor{currentstroke}%
\pgfsetdash{}{0pt}%
\pgfpathmoveto{\pgfqpoint{2.631001in}{1.277978in}}%
\pgfpathlineto{\pgfqpoint{2.698141in}{1.301489in}}%
\pgfpathlineto{\pgfqpoint{2.630749in}{1.350183in}}%
\pgfpathlineto{\pgfqpoint{2.563514in}{1.326824in}}%
\pgfpathclose%
\pgfusepath{fill}%
\end{pgfscope}%
\begin{pgfscope}%
\pgfpathrectangle{\pgfqpoint{0.150000in}{0.150000in}}{\pgfqpoint{4.700000in}{3.450000in}}%
\pgfusepath{clip}%
\pgfsetbuttcap%
\pgfsetroundjoin%
\definecolor{currentfill}{rgb}{0.561535,0.615533,0.691131}%
\pgfsetfillcolor{currentfill}%
\pgfsetlinewidth{0.000000pt}%
\definecolor{currentstroke}{rgb}{0.000000,0.000000,0.000000}%
\pgfsetstrokecolor{currentstroke}%
\pgfsetdash{}{0pt}%
\pgfpathmoveto{\pgfqpoint{1.484374in}{2.037108in}}%
\pgfpathlineto{\pgfqpoint{1.554193in}{2.062959in}}%
\pgfpathlineto{\pgfqpoint{1.486656in}{2.137174in}}%
\pgfpathlineto{\pgfqpoint{1.416678in}{2.111236in}}%
\pgfpathclose%
\pgfusepath{fill}%
\end{pgfscope}%
\begin{pgfscope}%
\pgfpathrectangle{\pgfqpoint{0.150000in}{0.150000in}}{\pgfqpoint{4.700000in}{3.450000in}}%
\pgfusepath{clip}%
\pgfsetbuttcap%
\pgfsetroundjoin%
\definecolor{currentfill}{rgb}{0.972013,0.975460,0.980285}%
\pgfsetfillcolor{currentfill}%
\pgfsetlinewidth{0.000000pt}%
\definecolor{currentstroke}{rgb}{0.000000,0.000000,0.000000}%
\pgfsetstrokecolor{currentstroke}%
\pgfsetdash{}{0pt}%
\pgfpathmoveto{\pgfqpoint{2.428507in}{1.352230in}}%
\pgfpathlineto{\pgfqpoint{2.496184in}{1.375556in}}%
\pgfpathlineto{\pgfqpoint{2.429012in}{1.424174in}}%
\pgfpathlineto{\pgfqpoint{2.361241in}{1.401001in}}%
\pgfpathclose%
\pgfusepath{fill}%
\end{pgfscope}%
\begin{pgfscope}%
\pgfpathrectangle{\pgfqpoint{0.150000in}{0.150000in}}{\pgfqpoint{4.700000in}{3.450000in}}%
\pgfusepath{clip}%
\pgfsetbuttcap%
\pgfsetroundjoin%
\definecolor{currentfill}{rgb}{0.517999,0.577359,0.660463}%
\pgfsetfillcolor{currentfill}%
\pgfsetlinewidth{0.000000pt}%
\definecolor{currentstroke}{rgb}{0.000000,0.000000,0.000000}%
\pgfsetstrokecolor{currentstroke}%
\pgfsetdash{}{0pt}%
\pgfpathmoveto{\pgfqpoint{1.416678in}{2.111236in}}%
\pgfpathlineto{\pgfqpoint{1.486656in}{2.137174in}}%
\pgfpathlineto{\pgfqpoint{1.419087in}{2.211423in}}%
\pgfpathlineto{\pgfqpoint{1.348943in}{2.185525in}}%
\pgfpathclose%
\pgfusepath{fill}%
\end{pgfscope}%
\begin{pgfscope}%
\pgfpathrectangle{\pgfqpoint{0.150000in}{0.150000in}}{\pgfqpoint{4.700000in}{3.450000in}}%
\pgfusepath{clip}%
\pgfsetbuttcap%
\pgfsetroundjoin%
\definecolor{currentfill}{rgb}{0.922258,0.931832,0.945236}%
\pgfsetfillcolor{currentfill}%
\pgfsetlinewidth{0.000000pt}%
\definecolor{currentstroke}{rgb}{0.000000,0.000000,0.000000}%
\pgfsetstrokecolor{currentstroke}%
\pgfsetdash{}{0pt}%
\pgfpathmoveto{\pgfqpoint{2.225919in}{1.426518in}}%
\pgfpathlineto{\pgfqpoint{2.294134in}{1.449658in}}%
\pgfpathlineto{\pgfqpoint{2.227183in}{1.498200in}}%
\pgfpathlineto{\pgfqpoint{2.158876in}{1.475213in}}%
\pgfpathclose%
\pgfusepath{fill}%
\end{pgfscope}%
\begin{pgfscope}%
\pgfpathrectangle{\pgfqpoint{0.150000in}{0.150000in}}{\pgfqpoint{4.700000in}{3.450000in}}%
\pgfusepath{clip}%
\pgfsetbuttcap%
\pgfsetroundjoin%
\definecolor{currentfill}{rgb}{0.468244,0.533732,0.625414}%
\pgfsetfillcolor{currentfill}%
\pgfsetlinewidth{0.000000pt}%
\definecolor{currentstroke}{rgb}{0.000000,0.000000,0.000000}%
\pgfsetstrokecolor{currentstroke}%
\pgfsetdash{}{0pt}%
\pgfpathmoveto{\pgfqpoint{1.348943in}{2.185525in}}%
\pgfpathlineto{\pgfqpoint{1.419087in}{2.211423in}}%
\pgfpathlineto{\pgfqpoint{1.351479in}{2.285835in}}%
\pgfpathlineto{\pgfqpoint{1.281187in}{2.259719in}}%
\pgfpathclose%
\pgfusepath{fill}%
\end{pgfscope}%
\begin{pgfscope}%
\pgfpathrectangle{\pgfqpoint{0.150000in}{0.150000in}}{\pgfqpoint{4.700000in}{3.450000in}}%
\pgfusepath{clip}%
\pgfsetbuttcap%
\pgfsetroundjoin%
\definecolor{currentfill}{rgb}{0.878722,0.893658,0.914568}%
\pgfsetfillcolor{currentfill}%
\pgfsetlinewidth{0.000000pt}%
\definecolor{currentstroke}{rgb}{0.000000,0.000000,0.000000}%
\pgfsetstrokecolor{currentstroke}%
\pgfsetdash{}{0pt}%
\pgfpathmoveto{\pgfqpoint{2.023237in}{1.500840in}}%
\pgfpathlineto{\pgfqpoint{2.091990in}{1.523793in}}%
\pgfpathlineto{\pgfqpoint{2.025260in}{1.572260in}}%
\pgfpathlineto{\pgfqpoint{1.956416in}{1.549459in}}%
\pgfpathclose%
\pgfusepath{fill}%
\end{pgfscope}%
\begin{pgfscope}%
\pgfpathrectangle{\pgfqpoint{0.150000in}{0.150000in}}{\pgfqpoint{4.700000in}{3.450000in}}%
\pgfusepath{clip}%
\pgfsetbuttcap%
\pgfsetroundjoin%
\definecolor{currentfill}{rgb}{0.424709,0.495558,0.594746}%
\pgfsetfillcolor{currentfill}%
\pgfsetlinewidth{0.000000pt}%
\definecolor{currentstroke}{rgb}{0.000000,0.000000,0.000000}%
\pgfsetstrokecolor{currentstroke}%
\pgfsetdash{}{0pt}%
\pgfpathmoveto{\pgfqpoint{1.281187in}{2.259719in}}%
\pgfpathlineto{\pgfqpoint{1.351479in}{2.285835in}}%
\pgfpathlineto{\pgfqpoint{1.283848in}{2.360154in}}%
\pgfpathlineto{\pgfqpoint{1.213392in}{2.334075in}}%
\pgfpathclose%
\pgfusepath{fill}%
\end{pgfscope}%
\begin{pgfscope}%
\pgfpathrectangle{\pgfqpoint{0.150000in}{0.150000in}}{\pgfqpoint{4.700000in}{3.450000in}}%
\pgfusepath{clip}%
\pgfsetbuttcap%
\pgfsetroundjoin%
\definecolor{currentfill}{rgb}{0.952512,0.913833,0.916896}%
\pgfsetfillcolor{currentfill}%
\pgfsetlinewidth{0.000000pt}%
\definecolor{currentstroke}{rgb}{0.000000,0.000000,0.000000}%
\pgfsetstrokecolor{currentstroke}%
\pgfsetdash{}{0pt}%
\pgfpathmoveto{\pgfqpoint{2.901270in}{1.154722in}}%
\pgfpathlineto{\pgfqpoint{2.967776in}{1.178573in}}%
\pgfpathlineto{\pgfqpoint{2.900005in}{1.227456in}}%
\pgfpathlineto{\pgfqpoint{2.833401in}{1.203759in}}%
\pgfpathclose%
\pgfusepath{fill}%
\end{pgfscope}%
\begin{pgfscope}%
\pgfpathrectangle{\pgfqpoint{0.150000in}{0.150000in}}{\pgfqpoint{4.700000in}{3.450000in}}%
\pgfusepath{clip}%
\pgfsetbuttcap%
\pgfsetroundjoin%
\definecolor{currentfill}{rgb}{0.381173,0.457384,0.564078}%
\pgfsetfillcolor{currentfill}%
\pgfsetlinewidth{0.000000pt}%
\definecolor{currentstroke}{rgb}{0.000000,0.000000,0.000000}%
\pgfsetstrokecolor{currentstroke}%
\pgfsetdash{}{0pt}%
\pgfpathmoveto{\pgfqpoint{1.213392in}{2.334075in}}%
\pgfpathlineto{\pgfqpoint{1.283848in}{2.360154in}}%
\pgfpathlineto{\pgfqpoint{1.216185in}{2.434507in}}%
\pgfpathlineto{\pgfqpoint{1.145576in}{2.408334in}}%
\pgfpathclose%
\pgfusepath{fill}%
\end{pgfscope}%
\begin{pgfscope}%
\pgfpathrectangle{\pgfqpoint{0.150000in}{0.150000in}}{\pgfqpoint{4.700000in}{3.450000in}}%
\pgfusepath{clip}%
\pgfsetbuttcap%
\pgfsetroundjoin%
\definecolor{currentfill}{rgb}{0.982904,0.968980,0.970083}%
\pgfsetfillcolor{currentfill}%
\pgfsetlinewidth{0.000000pt}%
\definecolor{currentstroke}{rgb}{0.000000,0.000000,0.000000}%
\pgfsetstrokecolor{currentstroke}%
\pgfsetdash{}{0pt}%
\pgfpathmoveto{\pgfqpoint{2.698647in}{1.229016in}}%
\pgfpathlineto{\pgfqpoint{2.765692in}{1.252682in}}%
\pgfpathlineto{\pgfqpoint{2.698141in}{1.301489in}}%
\pgfpathlineto{\pgfqpoint{2.631001in}{1.277978in}}%
\pgfpathclose%
\pgfusepath{fill}%
\end{pgfscope}%
\begin{pgfscope}%
\pgfpathrectangle{\pgfqpoint{0.150000in}{0.150000in}}{\pgfqpoint{4.700000in}{3.450000in}}%
\pgfusepath{clip}%
\pgfsetbuttcap%
\pgfsetroundjoin%
\definecolor{currentfill}{rgb}{0.331419,0.413756,0.529029}%
\pgfsetfillcolor{currentfill}%
\pgfsetlinewidth{0.000000pt}%
\definecolor{currentstroke}{rgb}{0.000000,0.000000,0.000000}%
\pgfsetstrokecolor{currentstroke}%
\pgfsetdash{}{0pt}%
\pgfpathmoveto{\pgfqpoint{1.145576in}{2.408334in}}%
\pgfpathlineto{\pgfqpoint{1.216185in}{2.434507in}}%
\pgfpathlineto{\pgfqpoint{1.148491in}{2.508894in}}%
\pgfpathlineto{\pgfqpoint{1.077720in}{2.482756in}}%
\pgfpathclose%
\pgfusepath{fill}%
\end{pgfscope}%
\begin{pgfscope}%
\pgfpathrectangle{\pgfqpoint{0.150000in}{0.150000in}}{\pgfqpoint{4.700000in}{3.450000in}}%
\pgfusepath{clip}%
\pgfsetbuttcap%
\pgfsetroundjoin%
\definecolor{currentfill}{rgb}{0.984452,0.986366,0.989047}%
\pgfsetfillcolor{currentfill}%
\pgfsetlinewidth{0.000000pt}%
\definecolor{currentstroke}{rgb}{0.000000,0.000000,0.000000}%
\pgfsetstrokecolor{currentstroke}%
\pgfsetdash{}{0pt}%
\pgfpathmoveto{\pgfqpoint{2.495931in}{1.303345in}}%
\pgfpathlineto{\pgfqpoint{2.563514in}{1.326824in}}%
\pgfpathlineto{\pgfqpoint{2.496184in}{1.375556in}}%
\pgfpathlineto{\pgfqpoint{2.428507in}{1.352230in}}%
\pgfpathclose%
\pgfusepath{fill}%
\end{pgfscope}%
\begin{pgfscope}%
\pgfpathrectangle{\pgfqpoint{0.150000in}{0.150000in}}{\pgfqpoint{4.700000in}{3.450000in}}%
\pgfusepath{clip}%
\pgfsetbuttcap%
\pgfsetroundjoin%
\definecolor{currentfill}{rgb}{0.822748,0.844577,0.875138}%
\pgfsetfillcolor{currentfill}%
\pgfsetlinewidth{0.000000pt}%
\definecolor{currentstroke}{rgb}{0.000000,0.000000,0.000000}%
\pgfsetstrokecolor{currentstroke}%
\pgfsetdash{}{0pt}%
\pgfpathmoveto{\pgfqpoint{1.820461in}{1.575196in}}%
\pgfpathlineto{\pgfqpoint{1.889752in}{1.597963in}}%
\pgfpathlineto{\pgfqpoint{1.822429in}{1.666577in}}%
\pgfpathlineto{\pgfqpoint{1.753106in}{1.640880in}}%
\pgfpathclose%
\pgfusepath{fill}%
\end{pgfscope}%
\begin{pgfscope}%
\pgfpathrectangle{\pgfqpoint{0.150000in}{0.150000in}}{\pgfqpoint{4.700000in}{3.450000in}}%
\pgfusepath{clip}%
\pgfsetbuttcap%
\pgfsetroundjoin%
\definecolor{currentfill}{rgb}{0.779213,0.806403,0.844470}%
\pgfsetfillcolor{currentfill}%
\pgfsetlinewidth{0.000000pt}%
\definecolor{currentstroke}{rgb}{0.000000,0.000000,0.000000}%
\pgfsetstrokecolor{currentstroke}%
\pgfsetdash{}{0pt}%
\pgfpathmoveto{\pgfqpoint{1.753106in}{1.640880in}}%
\pgfpathlineto{\pgfqpoint{1.822429in}{1.666577in}}%
\pgfpathlineto{\pgfqpoint{1.754876in}{1.740667in}}%
\pgfpathlineto{\pgfqpoint{1.685392in}{1.714769in}}%
\pgfpathclose%
\pgfusepath{fill}%
\end{pgfscope}%
\begin{pgfscope}%
\pgfpathrectangle{\pgfqpoint{0.150000in}{0.150000in}}{\pgfqpoint{4.700000in}{3.450000in}}%
\pgfusepath{clip}%
\pgfsetbuttcap%
\pgfsetroundjoin%
\definecolor{currentfill}{rgb}{0.287883,0.375582,0.498361}%
\pgfsetfillcolor{currentfill}%
\pgfsetlinewidth{0.000000pt}%
\definecolor{currentstroke}{rgb}{0.000000,0.000000,0.000000}%
\pgfsetstrokecolor{currentstroke}%
\pgfsetdash{}{0pt}%
\pgfpathmoveto{\pgfqpoint{1.077720in}{2.482756in}}%
\pgfpathlineto{\pgfqpoint{1.148491in}{2.508894in}}%
\pgfpathlineto{\pgfqpoint{1.080755in}{2.583446in}}%
\pgfpathlineto{\pgfqpoint{1.009845in}{2.557081in}}%
\pgfpathclose%
\pgfusepath{fill}%
\end{pgfscope}%
\begin{pgfscope}%
\pgfpathrectangle{\pgfqpoint{0.150000in}{0.150000in}}{\pgfqpoint{4.700000in}{3.450000in}}%
\pgfusepath{clip}%
\pgfsetbuttcap%
\pgfsetroundjoin%
\definecolor{currentfill}{rgb}{0.940916,0.948192,0.958379}%
\pgfsetfillcolor{currentfill}%
\pgfsetlinewidth{0.000000pt}%
\definecolor{currentstroke}{rgb}{0.000000,0.000000,0.000000}%
\pgfsetstrokecolor{currentstroke}%
\pgfsetdash{}{0pt}%
\pgfpathmoveto{\pgfqpoint{2.293120in}{1.377708in}}%
\pgfpathlineto{\pgfqpoint{2.361241in}{1.401001in}}%
\pgfpathlineto{\pgfqpoint{2.294134in}{1.449658in}}%
\pgfpathlineto{\pgfqpoint{2.225919in}{1.426518in}}%
\pgfpathclose%
\pgfusepath{fill}%
\end{pgfscope}%
\begin{pgfscope}%
\pgfpathrectangle{\pgfqpoint{0.150000in}{0.150000in}}{\pgfqpoint{4.700000in}{3.450000in}}%
\pgfusepath{clip}%
\pgfsetbuttcap%
\pgfsetroundjoin%
\definecolor{currentfill}{rgb}{0.735677,0.768229,0.813802}%
\pgfsetfillcolor{currentfill}%
\pgfsetlinewidth{0.000000pt}%
\definecolor{currentstroke}{rgb}{0.000000,0.000000,0.000000}%
\pgfsetstrokecolor{currentstroke}%
\pgfsetdash{}{0pt}%
\pgfpathmoveto{\pgfqpoint{1.685392in}{1.714769in}}%
\pgfpathlineto{\pgfqpoint{1.754876in}{1.740667in}}%
\pgfpathlineto{\pgfqpoint{1.687299in}{1.814665in}}%
\pgfpathlineto{\pgfqpoint{1.617644in}{1.788813in}}%
\pgfpathclose%
\pgfusepath{fill}%
\end{pgfscope}%
\begin{pgfscope}%
\pgfpathrectangle{\pgfqpoint{0.150000in}{0.150000in}}{\pgfqpoint{4.700000in}{3.450000in}}%
\pgfusepath{clip}%
\pgfsetbuttcap%
\pgfsetroundjoin%
\definecolor{currentfill}{rgb}{0.238128,0.331955,0.463312}%
\pgfsetfillcolor{currentfill}%
\pgfsetlinewidth{0.000000pt}%
\definecolor{currentstroke}{rgb}{0.000000,0.000000,0.000000}%
\pgfsetstrokecolor{currentstroke}%
\pgfsetdash{}{0pt}%
\pgfpathmoveto{\pgfqpoint{1.009845in}{2.557081in}}%
\pgfpathlineto{\pgfqpoint{1.080755in}{2.583446in}}%
\pgfpathlineto{\pgfqpoint{1.012998in}{2.657903in}}%
\pgfpathlineto{\pgfqpoint{0.941928in}{2.631569in}}%
\pgfpathclose%
\pgfusepath{fill}%
\end{pgfscope}%
\begin{pgfscope}%
\pgfpathrectangle{\pgfqpoint{0.150000in}{0.150000in}}{\pgfqpoint{4.700000in}{3.450000in}}%
\pgfusepath{clip}%
\pgfsetbuttcap%
\pgfsetroundjoin%
\definecolor{currentfill}{rgb}{0.891161,0.904565,0.923330}%
\pgfsetfillcolor{currentfill}%
\pgfsetlinewidth{0.000000pt}%
\definecolor{currentstroke}{rgb}{0.000000,0.000000,0.000000}%
\pgfsetstrokecolor{currentstroke}%
\pgfsetdash{}{0pt}%
\pgfpathmoveto{\pgfqpoint{2.090215in}{1.452106in}}%
\pgfpathlineto{\pgfqpoint{2.158876in}{1.475213in}}%
\pgfpathlineto{\pgfqpoint{2.091990in}{1.523793in}}%
\pgfpathlineto{\pgfqpoint{2.023237in}{1.500840in}}%
\pgfpathclose%
\pgfusepath{fill}%
\end{pgfscope}%
\begin{pgfscope}%
\pgfpathrectangle{\pgfqpoint{0.150000in}{0.150000in}}{\pgfqpoint{4.700000in}{3.450000in}}%
\pgfusepath{clip}%
\pgfsetbuttcap%
\pgfsetroundjoin%
\definecolor{currentfill}{rgb}{0.685922,0.724602,0.778753}%
\pgfsetfillcolor{currentfill}%
\pgfsetlinewidth{0.000000pt}%
\definecolor{currentstroke}{rgb}{0.000000,0.000000,0.000000}%
\pgfsetstrokecolor{currentstroke}%
\pgfsetdash{}{0pt}%
\pgfpathmoveto{\pgfqpoint{1.617644in}{1.788813in}}%
\pgfpathlineto{\pgfqpoint{1.687299in}{1.814665in}}%
\pgfpathlineto{\pgfqpoint{1.619685in}{1.888822in}}%
\pgfpathlineto{\pgfqpoint{1.549867in}{1.862889in}}%
\pgfpathclose%
\pgfusepath{fill}%
\end{pgfscope}%
\begin{pgfscope}%
\pgfpathrectangle{\pgfqpoint{0.150000in}{0.150000in}}{\pgfqpoint{4.700000in}{3.450000in}}%
\pgfusepath{clip}%
\pgfsetbuttcap%
\pgfsetroundjoin%
\definecolor{currentfill}{rgb}{0.847626,0.866391,0.892662}%
\pgfsetfillcolor{currentfill}%
\pgfsetlinewidth{0.000000pt}%
\definecolor{currentstroke}{rgb}{0.000000,0.000000,0.000000}%
\pgfsetstrokecolor{currentstroke}%
\pgfsetdash{}{0pt}%
\pgfpathmoveto{\pgfqpoint{1.887215in}{1.526539in}}%
\pgfpathlineto{\pgfqpoint{1.956416in}{1.549459in}}%
\pgfpathlineto{\pgfqpoint{1.889752in}{1.597963in}}%
\pgfpathlineto{\pgfqpoint{1.820461in}{1.575196in}}%
\pgfpathclose%
\pgfusepath{fill}%
\end{pgfscope}%
\begin{pgfscope}%
\pgfpathrectangle{\pgfqpoint{0.150000in}{0.150000in}}{\pgfqpoint{4.700000in}{3.450000in}}%
\pgfusepath{clip}%
\pgfsetbuttcap%
\pgfsetroundjoin%
\definecolor{currentfill}{rgb}{0.642387,0.686428,0.748085}%
\pgfsetfillcolor{currentfill}%
\pgfsetlinewidth{0.000000pt}%
\definecolor{currentstroke}{rgb}{0.000000,0.000000,0.000000}%
\pgfsetstrokecolor{currentstroke}%
\pgfsetdash{}{0pt}%
\pgfpathmoveto{\pgfqpoint{1.549867in}{1.862889in}}%
\pgfpathlineto{\pgfqpoint{1.619685in}{1.888822in}}%
\pgfpathlineto{\pgfqpoint{1.552048in}{1.962885in}}%
\pgfpathlineto{\pgfqpoint{1.482061in}{1.936996in}}%
\pgfpathclose%
\pgfusepath{fill}%
\end{pgfscope}%
\begin{pgfscope}%
\pgfpathrectangle{\pgfqpoint{0.150000in}{0.150000in}}{\pgfqpoint{4.700000in}{3.450000in}}%
\pgfusepath{clip}%
\pgfsetbuttcap%
\pgfsetroundjoin%
\definecolor{currentfill}{rgb}{0.598851,0.648254,0.717417}%
\pgfsetfillcolor{currentfill}%
\pgfsetlinewidth{0.000000pt}%
\definecolor{currentstroke}{rgb}{0.000000,0.000000,0.000000}%
\pgfsetstrokecolor{currentstroke}%
\pgfsetdash{}{0pt}%
\pgfpathmoveto{\pgfqpoint{1.482061in}{1.936996in}}%
\pgfpathlineto{\pgfqpoint{1.552048in}{1.962885in}}%
\pgfpathlineto{\pgfqpoint{1.484374in}{2.037108in}}%
\pgfpathlineto{\pgfqpoint{1.414227in}{2.011135in}}%
\pgfpathclose%
\pgfusepath{fill}%
\end{pgfscope}%
\begin{pgfscope}%
\pgfpathrectangle{\pgfqpoint{0.150000in}{0.150000in}}{\pgfqpoint{4.700000in}{3.450000in}}%
\pgfusepath{clip}%
\pgfsetbuttcap%
\pgfsetroundjoin%
\definecolor{currentfill}{rgb}{0.971507,0.948300,0.950138}%
\pgfsetfillcolor{currentfill}%
\pgfsetlinewidth{0.000000pt}%
\definecolor{currentstroke}{rgb}{0.000000,0.000000,0.000000}%
\pgfsetstrokecolor{currentstroke}%
\pgfsetdash{}{0pt}%
\pgfpathmoveto{\pgfqpoint{2.766453in}{1.179939in}}%
\pgfpathlineto{\pgfqpoint{2.833401in}{1.203759in}}%
\pgfpathlineto{\pgfqpoint{2.765692in}{1.252682in}}%
\pgfpathlineto{\pgfqpoint{2.698647in}{1.229016in}}%
\pgfpathclose%
\pgfusepath{fill}%
\end{pgfscope}%
\begin{pgfscope}%
\pgfpathrectangle{\pgfqpoint{0.150000in}{0.150000in}}{\pgfqpoint{4.700000in}{3.450000in}}%
\pgfusepath{clip}%
\pgfsetbuttcap%
\pgfsetroundjoin%
\definecolor{currentfill}{rgb}{0.549096,0.604626,0.682368}%
\pgfsetfillcolor{currentfill}%
\pgfsetlinewidth{0.000000pt}%
\definecolor{currentstroke}{rgb}{0.000000,0.000000,0.000000}%
\pgfsetstrokecolor{currentstroke}%
\pgfsetdash{}{0pt}%
\pgfpathmoveto{\pgfqpoint{1.414227in}{2.011135in}}%
\pgfpathlineto{\pgfqpoint{1.484374in}{2.037108in}}%
\pgfpathlineto{\pgfqpoint{1.416678in}{2.111236in}}%
\pgfpathlineto{\pgfqpoint{1.346372in}{2.085178in}}%
\pgfpathclose%
\pgfusepath{fill}%
\end{pgfscope}%
\begin{pgfscope}%
\pgfpathrectangle{\pgfqpoint{0.150000in}{0.150000in}}{\pgfqpoint{4.700000in}{3.450000in}}%
\pgfusepath{clip}%
\pgfsetbuttcap%
\pgfsetroundjoin%
\definecolor{currentfill}{rgb}{0.996890,0.997273,0.997809}%
\pgfsetfillcolor{currentfill}%
\pgfsetlinewidth{0.000000pt}%
\definecolor{currentstroke}{rgb}{0.000000,0.000000,0.000000}%
\pgfsetstrokecolor{currentstroke}%
\pgfsetdash{}{0pt}%
\pgfpathmoveto{\pgfqpoint{2.563514in}{1.254344in}}%
\pgfpathlineto{\pgfqpoint{2.631001in}{1.277978in}}%
\pgfpathlineto{\pgfqpoint{2.563514in}{1.326824in}}%
\pgfpathlineto{\pgfqpoint{2.495931in}{1.303345in}}%
\pgfpathclose%
\pgfusepath{fill}%
\end{pgfscope}%
\begin{pgfscope}%
\pgfpathrectangle{\pgfqpoint{0.150000in}{0.150000in}}{\pgfqpoint{4.700000in}{3.450000in}}%
\pgfusepath{clip}%
\pgfsetbuttcap%
\pgfsetroundjoin%
\definecolor{currentfill}{rgb}{0.505561,0.566452,0.651700}%
\pgfsetfillcolor{currentfill}%
\pgfsetlinewidth{0.000000pt}%
\definecolor{currentstroke}{rgb}{0.000000,0.000000,0.000000}%
\pgfsetstrokecolor{currentstroke}%
\pgfsetdash{}{0pt}%
\pgfpathmoveto{\pgfqpoint{1.346372in}{2.085178in}}%
\pgfpathlineto{\pgfqpoint{1.416678in}{2.111236in}}%
\pgfpathlineto{\pgfqpoint{1.348943in}{2.185525in}}%
\pgfpathlineto{\pgfqpoint{1.278480in}{2.159379in}}%
\pgfpathclose%
\pgfusepath{fill}%
\end{pgfscope}%
\begin{pgfscope}%
\pgfpathrectangle{\pgfqpoint{0.150000in}{0.150000in}}{\pgfqpoint{4.700000in}{3.450000in}}%
\pgfusepath{clip}%
\pgfsetbuttcap%
\pgfsetroundjoin%
\definecolor{currentfill}{rgb}{0.953355,0.959099,0.967142}%
\pgfsetfillcolor{currentfill}%
\pgfsetlinewidth{0.000000pt}%
\definecolor{currentstroke}{rgb}{0.000000,0.000000,0.000000}%
\pgfsetstrokecolor{currentstroke}%
\pgfsetdash{}{0pt}%
\pgfpathmoveto{\pgfqpoint{2.360479in}{1.328784in}}%
\pgfpathlineto{\pgfqpoint{2.428507in}{1.352230in}}%
\pgfpathlineto{\pgfqpoint{2.361241in}{1.401001in}}%
\pgfpathlineto{\pgfqpoint{2.293120in}{1.377708in}}%
\pgfpathclose%
\pgfusepath{fill}%
\end{pgfscope}%
\begin{pgfscope}%
\pgfpathrectangle{\pgfqpoint{0.150000in}{0.150000in}}{\pgfqpoint{4.700000in}{3.450000in}}%
\pgfusepath{clip}%
\pgfsetbuttcap%
\pgfsetroundjoin%
\definecolor{currentfill}{rgb}{0.462025,0.528278,0.621032}%
\pgfsetfillcolor{currentfill}%
\pgfsetlinewidth{0.000000pt}%
\definecolor{currentstroke}{rgb}{0.000000,0.000000,0.000000}%
\pgfsetstrokecolor{currentstroke}%
\pgfsetdash{}{0pt}%
\pgfpathmoveto{\pgfqpoint{1.278480in}{2.159379in}}%
\pgfpathlineto{\pgfqpoint{1.348943in}{2.185525in}}%
\pgfpathlineto{\pgfqpoint{1.281187in}{2.259719in}}%
\pgfpathlineto{\pgfqpoint{1.210560in}{2.233611in}}%
\pgfpathclose%
\pgfusepath{fill}%
\end{pgfscope}%
\begin{pgfscope}%
\pgfpathrectangle{\pgfqpoint{0.150000in}{0.150000in}}{\pgfqpoint{4.700000in}{3.450000in}}%
\pgfusepath{clip}%
\pgfsetbuttcap%
\pgfsetroundjoin%
\definecolor{currentfill}{rgb}{0.909819,0.920925,0.936474}%
\pgfsetfillcolor{currentfill}%
\pgfsetlinewidth{0.000000pt}%
\definecolor{currentstroke}{rgb}{0.000000,0.000000,0.000000}%
\pgfsetstrokecolor{currentstroke}%
\pgfsetdash{}{0pt}%
\pgfpathmoveto{\pgfqpoint{2.157351in}{1.403258in}}%
\pgfpathlineto{\pgfqpoint{2.225919in}{1.426518in}}%
\pgfpathlineto{\pgfqpoint{2.158876in}{1.475213in}}%
\pgfpathlineto{\pgfqpoint{2.090215in}{1.452106in}}%
\pgfpathclose%
\pgfusepath{fill}%
\end{pgfscope}%
\begin{pgfscope}%
\pgfpathrectangle{\pgfqpoint{0.150000in}{0.150000in}}{\pgfqpoint{4.700000in}{3.450000in}}%
\pgfusepath{clip}%
\pgfsetbuttcap%
\pgfsetroundjoin%
\definecolor{currentfill}{rgb}{0.412270,0.484651,0.585983}%
\pgfsetfillcolor{currentfill}%
\pgfsetlinewidth{0.000000pt}%
\definecolor{currentstroke}{rgb}{0.000000,0.000000,0.000000}%
\pgfsetstrokecolor{currentstroke}%
\pgfsetdash{}{0pt}%
\pgfpathmoveto{\pgfqpoint{1.210560in}{2.233611in}}%
\pgfpathlineto{\pgfqpoint{1.281187in}{2.259719in}}%
\pgfpathlineto{\pgfqpoint{1.213392in}{2.334075in}}%
\pgfpathlineto{\pgfqpoint{1.142611in}{2.307875in}}%
\pgfpathclose%
\pgfusepath{fill}%
\end{pgfscope}%
\begin{pgfscope}%
\pgfpathrectangle{\pgfqpoint{0.150000in}{0.150000in}}{\pgfqpoint{4.700000in}{3.450000in}}%
\pgfusepath{clip}%
\pgfsetbuttcap%
\pgfsetroundjoin%
\definecolor{currentfill}{rgb}{0.860064,0.877298,0.901425}%
\pgfsetfillcolor{currentfill}%
\pgfsetlinewidth{0.000000pt}%
\definecolor{currentstroke}{rgb}{0.000000,0.000000,0.000000}%
\pgfsetstrokecolor{currentstroke}%
\pgfsetdash{}{0pt}%
\pgfpathmoveto{\pgfqpoint{1.954128in}{1.477767in}}%
\pgfpathlineto{\pgfqpoint{2.023237in}{1.500840in}}%
\pgfpathlineto{\pgfqpoint{1.956416in}{1.549459in}}%
\pgfpathlineto{\pgfqpoint{1.887215in}{1.526539in}}%
\pgfpathclose%
\pgfusepath{fill}%
\end{pgfscope}%
\begin{pgfscope}%
\pgfpathrectangle{\pgfqpoint{0.150000in}{0.150000in}}{\pgfqpoint{4.700000in}{3.450000in}}%
\pgfusepath{clip}%
\pgfsetbuttcap%
\pgfsetroundjoin%
\definecolor{currentfill}{rgb}{0.368735,0.446477,0.555316}%
\pgfsetfillcolor{currentfill}%
\pgfsetlinewidth{0.000000pt}%
\definecolor{currentstroke}{rgb}{0.000000,0.000000,0.000000}%
\pgfsetstrokecolor{currentstroke}%
\pgfsetdash{}{0pt}%
\pgfpathmoveto{\pgfqpoint{1.142611in}{2.307875in}}%
\pgfpathlineto{\pgfqpoint{1.213392in}{2.334075in}}%
\pgfpathlineto{\pgfqpoint{1.145576in}{2.408334in}}%
\pgfpathlineto{\pgfqpoint{1.074632in}{2.382171in}}%
\pgfpathclose%
\pgfusepath{fill}%
\end{pgfscope}%
\begin{pgfscope}%
\pgfpathrectangle{\pgfqpoint{0.150000in}{0.150000in}}{\pgfqpoint{4.700000in}{3.450000in}}%
\pgfusepath{clip}%
\pgfsetbuttcap%
\pgfsetroundjoin%
\definecolor{currentfill}{rgb}{0.963909,0.934513,0.936841}%
\pgfsetfillcolor{currentfill}%
\pgfsetlinewidth{0.000000pt}%
\definecolor{currentstroke}{rgb}{0.000000,0.000000,0.000000}%
\pgfsetstrokecolor{currentstroke}%
\pgfsetdash{}{0pt}%
\pgfpathmoveto{\pgfqpoint{2.834419in}{1.130747in}}%
\pgfpathlineto{\pgfqpoint{2.901270in}{1.154722in}}%
\pgfpathlineto{\pgfqpoint{2.833401in}{1.203759in}}%
\pgfpathlineto{\pgfqpoint{2.766453in}{1.179939in}}%
\pgfpathclose%
\pgfusepath{fill}%
\end{pgfscope}%
\begin{pgfscope}%
\pgfpathrectangle{\pgfqpoint{0.150000in}{0.150000in}}{\pgfqpoint{4.700000in}{3.450000in}}%
\pgfusepath{clip}%
\pgfsetbuttcap%
\pgfsetroundjoin%
\definecolor{currentfill}{rgb}{0.325199,0.408303,0.524648}%
\pgfsetfillcolor{currentfill}%
\pgfsetlinewidth{0.000000pt}%
\definecolor{currentstroke}{rgb}{0.000000,0.000000,0.000000}%
\pgfsetstrokecolor{currentstroke}%
\pgfsetdash{}{0pt}%
\pgfpathmoveto{\pgfqpoint{1.074632in}{2.382171in}}%
\pgfpathlineto{\pgfqpoint{1.145576in}{2.408334in}}%
\pgfpathlineto{\pgfqpoint{1.077720in}{2.482756in}}%
\pgfpathlineto{\pgfqpoint{1.006636in}{2.456368in}}%
\pgfpathclose%
\pgfusepath{fill}%
\end{pgfscope}%
\begin{pgfscope}%
\pgfpathrectangle{\pgfqpoint{0.150000in}{0.150000in}}{\pgfqpoint{4.700000in}{3.450000in}}%
\pgfusepath{clip}%
\pgfsetbuttcap%
\pgfsetroundjoin%
\definecolor{currentfill}{rgb}{0.990502,0.982767,0.983379}%
\pgfsetfillcolor{currentfill}%
\pgfsetlinewidth{0.000000pt}%
\definecolor{currentstroke}{rgb}{0.000000,0.000000,0.000000}%
\pgfsetstrokecolor{currentstroke}%
\pgfsetdash{}{0pt}%
\pgfpathmoveto{\pgfqpoint{2.631256in}{1.205228in}}%
\pgfpathlineto{\pgfqpoint{2.698647in}{1.229016in}}%
\pgfpathlineto{\pgfqpoint{2.631001in}{1.277978in}}%
\pgfpathlineto{\pgfqpoint{2.563514in}{1.254344in}}%
\pgfpathclose%
\pgfusepath{fill}%
\end{pgfscope}%
\begin{pgfscope}%
\pgfpathrectangle{\pgfqpoint{0.150000in}{0.150000in}}{\pgfqpoint{4.700000in}{3.450000in}}%
\pgfusepath{clip}%
\pgfsetbuttcap%
\pgfsetroundjoin%
\definecolor{currentfill}{rgb}{0.275444,0.364675,0.489599}%
\pgfsetfillcolor{currentfill}%
\pgfsetlinewidth{0.000000pt}%
\definecolor{currentstroke}{rgb}{0.000000,0.000000,0.000000}%
\pgfsetstrokecolor{currentstroke}%
\pgfsetdash{}{0pt}%
\pgfpathmoveto{\pgfqpoint{1.006636in}{2.456368in}}%
\pgfpathlineto{\pgfqpoint{1.077720in}{2.482756in}}%
\pgfpathlineto{\pgfqpoint{1.009845in}{2.557081in}}%
\pgfpathlineto{\pgfqpoint{0.938601in}{2.530726in}}%
\pgfpathclose%
\pgfusepath{fill}%
\end{pgfscope}%
\begin{pgfscope}%
\pgfpathrectangle{\pgfqpoint{0.150000in}{0.150000in}}{\pgfqpoint{4.700000in}{3.450000in}}%
\pgfusepath{clip}%
\pgfsetbuttcap%
\pgfsetroundjoin%
\definecolor{currentfill}{rgb}{0.972013,0.975460,0.980285}%
\pgfsetfillcolor{currentfill}%
\pgfsetlinewidth{0.000000pt}%
\definecolor{currentstroke}{rgb}{0.000000,0.000000,0.000000}%
\pgfsetstrokecolor{currentstroke}%
\pgfsetdash{}{0pt}%
\pgfpathmoveto{\pgfqpoint{2.427998in}{1.279744in}}%
\pgfpathlineto{\pgfqpoint{2.495931in}{1.303345in}}%
\pgfpathlineto{\pgfqpoint{2.428507in}{1.352230in}}%
\pgfpathlineto{\pgfqpoint{2.360479in}{1.328784in}}%
\pgfpathclose%
\pgfusepath{fill}%
\end{pgfscope}%
\begin{pgfscope}%
\pgfpathrectangle{\pgfqpoint{0.150000in}{0.150000in}}{\pgfqpoint{4.700000in}{3.450000in}}%
\pgfusepath{clip}%
\pgfsetbuttcap%
\pgfsetroundjoin%
\definecolor{currentfill}{rgb}{0.231909,0.326501,0.458931}%
\pgfsetfillcolor{currentfill}%
\pgfsetlinewidth{0.000000pt}%
\definecolor{currentstroke}{rgb}{0.000000,0.000000,0.000000}%
\pgfsetstrokecolor{currentstroke}%
\pgfsetdash{}{0pt}%
\pgfpathmoveto{\pgfqpoint{0.938601in}{2.530726in}}%
\pgfpathlineto{\pgfqpoint{1.009845in}{2.557081in}}%
\pgfpathlineto{\pgfqpoint{0.941928in}{2.631569in}}%
\pgfpathlineto{\pgfqpoint{0.870536in}{2.605116in}}%
\pgfpathclose%
\pgfusepath{fill}%
\end{pgfscope}%
\begin{pgfscope}%
\pgfpathrectangle{\pgfqpoint{0.150000in}{0.150000in}}{\pgfqpoint{4.700000in}{3.450000in}}%
\pgfusepath{clip}%
\pgfsetbuttcap%
\pgfsetroundjoin%
\definecolor{currentfill}{rgb}{0.922258,0.931832,0.945236}%
\pgfsetfillcolor{currentfill}%
\pgfsetlinewidth{0.000000pt}%
\definecolor{currentstroke}{rgb}{0.000000,0.000000,0.000000}%
\pgfsetstrokecolor{currentstroke}%
\pgfsetdash{}{0pt}%
\pgfpathmoveto{\pgfqpoint{2.224645in}{1.354295in}}%
\pgfpathlineto{\pgfqpoint{2.293120in}{1.377708in}}%
\pgfpathlineto{\pgfqpoint{2.225919in}{1.426518in}}%
\pgfpathlineto{\pgfqpoint{2.157351in}{1.403258in}}%
\pgfpathclose%
\pgfusepath{fill}%
\end{pgfscope}%
\begin{pgfscope}%
\pgfpathrectangle{\pgfqpoint{0.150000in}{0.150000in}}{\pgfqpoint{4.700000in}{3.450000in}}%
\pgfusepath{clip}%
\pgfsetbuttcap%
\pgfsetroundjoin%
\definecolor{currentfill}{rgb}{0.878722,0.893658,0.914568}%
\pgfsetfillcolor{currentfill}%
\pgfsetlinewidth{0.000000pt}%
\definecolor{currentstroke}{rgb}{0.000000,0.000000,0.000000}%
\pgfsetstrokecolor{currentstroke}%
\pgfsetdash{}{0pt}%
\pgfpathmoveto{\pgfqpoint{2.021198in}{1.428880in}}%
\pgfpathlineto{\pgfqpoint{2.090215in}{1.452106in}}%
\pgfpathlineto{\pgfqpoint{2.023237in}{1.500840in}}%
\pgfpathlineto{\pgfqpoint{1.954128in}{1.477767in}}%
\pgfpathclose%
\pgfusepath{fill}%
\end{pgfscope}%
\begin{pgfscope}%
\pgfpathrectangle{\pgfqpoint{0.150000in}{0.150000in}}{\pgfqpoint{4.700000in}{3.450000in}}%
\pgfusepath{clip}%
\pgfsetbuttcap%
\pgfsetroundjoin%
\definecolor{currentfill}{rgb}{0.982904,0.968980,0.970083}%
\pgfsetfillcolor{currentfill}%
\pgfsetlinewidth{0.000000pt}%
\definecolor{currentstroke}{rgb}{0.000000,0.000000,0.000000}%
\pgfsetstrokecolor{currentstroke}%
\pgfsetdash{}{0pt}%
\pgfpathmoveto{\pgfqpoint{2.699158in}{1.155996in}}%
\pgfpathlineto{\pgfqpoint{2.766453in}{1.179939in}}%
\pgfpathlineto{\pgfqpoint{2.698647in}{1.229016in}}%
\pgfpathlineto{\pgfqpoint{2.631256in}{1.205228in}}%
\pgfpathclose%
\pgfusepath{fill}%
\end{pgfscope}%
\begin{pgfscope}%
\pgfpathrectangle{\pgfqpoint{0.150000in}{0.150000in}}{\pgfqpoint{4.700000in}{3.450000in}}%
\pgfusepath{clip}%
\pgfsetbuttcap%
\pgfsetroundjoin%
\definecolor{currentfill}{rgb}{0.984452,0.986366,0.989047}%
\pgfsetfillcolor{currentfill}%
\pgfsetlinewidth{0.000000pt}%
\definecolor{currentstroke}{rgb}{0.000000,0.000000,0.000000}%
\pgfsetstrokecolor{currentstroke}%
\pgfsetdash{}{0pt}%
\pgfpathmoveto{\pgfqpoint{2.495676in}{1.230588in}}%
\pgfpathlineto{\pgfqpoint{2.563514in}{1.254344in}}%
\pgfpathlineto{\pgfqpoint{2.495931in}{1.303345in}}%
\pgfpathlineto{\pgfqpoint{2.427998in}{1.279744in}}%
\pgfpathclose%
\pgfusepath{fill}%
\end{pgfscope}%
\begin{pgfscope}%
\pgfpathrectangle{\pgfqpoint{0.150000in}{0.150000in}}{\pgfqpoint{4.700000in}{3.450000in}}%
\pgfusepath{clip}%
\pgfsetbuttcap%
\pgfsetroundjoin%
\definecolor{currentfill}{rgb}{0.940916,0.948192,0.958379}%
\pgfsetfillcolor{currentfill}%
\pgfsetlinewidth{0.000000pt}%
\definecolor{currentstroke}{rgb}{0.000000,0.000000,0.000000}%
\pgfsetstrokecolor{currentstroke}%
\pgfsetdash{}{0pt}%
\pgfpathmoveto{\pgfqpoint{2.292099in}{1.305215in}}%
\pgfpathlineto{\pgfqpoint{2.360479in}{1.328784in}}%
\pgfpathlineto{\pgfqpoint{2.293120in}{1.377708in}}%
\pgfpathlineto{\pgfqpoint{2.224645in}{1.354295in}}%
\pgfpathclose%
\pgfusepath{fill}%
\end{pgfscope}%
\begin{pgfscope}%
\pgfpathrectangle{\pgfqpoint{0.150000in}{0.150000in}}{\pgfqpoint{4.700000in}{3.450000in}}%
\pgfusepath{clip}%
\pgfsetbuttcap%
\pgfsetroundjoin%
\definecolor{currentfill}{rgb}{0.891161,0.904565,0.923330}%
\pgfsetfillcolor{currentfill}%
\pgfsetlinewidth{0.000000pt}%
\definecolor{currentstroke}{rgb}{0.000000,0.000000,0.000000}%
\pgfsetstrokecolor{currentstroke}%
\pgfsetdash{}{0pt}%
\pgfpathmoveto{\pgfqpoint{2.088427in}{1.379878in}}%
\pgfpathlineto{\pgfqpoint{2.157351in}{1.403258in}}%
\pgfpathlineto{\pgfqpoint{2.090215in}{1.452106in}}%
\pgfpathlineto{\pgfqpoint{2.021198in}{1.428880in}}%
\pgfpathclose%
\pgfusepath{fill}%
\end{pgfscope}%
\begin{pgfscope}%
\pgfpathrectangle{\pgfqpoint{0.150000in}{0.150000in}}{\pgfqpoint{4.700000in}{3.450000in}}%
\pgfusepath{clip}%
\pgfsetbuttcap%
\pgfsetroundjoin%
\definecolor{currentfill}{rgb}{0.971507,0.948300,0.950138}%
\pgfsetfillcolor{currentfill}%
\pgfsetlinewidth{0.000000pt}%
\definecolor{currentstroke}{rgb}{0.000000,0.000000,0.000000}%
\pgfsetstrokecolor{currentstroke}%
\pgfsetdash{}{0pt}%
\pgfpathmoveto{\pgfqpoint{2.767220in}{1.106647in}}%
\pgfpathlineto{\pgfqpoint{2.834419in}{1.130747in}}%
\pgfpathlineto{\pgfqpoint{2.766453in}{1.179939in}}%
\pgfpathlineto{\pgfqpoint{2.699158in}{1.155996in}}%
\pgfpathclose%
\pgfusepath{fill}%
\end{pgfscope}%
\begin{pgfscope}%
\pgfpathrectangle{\pgfqpoint{0.150000in}{0.150000in}}{\pgfqpoint{4.700000in}{3.450000in}}%
\pgfusepath{clip}%
\pgfsetbuttcap%
\pgfsetroundjoin%
\definecolor{currentfill}{rgb}{0.996890,0.997273,0.997809}%
\pgfsetfillcolor{currentfill}%
\pgfsetlinewidth{0.000000pt}%
\definecolor{currentstroke}{rgb}{0.000000,0.000000,0.000000}%
\pgfsetstrokecolor{currentstroke}%
\pgfsetdash{}{0pt}%
\pgfpathmoveto{\pgfqpoint{2.563514in}{1.181316in}}%
\pgfpathlineto{\pgfqpoint{2.631256in}{1.205228in}}%
\pgfpathlineto{\pgfqpoint{2.563514in}{1.254344in}}%
\pgfpathlineto{\pgfqpoint{2.495676in}{1.230588in}}%
\pgfpathclose%
\pgfusepath{fill}%
\end{pgfscope}%
\begin{pgfscope}%
\pgfpathrectangle{\pgfqpoint{0.150000in}{0.150000in}}{\pgfqpoint{4.700000in}{3.450000in}}%
\pgfusepath{clip}%
\pgfsetbuttcap%
\pgfsetroundjoin%
\definecolor{currentfill}{rgb}{0.953355,0.959099,0.967142}%
\pgfsetfillcolor{currentfill}%
\pgfsetlinewidth{0.000000pt}%
\definecolor{currentstroke}{rgb}{0.000000,0.000000,0.000000}%
\pgfsetstrokecolor{currentstroke}%
\pgfsetdash{}{0pt}%
\pgfpathmoveto{\pgfqpoint{2.359712in}{1.256020in}}%
\pgfpathlineto{\pgfqpoint{2.427998in}{1.279744in}}%
\pgfpathlineto{\pgfqpoint{2.360479in}{1.328784in}}%
\pgfpathlineto{\pgfqpoint{2.292099in}{1.305215in}}%
\pgfpathclose%
\pgfusepath{fill}%
\end{pgfscope}%
\begin{pgfscope}%
\pgfpathrectangle{\pgfqpoint{0.150000in}{0.150000in}}{\pgfqpoint{4.700000in}{3.450000in}}%
\pgfusepath{clip}%
\pgfsetbuttcap%
\pgfsetroundjoin%
\definecolor{currentfill}{rgb}{0.909819,0.920925,0.936474}%
\pgfsetfillcolor{currentfill}%
\pgfsetlinewidth{0.000000pt}%
\definecolor{currentstroke}{rgb}{0.000000,0.000000,0.000000}%
\pgfsetstrokecolor{currentstroke}%
\pgfsetdash{}{0pt}%
\pgfpathmoveto{\pgfqpoint{2.155815in}{1.330759in}}%
\pgfpathlineto{\pgfqpoint{2.224645in}{1.354295in}}%
\pgfpathlineto{\pgfqpoint{2.157351in}{1.403258in}}%
\pgfpathlineto{\pgfqpoint{2.088427in}{1.379878in}}%
\pgfpathclose%
\pgfusepath{fill}%
\end{pgfscope}%
\begin{pgfscope}%
\pgfpathrectangle{\pgfqpoint{0.150000in}{0.150000in}}{\pgfqpoint{4.700000in}{3.450000in}}%
\pgfusepath{clip}%
\pgfsetbuttcap%
\pgfsetroundjoin%
\definecolor{currentfill}{rgb}{0.990502,0.982767,0.983379}%
\pgfsetfillcolor{currentfill}%
\pgfsetlinewidth{0.000000pt}%
\definecolor{currentstroke}{rgb}{0.000000,0.000000,0.000000}%
\pgfsetstrokecolor{currentstroke}%
\pgfsetdash{}{0pt}%
\pgfpathmoveto{\pgfqpoint{2.631512in}{1.131927in}}%
\pgfpathlineto{\pgfqpoint{2.699158in}{1.155996in}}%
\pgfpathlineto{\pgfqpoint{2.631256in}{1.205228in}}%
\pgfpathlineto{\pgfqpoint{2.563514in}{1.181316in}}%
\pgfpathclose%
\pgfusepath{fill}%
\end{pgfscope}%
\begin{pgfscope}%
\pgfpathrectangle{\pgfqpoint{0.150000in}{0.150000in}}{\pgfqpoint{4.700000in}{3.450000in}}%
\pgfusepath{clip}%
\pgfsetbuttcap%
\pgfsetroundjoin%
\definecolor{currentfill}{rgb}{0.972013,0.975460,0.980285}%
\pgfsetfillcolor{currentfill}%
\pgfsetlinewidth{0.000000pt}%
\definecolor{currentstroke}{rgb}{0.000000,0.000000,0.000000}%
\pgfsetstrokecolor{currentstroke}%
\pgfsetdash{}{0pt}%
\pgfpathmoveto{\pgfqpoint{2.427485in}{1.206708in}}%
\pgfpathlineto{\pgfqpoint{2.495676in}{1.230588in}}%
\pgfpathlineto{\pgfqpoint{2.427998in}{1.279744in}}%
\pgfpathlineto{\pgfqpoint{2.359712in}{1.256020in}}%
\pgfpathclose%
\pgfusepath{fill}%
\end{pgfscope}%
\begin{pgfscope}%
\pgfpathrectangle{\pgfqpoint{0.150000in}{0.150000in}}{\pgfqpoint{4.700000in}{3.450000in}}%
\pgfusepath{clip}%
\pgfsetbuttcap%
\pgfsetroundjoin%
\definecolor{currentfill}{rgb}{0.922258,0.931832,0.945236}%
\pgfsetfillcolor{currentfill}%
\pgfsetlinewidth{0.000000pt}%
\definecolor{currentstroke}{rgb}{0.000000,0.000000,0.000000}%
\pgfsetstrokecolor{currentstroke}%
\pgfsetdash{}{0pt}%
\pgfpathmoveto{\pgfqpoint{2.223362in}{1.281524in}}%
\pgfpathlineto{\pgfqpoint{2.292099in}{1.305215in}}%
\pgfpathlineto{\pgfqpoint{2.224645in}{1.354295in}}%
\pgfpathlineto{\pgfqpoint{2.155815in}{1.330759in}}%
\pgfpathclose%
\pgfusepath{fill}%
\end{pgfscope}%
\begin{pgfscope}%
\pgfpathrectangle{\pgfqpoint{0.150000in}{0.150000in}}{\pgfqpoint{4.700000in}{3.450000in}}%
\pgfusepath{clip}%
\pgfsetbuttcap%
\pgfsetroundjoin%
\definecolor{currentfill}{rgb}{0.982904,0.968980,0.970083}%
\pgfsetfillcolor{currentfill}%
\pgfsetlinewidth{0.000000pt}%
\definecolor{currentstroke}{rgb}{0.000000,0.000000,0.000000}%
\pgfsetstrokecolor{currentstroke}%
\pgfsetdash{}{0pt}%
\pgfpathmoveto{\pgfqpoint{2.699672in}{1.082422in}}%
\pgfpathlineto{\pgfqpoint{2.767220in}{1.106647in}}%
\pgfpathlineto{\pgfqpoint{2.699158in}{1.155996in}}%
\pgfpathlineto{\pgfqpoint{2.631512in}{1.131927in}}%
\pgfpathclose%
\pgfusepath{fill}%
\end{pgfscope}%
\begin{pgfscope}%
\pgfpathrectangle{\pgfqpoint{0.150000in}{0.150000in}}{\pgfqpoint{4.700000in}{3.450000in}}%
\pgfusepath{clip}%
\pgfsetbuttcap%
\pgfsetroundjoin%
\definecolor{currentfill}{rgb}{0.984452,0.986366,0.989047}%
\pgfsetfillcolor{currentfill}%
\pgfsetlinewidth{0.000000pt}%
\definecolor{currentstroke}{rgb}{0.000000,0.000000,0.000000}%
\pgfsetstrokecolor{currentstroke}%
\pgfsetdash{}{0pt}%
\pgfpathmoveto{\pgfqpoint{2.495418in}{1.157280in}}%
\pgfpathlineto{\pgfqpoint{2.563514in}{1.181316in}}%
\pgfpathlineto{\pgfqpoint{2.495676in}{1.230588in}}%
\pgfpathlineto{\pgfqpoint{2.427485in}{1.206708in}}%
\pgfpathclose%
\pgfusepath{fill}%
\end{pgfscope}%
\begin{pgfscope}%
\pgfpathrectangle{\pgfqpoint{0.150000in}{0.150000in}}{\pgfqpoint{4.700000in}{3.450000in}}%
\pgfusepath{clip}%
\pgfsetbuttcap%
\pgfsetroundjoin%
\definecolor{currentfill}{rgb}{0.940916,0.948192,0.958379}%
\pgfsetfillcolor{currentfill}%
\pgfsetlinewidth{0.000000pt}%
\definecolor{currentstroke}{rgb}{0.000000,0.000000,0.000000}%
\pgfsetstrokecolor{currentstroke}%
\pgfsetdash{}{0pt}%
\pgfpathmoveto{\pgfqpoint{2.291070in}{1.232173in}}%
\pgfpathlineto{\pgfqpoint{2.359712in}{1.256020in}}%
\pgfpathlineto{\pgfqpoint{2.292099in}{1.305215in}}%
\pgfpathlineto{\pgfqpoint{2.223362in}{1.281524in}}%
\pgfpathclose%
\pgfusepath{fill}%
\end{pgfscope}%
\begin{pgfscope}%
\pgfpathrectangle{\pgfqpoint{0.150000in}{0.150000in}}{\pgfqpoint{4.700000in}{3.450000in}}%
\pgfusepath{clip}%
\pgfsetbuttcap%
\pgfsetroundjoin%
\definecolor{currentfill}{rgb}{0.996890,0.997273,0.997809}%
\pgfsetfillcolor{currentfill}%
\pgfsetlinewidth{0.000000pt}%
\definecolor{currentstroke}{rgb}{0.000000,0.000000,0.000000}%
\pgfsetstrokecolor{currentstroke}%
\pgfsetdash{}{0pt}%
\pgfpathmoveto{\pgfqpoint{2.563514in}{1.107734in}}%
\pgfpathlineto{\pgfqpoint{2.631512in}{1.131927in}}%
\pgfpathlineto{\pgfqpoint{2.563514in}{1.181316in}}%
\pgfpathlineto{\pgfqpoint{2.495418in}{1.157280in}}%
\pgfpathclose%
\pgfusepath{fill}%
\end{pgfscope}%
\begin{pgfscope}%
\pgfpathrectangle{\pgfqpoint{0.150000in}{0.150000in}}{\pgfqpoint{4.700000in}{3.450000in}}%
\pgfusepath{clip}%
\pgfsetbuttcap%
\pgfsetroundjoin%
\definecolor{currentfill}{rgb}{0.953355,0.959099,0.967142}%
\pgfsetfillcolor{currentfill}%
\pgfsetlinewidth{0.000000pt}%
\definecolor{currentstroke}{rgb}{0.000000,0.000000,0.000000}%
\pgfsetstrokecolor{currentstroke}%
\pgfsetdash{}{0pt}%
\pgfpathmoveto{\pgfqpoint{2.358938in}{1.182704in}}%
\pgfpathlineto{\pgfqpoint{2.427485in}{1.206708in}}%
\pgfpathlineto{\pgfqpoint{2.359712in}{1.256020in}}%
\pgfpathlineto{\pgfqpoint{2.291070in}{1.232173in}}%
\pgfpathclose%
\pgfusepath{fill}%
\end{pgfscope}%
\begin{pgfscope}%
\pgfpathrectangle{\pgfqpoint{0.150000in}{0.150000in}}{\pgfqpoint{4.700000in}{3.450000in}}%
\pgfusepath{clip}%
\pgfsetbuttcap%
\pgfsetroundjoin%
\definecolor{currentfill}{rgb}{0.990502,0.982767,0.983379}%
\pgfsetfillcolor{currentfill}%
\pgfsetlinewidth{0.000000pt}%
\definecolor{currentstroke}{rgb}{0.000000,0.000000,0.000000}%
\pgfsetstrokecolor{currentstroke}%
\pgfsetdash{}{0pt}%
\pgfpathmoveto{\pgfqpoint{2.631770in}{1.058070in}}%
\pgfpathlineto{\pgfqpoint{2.699672in}{1.082422in}}%
\pgfpathlineto{\pgfqpoint{2.631512in}{1.131927in}}%
\pgfpathlineto{\pgfqpoint{2.563514in}{1.107734in}}%
\pgfpathclose%
\pgfusepath{fill}%
\end{pgfscope}%
\begin{pgfscope}%
\pgfpathrectangle{\pgfqpoint{0.150000in}{0.150000in}}{\pgfqpoint{4.700000in}{3.450000in}}%
\pgfusepath{clip}%
\pgfsetbuttcap%
\pgfsetroundjoin%
\definecolor{currentfill}{rgb}{0.972013,0.975460,0.980285}%
\pgfsetfillcolor{currentfill}%
\pgfsetlinewidth{0.000000pt}%
\definecolor{currentstroke}{rgb}{0.000000,0.000000,0.000000}%
\pgfsetstrokecolor{currentstroke}%
\pgfsetdash{}{0pt}%
\pgfpathmoveto{\pgfqpoint{2.426968in}{1.133118in}}%
\pgfpathlineto{\pgfqpoint{2.495418in}{1.157280in}}%
\pgfpathlineto{\pgfqpoint{2.427485in}{1.206708in}}%
\pgfpathlineto{\pgfqpoint{2.358938in}{1.182704in}}%
\pgfpathclose%
\pgfusepath{fill}%
\end{pgfscope}%
\begin{pgfscope}%
\pgfpathrectangle{\pgfqpoint{0.150000in}{0.150000in}}{\pgfqpoint{4.700000in}{3.450000in}}%
\pgfusepath{clip}%
\pgfsetbuttcap%
\pgfsetroundjoin%
\definecolor{currentfill}{rgb}{0.984452,0.986366,0.989047}%
\pgfsetfillcolor{currentfill}%
\pgfsetlinewidth{0.000000pt}%
\definecolor{currentstroke}{rgb}{0.000000,0.000000,0.000000}%
\pgfsetstrokecolor{currentstroke}%
\pgfsetdash{}{0pt}%
\pgfpathmoveto{\pgfqpoint{2.495159in}{1.083413in}}%
\pgfpathlineto{\pgfqpoint{2.563514in}{1.107734in}}%
\pgfpathlineto{\pgfqpoint{2.495418in}{1.157280in}}%
\pgfpathlineto{\pgfqpoint{2.426968in}{1.133118in}}%
\pgfpathclose%
\pgfusepath{fill}%
\end{pgfscope}%
\begin{pgfscope}%
\pgfpathrectangle{\pgfqpoint{0.150000in}{0.150000in}}{\pgfqpoint{4.700000in}{3.450000in}}%
\pgfusepath{clip}%
\pgfsetbuttcap%
\pgfsetroundjoin%
\definecolor{currentfill}{rgb}{0.996890,0.997273,0.997809}%
\pgfsetfillcolor{currentfill}%
\pgfsetlinewidth{0.000000pt}%
\definecolor{currentstroke}{rgb}{0.000000,0.000000,0.000000}%
\pgfsetstrokecolor{currentstroke}%
\pgfsetdash{}{0pt}%
\pgfpathmoveto{\pgfqpoint{2.563514in}{1.033591in}}%
\pgfpathlineto{\pgfqpoint{2.631770in}{1.058070in}}%
\pgfpathlineto{\pgfqpoint{2.563514in}{1.107734in}}%
\pgfpathlineto{\pgfqpoint{2.495159in}{1.083413in}}%
\pgfpathclose%
\pgfusepath{fill}%
\end{pgfscope}%
\end{pgfpicture}%
\makeatother%
\endgroup%

        \caption[Game Value of a Simple Tennis Game]{The dependence of the game value $\val(G_{\lambda, \mu})$ on learning parameters $\lambda, \mu \in [0, 1]$ for a tennis point with parameterised incompetence. Generated using \texttt{incompetent\_game\_plot.py}.}
        \label{fig:tennis-game-plot}
    \end{figure}

    \begin{figure}[t]
        \centering
        \begin{minipage}{\textwidth}
            \subbottom[\label{fig:tennis-equilibrium-a}]%
                {% ---------------------------------------------------------------------------- %
% Honours Thesis                                                               %
% Figure: Equilibrium in a Simple Tennis Game with Incremental Learning        %
% ---------------------------------------------------------------------------- %

\begin{tikzpicture}
    \newcommand\SCL{0.8}

    % Vertex Style
    \tikzset{vertex/.style = {
        draw,
        shape=circle,
        inner sep=0pt,
        minimum size=7pt
    }}
    
    % Empty Vertex Style
    \tikzset{empty/.style = {
        inner sep=0pt,
        minimum size=7pt,=
    }}

    % Edge Style
    \tikzset{edge/.style = {
        ->,
        > = latex',
        shorten <=7pt,
        shorten >=7pt
    }}

    % Diagonal Edge Style
    \tikzset{diagonal/.style = {
        ->,
        > = latex',
        shorten <=10pt,
        shorten >=10pt
    }}

    % Nodes
    \foreach \i in {0,1,2,3,4,5}
        \foreach \j in {0,1,2,3,4,5}
            \node[vertex] (\i, \j) at (\SCL * \i, \SCL* \j) {};

    % Edges
    \foreach \i in {0,1,2,3,4}
        \foreach \j in {5}
            \draw[edge] (\SCL * \i, \SCL * \j) to (\SCL * \i + \SCL, \SCL * \j);

    \foreach \j in {0,1,2,3,4}
        \foreach \i in {5}
            \draw[edge] (\SCL * \i, \SCL * \j) to (\SCL * \i, \SCL * \j + \SCL);

    \foreach \j in {0}
        \foreach \i in {1,2,3,4}
            \draw[diagonal] (\SCL * \i, \SCL * \j) to (\SCL * \i + \SCL, \SCL * \j + \SCL);

    \foreach \j in {1,2,3,4}
        \foreach \i in {0,1,2,3,4}
            \draw[diagonal] (\SCL * \i, \SCL * \j) to (\SCL * \i + \SCL, \SCL * \j + \SCL);

    \draw[edge] (0, 0) to [out=330,in=300,loop] (0, 0);
    \draw[edge] (\SCL * 5, \SCL * 5) to [out=330,in=300,loop] (\SCL * 5, \SCL * 5);


    % Labels
    \node[rotate=90] (P1) at (\SCL * -1.5, \SCL * 3) {Player 1};
    \foreach \j in {1,2,3,4}
        \node[] (L\j) at (\SCL * -0.5, \SCL * \j) {$\nicefrac{\j}{5}$};
    \node[] (L0) at (\SCL * -0.5, 0) {$0$};
    \node[] (L5) at (\SCL * -0.5, \SCL * 5) {$1$};


    \node[] (P2) at (\SCL * 3, \SCL * 6.5) {Player 2};
    \foreach \i in {1,2,3,4}
        \node[] (M\i) at (\SCL * \i, \SCL * 5.5) {$\nicefrac{\i}{5}$};
    \node[] (M0) at (0, \SCL * 5.5) {$0$};
    \node[] (M0) at (\SCL * 5, \SCL * 5.5) {$1$};

\end{tikzpicture}}
            \hfill
            \subbottom[\label{fig:tennis-equilibrium-b}]%
                {% ---------------------------------------------------------------------------- %
% Honours Thesis                                                               %
% Figure: Equilibrium in a Simple Tennis Game with Incremental Learning        %
% ---------------------------------------------------------------------------- %

\begin{tikzpicture}
    \newcommand\SCL{0.8}

    % Vertex Style
    \tikzset{vertex/.style = {
        draw,
        shape=circle,
        inner sep=0pt,
        minimum size=7pt
    }}
    
    % Empty Vertex Style
    \tikzset{empty/.style = {
        inner sep=0pt,
        minimum size=7pt,=
    }}

    % Edge Style
    \tikzset{edge/.style = {
        ->,
        > = latex',
        shorten <=7pt,
        shorten >=7pt
    }}

    % Diagonal Edge Style
    \tikzset{diagonal/.style = {
        ->,
        > = latex',
        shorten <=10pt,
        shorten >=10pt
    }}

    % Nodes
    \foreach \i in {0,1,2,3,4,5}
        \foreach \j in {0,1,2,3,4,5}
            \node[vertex] (\i, \j) at (\SCL * \i, \SCL* \j) {};

    % Edges
    \foreach \i in {0,1,2,3,4}
        \foreach \j in {5}
            \draw[edge] (\SCL * \i, \SCL * \j) to (\SCL * \i + \SCL, \SCL * \j);

    \foreach \j in {0,1,2,3,4}
        \foreach \i in {5}
            \draw[edge] (\SCL * \i, \SCL * \j) to (\SCL * \i, \SCL * \j + \SCL);
    
    \foreach \j in {0,1,2,3,4}
        \foreach \i in {0,1,2,3,4}
            \draw[diagonal] (\SCL * \i, \SCL * \j) to (\SCL * \i + \SCL, \SCL * \j + \SCL);

    \draw[edge] (\SCL * 5, \SCL * 5) to [out=330,in=300,loop] (\SCL * 5, \SCL * 5);
    \draw[edge, white] (0, 0) to [out=330,in=300,loop] (0, 0);

    % Labels
    \node[rotate=90] (P1) at (\SCL * -1.5, \SCL * 3) {Player 1};
    \foreach \j in {1,2,3,4}
        \node[] (L\j) at (\SCL * -0.5, \SCL * \j) {$\nicefrac{\j}{5}$};
    \node[] (L0) at (\SCL * -0.5, 0) {$0$};
    \node[] (L5) at (\SCL * -0.5, \SCL * 5) {$1$};


    \node[] (P2) at (\SCL * 3, \SCL * 6.5) {Player 2};
    \foreach \i in {1,2,3,4}
        \node[] (M\i) at (\SCL * \i, \SCL * 5.5) {$\nicefrac{\i}{5}$};
    \node[] (M0) at (0, \SCL * 5.5) {$0$};
    \node[] (M0) at (\SCL * 5, \SCL * 5.5) {$1$};

\end{tikzpicture}}

            \subbottom[\label{fig:tennis-equilibrium-c}]%
                {% ---------------------------------------------------------------------------- %
% Honours Thesis                                                               %
% Figure: Equilibrium in a Simple Tennis Game with Incremental Learning        %
% ---------------------------------------------------------------------------- %

\begin{tikzpicture}
    \newcommand\SCL{0.8}

    % Labels
    \node[rotate=90] (P1) at (\SCL * -1.5, \SCL * 3) {Player 1};
    \foreach \j in {1,2,3,4}
        \node[] (L\j) at (\SCL * -0.5, \SCL * \j) {$\nicefrac{\j}{5}$};
    \node[] (L0) at (\SCL * -0.5, 0) {$0$};
    \node[] (L5) at (\SCL * -0.5, \SCL * 5) {$1$};


    \node[] (P2) at (\SCL * 3, \SCL * 6.5) {Player 2};
    \foreach \i in {1,2,3,4}
        \node[] (M\i) at (\SCL * \i, \SCL * 5.5) {$\nicefrac{\i}{5}$};
    \node[] (M0) at (0, \SCL * 5.5) {$0$};
    \node[] (M0) at (\SCL * 5, \SCL * 5.5) {$1$};

    \node[rotate=90] at (7.35, 2.25) {\tiny $2.76 \times 10^{-3}$};
    \node[rotate=45] at (8.5, 2.5) {\tiny $7.68 \times 10^{-6}$};
    \node[] at (8.75, 1.55) {\tiny $2.76 \times 10^{-3}$};
    \node[] at (9.25, 0.5) {\tiny $9.94 \times 10^{-1}$};

    \begin{scope}[spy using outlines={rectangle, magnification=2.5, size=1cm, connect spies}]

    % Vertex Style
    \tikzset{vertex/.style = {
        draw,
        shape=circle,
        inner sep=0pt,
        minimum size=7pt
    }}

    \tikzset{bigvertex/.style = {
        draw,
        shape=circle,
        inner sep=0pt,
        minimum size=14pt
    }}
    
    % Empty Vertex Style
    \tikzset{empty/.style = {
        inner sep=0pt,
        minimum size=7pt,=
    }}

    % Edge Style
    \tikzset{edge/.style = {
        ->,
        > = latex',
        shorten <=7pt,
        shorten >=7pt
    }}

    % Diagonal Edge Style
    \tikzset{diagonal/.style = {
        ->,
        > = latex',
        shorten <=10pt,
        shorten >=10pt
    }}

    % Nodes
    \foreach \i in {0,1,2,3,4,5}
        \foreach \j in {0,1,2,3,4,5}
            \node[vertex] (\i, \j) at (\SCL * \i, \SCL* \j) {};

    % Edges
    \foreach \i in {0,1,2,3,4}
        \foreach \j in {5}
            \draw[edge] (\SCL * \i, \SCL * \j) to (\SCL * \i + \SCL, \SCL * \j);

    \foreach \j in {0,1,2,3,4}
        \foreach \i in {5}
            \draw[edge] (\SCL * \i, \SCL * \j) to (\SCL * \i, \SCL * \j + \SCL);

    \foreach \j in {0}
        \foreach \i in {1,2,3,4}
            \draw[diagonal] (\SCL * \i, \SCL * \j) to (\SCL * \i + \SCL, \SCL * \j + \SCL);

    \foreach \j in {1,2,3,4}
        \foreach \i in {0,1,2,3,4}
            \draw[diagonal] (\SCL * \i, \SCL * \j) to (\SCL * \i + \SCL, \SCL * \j + \SCL);

    \draw[edge] (0, 0) to [out=330,in=300,loop] (0, 0);
    \draw[edge] (0, 0) to (\SCL, 0);
    \draw[edge] (0, 0) to (0, \SCL) {};
    \draw[diagonal] (0, 0) to (\SCL, \SCL);
    \draw[edge] (\SCL * 5, \SCL * 5) to [out=330,in=300,loop] (\SCL * 5, \SCL * 5);

    \node[] (Z) at (0.5 * \SCL, 0.4 * \SCL) {};
    \spy[color4, width=4cm, height=4cm] on (Z) in node at (8.7, 2);

    \end{scope}



\end{tikzpicture}}
            \hfill
            \caption[Equilibria in a Tennis Game with Learning]{The learning strategies under various equilibria in a simple tennis game with incremental learning. Computed using \texttt{incremental\_solver.py}.}
            \label{fig:tennis-equilibria}
        \end{minipage}
    \end{figure}

    Assume that, to convert these probabilities into a matrix game, a player is given $-100$ utility after losing a point and a player is given $100$ utility after winning a point.
    Clearly, this situation can be expressed as a $3 \times 3$ matrix game $G$ with the utility matrix
    \[
        R
            =
            \begin{pmatrix}
                0    & 40   & 100 \\
                -40  & 0    & 100 \\
                -100 & -100 & 0   \\
            \end{pmatrix}.
    \]
    Unsurprisingly, the unique equilibrium $(\vec{x}^*, \vec{y}^*)$ of the competent game $G$ has $\vec{x}^* = (1, 0, 0)$ and $\vec{y}^* = (1, 0, 0)$; that is, both Player 1 and Player 2 should always select ``Good Shot''.
    Suppose that the incompetence of Player 1 is parameterised by $Q_1 : [0, 1] \to \RR^{3 \times 3}$ where, for all $\lambda \in [0, 1]$, we have
    \[
        Q_1(\lambda)
            =
            \begin{pmatrix}
                \nicefrac{3}{10} & \nicefrac{1}{10} & \nicefrac{3}{5}  \\
                \nicefrac{1}{10} & \nicefrac{3}{5}  & \nicefrac{3}{10} \\
                0 & 0 & 1 \\
            \end{pmatrix}
            (1 - \lambda) +
            \begin{pmatrix}
                1 & 0 & 0 \\
                0 & 1 & 0 \\
                0 & 0 & 1 \\
            \end{pmatrix}
            \lambda
    \]
    and the incompetence of Player 2 is parametersied by $Q_2 : [0, 1] \to \RR^{3 \times 3}$ where, for all $\mu \in [0, 1]$, we have
    \[
        Q_2(\mu)
            =
            \begin{pmatrix}
                \nicefrac{3}{10} & \nicefrac{1}{10} & \nicefrac{3}{5}  \\
                \nicefrac{1}{10} & \nicefrac{3}{5}  & \nicefrac{3}{10} \\
                0 & 0 & 1 \\
            \end{pmatrix}
            (1 - \mu) +
            \begin{pmatrix}
                1 & 0 & 0 \\
                0 & 1 & 0 \\
                0 & 0 & 1 \\
            \end{pmatrix}
            \mu.
    \]
    Although an incompetent player is more likely to accidentally execute ``Out'' when they select ``Good Shot'' than when they select ``Safe Shot'', this is compensated for by having ``Good Shot'' being more likely to win than ``Safe Shot''.
    The game value of the resulting parameterised incompetent game is shown in \autoref{fig:tennis-game-plot}.
    Now, to create a discounted incremental learning game $\Gamma_\beta$ from this tennis game, we allow Player 1 to attain the learning parameters
    \[
        \Gamma
            =
            \big\{ \lambda_i = \frac{i - 1}{5} : i = 1, 2, \ldots, 6 \big\}
    \]
    and Player 2 to attain the learning parameters
    \[
        \Mu
            =
            \big\{ \mu_j = \frac{j - 1}{5} : j = 1, 2, \ldots, 6 \big\}.
    \]
    Additionally, for any $i = 1, 2, \ldots, 6$ and $j = 1, 2, \ldots, 6$, the learning cost at state $(i, j)$ is $c^1(i, j) = 10$ and $c^2(i, j) = 10$ for Player 1 and Player 2, respectively.
    The future rewards are progressively discounted by a discount factor of $\beta = \nicefrac{19}{20}$.

    Next, applying the aforementioned backward induction algorithm to the tennis game with incremental learning, we find that there are a total of three equilibria, which are shown in \autoref{fig:tennis-equilibria}.
    A node indicates a  pair of learning parameters and an arc represents a transition realised by the equilibrium.
    So, a vertical arrow means that only Player 1 learns, a horizontal arrow means that only Player 2 learns, a diagonal arrow means that both players learn, and a loop means that neither player learns.
    An equilibrium in mixed strategies is shown in \autoref{fig:tennis-equilibrium-c} and the probabilities of each transition are included.

    The only differences between the equilibria in \autoref{fig:tennis-equilibria} are the strategies employed at the initial state $(0, 0)$ and, at the remaining states, the players always choose to learn whenever possible.
    Namely, at the initial state, neither player learns in \autoref{fig:tennis-equilibrium-a}, both players learn in \autoref{fig:tennis-equilibrium-b}, and learning is unlikely in \autoref{fig:tennis-equilibrium-c}.
    How can these seemingly opposite equilibria occur within the same game?
    The equilibrium shown in \autoref{fig:tennis-equilibrium-a} arises because, when your opponent initially chooses not to learn, forgoing the benefits of learning is preferable to paying the immediate learning cost.
    Similarly, the equilibrium shown in \autoref{fig:parameterised-incompetent-games-b} arises because, when your opponent initially chooses to learn, paying the immediate learning cost is preferable to forgoing the benefits of learning.
    
    \begin{table}[t]
        \centering
        % ------------------------------------------------------------------------------ %
% Honours Thesis                                                                 %
% Figure: Discounted Values of Equilibria in an Incremental Learning Tennis Game %
% ------------------------------------------------------------------------------ %

\setlength{\extrarowheight}{3pt}

\begin{tabular}{ccc}

    \toprule
    Equilibrium                        & Player 1 Discounted Value & Player 2 Discounted Value \\
    \midrule
    \autoref{fig:tennis-equilibrium-a} & $0$                       & $0$                       \\
    \autoref{fig:tennis-equilibrium-b} & $-45.24$                  & $-45.24$                  \\
    \autoref{fig:tennis-equilibrium-c} & $-4.42$                   & $-4.42$                   \\
    \bottomrule

\end{tabular}

\setlength{\extrarowheight}{0pt}

        \caption[Discounted Values of Equilibria in a Tennis Game with Learning]{The discounted values of various equilibria in a simple tennis game with incremental learning.}
        \label{tab:tennis-equilibria-values}
    \end{table}

    Here, we encounter a problem that can often arise in general-sum games with multiple equilibria; the concept of a Nash equilibrium cannot be used alone to recommend a single strategy for each player.
    Recall that a similar observation was made about ``Battle of the Sexes'' (shown in \autoref{fig:battle-of-the-sexes}) whose three equilibria produced different expected utilities.
    Still, does the incremental learning game have an equilibrium that both players would prefer over the alternatives?
    This question is answered by comparing the discounted values of each equilibrium in \autoref{tab:tennis-equilibria-values}.
    Notice that the discounted value awarded to both players is greatest when using the equilibrium in \autoref{fig:tennis-equilibrium-a} or, in other words, this equilibrium is \emph{Pareto superior} to the remaining alternatives.
    Precisely, a strategy profile $(\vec{f}^*, \vec{g}^*) \in \vec{F} \times \vec{G}$ is Pareto superior to an alternative strategy profile $(\vec{f} \times \vec{g}) \in \vec{F} \times \vec{G}$ whenever, for every $k = 1, 2$, we have
    \[
        \vec{v}^k_\beta(\vec{f}^*, \vec{g}^*)
            \ge \vec{v}^k_\beta(\vec{f}, \vec{g}),
    \]
    with a strict inequality holding for some $k = 1, 2$ \parencite{Osborne1994}.
    Thus, since the players both prefer the equilibrium in \autoref{fig:parameterised-incompetent-games-a}, we might recommend that Player 1 and Player 2 choose not to improve upon their initial levels of incompetence.
    Note that, as is the case with any general-sum equilibrium, this recommendation relies on a player's opponent following the same reasoning.