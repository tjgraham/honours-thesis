% ---------------------------------------------------------------------------- %
% Honours Thesis                                                               %
% Figure: Equilibrium in a Simple Tennis Game with Incremental Learning        %
% ---------------------------------------------------------------------------- %

\begin{tikzpicture}
    \newcommand\SCL{0.8}

    % Labels
    \node[rotate=90] (P1) at (\SCL * -1.5, \SCL * 3) {Player 1};
    \foreach \j in {1,2,3,4}
        \node[] (L\j) at (\SCL * -0.5, \SCL * \j) {$\nicefrac{\j}{5}$};
    \node[] (L0) at (\SCL * -0.5, 0) {$0$};
    \node[] (L5) at (\SCL * -0.5, \SCL * 5) {$1$};


    \node[] (P2) at (\SCL * 3, \SCL * 6.5) {Player 2};
    \foreach \i in {1,2,3,4}
        \node[] (M\i) at (\SCL * \i, \SCL * 5.5) {$\nicefrac{\i}{5}$};
    \node[] (M0) at (0, \SCL * 5.5) {$0$};
    \node[] (M0) at (\SCL * 5, \SCL * 5.5) {$1$};

    \node[rotate=90] at (7.35, 2.25) {\tiny $2.76 \times 10^{-3}$};
    \node[rotate=45] at (8.5, 2.5) {\tiny $7.68 \times 10^{-6}$};
    \node[] at (8.75, 1.55) {\tiny $2.76 \times 10^{-3}$};
    \node[] at (9.25, 0.5) {\tiny $9.94 \times 10^{-1}$};

    \begin{scope}[spy using outlines={rectangle, magnification=2.5, size=1cm, connect spies}]

    % Vertex Style
    \tikzset{vertex/.style = {
        draw,
        shape=circle,
        inner sep=0pt,
        minimum size=7pt
    }}

    \tikzset{bigvertex/.style = {
        draw,
        shape=circle,
        inner sep=0pt,
        minimum size=14pt
    }}
    
    % Empty Vertex Style
    \tikzset{empty/.style = {
        inner sep=0pt,
        minimum size=7pt,=
    }}

    % Edge Style
    \tikzset{edge/.style = {
        ->,
        > = latex',
        shorten <=7pt,
        shorten >=7pt
    }}

    % Diagonal Edge Style
    \tikzset{diagonal/.style = {
        ->,
        > = latex',
        shorten <=10pt,
        shorten >=10pt
    }}

    % Nodes
    \foreach \i in {0,1,2,3,4,5}
        \foreach \j in {0,1,2,3,4,5}
            \node[vertex] (\i, \j) at (\SCL * \i, \SCL* \j) {};

    % Edges
    \foreach \i in {0,1,2,3,4}
        \foreach \j in {5}
            \draw[edge] (\SCL * \i, \SCL * \j) to (\SCL * \i + \SCL, \SCL * \j);

    \foreach \j in {0,1,2,3,4}
        \foreach \i in {5}
            \draw[edge] (\SCL * \i, \SCL * \j) to (\SCL * \i, \SCL * \j + \SCL);

    \foreach \j in {0}
        \foreach \i in {1,2,3,4}
            \draw[diagonal] (\SCL * \i, \SCL * \j) to (\SCL * \i + \SCL, \SCL * \j + \SCL);

    \foreach \j in {1,2,3,4}
        \foreach \i in {0,1,2,3,4}
            \draw[diagonal] (\SCL * \i, \SCL * \j) to (\SCL * \i + \SCL, \SCL * \j + \SCL);

    \draw[edge] (0, 0) to [out=330,in=300,loop] (0, 0);
    \draw[edge] (0, 0) to (\SCL, 0);
    \draw[edge] (0, 0) to (0, \SCL) {};
    \draw[diagonal] (0, 0) to (\SCL, \SCL);
    \draw[edge] (\SCL * 5, \SCL * 5) to [out=330,in=300,loop] (\SCL * 5, \SCL * 5);

    \node[] (Z) at (0.5 * \SCL, 0.4 * \SCL) {};
    \spy[color4, width=4cm, height=4cm] on (Z) in node at (8.7, 2);

    \end{scope}



\end{tikzpicture}