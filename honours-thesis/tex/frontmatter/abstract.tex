% ---------------------------------------------------------------------------- %
% Honours Thesis                                                               %
% Abstract                                                                     %
% ---------------------------------------------------------------------------- %

\thispagestyle{plain}

\renewcommand{\abstractname}{Abstract}

\vspace*{0.1\textheight}

\begin{abstract}
    Although the strategic insights offered by game theory are often helpful for planning and decision making, traditional game-theoretic frameworks generally ignore the participants' skill levels.
    Beck and Filar \parencite{Beck2007} introduce the notion of incompetence into matrix games to capture accidental deviations.
    How can incompetence be counteracted, either through short-term immediate strategies or long-term learning strategies?
    We address this question by investigating the equilibria of strategic interactions involving incompetence and learning.
    Consequently, this leads us to establish various properties of incompetence, including an explanation of game value plateaus and conditions on optimal learning parameters.
    Moreover, we describe a model of incremental learning and develop a backward induction procedure to exhaustively compute learning strategies.
    This model's insights are illustrated using a simple tennis game adapted from \parencite{Beck2013}.
    Hopefully, by better understanding the strategic considerations arising from players' incompetence, we can develop more realistic and robust strategies.
\end{abstract}